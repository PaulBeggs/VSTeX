\documentclass[stu]{apa7}
\usepackage{amsmath}
\usepackage{amssymb}
\usepackage{enumitem}

\usepackage{geometry}
% \linespread{2}
\linespread{1}
\usepackage[style=apa,backend=biber]{biblatex}
\usepackage{hyperref}
\usepackage{xcolor}
\definecolor{horange}{HTML}{f58026}
\hypersetup{
	colorlinks=true,
	linkcolor=horange,
	filecolor=horange,      
	urlcolor=horange,
    citecolor=horange
}
\usepackage{draculatheme}
\addbibresource{references.bib}

\title{Mental Illness as a Catalyst for Empathetic Development: A Philosophical Analysis}
\shorttitle{Mental Illness for Empathetic Development}
\author{Paul Beggs}
\authorsaffiliations{Hendrix College}
\course{PHIL 390 01: Other Minds. Social Cognition, and Empathy}
\professor{Dr. James Dow}
\duedate{November 14, 2024}

\abstract{
    My paper examines the relationship between mental health and empathy. I argue that people who experience mental illness demonstrate higher levels of empathy toward others, including those without mental health challenges. Drawing from theoretical frameworks, empirical evidence, and philosophical perspectives, I show how mental illness enhances empathetic capabilities. I analyze potential implications for social and therapeutic practices, while addressing key counterarguments to strengthen my position. Through this comprehensive analysis, I establish a clear link between mental health experiences and increased empathetic capacity.
}



\begin{document}

\maketitle

\section{Introduction}
\begin{itemize}
    \item \textbf{Background Information:}
        \begin{itemize}
            \item Overview of mental health and its societal importance.
            \item Introduction to empathy and its role in human interactions.
        \end{itemize}
    \item \textbf{Thesis Statement:}
        \begin{itemize}
            \item Argue that individuals experiencing mental illness are more likely to exhibit higher levels of empathy towards others, including those without mental health challenges.
        \end{itemize}
    \item \textbf{Purpose and Scope:}
        \begin{itemize}
            \item Outline the objectives of the paper.
            \item Brief mention of the structure of the paper.
        \end{itemize}
\end{itemize}

\section{Definitions}
\begin{itemize}
    \item \textbf{Mental Health:}
        \begin{itemize}
            \item Define mental health.
            \item Differentiate between mental health and mental illness.
        \end{itemize}
    \item \textbf{Mental Illness:}
        \begin{itemize}
            \item Define various types of mental illnesses (e.g., depression, anxiety, bipolar disorder).
        \end{itemize}
    \item \textbf{Empathy:}
        \begin{itemize}
            \item Define empathy.
            \item Differentiate between cognitive empathy and emotional empathy.
        \end{itemize}
    \item \textbf{Additional Key Terms:}
        \begin{itemize}
            \item Define any other relevant terms (e.g., compassion, sympathy).
        \end{itemize}
\end{itemize}

\section{Literature Review}
\begin{itemize}
    \item \textbf{Existing Research on Mental Health and Empathy:}
        \begin{itemize}
            \item Summarize studies that explore the relationship between mental health and empathy.
            \item Highlight gaps in the current research that your paper aims to address.
        \end{itemize}
    \item \textbf{Theoretical Frameworks:}
        \begin{itemize}
            \item Discuss relevant psychological and philosophical theories that link mental health and empathy.
        \end{itemize}
\end{itemize}

\section{Argument 1: Enhanced Empathy Through Personal Experience}
\begin{itemize}
    \item \textbf{Personal Insight:}
        \begin{itemize}
            \item Argue that personal experiences with mental illness cultivate a deeper understanding of others' struggles, thereby enhancing empathy.
        \end{itemize}
    \item \textbf{Supporting Evidence:}
        \begin{itemize}
            \item Reference studies or philosophical arguments that support this perspective.
        \end{itemize}
\end{itemize}

\section{Argument 2: Emotional Regulation and Empathy}
\begin{itemize}
    \item \textbf{Emotional Awareness:}
        \begin{itemize}
            \item Discuss how managing one's own mental health conditions can lead to better emotional regulation, which is a key component of empathy.
        \end{itemize}
    \item \textbf{Empirical Support:}
        \begin{itemize}
            \item Present evidence from psychological research that links emotional regulation with increased empathetic responses.
        \end{itemize}
\end{itemize}

\section{Argument 3: Social Connection and Empathy}
\begin{itemize}
    \item \textbf{Building Social Bonds:}
        \begin{itemize}
            \item Argue that individuals with mental illnesses often seek and develop deeper social connections, fostering empathy towards others.
        \end{itemize}
    \item \textbf{Philosophical Perspectives:}
        \begin{itemize}
            \item Explore philosophical viewpoints on how social experiences shape empathetic capacities.
        \end{itemize}
\end{itemize}

\section{Counterarguments and Rebuttals}
\begin{itemize}
    \item \textbf{Counterargument 1: Mental Illness Diminishes Empathy}
        \begin{itemize}
            \item Present the opposing view that certain mental illnesses may impair empathetic abilities.
        \end{itemize}
    \item \textbf{Rebuttal to Counterargument 1:}
        \begin{itemize}
            \item Provide evidence or reasoning that counters this viewpoint, reinforcing your thesis.
        \end{itemize}
    \item \textbf{Counterargument 2: Empathy is Independent of Mental Health}
        \begin{itemize}
            \item Discuss the argument that empathy is a trait unrelated to one's mental health status.
        \end{itemize}
    \item \textbf{Rebuttal to Counterargument 2:}
        \begin{itemize}
            \item Argue how mental health experiences specifically influence empathetic development.
        \end{itemize}
\end{itemize}

\section{Implications of the Association}
\begin{itemize}
    \item \textbf{Social Implications:}
        \begin{itemize}
            \item Discuss how recognizing the link between mental health and empathy can inform social policies and support systems.
        \end{itemize}
    \item \textbf{Therapeutic Implications:}
        \begin{itemize}
            \item Explore how this association can impact therapeutic practices and interventions for individuals with mental illnesses.
        \end{itemize}
\end{itemize}

\section{Conclusion}
\begin{itemize}
    \item \textbf{Summary of Arguments:}
        \begin{itemize}
            \item Recap the main points that support your thesis.
        \end{itemize}
    \item \textbf{Restatement of Thesis:}
        \begin{itemize}
            \item Reinforce your primary argument about the association between mental health and empathy.
        \end{itemize}
    \item \textbf{Future Research Directions:}
        \begin{itemize}
            \item Suggest areas where further research is needed to deepen the understanding of this association.
        \end{itemize}
\end{itemize}

\newpage

\section{\textit{Empathy} by Heidi Maibom}

Consider the following definitions for empathy:

\begin{enumerate}[label=(D\arabic*)]
    \item ``We define empathy as an affective response that stems from the
    apprehension or comprehension of another's emotional state or condition
    and is similar to what the other person is feeling or would be
    expected to feel in a given situation.'' (Eisenberg 2005, 75)
    \item ``[\dots] the three essential features of empathy: affective matching, otheroriented
    perspective taking, and self-other differentiation.'' (Coplan 2011, 6)
    \item ``A perception-action model of empathy specifically states that attended
    perception of the object's state automatically activates the subject's representations
    of the state, situation, and object, and that activation of these
    representations automatically primes or generates the associated autonomic
    and somatic responses, unless inhibited.'' (Preston \& de Waal 2002, 4)
    \item ``Empathic concern refers to other-oriented emotion elicited by and
    congruent with the perceived welfare of a person in need.'' (Batson 2014, 41)
\end{enumerate}

\subsection{Emotional Contagion}

Maibom discusses the concept of emotional contagion, which is the phenomenon where individuals unconsciously mimic the emotions of others. This process is automatic and does not require higher-order cognitive processes. Emotional contagion is distinct from empathy, as it does not involve understanding or sharing the other person's mental state. However, it can be a precursor to empathy, especially in social interactions where emotional cues play a significant role. (Maibom, 2019)

For example, imagine you are watching the news at your home and the anchor reads the teleprompter wrong and stumbles over their words. You may feel embarrassment or discomfort for the anchor, even though you are not directly involved in the situation. This is often colloquially referred to as ``second-hand embarrassment'' or ``vicarious embarrassment.''

\subsection{Affective Empathy}

According to Maibom, an easy way to conceptualize affective empathy is by recognizing that the focal point of your emotion does not hinge on reflecting your friend's emotion. Rather, it is the emotion that you personally feel for the focus of their emotion, through them (e.g., through their verbal description, nonverbal communication, and any other identifier that you are privy to).

For example, consider a situation where your friend tells you about an argument with their mom. If the mom is treating them unfairly, you may feel anger or frustration toward the mom. This emotional response, despite not being directly involved, is an example of affective empathy.

\subsection{Cognitive Empathy}



% Placeholder for your bibliography.
% \printbibliography

\end{document}