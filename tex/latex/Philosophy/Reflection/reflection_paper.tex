\documentclass[stu]{apa7}
\usepackage{amsmath}
\usepackage{amssymb}
\usepackage{enumitem}

\usepackage{geometry}
\usepackage{draculatheme}
\usepackage{setspace}
\linespread{2}
\usepackage{hyperref}
\usepackage{xcolor}
\definecolor{horange}{HTML}{f58026}
\hypersetup{
	colorlinks=true,
	linkcolor=cyan,
	filecolor=cyan,      
	urlcolor=cyan,
    citecolor=cyan
}

\title{Reflection}
% \shorttitle{Mental Illness and Empathetic Concern}
\author{Paul Beggs}
\authorsaffiliations{Hendrix College}
\course{PHIL 390 01: Other Minds. Social Cognition, and Empathy}
\professor{Dr. James Dow}
\duedate{\today}

\begin{document}

\maketitle

Course Reflection Paper
Paul Beggs
Fall 2024
PHIL 390: Philosophy of Mind: Other Minds, Social Cognition, and Empathy
Introduction
	My initial expectations for this class were low. Generally, I did not know what to expect regarding course work, but I did have a notion about the instructor. That is, I heard from many students that Dr. Dow was a respectable professor and that if you came to class, completed assigned readings, and participate in classroom discussions, then he would treat you fairly or even like you. 
	I was afraid that the class would be very difficult. From the syllabus, I read that if you were late to class you must apologize for your tardiness, or if you have an unexcused absence then you would get a letter deducted from your overall grade. As a generally anxious person, this did not sit well with me. 
	I considered dropping the class because it was ``just Philosophy,''\footnote{This preconceived notion has been ingrained in me ever since I was a young kid. My cousin is extremely intelligent, and he decided to get his degree in Philosophy. His parents (and mine) did not understand his decision because they considered a Philosophy degree to be worthless compared to a degree like engineering. (The irony here is that his father was strong-armed into getting his degree in mechanical engineering, even though he wanted to be an economist. My uncle grew to resent his father because of this arbitrary decision, yet, he did the same thing to his son. Perhaps my uncle could have benefitted from taking this course as well.)} but I decided to stick it out because I was (and still am) genuinely interested in the topics of the course. 
	Before this course, I have always considered myself to be empathetic. My family have always said that I give the best advice because I have the most ``lived experience.'' All before my 18th birthday:
	\begin{enumerate}
		\item I would be hospitalized after getting hit by a freight train during a suicide attempt.
		\item I lost a fourth of my body weight because I was tired of the self-hatred for being over-weight.
		\item My weight training coach trapped me under a barbell and interrogated me on if I was involved in breaking a stall door. (``Trapped'' and ``interrogated'' meaning: held a heavy barbell that I could not lift off my chest while he screamed at me to suss out if I had broken the stall door.)
		\item Watched my dog---who was everything and more to me---die as he got hit by a car.
		\item Lost all my friends because of the stigma of mental illness. 
	\end{enumerate}
	These instances are just the tip of the iceberg, and I am not mentioning them for pity, but as a testament to the empathy I have acquired (albeit, quite forcibly). After this course, I realize that I still do possess empathy, but I am not as well-rounded as I thought I was. I learned about many facets of empathy that I had never encountered before. These include cognitive and affective empathy, theory-theory and simulation theory, and folk psychology. I also gained a deeper understanding of empathy and altruism, including what those concepts actually mean. Most importantly, I developed a desire to explore this topic further, beyond what is required in a class environment.
	Just today, I was talking with Dr. Zorwick about empathy and my final paper that I wrote about the relationship between empathetic concern and mental illness. Being able to have an intellectual conversation about something that I know is so important was very empowering. In fact, Dr. Zorwick recommended a book (``The War for Kindness'' by Jamil Zaki) that I would have never considered if I did not have a familiarity with the vocabulary that I learned from this class (e.g., empathetic concern, cognitive empathy, etc.).
	In closing, I just wanted share my observations with you regarding self-worth. Throughout the semester, you have mentioned several mental health struggles that you have been dealing with, and I wanted to ask you to try to not be so hard on yourself. It seems to me that you think that teaching is negligible or worthless, but that is not the case at all. Though I am just one student, I can tell you that I am grateful for the time, passion, and commitment you have given me throughout this semester. We have all been through hard times, but it is up to us on if we want to define ourselves based on things that are out of our control. Such as the opinions of others, for example. To have joy in life, we must first have compassion for ourselves. Is that not the first and most important step in fostering empathy for others? 
\end{document}