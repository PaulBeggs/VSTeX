\documentclass[stu]{apa7}
\usepackage{amsmath}
\usepackage{amssymb}
\usepackage{enumitem}
\usepackage{geometry}
\usepackage[style=apa,backend=biber]{biblatex}
\usepackage{hyperref}
\usepackage{xcolor}
\definecolor{horange}{HTML}{f58026}
\hypersetup{
    colorlinks=true,
    linkcolor=horange,
    filecolor=horange,      
    urlcolor=horange,
    citecolor=horange
}
\usepackage{draculatheme}
\addbibresource{references.bib}

\title{The Effect of ADHD on Working Memory}
\shorttitle{The Effect of ADHD on Working Memory}
\author{Paul Beggs}
\authorsaffiliations{Hendrix College}
\course{PSYC 319: Cognitive Psychology}
\professor{Dr.\ Carmen Merrick}
\duedate{\today}

% \abstract{
% 	TODO: Write abstract.
% }

\begin{document}

\maketitle

\section{Introduction}

Attention-Deficit/Hyperactivity Disorder (ADHD) is a prevalent neurodevelopmental disorder characterized by symptoms of inattention, hyperactivity, and impulsivity \parencite{faraone_attention-deficithyperactivity_2015}. Among the various cognitive functions affected by ADHD, working memory (WM) deficits have garnered significant attention due to their impact on academic and daily functioning. Working memory, the system responsible for temporarily holding and manipulating information, is crucial for tasks such as problem-solving, learning, and reasoning \parencite{skalski_impact_2021}.

This paper argues that individuals with ADHD exhibit impaired working memory. By synthesizing empirical studies, this literature review examines how different components of working memory—the central executive, phonological loop, and visuospatial sketchpad—are affected in individuals with ADHD. Understanding these impairments provides insights into the cognitive challenges faced by those with ADHD and informs strategies for diagnosis and intervention.

\section{Literature Review}

\subsection{Central Executive}

The central executive is pivotal in coordinating cognitive processes and managing attention \parencite{kofler_are_2018}. \textcite{kofler_working_2020} \textbf{(empirical)} provides robust evidence that ADHD is associated with significant deficits in the central executive component of working memory. Utilizing a bifactor modeling approach, Kofler et al. found that children with ADHD exhibit marked impairments in executive functions, such as inhibitory control and cognitive flexibility. These deficits are not only prevalent but also covary with the severity of ADHD symptoms across different settings, underscoring the central executive's role in the disorder's cognitive profile.

Similarly, \textcite{elosua_differences_2017} \textbf{(empirical)} investigates executive functioning in children with ADHD, focusing on divided attention, updating, attentional shifting, and inhibition. Their findings corroborate those of Kofler et al., demonstrating that children with ADHD consistently perform worse on tasks that require the central executive's involvement. This consistent impairment suggests that the central executive deficits are integral to the cognitive challenges experienced by individuals with ADHD.

\subsection{Phonological Loop}

The phonological loop is responsible for the temporary storage and manipulation of verbal and auditory information \parencite{baddeley_developments_1994}. \textcite{fried_clinical_2016} \textbf{(empirical)} explores the relationship between ADHD and phonological working memory. Their controlled study indicates that children with ADHD have significant difficulties with tasks that engage the phonological loop, such as verbal reasoning and immediate memory tasks. These impairments adversely affect academic performance, particularly in areas like reading comprehension and mathematical problem-solving.

In contrast, \textcite{kofler_working_2020} \textbf{(empirical)} presents a nuanced view by demonstrating that while the central executive is significantly impaired in ADHD, the phonological loop remains relatively intact. Their findings suggest that phonological short-term memory does not exhibit the same level of deficit as the central executive and visuospatial sketchpad. This distinction highlights the complexity of working memory impairments in ADHD, indicating that not all components are uniformly affected.

\subsection{Visuospatial Sketchpad}

The visuospatial sketchpad manages the temporary storage and manipulation of visual and spatial information \parencite{baddeley_developments_1994}. \textcite{butzbach_basic_2019} \textbf{(empirical)} examines visuospatial working memory in adults with ADHD and finds that individuals with ADHD exhibit significant deficits in tasks requiring spatial reasoning and visual tracking. These impairments are evident in cross-modality tasks that necessitate the integration of phonological and visuospatial information, suggesting that the visuospatial sketchpad is disproportionately affected in ADHD beyond general working memory impairments.

Furthermore, \textcite{elosua_differences_2017} \textbf{(empirical)} corroborates these findings by demonstrating that children with ADHD perform poorly on visuospatial tasks compared to their non-ADHD peers. The consistent evidence across studies reinforces the conclusion that the visuospatial sketchpad is a critical area of impairment in individuals with ADHD, contributing to the broader cognitive challenges associated with the disorder.

\section{Discussion}

The reviewed studies collectively underscore that individuals with ADHD exhibit significant impairments in working memory, specifically within the central executive and visuospatial sketchpad components. \textcite{kofler_working_2020} and \textcite{elosua_differences_2017} provide robust empirical evidence supporting central executive deficits, which are closely linked to the severity of ADHD symptoms. These deficits hinder cognitive processes such as inhibitory control, cognitive flexibility, and attentional regulation, which are essential for academic and everyday functioning.

In the phonological loop domain, \textcite{fried_clinical_2016} highlights difficulties in verbal reasoning and immediate memory tasks among children with ADHD. However, \textcite{kofler_working_2020} suggests that the phonological loop may remain relatively intact, indicating variability in how different components of working memory are affected. This discrepancy points to the heterogeneous nature of ADHD and suggests that working memory impairments may manifest differently across individuals.

The visuospatial sketchpad emerges as a significant area of impairment, with \textcite{butzbach_basic_2019} and \textcite{elosua_differences_2017} both reporting substantial deficits in spatial reasoning and visual tasks. These impairments affect the ability to integrate and manipulate visual and spatial information, further complicating the cognitive profile of ADHD.

Overall, these findings emphasize the necessity for targeted interventions that address specific working memory components. Enhancing central executive functions and visuospatial skills could lead to improved cognitive outcomes for individuals with ADHD. Additionally, the relative preservation of the phonological loop suggests that leveraging verbal strategies might be beneficial in educational and therapeutic settings.

\section{Conclusion}

This paper explored the intricate relationship between Attention-Deficit/Hyperactivity Disorder (ADHD) and working memory (WM) deficits. The literature reveals a consistent association between ADHD and impairments in working memory, particularly in the central executive and visuospatial sketchpad. However, the variability in findings across studies highlights the heterogeneity of ADHD and suggests that working memory deficits may not be uniformly present across all individuals with the disorder.

Recognizing the specific areas of working memory impairment has important implications for diagnosis and intervention. Tailored strategies that strengthen central executive functions and visuospatial abilities can enhance cognitive performance and academic achievement in individuals with ADHD. Future research should continue to dissect the multifaceted nature of working memory in ADHD, employing standardized methodologies and larger sample sizes to further elucidate these cognitive challenges.

\printbibliography

\end{document}