\renewcommand{\theenumi}{\arabic{enumi}}
\renewcommand{\labelenumi}{\theenumi.}
\section{Introduction to the Power Set}

The \textit{Power Set} is an integral concept that handles the subsets of a set. That is, for any set \(A\), the power set \(\mathcal{P}(A)\) includes every possible subset of \(A\), ranging from the empty set to \(A\) itself. This section will delve into the formal structure of power sets, and provide a basis that we can build upon for later concepts.

\vspace{0.5cm}

\begin{axiom}
    {Power Set}Suppose \(A\) is a set. Then, there exists a set \[\mathcal{P}(A)\]
    called the \textit{power set} of \(A\) whose elements are exactly the subsets of \(A\). Put formally, \[x\in \mathcal{P}(A) \iff x\subseteq A\]
\end{axiom}

\begin{note}
    \textbf{Note the following notation:}

\begin{enumerate}
        
    \item \(x\in A\): This denotes that \(x\) is an element of the set \(A\). It means \(x\) is one of the objects contained within the set \(A\). For example, if \(A=\{a,b,c\}\), then saying \(a\in A\) is correct because \(a\) is one of the elements in the set \(A\).

    \item \(x\subseteq A\): This denotes that \(x\) is a subset of the set \(A\). It means every element of \(x\) is also an element of \(A\), but \(x\) itself does not need to be a singular `element' in \(A\); rather, it can be a collection (or set) of elements that are all found in \(A\). For example, if \(A=\{a,b,c\}\), then \(\{a,b\} \subseteq A\) is correct because every element of the set \(\{a,b\}\) is also an element of \(A\).
\end{enumerate}

The key difference is that \(x\in A\) refers to membership (i.e., \(x\) is a single object within the collection \(A\)), whereas \(x\subseteq A\) refers to inclusion (i.e., all elements of the collection \(x\) are contained within the collection \(A\)).
\end{note}

What we really need the power set for is to establish an ordering on a set. Up to now, membership was the only defining characteristic of sets, but now, we can specify a set's \textit{order} using the following notation for an \textit{ordered pair} \((x,y)\): \[\{\{x\},\{x,y\}\}\]

This leads us to the following Theorem:

\begin{ntheorem}
    {Existence pt. 3}Suppose \(A\) and \(B\) are sets with \(a\in A\) and \(b\in B\). Then the set \(\{\{a\},\{a,b\}\}\) exists.
\end{ntheorem}

\tpf{
    Let \(A\) and \(B\) be sets such that \(a\in A\) and \(b\in B\). \\
    Then, \(A\cup B\) exists, and \(a,b\in A\cup B\) by the corollary to Axiom 1.1.1. 5. Consider \(\mathcal{P}(A\cup B)\). By Problem 34, \(\{a\}\in \mathcal{P}(P\cup B)\). And consider the set \(a,b\), so \(\{a,b\}\in \mathcal{P}(A\cup B)\) by Axiom 6. Consider \(\mathcal{P}(\mathcal{P}(A\cup B))\) which exists by Axiom 6. Since \(\{a\}, \{a,b\}\in \mathcal{P}(A\cup B)\), then \(\{\{a\},\{a,b\}\} \in \mathcal{P}(\mathcal{P}(A\cup B))\).
}

\begin{ntheorem}
    {Set Pair Equality}Let \(a,b,x,y\) are elements such that \(\{\{a\},\{a,b\}\}=\{\{x\},\{x,y\}\}\). Then, \[a=x \text{ and } b=y\] 
\end{ntheorem}

\tpf{
    Let \(a,b,x,y\) be elements such that \(\{\{a\},\{a,b\}\}=\{\{x\},\{x,y\}\}\). \\
    By Axiom 1, we must have \(\{a\} = \{x\}\) or \(\{a\} = \{x,y\}\). Consider the following cases: \\
    
    \noindent \textbf{Case 1:} \\
    \indent If \(\{a\} = \{x\}\), then by Axiom 1 \(a=x\). Also, in this case, \(\{a,b\} = \{x,y\}\). And since \(a=x\), then \(b=y\) by Axiom 1. \\
    
    \noindent \textbf{Case 2:} \\
    \indent If \(\{a\} = \{x,y\}\), then Axiom 1 tells us that either \(a=x\) and \(a=y\). Also in this case, the set containing \(x\) must equal the set containing \(a,b\) and thus, \(x=a\) and \(x=b\). \\
    
    Thus, \(b=x=a=y\).
}

\begin{definition}
    {Ordered Pairs (formal definition)}Let \(A\) and \(B\) be the sets with \(a\in A\) and \(b\in B\). Then, the \textit{ordered pair} \((a,b)\) is the set \(\{\{a\},\{a,b\}\}\).
\end{definition}

\newpage

\begin{ntheorem}
    {Product}Let \(A\) and \(B\) be sets. Then, \[\{(a,b)\colon a\in A \text{ and } b\in B\} \text{ exists.}\] 
\end{ntheorem}

\tpf{
Let \(A\) and \(B\) be sets. Theorem 3.1.2 gives that each ordered pair exists, and the existence Axiom---simply stated as, ``There exists a set''---gives the existence of the set as a whole. 
}

\begin{definition}
    {Cartesian Product}The set in Theorem 4 is called the (Cartesian) product. We denote this as \(A\times B\).
\end{definition}