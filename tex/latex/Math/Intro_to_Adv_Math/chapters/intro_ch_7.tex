    
            \begin{definition}
                {Equivalence Relation}Suppose \(E\) is a relation on \(A\). The statement that \(E\) is an \textit{equivalence relation} means that \(E\) is reflexive, symmetric, and transitive. 
            \end{definition}
        
            \begin{example}
                If \(H\) is the set of all Hendrix students, and \(E\) is the relation defined by \((a,b)\in E\) means \(a\) has the same advisor as \(b\).
            \end{example}
    
    
            \begin{definition}
               {Equivalence Class} Suppose \(E\) is an equivalence relation on \(A\). If \(a\in A\), we define the \textit{equivalence class} of \(a\) (relative to \(E\)) by \([a]_E := \{b\in A\colon (a,b) \in E\}\)
            \end{definition}
            \textbf{Note}: We denote the set of all equivalence classes by \(A/E\).
        
            \begin{example}
                \([\text{West}]_\text{Advisor} = \{\text{Yorgey's advisee}\}\). 
            \end{example}
    
        \section{Problem}

% ------------------------------------------------------------------------------
% Problem 40
% ------------------------------------------------------------------------------

            \begin{exercise}
                {Equivalence Relation I}Suppose \(E\) is an equivalence relation on \(A\). Then, for distinct elements \(a,b\in A\), either: \[[a]_E = [b]_E \text{ or } [a]_E \cap [b]_E = \emptyset\]
            \end{exercise}
            \textbf{Think}: Equivalence relations divide a set into classes. Kind dividing up sets into ``families.''

            \expf{
                Let \(E\) be equivalent on set \(A\) for \(a,b\in A\) and \(a\ne b\). \\

                Suppose \([a]_E \cap [b]_E \ne \emptyset\). We'll show that \([a]_E = [b]_E\). By the definition of intersection, and the negation of Problem 11, there exists \(x\in [a]_E\) and \(x\in [b]_E\). Then by the definition of equivalence classes, we have \((a,x) \in E\) and \((b,x) \in E\). Because \textit{E} is symmetric and transitive, \((x,b) \in E\) and if \((a,x) \in E\), and \((x,b)\in E\), then \((a,b)\in E\), respectively. The fact that \((a,b) \in E\) means both \(a\) and \(b\) are in the same equivalence class, so \([a]_E = [b]_E\). \\
                
                Therefore, for distinct elements \(a,b\in A\), either \([a]_E = [b]_E\), or their intersection is empty.
            }

% Proof: Need to add additional clarification about since the ordered pairs are in E, and the intersection cannot be empty, then the equivalence classes must be equal.


            \begin{definition}
                {Partition}Suppose \(A\) is a set and \(\mathcal{A}\) is a collection of subsets of \(A\). We say that \(\mathcal{A}\) is a \textit{partition} of \(A\) provided: 
                \begin{enumerate}
                    \item \(s\ne \emptyset\) for all \(s\in \mathcal{A}\), and
                    \item if \(s_1, s_2 \in \mathcal{A}\), then \(s_1\cap s_2 = \emptyset\) or \(s_1 = s_2\), and
                    \item if \(a\in A\), then there exists \(S\in \mathcal{A}\) such that \(a\in S\).
                \end{enumerate}
            \end{definition}
            \textbf{Note}: (1) generates no empty set. (2) Subsets are either the same or have no overlap (i.e., ``uniqueness''). (3) All elements of \(A\) are included somewhere.

        \vspace{0.5cm}

        \section{Problems}

% ------------------------------------------------------------------------------
% Problem 41
% ------------------------------------------------------------------------------

            \begin{exercise}
                {Partition}Suppose \(E\) is an equivalence relation on \(A\). Then, \[A/E \text{ is a partition of } A.\]
            \end{exercise}

            \expf{
                Let \textit{E} be an equivalence relation on \textit{A}. To show that \(A/E\) is a partition of \(A\), we need to prove three things:
                \begin{enumerate}
                    \item \textbf{Non-emptiness}: \\
                    Suppose \(a\in A\). By the definition of an equivalence relation and its equivalence classes, \([a]_E\) contains at least the element \(a\) itself due to the reflexivity of \textit{E} (i.e., \((a,a) \in E\)). Therefore, no equivalence class in \(A/E\) can be empty.
                    \item \textbf{Mutual exclusivity}: \\
                    From Problem 40, we know that for any two distinct elements \(a,b \in A\), their equivalence classes under \textit{E}, \([a]_E\) and \([b]_E\), are either identical or have no elements in common. This ensure that any two distinct equivalence classes do not overlap. 
                    \item \textbf{Completeness}: \\
                    By the definition of equivalence classes, every element \(a\in A\) must belong to the equivalence class \([a]_E\). This guarantees that the union of all equivalence classes in \(A/E\) covers the entire set of \textit{A}.
                \end{enumerate}

                Therefore, \(A/E\) is a partition of \textit{A}.
            }

% ------------------------------------------------------------------------------
% Problem 42
% ------------------------------------------------------------------------------

% ------------------------------------------------------------------------------
\newpage
% ------------------------------------------------------------------------------

            \begin{exercise}
                {Equivalence Relation II}Suppose \(A\) is a set and \(\mathcal{A}\) is a partition on \(A\). Define the relation, \(\sim : A \rightarrow A\) by \(x \sim y\) if, and only if, there exists an \(S\in A\) with \(x,y\in S\). Then, \[\text{``}\sim \text{''} \text{ is an equivalence relation.}\]
            \end{exercise}
            \textbf{Note}: \(\sim\): \(A\rightarrow A\) by \(x\sim y \iff \exists S\in A\) with \(x,y\in S\). \\
            Must prove that given this definition, \(\sim\) is an equivalence relation.


            \expf{
                Let \(A\) be a set with \(\mathcal{A}\) as a partition on \(A\). To show that \(\text{``}\sim \text{''} \text{ is an equivalence relation}\), we need to prove three things:
                \begin{enumerate}
                    \item \textbf{Reflexivity}: \\
                    For \(\sim\) to be reflexive, we need to show that every element \(x\in A\) is related to itself under \(\sim\). Additionally, since \(\mathcal{A}\) is a partition, then by definition, \(\mathcal{A}\) is a collection of subsets of \(A\). This is important because by the definition of partition 17 (3), it means for every element \(x\in A\), there exists a subset \(S\in A\) such that \(x\in S\). \\
                    Therefore, by definition, \(x\sim x\) for all \(x\in A\).
                    \item \textbf{Symmetry}: \\
                    For symmetry, we need to show that if \(x\sim y\), then \(y\sim x\) for any \(x,y\in A\). So, suppose that \(x\sim y\) by the definition of \(\text{``}\sim \text{''}\). This implies that there exists an \(S\in A\) with \(x,y \in S\). By Problem 10, we can rearrange this to be \(y,x\). \\
                    Therefore, if \(x\sim y\), then \(y\sim x\).
                    \item \textbf{Transitivity}: \\
                    For \(\sim\) to be transitive, we need to show that if \(x \sim y\) and \(y \sim z\), then \(x\sim z\) for any \(x,y,z\in A\). By the definition of \(\text{``}\sim \text{''}\), we know that there exists an \(s_1\in \mathcal{A}\) such that \(x,y \in s_1\) and there exists an \(s_2\in \mathcal{A}\) such that \(y,z\in s_2\). Given that \(y\) is an element of both \(s_1\) and \(s_2\), it follows that the intersection of these two partitions must not be empty. Hence, by Problem 40, we know that because \(s_1 \cap s_2 \ne \emptyset\), then it must be the case that \(s_1 = s_2\). \\
                    Therefore, since \(s_1 = s_2\), and both \(x\) and \(z\) belong to this \textit{same} subset, by definition of \(\text{``}\sim \text{''}\), \(x\sim z\).
                \end{enumerate}
            }

\newpage 