\begin{exercise}
    {2.2.2(b)}Verify, using \defref{2.2.3}, that
    the following sequences converge to the proposed limit.
    \begin{enumerate}
        \setcounter{enumi}{1}
        \item \(\limn \frac{2n^2{n^3+3}} = 0\)
    \end{enumerate}
\end{exercise}

\sol{
    \begin{enumerate}
        \setcounter{enumi}{1}\item
              \innerpf{

                  Let \(\epsilon > 0\). Choose \(N = \frac{2}{\epsilon}\). Then, for \(n > N\),
                  \begin{align*}
                      \disabs{\frac{2n^2}{n^3 + 3} - 0} & = \disabs{\frac{2n^2}{n^3 + 3}} \\
                                                        & = \frac{2n^2}{n^3 + 3}          \\
                                                        & < \frac{2n^2}{n^3}              \\
                                                        & = \frac{2}{n}                   \\
                                                        & < \frac{2}{N}                   \\
                                                        & = \frac{2}{2/\epsilon}          \\
                                                        & = \epsilon.
                  \end{align*}
                  Therefore, \(\limn \frac{2n^2{n^3 + 3}} = 0\). \qedhere

              }
    \end{enumerate}}

\begin{exercise}
    {2.2.3} Describe what we would have to demonstrate in order to disprove each of the following statements.
    \begin{enumerate}
        \item At every college in the United States, there is a student who is at least
              seven feet tall.
        \item For all colleges in the United States, there exists a professor who gives every student a grade of either A or B.
        \item There exists a college in the United States where every student is at least six feet tall.
    \end{enumerate}
\end{exercise}

\sol{
    \begin{enumerate}
        \item There is at least one college in the United States where all students are less than seven feet tall.
        \item There is at least one college in the United States where all professors give at least one student a grade of C or lower.
        \item For all colleges in the United States, there exists a student who is less than six feet tall.
    \end{enumerate}
}

\begin{exercise}
    {2.2.4} Give an example of each or state that the request is impossible.
    For any that are impossible, give a compelling argument for why that is the case.
    \begin{enumerate}
        \item A sequence with an infinite number of ones that converges to one.
        \item A sequence with an infinite number of ones that converges to a limit not equal to one.
        \item A divergent sequence such such that for every \(n \in \N\) it is possible to find \(n\) consecutive ones somewhere in the sequence.
    \end{enumerate}
\end{exercise}

\sol{
    \begin{enumerate}
        \item Possible. Example: \(a_n = 1 \text{ for all } n \in \N\).
        \item Impossible. We can use the \namrefthm{Uniqueness of Limits} to show that this is impossible. Essentially, because we found in (a) that the limit of an infinite number of ones converges to 1, it must be the case that if the limit is not equal to one, then the sequence cannot converge to one.
        \item Possible. Example: \(\{2,2,2,2,4,4,4,4,1,1,1,1,1,3,3,\dots\}\)
    \end{enumerate}
}


\begin{exercise}
    {2.3.2}Using only \defref{2.2.3}, prove that if \((x_n) \rightarrow 2\), then
    \begin{enumerate}
        \item \(\displaystyle \left(\frac{2x_n - 1}{3}\right) \rightarrow 1\);
        \item \(\displaystyle \left(1 / x_n\right) \rightarrow 1 / 2\).
    \end{enumerate}
    (For this exercise the Algebraic Limit Theorem is off-limits, so to speak.)
\end{exercise}

\sol{ \hfill
    \begin{enumerate}
        \item \textit{Proof.} Let \(\epsilon > 0\). Since \((x_n)\) converges to 2, there exists \(N \in \N\) such that for all \(n \geq N\), \(\disabs{x_n - 2} < \epsilon\). Now, for any \(n \geq N\),
              \begin{align*}
                  \disabs{\frac{2x_n - 1}{3} - 1} & = \disabs{\frac{2x_n - 1 - 3}{3}} \\
                                                  & = \disabs{\frac{2x_n - 4}{3}}     \\
                                                  & = \frac{2}{3}\disabs{x_n - 2}     \\
                                                  & < \disabs{x_n - 2}                \\
                                                  & < \epsilon
              \end{align*}
              Therefore, \(\frac{2x_n - 1}{3} \rightarrow 1\) \qed
        \item \textit{Proof.} Let \(\epsilon > 0\). Since \((x_n)\) converges to 2, there exists \(N \in \N\) such that for all \(n \geq N\), \(\disabs{x_n - 2} < \epsilon\). Now, for any \(n \geq N\),
              \begin{align*}
                  \disabs{\frac{1}{x_n} - \frac{1}{2}} & = \disabs{\frac{2 - x_n}{2x_n}}  \\
                                                       & = \frac{1}{2x_n}\disabs{2 - x_n} \\
              \end{align*} \qed
    \end{enumerate}
}