\documentclass[12pt]{article}

\usepackage{fix-cm}
\usepackage{tikz}
\usetikzlibrary{shapes,backgrounds,calc,patterns}
\usepackage{amsmath,amsthm} 
\usepackage{mathtools}
\usepackage{amssymb}
\usepackage[framemethod=tikz]{mdframed}
\usepackage{mathrsfs}
\usepackage{changepage}
\usepackage{multicol}
\usepackage{slashed}
\usepackage{enumerate}
\usepackage{booktabs}
\usepackage{enumitem}
\usepackage{kantlipsum}  %This package lets us generate random text for example purposes.
\usepackage{pgfplots}



%%% AESTHETICS %%%
%-%-%-%-%-%-%-%-%-%-%-%-%-%-%-%-%-%-%-%-%-%-%-%-%-%-%-%-%-%-%-%-%-%-%-%-%-%-%


%%% Dimensions and Spacing %%%
\usepackage[margin=1in]{geometry}
\usepackage{setspace}
\linespread{1}
\usepackage{listings}

%%% Define new colors %%%
\usepackage{xcolor}
\definecolor{orangehdx}{rgb}{0.96, 0.51, 0.16}

% Normal colors
\definecolor{xred}{HTML}{BD4242}
\definecolor{xblue}{HTML}{4268BD}
\definecolor{xgreen}{HTML}{52B256}
\definecolor{xpurple}{HTML}{7F52B2}
\definecolor{xorange}{HTML}{FD9337}
\definecolor{xdotted}{HTML}{999999}
\definecolor{xgray}{HTML}{777777}
\definecolor{xcyan}{HTML}{80F5DC}
\definecolor{xpink}{HTML}{F690EA}
\definecolor{xgrayblue}{HTML}{49B095}
\definecolor{xgraycyan}{HTML}{5AA1B9}

% Dark colors
\colorlet{xdarkred}{red!85!black}
\colorlet{xdarkblue}{xblue!85!black}
\colorlet{xdarkgreen}{xgreen!85!black}
\colorlet{xdarkpurple}{xpurple!85!black}
\colorlet{xdarkorange}{xorange!85!black}
\definecolor{xdarkcyan}{HTML}{008B8B}
\colorlet{xdarkgray}{xgray!85!black}

% Very dark colors
\colorlet{xverydarkblue}{xblue!50!black}

% Document-specific colors
\colorlet{normaltextcolor}{black}
\colorlet{figtextcolor}{xblue}

% Enumerated colors
\colorlet{xcol0}{black}
\colorlet{xcol1}{xred}
\colorlet{xcol2}{xblue}
\colorlet{xcol3}{xgreen}
\colorlet{xcol4}{xpurple}
\colorlet{xcol5}{xorange}
\colorlet{xcol6}{xcyan}
\colorlet{xcol7}{xpink!75!black}

% Blue-Purple (should just used colorbrewer...)
\definecolor{xrainbow0}{HTML}{e41a1c}
\definecolor{xrainbow1}{HTML}{a24057}
\definecolor{xrainbow2}{HTML}{606692}
\definecolor{xrainbow3}{HTML}{3a85a8}
\definecolor{xrainbow4}{HTML}{42977e}
\definecolor{xrainbow5}{HTML}{4aaa54}
\definecolor{xrainbow6}{HTML}{629363}
\definecolor{xrainbow7}{HTML}{7e6e85}
\definecolor{xrainbow8}{HTML}{9c509b}
\definecolor{xrainbow9}{HTML}{c4625d}
\definecolor{xrainbow10}{HTML}{eb751f}
\definecolor{xrainbow11}{HTML}{ff9709}

%%% FIGURES %%%
\usepackage{graphicx}  
\graphicspath{ {images/} }  
% \numberwithin{figure}{section}
\usepackage{float}
\usepackage{caption}

%%% Hyperlinks %%%
\usepackage{hyperref}
\definecolor{horange}{HTML}{f58026}
\hypersetup{
	colorlinks=true,
	linkcolor=horange,
	filecolor=horange,      
	urlcolor=horange,
}

\newcommand{\disabs}[1]{\ensuremath{\left|#1\right|}}
\newcommand{\limn}[1]{\ensuremath{\lim_{n \rightarrow \infty}}#1}
\newcommand{\limx}[2]{\lim_{x \rightarrow #1}#2}

\newcommand{\limsupn}[1]{\displaystyle\limsup_{n\rightarrow \infty}#1}
\newcommand{\liminfn}[1]{\displaystyle\liminf_{n\rightarrow \infty}#1}


\renewcommand{\theenumi}{\arabic{enumi}} 
\renewcommand{\labelenumi}{(\theenumi)}

\newenvironment{solution}{\textit{Solution.}}

\title{Real Analysis: Exam 2 Corrections}
\author{Paul Beggs}
\date{November 13, 2024}

%%% Custom Comands %%%
% Natural Numbers 
\newcommand{\N}{\ensuremath{\mathbb{N}}}

% Whole Numbers
\newcommand{\W}{\ensuremath{\mathbb{W}}}

% Integers
\newcommand{\Z}{\ensuremath{\mathbb{Z}}}

% Rational Numbers
\newcommand{\Q}{\ensuremath{\mathbb{Q}}}

% Real Numbers
\newcommand{\R}{\ensuremath{\mathbb{R}}}

% Complex Numbers
\newcommand{\C}{\ensuremath{\mathbb{C}}}

\newcommand{\I}{\ensuremath{\mathbb{I}}}


\begin{document}

\maketitle
\begin{enumerate}
    \item Let \(x_{1} = 6\), and for all \(n \in \N\), \(x_{n+1} = 2 + \sqrt{x_{n} - 2}\). Show that \((x_{n})\) converges and find its limit.

    \begin{solution}
        \textbf{Step 1: Show that \((x_n)\) is bounded and decreasing.}

        Notice that if \(x_{n} = 3\), then,
        \[
        x_{n + 1} = 2 + \sqrt{3 - 2} = 2 + 1 = 3.
        \]
        This indicates that if \(x_{n}\) ever reaches 3, then \(x_{n + 1} = 3\). Thus, making it a fixed point. We conjecture that this is the lower bound for the sequence.

        \begin{itemize}
            \item \textbf{Base Case:} \(x_1 = 6 > 3\).
            \item \textbf{Inductive Step:} Assume \(x_n > 3\) for some \(n \in \N\). Since \(x_{n} > 3\), we have \(x_{n} - 2 > 1\), which means \(\sqrt{x_{n} - 2} > 1\). Thus, 
            \[
                x_{n + 1} = 2 + \sqrt{x_{n} - 2} > 2 + 1 = 3.
            \]
            This shows that \(x_{n} > 3\) for all \(n\). 
        \end{itemize}

        Now we will show that \(x_{n + 1} < x_{n}\) when \(x_{n} > 3\), which will prove that the sequence is decreasing.

        Our goal is to prove 

        \[
        2 + \sqrt{x_{n} - 2} < x_{n}.
        \]
        Rearranging this inequality, we get
        \[
        \sqrt{x_{n} - 2} < x_{n} - 2.
        \]
        Since \(x_{n} > 3\), we know \(x_{n} - 2 > 1\), and for numbers greater than 1, it holds that \(\sqrt{x_{n} - 2} < x_{n} - 2\). Therefore, \(x_{n + 1} < x_{n}\), and the sequence is decreasing when \(x_{n} > 3\). Thus, the sequence is bounded and decreasing.

        \textbf{Step 2: Conclude that \((x_n)\) converges.}

        A monotonic sequence that is bounded converges by the Monotone Convergence Theorem. Therefore, \((x_n)\) converges to some limit \(L \geq 3\).

        \textbf{Step 3: Find the limit \(L\).}

        Taking the limit on both sides of the recursion and solving for \(L\):

        \begin{align*}
            \lim_{n \to \infty} x_{n + 1} &= \lim_{n \to \infty} \left(2 + \sqrt{x_n - 2}\right) \\
            L &= 2 + \sqrt{L - 2} \\
            L - 2 &= \sqrt{L - 2} \\
            (L - 2)^2 &= L - 2 \\
            L^2 - 4L + 4 &= L - 2 \\
            L^2 - 5L + 6 &= 0 \\
            (L - 2)(L - 3) &= 0.
        \end{align*}
        Therefore, \(L = 2\) or \(L = 3\). Since \(L \geq 3\), the limit is \(L = 3\). 
        
        The sequence \((x_n)\) converges to 3.
    \end{solution}

    \setcounter{enumi}{2}
    \item Let \(K \subset \R\) be a nonempty compact set, and let \(p \in K^{c}\). Define 
    \[
    d = \inf\{\disabs{x - p} \mid x \in K\}.
    \]
    \begin{enumerate}
        \item Show that there exists a sequence \((x_n)\) in \(K\) such that \(\displaystyle\lim_{n \to \infty} \disabs{x_n - p} = d\).
        \item Show there exists a point \(x_0\) in \(K\) such that \(\disabs{x_0 - p} = d\). \\
        \textit{We think of \(x_0\) as the closest point in \(K\) to \(p\).}
    \end{enumerate}
    
    \begin{solution}
        \begin{enumerate}[label=(\alph*)]
        \item By the definition of the infimum, we know for all \(\epsilon > 0\), there exists an \(x \in K\) such that 
        
        \[
        d \leq |x - p| < d + \epsilon.
        \]
        Use \(\epsilon = \frac{1}{n}\). Then, for each \(n \in \N\) there exists \(x_n \in K\) such that
        \[
        d \leq \disabs{x_n - p} < d + \frac{1}{n}.
        \]
        This implies
        \[
        \left| \disabs{x_n - p} - d \right| < \frac{1}{n}.
        \]
        Therefore,
        \[
        \lim_{n \to \infty} \disabs{x_n - p} = d.
        \]
        Hence, there exists a sequence \((x_n)\) in \(K\) such that \(\displaystyle\lim_{n \to \infty} \disabs{x_n - p} = d\).

        \item Since \(K\) is compact and \((x_n)\) is in \(K\), by the Heine–Borel theorem, there exists a subsequence \((x_{n_k})\) that converges to some point \(x_0 \in K\).

        We have
        \[
        \lim_{k \to \infty} \disabs{x_{n_k} - p} = d.
        \]
        Using the triangle inequality,
        \[
        \left| \disabs{x_0 - p} - \disabs{x_{n_k} - p} \right| \leq \disabs{x_{n_k} - x_0}.
        \]
        Taking the limit as \(k \to \infty\),
        \[
        \left| \disabs{x_0 - p} - d \right| \leq 0.
        \]
        Thus,
        \[
        \disabs{x_0 - p} = d.
        \]
        Therefore, there exists \(x_0 \in K\) such that \(\disabs{x_0 - p} = d\).
        \end{enumerate}
    \end{solution}
    
    \end{enumerate}


\end{document}