\documentclass{article}
\usepackage{amsthm} % For theorem-style environments
\usepackage{hyperref} % For cross-references
\usepackage{amsmath}
\usepackage{tcolorbox}
\usepackage{environ} % For capturing environments

% Create a new counter for definitions
\newcounter{BoxCounter}

% Your existing tcolorbox definition for definitions
\newtcolorbox[use counter=BoxCounter, number within=section]{definition}[1][]{%
    rounded corners,
    colframe=gray,
    colback=gray!15,
    coltitle=white,
    title={\large \textbf{Definition \theBoxCounter}},
    label={def:\theBoxCounter},
    #1
}

% Define a macro to store the definitions for the appendix
\newcommand{\storeappendix}[1]{%
    \expandafter\def\expandafter\appendixdefs\expandafter{\appendixdefs#1\par}%
}

% Initialize the appendix storage as empty
\newcommand{\appendixdefs}{}

% Environment for storing definitions (title + content)
\NewEnviron{storedefinition}[1][]{%
    \begin{definition}[#1]%
    \BODY % Display the definition in the main text
    \end{definition}%
    % Store the definition for the appendix
    \storeappendix{Definition \theBoxCounter: \BODY}
}

% Command to print the appendix definitions at the end of the document
\newcommand{\printappendixdefs}{
    \newpage
    \section*{Appendix: Definitions}
    \appendixdefs
}

\begin{document}

\renewcommand{\theenumi}{\arabic{enumi}}
\renewcommand{\labelenumi}{\theenumi.}
\section{Main Text}

Here is the main discussion of functions, domains, and ranges.

\begin{storedefinition}
    Given two sets \(A\) and \(B\), a function from \(A\) to \(B\) is a rule or mapping that takes each element \(x \in A\) and associates with it a single element of \(B\).
    The set \(A\) is called the domain of \(f\).
    The range of \(f\) is not necessarily equal to \(B\), but refers to the subset of \(B\) given by \(\{y \in B : y = f(x) \text{ for some } x \in A\}\).
\end{storedefinition}
\begin{storedefinition}
    Given two sets \(A\) and \(B\), a function from \(A\) to \(B\) is a rule or mapping that takes each element \(x \in A\) and associates with it a single element of \(B\).
    The set \(A\) is called the domain of \(f\).
    The range of \(f\) is not necessarily equal to \(B\), but refers to the subset of \(B\) given by \(\{y \in B : y = f(x) \text{ for some } x \in A\}\).
\end{storedefinition}

% Add appendix at the esnd of the document
\printappendixdefs

\end{document}
