\documentclass[11pt]{article}
\usepackage{listings}
\usepackage{amsmath}
\usepackage{amssymb}
\usepackage{enumerate}
\usepackage[margin=1in]{geometry}



\renewcommand{\theenumi}{\arabic{enumi}} 
\renewcommand{\labelenumi}{\theenumi)}
\begin{document}


\begin{center}
    \section*{Cryptography: Coding Lab 1}

    \large{Due: Tuesday, Sep. 24, 2024}
\end{center}
Goal: Implement a computer program utilizing modular arithmetic that can encrypt and decrypt messages using a Caesar shift cipher and a transposition cipher.
\subsection*{Part 1: Caesar Shift}

\begin{enumerate}
\item	Below, write pseudocode for a program that will ask for a plaintext message and an integer key between 1 and 26 and output the appropriate cipher text. \\
\textbf{Python: define a function \texttt{Cencrypt} that performs the above algorithm.}

\begin{lstlisting}[language=Python]
def Cencrypt():
    plaintext = input("Enter plaintext: ").upper()
    key = int(input("Enter key (1-26): "))

    # Ensure key is between 1 and 26
    if not 1 <= key <= 26:
        print("Key must be between 1 and 26.")
        return

    # Filter out non-alphabetic characters
    plaintext = ''.join(filter(str.isalpha, plaintext))

    # Shift each letter by the key
    ciphertext = ''.join(chr((ord(c) - ord('A') + key) % 26 
                              + ord('A')) for c in plaintext)

    print("Ciphertext:", ciphertext)
\end{lstlisting}





\item	How would this program need to be changed to decrypt a cipher text given a key? \\
\textbf{Python: define a function \texttt{Cdecrypt} and have the user choose encrypt/decrypt upon opening the program.}

\item	How would the decrypt function be changed to allow a brute force attack if we did not know the key? \\
\textbf{Python: Add an option to the decrypt function so if the person does not know the key, the program will perform a brute force attack.}

\item	Check that your program works with the following message/ciphertext pair
\texttt{e(3, `hello world') = KHOORZRUOG}

\item	Choose a message of your own to encrypt and then send the ciphertext to another group for them to decrypt (without the key). \\
Write down your message/ciphertext/key here.
\end{enumerate}


Write the other group's ciphertext and your decryption of it here, along with their key.


\newpage
\subsection*{Part 2: Transposition Cipher}
\begin{enumerate}
\item	Below, write pseudocode for a program that will ask for a plaintext message and an integer key and output the appropriate cipher text. \\
\textbf{Python: define a function \texttt{Tencrypt} that performs the above algorithm.}












\item	How would this program need to be changed to decrypt a cipher text given a key? \\
\textbf{Python: define a function \texttt{Tdecrypt} and have the user choose encrypt/decrypt upon opening the program.}

\item	How would the decrypt function be changed to allow a brute force attack if we did not know the key? \\
\textbf{Python: Add an option to the decrypt function so if the person does not know the key, the program will perform a brute force attack.}

\item	Check that your program works with the following message/ciphertext pair
\texttt{e(3, `hello world') = HOLEWDLOLR}

\item	Choose a message of your own to encrypt and then send the ciphertext to another group for them to decrypt (without the key). 
Write down your message/ciphertext/key here.


Write the other group's ciphertext and your decryption of it here, along with their key.
\end{enumerate}


\end{document}
