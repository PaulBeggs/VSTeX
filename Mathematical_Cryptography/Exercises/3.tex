\begin{exercise}
    {2.3} Let $g$ be a \hyperlink{section 2.2}{primitive root} for $\mathbb{F}_p$.
    \begin{enumerate}
        \setcounter{enumi}{1} % Start enumeration at (b)
        \item Prove that $\log_g(h_1 h_2) = \log_g(h_1) + \log_g(h_2)$ for all $h_1, h_2 \in \mathbb{F}_p^*$.
        \item Prove that $\log_g(h^n) = n \log_g(h)$ for all $h \in \mathbb{F}_p^*$ and $n \in \mathbb{Z}$.
    \end{enumerate}
\end{exercise}

\sol{
    \begin{enumerate}
        \setcounter{enumi}{1}
        \item \textit{Proof.} We know that \(g^x \equiv h \pmod{p}\), which means that \(x = \log_g(h)\). Similarly, if \(g^{x_1} \equiv h_1 \pmod{p}\) and \(g^{x_2} \equiv h_2 \pmod{p}\), then \(x_1 = \log_g(h_1)\) and \(x_2 = \log(h_2)\). Now, we can substitute these values into the first equation to get \(g^{x_1 + x_2} \equiv h_1h_2 \pmod{p} \equiv g^{\log_g(h_1) + \log_g(h_2)} \pmod{p}\). Then, from the properties of exponents, we can rewrite this equation as \(h_1 \cdot h_2 \pmod{p} \equiv g^{\log_g(h_1h_2)} \pmod{p}\). Therefore, \(\log_g(h_1 \cdot h_2) = \log_g(h_1) + \log_g(h_2)\).
        \item \textit{Proof} Following a similar process to \((b)\), we start with \(g^{n\log_g(h)}\). Then, by using the properties of logarithms, we can rewrite this as \(g^{\log_g(h^n)}\). Then, because \(g^{\log_g}\) cancel out, we see that \(g^{\log_g(h^n)} = h^n\). Putting everything together, we have \begin{align*}
                  g^{\log_g(h^n)} & \equiv h^n \pmod{p}         \\
                  \log_g(h^n)     & \equiv n\log_g(h) \pmod{p}.
              \end{align*}
    \end{enumerate}
}

\begin{exercise}
    {2.4} Compute the following \hyperlink{section 2.3}{discrete logarithms}.
    \begin{enumerate}
        \item $\log_2(13)$ for the prime $23$, i.e., $p = 23$, $g = 2$, and you must solve the congruence $2^x \equiv 13 \pmod{23}$.
        \item $\log_{10}(22)$ for the prime $p = 47$.
        \item $\log_{627}(608)$ for the prime $p = 941$. (Hint: Look in the second column of Table 2.1 on page 66.)
    \end{enumerate}
\end{exercise}

\sol{
    \begin{enumerate}
        \item We use Wolfram Alpha to solve for \(x\) in the equation \(2^x \equiv 13 \pmod{23}\): \(x = 7\).
        \item Solving for \(x\) we get \(x = 11\).
        \item \(x = 18\).
    \end{enumerate}
}

\begin{exercise}
    {2.6} Alice and Bob agree to use the prime $p = 1373$ and the base $g = 2$ for a \hyperlink{d-h key}{Diffie-Hellman key exchange}. Alice sends Bob the value $A = 974$. Bob asks for your assistance, so you tell him to use the secret exponent $b = 871$.
    \begin{enumerate}
        \item What value $B$ should Bob send to Alice, and what is their secret shared value?
        \item Can you figure out Alice's secret exponent?
    \end{enumerate}
\end{exercise}

\begin{center}
    \textbf{Added (a) and (b) for clarity in Exercise 2.6.}
\end{center}

\sol{
    \begin{enumerate}
        \item Bob sends \(B = g^b = 2^{871} \equiv 805 \pmod{1373}\). Their shared secret value is \(974^{871} \equiv 397 \pmod{1373}\) (via Wolfram Alpha).
        \item To get Alice's secret exponent, we need to solve \(a\) by brute force. Thus, we need to solve the equation \(2^a \equiv 974 \pmod{1373}\). We find that \(a = 587\).
    \end{enumerate}
}

\begin{exercise}
    {2.7} Let $p$ be a prime and let $g$ be an integer. The \textit{Decision Diffie-Hellman Problem} is as follows. Suppose that you are given three numbers $A$, $B$, and $C$, and suppose that $A$ and $B$ are equal to
    \[
        A \equiv g^a \pmod{p} \quad \text{and} \quad B \equiv g^b \pmod{p},
    \]
    but that you do not necessarily know the values of the exponents $a$ and $b$. Determine whether $C$ is equal to $g^{ab} \pmod{p}$. Notice that this is different from the Diffie-Hellman problem described on page 69. The Diffie-Hellman problem asks you to actually compute the value of $g^{ab}$.
    \begin{enumerate}
        \item Prove that an algorithm that solves the Diffie-Hellman problem can be used to solve the decision Diffie-Hellman problem.
        \item Do you think that the decision Diffie-Hellman problem is hard or easy? Why? See Exercise 6.40 for a related example in which the decision problem is easy, but it is believed that the associated computational problem is hard.
    \end{enumerate}
\end{exercise}

\sol{
    \begin{enumerate}
        \item Since we know that \(A = g^a \pmod{p}\) and \(B = g^b \pmod{p}\), then we just have to compare \(C\) to \(g^{ab} \pmod{p}\). If \(C = g^{ab} \pmod{p}\), then the algorithm solves the decision Diffie-Hellman problem.
        \item Because we only know \(A\) and \(B\) it makes determining \(C\) hard because we only know \(g^{a+b}\) and \textit{not} \(g^{ab}\). Thus, the decision Diffie-Hellman problem is hard because it is difficult to determine whether \(C = g^{ab} \pmod{p}\) without knowing \(a\) and \(b\).
    \end{enumerate}
}

\begin{exercise}
    {2.8} Alice and Bob agree to use the prime $p = 1373$ and the base $g = 2$ for communications using the \hyperlink{Elgamal}{Elgamal public key cryptosystem}.
    \begin{enumerate}
        \item Alice chooses $a = 947$ as her private key. What is the value of her public key $A$?
        \item Bob chooses $b = 716$ as his private key, so his public key is
              \[
                  B \equiv 2^{716} \equiv 469 \pmod{1373}.
              \]
              Alice encrypts the message $m = 583$ using the random element $k = 877$. What is the ciphertext $(c_1, c_2)$ that Alice sends to Bob?
        \item Alice decides to choose a new private key $a = 299$ with associated public key
              \[
                  A \equiv 2^{299} \equiv 34 \pmod{1373}.
              \]
              Bob encrypts a message using Alice's public key and sends her the ciphertext $(c_1, c_2) = (661, 1325)$. Decrypt the message.
        \item Now Bob chooses a new private key and publishes the associated public key $B = 893$. Alice encrypts a message using this public key and sends the ciphertext $(c_1, c_2) = (693, 793)$ to Bob. Eve intercepts the transmission. Help Eve by solving the discrete logarithm problem $2^b \equiv 893 \pmod{1373}$ and using the value of $b$ to decrypt the message.
    \end{enumerate}
\end{exercise}

\sol{
    \begin{enumerate}
        \item \(p = 1373, \ g = 2, \ a = 947 \Rightarrow A \equiv 2^{947} \pmod{1373} \equiv 177\).
        \item \(c_1 \equiv 2^{877} \pmod{1373} \equiv 719, \ c_2 \equiv 583 \cdot 469^{877} \pmod{1373} \equiv 623\). Alice sends ``\((719, 623)\)'' to Bob.
        \item To decrypt, we can use the \hyperlink{thm:Extended Euclidean Algorithm}{EEA} for the equation \(352x \equiv 1 \pmod{1373}\), which gives \(1 = 352 \cdot 667 - 1373 \cdot 171\). Thus, the modular inverse of 352 modulo 1373 is 667 because \(352 \cdot 667 \equiv 1 \pmod{1373}\). Now we solve \(667 \cdot 623 \pmod{1373}\), which gives us 895.
        \item Solving for \(b\) in \(2^b \equiv 893 \pmod{1373}\) gives \(b = 219\). Now we can decrypt: 
        \[
        (c_1^a)^{-1} \cdot c_2 \equiv (693^{219})^{-1} \equiv 431^{-1} \cdot 793 \equiv 532 \cdot 793 \equiv 365 \pmod{1373}.
        \] Alice's private message to Bob is \(m = 365\).
    \end{enumerate}
}

\begin{exercise}
    {1.46}
    \begin{enumerate}
        \item Convert the 12-bit binary number $110101100101$ into a decimal integer between $0$ and $2^{12} - 1$.
              \setcounter{enumi}{4}

        \item Convert the decimal numbers $8734$ and $5177$ into binary numbers, combine them using XOR, and convert the result back into a decimal number.
    \end{enumerate}
\end{exercise}

For this exercise, I will be using these python functions that I wrote: 
\begin{lstlisting}[language=python]
    def compute_binary(num):
        binary_representation = []
        while num != 0:
            result = num % 2
            num = num // 2
            binary_representation.append(result)
        binary_str = ''.join(map(str, binary_representation))
        print(binary_str)

    def compute_decimal_int(binary_str):
        int_representation = int(binary_str, 2)
        print(int_representation)
\end{lstlisting}
% Note that this is just what \texttt{bin} does, but I am providing this function to show work for the problem. 

\sol{

    \begin{enumerate}
        \item 3429
              \setcounter{enumi}{4}
        \item Note the added 0 for the second binary number so the two numbers add properly. 
        \begin{align*}
            8734 & = 10001000011110 \\
            5177 & = 01010000111001 \\
            8734 \oplus 5177 & = 11011001010111\\
            13863 & = 11011001010111
        \end{align*}
    \end{enumerate}
}

\begin{exercise}
    {2.17} Use \hyperlink{2.21}{Shanks's babystep-giantstep} method to solve the following discrete logarithm problems. 
    \begin{enumerate}
        \item $11^x = 21$ in $\mathbb{F}_{71}$.
    \end{enumerate}
\end{exercise}

\sol{
    \begin{enumerate}
        \item \begin{enumerate}[label=(\arabic*)]
            \item Let \(m = \lceil \sqrt{70} \rceil = 9\)
            \item Create two lists: \begin{itemize}
                \item \textbf{Baby steps:} \[\{11^0, 11^1,\dots,11^9\} \pmod{71} \equiv \{1, \boxed{11}, 27, 15, 24, 29, 37, 31, 12, 38\}\]
                \item \textbf{Giant steps:} \[\{21, 21 \cdot 11^{-9}, 21 \cdot 11^{-18}, \dots\} \pmod{71} \equiv \{21, 5, 35, 32, \boxed{11}\}\]
            \end{itemize}
            \item Find a match between the two lists: \(\boxed{11}\)
            \item Substitute values for \(i + jm = 1 + 4(9) = 37\). So, \(11^{37} \equiv 21 \pmod{71}\)  
        \end{enumerate}  
    \end{enumerate}
}

\begin{exercise}
    {2.18} Solve each of the following simultaneous systems of congruences (or explain why no solution exists).
    \begin{enumerate}
        \setcounter{enumi}{1} % Start enumeration at (b)
        \item $x \equiv 137 \pmod{423}$ and $x \equiv 87 \pmod{191}$.
              \setcounter{enumi}{3} % Start enumeration at (d)
        \item $x \equiv 5 \pmod{9}$, $x \equiv 6 \pmod{10}$, and $x \equiv 7 \pmod{11}$.
    \end{enumerate}
\end{exercise}

%%
% Variables
% m_1 = 423
% m_2 = 191
% m = 80793
% n_1 = 191
% n_2 = 423
% a_1 = 137
% a_2 = 87
%%


\sol{
    \begin{enumerate}
        \setcounter{enumi}{1}
        \item \begin{enumerate}[label=(\arabic*)]
            \item Let \(m = 423 \cdot 191 = 80793\).
            \item Compute \(n_1 = \frac{m}{m_1} = \frac{80793}{423} = 191\), and \(n_2 = \frac{80793}{191} = 423\)
            \item Compute \(y_1 = 191^{-1} \pmod{423} \equiv 392\) and \(y_2 = 423^{-1} \pmod{191} \equiv 14\).
            \item Compute \(x = (137)(191)(392) + (87)(423)(14) \pmod{80793} \equiv 27209\)
        \end{enumerate}
        \setcounter{enumi}{3} % Start enumeration at (d)
        \item \begin{enumerate}[label=(\arabic*)]
            \item Let \(m = 9 \cdot 10 \cdot 11 = 990\).
            \item Compute \(n_1 = \frac{990}{9} = 110, \ n_2 = \frac{990}{10} = 99, \text{ and } n_3 = \frac{990}{11} = 90\).
            \item Compute \(y_1 = 110^{-1} \pmod{9} \equiv 5, \ y_2 = 99^{-1} \pmod{10} \equiv 9\), and \(y_3 = 90^{-1} \pmod{11} \equiv 6\)
            \item Compute \(x = (5)(110)(5) + (6)(99)(9) + (7)(90)(6) \pmod{990} = 986\)
        \end{enumerate}

    \end{enumerate}
}

\begin{exercise}
    {2.28} Use the \hyperlink{Pohlig-Hellman}{Pohlig-Hellman algorithm} (Theorem 2.31) to solve the discrete logarithm problem $g^x = a$ in $\mathbb{F}_p$ in each of the following cases.
    \begin{enumerate}
        \item $p = 433$, $g = 7$, $a = 166$.
    \end{enumerate}
\end{exercise}

\sol{
    We start by writing the given information into an equation that we can work with. Thus, \(7^x \equiv 166 \pmod{433}\). Since \(433\) is prime, \(\varphi(433) = 432\), which we can factor to \(16 \cdot 27\). Let \(x = a_0 + 16a_1\):
    \begin{align}
        (7^{a_0 + 16a_1})^{27} &\equiv 166^{27} \pmod{433} \\
        (7^{27a_0 + 432a_1}) &\equiv \\
        (7^{27a_0} \cdot 7^{432a_1}) &\equiv \\
        (7^{27})^{a_0}(7^{a_1})^{432} &\equiv \\ 
        (265)^{a_0} &\equiv 250 \pmod{433}
    \end{align}
    For (3.1), we got the expression by substituting \(x\) for \(a_0 + 16a_1\) for the exponent in \(g^x \equiv \dots\). From there, (3.2) --- (3.4) is simple algebra. Then for (3.5), because \(7^{a_1}\) is congruent to 1, \((7^{a_1})^{432} = 1\). Additionally, we take \(7^{27} \pmod{432}\) to get \(265^{a_0}\). Then, we do the same thing for the other side of the equation. Now, we can brute force this by setting \(a_1 = \{0,1,2,3,\dots, 27\}\). We find that when \(a_0 = 15\), \(265^{a_0} \equiv 250 \pmod{433}\). Seeing that we have \(a_0\), we need to find \(b_0\): 
    \begin{align*}
        (7^{b_0 + 27b_1})^{16} &\equiv 166^{16} \pmod{433} \\
        374^{b_0} &\equiv 335 \pmod{433}
    \end{align*} 
    Thus, through brute force, we find that \(b_0 = 20\). We can take our \(a_0\) and \(b_0\) to solve for \(x_1,x_2\): 
    \begin{align*}
        x_1 &=  a_0 + 16a_1 \\
        x_1 &= 15 \pmod{16}
    \end{align*} and \begin{align*}
        x_2 &= b_0 + 27b_1 \\
        x_2 &= 20 \pmod{27}
    \end{align*}
    At this point, we can solve the \hyperlink{thm:Chinese Remainder}{CRT}. 
    \begin{enumerate}
        \item Let \(m = 16 \cdot 27 = 432\).
        \item Compute \(n_1 = \frac{432}{16} = 27\) and \(n_2 = \frac{432}{27} = 16\).
        \item Compute \(y_1 = 27^{-1} \pmod{16} \equiv 3\) and \(y_2 = 16^{-1} \pmod{27} \equiv 22\).
        \item Compute \(x = (27)(15)(3) + (20)(16)(22) \pmod{432} = 47\)
    \end{enumerate} I relied on \href{https://www.youtube.com/watch?v=CHfP5tBbiAg}{this video} heavily.
}

