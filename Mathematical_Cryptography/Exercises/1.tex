\begin{exercise}
    {1.1}Build a cipher wheel as illustrated in Figure 1.1, but with an inner wheel that rotates, and use it to complete the following tasks. 
    \begin{enumerate}
        \item Encrypt the following plaintext using a rotation of 11 clockwise.
        \begin{center}
            ``A page of history is worth a volume of logic.''
        \end{center}
        \item Decrypt the following message, which was encrypted with a rotation of 7 clockwise.
        \begin{center}
            \texttt{AOLYLHYLUVZLJYLAZILAALYAOHUAOLZLJYLALZAOHALCLYFIVKFNBLZZLZ}
        \end{center}
    \end{enumerate}
\end{exercise}

\sol{
\begin{enumerate}
    \item \texttt{L ALRP ZQ STDEZCJ TD HZCES L GZWFXP ZQ WZRTN}
    \item THERE ARE NO SECRETS BETTER THAN THE SECRETES [sic] THAT EVERY BODY GUESSES \\
    In the encrypted text, ``Secrets'' is \texttt{ZLJYLAZ}. Then, they use an incorrect spelling of the word, \texttt{ZLJYLALZ}, of which has an extra `e' in it. That is what the ``[sic]'' is for. 
\end{enumerate}
}

\begin{exercise}
    {1.2}Decrypt each of the following Caesar encryptions by trying the various possible
shifts until you obtain readable text.
    \begin{enumerate}
        \item \texttt{LWKLQNWKDWLVKDOOQHYHUVHHDELOOERDUGORYHOBDVDWUHH}
        \item \texttt{UXENRBWXCUXENFQRLQJUCNABFQNWRCJUCNAJCRXWORWMB}
    \end{enumerate}
\end{exercise}

\sol{
\begin{enumerate}
    \item I THINK THAT I SHALL NEVER SEE A BILLBOARD LOVELY AS A TREE
    \item LOVE IS NOT LOVE WHICH ALTERS WHEN IT ALTERATION FINDS
\end{enumerate}
}


\begin{exercise}
    {1.3}For this exercise, use the simple substitution table given in Table 1.11.
    \begin{enumerate}
        \item Encrypt the plaintext message:
        \center{The gold is hidden in the garden}
    \end{enumerate}
\end{exercise}

\sol{
\begin{enumerate}
    \item \texttt{IBX FEPA QL BQAAXW QW IBX FSVAXW}
\end{enumerate}
}

\begin{exercise}
    {1.4} Each of the following messages has been encrypted using a simple substitution cipher. Decrypt them. For your convenience, we have given you a frequency table and a list of the most common bigrams that appear in the ciphertext. (If you do not want to recopy the ciphertexts by hand, they can be downloaded or printed from the web site listed in the preface.)
    \begin{enumerate}
        \item “A Piratical Treasure”
        \begin{flushleft}
            \texttt{JNRZR BNIGI BJRGZ IZLQR OTDNJ GRIHT USDKR ZZWLG OIBTM NRGJN} \\
            \texttt{IJTZJ LZISJ NRSBL QVRSI ORIQT QDEKJ JNRQW GLOFN IJTZX QLFQL} \\
            \texttt{WBIMJ ITQXT HHTBL KUHQL JZKMM LZRNT OBIMI EURLW BLQZJ GKBJT} \\
            \texttt{QDIQS LWJNR OLGRI EZJGK ZRBGS MJLDG IMNZT OIHRK MOSOT QHIJL} \\ 
            \texttt{QBRJN IJJNT ZFIZL WIZTO MURZM RBTRZ ZKBNN LFRVR GIZFL KUHIM} \\ 
            \texttt{MRIGJ LJNRB GKHRT QJRUU RBJLW JNRZI TULGI EZLUK JRUST QZLUK} \\ 
            \texttt{EURFT JNLKJ JNRXR S}
        \end{flushleft}
    \end{enumerate}
\end{exercise}

\sol{
\begin{enumerate}
    \item THESE CHARACTERS AS ONE MIGHT READILY GUESS FORM A CIPHER THAT IS TO SAY THEY CONVEY A MEANING BUT THEN FROM WHAT IS KNOWN OF CAPTAIN KIDD I COULD NOT SUPPOSE HIM CAPABLE OF CONSTRUCTING ANY OF THE MORE ABSTRUSE CRYPTOGRAPHS I MADE UP MY MIND AT ONCE THAT THIS WAS OF A SIMPLE SPECIES SUCH HOW EVER AS WOULD APPEAR TO THE CRUDE INTELLECT OF THE SAILOR ABSOLUTELY INSOLUBLE WITHOUT THE KEY    
\end{enumerate}
\href{https://planetcalc.com/8047/}{Solver}.

}

\begin{exercise}
    {1.5}Suppose that you have an alphabet of 26 letters. 
    \begin{enumerate}
        \item How many possible simple substitution ciphers are there?
        \item A letter in the alphabet is said to be fixed if the encryption of the letter is the letter itself. \textit{Show an example of how the pieces work together}
    \end{enumerate}
\end{exercise}

\sol{
\begin{enumerate}
    \item $26!$
    \item This is the formula used for solving for derangements (where $n$ is the number of elements in the set, and $!n$ is the number of derangements [Definition: A permutation with no fixed points]): $!n=n!\sum_{i=0}^{n}{\frac{(-1)^{i}}{i!}}$. From \href{https://en.wikipedia.org/wiki/Derangement}{Wikipedia}. For $n = 2$, we can run through the following:
    \begin{enumerate}[label=(\arabic*)]
        \item $i = 0$: $\frac{(-1)^0}{0!} = 1$
        \item $i = 1$: $\frac{(-1)^1}{1!} = -1$
        \item $i = 2$: $\frac{(-1)^2}{2!} = 0.5$
    \end{enumerate}
    Sum them together: $1 - 1 + 0.5 = 0.5$. Now we can get $!2$: $$!2 = 2! \times (1 - 1 + 0.5) = 2 \times 0.5 = 1$$
\end{enumerate}
}

\begin{exercise}
    {1.6}Let $a, b, c \in Z$. Use the definition of divisibility to directly prove the following properties of divisibility. (This is Proposition 1.4.)
    \begin{enumerate}
        \item If $a \mid b$ and $b \mid c$, then $a \mid c$.
        \item If $a \mid b$ and $b \mid a$, then $a = \pm b$.
        \item If $a \mid b$ and $a \mid c$, then $a \mid (b + c)$ and $a \mid (b - c)$.
    \end{enumerate}
\end{exercise}

\sol{
    \begin{enumerate}
        \item Let $a,b,c \in \Z$ such that $a \mid b$ and $b \mid c$. We know there exists an $n \in \Z$ such that $a \times n = b$. Similarly, $b \mid c$ means there exists an $k \in \Z$ such that $b\times k = c$. We can use the commutative property to show that:
        \begin{align*}
            k(an) &= (b)k \\
            ank &= bk \\
            bk &= c \\
            a(nk) &= c \\
            a &\mid c
        \end{align*}
        \item From the problem statement, we can intuitively ascertain that because both $a,b$ divide each other, then it must be the case that they are the same number. Moreover, because the criteria for dividing is not pertinent to whether the quotient is negative or positive, this number can be positive or negative. Now that we have an idea of what we are trying to accomplish, we can begin the proof: Let $a,b \in \Z$ such that $a \mid b$ and $b \mid a$. We know there exists an $n \in \Z$ such that $a \times n = b$. Similarly, $b \mid a$ means there exists an $k \in \Z$ such that $b\times k = a$. Then, we can utilize substitution to get the following:
        \begin{align*}
            bk &= a \\
            (an)k &= a 
        \end{align*} From here, we know that because $a$ is present on both sides of the equation, we should divide by $a$ to simplify. Thus, consider the following two cases.
        \begin{itemize}
            \item \textbf{Case 1:} $a \ne 0$ \\
            Since $a$ is not zero, we can divide both sides by $a$ to get $nk = 1$. Since, $n,k \in \Z$, we do not need to worry about fractional reciprocals. Instead, we know from the identity property of multiplication, that $n,k$ must both be $\pm 1$. \begin{itemize}
                \item \textbf{$n,k$ are both $+1$:} Then, $a = b \times 1 \text{ and } b = a \times 1$. Which simplifies to $a = b$ in both cases.
                \item \textbf{$n,k$ are both $-1$:} Then, $a = b \times (-1) \text{ and } b = a \times (-1)$ Which simplifies to $a = -b$ in both cases.
            \end{itemize}
            \item \textbf{Case 2:} $a = 0$ \\
            If $a = 0$, then $b = 0 \times n = 0$ and $0 = b \times k = 0$. Therefore, $a = b = 0$, and $a = \pm b$ is still true.
        \end{itemize}
        We have shown that in either case if $a \mid b$ and $b \mid a$, then $a = \pm b$.
        \item Because $a$ needs to be divisible by both $b$ and $c$, we know that there must exist an $n,k \in \Z$ such that $b = an$ and $c = ak$. Our goal is to get to the form, $a \times \text{ some integer } = b + c$ and $a \times \text{ some integer } = b - c$ so we can use the definition of divides to help us out here. Therefore, let us consider both $b + c$ and $b - c$ in two separate cases:
        \begin{itemize}
            \item \textbf{Case 1:} $b + c$
            \begin{align*}
                b + c &= (an) + (ak) \\
                b + c &= a(n + k) \\
                a &\mid (b + c) 
            \end{align*}
            \item \textbf{Case 2:} $b - c$
            \begin{align*}
                b - c &= (an) - (ak) \\
                b - c &= a(n - k) \\
                a &\mid (b - c) 
            \end{align*}
        \end{itemize}
        We have shown that if $a \mid b$ and $a \mid c$, then $a \mid (b + c)$ and $a \mid (b - c)$.
    \end{enumerate}
}

\begin{exercise}
    {1.7}Use a calculator and the method described in Remark 1.9 to compute the following quotients and remainders.
    \begin{enumerate}
        \item 34787 divided by 353.
        \item 238792 divided by 7843.
    \end{enumerate}
\end{exercise}

\sol{
    \begin{enumerate}
        \item $a = 34787$ and $b = 353$. Then $a / b \approx 98.54674220$, so $q = 98$ and $r = a - b \cdot q = 34787 - 353 \cdot 98 = 193$.
        \item $a = 238792$ and $b = 7843$. Then $a / b \approx 30.446512$, so $q = 30$ and $r = a - b \cdot q = 238792 - 7843 \cdot 30 = 3502$.
    \end{enumerate}
}

\begin{exercise}
    {1.9}Use the Euclidean algorithm to compute the following greatest common divisors.
    \begin{enumerate}
        \item $\gcd(291, 252)$.
        \item $\gcd(16261, 85652)$.
    \end{enumerate}
\end{exercise}

\sol{
\begin{enumerate}
    \item $\gcd(291,252)$
    \begin{enumerate}[label=(\arabic*)]
        \item $r_0 = 291$, $r_1 = 252$.
        \item $i = 1$.
        \item Divide $r_0$ by $r_1$ to get a quotient, $q_1$ and a remainder, $r_{2}$:
        \begin{align*}
            291 / 252 &= 1 = q_1 \\
            291 - (252 \times 1) &= 39 = r_2
        \end{align*}
        \item $r_2 \ne 0$. So, we continue.
        \item $i = 2 + 1 = 3$.
    \end{enumerate}
    \pfs
    \begin{enumerate}[label=(\arabic*)]
        \setcounter{enumii}{2}
        \item Divide $r_1$ by $r_2$ to get quotient, $q_2$ and a remainder, $r_3$:
        \begin{align*}
            252 / 39 &= 6 = q_2 \\
            252 - (39 \times 6) &= 18 = r_3
        \end{align*}
        \item $r_3 \ne 0$. So, we continue.
        \item $i = 3 + 1 = 4$.
    \end{enumerate}
    \pfs
    \begin{enumerate}[label=(\arabic*)]
        \setcounter{enumii}{2}
        \item Divide $r_2$ by $r_3$ to get quotient $q_3$ and a remainder, $r_4$:
        \begin{align*}
            39 / 18 &= 2 = q_3 \\
            39 - (18 \times 2) &= 3 = r_4
        \end{align*}
        \item $r_4 \ne 0$. So, we continue.
        \item $i = 3 + 1 = 5$.
    \end{enumerate}
    \pfs
    \begin{enumerate}[label=(\arabic*)]
        \setcounter{enumii}{2}
        \item Divide $r_3$ by $r_4$ to get quotient $q_4$ and a remainder, $r_5$:
        \begin{align*}
            18 / 3 &= 6 = q_4 \\
            18 - (3 \times 6) &= 0 = r_5
        \end{align*}
        \item $r_4 = 0$. So, we stop.
    \end{enumerate}
    We have found that the greatest common divisor is $3$.
    \item \textit{To cut back on paper, I am going to avoid reiterating the steps of 4 and 5. If there is a continuation in the enumeration process, then $r_i \ne 0$, and the process needs to continue}: $\gcd(16261,85652) \Rightarrow \gcd(85652,16261)$.
    \begin{enumerate}[label=(\arabic*)]
    \item
    \begin{align*}
        85652 / 16261 &= 5 = q_1 \\
        85652 - (16261 \times 5) &= 4347 = r_2
    \end{align*}
    \item 
    \begin{align*}
        16261 / 4347 &= 3 = q_2 \\
        16261 - (4347 \times 3) &= 3220 = r_3
    \end{align*}
    \item 
    \begin{align*}
        4347 / 3220 &= 1 = q_3 \\
        4347 - (3220 \times 1) &= 1127 = r_4
    \end{align*}
    \item 
    \begin{align*}
        3220 / 1127 &= 2 = q_4 \\
        3220 - (1127 \times 2) &= 966 = r_5
    \end{align*}
    \item 
    \begin{align*}
        1127 / 966 &= 1 = q_5 \\
        1127 - (966 \times 1) &= 161 = r_6
    \end{align*}
    \item 
    \begin{align*}
        966 / 161 &= 6 = q_6 \\
        966 - (161 \times 6) &= 0 = r_7 
    \end{align*}
    \end{enumerate}
    We have found that $\gcd(85652,16261) = 161$.
    
\end{enumerate}
}

\begin{exercise}
    {1.10} For each of the $\gcd(a, b)$ values in Exercise 1.9, use the extended Euclidean
algorithm (Theorem 1.11) to find integers $u$ and $v$ such that $au + bv = \gcd(a, b)$.
\end{exercise}

\sol{
\begin{enumerate}
    \item We need to solve for the various $u_i$ and $v_i$. We will start at $i = 2$. \\
    \begin{tabular}{c|c|c|c|c|c|c}
        $i$ & $u_i$ & \textit{Formula} & \textit{Evaluation} & $v_i$ & \textit{Formula} & \textit{Evaluation} \\ \hline
        \textit{2} & $1$ & $u_0 - q_1 \times u_1$ & $1 - 1 \times 0$ & $-1$ & $v_0 - q_1 \times v_1$ & $0 - 1 \times 1$ \\ \hline
        \textit{3} & $-6$ & $u_1 - q_2 \times u_2$ & $0 - 6 \times 1$ & $7$ & $v_1 - q_2 \times v_2$ & $1 - 6 \times (-1)$ \\ \hline
        \textit{4} & $13$ & $u_2 - q_3 \times u_3$ & $1 - 2 \times (-6)$ & $-15$ & $v_2 - q_3 \times v_3$ & $-1 - 2 \times 7$ \\
    \end{tabular} \\

    \pfs
    
    Thus, we can now fill out the table in full:
    \begin{tabular}{c|c|c|c|c|c}
        \textit{i} & \textit{$r_i$} & \textit{$q_i$} & $r_{i + 1}$ & \textit{$u_i$} & \textit{$v_i$} \\ \hline
        \textit{0} & $291$ & $-$ & $-$ & $1$ & $0$ \\ \hline
        \textit{1} & $252$ & $1$ & $39$ & $0$ & $1$ \\ \hline
        \textit{2} & $39$ & $6$ & $18$ & $1$ & $-1$ \\ \hline
        \textit{3} & $18$ & $2$ & $3$ & $-6$ & $7$ \\ \hline
        \textit{4} & $3$ & $6$ & $0$ & $13$ & $-15$ \\
    \end{tabular} \\

    Now, we need to solve: $au + bv = \gcd(a,b) \Rightarrow 291(13) + 252(-15) = 3$. $3$ matches the gcd that we found in Exercise 1.9, so this is the correct solution.
    
    \item 
    
    \begin{tabular}{c|c|c|c|c|c}
        \textit{i} & \textit{$r_i$} & \textit{$q_i$} & $r_{i + 1}$ & \textit{$u_i$} & \textit{$v_i$} \\ \hline
        \textit{0} & $85652$ & $-$ & $-$ & $1$ & $0$ \\ \hline
        \textit{1} & $16261$ & $5$ & $4347$ & $0$ & $1$ \\ \hline
        \textit{2} & $4347$ & $3$ & $3220$ & $1$ & $-5$ \\ \hline
        \textit{3} & $3220$ & $1$ & $1127$ & $-3$ & $16$ \\ \hline
        \textit{4} & $1127$ & $2$ & $966$ & $4$ & $-21$ \\ \hline
        \textit{5} & $966$ & $1$ & $161$ & $-11$ & $58$ \\ \hline
        \textit{6} & $161$ & $6$ & $0$ & $15$ & $-79$ \\
    \end{tabular} \\

    $85652(15) + 16261(-79) = 161$.    
\end{enumerate}
}

\begin{exercise}
    {1.11} Let $a$ and $b$ be positive integers.
    \begin{enumerate}
        \item Suppose that there are integers $u$ and $v$ satisfying $au + bv = 1$. Prove that $\gcd(a, b) = 1$.
    \end{enumerate}
\end{exercise}

\expf{
\begin{enumerate}
    \item Suppose there are integers $a,b,u,v$ such that $av + bv = 1$. Assume $d \in \Z$ such that $d = \gcd(a,b)$. Since $d$ divides both $a$ and $b$ by definition of common divisor, it must also divide $av$ and $bv$ by definition of divisibility. Moreover, because $au + bv = 1$ and $d$ is a common divisor of both $av$ and $bv$, it must also divide $1$ by Proposition 1.4 (c). Then, the only positive integer that divides $1$ is $1$ itself, so it must be the case that $d = 1$. Therefore, since $d = 1$ and $\gcd(a,b) = d$, it follows that $\gcd(a,b) = 1$. \qedhere
\end{enumerate}
}

\begin{exercise}
    {1.14}Let $m \geq 1$ be an integer and suppose that $$a_1 \equiv a_2 \ (\text{mod } m) \text{ and } b_1 \equiv b_2 \ (\text{mod } m).$$ Prove that $$a_1 \pm b_1 \equiv a_2 \pm b_2 \ (\text{mod } m) \text{ and } a_1 \cdot b_1 \equiv a_2 \cdot b_2 \ (\text{mod } m).$$ (This is Proposition 1.13(a).)
\end{exercise}

\expf{
    Let $m \geq 1$ be an integer and suppose that $a_1 \equiv a_2 \ (\text{mod } m) \text{ and } b_1 \equiv b_2 \ (\text{mod } m)$. From the definition of modulo, we know the difference of $a_1 - a_2$ and $b_1 - b_2$ is divisible by $m$. 
    \begin{itemize}
        \item \textbf{Addition:} We want to show that $a_1 + b_1 \equiv a_2 + b_2$. So, our goal is to achieve $m \mid ((a_1 + b_1) - (a_2 + b_2))$. Thus, consider $(a_1 + b_1) - (a_2 + b_2)$. We can distribute the minus sign to get $(a_1 - a_2) + (b_1 - b_2)$. From Proposition 1.4 (c), because we know that $m \mid (a_1 - a_2)$ and $m \mid (b_1 - b_2)$, we can write this as $m \mid ((a_1 - a_2) + (b_1 - b_2))$ which implies $m \mid ((a_1 + b_1) - (a_2 + b_2))$. This shows $a_1 + b_1 \equiv a_2 + b_2 \ (\text{mod } m)$.
        \item \textbf{Subtraction:} Similarly to addition, we want to show $a_1 - b_1 \equiv a_2 - b_2 \ (\text{mod } m)$. Thus, consider $(a_1 - b_1) - (a_2 - b_2) $ which implies $ (a_1 - a_2) - (b_1 - b_2)$. From Proposition 1.4 (c), because we know that $m \mid (a_1 - a_2)$ and $m \mid (b_1 - b_2)$, we can write this as $m \mid ((a_1 - a_2) - (b_1 - b_2))$ which implies $m \mid ((a_1 - b_1) - (a_2 - b_2))$, so $a_1 - b_1 \equiv a_2 - b_2 \ (\text{mod } m)$.
    \end{itemize}
    Therefore we have shown $a_1 \pm b_1 \equiv a_2 \pm b_2 \ (\text{mod } m)$
    \begin{itemize}
        \item \textbf{Product} We want to show that $a_1 \cdot a_2 \equiv a_2 \cdot b_2 \ (\text{mod } m)$. Thus, consider $a_1 \cdot b_1 - a_2 \cdot b_2$:
    \begin{align*}
        a_1 \cdot b_1 - a_2 \cdot b_2 &= a_1 \cdot b_1 - a_1 \cdot b_2 + a_1 \cdot b_2 - a_2 \cdot b_2 \\
        &= a_1 \cdot (b_1 - b_2) + b_2 \cdot (a_1 - a_2)
    \end{align*}
    Because $m \mid (b_1 - b_2)$ and $m \mid (a_1 - a_2)$, we know from the definition of division that when we multiply those numbers by an integer like $a_1$ and $b_2$, $m$ still divides the expression. Hence, $m \mid (a_1 \cdot (b_1 - b_2))$ and $m \mid (b_2 \cdot (a_1 - a_2))$. Therefore, $m \mid (a_1 \cdot b_1 - a_2 \cdot b_2)$ and $a_1 \cdot b_1 \equiv a_2 \cdot b_2 \ (\text{mod } m)$. \qedhere
    \end{itemize}
}

\begin{exercise}
    {1.16}Do the following modular computations. In each case, fill in the box with an integer between 0 and $m - 1$, where $m$ is the modulus.
    \begin{enumerate}
        \item $347 + 513 \equiv \boxed{\phantom{000}} \ (\text{mod } 763)$.
        \setcounter{enumi}{2}
        \item $153 \cdot 287 \equiv \boxed{\phantom{000}} \ (\text{mod } 353)$
        \setcounter{enumi}{4}
        \item $5327 \cdot 6135 \cdot 7139 \cdot 2187 \cdot 5219 \cdot 1873 \equiv \boxed{\phantom{000}} \ (\text{mod } 8157)$ (\textit{Hint:} After each multiplication, reduce modulo $8157$ before doing the next multiplication.)
        \setcounter{enumi}{6}
        \item $373^6 \equiv \boxed{\phantom{000}} \ (\text{mod } 581)$.
    \end{enumerate}
\end{exercise}

\sol{
    \begin{enumerate}
        \item $347 + 513 \ (\text{mod } 763) = 97$
        \setcounter{enumi}{2}
        \item $153 \cdot 287 \ (\text{mod } 353) = 139$
        \setcounter{enumi}{4}
        \item 
        \begin{align*}
            5327 \cdot 6135 \ (\text{mod } 8157) &= 4203 \\
            4203 \cdot 7139 \ (\text{mod } 8157) &= 3771 \\
            3771 \cdot 2187 \ (\text{mod } 8157) &= 450 \\
            450 \cdot 5219 \ (\text{mod } 8157) &= 7491 \\
            7491 \cdot 1873 \ (\text{mod } 8157) &= \boxed{603}
        \end{align*}
        \setcounter{enumi}{6}
        \item $373^6 \ (\text{mod } 581) = 463$
    \end{enumerate}
}

\begin{exercise}
    {1.17}Find all values of $x$ between $0$ and $m - 1$ that are solutions of the following congruences. (\textit{Hint:} If you can't figure out a clever way to find the solution(s), you can just substitute each value $x = 1$, $x = 2,\dots$, $x = m - 1$ and see which ones work.)
    \begin{enumerate}
        \item $x + 17 \equiv 23 \ (\text{mod } 37)$.
        \setcounter{enumi}{2}
        \item $x^2 \equiv 3 \ (\text{mod } 11)$
        \setcounter{enumi}{6}
        \item $x \equiv 1 \ (\text{mod } 5)$ and also, $x \equiv 2 \ (\text{mod } 7)$. (Find all solutions modulo 35, that is, find the solutions satisfying $0 \leq x \leq 34$.
    \end{enumerate}
\end{exercise}

\sol{
    \begin{enumerate}
        \item $x = 6$
        \setcounter{enumi}{2}
        \item We know that $x$ cannot be any number who's square does not exceed 11 because we cannot square a number to get $3$ (other than $\sqrt{3}$, but these are only the integers, so we cannot use that). Hence, $x \ne 1,2,3$ because we know that the results of these, $1^2 = 1$, $2^2 = 4$, $3^2 = 9$ are not equivalent to 3. Let's try some more values: $x = 4$: $4^2 = 16 \ (\text{mod } 11) = 5 \ \slashed{\equiv} \ 3$; $x = 5$: $5^2 = 25 \ (\text{mod } 11) = 3 \equiv 3$; $x = 6$: $6^2 = 36 \ (\text{mod } 11) = 3 \equiv 3$; $x = 7$: $7^2 = 49 \ (\text{mod } 11) = 5 \ \slashed{\equiv} \ 3$; $x = 8$: $8^2 = 64 \ (\text{mod } 11) = 9 \ \slashed{\equiv} \ 3$; $x = 9$: $9^2 = 81 \ (\text{mod } 11) = 4 \ \slashed{\equiv} \ 3$; $x = 10$: $10^2 = 100 \ (\text{mod } 11) = 1 \ \slashed{\equiv} \ 3$. 

        Thus, we have it that $x = 5,6$.  
        \setcounter{enumi}{6}
        \item Because we have $x \equiv 1 \ (\text{mod } 5)$, verifying correct $x$'s are straightforward. All we need to check for is if a multiple of $5 + 1$ satisfies $x \equiv 2 \ (\text{mod } 7)$. Thus, the only solution is $x = 16$
    \end{enumerate}
}

\begin{exercise}
    {1.19}Prove that if $a_1$ and $a_2$ are units modulo $m$, then $a_1a_2$ is a unit modulo $m$.
\end{exercise}

\expf{
    Suppose $a_1$ and $a_2$ are units modulo $m$. This means $a \in \Z/m\Z \colon gcd(a,m) = 1$. In other words, $a_1b_1 \equiv 1 \ (\text{mod } m)$ and $a_2b_2 \equiv 1 \ (\text{mod } m)$ for some $b_1, b_2 \in \Z$. When we multiply the equations together, we get $(a_1b_1)(a_2b_2) \equiv 1 \ (\text{mod } m)$ which can be rewritten as $(a_1a_2)(b_1b_2) \equiv 1 \ (\text{mod } m)$. We can multiply $b_1$ and $b_2$ to get an integer $b_3$. Thus, when we multiply $a_1a_2$ by $b_3$ and get $1$, we have shown that $b_3$ is a multiplicative inverse, and $a_1a_2$ is a unit modulo $m$.  
}

\begin{exercise}
    {(Additional)}Decide whether each of the following is a group:
    \begin{enumerate}
        \item \texttt{All 2x2 matrices with real number entries} with operation matrix addition
        \item \texttt{All 2x2 matrices with real number entries} with operation matrix multiplication
    \end{enumerate}

\end{exercise}

\sol{
    \begin{enumerate}
        \item \textbf{Matrix Addition:} \cmark
        \begin{enumerate}[label=(\arabic*)]
            \item \textbf{Closure:} For addition to work between matrices, they must be of dimension $2 \times 2 + 2 \times 2$. Therefore, the dimensions do not change, and it is closed.
            \item \textbf{Associativity:} $2 \times 2$ matrix addition is associative, as it inherits this property from the properties of matrices.
            \item \textbf{Identity Element:} We can add a matrix $Z$ that consists of only 0s to a matrix $A$, and matrix $A$ will remain unchanged.
            \item \textbf{Inverse Element:} True. Consider the matrices, 
            $\begin{bmatrix}
                a & b \\
                c & d
            \end{bmatrix}$, and 
            $\begin{bmatrix}
                -a & -b \\
                -c & -d 
            \end{bmatrix}$. When we add these two together we get
            $\begin{bmatrix}
                0 & 0 \\
                0 & 0
            \end{bmatrix}$. This shows that $2\times 2$ matrices have additive inverses.


            
        \end{enumerate}
        \item \textbf{Matrix Multiplication:} \xmark
        \begin{enumerate}[label=(\arabic*)]
            \item \textbf{Closure:} The dimensions will stay the same during multiplication because it is an $n \times n$ matrix. 
            \item \textbf{Associativity:} $2 \times 2$ matrix multiplication is associative, as it inherits this property from the properties of matrices.
            \item \textbf{Identity Element:} True. Consider the identity matrix, $I_2 =
            \begin{bmatrix}
                1 & 0 \\
                0 & 1
            \end{bmatrix}$. When we multiply a matrix $\begin{bmatrix}
                a & b \\
                c & d
            \end{bmatrix}$ by $I$, we get $$\begin{bmatrix}
                a & b \\
                c & d
            \end{bmatrix} \cdot
            \begin{bmatrix}
                1 & 0 \\
                0 & 1
            \end{bmatrix} = \begin{bmatrix}
                a & b \\
                c & d
            \end{bmatrix},$$
            \item \textbf{Inverse Element:} False. Matrices with a non-zero determinant fail this criteria. Consider the matrix $B =$
            $\begin{bmatrix}
                1 & 2 \\
                3 & 4 
            \end{bmatrix}$. The determinant would be $\det((1)(4) - (2)(3) = -2$. Therefore, this matrix would not have an inverse. 
        \end{enumerate}
    \end{enumerate}
}

\begin{exercise}
    {(Additional)}Is \texttt{All 2x2 matrices with real number entries} a ring with operations matrix addition and matrix multiplication? Justify your answer.
\end{exercise}

\sol{ \cmark
    \begin{enumerate}[label=(\arabic*)]
        \item \textbf{Additive Closure:} True. (See the previous exercise (a), (1)).
        \item \textbf{Additive Associativity:} True. Inherited from the properties of matrices.
        \item \textbf{Additive Identity:} True. (See the previous exercise (a), (3)).
        \item \textbf{Additive Inverse:} True. You can take the difference between a matrix and its inverted duplicate (e.g., $-[A]$) and get $0$.
        \item \textbf{Multiplicative Closure:} True. (See the previous exercise (b), (4)).
        \item \textbf{Distributive Property:} True. While this will pose an error on a calculator, you can do the equivalent: $[A]([B] + [C]) = [A][B] + [A][C]$. This is true because you are still taking the summation of each $a_{ij}$, $b_{ij}$ and $c_{ij}$. 
    \end{enumerate}
}