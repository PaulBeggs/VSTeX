\setcounter{chapter}{99}
\setcounter{ExampleCounter}{50}


\section*{Chapter 2 Review}

\begin{enumerate}
    \item Discrete logarithm
    \item Diffie-Hellman key
    \item El-gamal Encryption
    \item Attacks on DLP
          \begin{itemize}
              \item Babystep/Giantstep
              \item Chinese Remainder Th
              \item Pohlig-Hellman Alg
          \end{itemize}
\end{enumerate}



\begin{center}
    \textbf{Discrete Logarithm}
\end{center}

Solve for \(g^x \equiv h \pmod{p}\) by using logarithms. Specifically, find for an example, if we have \(2^x \equiv 10 \pmod{11}\), we would find \(\log_2(10) \pmod{11}\). Then, we would solve for various powers of 2: \(2^1 = 2 \pmod{11}\)\dots up to \(2^5 = 10 \pmod{11}\). Thus, we have found an \(x\), \(x = 5\) for which \(2^x \equiv 10 \pmod{11}\) is true.


\begin{center}
    \textbf{Diffie-Hellman Keys Exchange}
\end{center}

Straightforward algorithm that really only requires computation of 4 numbers. That is, if we are given \(p, \ g, \ a, \ b\), where \(p\) is the modulus and \(g\) is the agreed primitive root base and \(A\) and \(B\) are respective private keys. (Recall the \hyperlink{thm:Primitive Root Theorem}{Primitive Root Theorem} for how we find \(g\).)

This algorithm involves solving the discrete logarithm problem because if Eve were to obtain \(a\) and \(b\), she would still have to calculate \(a = \log_g(A)\) and \(b = \log_g(B)\) where \(A\) and \(B\) are respective public keys.

Formally, the algorithm goes like this:

\begin{enumerate}
    \item Choose large prime \(p\) and primitive root \(g\) and make a public key, \(k_{\text{pub}} = (p,g)\).
    \item Pick ints \(a\) and \(b\) such that \(k_{\text{priv} A} = a\) and \(k_{\text{priv} B} = b\). Compute \(g^a \pmod{p} = A\) and \(g^b \pmod{p} = B\).
    \item Exchange \(A\) and \(B\).
    \item Compute \(B^a \pmod{p}\) to get \(A'\), and compute \(A^b \pmod{p}\) to get \(B'\).
    \item Now \(A' = B'\) and you have created a shared key.
\end{enumerate}


\begin{center}
    \textbf{Elgamal Public Key Cryptosystem}
\end{center}

This algorithm is a little more convoluted. To set the stage, for Alice and Bob to communicate, only Alice will use a private key \(a\) to send a private message. (In the other instance, Bob would calculate some private key \(b\), and the roles would be swapped. For this instance, we will only be  doing \textit{one} message transfer.) Also notice that we have similar variables from the D-H key exchange, \(p\) and \(q\) which are both publicly known information. Alice will publish her public key so Bob can use it. She calculates it with \(A \equiv g^a \pmod{p}\)---notice the resemblance to D-H.

Now, suppose Bob wants to encrypt a message \(m\) to send to Alice. This message will be between \(2\) and \(p\) (so our message must not be greater than \(p\)). Bob randomly chooses a number \(k \pmod{p}\) to encrypt \textit{one} message, and then he discards it. Then, Bob takes his plaintext message \(m\), his random \(k\) and Alice's public key \(A\), and uses them to compute to quantities \(c_1 \equiv g^k \pmod{p}\) and \(c_2 \equiv mA^k \pmod{p}\). Bob now sends the pair of numbers \(c_1, c_2\) to Alice.

Alice decrypts this message because she knows \(a\). She solves the quantity \(x \equiv (c^a_1)^{-1} \pmod{p}\). She solves this quantity by computing \(c^{p - 1 - a}_1 \pmod{p}\) using the fast power algorithm. Finally, Alice multiplies this \(x\) by \(c_2\) to get \(m\). \textbf{Refer to \hyperlink{Elgamal}{this table} for a more precise algorithm.}

\begin{center}
    \textbf{Shanks's Babystep-Giantstep Algorithm}
\end{center}

Refer to \hyperlink{2.21}{Proposition 2.21} for the algorithm. In this section, we will be going over specific examples for the \hyperlink{Discrete Logarithm Problem}{DLP}.

\begin{example}
    {Babystep-Giantstep Exam Practice} ttt
\end{example}

\begin{center}
    \textbf{Chinese Remainder Theorem}
\end{center}

In general, to solve for CRT such that \(x \equiv a_1 \ (\text{mod } m_1)\dots, x \equiv a_k \ (\text{mod } m_k)\) we follow the algorithm below:
\begin{enumerate}
    \item Let \(m = m_1 \cdot m_2 \cdots m_k\).
    \item Take \(n_i = \frac{m}{m_i}\).
    \item  Check to see if there is a solution, \(y_i\). \(y_i = n_i^{-1} \ (\text{mod } m_i)\). Note that the inverse exists because \(m_i\) and \(n_i\) are relatively prime.
    \item Compute \(x = a_1n_1y_1 + a_2n_2y_2 + \cdots + a_kn_ky_k \ (\text{mod } m)\).
\end{enumerate}

\begin{center}
    \textbf{Pohlig-Hellman Algorithm}
\end{center}

% Suppose that \(N\) factors into a product of prime powers as \(N = q^{e_1}_1 \cdot q^{e_2}_2 \cdots  q^{e_t}_t\). Now, we follow the algorithm as shown below:
% \begin{enumerate}
%     \item For each \(1 \leq i \leq t\), let 
%     \[
%         g_i = g^{N / q^{e_i}_i}\quad \textit{ and }\quad h_i = h^{N / q^{e_i}_i}
%     \]
%     Notice that \(g_i\) has prime power order \(q^{e_i}_i\), so we use the given algorithm to solve the discrete logarithm problem \(g^y_i = h_i\). Let \(y = y_i\) be a solution for the discrete logarithm problem.
%     \item Use the Chinese remainder theorem to solve 
%     \[\begin{array}{cccc}
%         x \equiv y_1 \pmod{q^{e^1}_1}, & x \equiv y_2 \pmod{q^{e^2}_2}, & \dots, & x \equiv y_t \pmod{q^{e^t}_t}
%     \end{array}\]
% \end{enumerate}

\begin{example}
    {Pohlig-Hellman Algorithm} (Continuation of \hyperlink{exerc:2.18}{Exercise 2.18}.) Use the Pohlig-Hellman Algorithm to solve the discrete logarithm problem \(g^x = a\) in \(\F_p\) in each of the following cases:
    \begin{enumerate}[label=(\alph*)]
        \setcounter{enumi}{1}
        \item \(p = 746497\), \(g = 10\), \(a = 243278\)
        \item \(p = 41022299\), \(g = 2\), \(a = 39183497\) (Hint: \(p = 2 \cdot 29^5 + 1\))
        \item \(p = 1291799\), \(g = 17\), \(a = 192988\) (Hint: \(p - 1\) has factor of 709)
    \end{enumerate}
\end{example}

\section*{Chapter 3 Review}

\begin{enumerate}
    \item Semi-primes \((N = pq)\)
    \item RSA-Encryption and Decryption
    \item Roots of \(pq\)
    \item Primality testing
          \begin{itemize}
              \item FLT
              \item M-R
          \end{itemize}
    \item Factoring semi-primes
          \begin{itemize}
              \item Pollard's \(p - 1\) method
                    Difference of 2 squares
          \end{itemize}
\end{enumerate}

\begin{center}
    \textbf{Semi-Primes}
\end{center}

\begin{example}
    {Congruence Ex From Book}Solve the congruence
    \[
        x^{1583} \equiv 4714 \pmod{7919}
    \]
\end{example}

\lesol{
    Because \(p = 7919\) is prime, we need to solve the congruence
    \[
        1583d \equiv 1 \pmod{7918} \quad \Rightarrow \quad d \equiv 1583^{-1} \pmod{7918}
    \]

    (Notice that for \(d\), we use a modulus \(p - 1\)!) We can use the extended \hyperlink{thm:Extended Euclidean Algorithm}{Extended Euclidean Algorithm} for to solve for \(d\):
    For the extended Euclidean algorithm:
    \begin{center}
        \begin{tabular}{ccccccc}
            \(n\) & \(b\) & \(q\) & \(r\) & \(t_1\) & \(t_2\) & \(t_3\)           \\ \toprule
            7918  & 1583  & 5     & 3     & 0       & 1       & \(-5\)            \\ \midrule
            1583  & 3     & 527   & 2     & 1       & \(-5\)  & 2636              \\ \midrule
            3     & 2     & 1     & 1     & \(-5\)  & 2636    & \(\boxed{-2641}\) \\ \midrule
            2     & 1     & 2     & 0     &         &         &                   \\ \midrule
        \end{tabular}
    \end{center} A step by tep break down for how this table got its values:
    \begin{enumerate}[label=\arabic*.]
        \item \textbf{Initial Setup}

              Start with \(t_1 = 0\) and \(t_2 = 1\). Calculate \(t_3\) with the recursive formula \(t_3 = t_1 - q \cdot t_2\) where \(q\) is the quotient from the division of the 2 numbers (\(n\) divided by \(b\)), and the \(t\) values shift as you move to the next row.

        \item \textbf{Calculate \(t_3\) Values}

              \textit{First row:}
              \begin{itemize}
                  \item We start with \(n = 7919\) and \(b = 1583\).
                  \item The quotient \(q = \lfloor \frac{7919}{1583} \rfloor = 5\).
                  \item The values of \(t_1 = 0\) and \(t_2 = 1\) are given by default.
                  \item Now, calculate the \(t_3\) using the formula
                        \[
                            t_3 = t_1 - q \cdot t_2 = 0 - 5 \cdot 1 = -5
                        \]
                  \item This leaves us with \(t_2 = 1, \ t_3 = -5\). These will become \(t_1\) and \(t_2\) in the next row.
              \end{itemize}
              \textit{Second row:}
              \begin{itemize}
                  \item Now, \(n = 1583\) and \(b = 3\).
                  \item The quotient \(q = \lfloor \frac{1583}{3} \rfloor = 527\).
                  \item The values of \(t_1 = 1\) and \(t_2 = -5\) from the previous row.
                  \item Now, calculate the \(t_3\) using the formula
                        \[
                            t_3 = t_1 - q \cdot t_2 = 1 - 527 \cdot -5 = 2636
                        \]
                  \item This leaves us with \(t_2 = -5, \ t_3 = 2636\). These will become \(t_1\) and \(t_2\) in the next row.
              \end{itemize}
              \textit{Third row:}
              \begin{itemize}
                  \item Now, \(n = 3\) and \(b = 2\).
                  \item The quotient \(q = \lfloor \frac{3}{2} \rfloor = 1\).
                  \item The values of \(t_1 = -5\) and \(t_2 = 2636\) from the previous row.
                  \item Now, calculate the \(t_3\) using the formula
                        \[
                            t_3 = t_1 - q \cdot t_2 = -5 - 1 \cdot 2636 = -2641
                        \]
              \end{itemize}
              \textit{Fourth row:}
        \item Now, \(n = 2\) and \(b = 1\).
        \item The quotient \(q = \lfloor \frac{2}{1} \rfloor = 2\).
        \item Because get a remainder of 0 from the previous calculation, we stop, and record the largest \(t\) value.
    \end{enumerate}
    Now we have our \(t = -2641\), but we still need to find the modulo. Simply subtract 7919 by 2641, and you get the inverse, 5277. Thus, \(d = 5277\). Then we solve for
    \[
        x^{5277} \equiv 4714 \pmod{7919} \quad \Rightarrow \quad x \equiv 4714^{5277} \pmod{7919} \quad \Rightarrow \quad x \equiv 6059.
    \]
}

\wrpprop{3.5}{Let \(p\) and \(q\) be distinct primes and let \(e \geq 1\)   satisfy
    \[
        \gcd(e, (p - 1)(q - 1)) = 1.
    \]
    Then, \refprop{1.13} tells us that \(e\) has an inverse modulo \((p-1)(q-1)\), say
    \[
        de \equiv 1 \pmod{(p-1)(q-1)}.
    \]
    Then the congruence
    \[
        x^e \equiv c \pmod{pq}
    \]
    has the unique solution \(x \equiv c^d \pmod{pq}\).
}%

\begin{example}
    {Congruence with p and q}Solve the congruence \(x^{17389} \equiv 43927 \pmod{64349}\)
\end{example}

\lesol{
    We factor the number 64349 into its prime factors so we can subtract 1 from both factors to get the value of the modulus for \(d\). We get
    \begin{align*}
        p \cdot q             & = 64349 = 229 \cdot 281 \\
        (p - 1) \cdot (q - 1) & = 63840 = 228 \cdot 280
    \end{align*}
    From here, we need to solve the following equation:
    \[
        17389d \equiv 1 \pmod{63840},
    \]
    We solve this \(d\) to be 53509. Then, \refprop{3.5} tells us that \(x \equiv 43927^{53509} \equiv 14458 \pmod{64349}\) is the solution to \(x^{17389} \equiv 43927 \pmod{64349}\)

    Now, we could have saved ourselves some work by calculating \(\gcd(p-1, q-1)\). This gives \(\gcd(228, 280) = 4\), so \((p-1)(q-1)/g = (228)(280)/4 = 15960\), which means that we can find a value \(d\) by solving the congruence
    \[
    17389d \equiv 1 \pmod{15960}.
    \]
    The solution is \(d \equiv 5629 \pmod{15960}\), and then 
\[
x \equiv 43927^{5629} \equiv 14458 \pmod{64349}.
\]
}

\begin{center}
    \textbf{RSA Encryption and Decryption}
\end{center}

The algorithm for RSA key generation, encryption, and decryption is as follows: \\

\textit{Key Generation:}
\begin{enumerate}
    \item Choose two distinct primes \(p\) and \(q\), and compute \(N = pq\).
    \item Choose encryption exponent \(e\) with \(\gcd(e,(p-1)(q-1)) = 1\). 
    \item Publish \(N = pq\) and \(e\).
\end{enumerate}

\textit{Encryption:}
\begin{enumerate}
    \item Choose plaintext \(m\). 
    \item Use the recipient's public key \((N,e)\) to compute the ciphertext \(c \equiv m^e \pmod{N}\).
    \item Send \(c\) to the recipient.
\end{enumerate}

\textit{Decryption:}
\begin{enumerate}
    \item Compute \(d\) satisfying \(ed \equiv 1 \pmod{(p-1)(q-1)}\).
    \item Compute \(m' \equiv c^d \pmod{N}\).
    \item Then \(m'\) equals the plaintext \(m\).
\end{enumerate}

\begin{example}
    {RSA} (a) \textbf{Key Creation} Use two secret primes \(p = 1223\), \(q = 1987\) to make a public modulus. Also, use the public encryption exponent \(e = 948047\) with the property that \(\gcd(e,(p - 1)(q - 1)) = 1\). (b) \textbf{Encryption} Use the plaintext message \(m = 1070777\) to compute the message to ciphertext. (c) \textbf{Decryption} Use the ciphertext to decrypt the message.
\end{example}

\lesol{
    \begin{enumerate}[label=(\alph*)]
        \item Calculate \(N = p \cdot q = 1223 \cdot 1987 = 2430101\). This is the public modulus that will be sent to person B. Now, we need to test if our \(e = 948047\) works with our \(p - 1\) and \(q - 1\): 
        \[
            \gcd(948047, 1222 \cdot 1986) = 1;
        \]
        it does, so we can move on to encryption.
        \item Now that person A has calculated the public modulus, they can encrypt their message. The message is \(m = 1070777\). Person A uses the public key \((N,e) = (2430101, 948047)\) to encrypt the message:
        \begin{align*}
            c &\equiv m^e \pmod{N} \\
            c &\equiv 1070777^{948047} \pmod{2430101} \\
            c &\equiv 1473513 \pmod{2430101}.
        \end{align*}
        Person A sends the ciphertext \(c = 1473513\) to person B.
        \item Person B receives the ciphertext and decrypts it using their private key. They calculate the decryption exponent \(d\) such that \(ed \equiv 1 \pmod{(p - 1)(q - 1)}\):
        \begin{align*}
            ed &\equiv 1 \pmod{(p - 1)(q - 1)} \\
            948047 \cdot d &\equiv 1 \pmod{2426892} \\
            d &\equiv 948047^{-1} \pmod{2426892} \\
            d &\equiv 1051235 \pmod{2426892}.
        \end{align*} Person B found that \(d = 1051235\). Now, they take the ciphertext \(c = 1473513\) and compute the plaintext message:
        \begin{align*}
            c^d &\equiv 1473513^{1051235} \pmod{2430101} \\
            m &\equiv 1070777 \pmod{2430101}
        \end{align*}
    \end{enumerate}
}

\begin{center}
    \textbf{Primality Testing}
\end{center}

\noindent\textbf{FLT} \\

This is for \hyperlink{thm:Fermat's Little Theorem, Version 2}{Fermat's Little Theorem V2}. This algorithm is pretty straightforward for evaluating if a number is prime or not. Note that it specifically checks for the compositeness of a number, it does not \textit{guarantee} that number is prime. We have the following expression to test if a number is prime:
