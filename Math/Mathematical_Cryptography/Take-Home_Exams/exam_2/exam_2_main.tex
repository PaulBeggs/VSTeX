\documentclass[11pt]{article}
% \usepackage{draculatheme}
\usepackage{fancyhdr}  % For headers and footers
\usepackage[margin=1in]{geometry}  % Set margins
\usepackage{tcolorbox}  % For creating the double-lined box
\usepackage{graphicx}   % For extending lines within margins
\usepackage{amsmath}
\usepackage{amssymb}
\usepackage{amsthm}
\usepackage{enumitem}
\usepackage{lastpage}
\usepackage{array}
\usepackage{booktabs}
\usepackage{pgfplots}
\usepackage{tikz}
\usepackage{hyperref}
\usepackage{xcolor}
\usepackage{mdframed}
\usepackage{listings}
\usepackage{booktabs}

\definecolor{horange}{HTML}{f58026}

\hypersetup{
	colorlinks=true,
	linkcolor=horange,
	filecolor=horange,      
	urlcolor=horange,
}

\setlist[enumerate]{left=0.2cm}

% Need to add in customization for examples
\newcommand{\sol}[1]{
    \begin{customframedproof}
        \begin{solution}
        #1
        \end{solution}
    \end{customframedproof}
}

% \newmdenv[
%   topline=false,
%   bottomline=false,
%   rightline=false,
%   leftline=true,
%   linewidth=1.5pt,
%   linecolor=black!45, % default color, will be overridden in custom commands
%   innertopmargin=0pt,
%   innerbottommargin=0pt,
%   innerrightmargin=10pt,
%   innerleftmargin=10pt,
%   leftmargin=0pt,
%   rightmargin=0pt,
%   skipabove=\topsep,
%   skipbelow=\topsep,
% ]{customframedproof}

\newmdenv[
  topline=false,
  bottomline=true,
  rightline=false,
  leftline=true,
  linewidth=1.5pt,
  linecolor=black!25, % default color, will be overridden in custom commands
%   backgroundcolor=draculabg, % Needed for Dracula theme
%   fontcolor=draculafg, % Needed for Dracula theme
  innertopmargin=0pt,
  innerbottommargin=5pt,
  innerrightmargin=10pt,
  innerleftmargin=10pt,
  leftmargin=0pt,
  rightmargin=0pt,
  skipabove=\topsep,
  skipbelow=\topsep,
]{customframedproof}

\newenvironment{proofpart}[2][black]{
    \begin{mdframed}[
        topline=false,
        bottomline=false,
        rightline=false,
        leftline=true,
        linewidth=1pt,
        linecolor=#1!40, % Custom color
        innertopmargin=10pt,
        innerbottommargin=10pt,
        innerleftmargin=10pt,
        innerrightmargin=10pt,
        leftmargin=0pt,
        rightmargin=0pt,
        skipabove=\topsep,
        skipbelow=\topsep%
    ]
    \noindent
    \begin{minipage}[t]{0.08\textwidth}%
        \textbf{#2}%
    \end{minipage}%
    \begin{minipage}[t]{0.90\textwidth}%
        \begin{adjustwidth}{0pt}{0pt}%
}{
    \end{adjustwidth}
    \end{minipage}
    \end{mdframed}
}

\newenvironment{solution}
  {\textit{Solution.}}

  % Command for theorems proofs with a predefined color
\newcommand{\pf}[1]{
    \begin{customframedproof}
        \begin{proof}
        #1
        \end{proof}
    \end{customframedproof}
}


% Header formatting for all pages except the title page
\fancypagestyle{standard}{
    \fancyhf{}
    \fancyhead[L]{Exam II -- Take Home}
    \fancyhead[R]{Cryptography}
    \fancyfoot[R]{\rule{\textwidth}{0.5pt} \\ Page \thepage\ of 3} % Right-aligned footer with page numbers
}

\newcommand{\points}[1]{\marginpar{[#1]}} % Define a command for margin points

% Header formatting for the title page
\fancypagestyle{titlepage}{
    \fancyhf{}
    \fancyhead[L]{Hendrix}
    \fancyhead[C]{Oct. 24, 2024}  % Centered date in title page header
    \fancyhead[R]{Mathematical Cryptography}
}

\setlength{\footskip}{0in}

% Natural Numbers 
\newcommand{\N}{\mathbb{N}}

% Whole Numbers
\newcommand{\W}{\mathbb{W}}

% Integers
\newcommand{\Z}{\mathbb{Z}}

% Rational Numbers
\newcommand{\Q}{\mathbb{Q}}

% Real Numbers
\newcommand{\R}{\mathbb{R}}

% Complex Numbers
\newcommand{\C}{\mathbb{C}}

\linespread{1}

\pgfplotsset{compat=1.18}

\begin{document}

% Title page
\thispagestyle{titlepage}

\begin{center}
    {\large Exam 2 -- Take home}
\end{center}

\noindent Printed Name: Paul Beggs \underline{\hspace{341.49907pt}}



\begin{center}
    \fbox{%
        \fbox{%
            \parbox{15.2cm}{%
                \textbf{Instructions:} This is an individual, open-book, open-notes exam. You may not give help, receive help, or discuss this exam with anyone apart from me. \\

                You may use a calculator or the computer to help with computations, just be sure to cite when you do so. \\

                You may consult your textbook, your personal class notes, and your previous homework assignments while working on this exam. You may also consult anything posted on our Teams page. Only these resources are allowed. \\

                You may NOT consult any other sources. No internet sources. No other textbooks. No notes or any other material from other students. No consulting with another student. No consulting with another professor. \\

                As always, show work to get full credit (the correct answer may NOT be enough). \\
                Write clearly! Double check your answers! \\

                \textbf{Honor Pledge: I have neither received nor given aid on this work, nor have I witnessed any such violation of the Honor Code.}\\

                \textbf{Honor Pledge Signature:}\underline{\hspace{299.05994pt}}
                \vspace{0.00001cm}
            }%
        }%
    }%
\end{center}
\vspace{0.9cm}
\begin{table}[h]
    \centering
    \renewcommand{\arraystretch}{1.5} % Increase row height
    \begin{tabular}{ccc}
        \toprule
        Question & Points & Score \\ \midrule
        1        & 12     &       \\ \midrule
        2        & 12     &       \\ \midrule
        Total:   & 24     &       \\ \bottomrule
    \end{tabular}
\end{table}

\vfill

\newpage



% Following pages use the standard header
\pagestyle{standard}
\begin{enumerate}
    \item Pirates hid a treasure somewhere on campus. They broke the location up into two pieces\points{12} and encrypted them. An ElGamal Encryption scheme with the public key \(p = 24691\), \(g = 106\), and \(A = 12375\) was used to encrypt the first part of a hidden message. Here is the resulting ciphertext: \(c_1 = 3799\), \(c_2 = 5973\).
          \begin{enumerate}
              \item Use the Pohlig-Hellman algorithm to find the discrete logarithm problem to find the private key \(a\) such that \(A = g^a \pmod{p}\).

                    %% SOLUTION HERE! %%
                    \sol{
                        To use the Pohlig-Hellman algorithm, we need to find the order of the prime number \(p = 24691\). Since it is prime, we calculate the order to be \(\varphi(24691) = 24691 - 1 = 24690\). We can factor this number to \(30 \cdot 823\). Let \(x = a_0 + 30a_1\):
                        \begin{align}
                            (106^{a_0 + 30a_1})^{823}                   & \equiv 12375^{823} \pmod{24691} \\
                            (106^{823a_0 + 24690a_1})                   & \equiv 24143 \pmod{24691}       \\
                            (106^{823a_0} \cdot 106^{24690a_1})         & \equiv 24143 \pmod{24691}       \\
                            (106^{823})^{a_0} \cdot (106^{24690})^{a_1} & \equiv 24143 \pmod{24691}       \\
                            (1410)^{a_0}                                & \equiv 24143 \pmod{24691}.
                        \end{align}
                        For (1), we got the expression by substituting \(x\) for \(a_0 + 30a_1\) for the exponent in \(g^x \equiv \dots\). From there, (2) --- (4) is simple algebra. Then for (5), because \(106^{a_1}\) is congruent to 1, \((106^{a_1})^{24690} = 1\). Additionally, we take \(106^{823} \pmod{24691}\) to get \(1410^{a_0}\). Then, we do the same thing for the other side of the equation. Now, we can brute force this by setting \(a_0 = \{0,1,2,3,\dots, 823\}\). We find that when \(a_0 = 12\), \(1410^{a_0} \equiv 24143 \pmod{24691}\). Seeing that we have \(a_0\), we need to find \(b_0\):
                        \begin{align*}
                            (106^{b_0 + 823b_1})^{30} & \equiv 12375^{30} \pmod{24691} \\
                            15097^{b_0}               & \equiv 7229 \pmod{24691}.
                        \end{align*}
                        Thus, through brute force, we find that \(b_0 = 171\). We can take our \(a_0\) and \(b_0\) to solve for \(x_1,x_2\):
                        \begin{align*}
                            x_1 & =  a_0 + 30a_1 \\
                            x_1 & = 12 \pmod{30}
                        \end{align*} and \begin{align*}
                            x_2 & = b_0 + 823b_1   \\
                            x_2 & = 171 \pmod{823}.
                        \end{align*}
                        At this point, we can solve the Chinese Remainder Theorem.
                        \begin{enumerate}[label=\arabic*.]
                            \item Let \(m = 30 \cdot 823 = 24690\).
                            \item Compute \(n_1 = \frac{24690}{30} = 823\) and \(n_2 = \frac{24690}{823} = 30\).
                            \item Compute \(y_1 = 823^{-1} \pmod{30} \equiv 7\) and \(y_2 = 30^{-1} \pmod{823} \equiv 631\).
                            \item Compute \(x = (12)(823)(7) + (171)(30)(631) \pmod{24690} \equiv 22392\).
                        \end{enumerate} Therefore, \(a = 22392\).
                    }


                    \newpage
              \item Use the private key found in part (a) to decrypt the ciphertext (be sure to convert from ascii back to text).

                    %% SOLUTION HERE! %%
                    \sol{
                        to decrypt the message, we will compute \((c_1^a)^{-1} \cdot c_2 \pmod{p}\). Substituting values, we get \((3799^{22392})^{-1} \cdot 5973 \pmod{24691} \equiv 7378\). Turning this from ascii into regular text, we get ``\texttt{IN-}''.
                    }


          \end{enumerate}
    \item The pirates used an RSA Encryption scheme with public key \(N = 57247159\) and \(e = 11\) \points{12}to encode the second part of the message. Here is the resulting cipher text: \(c = 2772237\).
          \begin{enumerate}
              \item Use your knowledge of algorithms to factor \(N\). Show all work.

                    %% SOLUTION HERE! %%
                    \sol{
                        If we use an method like factoring by difference of squares, we can set up a brute force algorithm to decrypt the factors of \(N\). To compute this algorithmically, we will calculate values up to \texttt{i < limit} for which we add \(N + i^2\). We square root this value, and take the integer value so we cut off the decimal. Then, we compare the squared value to the original. To do this in Python, we can look to the code below. From the algorithm we get \((p,q) = (421, 135979)\).
                    }
\begin{lstlisting}[language=Python,numbers=left]
    import math

    def factor_using_diff(n, limit):
        i = 1
        while i < limit:
            prod = n + i ** 2
            sqrt_prod = int(math.sqrt(prod))
            
            # Check if prod is a perfect square
            if sqrt_prod ** 2 == prod:
                factor1 = sqrt_prod - i
                factor2 = sqrt_prod + i
                return factor1, factor2  # Return the two factors
            i += 1
        return None  # If no factors are found within the limit
    
    if __name__ == "__main__":
        n = 57247159
        limit = 100000
        print(factor_using_diff(n, limit))        
\end{lstlisting}
              \item Use the factors of \(N\) from part (b) to find the private decryption exponent \(d\).

                    %% SOLUTION HERE! %%
                    \sol{
                    We can compute \(d = e^{-1} \pmod{420 \cdot 135978} = 11^{-1} \pmod{57110760} \equiv 36343211\).
                        }

              \item Use the private key found in parts (a) and (b) to decrypt the ciphertext \(c\) and discover fully where the pirates hid their treasure.

                    %% SOLUTION HERE! %%
                    \sol{
                        To decrypt, we compute \(m' = c^d \pmod{N} = 2772237^{36343211} \pmod{57247159} \equiv 667988\). Turning this from ascii into regular text, we get ``\texttt{BOX}.'' Thus, from the first question, the full text is ``\texttt{IN-BOX}.''
                    }
          \end{enumerate}
\end{enumerate}



\end{document}