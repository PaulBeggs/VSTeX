%-%-%-%-%-%-%-%-%-%-%-%-%-%-%-%-%-%-%-%-%-%-%-%-%-%-%-%-%-%-%-%-%-%-%-%-%-%-%
%%% MAIN DOCUMENT %%%

%-%-%-%-%-%-%-%-%-%-%-%-%-%-%-%-%-%-%-%-%-%-%-%-%-%-%-%-%-%-%-%-%-%-%-%-%-%-%

\documentclass[12pt, oneside]{book}
\usepackage{bold-extra}
\usepackage{draculatheme}

%%% AESTHETICS %%%
%-%-%-%-%-%-%-%-%-%-%-%-%-%-%-%-%-%-%-%-%-%-%-%-%-%-%-%-%-%-%-%-%-%-%-%-%-%-%


%%% Dimensions and Spacing %%%
\usepackage[margin=1in]{geometry}
\usepackage{setspace}
\linespread{1}

\setlength{\headheight}{18.5764pt}

%%% Define new colors %%%
\usepackage{xcolor}
\definecolor{orangehdx}{rgb}{0.96, 0.51, 0.16}

% Normal colors
\definecolor{xred}{HTML}{BD4242}
\definecolor{xblue}{HTML}{4268BD}
\definecolor{xgreen}{HTML}{52B256}
\definecolor{xpurple}{HTML}{7F52B2}
\definecolor{xorange}{HTML}{FD9337}
\definecolor{xdotted}{HTML}{999999}
\definecolor{xgray}{HTML}{777777}
\definecolor{xcyan}{HTML}{80F5DC}
\definecolor{xpink}{HTML}{F690EA}
\definecolor{xgrayblue}{HTML}{49B095}
\definecolor{xgraycyan}{HTML}{5AA1B9}

% Dark colors
\colorlet{xdarkred}{red!85!black}
\colorlet{xdarkblue}{xblue!85!black}
\colorlet{xdarkgreen}{xgreen!85!black}
\colorlet{xdarkpurple}{xpurple!85!black}
\colorlet{xdarkorange}{xorange!85!black}
\definecolor{xdarkcyan}{HTML}{008B8B}
\colorlet{xdarkgray}{xgray!85!black}

% Very dark colors
\colorlet{xverydarkblue}{xblue!50!black}

% Document-specific colors
\colorlet{normaltextcolor}{black}
\colorlet{figtextcolor}{xblue}

% Light colors
\colorlet{xlightblue}{xblue!85!white}

% Enumerated colors
\colorlet{xcol0}{black}
\colorlet{xcol1}{xred}
\colorlet{xcol2}{xblue}
\colorlet{xcol3}{xgreen}
\colorlet{xcol4}{xpurple}
\colorlet{xcol5}{xorange}
\colorlet{xcol6}{xcyan}
\colorlet{xcol7}{xpink!75!black}

% Blue-Purple (should just used colorbrewer...)
\definecolor{xrainbow0}{HTML}{e41a1c}
\definecolor{xrainbow1}{HTML}{a24057}
\definecolor{xrainbow2}{HTML}{606692}
\definecolor{xrainbow3}{HTML}{3a85a8}
\definecolor{xrainbow4}{HTML}{42977e}
\definecolor{xrainbow5}{HTML}{4aaa54}
\definecolor{xrainbow6}{HTML}{629363}
\definecolor{xrainbow7}{HTML}{7e6e85}
\definecolor{xrainbow8}{HTML}{9c509b}
\definecolor{xrainbow9}{HTML}{c4625d}
\definecolor{xrainbow10}{HTML}{eb751f}
\definecolor{xrainbow11}{HTML}{ff9709}

%------- %
% XHFILL %
%------- %


%%% Chapter Headings %%%
\usepackage[Glenn]{fncychap}
\ChTitleVar{\bfseries\scshape\color{draculafg}} % Needed for Dracula theme
\ChNumVar{\large\selectfont\color{draculafg}} % Needed for Dracula theme
\ChNameVar{\large\color{draculafg}} % Needed for Dracula theme
% \ChTitleVar{\bfseries\scshape\color{black}}
% \ChNumVar{\large\selectfont\color{black}}
% \ChNameVar{\large\color{black}}
\usepackage{xpatch}
\xpatchcmd\DOCH
{\mghrulefill}{\color{orangehdx}\mghrulefill}
{}{\PatchFailed}
\xpatchcmd\DOTI
{\mghrulefill}{\color{orangehdx}\mghrulefill}
{}{\PatchFailed}
\xpatchcmd\DOTIS
{\mghrulefill}{\color{orangehdx}\mghrulefill}
{}{\PatchFailed}

%% Change Chapter Heading Placement %%
\usepackage{etoolbox}
\makeatletter
\patchcmd{\@makechapterhead}{\vspace*{50\p@}}{\vspace*{-20\p@}}{}{}
\patchcmd{\@makeschapterhead}{\vspace*{50\p@}}{\vspace*{-20\p@}}{}{}
\patchcmd{\DOTI}{\vskip 80\p@}{\vskip 40\p@}{}{}
\patchcmd{\DOTIS}{\vskip 40\p@}{\vskip 0\p@}{}{}
\makeatother

\renewcommand{\thesection}{\thechapter.\arabic{section}} %% Chapter.Section Numbering


%%% FIGURES %%%
\usepackage{graphicx}  
\graphicspath{ {images/} }  
% \numberwithin{figure}{section}
\usepackage{float}
\usepackage{caption}

%%% Hyperlinks %%%
\usepackage{hyperref}
\definecolor{horange}{HTML}{f58026}
\hypersetup{
	colorlinks=true,
	linkcolor=horange,
	filecolor=horange,      
	urlcolor=horange,
}

%% Headers and Footers %%
\usepackage{fancyhdr} % This should be set AFTER setting up the page geometry
\pagestyle{fancy} % options: empty , plain , fancy
\renewcommand{\chaptermark}[1]{\markboth{\thechapter.\ #1}{}}
\fancyhead[R]{\leftmark}
\fancyhead[L]{Hendrix College}
\fancyhead[C]{\includegraphics[height=.50cm]{images/small logo.png}}
\usepackage{xpatch}
\xpretocmd\headrule{\color{orangehdx}}{}{\PatchFailed}
\setlength{\footskip}{0.5in}

%%%%% Colored Boxes %%%%%
\usepackage{tcolorbox}
\tcbuselibrary{skins}
\tcbuselibrary{theorems}
\tcbuselibrary{minted}
% \numberwithin{BoxCounter}{section}


\newtcolorbox{note}{colback=black!15!white, colframe=black, boxrule=0.5mm, arc=4mm, left=4mm, right=4mm, top=2mm, bottom=2mm}

%% Theorems %%
\newtcolorbox[]{theorem}[2][]{%
    enhanced,
    sharp corners=uphill,
    rounded corners,
    colframe=xblue, %color of frame
    colback=xblue!15, %color of background
    coltitle=white, %color of title text
    label={thm:#2},
    title={\large \textbf{Theorem: #2}},
    attach boxed title to top center,
    boxed title style={colback=xblue},
    #1
}

% %% This command needs to exist for the square root of 2 theorem to work

% \newtcolorbox[]{sqrtTh}[2][]{%
%     rounded corners,
%     colframe=xblue, %color of frame
%     colback=xblue!15, %color of background
%     coltitle=white, %color of title text
%     label={thm:#2},
%     title={\large \textbf{Theorem: \texorpdfstring{The Irrationality of \(\sqrt{2}\)}{The Irrationality of sqrt(2)}}},
%     #1
% }



%% Axioms %%
\newtcolorbox[]{axiom}[2][]{%
    enhanced,
    sharp corners=downhill
    colframe=xdarkgreen,
    colback=xdarkgreen!15,
    coltitle=white,
    label={ax:#2},
    title={\large \textbf{Axiom of #2}},
    attach boxed title to top center,
    boxed title style={colback=xdarkgreen, sharp corners=downhill},
    #1
}

%% Exercises %%
\newtcolorbox[]{exercise}[2][]{%
    enhanced,
    sharp corners=downhill,
    colframe=orangehdx!75,
    colback=orangehdx!10,
    coltitle=white,
    label={exerc:#2},
    title={\large \textbf{Exercise #2}},
    attach boxed title to top left={xshift=0.5cm},
    boxed title style={colback=orangehdx!75},
    #1
}

%% Examples %%
\newcounter{ExampleCounter}
\newtcolorbox[use counter=ExampleCounter, number within=chapter]{example}[2][]{%
enhanced,
    sharp corners=downhill,
    colframe=teal,
    colback=teal!15,
    coltitle=white,
    label={ex:\theExampleCounter},
    title={\large \textbf{Example \theExampleCounter: #2}},
    attach boxed title to top center,
    boxed title style={colback=teal},
    #1,
}


%% Corollaries %%
\newtcolorbox[]{corollary}[2][]{%
    enhanced,
    sharp corners=downhill,
    colframe=xpurple,  
    colback=xpurple!15,        
    coltitle=white,
    title={\large \textbf{Corollary #2}},
    attach boxed title to top left={xshift=0.5cm},
    boxed title style={colback=xpurple},
    #1
}

%% Propositions %%
\newenvironment{proposition}[2][]{ % Optional title argument
    \vspace{0.5em}
    \noindent\quad\quad\textbf{Proposition \textcolor{horange}{#2}:} #1
    \par\vspace{0.5em}\noindent
}{\par\vspace{1em}\noindent}

\newcommand{\wrpprop}[2]{%
    \begin{proposition}
        {#1}\label{prop:#1}#2
    \end{proposition}
}

\newcommand{\refprop}[1]{%
    \hyperref[prop:#1]{Proposition #1}%
}

%% Definitions %%
\newenvironment{definition}[2][]{%
    \vspace{0.5em}
    \noindent\quad\quad\textbf{Definition \textcolor{horange}{#2}:} #1
    \par\vspace{0.5em}\noindent
}{\par\vspace{0.3em}\noindent}

\newcommand{\wrpdef}[2]{%
    \begin{definition}
        {#1}\label{#1}#2
    \end{definition}
}

%-%-%-%-%-%-%-%-%-%-%-%-%-%-%-%-%-%-%-%-%-%-%-%-%-%-%-%-%-%-%-%-%-%-%-%-%-%-%
%% MATH PACKAGES, ENVIRONMENTS, COMMANDS %%
%-%-%-%-%-%-%-%-%-%-%-%-%-%-%-%-%-%-%-%-%-%-%-%-%-%-%-%-%-%-%-%-%-%-%-%-%-%-%
%You'll need your own packages, theroem types, and commands.

\usepackage{tikz}
\usepackage{pifont}
\usetikzlibrary{shapes,backgrounds,calc,patterns}
\usepackage{amsmath,amsthm} 
\usepackage{mathtools}
\usepackage{amssymb}
\usepackage{mdframed}
\usepackage{tabularx}
\usepackage{mathrsfs}
\usepackage{multicol}
\usepackage{slashed}
\usepackage{enumerate}
\usepackage{enumitem}
\usepackage{lipsum}  %This package lets us generate random text for example purposes.
\usepackage{booktabs}   
\usepackage{ragged2e}
\usepackage{multirow}




%%% New Environments %%%
\newtheorem{exmp}{Example}[section]

%%% Custom Comands %%%

\newcommand{\cmark}{\ding{51}}%
\newcommand{\xmark}{\ding{55}}%

% Natural Numbers 
\newcommand{\N}{\ensuremath{\mathbb{N}}}

% Whole Numbers
\newcommand{\W}{\ensuremath{\mathbb{W}}}

% Integers
\newcommand{\Z}{\ensuremath{\mathbb{Z}}}

% Finite fields
\newcommand{\F}{\ensuremath{\mathbb{F}}}

% Rational Numbers
\newcommand{\Q}{\ensuremath{\mathbb{Q}}}

% Real Numbers
\newcommand{\R}{\ensuremath{\mathbb{R}}}

% Complex Numbers
\newcommand{\C}{\ensuremath{\mathbb{C}}}

\newcommand{\pfs}{\noindent\makebox[\linewidth]{\rule{\textwidth}{0.4pt}}\vspace{0.5cm}}

% \newcommand{}{\marginnote{\includegraphics[width=2em]{caution.png}}}

\newmdenv[
  topline=false,
  bottomline=true,
  rightline=false,
  leftline=true,
  linewidth=1.5pt,
  linecolor=black, % default color, will be overridden in custom commands
  backgroundcolor=draculabg, % Needed for Dracula theme
  fontcolor=draculafg, % Needed for Dracula theme
  innertopmargin=0pt,
  innerbottommargin=5pt,
  innerrightmargin=10pt,
  innerleftmargin=10pt,
  leftmargin=0pt,
  rightmargin=0pt,
  skipabove=\topsep,
  skipbelow=\topsep,
]{customframedproof}

\newenvironment{proofpart}[2][black]{
    \begin{mdframed}[
        topline=false,
        bottomline=false,
        rightline=false,
        leftline=true,
        linewidth=1pt,
        linecolor=#1!40, % Custom color
        innertopmargin=10pt,
        innerbottommargin=10pt,
        innerleftmargin=10pt,
        innerrightmargin=10pt,
        leftmargin=0pt,
        rightmargin=0pt,
        skipabove=\topsep,
        skipbelow=\topsep%
    ]
    \noindent
    \begin{minipage}[t]{0.08\textwidth}%
        \textbf{#2}%
    \end{minipage}%
    \begin{minipage}[t]{0.90\textwidth}%
        \begin{adjustwidth}{0pt}{0pt}%
}{
    \end{adjustwidth}
    \end{minipage}
    \end{mdframed}
}

\newcommand{\sbseteqpf}[4]{%
    \begin{customframedproof}[linecolor=#4]
        \begin{proof}
                We need to show these expressions are subsets of each other so we can utilize \thm{subset} to prove they are equal: 
            \begin{proofpart}[#4]{(\(\subseteq\))}
                #1
            \end{proofpart}
    \vspace{2mm}
            \begin{proofpart}[#4]{(\(\supseteq\))}
                #2
            \end{proofpart}
    \vspace{2mm}
            Therefore, by \thm{subset}, #3
        \end{proof}
    \end{customframedproof}
}

\newcommand{\iffpf}[4]{
    \begin{customframedproof}[linecolor=#4]
        \begin{proof}
            We show this by proving both implications:
            \begin{proofpart}[#4]{(\(\Rightarrow\))}
                #1
            \end{proofpart}
    \vspace{2mm}
            \begin{proofpart}[#4]{(\(\Leftarrow\))}
                #2
            \end{proofpart}
    \vspace{2mm}
            #3
        \end{proof}
    \end{customframedproof}
}

% Command for lemma proofs with a predefined color
\newcommand{\expf}[1]{
    \begin{customframedproof}[linecolor=orangehdx!75]
        \begin{proof}
        #1
        \end{proof}
    \end{customframedproof}
}


% Command for theorems proofs with a predefined color
\newcommand{\tpf}[1]{
    \begin{customframedproof}[linecolor=xblue]
        \begin{proof}
        #1
        \end{proof}
    \end{customframedproof}
}
% Command for corollaries proofs with a predefined color
\newcommand{\cpf}[1]{
    \begin{customframedproof}[linecolor=xpurple]
        \begin{proof}
        #1
        \end{proof}
    \end{customframedproof}
}

% Need to add in customization for examples
\newcommand{\sol}[1]{
    \begin{customframedproof}[linecolor=orangehdx!75]
        \begin{solution}
        #1
        \end{solution}
    \end{customframedproof}
}

\newcommand{\lesol}[1]{
    \begin{customframedproof}[linecolor=teal]
        \begin{solution}
        #1
        \end{solution}
    \end{customframedproof}
}

\newcommand{\pb}[2]{
    \begin{exercise}{#1}[label=ex:#1]
    #2
    \end{exercise}
}

\def \proofDistance {10pt}
\newenvironment{solution}
  {\textit{Solution.}}

% Command for theorems proofs with a predefined color
\newcommand{\epf}[1]{
    \begin{customframedproof}[linecolor=teal]
        \begin{proof}
        #1
        \end{proof}
    \end{customframedproof}
}

\renewcommand{\theenumi}{\arabic{enumi}}
\renewcommand{\labelenumi}{\theenumi.}



%-%-%-%-%-%-%-%-%-%-%-%-%-%-%-%-%-%-%-%-%-%-%-%-%-%-%-%-%-%-%-%-%-%-%-%-%-%-%
   %% CODE FORMATING %%%
%-%-%-%-%-%-%-%-%-%-%-%-%-%-%-%-%-%-%-%-%-%-%-%-%-%-%-%-%-%-%-%-%-%-%-%-%-%-%

\usepackage{listings}
\lstdefinestyle{javaStyle}{
    language=Java,
    basicstyle=\ttfamily\small,
    keywordstyle=\bfseries\color{blue},
    commentstyle=\color{green},
    stringstyle=\color{red},
    showstringspaces=false,
    breaklines=true,
    numbers=left,
    numberstyle=\tiny\color{gray},
    stepnumber=1,
    numbersep=5pt,
    captionpos=b,
    frame=single,
    tabsize=4,
    upquote=true
}

% \definecolor{stringcolor}{RGB}{248, 217, 60}
\definecolor{functioncolor}{RGB}{97, 227, 108}
\definecolor{variablecolor}{RGB}{248, 248, 242}
\definecolor{ifelsecolor}{RGB}{255, 121, 198}
\definecolor{passedvarcolor}{RGB}{255, 184, 108}
\definecolor{bordercolor}{RGB}{197, 31, 250}
\definecolor{black}{RGB}{0, 0, 0}
\definecolor{commentcolor}{RGB}{98, 114, 164}
\definecolor{backcolor}{rgb}{0.95,0.95,0.92}
\definecolor{stringcolor}{rgb}{0,0.6,0}

% Define a custom style
\lstdefinestyle{myStyle}{
    backgroundcolor=\color{backcolor},   
    commentstyle=\color{commentcolor},
    basicstyle=\ttfamily\footnotesize,
    breakatwhitespace=false,         
    breaklines=true,       
    keywordstyle=\color{blue},          
    keepspaces=true,
    numberstyle=\color{commentcolor},                 
    numbers=left,
    stringstyle=\color{stringcolor},
    numbersep=5pt,                  
    showspaces=false,                
    showstringspaces=false,
    showtabs=false,            
    tabsize=4,
    frame=tlrb      % Adds black bars on top and bottom
    rulecolor=\color{black},         % Sets the frame color to black
    xleftmargin=5.5ex,               % Extends background to include line numbers
    framexleftmargin=5.5ex,          % Ensures the frame includes the margin area
    framexrightmargin=0pt,           % Right margin
    framextopmargin=1pt,             % Adjusts top margin of frame
    framexbottommargin=1pt           % Adjusts bottom margin of frame
}

\lstset{style=myStyle}

%-%-%-%-%-%-%-%-%-%-%-%-%-%-%-%-%-%-%-%-%-%-%-%-%-%-%-%-%-%-%-%-%-%-%-%-%-%-%




% end of preamble
%-%-%-%-%-%-%-%-%-%-%-%-%-%-%-%-%-%-%-%-%-%-%-%-%-%-%-%-%-%-%-%-%-%-%-%-%-%-%

\begin{document}

%-%-%-%-%-%-%-%-%-%-%-%-%-%-%-%-%-%-%-%-%-%-%-%-%-%-%-%-%-%-%-%-%-%-%-%-%-%-%
%%% COVER PAGE %%%
%-%-%-%-%-%-%-%-%-%-%-%-%-%-%-%-%-%-%-%-%-%-%-%-%-%-%-%-%-%-%-%-%-%-%-%-%-%-%

% Do not use all caps.
\newcommand{\titlestandin}[0]{Mathematical Cryptography}
\newcommand{\cussubtitle}[0]{MATH 490}
% Date format should be like: "January 1, 2001"
\newcommand{\startdate}[0]{August 26, 2024}
\newcommand{\customenddate}[0]{December 2, 2024}
% Be sure to include degree recognition (e.g., B.S., M.S., Ph.D)
\newcommand{\professor}[0]{Prof. Allie Ray, Ph.D.}

%-%-%-%-%-%-%-%-%-%-%-%-%-%-%-%-%-%-%-%-%-%-%-%-%-%-%-%-%



\begin{titlepage}
    \begin{center}

        \vspace*{-2cm}
        \includegraphics[width=0.8\textwidth]{images/Hendrix Logo.png}\\
        \vfill

        % Horizontal line above the title in 'horange' color
        \textcolor{horange}{\rule{\textwidth}{1.0pt}}

        \vspace{2em}

        {\huge \textbf{\titlestandin}}

        \vspace{1em} % Space between the title and the bottom line

        \textcolor{horange}{\rule{\textwidth}{1.0pt}}

        \vspace*{1\baselineskip}

        {\LARGE \textbf{\cussubtitle}}

        \begin{large}
            \vspace*{2\baselineskip}

            \textit{Start}  \\[1ex]
            {\scshape \startdate} \\[0.3\baselineskip] % Year published

            \vspace*{1\baselineskip}

            \emph{Author} \\[1ex]
            %Submitted by \\[\baselineskip]
            {\Large Paul Beggs \\ \par} % Editor list
            {\href{mailto:BeggsPA@Hendrix.edu}{{BeggsPA@Hendrix.edu}}}\\ % Editor affiliation

            \vspace*{1\baselineskip}

            \textit{Instructor} \\[1ex] % Tagline(s) or further description
            \professor

            \vspace*{1\baselineskip}

            \textit{End}\\[1ex]
            {\scshape  \customenddate} \\[0.3\baselineskip] % Year published

            \thispagestyle{empty}

        \end{large}
    \end{center}
\end{titlepage}
\pagebreak

%-%-%-%-%-%-%-%-%-%-%-%-%-%-%-%-%-%-%-%-%-%-%-%-%-%-%-%-%-%-%-%-%-%-%-%-%-%-%

%%% Table of Contents %%%

% No need to edit.

%-%-%-%-%-%-%-%-%-%-%-%-%-%-%-%-%-%-%-%-%-%-%-%-%-%-%-%-%-%-%-%-%-%-%-%-%-%-%

% \thispagestyle{plain}
% % Table of Contents will automatically update based on included chapters
% % You may need to compile twice for the toc to update
% \begin{spacing}{1} %Can change spacing between entries in toc
%     \renewcommand{\contentsname}{\Large\textbf{Table of Contents}} %Can rename here
%     \tableofcontents
%     \addtocontents{toc}{\protect\enlargethispage{\baselineskip}}
% \end{spacing}


%-%-%-%-%-%-%-%-%-%-%-%-%-%-%-%-%-%-%-%-%-%-%-%-%-%-%-%-%-%-%-%-%-%-%-%-%-%-%



%-%-%-%-%-%-%-%-%-%-%-%-%-%-%-%-%-%-%-%-%-%-%-%-%-%-%-%-%-%-%-%-%-%-%-%-%-%-%



%%% CHAPTERS %%%

% The document will automatically update (including numbering and toc)
% You may need to compile twice to update the toc
% Witness magic!

% Chapter files -- containing the actual text of the document
% These are in the chapters folder
% All images (for any chapter) must be placed in the images folder

% Example chapters are included with examples on formatting boxes and figures


%-%-%-%-%-%-%-%-%-%-%-%-%-%-%-%-%-%-%-%-%-%-%-%-%-%-%-%-%-%-%-%-%-%-%-%-%-%-%



% \chapter{The Real Numbers}\label{Real nums}
% \vspace*{-0.25in}
% \renewcommand{\theenumi}{\arabic{enumi}}
\renewcommand{\labelenumi}{\theenumi.}
\section{Introduction}

\begin{note}
    Before starting the course, it is important to understand that this document is for note-taking, and will therefore be a bit more informal than the actual textbook. There may be missing sections in the chapters, and there will certainly be missing chapters when we finish for the semester. The textbook is \textit{An Introduction to Mathematical Cryptography} by Hoffstein, Pipher, and Silverman.
\end{note}

\wrpdef{Caesar Shift Cipher}{An encrypted text (by \textbf{shifting}), and you match it up with the alphabet. To encrypt, you write out a sentence, match it with a random assortment of letters by shifting the letters by a predetermined amount.}

\wrpdef{Code}{Replace words / concepts. Example: Eagle has landed.}

\wrpdef{Cipher}{Replacing characters or letters. Simply, replacing one letter for another.}

\wrpdef{Scytale Cipher}{Used by the Spartans in the \(5^{\text{th}}\) century B.C.. This is also known as a \textbf{Transposition Cipher}.}

\wrpdef{Transposition Cipher}{Changed order, but the stayed the same.}

\wrpdef{Plain Text}{Original message that is readable to humans. Abbreviated as [pt].}

\wrpdef{Cipher Text}{Encrypted message that is unreadable to humans. Abbreviated as [ct].}

\wrpdef{Encrypting}{From plain text to cipher text. The inverse of encrypting is decrypting; which is going from the Plain Text [pt] to the Cipher Text [ct].}

\wrpdef{Key}{A secret number or word used in encoding and decoding using a certain algorithm. Example: Caesar shift: \(A \rightarrow R\) rotated clockwise by 17}

\wrpdef{Key Space}{The set of all keys, notated \(\mathcal{K}\). The cardinality (amount of different keys) is notated with absolute value symbols. (E.g., for the Caesar shift, \(|\mathcal{K}| = 26\) because there are 26 letters in the alphabet. Similarly, Scytale \(|\mathcal{K}| = \text{pt}\).)}

\wrpdef{Brute Force Attack}{[During decryption] Trying all possible keys.}

\subsection{Goals of Cryptography}

\begin{enumerate}
    \item Provide confidentiality -- You can't read the message.
    \item Provide integrity -- You can't change the message.
    \item Provide authenticity -- You can't forge the message.
\end{enumerate}

Generally, these are the people and setting that will be used in examples: Alice and Bob are trying to communicate. Eve is trying to eavesdrop on the conversation. \\

\subsection{Simple Substitution Ciphers (Mono-alphabetic Cipher)}

Each letter can be replaced with any other letter. For example, you may have a key: \(\{a,b,c,\dots,z\} \rightarrow \{q,m,w,\dots,t\}\). Its cardinality is \(|\mathcal{K}| = 26!\). \\

\wrpdef{Cryptanalysis}{Process of decrypting without a key.}

\newpage

\wrpdef{Bigrams}{Two letters that are commonly placed together in language. For example, ``Th'', ``is'', or ``He''.}

\wrpdef{Frequency Analysis}{English language patterns. Note that \(13\%\) of letters that are used in the alphabet are (in order from greatest to least): E, T, A, O, N. In brief, the longer the text, the more likely these letters will pop up.} 

\renewcommand{\theenumi}{\alph{enumi}}
\renewcommand{\labelenumi}{(\theenumi)}
\subsection{Exercises}

\begin{exercise}
    {1.1}Build a cipher wheel as illustrated in Figure 1.1, but with an inner wheel that rotates, and use it to complete the following tasks.
    \begin{enumerate}
        \item Encrypt the following plaintext using a rotation of 11 clockwise.
              \begin{center}
                  ``A page of history is worth a volume of logic.''
              \end{center}
        \item Decrypt the following message, which was encrypted with a rotation of 7 clockwise.
              \begin{center}
                  \texttt{AOLYLHYLUVZLJYLAZILAALYAOHUAOLZLJYLALZAOHALCLYFIVKFNBLZZLZ}
              \end{center}
    \end{enumerate}
\end{exercise}

\sol{
    \begin{enumerate}
        \item \texttt{L ALRP ZQ STDEZCJ TD HZCES L GZWFXP ZQ WZRTN}
        \item THERE ARE NO SECRETS BETTER THAN THE SECRETES [sic] THAT EVERY BODY GUESSES \\
              In the encrypted text, ``Secrets'' is \texttt{ZLJYLAZ}. Then, they use an incorrect spelling of the word, \texttt{ZLJYLALZ}, of which has an extra `e' in it. That is what the ``[sic]'' is for.
    \end{enumerate}
}

\newpage
\vfill

\begin{exercise}
    {1.2}Decrypt each of the following Caesar encryptions by trying the various possible
    shifts until you obtain readable text.
    \begin{enumerate}
        \item \texttt{LWKLQNWKDWLVKDOOQHYHUVHHDELOOERDUGORYHOBDVDWUHH}
        \item \texttt{UXENRBWXCUXENFQRLQJUCNABFQNWRCJUCNAJCRXWORWMB}
    \end{enumerate}
\end{exercise}

\sol{
    \begin{enumerate}
        \item I THINK THAT I SHALL NEVER SEE A BILLBOARD LOVELY AS A TREE
        \item LOVE IS NOT LOVE WHICH ALTERS WHEN IT ALTERATION FINDS
    \end{enumerate}
}
\vfill


\begin{exercise}
    {1.3}For this exercise, use the simple substitution table given in Table 1.11.
    \begin{enumerate}
        \item Encrypt the plaintext message:
              \center{The gold is hidden in the garden}
    \end{enumerate}
\end{exercise}

\sol{
    \begin{enumerate}
        \item \texttt{IBX FEPA QL BQAAXW QW IBX FSVAXW}
    \end{enumerate}
}
\vfill
\newpage

\begin{exercise}
    {1.4} Each of the following messages has been encrypted using a simple substitution cipher. Decrypt them. For your convenience, we have given you a frequency table and a list of the most common bigrams that appear in the ciphertext.\ (If you do not want to recopy the ciphertexts by hand, they can be downloaded or printed from the web site listed in the preface.)
    \begin{enumerate}
        \item “A Piratical Treasure”
              \begin{flushleft}
                  \texttt{JNRZR BNIGI BJRGZ IZLQR OTDNJ GRIHT USDKR ZZWLG OIBTM NRGJN} \\
                  \texttt{IJTZJ LZISJ NRSBL QVRSI ORIQT QDEKJ JNRQW GLOFN IJTZX QLFQL} \\
                  \texttt{WBIMJ ITQXT HHTBL KUHQL JZKMM LZRNT OBIMI EURLW BLQZJ GKBJT} \\
                  \texttt{QDIQS LWJNR OLGRI EZJGK ZRBGS MJLDG IMNZT OIHRK MOSOT QHIJL} \\
                  \texttt{QBRJN IJJNT ZFIZL WIZTO MURZM RBTRZ ZKBNN LFRVR GIZFL KUHIM} \\
                  \texttt{MRIGJ LJNRB GKHRT QJRUU RBJLW JNRZI TULGI EZLUK JRUST QZLUK} \\
                  \texttt{EURFT JNLKJ JNRXR S}
              \end{flushleft}
    \end{enumerate}
\end{exercise}

\sol{
    \begin{enumerate}
        \item THESE CHARACTERS AS ONE MIGHT READILY GUESS FORM A CIPHER THAT IS TO SAY THEY CONVEY A MEANING BUT THEN FROM WHAT IS KNOWN OF CAPTAIN KIDD I COULD NOT SUPPOSE HIM CAPABLE OF CONSTRUCTING ANY OF THE MORE ABSTRUSE CRYPTOGRAPHS I MADE UP MY MIND AT ONCE THAT THIS WAS OF A SIMPLE SPECIES SUCH HOW EVER AS WOULD APPEAR TO THE CRUDE INTELLECT OF THE SAILOR ABSOLUTELY INSOLUBLE WITHOUT THE KEY
    \end{enumerate}
    \href{https://planetcalc.com/8047/}{Solver}.

}
\newpage

\begin{exercise}
    {1.5}Suppose that you have an alphabet of 26 letters.
    \begin{enumerate}
        \item How many possible simple substitution ciphers are there?
        \item A letter in the alphabet is said to be fixed if the encryption of the letter is the letter itself. \textit{Show an example of how the pieces work together}
    \end{enumerate}
\end{exercise}

\sol{
    \begin{enumerate}
        \item \(26!\)
        \item This is the formula used for solving for derangements (where \(n\) is the number of elements in the set, and \(!n\) is the number of derangements [Definition: A permutation with no fixed points]): \(!n=n!\sum_{i=0}^{n}{\frac{(-1)^{i}}{i!}}\). From \href{https://en.wikipedia.org/wiki/Derangement}{Wikipedia}. For \(n = 2\). We can run through the following:
              \begin{enumerate}[label=(\arabic*)]
                  \item \(i = 0\): \(\frac{(-1)^0}{0!} = 1\)
                  \item \(i = 1\): \(\frac{(-1)^1}{1!} = -1\)
                  \item \(i = 2\): \(\frac{(-1)^2}{2!} = 0.5\)
              \end{enumerate}
              Sum them together: \(1 - 1 + 0.5 = 0.5\). Now we can get \(!2\): \[!2 = 2! \times (1 - 1 + 0.5) = 2 \times 0.5 = 1\]
    \end{enumerate}
}

\renewcommand{\theenumi}{\arabic{enumi}}
\renewcommand{\labelenumi}{\theenumi.}
\section{Divisibility and Greatest Common Denominators}

Can assume all the properties of \(\R, \Z, \text{ and } \N\). Note that \(\N\) does not include 0. \\

\wrpdef{Divides}{Let \(a\) and \(b\) be integers with \(b \ne 0\). We say that \(b \textit{ divides } a\) or that \(a \) is divisible by \(b\), denoted by \(b \mid a\), if there exists an integer \(n\) such that \(a = nb\).}

\begin{example}
    {Divisibility}Let \(a = 100\) and \(b = 4\). Is \(b \mid a\)?
\end{example}

\lesol{
    Yes, because \(100 = 4 \times 25\).
}

\begin{example}
    {Divisibility}Let \(a = 100\) and \(b = 8\). Is \(b \mid a\)?
\end{example}

\lesol{
    No, because \(100 = 8 \times 12 + 4\).
}

\wrpprop{1.4}{Let \(a,b,c \in \Z\):
    \begin{enumerate}
        \item If \(a \mid b\) and \(b \mid c\), then \(a \mid c\).
        \item If \(a \mid b\) and \(b \mid a\), then \(a = \pm 2\).
        \item If \(a \mid b\) and \(a \mid c\), then \(a \mid (b + c) \text{ and } a \mid (b - c)\).
    \end{enumerate}}


\wrpdef{Greatest Common Divisor}{A \textit{common divisor} of two integers \(a\) and \(b\) is a positive integer \(d\) that divides both of them. The \textit{greatest common divisor} of \(a\) and \(b\), denoted by \(\gcd(a,b)\), is the largest positive integer such that \(d \mid a \) and \(d \mid b\).} \\

This is less complicated than it sounds: we are simply factoring the integers and finding the largest common divisor between the two numbers. \\

\wrpdef{Divison with Remainder}{Let \(a,b\) be positive integers. Then we say that \(a\) \textit{divided by} \(b\) gives a \textit{quotient} \(q\) and a \textit{remainder} \(r\) if \(a = bq + r\) and \(0 \leq r < b\).} \\

\begin{example}
    {Division with Remainder}Let \(a = 24\) and \(b = 16\). Find the quotient and remainder.
\end{example}
\lesol{
    \(24 = 16(1) + 8\). Therefore, the quotient is 1 and the remainder is 8.
}
\newpage
\begin{theorem}
    {Euclidean Algorithm}Let \(a,b \in \Z^+\) with \(a \geq b\). The following algorithm computes \(\gcd(a,b)\) in a finite number of steps. \begin{enumerate}
        \item Let \(r_0 = a, r_1 = b\);
        \item Set \(i = 1\);
        \item Divide \(r_{i - 1}\) by \(r_i\) to get quotient \(q_i\) and remainder \(r_{i + 1}\);
        \item If \(r_{i + 1} = 0\), stop, and \(\gcd(a,b) = r_i\);
        \item Otherwise, \(r_{i + 1} > 0\). Set \(i = i + 1\), and go back to step 3.
        \item Step 3 is executed at most \(2\log_2(b) + 2\) times.
    \end{enumerate}
\end{theorem}


% \begin{theorem}
%     {}Let \(a,b \in \Z^+\) with \(a = bq + r\), \(0 \leq r < b\). Then, \(\gcd(a,b) = \gcd(b,r)\)
% \end{theorem}

% \tpf{
%     Let \(d = \gcd(a,b) \Rightarrow d \mid a \text{ and } d \mid b \Rightarrow a = dk \text{ and } b = dl \text{ for some } k,l \in \Z\). We need to show \(d \mid b\) and \(d \mid r\). We already have \(d \mid b\), so we just need to solve for \(r\): \begin{align*}
%         r & = a - bq    \\
%           & = dk - dlq  \\
%           & = d(k - lq)
%     \end{align*} Notice that \((k - lq) \in \Z\). However, we also need to prove that \(d\) is the largest factor that Divides both. Hence, assume there exists a \(D \in \Z^+\) such that \(D\mid b\), \(D \mid r\), and \(D > d\). Then, \(D \times m = b\) and \(D \times n = r \Rightarrow a = D(mq) + D(n) \Rightarrow D \mid a\), but this leads to a contradiction because we said that \(d\) was the Greatest Common Divisor of \(a\). Hence, \(D = d\).
% }

% Example: \(150 = 4(36) + 6\) (Def 1.1.3) \(\Rightarrow \gcd(150,36) = \gcd(36,6) \Rightarrow 36 = 6(6) + 0 \Rightarrow \gcd(36,6) = \gcd(6,0) = 6\). \\
\vfill
\begin{example}
    {Euclidean Algorithm} Compute \(\gcd(2024, 748)\) using the Euclidean Algorithm.
\end{example}

\lesol{
    Notice how the \(b\) and \(r\) values on each line become the new \(a\) and \(b\) values on the subsequent line:
    \begin{align*}
        2024 & = 2(748) + 528       \\
        748  & = 1(528) + 220       \\
        528  & = 2(220) + 88        \\
        220  & = 2(88) + \boxed{44} \\
        88   & = 2(44) + 0
    \end{align*}
    Therefore, the greatest common divisor is 44.
}
\vfill

\newpage

\begin{example}
    {Linear Combinations}Use Example 1.2 in determining the linear combination of 2024 and 748 that equals 44.
\end{example}

\lesol{
    We let \(a = 2024\) and \(b = 748\). From the equation in Example 1.2, we read the first line:
    \[
        528 = a - 2b.
    \]
    We substitue this into the second line to get
    \[
        b = (a - 2b) \cdot 1 + 220, \quad \text{ so } \quad 220 = 2b - a.
    \]
    We next substitute the expressions \(528 = a - 2b\) and \(220 = 2b - a\) into the third line to get
    \[
        a - 2b = (-a + 3b) \cdot 2 + 88, \quad \text{ so } \quad 88 = 3a - 8b.
    \]
    Finally, we substitute the expressions \(220 = -a + 3b\) and \(88 = 3a - 8b\) into the fourth line to get
    \[
        -a + 3b = (3a - 8b) \cdot 2 + 44, \quad \text{ so } \quad 44 = 7a - 19b.
    \]
    In other words,
    \[
        -7 \cdot 2024 + 19 \cdot 748 = 44 = \gcd(2024, 748),
    \]
    so we have found a way to write \(\gcd(a, b)\) as a linear combination of \(a\) and \(b\) using integer coefficients. \\
}

\wrpdef{Relatively Prime}{Let \(a\) and \(b\) be \textit{relatively prime} if \(\gcd(a,b) = 1\). More generally, any equation \(Au + Bv = \gcd(A,B)\) can be reduced to the case of relatively prime numbers by dividing both sides by \(\gcd(A,B)\). Thus, \(\frac{A}{\gcd(A,B)}u + \frac{B}{\gcd(A,B)} = 1\) where \(a = A / \gcd(A,B)\) and \(b = B / \gcd(A,B)\) are relatively prime and satisfy \(au + bv = 1\).}

\begin{theorem}
    {Extended Euclidean Algorithm}Let \(a,b \in \Z^+\) with \(a \geq b\). Then the equation \(\gcd(a,b) = ua + vb\) always has a solution in integers \(u\) and \(v\). If \(u_0,v_0\) is any one solution, then every solution has the form 
    \[
        u = u_0 + \frac{b \cdot t}{\gcd(a,b)} \quad \text{and} \quad v = v_0 - \frac{a \cdot t}{\gcd(a,b)}, \text{ for some integer } t \in \Z.
    \]
\end{theorem}

\begin{example}
    {Inverse Modulo Application with EEA}Solve for \(d\) using the extended Euclidean algorithm:
    \[
        d \cdot 137 \equiv 1 \pmod{540} \quad \Rightarrow \quad d \equiv 137^{-1} \pmod{540}
    \]
\end{example}

\begin{center}
    \begin{tabular}{ccccccc}
        \(n\) & \(b\) & \(q\) & \(r\) & \(t_1\) & \(t_2\) & \(t_3\)         \\ \toprule
        540   & 137   & 3     & 129   & 0       & 1       & \(-3\)          \\ \midrule
        137   & 129   & 1     & 8     & 1       & \(-3\)  & 4               \\ \midrule
        129   & 8     & 16    & 1     & \(-3\)  & 4       & \(\boxed{-67}\) \\ \midrule
        8     & 1     & 8     & 0     &         &         &                 \\ \midrule
    \end{tabular}
\end{center} A step by tep break down for how this table got its values:
\begin{enumerate}[label=\arabic*.]
    \item \textbf{Initial Setup}

          Start with \(t_1 = 0\) and \(t_2 = 1\). Calculate \(t_3\) with the recursive formula \(t_3 = t_1 - q \cdot t_2\) where \(q\) is the quotient from the division of the 2 numbers (\(n\) divided by \(b\)), and the \(t\) values shift as you move to the next row.

    \item \textbf{Calculate \(t_3\) Values}

          \textit{First row:}
          \begin{itemize}
              \item We start with \(n = 540\) and \(b = 137\).
              \item The quotient \(q = \lfloor \frac{540}{137} \rfloor = 3\).
              \item The values of \(t_1 = 0\) and \(t_2 = 1\) are given by default.
              \item Now, calculate the \(t_3\) using the formula
                    \[
                        t_3 = t_1 - q \cdot t_2 = 0 - 3 \cdot 1 = -3
                    \]
              \item This leaves us with \(t_2 = 1, \ t_3 = -3\). These will become \(t_1\) and \(t_2\) in the next row.
          \end{itemize}
          \textit{Second row:}
          \begin{itemize}
              \item Now, \(n = 137\) and \(b = 129\). We get \(b\) by solving \(b = n - b \cdot q \Rightarrow b = 540 - 137 \cdot 3 = 129\).
              \item The quotient \(q = \lfloor \frac{137}{129} \rfloor = 1\).
              \item The values of \(t_1 = 1\) and \(t_2 = -3\) from the previous row.
              \item Now, calculate the \(t_3\) using the formula
                    \[
                        t_3 = t_1 - q \cdot t_2 = 1 - 1 \cdot -3 = 4
                    \]
              \item This leaves us with \(t_2 = -3, \ t_3 = 4\). These will become \(t_1\) and \(t_2\) in the next row.
          \end{itemize}
          \textit{Third row:}
          \begin{itemize}
              \item Now, \(n = 129\) and \(b = 8\).
              \item The quotient \(q = \lfloor \frac{128}{8} \rfloor = 16\).
              \item The values of \(t_1 = -3\) and \(t_2 = 4\) from the previous row.
              \item Now, calculate the \(t_3\) using the formula
                    \[
                        t_3 = t_1 - q \cdot t_2 = -3 - 16 \cdot 4 = -67
                    \]
          \end{itemize}
          \textit{Fourth row:}
    \item Now, \(n = 8\) and \(b = 1\).
    \item The quotient \(q = \lfloor \frac{8}{1} \rfloor = 8\).
    \item Because get a remainder of 0 from the previous calculation, we stop, and record the largest \(t\) value.
\end{enumerate}
Now we have our \(t = -67\), but we still need to find the modulo. Take the modulo of \(t\) (\(-67 \pmod{540} = 473\)), and you get the inverse, 473.

\textbf{See Exercise~\hyperref[exerc:1.10]{1.10} for a detailed example.}

\begin{example}
    {Relative Prime}What are the relative prime numbers of 2024 and 748?
\end{example}

\lesol{
    In Example 1.2, we found that the gcd of 2024 and 748 have greatest common divisor of 44 and satisfy the equation \(-7 \cdot 2024 + 19 \cdot 748 = 44\). We can divide both sides by 44 to get \(46u + 17v = 1\). Therefore, 46 and 17 are relatively prime and \(u = -7\) and \(v = 19\) are the coefficients of a linear combination of 46 and 17 that equals 1.
}

\renewcommand{\theenumi}{\alph{enumi}}
\renewcommand{\labelenumi}{(\theenumi)}
\subsection{Exercises}
\begin{exercise}
    {1.6}Let \(a, b, c \in Z\). Use the definition of divisibility to directly prove the following properties of divisibility.\ (This is \refprop{1.4})
    \begin{enumerate}
        \item If \(a \mid b\) and \(b \mid c\). Then \(a \mid c\).
        \item If \(a \mid b\) and \(b \mid a\). Then \(a = \pm b\).
        \item If \(a \mid b\) and \(a \mid c\). Then \(a \mid (b + c)\) and \(a \mid (b - c)\).
    \end{enumerate}
\end{exercise}

\sol{
    \begin{enumerate}
        \item Let \(a,b,c \in \Z\) such that \(a \mid b\) and \(b \mid c\). We know there exists an \(n \in \Z\) such that \(a \times n = b\). Similarly, \(b \mid c\) means there exists an \(k \in \Z\) such that \(b\times k = c\). We can use the commutative property to show that:
              \begin{align*}
                  k(an) & = (b)k \\
                  ank   & = bk   \\
                  bk    & = c    \\
                  a(nk) & = c    \\
                  a     & \mid c
              \end{align*}
        \item From the problem statement, we can intuitively ascertain that because both \(a,b\) divide each other, then it must be the case that they are the same number. Moreover, because the criteria for dividing is not pertinent to whether the quotient is negative or positive, this number can be positive or negative. Now that we have an idea of what we are trying to accomplish, we can begin the proof: Let \(a,b \in \Z\) such that \(a \mid b\) and \(b \mid a\). We know there exists an \(n \in \Z\) such that \(a \times n = b\). Similarly, \(b \mid a\) means there exists an \(k \in \Z\) such that \(b\times k = a\). Then, we can utilize substitution to get the following:
              \begin{align*}
                  bk    & = a \\
                  (an)k & = a
              \end{align*} From here, we know that because \(a\) is present on both sides of the equation, we should divide by \(a\) to simplify. Thus, consider the following two cases.
              \begin{itemize}
                  \item \textbf{Case 1:} \(a \ne 0\) \\
                        Since \(a\) is not zero, we can divide both sides by \(a\) to get \(nk = 1\). Since, \(n,k \in \Z\). We do not need to worry about fractional reciprocals. Instead, we know from the identity property of multiplication, that \(n,k\) must both be \(\pm 1\).\ \begin{itemize}
                            \item \textbf{\(n,k\) are both \(+1\):} Then, \(a = b \times 1 \text{ and } b = a \times 1\). Which simplifies to \(a = b\) in both cases.
                            \item \textbf{\(n,k\) are both \(-1\):} Then, \(a = b \times (-1) \text{ and } b = a \times (-1)\) Which simplifies to \(a = -b\) in both cases.
                        \end{itemize}
                  \item \textbf{Case 2:} \(a = 0\) \\
                        If \(a = 0\). Then \(b = 0 \times n = 0\) and \(0 = b \times k = 0\). Therefore, \(a = b = 0\). And \(a = \pm b\) is still true.
              \end{itemize}
              We have shown that in either case if \(a \mid b\) and \(b \mid a\). Then \(a = \pm b\).
        \item Because \(a\) needs to be divisible by both \(b\) and \(c\). We know that there must exist an \(n,k \in \Z\) such that \(b = an\) and \(c = ak\). Our goal is to get to the form, \(a \times \text{ some integer } = b + c\) and \(a \times \text{ some integer } = b - c\) so we can use the definition of divides to help us out here. Therefore, let us consider both \(b + c\) and \(b - c\) in two separate cases:
              \begin{itemize}
                  \item \textbf{Case 1:} \(b + c\)
                        \begin{align*}
                            b + c & = (an) + (ak) \\
                            b + c & = a(n + k)    \\
                            a     & \mid (b + c)
                        \end{align*}
                  \item \textbf{Case 2:} \(b - c\)
                        \begin{align*}
                            b - c & = (an) - (ak) \\
                            b - c & = a(n - k)    \\
                            a     & \mid (b - c)
                        \end{align*}
              \end{itemize}
              We have shown that if \(a \mid b\) and \(a \mid c\). Then \(a \mid (b + c)\) and \(a \mid (b - c)\).
    \end{enumerate}
}

\begin{exercise}
    {1.7}Use a calculator and the method described in Remark 1.9 to compute the following quotients and remainders.
    \begin{enumerate}
        \item 34787 divided by 353.
        \item 238792 divided by 7843.
    \end{enumerate}
\end{exercise}

\sol{
    \begin{enumerate}
        \item \(a = 34787\) and \(b = 353\). Then \(a / b \approx 98.54674220\). So \(q = 98\) and \(r = a - b \cdot q = 34787 - 353 \cdot 98 = 193\).
        \item \(a = 238792\) and \(b = 7843\). Then \(a / b \approx 30.446512\). So \(q = 30\) and \(r = a - b \cdot q = 238792 - 7843 \cdot 30 = 3502\).
    \end{enumerate}
}

\begin{exercise}
    {1.9}Use the Euclidean algorithm to compute the following greatest common divisors.
    \begin{enumerate}
        \item \(\gcd(291, 252)\).
        \item \(\gcd(16261, 85652)\).
    \end{enumerate}
\end{exercise}

\sol{
    \begin{enumerate}
        \item \(\gcd(291,252)\)
              \begin{enumerate}[label=(\arabic*)]
                  \item \(r_0 = 291\).\ \(r_1 = 252\).
                  \item \(i = 1\).
                  \item Divide \(r_0\) by \(r_1\) to get a quotient, \(q_1\) and a remainder, \(r_{2}\):
                        \begin{align*}
                            291 / 252            & = 1 = q_1  \\
                            291 - (252 \times 1) & = 39 = r_2
                        \end{align*}
                  \item \(r_2 \ne 0\). So, we continue.
                  \item \(i = 2 + 1 = 3\).
              \end{enumerate}
              \pfs
              \begin{enumerate}[label=(\arabic*)]
                  \setcounter{enumii}{2}
                  \item Divide \(r_1\) by \(r_2\) to get quotient, \(q_2\) and a remainder, \(r_3\):
                        \begin{align*}
                            252 / 39            & = 6 = q_2  \\
                            252 - (39 \times 6) & = 18 = r_3
                        \end{align*}
                  \item \(r_3 \ne 0\). So, we continue.
                  \item \(i = 3 + 1 = 4\).
              \end{enumerate}
              \pfs
              \begin{enumerate}[label=(\arabic*)]
                  \setcounter{enumii}{2}
                  \item Divide \(r_2\) by \(r_3\) to get quotient \(q_3\) and a remainder, \(r_4\):
                        \begin{align*}
                            39 / 18            & = 2 = q_3 \\
                            39 - (18 \times 2) & = 3 = r_4
                        \end{align*}
                  \item \(r_4 \ne 0\). So, we continue.
                  \item \(i = 3 + 1 = 5\).
              \end{enumerate}
              \pfs
              \begin{enumerate}[label=(\arabic*)]
                  \setcounter{enumii}{2}
                  \item Divide \(r_3\) by \(r_4\) to get quotient \(q_4\) and a remainder, \(r_5\):
                        \begin{align*}
                            18 / 3            & = 6 = q_4 \\
                            18 - (3 \times 6) & = 0 = r_5
                        \end{align*}
                  \item \(r_4 = 0\). So, we stop.
              \end{enumerate}
              We have found that the greatest common divisor is \(3\).
        \item \textit{To cut back on paper, I am going to avoid reiterating the steps of 4 and 5. If there is a continuation in the enumeration process, then \(r_i \ne 0\). And the process needs to continue}: \(\gcd(16261,85652) \Rightarrow \gcd(85652,16261)\).
              \begin{enumerate}[label=(\arabic*)]
                  \item
                        \begin{align*}
                            85652 / 16261            & = 5 = q_1    \\
                            85652 - (16261 \times 5) & = 4347 = r_2
                        \end{align*}
                  \item
                        \begin{align*}
                            16261 / 4347            & = 3 = q_2    \\
                            16261 - (4347 \times 3) & = 3220 = r_3
                        \end{align*}
                  \item
                        \begin{align*}
                            4347 / 3220            & = 1 = q_3    \\
                            4347 - (3220 \times 1) & = 1127 = r_4
                        \end{align*}
                  \item
                        \begin{align*}
                            3220 / 1127            & = 2 = q_4   \\
                            3220 - (1127 \times 2) & = 966 = r_5
                        \end{align*}
                  \item
                        \begin{align*}
                            1127 / 966            & = 1 = q_5   \\
                            1127 - (966 \times 1) & = 161 = r_6
                        \end{align*}
                  \item
                        \begin{align*}
                            966 / 161            & = 6 = q_6 \\
                            966 - (161 \times 6) & = 0 = r_7
                        \end{align*}
              \end{enumerate}
              We have found that \(\gcd(85652,16261) = 161\).

    \end{enumerate}
}

\begin{exercise}
    {1.10} For each of the \(\gcd(a, b)\) values in Exercise 1.9, use the extended Euclidean
    algorithm (Theorem 1.11) to find integers \(u\) and \(v\) such that \(au + bv = \gcd(a, b)\).
\end{exercise}

\sol{
    \begin{enumerate}
        \item We need to solve for the various \(u_i\) and \(v_i\). We will start at \(i = 2\).\ \\
              \begin{tabular}{c|c|c|c|c|c|c}
                  \(i\)      & \(u_i\) & \textit{Formula}         & \textit{Evaluation}   & \(v_i\) & \textit{Formula}         & \textit{Evaluation}   \\ \hline
                  \textit{2} & \(1\)   & \(u_0 - q_1 \times u_1\) & \(1 - 1 \times 0\)    & \(-1\)  & \(v_0 - q_1 \times v_1\) & \(0 - 1 \times 1\)    \\ \hline
                  \textit{3} & \(-6\)  & \(u_1 - q_2 \times u_2\) & \(0 - 6 \times 1\)    & \(7\)   & \(v_1 - q_2 \times v_2\) & \(1 - 6 \times (-1)\) \\ \hline
                  \textit{4} & \(13\)  & \(u_2 - q_3 \times u_3\) & \(1 - 2 \times (-6)\) & \(-15\) & \(v_2 - q_3 \times v_3\) & \(-1 - 2 \times 7\)   \\
              \end{tabular} \\

              \pfs

              Thus, we can now fill out the table in full:
              \begin{tabular}{c|c|c|c|c|c}
                  \textit{i} & \textit{\(r_i\)} & \textit{\(q_i\)} & \(r_{i + 1}\) & \textit{\(u_i\)} & \textit{\(v_i\)} \\ \hline
                  \textit{0} & \(291\)          & \(-\)            & \(-\)         & \(1\)            & \(0\)            \\ \hline
                  \textit{1} & \(252\)          & \(1\)            & \(39\)        & \(0\)            & \(1\)            \\ \hline
                  \textit{2} & \(39\)           & \(6\)            & \(18\)        & \(1\)            & \(-1\)           \\ \hline
                  \textit{3} & \(18\)           & \(2\)            & \(3\)         & \(-6\)           & \(7\)            \\ \hline
                  \textit{4} & \(3\)            & \(6\)            & \(0\)         & \(13\)           & \(-15\)          \\
              \end{tabular} \\

              Now, we need to solve: \(au + bv = \gcd(a,b) \Rightarrow 291(13) + 252(-15) = 3\).\ \(3\) matches the gcd that we found in Exercise 1.9, so this is the correct solution.

        \item

              \begin{tabular}{c|c|c|c|c|c}
                  \textit{i} & \textit{\(r_i\)} & \textit{\(q_i\)} & \(r_{i + 1}\) & \textit{\(u_i\)} & \textit{\(v_i\)} \\ \hline
                  \textit{0} & \(85652\)        & \(-\)            & \(-\)         & \(1\)            & \(0\)            \\ \hline
                  \textit{1} & \(16261\)        & \(5\)            & \(4347\)      & \(0\)            & \(1\)            \\ \hline
                  \textit{2} & \(4347\)         & \(3\)            & \(3220\)      & \(1\)            & \(-5\)           \\ \hline
                  \textit{3} & \(3220\)         & \(1\)            & \(1127\)      & \(-3\)           & \(16\)           \\ \hline
                  \textit{4} & \(1127\)         & \(2\)            & \(966\)       & \(4\)            & \(-21\)          \\ \hline
                  \textit{5} & \(966\)          & \(1\)            & \(161\)       & \(-11\)          & \(58\)           \\ \hline
                  \textit{6} & \(161\)          & \(6\)            & \(0\)         & \(15\)           & \(-79\)          \\
              \end{tabular} \\

              \(85652(15) + 16261(-79) = 161\).
    \end{enumerate}
}

\begin{exercise}
    {1.11} Let \(a\) and \(b\) be positive integers.
    \begin{enumerate}
        \item Suppose that there are integers \(u\) and \(v\) satisfying \(au + bv = 1\). Prove that \(\gcd(a, b) = 1\).
    \end{enumerate}
\end{exercise}

\expf{
    \begin{enumerate}
        \item Suppose there are integers \(a,b,u,v\) such that \(av + bv = 1\). Assume \(d \in \Z\) such that \(d = \gcd(a,b)\). Since \(d\) divides both \(a\) and \(b\) by definition of common divisor, it must also divide \(av\) and \(bv\) by definition of divisibility. Moreover, because \(au + bv = 1\) and \(d\) is a common divisor of both \(av\) and \(bv\). It must also divide \(1\) by Proposition 1.4 (c). Then, the only positive integer that divides \(1\) is \(1\) itself, so it must be the case that \(d = 1\). Therefore, since \(d = 1\) and \(\gcd(a,b) = d\). It follows that \(\gcd(a,b) = 1\).\ \qedhere
    \end{enumerate}
}

\begin{exercise}
    {1.14}Let \(m \geq 1\) be an integer and suppose that \[a_1 \equiv a_2 \ (\text{mod } m) \text{ and } b_1 \equiv b_2 \ (\text{mod } m).\]Prove that \[a_1 \pm b_1 \equiv a_2 \pm b_2 \ (\text{mod } m) \text{ and } a_1 \cdot b_1 \equiv a_2 \cdot b_2 \ (\text{mod } m).\](This is Proposition 1.13(a).)
\end{exercise}

\expf{
    Let \(m \geq 1\) be an integer and suppose that \(a_1 \equiv a_2 \ (\text{mod } m) \text{ and } b_1 \equiv b_2 \ (\text{mod } m)\). From the definition of modulo, we know the difference of \(a_1 - a_2\) and \(b_1 - b_2\) is divisible by \(m\).
    \begin{itemize}
        \item \textbf{Addition:} We want to show that \(a_1 + b_1 \equiv a_2 + b_2\). So, our goal is to achieve \(m \mid ((a_1 + b_1) - (a_2 + b_2))\). Thus, consider \((a_1 + b_1) - (a_2 + b_2)\). We can distribute the minus sign to get \((a_1 - a_2) + (b_1 - b_2)\). From Proposition 1.4 (c), because we know that \(m \mid (a_1 - a_2)\) and \(m \mid (b_1 - b_2)\). We can write this as \(m \mid ((a_1 - a_2) + (b_1 - b_2))\) which implies \(m \mid ((a_1 + b_1) - (a_2 + b_2))\). This shows \(a_1 + b_1 \equiv a_2 + b_2 \ (\text{mod } m)\).
        \item \textbf{Subtraction:} Similarly to addition, we want to show \(a_1 - b_1 \equiv a_2 - b_2 \ (\text{mod } m)\). Thus, consider \((a_1 - b_1) - (a_2 - b_2)\) which implies \( (a_1 - a_2) - (b_1 - b_2)\). From Proposition 1.4 (c), because we know that \(m \mid (a_1 - a_2)\) and \(m \mid (b_1 - b_2)\). We can write this as \(m \mid ((a_1 - a_2) - (b_1 - b_2))\) which implies \(m \mid ((a_1 - b_1) - (a_2 - b_2))\). So \(a_1 - b_1 \equiv a_2 - b_2 \ (\text{mod } m)\).
    \end{itemize}
    Therefore we have shown \(a_1 \pm b_1 \equiv a_2 \pm b_2 \ (\text{mod } m)\)
    \begin{itemize}
        \item \textbf{Product} We want to show that \(a_1 \cdot a_2 \equiv a_2 \cdot b_2 \ (\text{mod } m)\). Thus, consider \(a_1 \cdot b_1 - a_2 \cdot b_2\):
              \begin{align*}
                  a_1 \cdot b_1 - a_2 \cdot b_2 & = a_1 \cdot b_1 - a_1 \cdot b_2 + a_1 \cdot b_2 - a_2 \cdot b_2 \\
                                                & = a_1 \cdot (b_1 - b_2) + b_2 \cdot (a_1 - a_2)
              \end{align*}
              Because \(m \mid (b_1 - b_2)\) and \(m \mid (a_1 - a_2)\). We know from the definition of division that when we multiply those numbers by an integer like \(a_1\) and \(b_2\).\ \(m\) still divides the expression. Hence, \(m \mid (a_1 \cdot (b_1 - b_2))\) and \(m \mid (b_2 \cdot (a_1 - a_2))\). Therefore, \(m \mid (a_1 \cdot b_1 - a_2 \cdot b_2)\) and \(a_1 \cdot b_1 \equiv a_2 \cdot b_2 \ (\text{mod } m)\).\ \qedhere
    \end{itemize}
}
\newpage

\renewcommand{\theenumi}{\arabic{enumi}}
\renewcommand{\labelenumi}{\theenumi.}
\section{Modular Arithmetic}

\wrpdef{Modular Arithmetic}{Let \(m \geq 1\) be an integer. We say that the integers \(a\) and \(b\) are congruent modulo \(m\) if their difference is divisible by \(m\): \(m \mid (b - a)\) or \(m \mid (a - b)\). Notated as \(a = b\bmod m\)}

By the definition of division, we can write this as \(b - a = mk\) for some \(k \in \Z\) and \(b = mk + a\) for some \(k \in \Z\). For example, \(17\bmod 4 \equiv 1\). Or for another example, \(-17\bmod 4 \equiv -1 \equiv 3\). In addition, you can go in two separate directions. For the first, sequence of operations, we could add the numbers inside the parentheses and then take the mod of the number as demonstrated \((26 + 14)\bmod 5 \equiv 40 \bmod 5 \equiv 0\), or we could take the mod of both numbers inside the parentheses, demonstrated as \(26\bmod + 14\bmod 5 = 1 + 4 = 0\). \\

\wrpprop{1.13}{Let \(m \in \Z^+\)
    \begin{enumerate}
        \item If \(a_1 \equiv a_2 \ (\text{mod } m)\) and \(b_1 \equiv b_2 \ (\text{mod } m)\) then \(a_1 \pm b_1 \equiv a_2 \pm b_2 \ (\text{mod } m) \equiv a_2\ (\text{mod } m) + b_2\ (\text{mod } m)\). Also, \(a_1b_1 = a_2b_2\ (\text{mod } m)=a_2\ (\text{mod } m) b_2\ (\text{mod } m)\)
        \item Let \(a \in \Z\). Then \(ab \equiv 1\ (\text{mod } m)\) for some \(b\equiv \Z \iff \gcd(a,m)=1\)
    \end{enumerate}}
\vspace*{-1.5em}

\wrpdef{Ring}{We write \(\Z/m\Z = \{0,1,2,\dots,m-1\}\) and call \(\Z / m\Z\) the \textit{ring of integers modulo} \(m\). We add and multiplying them as integers and then dividing the result by \(m\) and taking the remainder in order to obtain an element in \(\Z/m\Z\).}

Note that we will be finding the more traditional rings that are brought up in Algebra. Thus, for a ring to be a ring, it must have the following properties: \begin{enumerate}[label=\arabic*.]
    \item \textbf{Additive Closure:} For any \( a, b \in R \), the sum \( a + b \) is also in \( R \).
    \item \textbf{Associativity of Addition:} For any \( a, b, c \in R \), \( (a + b) + c = a + (b + c) \).
    \item \textbf{Commutativity of Addition:} For any \( a, b \in R \), \( a + b = b + a \).
    \item \textbf{Additive Identity:} There exists an element \( 0 \in R \) such that for any \( a \in R \), \( a + 0 = a \).
    \item \textbf{Additive Inverses:} For each \( a \in R \), there exists an element \( -a \in R \) such that \( a + (-a) = 0 \).
    \item \textbf{Multiplicative Closure:} For any \( a, b \in R \), the product \( a \cdot b \) is also in \( R \).
    \item \textbf{Associativity of Multiplication:} For any \( a, b, c \in R \), \( (a \cdot b) \cdot c = a \cdot (b \cdot c) \).
    \item \textbf{Distributive Property:} For any \( a, b, c \in R \), \( a \cdot (b + c) = a \cdot b + a \cdot c \) and \( (a + b) \cdot c = a \cdot c + b \cdot c \).
\end{enumerate}

\begin{example}
    {\((\Z_6, \cdot)\)}Find the ring of integers modulo 6 bound under multiplication.
\end{example}

\lesol{
    Note that from the following table, 6 only has 2 inverses, \(1\) and \(5\). This is because in their respective column, there is a 1 in each row.
    \begin{center}
        \begin{tabular}{|c|c|c|c|c|c|c|}
            \hline
            \(a\)         & 0 & 1 & 2 & 3 & 4 & 5 \\
            \hline
            \(a \cdot 0\) & 0 & 0 & 0 & 0 & 0 & 0 \\
            \hline
            \(a \cdot 1\) & 0 & 1 & 2 & 3 & 4 & 5 \\
            \hline
            \(a \cdot 2\) & 0 & 2 & 4 & 0 & 2 & 4 \\
            \hline
            \(a \cdot 3\) & 0 & 3 & 0 & 3 & 0 & 3 \\
            \hline
            \(a \cdot 4\) & 0 & 4 & 2 & 0 & 4 & 2 \\
            \hline
            \(a \cdot 5\) & 0 & 5 & 4 & 3 & 2 & 1 \\
            \hline
        \end{tabular}
    \end{center}

}

\wrpdef{Unit}{Recall from \refprop{1.13} that \(a\) has an inverse modulo \(m\) if and only if \(\gcd(a, m) = 1\). Numbers that have inverses are called \textit{units}. We denote the set of all units by
    \begin{align*}
        {(\Z/m\Z)}^* & = \{a \in \Z/m\Z \mid \gcd(a,m) = 1\}                      \\
                     & = \{a \in \Z/m\Z \mid \text{a has an inverse modulo } m.\}
    \end{align*}}
\begin{center}
    \textbf{Click \hyperref[sec:Theory groups]{this link} to be teleported to section 2.5 where we discuss Groups. We use groups throughout the remainder of Chapter 1.} \\
\end{center}


\wrpdef{Euler's Phi Function}{The function \(\varphi(m)\) defined by the rule \[\varphi(m) = |\{0 \leq a < m \mid \gcd(a,m) = 1\}| \]}
\vspace*{-1em}

\subsection{Modular Arithmetic and Shift Ciphers}

\begin{example}
    {Shift Cipher} Let's say we have a shift cipher with a key of 3. We want to encrypt the message ``HELLO''. What is the encrypted message?
\end{example}

\lesol{
    By shifting the letters by 3, we get ``KHOOR''. For the decryption, we would shift the letters by -3 to get the original message.
}


\subsection{Fast Powering Algorithm}
\label{fast powering algorithm}

We can write the algorithm as follows: \begin{enumerate}[label=\arabic*.]
    \item Write the exponent in binary.
    \item Compute the powers of the base in binary. For example,
          \begin{align*}
              k_0 & = a \ (\text{mod } m)                                              \\
              k_1 & = k_0^2 \ (\text{mod } m) = a^2 \ (\text{mod } m)                  \\
              k_2 & = k_1^2 \ (\text{mod } m) = (a^2)^{2} \ (\text{mod } m)            \\
                  & \vdots                                                             \\
              k_r & = k_{r-1}^2 \ (\text{mod } m) = (a^{2^{n-1}})^2 \ (\text{mod } m).
          \end{align*}
    \item Multiply the powers of the base that correspond to the 1s in the binary representation of the exponent.
          Compute \(a_b \ (\text{mod } m)\) by using \[a^b = a^{b_0 \cdot b_{12} + \cdots + b_r2^r}\]
\end{enumerate}

\begin{example}
    {Fast Powering Algorithm} Compute \(3^{218} \ (\text{mod } 1000)\).
\end{example}


\lesol{
    The first step is to write 218 in binary: \(218 = 11011010\). Then, we can write the powers of 3 in binary: \begin{align*}
        3^1     & = 3 \ (\text{mod } 1000) = 3       \\
        3^2     & = 3^2 \ (\text{mod } 1000) = 9     \\
        3^4     & = 9^2 \ (\text{mod } 1000) = 81    \\
        3^8     & = 81^2 \ (\text{mod } 1000) = 561  \\
        3^{16}  & = 561^2 \ (\text{mod } 1000) = 721 \\
        3^{32}  & = 721^2 \ (\text{mod } 1000) = 841 \\
        3^{64}  & = 841^2 \ (\text{mod } 1000) = 281 \\
        3^{128} & = 281^2 \ (\text{mod } 1000) = 961
    \end{align*} Now, we can calculate the value of \(3^{218}\) by multiplying the values of \(3^{128}\), \(3^{64}\), \(3^{16}\), \(3^{8}\), and \(3^{2}\) together: \begin{align*}
        3^{218} & = 3^{128} \times 3^{64} \times 3^{16} \times 3^{8} \times 3^{2} \\
                & = 961 \times 281 \times 721 \times 561 \times 9                 \\
                & = 489 \ (\text{mod } 1000)
    \end{align*}
}

\renewcommand{\theenumi}{\alph{enumi}}
\renewcommand{\labelenumi}{(\theenumi)}
\subsection{Exercises}

\begin{exercise}
    {1.17}Find all values of \(x\) between \(0\) and \(m - 1\) that are solutions of the following congruences.\ (\textit{Hint:} If you can't figure out a clever way to find the solution(s), you can just substitute each value \(x = 1\).\ \(x = 2,\dots\).\ \(x = m - 1\) and see which ones work.)
    \begin{enumerate}
        \item \(x + 17 \equiv 23 \ (\text{mod } 37)\).
              \setcounter{enumi}{2}
        \item \(x^2 \equiv 3 \ (\text{mod } 11)\)
              \setcounter{enumi}{6}
        \item \(x \equiv 1 \ (\text{mod } 5)\) and also, \(x \equiv 2 \ (\text{mod } 7)\).\ (Find all solutions modulo 35, that is, find the solutions satisfying \(0 \leq x \leq 34\).)
    \end{enumerate}
\end{exercise}

\sol{
    \begin{enumerate}
        \item \(x = 6\)
              \setcounter{enumi}{2}
        \item We know that \(x\) cannot be any number who's square does not exceed 11 because we cannot square a number to get \(3\) (other than \(\sqrt{3}\). But these are only the integers, so we cannot use that). Hence, \(x \ne 1,2,3\) because we know that the results of these, \(1^2 = 1\).\ \(2^2 = 4\).\ \(3^2 = 9\) are not equivalent to 3. Let's try some more values: \(x = 4\): \(4^2 = 16 \ (\text{mod } 11) = 5 \ \slashed{\equiv} \ 3\); \(x = 5\): \(5^2 = 25 \ (\text{mod } 11) = 3 \equiv 3\); \(x = 6\): \(6^2 = 36 \ (\text{mod } 11) = 3 \equiv 3\); \(x = 7\): \(7^2 = 49 \ (\text{mod } 11) = 5 \ \slashed{\equiv} \ 3\); \(x = 8\): \(8^2 = 64 \ (\text{mod } 11) = 9 \ \slashed{\equiv} \ 3\); \(x = 9\): \(9^2 = 81 \ (\text{mod } 11) = 4 \ \slashed{\equiv} \ 3\); \(x = 10\): \(10^2 = 100 \ (\text{mod } 11) = 1 \ \slashed{\equiv} \ 3\).

              Thus, we have it that \(x = 5,6\).
              \setcounter{enumi}{6}
        \item Because we have \(x \equiv 1 \ (\text{mod } 5)\). Verifying correct \(x\)'s are straightforward. All we need to check for is if a multiple of \(5 + 1\) satisfies \(x \equiv 2 \ (\text{mod } 7)\). Thus, the only solution is \(x = 16\)
    \end{enumerate}
}

\begin{exercise}
    {1.19}Prove that if \(a_1\) and \(a_2\) are units modulo \(m\). Then \(a_1a_2\) is a unit modulo \(m\).
\end{exercise}

\expf{
    Suppose \(a_1\) and \(a_2\) are units modulo \(m\). This means \(a \in \Z/m\Z \colon gcd(a,m) = 1\). In other words, \(a_1b_1 \equiv 1 \ (\text{mod } m)\) and \(a_2b_2 \equiv 1 \ (\text{mod } m)\) for some \(b_1, b_2 \in \Z\). When we multiply the equations together, we get \((a_1b_1)(a_2b_2) \equiv 1 \ (\text{mod } m)\) which can be rewritten as \((a_1a_2)(b_1b_2) \equiv 1 \ (\text{mod } m)\). We can multiply \(b_1\) and \(b_2\) to get an integer \(b_3\). Thus, when we multiply \(a_1a_2\) by \(b_3\) and get \(1\). We have shown that \(b_3\) is a multiplicative inverse, and \(a_1a_2\) is a unit modulo \(m\).
}

\begin{exercise}
    {1.26}Use the square-and-multiply algorithm described in \hyperref[fast powering algorithm]{Section 1.3.2}, or the more efficient version in Exercise 1.25, to compute the following powers.
    \begin{enumerate}
        \item \(17^{183} \ (\text{mod } 256)\)
        \item \(2^{477} \ (\text{mod } 1000)\)
    \end{enumerate}
\end{exercise}




\sol{ \hfill
    \begin{enumerate}
        \item \(a = 17\), \(b = 1\), \(A = 183\), and \(N = 256\). We will now begin the loop because \(A > 0\)
              \begin{itemize}
                  \item \(A = 183\) is odd. Therefore, \(b = b \cdot a \ (\text{mod } 256) = 1 \cdot 17 \ (\text{mod } 256) = 17\); and \(a = a^2 \ (\text{mod } 256) = 17^2 \ (\text{mod 256}) = 33\). \(A = \lfloor 182 / 2 \rfloor = 91\).
                  \item \(A = 91\) is odd. Therefore, \(b = b \cdot a \ (\text{mod } 256) = 17 \cdot 33 \ (\text{mod } 256) = 49\); and \(a = a^2 \ (\text{mod } 256) = 33^2 \ (\text{mod 256}) = 65\). \(A = \lfloor 91 / 2 \rfloor = 45\).
                  \item \(A = 45\) is odd. Therefore, \(b = b \cdot a \ (\text{mod } 256) = 49 \cdot 65 \ (\text{mod } 256) = 113\); and \(a = a^2 \ (\text{mod } 256) = 65^2 \ (\text{mod 256}) = 129\). \(A = \lfloor 45 / 2 \rfloor = 22\).
                  \item \(A = 22\) is even. Therefore, \(a = a^2 \ (\text{mod } 256) = 129^2 \ (\text{mod 256}) = 1\). \(A = \lfloor 22 / 2 \rfloor = 11\).
                  \item \(A = 11\) is odd. Therefore, \(b = b \cdot a \ (\text{mod } 256) = 113 \cdot 1 \ (\text{mod } 256) = 113\); and \(a = a^2 \ (\text{mod } 256) = 1^2 \ (\text{mod 256}) = 1\). \(A = \lfloor 11 / 2 \rfloor = 5\).
                  \item \(A = 5\) is odd. Therefore, \(b = b \cdot a \ (\text{mod } 256) = 113 \cdot 1 \ (\text{mod } 256) = 113\); and \(a = a^2 \ (\text{mod } 256) = 1^2 \ (\text{mod 256}) = 1\). \(A = \lfloor 5 / 2 \rfloor = 2\).
                  \item \(A = 2\) is even. Therefore, \(a = a^2 \ (\text{mod } 256) = 1^2 \ (\text{mod 256}) = 1\). \(A = \lfloor 2 / 2 \rfloor = 1\).
                  \item \(A = 1\) is odd. Therefore, \(b = b \cdot a \ (\text{mod } 256) = 113 \cdot 1 \ (\text{mod } 256) = 113\); and \(a = a^2 \ (\text{mod } 256) = 1^2 \ (\text{mod 256}) = 1\). \(A = \lfloor 1 / 2 \rfloor = 0\).
              \end{itemize}
              Since \(A = 0\), we can report the value of \(b\), which is \(113\). Hence, \(17^183 \ (\text{mod }256) = 113\).
        \item \(a = 2\), \(b = 1\), \(A = 477\), \(N = 1000\). We will now begin the loop because \(A > 0\)
              \begin{itemize}
                  \item \(A = 477\) is odd. Therefore, \(b = b \cdot a \ (\text{mod } 1000) = 1 \cdot 2 \ (\text{mod } 1000) = 2\); and \(a = a^2 \ (\text{mod } 1000) = 2^2 \ (\text{mod } 1000) = 4\). \(A = \lfloor 477 / 2 \rfloor = 238\).
                  \item \(A = 238\) is even. Therefore, \(a = a^2 \ (\text{mod } 1000) = 4^2 \ (\text{mod } 1000) = 16\). \(A = \lfloor 238 / 2 \rfloor = 119\).
                  \item \(A = 119\) is odd. Therefore, \(b = b \cdot a \ (\text{mod } 1000) = 2 \cdot 16 \ (\text{mod } 1000) = 32\); and \(a = a^2 \ (\text{mod } 1000) = 16^2 \ (\text{mod } 1000) = 256\). \(A = \lfloor 119 / 2 \rfloor = 59\).
                  \item \(A = 59\) is odd. Therefore, \(b = b \cdot a \ (\text{mod } 1000) = 32 \cdot 256 \ (\text{mod } 1000) = 192\); and \(a = a^2 \ (\text{mod } 1000) = 256^2 \ (\text{mod } 1000) = 536\). \(A = \lfloor 59 / 2 \rfloor = 29\).
                  \item \(A = 29\) is odd. Therefore, \(b = b \cdot a \ (\text{mod } 1000) = 192 \cdot 536 \ (\text{mod } 1000) = 912\); and \(a = a^2 \ (\text{mod } 1000) = 536^2 \ (\text{mod } 1000) = 296\). \(A = \lfloor 29 / 2 \rfloor = 14\).
                  \item \(A = 14\) is even. Therefore, \(a = a^2 \ (\text{mod } 1000) = 296^2 \ (\text{mod } 1000) = 616\). \(A = \lfloor 14 / 2 \rfloor = 7\).
                  \item \(A = 7\) is odd. Therefore, \(b = b \cdot a \ (\text{mod } 1000) = 912 \cdot 616 \ (\text{mod } 1000) = 792\); and \(a = a^2 \ (\text{mod } 1000) = 616^2 \ (\text{mod } 1000) = 456\). \(A = \lfloor 7 / 2 \rfloor = 3\).
                  \item \(A = 3\) is odd. Therefore, \(b = b \cdot a \ (\text{mod } 1000) = 792 \cdot 456 \ (\text{mod } 1000) = 152\); and \(a = a^2 \ (\text{mod } 1000) = 456^2 \ (\text{mod } 1000) = 936\). \(A = \lfloor 3 / 2 \rfloor = 1\).
                  \item \(A = 1\) is odd. Therefore, \(b = b \cdot a \ (\text{mod } 1000) = 152 \cdot 936 \ (\text{mod } 1000) = 272\); and \(a = a^2 \ (\text{mod } 1000) = 936^2 \ (\text{mod } 1000) = 96\). \(A = \lfloor 1 / 2 \rfloor = 0\).
              \end{itemize}
              Since \(A = 0\), we can report the value of \(b\), which is \(272\). Hence, \(2^477 \ (\text{mod }1000) = 272\).
    \end{enumerate}
}

\renewcommand{\theenumi}{\arabic{enumi}}
\renewcommand{\labelenumi}{\theenumi.}
\section{Prime Numbers, Unique Factorization, and Finite Fields}

\wrpdef{Prime Number}{A prime number is a positive integer greater than 1 whose only divisors are 1 and itself.}

\wrpprop{1.19}{Let \(p\) be a prime number with \(a,b \in \Z\) such that \(p \mid ab\). Then \(p \mid a\) or \(p \mid b\).}

\begin{theorem}
    {Fundamental Theorem of Arithmetic}Let \(a \geq 2\) be an integer. Then \(a\) can be factored as a product of prime numbers \[a = p_1^{e_1} \cdot p_2^{e_2} \cdots p_r^{e_r}.\] Further, other than rearranging the order of the factors, this factorization is unique.
\end{theorem}

\wrpprop{1.21}{Let \(p\) be prime. Then every non-zero element of \(\Z/p\Z\) has a multiplicative inverse. Put another way, \(x \in \Z_p\) has a multiplicative inverse if and only if \(\gcd(x,p) = 1\) because of \refprop{1.13}.}

\newpage

\wrpdef{Field}{If \(p\) is prime, then \(\Z/p\Z\) of integers modulo \(p\) with its addition, subtraction, multiplication, and division rules is a \textit{field}. See the definition of \hyperref[Ring]{Ring} for more information on the properties of a field. (Note that a field is a type of ring, called a communicative ring.) We often notate fields as \(\F_p\) and \(\F_p^*\) as the group of units.}

\renewcommand{\theenumi}{\arabic{enumi}}
\renewcommand{\labelenumi}{\theenumi.}
\section{Powers and Primitive Roots in Finite Fields}
\label{thm:Fermat's Little Theorem}

\begin{theorem}
    {Fermat's Little Theorem} Let \(p\) be a prime number and \(a\) be an integer. Then, \[a^{p - 1} \equiv \begin{cases}
            1 \ (\text{mod } p) & \text{if } p \nmid a \\
            0 \ (\text{mod } p) & \text{if } p \mid a
        \end{cases}\]
\end{theorem}

\begin{example}
    {Order 1}Find the order of 2 modulo 7.
\end{example}

\lesol{\(2^3 = 8 \ (\text{mod } 7) = 1\). Thus, the order of 2 modulo 7 is 3.}

\begin{example}
    {Order 2}Find the order of 5 modulo 7.
\end{example}

\lesol{\(5^6 = 15625 \ (\text{mod } 7) = 1\). Thus, the order of 5 modulo 7 is 6.}

\begin{theorem}
    {Primitive Root Theorem}Let \(p\) be a prime number. Then there exists an integer \(g\) such that there exists an \(x \in \F_p^*\) whose powers give every element of \(\F_p^*\), i.e., \[\F_p^* = \{1,g,g^2,g^3.\dots,g^{p - 2}\}.\] Elements with this property are called the \textit{primitive roots of} \(\F_p\) or \textit{generators of} \(\F_p^*\). They are elements of order \(p - 1\).
\end{theorem}

\newpage

\begin{example}
    {Primitive Root}Find a primitive root modulo 7.
\end{example}

\lesol{We can look to the \hyperref[thm:Primitive Root Theorem]{Primitive Root Theorem} to see how many primitive roots are \(7\) has. Because 7 is prime, we know that \(\varphi = 6\). Thus, we know that 7 will have 6 primitive roots. Because there are 6 primitive roots in total, and \(\F_7^*\) has 7 elements (including 0), we know that 1-6 will be the primitive roots.}

\renewcommand{\theenumi}{\alph{enumi}}
\renewcommand{\labelenumi}{(\theenumi)}
\subsection{Exercises}

\begin{exercise}
    {1.32}For each of the following primes \( p \) and numbers \( a \), compute \( a^{-1} \bmod p \) in two ways: (i) the extended Euclidean algorithm. (ii) Use the fast power algorithm and Fermat's little theorem. (See Example 1.27.)
    \begin{enumerate}
        \item \( p = 47 \) and \( a = 11 \).
        \item \( p = 587 \) and \( a = 345 \).
    \end{enumerate}
\end{exercise}

\sol{ \hfill
    \begin{enumerate}
        \item \begin{enumerate}[label=(\roman*)]
                  \item For the extended Euclidean algorithm, we can start by filling out this table:
                        \begin{center}
                            \begin{tabular}{c|c|c|c|c|c}
                                \textit{i} & \textit{\(r_i\)} & \textit{\(q_i\)} & \(r_{i + 1}\) & \textit{\(u_i\)}        & \textit{\(v_i\)}          \\ \hline
                                \textit{0} & \(47\)           & \(-\)            & \(-\)         & \(1\)                   & \(0\)                     \\ \hline
                                \textit{1} & \(11\)           & \(4\)            & \(3\)         & \(0\)                   & \(1\)                     \\ \hline
                                \textit{2} & \(3\)            & \(3\)            & \(2\)         & \(1\)                   & \(0 - 4 \cdot 1 = -4\)    \\ \hline
                                \textit{3} & \(2\)            & \(1\)            & \(1\)         & \(0 - 3 \cdot 1 = -4\)  & \(1 - 3 \cdot -4 = 13\)   \\ \hline
                                \textit{4} & \(1\)            & \(2\)            & \(0\)         & \(1 - 1 \times -4 = 4\) & \(-4 - 1 \cdot 13 = -17\)
                            \end{tabular} \\
                        \end{center}
                        Thus, to find  \(a^{-1} \ (\text{mod } p)\), we need an \(x\) such that \(11x \equiv 1 \ (\text{mod } 47)\). From our table, we know that number to be \(-17\) because \(1 = 4 \times 47 - 17 \times 11\). Then, as a positive number \(\text{mod } 47\), we get \(x \equiv -17 \equiv 30 \ (\text{mod } 47)\).
                  \item For Fermat's Little Theorem:
                        \begin{align*}
                            a^{p - 1} & \equiv 1 \ (\text{mod } p)      \\
                            a^{p - 2} & \equiv a^{-1} \ (\text{mod } p) \\
                            11^{45}   & \ (\text{mod } 47)
                        \end{align*}
                        By the fast power algorithm, we need to find the binary representation of 45. Thus, \(45 \ (\text{mod } 2) = 1\), \(22 \ (\text{mod } 2) = 0\), \(11 \ (\text{mod } 2) = 1\), \(5 \ (\text{mod } 2) = 1\), \(2 \ (\text{mod } 2) = 0\), \(1 \ (\text{mod } 2) = 1\). From this, we have the binary representation of \(45_{10}\) as \(101101_2\). Hence, we will need to calculate \(2^5 \cdot 2^3 \cdot 2^2 \cdot 2^1 \cdot 2^0 \Rightarrow 11^{2^5} \cdot 11^{2^3} \cdot 11^{2^0} \Rightarrow 11^{32} \cdot 11^8 \cdot 11^4 \cdot 11 \equiv 17 \cdot 14 \cdot 25 \cdot 11\). Now, these are:
                        \begin{align*}
                            11^1    & \equiv 11 \ (\text{mod } 47) \\
                            11^2    & \equiv 27 \ (\text{mod } 47) \\
                            11^4    & \equiv 25 \ (\text{mod } 47) \\
                            11^8    & \equiv 14 \ (\text{mod } 47) \\
                            11^{16} & \equiv 8 \ (\text{mod } 47)  \\
                            11^{32} & \equiv 17 \ (\text{mod } 47)
                        \end{align*}
                        Now, we can calculate \(11^{45} = 11^{32} \cdot 11^8 \cdot 11^4 \cdot 11 \equiv 17 \cdot 14 \cdot 25 \cdot 11 \ (\text{mod } 47)\). Further, multiplying we get \(17 \cdot 14 \ (\text{mod } 47) \equiv 3\). Then, \(3 \cdot 25 \ (\text{mod } 47) \equiv 28\). Finally, \(28 \cdot 11 \ (\text{mod } 47) \equiv 30\). Thus confirming our previous answer in (i).
              \end{enumerate}
              %
              %
              %
              %
              %   
        \item \begin{enumerate}[label=(\roman*)]
                  \item For the extended Euclidean algorithm:
                        \begin{center}
                            \begin{tabular}{c|c|c|c|c|c}
                                \textit{i} & \textit{\(r_i\)} & \textit{\(q_i\)} & \(r_{i + 1}\) & \textit{\(u_i\)}          & \textit{\(v_i\)}           \\ \hline
                                \textit{0} & \(587\)          & \(-\)            & \(-\)         & \(1\)                     & \(0\)                      \\ \hline
                                \textit{1} & \(345\)          & \(1\)            & \(242\)       & \(0\)                     & \(1\)                      \\ \hline
                                \textit{2} & \(242\)          & \(1\)            & \(103\)       & \(1\)                     & \(0 - 1 \cdot 1 = -1\)     \\ \hline
                                \textit{3} & \(103\)          & \(2\)            & \(36\)        & \(0 - 1 \cdot 1 = -1\)    & \(1 - 1 \cdot -1 = 2\)     \\ \hline
                                \textit{4} & \(36\)           & \(2\)            & \(31\)        & \(1 - 2 \cdot -1 = 3\)    & \(-1 - 2 \cdot 2 = -5\)    \\ \hline
                                \textit{5} & \(31\)           & \(1\)            & \(5\)         & \(-1 - 2 \cdot 3 = -7\)   & \(2 - 2 \cdot -5 = 12\)    \\ \hline
                                \textit{6} & \(5\)            & \(6\)            & \(1\)         & \(3 - 1 \cdot -7 = 10\)   & \(-5 - 1 \cdot 12 = -17\)  \\ \hline
                                \textit{7} & \(1\)            & \(5\)            & \(0\)         & \(10 - 6 \cdot 10 = -67\) & \(12 - 6 \cdot -17 = 114\)
                            \end{tabular}
                        \end{center}
                        Thus, the inverse of \(345 \ (\text{mod } 587)\) is 114.

                  \item The binary expression for \(585_{10}\) is \(1001001001_2\) (I just used my calculator this time and kept dividing ans by 2 while keeping track of the odd vs even quotients). This leaves us with \(345^{2^9} \cdot 345^{2^6} \cdot 345^{2^3} \cdot 345^{2^0}\).
                        \begin{align*}
                            345^1     & \equiv 345 \ (\text{mod } 587) \\
                            345^2     & \equiv 451 \ (\text{mod } 587) \\
                            345^4     & \equiv 299 \ (\text{mod } 587) \\
                            345^8     & \equiv 177 \ (\text{mod } 587) \\
                            345^{16}  & \equiv 218 \ (\text{mod } 587) \\
                            345^{64}  & \equiv 529 \ (\text{mod } 587) \\
                            345^{128} & \equiv 429 \ (\text{mod } 587) \\
                            345^{256} & \equiv 310 \ (\text{mod } 587) \\
                            345^{512} & \equiv 419 \ (\text{mod } 587)
                        \end{align*}
                        Now we can multiply and solve: \(345^{2^9} \cdot 345^{2^6} \cdot 345^{2^3} \cdot 345^{2^0} \Rightarrow 419 \cdot 529 \cdot 177 \cdot 345 \ (\text{mod } 587) = 114\). Therefore, the inverse of \(345 \ (\text{mod } 587)\) is \(114\).
              \end{enumerate}
    \end{enumerate}
}

\begin{exercise}
    {1.34}Recall that \( g \) is called a primitive root modulo \( p \) if the powers of \( g \) give all nonzero elements of \( \mathbb{F}_p \).
    \begin{enumerate}[label=(\alph*)]
        \item For which of the following primes is 2 a primitive root modulo \( p \)?
              \[
                  \begin{array}{llllllll}
                      \text{(i)} & p = 7 & \text{(ii)} & p = 13 & \text{(iii)} & p = 19 & \text{(iv)} & p = 23
                  \end{array}
              \]
        \item For which of the following primes is 3 a primitive root modulo \( p \)?
              \[
                  \begin{array}{llllllll}
                      \text{(i)} & p = 5 & \text{(ii)} & p = 7 & \text{(iii)} & p = 11 & \text{(iv)} & p = 17
                  \end{array}
              \]
        \item Find a primitive root for each of the following primes.
              \[
                  \begin{array}{llllllll}
                      \text{(i)} & p = 23 & \text{(ii)} & p = 29 & \text{(iii)} & p = 41 & \text{(iv)} & p = 43
                  \end{array}
              \]
        \item Find all primitive roots modulo 11. Verify that there are exactly \(\phi(10)\) of them, as asserted in Remark 1.32.
    \end{enumerate}
\end{exercise}

\sol{ \hfill
    \begin{enumerate}
        \item \begin{enumerate}[label=(\roman*)]
                  \item \(p = 7\): \(2^1 \equiv 2\), \(2^2 \equiv 4\), \(2^3 \equiv 1 \ (\text{mod } 7)\). Because \(2\) does not cover every nonzero elements of \( \mathbb{F}_p \), \(2\) is not a primitive root.
                  \item \(p = 13\): \(2^1 \equiv 2\), \(2^2 \equiv 4\), \(2^3 \equiv 8\), \(2^4 \equiv 3\), \(2^5 \equiv 6\), \(2^6 \equiv 12\), \(2^7 \equiv 11\), \(2^8 \equiv 9\), \(2^9 \equiv 5\), \(2^{10} \equiv 10\), \(2^{11} \equiv 7\), \(2^{12} \equiv 1\), \(2\) is a primitive root.
                  \item \(p = 19\): \(2^1 \equiv 2\), \(2^2 \equiv 4\), \(2^3 \equiv 8\), \(2^{4} \equiv 16\), \(2^{5} \equiv 13\), \(2^{6} \equiv 7\), \(2^{7} \equiv 14\), \(2^{8} \equiv 9\), \(2^{9} \equiv 18\), \(2^{10} \equiv 17\), \(2^{11} \equiv 15\), \(2^{12} \equiv 11\), \(2^{13} \equiv 3\), \(2^{14} \equiv 6\), \(2^{15} \equiv 12\), \(2^{16} \equiv 5\), \(2^{17} \equiv 10\), \(2^{18} \equiv 1\), 2 is a primitive root.
                  \item \(p = 23\): \(2^1 \equiv 2\), \(2^2 \equiv 4\), \(2^3 \equiv 8\), \(2^{4} \equiv 16\), \(2^{5} \equiv 9\), \(2^{6} \equiv 18\), \(2^{7} \equiv 13\), \(2^{8} \equiv 3\), \(2^{9} \equiv 6\), \(2^{10} \equiv 12\), \(2^{11} \equiv 1\), \(2\) is not a primitive root.
              \end{enumerate}
        \item \begin{enumerate}[label=(\roman*)]
                  \item \( p = 5 \): \( 3^1 \equiv 3, 3^2 \equiv 4, 3^3 \equiv 2, 3^4 \equiv 1 \), 3 is a primitive root.
                  \item \( p = 7 \): \( 3^1 \equiv 3, 3^2 \equiv 2, \ldots, 3^6 \equiv 1 \), 3 is a primitive root.
                  \item \( p = 11 \): \( 3^1 \equiv 3, 3^2 \equiv 9, 3^3 \equiv 5, \ldots, 3^5 \equiv 1 \), 3 is not a primitive root.
                  \item \( p = 17 \): \( 3^1 \equiv 3, 3^2 \equiv 9, \ldots, 3^{16} \equiv 1 \), 3 is a primitive root.
              \end{enumerate}
        \item \begin{enumerate}[label=(\roman*)]
                  \item \(p = 23\): \(5^1 \equiv 5\), \(5^{2} \equiv 2\), \(5^{3} \equiv 10\), \(5^{4} \equiv 4\), \(5^{5} \equiv 20\), \(5^{6} \equiv 8\), \(5^{7} \equiv 17\), \(5^{8} \equiv 16\), \(5^{9} \equiv 11\), \(5^{10} \equiv 9\), \(5^{11} \equiv 22\), \(5^{12} \equiv 18\), \(5^{13} \equiv 21\), \(5^{14} \equiv 13\), \(5^{15} \equiv 19\), \(5^{16} \equiv 3\), \(5^{17} \equiv 15\), \(5^{18} \equiv 6\), \(5^{19} \equiv 7\), \(5^{20} \equiv 12\), \(5^{21} \equiv 14\), \(5^{22} \equiv 1\).\\

                        \(5\) is a primitive root.
                  \item \(2^1 \equiv 2\), \(2^{2} \equiv 4\), \(2^{3} \equiv 8\), \(2^{4} \equiv 16\), \(2^{5} \equiv 3\), \(2^{6} \equiv 6\), \(2^{7} \equiv 12\), \(2^{8} \equiv 24\), \(2^{9} \equiv 19\), \(2^{10} \equiv 9\), \(2^{11} \equiv 18\), \(2^{12} \equiv 7\), \(2^{13} \equiv 14\), \(2^{14} \equiv 28\), \(2^{15} \equiv 27\), \(2^{16} \equiv 25\), \(2^{17} \equiv 21\), \(2^{18} \equiv 13\), \(2^{19} \equiv 26\), \(2^{20} \equiv 23\), \(2^{21} \equiv 17\), \(2^{22} \equiv 5\), \(2^{23} \equiv 10\), \(2^{24} \equiv 20\), \(2^{25} \equiv 11\), \(2^{26} \equiv 22\), \(2^{27} \equiv 15\),
                        \(2^{28} \equiv 1\). \\

                        \(2\) is a primitive root.
                  \item \(6^{1} \equiv 6\), \(6^{2} \equiv 36\), \(6^{3} \equiv 11\), \(6^{4} \equiv 25\), \(6^{5} \equiv 27\), \(6^{6} \equiv 39\), \(6^{7} \equiv 29\), \(6^{8} \equiv 10\), \(6^{9} \equiv 19\), \(6^{10} \equiv 32\), \(6^{11} \equiv 28\), \(6^{12} \equiv 4\), \(6^{13} \equiv 24\), \(6^{14} \equiv 21\), \(6^{15} \equiv 3\), \(6^{16} \equiv 18\), \(6^{17} \equiv 26\), \(6^{18} \equiv 33\), \(6^{19} \equiv 34\), \(6^{20} \equiv 40\), \(6^{21} \equiv 35\), \(6^{22} \equiv 5\), \(6^{23} \equiv 30\), \(6^{24} \equiv 16\), \(6^{25} \equiv 14\), \(6^{26} \equiv 2\), \(6^{27} \equiv 12\), \(6^{28} \equiv 31\), \(6^{29} \equiv 22\), \(6^{30} \equiv 9\), \(6^{31} \equiv 13\), \(6^{32} \equiv 37\), \(6^{33} \equiv 17\), \(6^{34} \equiv 20\), \(6^{35} \equiv 38\), \(6^{36} \equiv 23\), \(6^{37} \equiv 15\), \(6^{38} \equiv 8\), \(6^{39} \equiv 7\), \(6^{40} \equiv 1\).\\

                        \(6\) is a primitive root.
                  \item \(3^{1} \equiv 3\), \(3^{2} \equiv 9\), \(3^{3} \equiv 27\), \(3^{4} \equiv 38\), \(3^{5} \equiv 28\), \(3^{6} \equiv 41\), \(3^{7} \equiv 37\), \(3^{8} \equiv 25\), \(3^{9} \equiv 32\), \(3^{10} \equiv 10\), \(3^{11} \equiv 30\), \(3^{12} \equiv 4\), \(3^{13} \equiv 12\), \(3^{14} \equiv 36\), \(3^{15} \equiv 22\), \(3^{16} \equiv 23\), \(3^{17} \equiv 26\), \(3^{18} \equiv 35\), \(3^{19} \equiv 19\), \(3^{20} \equiv 14\), \(3^{21} \equiv 42\), \(3^{22} \equiv 40\), \(3^{23} \equiv 34\), \(3^{24} \equiv 16\), \(3^{25} \equiv 5\), \(3^{26} \equiv 15\), \(3^{27} \equiv 2\), \(3^{28} \equiv 6\), \(3^{29} \equiv 18\), \(3^{30} \equiv 11\), \(3^{31} \equiv 33\), \(3^{32} \equiv 13\), \(3^{33} \equiv 39\), \(3^{34} \equiv 31\), \(3^{35} \equiv 7\), \(3^{36} \equiv 21\), \(3^{37} \equiv 20\), \(3^{38} \equiv 17\), \(3^{39} \equiv 8\), \(3^{40} \equiv 24\), \(3^{41} \equiv 29\),
                        \(3^{42} \equiv 1\). \\

                        \(3\) is a primitive root.
              \end{enumerate}
        \item The primitive roots modulo 11 are 2, 6, 7, 8. To check: \\
              \(2^1 \equiv 11 = 2\), \(2^2 \equiv 11 = 4\), \(2^3 \equiv 11 = 8\), \(2^4 \equiv 11 = 5\), \(2^5 \equiv 11 = 10\), \(2^6 \equiv 11 = 9\), \(2^7 \equiv 11 = 7\), \(2^8 \equiv 11 = 3\), \(2^9 \equiv 11 = 6\), \(2^{10} \equiv 11 = 1\),

              2 is a primitive root. \\

              \(6^1 \equiv 11 = 6\), \(6^2 \equiv 11 = 3\), \(6^3 \equiv 11 = 7\), \(6^4 \equiv 11 = 9\), \(6^5 \equiv 11 = 10\), \(6^6 \equiv 11 = 5\), \(6^7 \equiv 11 = 8\), \(6^8 \equiv 11 = 4\), \(6^9 \equiv 11 = 2\), \(6^{10} \equiv 11 = 1\),

              6 is a primitive root. \\

              \(7^1 \equiv 11 = 7\), \(7^2 \equiv 11 = 5\), \(7^3 \equiv 11 = 2\), \(7^4 \equiv 11 = 3\), \(7^5 \equiv 11 = 10\), \(7^6 \equiv 11 = 4\), \(7^7 \equiv 11 = 6\), \(7^8 \equiv 11 = 9\), \(7^9 \equiv 11 = 8\), \(7^{10} \equiv 11 = 1\),

              7 is a primitive root. \\

              \(8^1 \equiv 11 = 8\), \(8^2 \equiv 11 = 9\), \(8^3 \equiv 11 = 6\), \(8^4 \equiv 11 = 4\), \(8^5 \equiv 11 = 10\), \(8^6 \equiv 11 = 3\), \(8^7 \equiv 11 = 2\), \(8^8 \equiv 11 = 5\), \(8^9 \equiv 11 = 7\), \(8^{10} \equiv 11 = 1\), 8 is a primitive root.
    \end{enumerate}
}
\renewcommand{\theenumi}{\arabic{enumi}}
\renewcommand{\labelenumi}{\theenumi.}
\section{Cryptography By Hand}

\renewcommand{\theenumi}{\alph{enumi}}
\renewcommand{\labelenumi}{(\theenumi)}
\subsection{Exercises}

\begin{exercise}
    {1.41}A \textit{transposition cipher} is a cipher in which the letters of the plaintext remain the same, but their order is rearranged. Here is a simple example in which the message is encrypted in blocks of 25 letters at a time. Take the given 25 letters and arrange them in a 5-by-5 block by writing the message horizontally on the lines. For example, the first 25 letters of the message
    \[\texttt{Now is the time for all good men to come to the aid...} \]
    is written as
    \begin{center}
        \vspace*{-1cm}
        \[
            \begin{array}{ccccc}
                \texttt{N} & \texttt{0} & \texttt{W} & \texttt{I} & \texttt{S} \\
                \texttt{T} & \texttt{H} & \texttt{E} & \texttt{T} & \texttt{I} \\
                \texttt{M} & \texttt{E} & \texttt{F} & \texttt{O} & \texttt{R} \\
                \texttt{A} & \texttt{L} & \texttt{L} & \texttt{G} & \texttt{O} \\
                \texttt{O} & \texttt{D} & \texttt{M} & \texttt{E} & \texttt{N} \\
            \end{array}
        \]
    \end{center}
    Now the ciphertext is formed by reading the letters down the columns, which gives the ciphertext
    \[\texttt{NTMAO OHELD WEFLM ITOGE SIRON.}\]
    \begin{enumerate}
        \item Use this transposition cipher to encrypt the first 25 letters of the message
              \[\texttt{Four score and seven years ago our fathers...}\]
        \item The following message was encrypted using this transposition cipher. Decrypt it.
              \[\texttt{WNOOA HTUFN EHRHE NESUV ICEME.}\]
        \item There are many variations on this type of cipher. We can form the letters into a rectangle instead of a square, and we can use various patterns to place the letters into the rectangle and to read them back out. Try to decrypt the following  ciphertext, in which the letters were placed horizontally into a rectangle of some size and then read off vertically by columns.
              \[\texttt{WHNCE STRHT TEOOH ALBAT DETET SADHE}\]
              \vspace*{-1cm}
              \[\texttt{LEELL QSFMU EEEAT VNLRI ATUDR HTEEA}\]
              (For convenience, we’ve written the ciphertext in 5 letter blocks, but that doesn’t necessarily mean that the rectangle has a side of length 5.)
    \end{enumerate}
\end{exercise}

\sol{ \hfill
    \begin{enumerate}
        \item The first 25 letters of the sentence given is, ``Four score and seven year ago''. Written in block form:
              \begin{center}
                  \vspace*{-1cm}
                  \[
                      \begin{array}{ccccc}
                          \texttt{F} & \texttt{O} & \texttt{U} & \texttt{R} & \texttt{S} \\
                          \texttt{C} & \texttt{O} & \texttt{R} & \texttt{E} & \texttt{A} \\
                          \texttt{N} & \texttt{D} & \texttt{S} & \texttt{E} & \texttt{V} \\
                          \texttt{E} & \texttt{N} & \texttt{Y} & \texttt{E} & \texttt{A} \\
                          \texttt{R} & \texttt{S} & \texttt{A} & \texttt{G} & \texttt{O} \\
                      \end{array}
                  \]
              \end{center}
              Now the ciphertext is formed by reading the letters down the columns, which gives the ciphertext
              \[\texttt{FCNER OODNS URSYA REEEG SAVAO}\]
        \item To create a transposition cipher, we can read the sentence from the starting letter of each word from the sentence, and move inward. For example,
              \begin{enumerate}[label=(\arabic*)]
                  \item \boxed{W}NOOA \boxed{H}TUFN \boxed{E}HRHE \boxed{N}ESUV \boxed{I}CEME \(\Rightarrow\) ``When I...''
                  \item W\boxed{N}OOA H\boxed{T}UFN E\boxed{H}RHE N\boxed{E}SUV I\boxed{C}EME \(\Rightarrow\) ``...N The C...''
                  \item WN\boxed{O}OA HT\boxed{U}FN EH\boxed{R}HE NE\boxed{S}UV IC\boxed{E}ME \(\Rightarrow\) ``...ourse...''
                  \item WNO\boxed{O}A HTU\boxed{F}N EHR\boxed{H}E NES\boxed{U}V ICE\boxed{M}E \(\Rightarrow\) ``...Of Hum...''
                  \item WNOO\boxed{A} HTUF\boxed{N} EHRH\boxed{E} NESU\boxed{V} ICEM\boxed{E} \(\Rightarrow\) ``...an Eve.''
              \end{enumerate}
              Putting it all together, we get, ``\href{https://www.archives.gov/founding-docs/declaration-transcript}{When in the course of human eve...}'' (Declaration of Independence)
        \item This cipher is going to be a little difficult as we have to go through a couple of cases to see which pattern makes the most sense. If we consider the number of letters, 60, we can start at the lowest divisor, 2, and work our way up through the other divisors (this way, we won't start with extremely long ciphers that will each take up a whole page on their lonesome). Thus,
              \begin{itemize}
                  \item Reading by 2s (to get a \(2\times 30\) matrix)
                        \begin{center}
                            \vspace*{-1cm}
                            \[
                                \begin{array}{cccccccccccccccc}
                                    \texttt{W} & \texttt{N} & \texttt{E} & \texttt{T} & \texttt{H} & \texttt{T} & \texttt{O} & \texttt{H} & \texttt{L} & \texttt{A} & \texttt{D} & \texttt{T} & \texttt{A} & \texttt{H} & \texttt{L} & \texttt{E} \\
                                    \texttt{H} & \texttt{C} & \texttt{S} & \texttt{R} & \texttt{T} & \texttt{E} & \texttt{O} & \texttt{A} & \texttt{B} & \texttt{T} & \texttt{E} & \texttt{S} & \texttt{D} & \texttt{E} & \texttt{E} & \texttt{L}
                                \end{array}
                            \]
                            \vspace*{-1cm}
                        \end{center}
                        Clearly, even though we do not have all the letters, there are no words being formed. Let's move on.
                  \item Reading by 3s (to get a \(3\times 20\) matrix)
                        \begin{center}
                            \vspace*{-1cm}
                            \[
                                \begin{array}{cccccccccccccccccccc}
                                    \texttt{W} & \texttt{C} & \texttt{T} & \texttt{T} & \texttt{O} & \texttt{A} & \texttt{A} & \texttt{E} & \texttt{T} & \texttt{D} & \texttt{L} & \texttt{L} & \texttt{S} & \texttt{U} & \texttt{E} & \texttt{V} & \texttt{R} & \texttt{T} & \texttt{R} & \texttt{E} \\

                                    \texttt{H} & \texttt{E} & \texttt{R} & \texttt{T} & \texttt{O} & \texttt{L} & \texttt{T} & \texttt{T} & \texttt{S} & \texttt{H} & \texttt{E} & \texttt{L} & \texttt{F} & \texttt{E} & \texttt{A} & \texttt{N} & \texttt{I} & \texttt{U} & \texttt{H} & \texttt{E} \\

                                    \texttt{N} & \texttt{S} & \texttt{H} & \texttt{E} & \texttt{H} & \texttt{B} & \texttt{D} & \texttt{E} & \texttt{A} & \texttt{E} & \texttt{E} & \texttt{Q} & \texttt{M} & \texttt{E} & \texttt{T} & \texttt{L} & \texttt{A} & \texttt{D} & \texttt{T} & \texttt{A}
                                \end{array}
                            \]
                            \vspace*{-1cm}
                        \end{center}
                        There are no words here.
                  \item \(4 \times 15\) matrix
              \end{itemize}
              \begin{center}
                  \vspace*{-1cm}
                  \[
                      \begin{array}{ccccccccccccccc}
                          \texttt{W} & \texttt{E} & \texttt{H} & \texttt{O} & \texttt{L} & \texttt{D} & \texttt{T} & \texttt{H} & \texttt{E} & \texttt{S} & \texttt{E} & \texttt{T} & \texttt{R} & \texttt{U} & \texttt{T} \\

                          \texttt{H} & \texttt{S} & \texttt{T} & \texttt{O} & \texttt{B} & \texttt{E} & \texttt{S} & \texttt{E} & \texttt{L} & \texttt{F} & \texttt{E} & \texttt{V} & \texttt{I} & \texttt{D} & \texttt{E} \\

                          \texttt{N} & \texttt{T} & \texttt{T} & \texttt{H} & \texttt{A} & \texttt{T} & \texttt{A} & \texttt{L} & \texttt{L} & \texttt{M} & \texttt{E} & \texttt{N} & \texttt{A} & \texttt{R} & \texttt{E} \\

                          \texttt{C} & \texttt{R} & \texttt{E} & \texttt{A} & \texttt{T} & \texttt{E} & \texttt{D} & \texttt{E} & \texttt{Q} & \texttt{U} & \texttt{A} & \texttt{L} & \texttt{T} & \texttt{H} & \texttt{A}
                      \end{array}
                  \]
                  \vspace*{-1cm}
              \end{center}
              There we finally have our transcription! ``We hold these truths to be self-evident that all men are created equal tha\ldots'' (Declaration of Independence again).
    \end{enumerate}
}

\begin{exercise}
    {Additional Problem} Write down the steps for an algorithm to encrypt a plaintext using a transposition/Scytale cipher using \(n\) columns. \textit{Hint: This should involve some modular arithmetic.}
\end{exercise}

\sol{
    \begin{enumerate}[label=(\arabic*)]
        \item Remove all spaces from the plaintext message.
        \item Let \(i\) be the length of the message after removing spaces.
        \item Find the number of rows \(m\) required for the transposition cipher matrix:
              \[
                  m = \left\lceil \frac{i}{n} \right\rceil
              \]
        \item Create an \(m \times n\) matrix to hold the characters.
        \item For each character in the message, place it into the matrix. Let the index of the current character in the message be \(k\), where \(0 \leq k < i\). Calculate the row and column for this character as follows:
              \[
                  \text{row} = \left\lfloor \frac{k}{n} \right\rfloor
              \]
              \[
                  \text{column} = k \ (\text{mod } n)
              \]
        \item Once all characters are placed into the matrix, read the matrix column by column to form the ciphertext. For each column \(j\), loop through the rows and append each character to the ciphertext in the following order:
              \[
                  \text{character index} = j + n \times \text{row}
              \]
        \item If there are empty cells (i.e., \(n\) does not divide the message length \(i\)), fill these cells with random characters or leave them blank.
        \item Concatenate the characters column by column to form the final ciphertext.
    \end{enumerate}

}

\renewcommand{\theenumi}{\arabic{enumi}}
\renewcommand{\labelenumi}{\theenumi.}
\section{Symmetric and Antisymmetric Ciphers}

\subsection{Symmetric Ciphers}

\wrpdef{Symmetric Cipher}{A cipher that uses the same key for both encryption and decryption.}

\begin{center}
    \subsubsection{Review of Notation}
\end{center}

\begin{itemize}
    \item \(k\) implies \hyperref[Key]{key};
    \item \(\mathcal{K}\) implies \hyperref[Key Space]{key space};
    \item \(m\) implies plaintext \hyperref[Plain Text]{plain text};
    \item \(c\) implies \hyperref[Cipher Text]{cipher text};
    \item \(\mathcal{M}\) implies all possible messages (message space);
    \item \(\mathcal{C}\) implies all possible cipher texts (cipher space);
    \item Encryption is a function that is defined as: \[e: \mathcal{K} \times \mathcal{M} \rightarrow \mathcal{C} \text{ such that } d(k(e(k,m))) = m\]
    \item Decryption is a function that is defined as: \[d: \mathcal{K} \times \mathcal{C} \rightarrow \mathcal{M} \text{ such that } e(k,d(k,c)) = c\]
\end{itemize}

\subsection{Successful Ciphers}
\label{Successful Ciphers}

This brings us to what it means to be a \textit{successful} cipher. Thus, we look to \textit{Kerckhoff's Principle} which states that the security of a cipher should not depend on the secrecy of the algorithm, but rather on the secrecy of the key. Therefore, we have the following properties that are required for each cipher:
\begin{enumerate}[label=\arabic*.]
    \item For all \(k\) and \(m\), it is easy to compute \(e_k(m)\). (Note that \textit{easy} is relative to the computational power of the adversary. For this course, \textit{easy} denotes a decryption time of less than a second.)
    \item For all \(k\) and \(c\), it is easy to compute \(e_k(c)\).
    \item Given one or more ciphertext, \(c_1,\dots c_n\), all encrypted with the same key, it is hard to compute any plaintext, \(m\), such that \(d_k(m)\) without knowing the key.
\end{enumerate}

The next traits are desired, but not required:

\begin{enumerate}[label=\arabic*.]
    \setcounter{enumi}{3}
    \item Given one or more PT and CT pair, \((m_1,c_1),\dots,(m_n,c_n)\) it is decrypt another CT not on this list without knowing the key. An example would be the \textit{enigma cipher} from WW-II\@.
    \item For any PT chosen by the adversary and their CT's, \((c_1,\dots c_n)\), it should be hard to decrypt any CT not in this list.
\end{enumerate}

\begin{center}
    \subsubsection{Types of Attacks}
\end{center}

\begin{itemize}
    \item \textit{Brute Force Attack}: Trying all possible keys.
    \item \textit{Known PT Attack}: The adversary knows the plaintext and the corresponding ciphertext.
    \item \textit{Chosen PT Attack}: The adversary chooses the plaintext and receives the corresponding ciphertext.
\end{itemize}

\begin{center}
    \subsubsection{Types of Ciphers}
\end{center}

\begin{enumerate}[label=\arabic*.]
    \item Multiplication modulo \(m\): \(c = m \times k \ (\text{mod } m)\).
          \begin{itemize}
              \item \(\mathcal{K} = \mathcal{M} = \mathcal{C} = \F_p^*\)
              \item \(k \in \F_p^*\)
              \item \(e_k(m) = k \cdot m \ (\text{mod } p)\)
              \item \(d_k(m) = k^{-1} \cdot c \ (\text{mod } p)\)
              \item Example: \(\F_{307}, k = 258, m = 444, e_{258}(444) = 258.444 \ (\text{mod } 1307) = 843\). To find decrypt this message, Eve needs to iterate through \(1307 - 1\) different keys. (Easy.)
          \end{itemize}
    \item Add \((\text{mod } m)\): \(c = m + k \ (\text{mod } m)\). (Caesar Cipher.)
    \item Affine Cipher: Key \(= (k_1,k_2) \in \Z \times \Z\).
    \item Hill Cipher.
    \item Vernam's One-time Pad.
\end{enumerate}

\subsection{Encoding Schemes}

\begin{definition}
    {Encoding Scheme}An \textit{encoding scheme} is a method of converting plaintext into a form that can be transmitted over a channel. An encoding scheme is assumed to be entirely public knowledge and used by everyone for the same purposes. An encryption scheme is designed to hide information from anyone who dopes not know the key. Thus, an encoding scheme, like an encryption scheme, consists of an encoding functions are public knowledge and should be fast and easy to compute.
\end{definition}

This section will cover ASCII\@: We can take strings of 8 bits and convert them into a single character. From 0 to 255, and use them to represent the letters of the alphabet via \(\texttt{a} = 00000000, \texttt{b} = 00000001, \texttt{c} = 00000010, \dots, \texttt{z} = 00011001\). To distinguish between upper and lower case letters, we can use the first bit to represent the case.

\renewcommand{\theenumi}{\alph{enumi}}
\renewcommand{\labelenumi}{(\theenumi)}
\subsection{Exercises}

\begin{exercise}
    {1.46}
    \begin{enumerate}
        \item Convert the 12-bit binary number \(110101100101\) into a decimal integer between \(0\) and \(2^{12} - 1\).
              \setcounter{enumi}{4}

        \item Convert the decimal numbers \(8734\) and \(5177\) into binary numbers, combine them using XOR, and convert the result back into a decimal number.
    \end{enumerate}
\end{exercise}
\vspace{1em}
For this exercise, I will be using these python functions that I wrote:
\begin{lstlisting}[language=Python,numbers=left]
def compute_binary(num):
    binary_representation = []
    while num != 0:
        result = num % 2
        num = num // 2
        binary_representation.append(result)
    binary_str = ''.join(map(str, binary_representation))
    print(binary_str)

def compute_decimal_int(binary_str):
    int_representation = int(binary_str, 2)
    print(int_representation)
\end{lstlisting}
% Note that this is just what \texttt{bin} does, but I am providing this function to show work for the problem. 

\sol{

    \begin{enumerate}
        \item 3429
              \setcounter{enumi}{4}
        \item Note the added 0 for the second binary number so the two numbers add properly.
              \begin{align*}
                  8734             & = 10001000011110 \\
                  5177             & = 01010000111001 \\
                  8734 \oplus 5177 & = 11011001010111 \\
                  13863            & = 11011001010111
              \end{align*}
    \end{enumerate}
}

% \chapter{Discrete Logarithms and Diffie-Hellman}\label{discrete log}
% \vspace*{-0.25in}
% \renewcommand{\theenumi}{\arabic{enumi}}
\renewcommand{\labelenumi}{\theenumi.}
\section{The Birth of Public Key Cryptography}

\wrpdef{One-way Function}{A \textit{one-way function} that is easy to computer, but whose inverse is difficult.}

\wrpdef{Trap-door Function}{A \textit{trap-door function} is a one-way function with an extra piece of information that makes \(f^{-1}\) easy.}

\renewcommand{\theenumi}{\arabic{enumi}}
\renewcommand{\labelenumi}{\theenumi.}
\section{Discrete Logarithm Problem (DLP)}
\label{sec:DLP}

\wrpdef{Discrete Logarithm Problem}{Let \(g\) be a primitive root for \(\F_p\) and let \(h\) be a nonzero element of \(\F_p\). The \textit{Discrete Logarithm Problem} is the problem of finding an exponent \(x\) such that \[g^x \equiv h \ (\text{mod } p).\] The number \(x\) is called the \textit{discrete logarithm} of \(h\) to the base \(g\) and is denoted by \(\log_g(h)\).}

Remember the rules of logarithms: \begin{align*}
    \log_b(a \cdot c) & = \log_b(a) + \log_b(c) \\
    \log_b(a^c)       & = c \cdot \log_b(a)     \\
    \log_b(a / c)     & = \log_b(a) - \log_b(c)
\end{align*}

What is the value of \(x\) such that \(20^x = 21 \ (\text{mod } 23) \overset{\text{Brute-f}}{\Longrightarrow} \log_{20}21 = \boxed{7} \ (\text{mod } 23)\).~(Where \(7\) is from wolfram.)

\begin{example}
    {DLP}Find \(\log_2(10) \ (\text{mod } 11)\). In other words, find the value of \(x\) such that \(2^x \equiv 10 \ (\text{mod } 11)\).
\end{example}

\lesol{
    \begin{align*}
        2^1        & = 2 \ (\text{mod } 11)          \\
        2^2        & = 4 \ (\text{mod } 11)          \\
        2^3        & = 8 \ (\text{mod } 11)          \\
        2^4        & = 5 \ (\text{mod } 11)          \\
        2^5        & = 10 \ (\text{mod } 11)         \\
        \log_2(10) & = \boxed{5} \ (\text{mod } 11).
    \end{align*}
}

\renewcommand{\theenumi}{\alph{enumi}}
\renewcommand{\labelenumi}{(\theenumi)}
\subsection{Exercises}

\begin{exercise}
    {2.3} Let \(g\) be a \hyperref[Relatively Prime]{primitive root} for \(\mathbb{F}_p\).
    \begin{enumerate}
        \setcounter{enumi}{1} % Start enumeration at (b)
        \item Prove that \(\log_g(h_1 h_2) = \log_g(h_1) + \log_g(h_2)\) for all \(h_1, h_2 \in \mathbb{F}_p^*\).
        \item Prove that \(\log_g(h^n) = n \log_g(h)\) for all \(h \in \mathbb{F}_p^*\) and \(n \in \mathbb{Z}\).
    \end{enumerate}
\end{exercise}

\sol{
    \begin{enumerate}
        \setcounter{enumi}{1}
        \item \textit{Proof.} We know that \(g^x \equiv h \pmod{p}\), which means that \(x = \log_g(h)\). Similarly, if \(g^{x_1} \equiv h_1 \pmod{p}\) and \(g^{x_2} \equiv h_2 \pmod{p}\), then \(x_1 = \log_g(h_1)\) and \(x_2 = \log(h_2)\). Now, we can substitute these values into the first equation to get \(g^{x_1 + x_2} \equiv h_1h_2 \pmod{p} \equiv g^{\log_g(h_1) + \log_g(h_2)} \pmod{p}\). Then, from the properties of exponents, we can rewrite this equation as \(h_1 \cdot h_2 \pmod{p} \equiv g^{\log_g(h_1h_2)} \pmod{p}\). Therefore, \(\log_g(h_1 \cdot h_2) = \log_g(h_1) + \log_g(h_2)\).
        \item \textit{Proof} Following a similar process to \((b)\), we start with \(g^{n\log_g(h)}\). Then, by using the properties of logarithms, we can rewrite this as \(g^{\log_g(h^n)}\). Then, because \(g^{\log_g}\) cancel out, we see that \(g^{\log_g(h^n)} = h^n\). Putting everything together, we have \begin{align*}
                  g^{\log_g(h^n)} & \equiv h^n \pmod{p}         \\
                  \log_g(h^n)     & \equiv n\log_g(h) \pmod{p}.
              \end{align*}
    \end{enumerate}
}

\begin{exercise}
    {2.4} Compute the following \hyperref[sec:DLP]{discrete logarithms}.
    \begin{enumerate}
        \item \(\log_2(13)\) for the prime \(23\), i.e., \(p = 23\), \(g = 2\), and you must solve the congruence \(2^x \equiv 13 \pmod{23}\).
        \item \(\log_{10}(22)\) for the prime \(p = 47\).
        \item \(\log_{627}(608)\) for the prime \(p = 941\). (Hint: Look in the second column of Table 2.1 on page 66.)
    \end{enumerate}
\end{exercise}

\sol{
    \begin{enumerate}
        \item We use Wolfram Alpha to solve for \(x\) in the equation \(2^x \equiv 13 \pmod{23}\): \(x = 7\).
        \item Solving for \(x\) we get \(x = 11\).
        \item \(x = 18\).
    \end{enumerate}
}

\renewcommand{\theenumi}{\arabic{enumi}}
\renewcommand{\labelenumi}{\theenumi.}
\section{Diffie-Hellman Key Exchange}
\label{sec:d-h key}

D-H gives a way for Alice and Bob to get a secret shared key in an unsecure environment (i.e., when Eve is listening). Now, we will follow the steps of D-H below:

\begin{enumerate}[label=\arabic*.]
    \item Alice and Bob choose large prime \(p\) and primitive root \(g\) and make public \(k_{\text{pub}} = (p,g)\).
    \item Alice and Bob each pick their own secret integers, \(a,b\) such that \(k_{\text{priv } A} = a\) and \(k_{\text{priv } B} = b\). Compute \(g^a \ (\text{mod } p) = A\) and \(g^b \ (\text{mod } p) = B\).
    \item Exchange \(A\) and \(B\) over an insecure channel.
          \begin{enumerate}[label=\roman*.]
              \item Note that Eve would have to solve \hyperref[Discrete Logarithm Problem]{DLP} if she obtained \(A\) and \(B\) where \(a = \log_g(A)\) and \(b = \log_g(B)\).
              \item Guidelines \(\approx 2^{1000} g \approx p/2\).
          \end{enumerate}
    \item Alice computes \(B^a \ (\text{mod } p) = A'\) and Bob computes \(A^b \ (\text{mod } p) = B'\).
\end{enumerate}

\begin{example}
    {D-H}Let \(p = 23\) and \(g = 5\). Alice chooses \(a = 6\) and Bob chooses \(b = 15\). Compute the shared secret key.
\end{example}

\lesol{
    \begin{align*}
        A  & = 5^6 \ (\text{mod } 23) = 8     \\
        B  & = 5^{15} \ (\text{mod } 23) = 19 \\
        A' & = 19^6 \ (\text{mod } 23) = 2    \\
        B' & = 8^{15} \ (\text{mod } 23) = 2.
    \end{align*}
}

\wrpdef{Diffie-Hellman Problem}{Let \(p\) be a prime number and \(g\) an integer. The \textit{Diffie-Hellman Problem} is the problem of computing the value of \(g^{ab} \ (\text{mod } p)\) from the known values of \(g^a \ (\text{mod } p)\) and \(g^b \ (\text{mod } p)\).}

\renewcommand{\theenumi}{\alph{enumi}}
\renewcommand{\labelenumi}{(\theenumi)}
\subsection{Exercises}



\renewcommand{\theenumi}{\arabic{enumi}}
\renewcommand{\labelenumi}{\theenumi.}
\section{Elgamal Public Key Cryptosystem}
\label{sec:Elgamal}

\begin{table}[h!]
    \centering
    \begin{tabular}{@{}p{0.9\textwidth}l@{}}
        \toprule
        \multicolumn{2}{c}{\parbox[t]{0.9\textwidth}{\centering \textbf{Public parameter creation}}}                                                                                  \\ \midrule
        \multicolumn{2}{c}{\parbox[t]{0.9\textwidth}{\centering A trusted party chooses and publishes a large prime \(p\) and an element \(g\) modulo \(p\) of large (prime) order.}} \\ \midrule
        \multicolumn{2}{c}{\parbox[t]{0.9\textwidth}{\centering \textbf{Key creation}}}                                                                                               \\ \midrule
        \multicolumn{1}{c}{\parbox[t]{0.45\textwidth}{\centering \textbf{Alice}}} & \multicolumn{1}{c}{\parbox[t]{0.45\textwidth}{\centering \textbf{Bob}}}                           \\ \midrule
        \multicolumn{1}{l}{\parbox[t]{0.45\textwidth}{Choose private key \(1 \leq a \leq p -1.\)                                                                                      \\ Compute \(A = g^a \ (\text{mod } p)\). \\ Publish the public key \(A\).}} &  \\ \midrule
        \multicolumn{2}{c}{\parbox[t]{0.9\textwidth}\centering\textbf{Encryption}}                                                                                                    \\ \midrule
        \multicolumn{1}{l}{\parbox[t]{0.45\textwidth}}                          & \multicolumn{1}{r}{\parbox[t]{0.45\textwidth}{\raggedright Choose plaintext \(m\).                \\ Choose random element \(k\). \\ Use Alice's public key \(A\) \\ \quad to compute \(c_1 = g^k \ (\text{mod } p)\) \\ \quad and \(c_2 = mA^k \ (\text{mod } p). \) \\ Send ciphertext \(c_1,c_2\) to Alice.}}                                            \\ \midrule
        \multicolumn{2}{c}{\parbox[t]{0.9\textwidth}{\centering\textbf{Decryption}}}                                                                                                  \\ \midrule
        \multicolumn{1}{l}{\parbox[t]{0.45\textwidth}{Compute \({(c_1^a)}^{-1} \cdot c_2 \ (\text{mod } p)\).                                                                         \\  This quantity is equal to \(m\).}}                     &                                                                                             \\ \bottomrule
    \end{tabular}
    \caption{Elgamal Key Creation, Encryption, and Decryption}
\end{table}

\begin{example}
    {Elgamal}Let \(p = 29\) and \(g = 2\). Alice chooses \(a = 12\) and Bob chooses \(k = 5\) and wants to send secret message \(m = 26\). Compute the shared secret key.
\end{example}

\lesol{First, we need to calculate Alice's \(A\) and Bob's \(B\). Then, we can calculate the ciphertexts \(c_1\) and \(c_2\):
\begin{align*}
    A   & = g^{a} \ (\text{mod } p)  & = & \ 2^{12} \ (\text{mod } 29) = 7     \\
    B   & = g^k \ (\text{mod } p)    & = & \ 2^5 \ (\text{mod } 29) = 3        \\
    c_1 & = g^k \ (\text{mod } p)    & = & \ 2^5 \ (\text{mod } 29) = 3        \\
    c_2 & = m(A^k) \ (\text{mod } p) & = & \ 26(7^{5} \ (\text{mod } 29)) = 10 \\
\end{align*}
Now, for Alice to decrypt the message, she must compute \({(c_1^a)}^{-1} \cdot c_2 \ (\text{mod } p)\):
\[
    {(c_1^a)}^{-1} \cdot c_2 \ (\text{mod } p) = {(3^{12})}^{-1} \cdot 10 \ (\text{mod } 29)
\]
The order of operations to compute this is as follows:
\begin{enumerate}[label=\arabic*.]
    \item \textbf{Compute} \(3^{12} \ (\text{mod } 29) = 16\);
    \item \textbf{Compute} \(16^{-1} \ (\text{mod } 29) = 20\);
    \item \textbf{Finish by multiplying} \(20 \cdot 10 \ (\text{mod } 29) = 26\).
\end{enumerate}
}

\begin{note}
    Be aware: You should only use this encryption scheme once. If you use it more than once, it is possible for an attacker to decrypt the message. For example, Eve knows \(m_1(c_1, c_2) \rightarrow\) Eve finds \(A\) by keeping record of the first message, then by solving for \(d_2\) such that \(c_1, d_2\) (where \(c_1\) is the \textit{same} as the first message) and \(d_2\) is the second message. Then, Eve can solve for \(m_2\) by computing \({(c_1^d)}^{-1} \cdot d_2 \ (\text{mod } p)\).
\end{note}



\renewcommand{\theenumi}{\alph{enumi}}
\renewcommand{\labelenumi}{(\theenumi)}
\subsection{Exercises}

\begin{exercise}
    {2.8} Alice and Bob agree to use the prime \(p = 1373\) and the base \(g = 2\) for communications using the \hyperref[sec:Elgamal]{Elgamal public key cryptosystem}.
    \begin{enumerate}
        \item Alice chooses \(a = 947\) as her private key. What is the value of her public key \(A\)?
        \item Bob chooses \(b = 716\) as his private key, so his public key is
              \[
                  B \equiv 2^{716} \equiv 469 \pmod{1373}.
              \]
              Alice encrypts the message \(m = 583\) using the random element \(k = 877\). What is the ciphertext \((c_1, c_2)\) that Alice sends to Bob?
        \item Alice decides to choose a new private key \(a = 299\) with associated public key
              \[
                  A \equiv 2^{299} \equiv 34 \pmod{1373}.
              \]
              Bob encrypts a message using Alice's public key and sends her the ciphertext \((c_1, c_2) = (661, 1325)\). Decrypt the message.
        \item Now Bob chooses a new private key and publishes the associated public key \(B = 893\). Alice encrypts a message using this public key and sends the ciphertext \((c_1, c_2) = (693, 793)\) to Bob. Eve intercepts the transmission. Help Eve by solving the discrete logarithm problem \(2^b \equiv 893 \pmod{1373}\) and using the value of \(b\) to decrypt the message.
    \end{enumerate}
\end{exercise}

\sol{
    \begin{enumerate}
        \item \(p = 1373, \ g = 2, \ a = 947 \Rightarrow A \equiv 2^{947} \pmod{1373} \equiv 177\).
        \item \(c_1 \equiv 2^{877} \pmod{1373} \equiv 719, \ c_2 \equiv 583 \cdot 469^{877} \pmod{1373} \equiv 623\). Alice sends ``\((719, 623)\)'' to Bob.
        \item To decrypt, we can use the \hyperref[thm:Extended Euclidean Algorithm]{EEA} to find the inverse of \(661^{299} \pmod{1373} \equiv 645^{-1} \equiv 794\). From here, we solve for the message: \(1325 \cdot 794 \pmod{1373} \equiv 332\).
        \item Solving for \(b\) in \(2^b \equiv 893 \pmod{1373}\) gives \(b = 219\). Now we can decrypt:
              \[
                  (c_1^a)^{-1} \cdot c_2 \equiv (693^{219})^{-1} \equiv 431^{-1} \cdot 793 \equiv 532 \cdot 793 \equiv 365 \pmod{1373}.
              \] Alice's private message to Bob is \(m = 365\).
    \end{enumerate}
}



\renewcommand{\theenumi}{\arabic{enumi}}
\renewcommand{\labelenumi}{\theenumi.}
\section{An Overview of the Theory of Groups}
\label{sec:Theory groups}

\wrpdef{Group}{A set \(G\) along with a binary operation (closure) such that for all \(a,b \in G\), \(a\times b \in G\) (closure), and there exists an \(e \in G\) such that \(a \times e = a\) and \(e \times a = a\) (identity), for all \(a \in G\), there exists \(a^{-1} \in G\) such that \(a \times a^{-1} = a^{-1} \times a = e\) (inverse), and for all \(a,b,c \in G\), \((a \times b) \times c = a \times (b \times c)\) (associativity)}

For commutativity, for all \(a,b \in G\), \(a\times b = b \times a\). Some groups have this, some do not. \\

\begin{example}
    {Integer Addition as a Group}Lets check to see addition among the integers are a group: \((\Z, +)\)
\end{example}

\lesol{
    \begin{enumerate}
        \item True. Let \(a,b \in \Z\) \(a + b \in \Z\).
        \item True. \(e = 0 \in \Z\), \(a + 0 = a\) and \(0 + a = a\)
        \item True. For all \(a\in \Z\), \(a^{-1} = -a\) because \(a + (-a) = 0 = -a + a\)
        \item True. For all \(a,b,c \in \Z\), \((a + b) + c = a + (b + c)\)
    \end{enumerate}
    Therefore, the additive property of the integers are a group. In fact, because \(a + b = b + a\) \(\Z\) are a commutative group (abelian group).
}

\begin{example}
    {Integer Multiplication}Lets check to see multiplication among the integers are a group: \((\Z, +)\)
\end{example}
\lesol{
    \begin{enumerate}
        \item True. Let \(a,b \in \Z\) \(ab \in \Z\).
        \item True. \(e = 1 \in \Z\), \(a * 1 = a\) and \(1 * a = a\)
        \item False. Counterexample: consider \(2^{-1} = \frac{1}{2}\) because \(2(\frac{1}{2}) = 1\) but \(\frac{1}{2} \notin \Z\)
    \end{enumerate}
}

\wrpdef{Order}{The \textit{order} of an element \(a \ (\text{mod } p)\) is the smallest exponent \(k \geq 1\) such that \(a^k \equiv 1 \ (\text{mod } p)\).}

\renewcommand{\theenumi}{\alph{enumi}}
\renewcommand{\labelenumi}{(\theenumi)}
\subsection{Exercises}

\begin{exercise}
    {(Additional)}Decide whether each of the following is a group:
    \begin{enumerate}
        \item \texttt{All 2\(\times\)2 matrices with real number entries} with operation matrix addition
        \item \texttt{All 2 \(\times\) 2 matrices with real number entries} with operation matrix multiplication
    \end{enumerate}

\end{exercise}

\sol{
    \begin{enumerate}
        \item \textbf{Matrix Addition:} \cmark
              \begin{enumerate}[label=(\arabic*)]
                  \item \textbf{Closure:} For addition to work between matrices, they must be of dimension \(2 \times 2 + 2 \times 2\). Therefore, the dimensions do not change, and it is closed.
                  \item \textbf{Associativity:} \(2 \times 2\) matrix addition is associative, as it inherits this property from the properties of matrices.
                  \item \textbf{Identity Element:} We can add a matrix \(Z\) that consists of only 0s to a matrix \(A\). And matrix \(A\) will remain unchanged.
                  \item \textbf{Inverse Element:} True. Consider the matrices,
                        \(\begin{bmatrix}
                            a & b \\
                            c & d
                        \end{bmatrix}\). And
                        \(\begin{bmatrix}
                            -a & -b \\
                            -c & -d
                        \end{bmatrix}\). When we add these two together we get
                        \(\begin{bmatrix}
                            0 & 0 \\
                            0 & 0
                        \end{bmatrix}\). This shows that \(2\times 2\) matrices have additive inverses.



              \end{enumerate}
        \item \textbf{Matrix Multiplication:} \xmark
              \begin{enumerate}[label=(\arabic*)]
                  \item \textbf{Closure:} The dimensions will stay the same during multiplication because it is an \(n \times n\) matrix.
                  \item \textbf{Associativity:} \(2 \times 2\) matrix multiplication is associative, as it inherits this property from the properties of matrices.
                  \item \textbf{Identity Element:} True. Consider the identity matrix, \(I_2 =
                        \begin{bmatrix}
                            1 & 0 \\
                            0 & 1
                        \end{bmatrix}\). When we multiply a matrix \(\begin{bmatrix}
                            a & b \\
                            c & d
                        \end{bmatrix}\) by \(I\). We get \[\begin{bmatrix}
                                a & b \\
                                c & d
                            \end{bmatrix}\] \(\cdot
                        \begin{bmatrix}
                            1 & 0 \\
                            0 & 1
                        \end{bmatrix} = \begin{bmatrix}
                            a & b \\
                            c & d
                        \end{bmatrix}\)
                  \item \textbf{Inverse Element:} False. Matrices with a non-zero determinant fail this criteria. Consider the matrix \(B =\)
                        \(\begin{bmatrix}
                            1 & 2 \\
                            3 & 4
                        \end{bmatrix}\). The determinant would be \(\det((1)(4) - (2)(3) = -2\). Therefore, this matrix would not have an inverse.
              \end{enumerate}
    \end{enumerate}
}

\begin{exercise}
    {(Additional)}Is \texttt{All 2x2 matrices with real number entries} a ring with operations matrix addition and matrix multiplication? Justify your answer.
\end{exercise}

\sol{ \cmark
    \begin{enumerate}[label=(\arabic*)]
        \item \textbf{Additive Closure:} True.\ (See the previous exercise (a), (1)).
        \item \textbf{Additive Associativity:} True. Inherited from the properties of matrices.
        \item \textbf{Additive Identity:} True.\ (See the previous exercise (a), (3)).
        \item \textbf{Additive Inverse:} True. You can take the difference between a matrix and its inverted duplicate (e.g., \(-[A]\)) and get \(0\).
        \item \textbf{Multiplicative Closure:} True.\ (See the previous exercise (b), (4)).
        \item \textbf{Distributive Property:} True. While this will pose an error on a calculator, you can do the equivalent: \([A]([B] + [C]) = [A][B] + [A][C]\). This is true because you are still taking the summation of each \(a_{ij}\).\ \(b_{ij}\) and \(c_{ij}\).
    \end{enumerate}
}

\setcounter{section}{6}
\renewcommand{\theenumi}{\arabic{enumi}}
\renewcommand{\labelenumi}{\theenumi.}
\section{A Collision Algorithm for the DLP}

\begin{note}
    Recall that the DLP is the problem of finding \(x\) such that \(g^x \equiv h \ (\text{mod } p)\) for a given value of \(h\).
\end{note}

Remember that to brute force the DLP, it takes \(P - 1\) steps. Recall \(g^{p-1} \ (\text{mod } p) \equiv 1\). In computational complexity, \textbf{we say that the DLP is \(\mathcal{O}(P)\).}

\wrpprop{2.21}{(Shanks Baby-Step Giant-Step Algorithm) Computational time of \(\mathcal{O}(\sqrt{P})\). Below is the algorithm: \begin{enumerate}
        \item Let \(m = \lceil \sqrt{P} \rceil\).
        \item Create two lists:
              \begin{enumerate}
                  \item Baby steps: \(\{g^0, g^1, g^2, \ldots, g^{m}\}\).
                  \item Giant steps: \(\{h, h \cdot g^{-m}, h \cdot g^{-2m}, \ldots, h \cdot g^{-m^2}\}\).
              \end{enumerate}
        \item Find a match between the two lists: \(g^i \equiv hg^{-im} \ (\text{mod } p)\)
        \item \(x = i + jm\) is a solution for \(g^x \equiv h \ (\text{mod } p)\) (another way of saying ``\(x = i + jm\) is a solution for the \hyperref[Discrete Logarithm Problem]{DLP}'').
    \end{enumerate}}

\begin{example}
    {Baby-step, Giant-step}Use the Baby-step, Giant-step algorithm to solve for \(13^x \equiv 5 \ (\text{mod } 47)\).
\end{example}

\lesol{
    \begin{enumerate}
        \item Let \(m = \lceil \sqrt{47} \rceil = 7\).
        \item Create the two lists:
              \begin{enumerate}
                  \item Baby steps: \(\{13^0, 13^1, 13^2, \ldots, 13^6\}\ (\text{mod } 47) \equiv \{1,13,28,35,32,40,3,39\}\).
                  \item Giant steps: \(\{5, 5 \cdot 13^{-7}, 5 \cdot 13^{-14}, \ldots, 5 \cdot 13^{-49}\} \ (\text{mod } 47) \equiv \{5, 17, 39, 1, 48, 36, 19, 27\}\).
              \end{enumerate}
        \item Find a match between the two lists: \(39\) and \(39\), or \(1\) and \(1\).
        \item Substitute the following variables: \(i = 7, \ j = 2,\  n = 7\) for the equation \(x = i + jm \Rightarrow x = 7 + 2(7) = \boxed{21}\). So, \(13^{21} \equiv 5 \ (\text{mod } 47)\).
    \end{enumerate}
}

\begin{example}
    {(From Book) Baby-step, Giant-step} Solve the discrete logarithm problem with these values: \(g = 9704, h = 13896, p = 17389\).
\end{example}

\lesol{
The number \(9704\) has order 1242 in \(\F_{17389}^*\). Set \(n = \lceil \sqrt{1242} \rceil = 36\) and \(u = g^{-n} = 9704{-36} = 2494\). Table 2.4 in the book lists the values of \(g^k\) and \(h\cdot u^{k}\) for \(k = 1,2 \dots\). From the table, we find the collision \[9704^7 = 14567 = 13896 \cdot 2494^{32} \text{ in } \F_{17389}.\] Using the fact that \(2494 = 9704^{-36}\), we compute \[13896 = 9704^7 \cdot 2494^{-32} = 9704^7 \cdot {(9704^{36}})^{32} = 9704^{1159} \text{ in } \F_{17389}.\] Hence, \(x = 1159\) solves the problem \(9704^x = 13896\) in \(\F_{17389}\).
    }

\renewcommand{\theenumi}{\alph{enumi}}
\renewcommand{\labelenumi}{(\theenumi)}
\subsection{Exercises}

\begin{exercise}
    {2.17} Use Shanks's babystep-giantstep method (\refprop{2.21}) to solve the following discrete logarithm problems.
    \begin{enumerate}
        \item \(11^x = 21\) in \(\mathbb{F}_{71}\).
    \end{enumerate}
\end{exercise}

\sol{
    \begin{enumerate}
        \item \begin{enumerate}[label=\arabic*.]
                  \item Let \(m = \lceil \sqrt{70} \rceil = 9\)
                  \item Create two lists: \begin{itemize}
                            \item \textbf{Baby steps:} \[\{11^0, 11^1,\dots,11^9\} \pmod{71} \equiv \{1, \boxed{11}, 50, 53, 15, 23, 40, 14, 12\}\]
                            \item \textbf{Giant steps:} \[\{21, 21 \cdot 11^{-9}, 21 \cdot 11^{-18}, \dots\} \pmod{71} \equiv \{21, 5, 35, 32, \boxed{11}\dots\}\]
                        \end{itemize}
                  \item Find a match between the two lists: \(\boxed{11}\)
                  \item Substitute values for \(i + jm = 1 + 4(9) = 37\). So, \(11^{37} \equiv 21 \pmod{71}\)
              \end{enumerate}
    \end{enumerate}
}

\renewcommand{\theenumi}{\arabic{enumi}}
\renewcommand{\labelenumi}{\theenumi.}
\section{Chinese Remainder Theorem (CRT)}

\begin{theorem}
    {Chinese Remainder} Let \(n_1, n_2, \ldots, n_k\) be pairwise relatively prime integers. This means that \(\gcd(m_i,m_j) = 1 \textit{ for all } i \ne j\). Then, for any integers \(a_1, a_2, \ldots, a_k\), the system of congruences \begin{align*}
        x & \equiv a_1 \ (\text{mod } n_1) \\
        x & \equiv a_2 \ (\text{mod } n_2) \\
          & \vdots                         \\
        x & \equiv a_k \ (\text{mod } n_k)
    \end{align*} has a unique solution \(c \ (\text{mod } n_1n_2 \ldots n_k)\).
\end{theorem}

\begin{example}
    {CRT}Solve the following system of congruences: \begin{align*}
        x & \equiv 6 \ (\text{mod } 7) \\
        x & \equiv 4 \ (\text{mod } 8)
    \end{align*}
\end{example}

\lesol{
    Note that \(x \equiv 6 \ (\text{mod } 7)\) means \begin{align*}
        x      & = 7n + 6                                \\
        7n + 6 & \equiv 4 \ (\text{mod } 8)              \\
        7n     & \equiv 6 \ (\text{mod } 8)              \\
        n      & \equiv 6 \cdot 7^{-1} \ (\text{mod } 8) \\
               & \equiv 6 \cdot 7\ (\text{mod } 8)       \\
               & \equiv 2 \ (\text{mod } 8),
    \end{align*} where \(2\ (\text{mod } 8) = 8m + 2 = n\). We plug this back into the following: \begin{align*}
        x & = 7(8m + 2) + 6 \ (\text{mod } 7 \cdot 8) \\
          & = 56m + 14 + 6 \ (\text{mod } 56)         \\
          & = 56m + 20 \ (\text{mod } 56).
    \end{align*} Note that \(56m\) is a multiple of \(56\) and so it will always be equal to 0. Thus, \(x = 20 \ (\text{mod } 56)\).
}

\begin{note}
    In general, to solve for CRT such that \(x \equiv a_1 \ (\text{mod } m_1)\dots, x \equiv a_k \ (\text{mod } m_k)\) we follow the algorithm below:
    \begin{enumerate}
        \item Let \(m = m_1 \cdot m_2 \cdots m_k\).
        \item Take \(n_i = \frac{m}{m_i}\).
        \item  Check to see if there is a solution, \(y_i\). \(y_i = n_i^{-1} \ (\text{mod } m_i)\). Note that the inverse exists because \(m_i\) and \(n_i\) are relatively prime.
        \item Compute \(x = a_1n_1y_1 + a_2n_2y_2 + \cdots + a_kn_ky_k \ (\text{mod } m)\).
    \end{enumerate}
\end{note}

\begin{example}
    {CRT with New Algorithm}
    Solve the following system of congruences: \(x \equiv a_1 \ (\text{mod } m_1)\) and \(x \equiv a_2 \ (\text{mod } m_2)\) where \(a_1 = 6, m_1 = 7, a_2 = 4, m_2 = 8\).
\end{example}

\lesol{
    \begin{enumerate}
        \item Let \(m = 7 \cdot 8 = 56\).
        \item Compute \(n_1 = 8\) and \(n_2 = 7\).
        \item Compute \(y_1 = 8^{-1} \ (\text{mod } 7) = 1\) and \(y_2 = 7^{-1} \ (\text{mod } 8) = 7\).
        \item Compute \begin{align*}
                  x & = 6 \cdot 8 \cdot 1 + 4 \cdot 7 \cdot 7 \ (\text{mod } 56) \\
                    & = 48 + 196 \ (\text{mod } 56)                              \\
                    & = 244 \ (\text{mod } 56)                                   \\
                    & = \boxed{20}.
              \end{align*}
    \end{enumerate}
}

\renewcommand{\theenumi}{\alph{enumi}}
\renewcommand{\labelenumi}{(\theenumi)}
\subsection{Exercises}



\begin{exercise}
    {2.18} Solve each of the following simultaneous systems of congruences (or explain why no solution exists).
    \begin{enumerate}
        \setcounter{enumi}{1} % Start enumeration at (b)
        \item \(x \equiv 137 \pmod{423}\) and \(x \equiv 87 \pmod{191}\).
              \setcounter{enumi}{3} % Start enumeration at (d)
        \item \(x \equiv 5 \pmod{9}\), \(x \equiv 6 \pmod{10}\), and \(x \equiv 7 \pmod{11}\).
    \end{enumerate}
\end{exercise}

%%
% Variables
% m_1 = 423
% m_2 = 191
% m = 80793
% n_1 = 191
% n_2 = 423
% a_1 = 137
% a_2 = 87
%%


\sol{
    \begin{enumerate}
        \setcounter{enumi}{1}
        \item \begin{enumerate}[label=\arabic*.]
                  \item Let \(m = 423 \cdot 191 = 80793\).
                  \item Compute \(n_1 = \frac{m}{m_1} = \frac{80793}{423} = 191\), and \(n_2 = \frac{80793}{191} = 423\)
                  \item Compute \(y_1 = 191^{-1} \pmod{423} \equiv 392\) and \(y_2 = 423^{-1} \pmod{191} \equiv 14\).
                  \item Compute \(x = (137)(191)(392) + (87)(423)(14) \pmod{80793} \equiv 27209\)
              \end{enumerate}
              \setcounter{enumi}{3} % Start enumeration at (d)
        \item \begin{enumerate}[label=\arabic*.]
                  \item Let \(m = 9 \cdot 10 \cdot 11 = 990\).
                  \item Compute \(n_1 = \frac{990}{9} = 110, \ n_2 = \frac{990}{10} = 99, \text{ and } n_3 = \frac{990}{11} = 90\).
                  \item Compute \(y_1 = 110^{-1} \pmod{9} \equiv 5, \ y_2 = 99^{-1} \pmod{10} \equiv 9\), and \(y_3 = 90^{-1} \pmod{11} \equiv 6\)
                  \item Compute \(x = (5)(110)(5) + (6)(99)(9) + (7)(90)(6) \pmod{990} = 986\)
              \end{enumerate}

    \end{enumerate}
}

\begin{exercise}
    {2.20}Let \( a, b, m, n \) be integers with \( \gcd(m, n) = 1 \). Let
    \[
        c \equiv (b - a) \cdot m^{-1} \pmod{n}.
    \]
    Prove that \( x = a + cm \) is a solution to
    \[
        x \equiv a \pmod{m} \quad \text{and} \quad x \equiv b \pmod{n}, \tag{2.24}
    \]
    and that every solution to (2.24) has the form \( x = a + cm + ymn \) for some \( y \in \mathbb{Z} \).
\end{exercise}

\expf{
Let \(a,b,m,n \in \Z\) with \(\gcd(m,n) = 1\). Let \(c \equiv (b - a)m^{-1} \pmod{n}\). Review the following: \begin{align*}
    x      & \equiv a \pmod{m}       \\
    a + cm & \equiv a \pmod{m}       \\
    a      & \equiv a \pmod{m} - cm.
\end{align*} Then, because \(cm\) is a multiple of \(m\), when we take the mod of \(a - cm \pmod{m}\), we will always get \(a\). Hence, \(a \equiv a \pmod{m}\). For the other equation, we will be using the def of \(c\) in our proof: \begin{align*}
    a + cm & \equiv b \pmod{n}             \\
    cm     & \equiv b - a \pmod{n}         \\
    c      & \equiv (b - a)m^{-1}\pmod{n}.
\end{align*} So, now when we multiply by \(m\) on both sides, we get \(cm \equiv b - a \pmod{n}\). Rearranging, we see \(cm + a \equiv b \pmod{n}\), so \(x\) is a solution.

For the second half of the proof, suppose we have \(x'\) as the solution for \(x' \equiv a \pmod{m}\) and \(x' \equiv b \pmod{n}\). We want to show that \(x' = x\). Thus, we subtract \(x\) from \(x'\) and get the following: \[
    x' - x \equiv a - a = 0 \pmod{m}, \text{ which implies } x' - x = km \text{ for some } k \in \Z.
\] This is the outcome because we know that for anything to be equal to 0 in modulus, the number itself must be a multiple of the modulus, \(m\). We can follow the same logic for \(b\), and see that \(x' - x \equiv b - b = lm\) for some \(l \in \Z\). Since \(\gcd(m,n) = 1\), \(x' - x\) must be a multiple of \(m\) and \(n\), meaning \(x' - x = ymn\) for some \(y \in \Z\). Therefore, \(x' = x + ymn, = a + cm + ymn\).
}

\renewcommand{\theenumi}{\arabic{enumi}}
\renewcommand{\labelenumi}{\theenumi.}
\section{Pohlig-Hellman Algorithm}
\label{sec:Pohlig-Hellman}

This algorithm is used to solve \(g^x \equiv h \ (\text{mod } p)\) for \(p\) prime and \(g\) primitive root. This has computational time of \(\mathcal{O}(\sqrt{p-1})\). Order of \(g\) is \(p-1\) is composite. This is most efficient when \(p - 1\) has small prime factors. Look below for the algorithm:

\begin{enumerate}
    \item Factor \(p - 1 = n_1^{e_1} \cdot n_2^{e_2} \cdots n_k^{e_k}\). (Note that \(\gcd(q_j, q_i) = 1 \  \forall i \ne j\).)
    \item For each \(1 \leq i \leq k\), let \(m_i = \frac{p-1}{n_i^{e_i}}\).
    \item Solve \(g^{x_i} \equiv h^{m_i} \ (\text{mod } p)\) for \(x_i\). (Note this DLP is easier because order of \(g_i\) is way less than the order of \(g\).)
    \item Use CRT to find \(x\) such that \(x \equiv x_1 \ (\text{mod } q_1^{e+1}), \dots, x_i \ (\text{mod } n_i^{e_i})\).
\end{enumerate}

\begin{example}
    {Pohlig-Hellman Algorithm} Solve the following DLP: \(7^x \equiv 12 \ (\text{mod } 41)\).
\end{example}

\lesol{
    \begin{enumerate}
        \item Factor \(41 - 1 = 40 = 2^3 \cdot 5 = 8 \cdot 5\).
        \item Find \(g_1\): \(g_1 = 7^{40/8} \ (\text{mod } 41) = 7^5 \ (\text{mod } 41) = 38 \). Find \(h_1\): \(h_1 = 12^5 \ (\text{mod } 41) = 3\).
        \item We can brute force solve for \(x_1\) up to 8 steps:
              \begin{align*}
                  38^{x_1} & \equiv 3 \ (\text{mod } 41) \\
                  x_1      & = 5 \ (\text{mod } 8).
              \end{align*} Continue for \(g_2\) and \(h_2\):
              \begin{align*}
                  g_2 & = 7^{40/5} \ (\text{mod } 41) = 7^8 \ (\text{mod } 41) = 37 \\
                  h_2 & = 12^8 \ (\text{mod } 41) = 18.
              \end{align*} Solve for \(x_2\):
              \begin{align*}
                  37^{x_2} & \equiv 18 \ (\text{mod } 41) \\
                  x_2      & = 3 \ (\text{mod } 5).
              \end{align*}
        \item Use CRT to solve for \(x\):
              \begin{align*}
                  x & \equiv 5 \ (\text{mod } 8)  \\
                  x & \equiv 3 \ (\text{mod } 5).
              \end{align*} This gives us \(M = 8\cdot 5 = 40\) with \(n_1 = 40/8 = 5\), \(n_2 = 40/ 5 = 8\), and \(y_1 \equiv 5^{-1} \ (\text{mod } 8) \equiv 5\), \(y_2 \equiv 8^{-1} \ (\text{mod } 5) \equiv 2\). Now we can substitute and see that \(x = 5 \cdot 5 \cdot 5 + 3 \cdot 8 \cdot 2 \ (\text{mod } 40) = \boxed{13}\).

    \end{enumerate}
}

\renewcommand{\theenumi}{\alph{enumi}}
\renewcommand{\labelenumi}{(\theenumi)}
\subsection{Exercise}

\begin{exercise}
    {2.28} Use the \hyperref[sec:Pohlig-Hellman]{Pohlig-Hellman algorithm} to solve the discrete logarithm problem \(g^x = a\) in \(\mathbb{F}_p\) in each of the following cases.
    \begin{enumerate}
        \item \(p = 433\), \(g = 7\), \(a = 166\).
    \end{enumerate}
\end{exercise}

\sol{
    We start by writing the given information into an equation that we can work with. Thus, \(7^x \equiv 166 \pmod{433}\). Since \(433\) is prime, \(\varphi(433) = 432\), which we can factor to \(16 \cdot 27\). Let \(x = a_0 + 16a_1\):
    \begin{align}
        (7^{a_0 + 16a_1})^{27}        & \equiv 166^{27} \pmod{433} \\
        (7^{27a_0 + 432a_1})          & \equiv                     \\
        (7^{27a_0} \cdot 7^{432a_1})  & \equiv                     \\
        (7^{27})^{a_0}(7^{a_1})^{432} & \equiv                     \\
        (265)^{a_0}                   & \equiv 250 \pmod{433}
    \end{align}
    For (3.1), we got the expression by substituting \(x\) for \(a_0 + 16a_1\) for the exponent in \(g^x \equiv \dots\). From there, (3.2) --- (3.4) is simple algebra. Then for (3.5), because \(7^{a_1}\) is congruent to 1, \((7^{a_1})^{432} = 1\). Additionally, we take \(7^{27} \pmod{432}\) to get \(265^{a_0}\). Then, we do the same thing for the other side of the equation. Now, we can brute force this by setting \(a_0 = \{0,1,2,3,\dots, 27\}\). We find that when \(a_0 = 15\), \(265^{a_0} \equiv 250 \pmod{433}\). Seeing that we have \(a_0\), we need to find \(b_0\):
    \begin{align*}
        (7^{b_0 + 27b_1})^{16} & \equiv 166^{16} \pmod{433} \\
        374^{b_0}              & \equiv 335 \pmod{433}
    \end{align*}
    Thus, through brute force, we find that \(b_0 = 20\). We can take our \(a_0\) and \(b_0\) to solve for \(x_1,x_2\):
    \begin{align*}
        x_1 & =  a_0 + 16a_1 \\
        x_1 & = 15 \pmod{16}
    \end{align*} and \begin{align*}
        x_2 & = b_0 + 27b_1  \\
        x_2 & = 20 \pmod{27}
    \end{align*}
    At this point, we can solve the \hyperref[thm:Chinese Remainder]{CRT}.
    \begin{enumerate}[label=\arabic*.]
        \item Let \(m = 16 \cdot 27 = 432\).
        \item Compute \(n_1 = \frac{432}{16} = 27\) and \(n_2 = \frac{432}{27} = 16\).
        \item Compute \(y_1 = 27^{-1} \pmod{16} \equiv 3\) and \(y_2 = 16^{-1} \pmod{27} \equiv 22\).
        \item Compute \(x = (27)(15)(3) + (20)(16)(22) \pmod{432} = 47\)
    \end{enumerate} I relied on \href{https://www.youtube.com/watch?v=CHfP5tBbiAg}{this video} heavily.
}

% \chapter{Integer Factorization and RSA}\label{RSA}
% \vspace*{-0.25in}
% \section{Euler's Totient Theorem and Roots Modulo pq}

\renewcommand{\theenumi}{\arabic{enumi}}
\renewcommand{\labelenumi}{\theenumi.}

Remember \hyperref[thm:Fermat's Little Theorem]{FLT} and \hyperref[Discrete Logarithm Problem]{DLP}? Can we use FLT for non-prime moduli? No, FLT is only true for prime moduli: \(2^5 \ (\text{mod 6}) = 32 \). Thus, we need to look to other ways of solving this problem for non-primes.

\begin{theorem}
    {Euler's Totient Theorem for phi}Let \(N\) be a positive integer and let \(a\) be an integer such that \(\gcd(a, N) = 1\). Then, \[
        a^{\varphi(N)} \equiv 1 \ (\text{mod }{n}).
    \] where \(\varphi(N) = \#\{0 \leq x < N \mid \gcd(x,N) = 1\}\)
\end{theorem}

\begin{corollary}
    {1: Euler's Totient Theorem for \textit{phi}}If \(N\) is prime, \(a^{\varphi(N)} \equiv 1 \ (\text{mod } N)\). For \(\varphi(N)\), we can express it as \(N-1\) from \hyperref[thm:Fermat's Little Theorem]{FLT}.
\end{corollary}

\cpf{
    If \(N\) is prime, then \(\gcd(a,N) = 1\). Thus, for all \(a\), such that \(0 < a < N\), \(\gcd(a,N) = 1\). Thus, \(\varphi(N) = N - 1\).
}

\begin{corollary}
    {2: Euler's Totient Theorem for \textit{phi}}If \(p,q\) are distinct primes, then \[
        a^{(p-1)(q-1)} \equiv 1 \ (\text{mod } pq) \quad \text{for all } a  \text{ such that } \gcd(a, pq) = 1.
    \] From this, we can state that \(N = pq\).
\end{corollary}

\cpf{
    For this proof, we would show that \(\varphi(pq) = \varphi(q) \cdot \varphi(p) = (p - 1)(q - 1)\).
}

\begin{example}
    {ETT for \textit{phi}}Find \(a^x \ (\text{mod } 15)\).
\end{example}

\lesol{
    \(15 = 3 \cdot 5\) (relatively prime); \(\varphi(15) = 2 \cdot 4 = 8\). Thus, \(a^8 \equiv 1 \ (\text{mod } 15)\).
}


\begin{theorem}
    {Euler's Totient Theorem for pq}Let \(p\) and \(q\) be distinct primes and let \[
        g = \gcd(p-1, q-1).
    \]Then, \[
        a^{(p-1)(q-1)/g} \equiv 1 \ (\text{mod } pq) \quad \text{for all } a  \text{ such that } \gcd(a, pq) = 1.
    \] In particular, if \(p\) and \(q\) are distinct primes, then \[
        a^{(p-1)(q-1)} \equiv 1 \ (\text{mod } pq) \quad \text{for all } a  \text{ such that } \gcd(a, pq) = 1.
    \]
\end{theorem}

Recall that for Diffie Hellman Elgamal we need to find \(a^x \ (\text{mod } p)\). Now, for RSA, we will be solving \(x^e \ (\text{mod } N)\) where \(N = pq\).

\wrpprop{3.2}{Let \(p\) be prime and let \(e\) be defined as \(\{e \in \Z \geq 1 \ \mid \ \gcd(e,p-q) = 1\}\). (Note this means that \(e^{-1} \text{mod } p - 1\) exists) Then, call \(d = e^{-1}\) i.e, \(d \equiv e^{-1} \ (\text{mod } p-1)\). Then, \(x^e \equiv c \ (\text{mod } p)\) has the unique solution \(x = c^d \ (\text{mod } p)\).}

\begin{example}
    {RSA 1}Find \(x\): \(x^3 \equiv 2 \ (\text{mod } 17)\).
\end{example}

\lesol{
    \begin{enumerate}
        \item Check with \(\gcd(3, 17-1) = \gcd(3,16)\) and 17 is a prime number. Thus, we can use \refprop{3.2}.
        \item Let \(d \cdot 3 \equiv 1 \ (\text{mod } 16) \equiv d \equiv 3^{-1} \ (\text{mod } 16) \equiv 11\) (Found with \hyperref[thm:Extended Euclidean Algorithm]{EEA})
        \item Then, \(x^3 \equiv 2 \ (\text{mod } 17) \Rightarrow x \equiv 2^{11} \ (\text{mod } 17) \equiv 8\).
        \item To check: \(8^3 = 512 \ (\text{mod } 17) = 2\) \cmark
    \end{enumerate}
}

\wrpprop{3.5}{Let \(p\) and \(q\) be distinct primes and let \(e \geq 1\) satisfy
    \[
        \gcd(e, (p - 1)(q - 1)) = 1.
    \]
    Then, \refprop{1.13} tells us that \(e\) has an inverse modulo \((p-1)(q-1)\), say
    \[
        de \equiv 1 \pmod{(p-1)(q-1)}.
    \]
    Then the congruence
    \[
        x^e \equiv c \pmod{pq}
    \]
    has the unique solution \(x \equiv c^d \pmod{pq}\).
}%

\begin{example}
    {RSA 2}Solve \(x^{169} \equiv 1000 \ (\text{mod } 6887)\)
\end{example}

\lesol{
    \begin{enumerate}
        \item Check \(\gcd(169, (70)(96)) = \gcd(169, 6720) = 1\) where 70 and 96 are from the prime factorization of \(6887\) such that \((71 - 1)(97 - 1)\) are from \((p - 1)(q - 1)\).
        \item Solve for \(d\) such that \(d169 \equiv 1 \pmod{6720}\). Using \hyperref[thm:Extended Euclidean Algorithm]{EEA}, we find that \(d \equiv -1511 \ (\text{mod} 6720) \equiv 5209\).
        \item Solve \(x \equiv 1000^{5209} \ (\text{mod } 6887) = 4055\)
    \end{enumerate}
}

\renewcommand{\theenumi}{\alph{enumi}}
\renewcommand{\labelenumi}{(\theenumi)}
\subsection{Exercises}

\begin{exercise}
    {3.1} {Solve the following congruences (use \hyperref[ex:3.3]{this example} for help).}
    \begin{enumerate}
        \item \(x^{19} \equiv 36 \pmod{97}\).
        \item \(x^{137} \equiv 428 \pmod{541}\).
        \item \(x^{73} \equiv 614 \pmod{1159}\).
        \item \(x^{751} \equiv 677 \pmod{8023}\).
    \end{enumerate}
\end{exercise}

\sol{
    \begin{enumerate}
        \item Because we know 97 is a prime number, and that \(\gcd(19, 96) = 1\), we can use \refprop{3.2}. Thus, we need to solve for \(d\): \begin{align*}
                  d \cdot 19 & \equiv 1 \pmod{96}       \\
                  d          & \equiv 19^{-1} \pmod{96} \\
                  d          & \equiv 91.
              \end{align*} Then, we can solve for \(x\): \begin{align*}
                  x^{19} & \equiv 36 \pmod{97}      \\
                  x      & \equiv 36^{91} \pmod{97} \\
                  x      & = 36.
              \end{align*}

        \item Using \refprop{3.2}, again:
              \begin{align*}
                  d \cdot 137 & \equiv 1 \pmod{540}        \\
                  d           & \equiv 137^{-1} \pmod{540} \\
                  d           & \equiv 473.
              \end{align*} Then, for \(x\):
              \begin{align*}
                  x^{137} & \equiv 428 \pmod{541}       \\
                  x       & \equiv 428^{473} \pmod{541} \\
                  x       & \equiv 213.
              \end{align*}
        \item \begin{align*}
                  d & \equiv 73^{-1} \pmod{1158} \\
                  d & \equiv 349.
              \end{align*} Then,
              \begin{align*}
                  x & \equiv 614^{349} \pmod{1159} \\
                  x & \equiv 598. 
              \end{align*}
        \item \begin{align*}
                  d & \equiv 751^{-1} \pmod{8022} \\
                  d & \equiv 235.
              \end{align*} Then, \begin{align*}
                  x & \equiv 677^{235} \pmod{8023} \\
                  x & \equiv 4858. 
              \end{align*}
    \end{enumerate}
}

\begin{exercise}
    {Additional Problem 1} Prove that \(\varphi(pq) = (p-1)(q-1)\) when \(p\) and \(q\) are distinct primes (where \(\varphi\) is \hyperref[Euler's Phi Function]{Euler's Phi Function}).
\end{exercise}

\expf{
    Let \(p\) and \(q\) be distinct primes. From the definition of \hyperref[Euler's Phi Function]{Euler's Phi Function}, we know \(\varphi(pq)\) counts the number of integers from 1 to \(pq - 1\) that are co-prime to \(pq\). Since \(p\) and \(q\) are distinct primes, the only numbers that are not co-prime to \(pq\) are multiples of \(p\) and multiples of \(q\).

    The numbers divisible by \(p\) are \(p, 2p, 3p, \dots, (q - 1)p\), for a total of \(q - 1\) multiples of \(p\). Similarly, the numbers divisible by \(q\) are \(q, 2q, 3q, \dots, (p - 1)q\), for a total of \(p - 1\) multiples of \(q\).

    Notice that the number \(pq\) itself is counted twice, once as a multiple of \(p\) and once as a multiple of \(q\). To correct this over-counting, we apply the \href{https://en.wikipedia.org/wiki/Inclusion\%E2\%80\%93exclusion_principle}{Inclusion-Exclusion Principle}. Thus, the total number of integers divisible by either \(p\) or \(q\) is:
    \[
        p + q - 1
    \]

    Since there are \(pq\) integers in total from 1 to \(pq - 1\), the number of integers that are co-prime to \(pq\) is:
    \[
        \varphi(pq) = pq - (p + q - 2) = pq - p - q + 1
    \]

    Simplifying the expression gives:
    \[
        \varphi(pq) = (p - 1)(q - 1) \qedhere
    \]
}

\renewcommand{\theenumi}{\arabic{enumi}}
\renewcommand{\labelenumi}{\theenumi.}
\section{RSA Cryptosystem}

\label{sec:RSA Algorithm}

\subsection{RSA Algorithm}

Remember, multiplying is easy; we have tools to compute large numbers. However, factoring is hard. The RSA Cryptosystem is based on the difficulty of factoring large numbers. Thus, the RSA Cryptosystem is based on the following steps:
\begin{enumerate}
    \item Alice chooses two distinct primes \(p\) and \(q\) (private key).
    \item With integer \(e\) such that \(\gcd(e, (p-1)(q-1)) = 1\).
    \item Alice computes \(N = pq\) and \(d = e^{-1} \ (\text{mod } (p-1)(q-1))\).
    \item For Bob to send ``\(m\)'' to Alice, he computes \(c = m^e \ (\text{mod } N)\).
    \item For Alice to decrypt, she computes \(m' = c^d \ (\text{mod } N)\).
\end{enumerate}
\pfs

We claim that \(m' = m\). \\

\noindent \textit{Proof.} \begin{align*}
    d  & \equiv e^{-1}\pmod{(p-1)(q-1)}                             \\
    de & \equiv 1                       & \pmod{(p-1)(q-1)}         \\
    de & = 1 + k(p-1)(q-1)              & \text{for some } k \in \Z
\end{align*} Now that we know \(de\), we can compute \(m'\):

\begin{align*}
    m' & = c^d \pmod{N}                                  \\
       & = m^{ed} \pmod{N}                               \\
       & = m^{1 + k(p-1)(q-1)} \pmod{N}                  \\
       & = m \cdot m^{k(p-1)(q-1)} \pmod{N}              \\
       & = m \cdot (m^{(p-1)(q-1)})^k \pmod{N} \tag{3.1} \\
       & = m \cdot 1^k \pmod{N}                          \\
       & = m \pmod{N}.
\end{align*} Note for (3.1), we got 1 because we substituted for \(\varphi\). Hence, \begin{align*}
    m (m^{(p-1)(q-1)}) \pmod{N} & \equiv m (m^{\varphi(pq)} \pmod{N}) \\
                                & \equiv m \cdot 1 \pmod{N}.
\end{align*} \qed

Remember from Chapter 1, for a Cryptosystem to be a \hyperref[Successful Ciphers]{successful cipher}, it must satisfy the following properties:

\begin{enumerate}
    \item \textbf{Encryption:} Must be easy to encrypt: \(c \equiv m^e \pmod{N}\).
    \item \textbf{Decryption:} Must be hard to decrypt:
          \begin{itemize}
              \item Need to find \(d \equiv e^{-1}\pmod{(p-1)(q-1)}\) which requires knowing \(p\) and \(q\) separately. Remember that factoring is hard.
          \end{itemize}
\end{enumerate}

\begin{example}
    {RSA 3}Let \(N = (47)(43) = 2021\) with private key \(47,43\) and \(e = 11\) (Note this means that \(e \ne 2 \cdot 23, \ 6 \cdot 7\), and so on). \begin{enumerate}[label=(\alph*)]
        \item Calculate Bob's public key.
        \item Encrypt the message \(m = 1050\) that Bob sends to Alice.
        \item Decrypt the message.
    \end{enumerate}
\end{example}

\lesol{
    \begin{enumerate}[label=(\alph*)]
        \item Calculate \(c\): \(c = 1050^{11} \pmod{2021} = 435\).
        \item Alice first finds \(d\): \(d \equiv 11^{-1} \pmod{(47 -1) (43 - 1)} \equiv 11^{-1} \pmod{1932}\). Using \hyperref[thm:Extended Euclidean Algorithm]{EEA}, we find that \(d \equiv 527\).
        \item Then, \(m' = 435^{527} \pmod{2021} = 1050\).
    \end{enumerate}
}

\begin{center}
    \textbf{Remarks}:
\end{center}
\begin{itemize}
    \item We often choose small \(e\) to make the encryption process faster.
    \item Most agree that a smaller \(e\) is just as secure. The normal choice for \(e\) is 65537.
    \item What is the smallest possible \(e\) for a semiprime \(pq\)? \(e = 3\)
    \item This cryptosystem may be insecure if \(d < N^{1/4}\).
\end{itemize}

\renewcommand{\theenumi}{\alph{enumi}}
\renewcommand{\labelenumi}{(\theenumi)}
\subsection{Exercises}

\begin{exercise}
    {3.7} {Alice publishes her \hyperref[sec:RSA Algorithm]{RSA} public key: modulus \(N = 2038667\) and exponent \(e = 103\).}
    \begin{enumerate}
        \item Bob wants to send Alice the message \(m = 892383\). What ciphertext does Bob send to Alice?
        \item Alice knows that her modulus factors into a product of two primes, one of which is \(p = 1301\). Find a decryption exponent \(d\) for Alice.
        \item Alice receives the ciphertext \(c = 317730\) from Bob. Decrypt the message.
    \end{enumerate}
\end{exercise}

\sol{
    \begin{enumerate}
        \item Bob needs to calculate \(c \equiv 892383^{103} \pmod{2038667} \equiv 45293\)
        \item To get the other prime factor of the modulus, Alice simply divides 2038667 by 1301. She gets 1567. Then Alice finds: \begin{align*}
                  d & \equiv 103^{-1} \pmod{(1301 - 1)(1567 - 1)} \\
                  d & \equiv 810367
              \end{align*}
        \item To decrypt, Alice solves for \(m\): \begin{align*}
                  m & \equiv 317730^{810367} \pmod{2038667} \\
                  m & \equiv 514407
              \end{align*}
    \end{enumerate}
}

\begin{exercise}
    {3.8} {Bob's RSA public key has modulus \(N = 12191\) and exponent \(e = 37\). Alice sends Bob the ciphertext \(c = 587\). Unfortunately, Bob has chosen too small a modulus. Help Eve \hyperref[sec:Ciphertext Attack]{by factoring} \(N\) and decrypting Alice's message. (Hint: \(N\) has a factor smaller than \(100\).)}
    \begin{enumerate}
        \item Factor \(N\).
        \item Decrypt Alice's message.
    \end{enumerate}
\end{exercise}

\sol{
    \begin{enumerate}
        \item \(N\)'s factors are \(p = 73\) and \(q = 167\). (This was found by simply looping through all integers less than 100 and dividing \(N\) by each. If I got a number that was not a decimal (i.e., \texttt{if m \% a == 0}), then the loop broke, and \(p\) and \(q\) were printed.)
        \item Eve will solve for \(d\): \begin{align*}
                  d & \equiv 37^{-1} \pmod{(73 - 1)(167 - 1)} \\
                  d & \equiv 11629
              \end{align*}


              Then she will solve for \(m\):
              \begin{align*}
                  m & \equiv 587^{11629} \pmod{12191} \\
                  m & \equiv 4894
              \end{align*}
    \end{enumerate}
}

\begin{exercise}
    {3.9} {For each of the given values of \(N = pq\) and \((p - 1)(q - 1)\), use the method described in \hyperref[sec:Ciphertext Attack]{Remark 3.11} to determine \(p\) and \(q\).}
    \begin{enumerate}
        \item \(N = pq = 352717\) and \((p - 1)(q - 1) = 351520\).
    \end{enumerate}
\end{exercise}

\sol{
    \begin{enumerate}
        \item First, we need get \((p + q)\) alone: \begin{align*}
                  (p-1)(q-1) & = pq - (p + q) + 1 \\
                  351520     & = 352718 - (p + q) \\
                  1198       & = p + q.
              \end{align*} Now, we can solve the quadratic equation \(x^2 -1198x + 352717\) to find \(p,q\). The answer is \(p = 521, \ q = 677\). To check: \(521 \cdot 677 = 352717\). \cmark%
    \end{enumerate}
}


\renewcommand{\theenumi}{\arabic{enumi}}
\renewcommand{\labelenumi}{\theenumi.}
\section{Implementation and Security Issues}

\subsubsection{Brute Force Attack}

This involves factoring \(N = pq\). We only consider numbers below \(\sqrt{N}\).


\subsection{Chosen Ciphertext Attack}
\label{sec:Ciphertext Attack}

If Eve can find \((p-1)(q-1)\), then \(p\) and \(q\) are easy.  \\

\noindent \textit{Proof.} \begin{align*}
    (p-1)(q-1) & = pq - p - q + 1   \\
               & = pq - (p +q) + 1.
\end{align*} Thus, Eve just needs to solve the quadratic equation \((x-p)(x-q) = x^2 - (p+q)x + pq = 0\). \qed

\begin{example}
    {CCA}Factor \(N = 2701\) such that \((p-1)(q - 1) = 2592\).
\end{example}

\lesol{
    Find \begin{align*}
        (p-1)(q-1) & = pq - (p + q) + 1   \\
        2592       & = 2701 - (p + q) + 1 \\
        110        & = p + q.
    \end{align*}
    Then, solve the quadratic equation \(x^2 - 110x + 2701 = 0\) to find \(p\) and \(q\). Thus, \(x = 37, 73\).
}

\subsection{Timing Attacks (Ex. of Side Channel Attack)}

This attack is based on the time it takes to decrypt a message. More specifically, it monitors how long it takes a computation to complete. Then, it derives the relative sizes of \(p\) and \(q\).

\renewcommand{\theenumi}{\arabic{enumi}}
\renewcommand{\labelenumi}{\theenumi.}
\section{Primality Testing}

The only primality tests that we know so far is the brute force method. In other words, we brute force finding prime numbers. We do this by dividing \(N\) by primes. If no numbers \hyperref[Divides]{divide} \(N\), then \(N\) is prime. This test is not efficient for large numbers.

Thus, consider a new test that utilizes \hyperref[thm:Fermat's Little Theorem]{FLT}.

\begin{theorem}
    {Fermat's Little Theorem, Version 2}If \(p\) is prime, then \(a^P \equiv a \pmod{p}\). Consider the contrapositive to this statement: if \(a^P \not\equiv 1 \pmod{p}\) for some \(a\), then \(P\) is \textit{not} prime.
\end{theorem}

\begin{example}
    {FLT}Let \(N = 21\). Is \(N\) prime?
\end{example}

\lesol{
    Pick \(a = 3\). Check \(3^{21} \stackrel{?}{\equiv} 3 \pmod{21}\). No, \(3^{21} \equiv 6 \pmod{21}\). Hence, 21 is not prime.
}

\wrpdef{Fermat Witness}{We call \(a\) a \textit{witness} to the fact that \(N\) is composite if \(a^N \not\equiv a \pmod{N}\).}

\wrpdef{Fermat Liar}{A \textit{Fermat liar} is a number \(a\) such that, even though \(N\) is composite, it satisfies \(a^{-1} \equiv 1 \pmod{n}\). This gives a false indication that \(N\) is prime. Simply put, \(a\) ``lies'' by making \(N\) appear prime when it is not.}

However, there are some numbers will force all \(a\)'s to be fermat liars. These are called \textit{Carmichael Numbers}.

\wrpdef{Carmichael Numbers}{A composite number \(N\) is a \textit{Carmichael Number} if \(a^N \equiv a \pmod{N}\) for all \(a\) such that \(\gcd(a,N) = 1\).}
\newpage
\begin{example}
    {Carmichael Numbers}Solve \(37^{561} \equiv \pmod{561}\). What can you conclude about 561? Is it prime?
\end{example}

\lesol{
    \begin{enumerate}
        \item Check if 561 is prime. Pick \(a = 37\). Check \(37^{561} \stackrel{?}{\equiv} 37 \pmod{561}\). Yes, \(37^{561} \equiv 37 \pmod{561}\). Thus, 561 is prime. Right?
        \item If we check for all \(a < 561\), we will see that 561 has no \hyperref[Fermat Witness]{Fermat witnesses}. However, we know that 561 is composite, so that means 561 is a Carmichael Number.
    \end{enumerate}
}
\subsection{Miller-Rabin Primality Test}
\label{sec:Miller-Rabin}

\wrpprop{3.17}{Let \(p\) be an odd prime, and write \(p - 1 = 2^kq\) with \(q\) being odd. Then, let \(a\) be any number not divisible by \(p\). Then, one of the following must hold: \begin{enumerate}[label=(\roman*)]
        \item \(a^q \equiv 1 \pmod{p}\).
        \item One of \(a^q,a^{2q},a^{4q},\dots,a^{2^{k-1}q}\) is congruent to \(-1\) modulo \(p\). 
    \end{enumerate}}

\begin{center}
    \textbf{Miller-Rabin Algorithm }    
\end{center}

\begin{enumerate}
    \item Choose an odd integer \(n > 2\) to test for primality.
    \item Write \(n - 1 = 2^k \cdot q\) where \(q\) is odd and \(k \geq 1\). 
    \item Pick a random integer \(a\) such that \(2 \leq a \leq n - 2\).
    \item compute \(x = a^q \pmod{n}\).
    \begin{itemize}
        \item If \(x = 1\) or \(x = n - 1\), \textbf{continue to the next base because this indicates that the number is probably prime.}
    \end{itemize}
    \item Square \(x\), up to \(k - 1\) times:
    \begin{itemize}
        \item If \(x = n - 1\) at any step, \textbf{continue to the next base for the same reason.}
        \item If \(x \ne n - 1\) for all squarings, \textbf{\(n\) is composite.}
        
        For example, if we have a number and we keep squaring \(x\) up to \(k - 1\), and we never encounter a \(n - 1\), but we encounter other numbers like 1 or anything else, then the number is composite. Remember: ONLY composite if there was NO record of a \(n - 1\) for a given base \(x\).
    \end{itemize}
    \item Repeat the test with different random bases \(a\).
\end{enumerate}

\wrpdef{Miller-Rabin Witness}{Let \(N\) be an odd number. Write \(N - 1 = 2^kq\) with \(q\) odd. If \(a\) is an integer that satisfies \(\gcd(a,N) = 1\), then \(a\) is a \textit{Miller-Rabin Witness} for \(N\) if both of the following hold: \begin{enumerate}[label=(\roman*)]
        \item \(a^q \not\equiv 1 \pmod{N}\).
        \item \(a^{2^iq} \not\equiv -1 \pmod{N}\) for all \(i\) such that \(0 \leq i \leq k-1\).
    \end{enumerate}}

\begin{example}
    {Miller-Rabin Test 1}Let \(N = 561\). Is \(N\) prime?
\end{example}

\lesol{
    \begin{enumerate}
        \item Write \(N - 1 = 2^4 \cdot 35\).
        \item Pick \(a = 7\). Start by solving \(7^{35} \pmod{561} \equiv 241\). If it was congruent to 1 or \(-1\), then \(N\) is a possible prime.
        \item If the number is neither congruent to 1 or \(-1\), keep squaring and checking up to \(k - 1\) times.
              \begin{itemize}
                  \item Like this: \((7^{36})^2 \pmod{561} \equiv 298\). If this number is congruent to 1, then we are done and the number is composite.
              \end{itemize}
        \item If it is congruent to \(-1\), then we choose another \(a\), and note this is a possible prime. If it is not congruent to either \(-1\) or 1, then keep squaring up to \(k - 1\) times.
        \item Thus, \((7^{36})^4 \pmod{561} \equiv 166\), \((7^{36})^8 \pmod{561} \equiv 67\), and \((7^{36})^{16} \pmod{561} \equiv 1\), so 561 is composite and 7 is a \hyperref[Miller-Rabin Witness]{M-R witness} because we found a 1 before we found a \(-1\) in the sequence.
    \end{enumerate}
}

\begin{example}
    {Miller-Rabin Test 2}Let \(N = 563\). Let \(a = 7\). Is \(N\) prime?
\end{example}

\lesol{
    \(N - 1 = 562 = 2(281)\). Start by solving \(7^{281} \pmod{563} \equiv 1\). Thus, 563 is a possible prime. Choose \(a = 8\). Then, \(8^{281} \pmod{563} \equiv -1\), so \(N\) is a possible prime.
}

\begin{example}
    {Miller-Rabin Test 3}Let \(N = 31001\). Determine if \(N\) is prime. 
\end{example}

\lesol{
    Start by finding \(p - 1 = 31000 = 2^3 \cdot 3875\). Then, use the fast powering algorithm to find \(b_0 = 2^{3875} \pmod{31001} \equiv 5799\) for \(b_1\). Since \(b_1\) is not 1 or \(-1\), we will try the other bases up to \(k - 1 = 3 - 1 = 2\). Thus, we use the fast powering algorithm to solve for subsequent \(b\)'s: 
    \[\begin{array}{lccccr}
        b_0^{3875} &=& 2^{3875} &\equiv& 5799 \pmod{31001} &= b_1 \\
        b^2_1 &=& 5799^2 &\equiv& 13026 \pmod{31001} &= b_2 \\
        b^2_2 &=& 13026^2 &\equiv& 8203 \pmod{31001} &= b_3
    \end{array}\] 
    Since no \(n - 1\) appeared in these steps, and there was not a 1 that appeared before a \(-1\), then this number is composite.
}

\begin{note}
    \textbf{Note:} Let \(N\) be an odd positive integer. Then, \(\geq 75\%\) of \(a\)'s are between \(1\) and \(n - 1\) are \hyperref[Miller-Rabin Witness]{Miller-Rabin Witnesses} for \(N\).
\end{note}

\renewcommand{\theenumi}{\alph{enumi}}
\renewcommand{\labelenumi}{(\theenumi)}
\subsection{Exercises}

\begin{exercise}
    {Additional Problem 2} Decide whether each of the following numbers is composite or probably prime by finding a witness or finding 5 liars using \hyperref[thm:Fermat's Little Theorem, Version 2]{Fermat's Little Theorem, V2}:
    \[8911; 13291; 415693\]
    For each of the possible primes left over from the previous part, use the \hyperref[sec:Miller-Rabin]{Miller-Rabin} test to either find a witness for the compositeness of the number or give 5 more liars to show the number is probably prime.
\end{exercise}

\sol{
    \begin{enumerate}
        \item For 8911: pick \(a = 2\). Thus, \(2^{8911} \stackrel{?}{\equiv} 2 \pmod{8911} \equiv 2\) \cmark. Thus, 2 is either a Fermat liar, or 8911 is prime. Let's continue for 4 more \(a\)'s:
              \begin{itemize}
                  \item \(a = 3^{8911} \pmod{8911} \stackrel{?}{\equiv} 3\)\ \cmark%
                  \item \(a = 4^{8911} \pmod{8911} \stackrel{?}{\equiv} 4\)\ \cmark%
                  \item \(a = 5^{8911} \pmod{8911} \stackrel{?}{\equiv} 5\)\ \cmark%
                  \item \(a = 6^{8911} \pmod{8911} \stackrel{?}{\equiv} 6\)\ \cmark%
              \end{itemize}
              Since every \(a\) we tested was true for the equation \(a^{N} \equiv a \pmod{N}\), 8911 is either prime, or all the \(a\)'s are liars. Now, we can use the Miller-Rabin test to further probe the possibility that 8911 is prime. We start by writing \(N - 1 = 2^kq\) with \(q\) odd. Thus, \(8911 = 2^1 \cdot 4455\). Now, we pick random \(a\)'s and solve for \(a^{4455} \pmod{8911} \stackrel{?}{\equiv} 1\):
              \begin{itemize}
                \item \(a = 3^{4455} \pmod{8911} \stackrel{?}{\equiv} 1\). Probably prime because \(\not\equiv 1\) or \(N - 1\).
                \item \(a = 5^{4455} \pmod{8911} \stackrel{?}{\equiv} 1\)\ \xmark. This gives 2813. Now, we need to use the \hyperref[fast powering algorithm]{fast powering algorithm} to check if for any \(k\), \(5^{2^k \cdot q}\) is equal to \(N - 1\) or \(1\):
                \begin{align*}
                    k_0 &= 2813 \pmod{8911} \\
                    k_1 &= 2813^2 \pmod{8911} \equiv 1 \\
                    k_2 &= 1^2 \pmod{8911} \equiv 1 \\
                    &\vdots
                \end{align*} Since we did not have any result of \(a^{2^k \cdot q} = N - 1\), 8911 is not prime with 5 as a Miller-Rabin witness.
              \end{itemize} 
        \item For 13291: pick \(a = 2\): \(2^{13291} \stackrel{?}{\equiv} 2 \pmod{13291} \equiv 2\) \cmark. Thus, 2 is either a Fermat liar, or 13291 is prime. Let's continue for 4 more \(a\)'s:
              \begin{itemize}
                  \item \(a = 3^{13291} \pmod{13291} \stackrel{?}{\equiv} 3\)\ \cmark%
                  \item \(a = 4^{13291} \pmod{13291} \stackrel{?}{\equiv} 4\)\ \cmark%
                  \item \(a = 5^{13291} \pmod{13291} \stackrel{?}{\equiv} 5\)\ \cmark%
                  \item \(a = 6^{13291} \pmod{13291} \stackrel{?}{\equiv} 6\)\ \cmark%
              \end{itemize}
              We can use the Miller-Rabin test to see if 13291 is prime. From the formula \(N - 1 = 2^kq\) with \(q\) odd: \(13290 = 2^1 \cdot 6645\). Pick random \(a\)'s and solve for \(a^{6645} \pmod{13291} \stackrel{?}{\equiv} 1\):
              \begin{itemize}
                  \item \(a = 3^{6645} \pmod{13291} \equiv 13290\)\ \cmark%
                  \item \(a = 7^{6645} \pmod{13291} \equiv 1\)\ \cmark%
                  \item \(a = 11^{6645} \pmod{13291} \equiv 13290\)\ \cmark%
                  \item \(a = 13^{6645} \pmod{13291} \equiv 13290\)\ \cmark%
                  \item \(a = 17^{6645} \pmod{13291} \equiv 13290\)\ \cmark%
              \end{itemize} Either 13290 is prime, or all \(a\)'s that we have chosen are Miller-Rabin liars.
        \item For 415693: pick \(a = 2\): \(2^{415693} \stackrel{?}{\equiv} 2 \pmod{415693} \equiv 116692\) \xmark. Therefore, 2 is a witness that 415693 is composite.
    \end{enumerate}
}

\renewcommand{\theenumi}{\arabic{enumi}}
\renewcommand{\labelenumi}{\theenumi.}
\section{Pollard's Factorization Algorithm}

For \(N = pq\) semiprime, this algorithm is efficient when \(p -1\) only has small prime factors.

This algorithm is based on the following steps:

\begin{itemize}
    \item \textbf{Step 1:} \textit{Choose a smooth bound B:} The smooth bound is a small number such that the prime factors of \(p - 1\) (where \(p\) is a prime factor of \(n\)) are less than or equal to \(B\).
    \item \textbf{Step 2:} \textit{Compute the least common multiple of all integers up to \(B\), call this number \(k\).}
    \item \textbf{Step 3:} \textit{Pick a random base a:} Choose an integer \(a\) such that \(2 \leq a \leq n - 2\).
    \item \textbf{Step 4:} \textit{Compute \(a^{k} \pmod{n}\)}.
    \item \textbf{Step 5:} \textit{Find the GCD of \(a^k - 1\) and \(n\):} The greatest common divisor \(\gcd(a^k - 1, n)\) may give a non-trivial factor of \(n\).
    \item \textbf{Step 6:} If \(\gcd(a^k - 1,n) = n\), try a different \(a\) or increase \(B\).
\end{itemize}

\begin{center}
    \textbf{In Python:}
\end{center}
\label{pollard's p python}
\begin{lstlisting}[language=Python, numbers=left]
def pollards_p_minus_1(base, mod, limit):
    lcm_val = 1
    for i in range(1, limit + 1):
        lcm_val = math.lcm(lcm_val, i)
        pow_val = pow(base, lcm_val, mod) - 1
        gcd_val = math.gcd(pow_val, mod)
        print(f"lcm({i}) = {lcm_val}, gcd: {gcd_val}")
        if 1 < gcd_val < mod:
            return gcd_val
    return None
\end{lstlisting}

\begin{example}
    {Pollard's Factorization Algorithm}Factor 540143 with \(a = 2\).
\end{example}

\lesol{ \vspace{0.5cm} \newline
    \begin{tabular}{l|l|c|r}
        \(B\) & \(\text{lcm}(k = 1,2\dots, 13)\)                                   & \(2^k - 1 \pmod{510143}\)             & \(\gcd(2^k - 1, 510143)\)      \\ \midrule
        5     & \(\text{lcm}(1,2,\underline{3},\underline{4},\underline{5}) = 60\) & \(2^{60 - 1 \pmod{540143}}\)          & \(\gcd(338353, 540143) = 1\)   \\ \midrule
        7     & \(\text{lcm}(1,2,\dots,7) = 420\)                                  & \(2^{420} - 1 \pmod{540143} = 42942\) & \(\boxed{421}\  \text{Stop!}\) \\  \bottomrule
    \end{tabular} \newline

    Thus, we found that 421 is a factor of 540143.
}

\renewcommand{\theenumi}{\alph{enumi}}
\renewcommand{\labelenumi}{(\theenumi)}
\subsection{Exercise}

\begin{exercise}
    {3.22} Use \hyperref[ex:3.10]{Pollard's} \(p - 1\) method to factor each of the following numbers.
    \begin{enumerate}
        \item \(n = 1739\)
        \item \(n = 220459\)
    \end{enumerate}
    Be sure to show your work and indicate which prime factor \(p\) of \(n\) has the property that \(p - 1\) is a product of small primes.
\end{exercise}

\sol{
    \begin{enumerate}
        \item
              \begin{itemize}
                  \item \textbf{Step 1:} Choose a smooth \(B = 5\). We assume that one of the prime factors \(p\) of \(n = 1739\) has the property \(p - 1\) is divisible by small primes, and we will try a small \(B\).
                  \item \textbf{Step 2:} Compute the least common multiple of all integers up to \(B = 5\): \[k = \text{lcm}(1,2,3,4,5) = 60.\]
                  \item \textbf{Step 3:} Pick a random base \(a = 2\).
                  \item \textbf{Step 4:} Compute \(a^{k} - 1 \pmod{n}\): \[2^{60} - 1 \pmod{1739} = 638.\]
                  \item \textbf{Step 5:} Compute the greatest common divisor of \(n\) and \(a^{k} - 1\): \[\gcd(1739, 638) = 1\]
                  \item Since the greatest common divisor is 1, we need to try a larger \(B\). We will try \(B = 7\): \[k = \text{lcm}(1,2,3,4,5,6,7) = 420.\]
                  \item Compute \[2^{420} - 1 \pmod{1739} = 157, \ \gcd(157, 1739) = 1.\]
                  \item Try \(B = 11\), and compute: \[2^{27720} - 1 \pmod{1739} = 407, \ \gcd(407, 1739) = 37\]
              \end{itemize}
              Thus, we have found a factor of 1739: 37. By dividing 1739 by 37, we get 47. Thus, 1739 factors into 37 and 47. Note that \(p - 1 = 36 = 2^2 \cdot 3^2\) and \(q - 1 = 46 = 2 \cdot 23\).
              %   \(B = 13\): \(2^{360360} \pmod{1739} = 1037\), \(\gcd(1037, 1739) = 37\) \\
              %   \(B = 17\): \(2^{12252240} \pmod{1739} = 1555\), \(\gcd(1555, 1739) = 37\) \\
              %   \(B = 19\): \(2^{232792560} \pmod{1739} = 667\), \(\gcd(666, 1739) = 37\) \\
        \item We will still be using the same steps from part (a), except this time we will skip straight to trying different \(B\)'s.
              \begin{itemize}
                  \item \(B = 5\): \(2^{60} - 1 \pmod{220459} = 120157\), \(\gcd(120157, 220459) = 1\).
                  \item \(B = 7\): \(2^{420} - 1 \pmod{220459} = 89023\), \(\gcd(89023, 220459) = 1\).
                  \item \(B = 11\): \(2^{27720} - 1 \pmod{220459} = 155869\), \(\gcd(155869, 220459) = 1\).
                  \item \(B = 13\): \(2^{360360} - 1 \pmod{220459} = 97948\), \(\gcd(97948, 220459) = 1\).
                  \item \(B = 17\): \(2^{12252240} - 1 \pmod{220459} = 144576\), \(\gcd(144576, 220459) = 1\).
                  \item \(B = 19\): \(2^{232792560} - 1 \pmod{220459} = 71838\), \(\gcd(71838, 220459) = 1\).
                  \item \(B = 23\): \(2^{5354228880} - 1 \pmod{220459} = 201150\), \(\gcd(201150, 220459) = 1\).
              \end{itemize}
              If we continue this process, we find a \(B = 5342931457063200\), which gives the factor 449. Divide 220459 by 449, and we get 491. Note that \(p - 1 = 448 = 2^6 \cdot 7\) and \(q - 1 = 490 = 2 \cdot 5 \cdot 7^2\). (See this \hyperref[pollard's p python]{Python code} for how I automated finding the factors of 220459.)
    \end{enumerate}
}



\renewcommand{\theenumi}{\arabic{enumi}}
\renewcommand{\labelenumi}{\theenumi.}
\section{Factorization via Difference of Squares}

We would only use this method if prime numbers, \(p\) and \(q\) are close to each other.

To factor \(N\) we find a value \(k\) such that \(N + k^2 = m^2\) for some \(m \in \Z\). This is equal to \(N = m^2 - k^2 = (m+k)(m-k)\).

\begin{example}
    {Factorization 1} Factor 45.
\end{example}

\lesol{
    \begin{align*}
        45 + 1^2 & = 46 \ \text{\xmark} \\
        45 + 2^2 & = 49 \ \text{\cmark} \\
        45       & = 7^2 - 2^2          \\
        45       & = (7 + 2)(7 - 2)     \\
        45       & = (9)(5)
    \end{align*}
}


\begin{example}
    {Factorization 2} Factor \(N = 25217\) by looking for an integer \(b\) making \(N + b^2\) a perfect square.
\end{example}

\lesol{
    We factor \( N = 25217 \) by looking for an integer \( b \) making \( N + b^2 \)
    a perfect square:


    \begin{align*}
        25217 + 1^2 & = 25218         & \text{not a square},            \\
        25217 + 2^2 & = 25221         & \text{not a square},            \\
        25217 + 3^2 & = 25226         & \text{not a square},            \\
        25217 + 4^2 & = 25233         & \text{not a square},            \\
        25217 + 5^2 & = 25242         & \text{not a square},            \\
        25217 + 6^2 & = 25253         & \text{not a square},            \\
        25217 + 7^2 & = 25266         & \text{not a square},            \\
        25217 + 8^2 & = 25281 = 159^2 & \text{Eureka! \textbf{square}.}
    \end{align*}


    Then we compute

    \[
        25217 = 159^2 - 8^2 = (159 + 8)(159 - 8) = 167 \cdot 151.
    \]

}

\renewcommand{\theenumi}{\alph{enumi}}
\renewcommand{\labelenumi}{(\theenumi)}
\subsection{Exercise}

\begin{exercise}
    {3.24} {For each of the following numbers \(N\), compute the values of \(N + 1^2, N + 2^2, N + 3^2, N + 4^2\),\dots as we did in Example 3.34 (\hyperref[ex:3.11]{Example 3.11 in notes}), until you find a value \(N + b^2\) that is a perfect square \(a^2\). Then use the values of \(a\) and \(b\) to factor \(N\).}
    \begin{enumerate}
        \item \(N = 53357\)
        \item \(N = 34571\)
    \end{enumerate}
\end{exercise}

\sol{
    \begin{enumerate}
        \item     \begin{align*}
                  53357 + 1^2 & = 53358         & \text{not a square},            \\
                  53357 + 2^2 & = 53361 = 231^2 & \text{Eureka! \textbf{square}.}
              \end{align*} \(53357 = 231^2 - 2^2 = (231 + 2)(231 - 2) = 233 \cdot 229\). Therefore, 53357 factors into 233 and 229.
        \item \begin{align*}
                  34571 + 1^2 & = 34572         & \text{not a square},            \\
                  34571 + 2^2 & = 34575         & \text{not a square},            \\
                  34571 + 3^2 & = 34580         & \text{not a square},            \\
                  34571 + 4^2 & = 34587         & \text{not a square},            \\
                  34571 + 5^2 & = 34596 = 186^2 & \text{Eureka! \textbf{square}.}
              \end{align*} \(34571 = 186^2 - 5^2 = (186 + 5)(186 - 5) = 191 \cdot 181\). Therefore, 34571 factors into 191 and 181.
    \end{enumerate}
}



% \chapter{Digitial Signatures}\label{dig sig}
% \vspace*{-0.25in}
% \section{What Is a Digital Signature?}

We used RSA and Elgamal for confidentiality, whereas we use digital signatures for authentication. A digital signature is a way to ensure that a message is authentic, has not been tampered with, and is from the person who claims to have sent it.

\section{RSA Digital Signatures}

Recall RSA encryption and decryption. We have public key \((N = pq, e)\) where \(N\) is the modulus, \(e\) is the public exponent, and \(p, q\) are the prime factors of \(N\). We also have private key \(p,q\) where \(e\) has the following property: \(\gcd(e, (p-1)(q-1)) = 1\). This ensures a \(d\) exists such that \(d \equiv e^{-1} \pmod{(p-1)(q-1)}\).

\begin{note}
    Note: To gain a bit of efficiency, choose a \(d\) and \(e\) to satisfy \[
        de \equiv 1 \left(\text{mod}\frac{(p-1)(q-1)}{\gcd(p-1,q-1)}\right)
    \]
\end{note}

To sign a document \(D\), which we assume to be an integer in the range \(1 < D < N\), we compute the signature \(S\) as follows:
\[
    S \equiv D^{d} \pmod{N}
\]
To verify this signature, we compute:
\[
    D \equiv S^{e} \pmod{N}
\]

\begin{example}
    {RSA Digital Signature}Given the following \((p,q,a)\) as \((1223,1987,2430101)\) with the verification exponent \(e = 948074\), publish a document and verify its signature.
\end{example}

\lesol{
    Samantha computers her private signing key \(d\) using secret values of \(p\) and \(q\) to compute \((p-1)(q-1) = 1222 \cdot 1986 = 2426892\) and then solving the congruence
    \[
        ed \equiv 1 \pmod{(p-1)(q-1)}, \quad 948074d \equiv 1 \pmod{2426892}
    \]
    She finds that \(d = 1051235\). Samantha selects a digital document to sign, \[
        D = 1070777 \quad \text{with} \quad 1 \leq D < N.
    \]
    She computes the digital signature
    \[
        S \equiv D^{d}\pmod{N}, \quad S \equiv 1070777^{1051235} \equiv 153337 \pmod{2430101}.
    \]
    She then publishes the document and signature
    \[
        D = 1070777, \quad S = 153337.
    \]
    To verify the signature, the recipient computes
    \[
        S^{e} \pmod{N}, \quad 153337^{948074} \equiv 1070777 \pmod{2430101}.
    \]
    He verifies that the value of \(S^{e}\) modulo \(N\) is the same as the value of the digital document \(D = 1070777\).
}

\begin{table}[h!]
    \centering
    \begin{tabular}{@{}p{0.9\textwidth}l@{}}
        \toprule
        \multicolumn{2}{c}{\parbox[t]{0.9\textwidth}{\centering \textbf{Key creation}}}                                                                                                 \\ \midrule
        \multicolumn{1}{c}{\parbox[t]{0.45\textwidth}{\centering \textbf{Samantha}}} & \multicolumn{1}{c}{\parbox[t]{0.45\textwidth}{\centering \textbf{Victor}}}                       \\ \midrule
        \multicolumn{1}{l}{\parbox[t]{0.45\textwidth}{\raggedright Choose secret primes \(p\) and \(q\).                                                                                \\ Choose encryption exponent \(e\)  \\ \quad with \(\gcd(e, (p-1)(q-1)) = 1\). \\
        Publish \(N = pq\) and \(e\).}}                                              &                                                                                                  \\ \midrule
        \multicolumn{2}{c}{\parbox[t]{0.9\textwidth}\centering\textbf{Signing}}                                                                                                         \\ \midrule
        \multicolumn{1}{l}{\parbox[t]{0.45\textwidth}{\raggedright
        Compute \(d\) satisfying                                                                                                                                                        \\ \quad \(ed \equiv 1 \pmod{(p-1)(q-1)}\). \\ Sign document \(D\) by computing \\ \quad \(S \equiv D^{d} \pmod{N}\). }} &  \\ \midrule
        \multicolumn{2}{c}{\parbox[t]{0.9\textwidth}{\centering\textbf{Verification}}}                                                                                                  \\ \midrule
        \multicolumn{1}{l}{\parbox[t]{0.45\textwidth}}                               & \multicolumn{1}{r}{\parbox[t]{0.45\textwidth}{\raggedright Compute \(S^{e} \pmod{N}\) and verify \\ \quad that it equals \(D\).}}
        \\ \bottomrule
    \end{tabular}
    \caption{RSA Digital Signatures}
\end{table}


\section{Elgamal Digital Signatures}

Elgamal digital signatures are similar to RSA digital signatures. We have public key \((p, g, A)\). Where \(A\) is the public key from the expression \(A \equiv g^{a} \pmod{p}\) and private key \(a\). To sign a document \(D\), where \(1 < D < p\), choose a random \(k\) with \(\gcd(k,p-1) = k\).  Compute
\[
    S_{1} \equiv g^{k} \pmod{p} \quad \text{ and } \quad S_{2} \equiv (D - aS_{1})k^{-1} \pmod{p-1}.
\]

Victor verifies the signature by checking that
\[
    A^{S_{1}} S^{S_{2}}_{1} \pmod{p} \text{ is equal to } g^{D} \pmod{p}.
\]

\begin{example}
    {Elgamal Digital Signature}Given the following \((p,g,a)\) as \((21739,7,15140)\), sign a document and verify its signature.
\end{example}

\lesol{
First, we need to calculate \(A\):
\[
    A \equiv g^{a} \pmod{p}, \quad A \equiv 7^{15140} \pmod{21739} \equiv 17702.
\]
Next, we sign the digital document \(D = 5331\) using the random element \(k = 10727\) by computing
\begin{align*}
     & S_{1} \equiv g^{k} \equiv 7^{10727} \equiv 15775 \pmod{21739},                                          \\
     & S_{2} \equiv (D - aS_{1})k^{-1} \equiv (5331 - 15140 \cdot 15775) \cdot 6353 \equiv 10727 \pmod{21739}.
\end{align*}
Verify the signature by computing
\[
    A^{S_{1}}S^{S_{2}} \equiv 17702^{15775} \cdot 15775^{10727} \equiv 7^{5331} \equiv 13897 \pmod{21739}
\]
and verifying that it agrees with
\[
    g^{D} \equiv 7^{5331} \equiv 13897 \pmod{21739}.
\]
}

\begin{table}[h!]
    \centering
    \begin{tabular}{@{}p{0.9\textwidth}l@{}}
        \toprule
        \multicolumn{2}{c}{\parbox[t]{0.9\textwidth}{\centering \textbf{Public parameter creation}}}                                                                                         \\ \midrule
        \multicolumn{2}{c}{\parbox[t]{0.9\textwidth}{\centering A trusted party chooses and publishes a large prime \(p\) and an element \(g\) modulo \(p\) of large (prime) order.}}        \\ \midrule
        \multicolumn{2}{c}{\parbox[t]{0.9\textwidth}{\centering \textbf{Key creation}}}                                                                                                 \\ \midrule
        \multicolumn{1}{c}{\parbox[t]{0.45\textwidth}{\centering \textbf{Samantha}}} & \multicolumn{1}{c}{\parbox[t]{0.45\textwidth}{\centering \textbf{Victor}}}                            \\ \midrule
        \multicolumn{1}{l}{\parbox[t]{0.45\textwidth}{\raggedright Choose secret signing key                                                                                                 \\ \quad \(1 \leq a \leq p - 1\). \\ compute \(A = g^{a} \pmod{p}\).  \\ Publish the verification key \(A\).}}                                              &                                                                                                  \\ \midrule
        \multicolumn{2}{c}{\parbox[t]{0.9\textwidth}\centering\textbf{Signing}}                                                                                                              \\ \midrule
        \multicolumn{1}{l}{\parbox[t]{0.45\textwidth}{\raggedright
        Choose document \(D \pmod{p}\).                                                                                                                                                      \\ Choose random element \(1 < k < p\) \\ \quad satisfying \(\gcd(k,p-1) = 1.\) \\ Compute signature \\ \quad \(S_{1} \equiv g^{k} \pmod{p}\) and \\ \quad \(S_{2} \equiv (D - aS_{1})k^{-1} \pmod{p - 1}\).}} &  \\ \midrule
        \multicolumn{2}{c}{\parbox[t]{0.9\textwidth}{\centering\textbf{Verification}}}                                                                                                       \\ \midrule
        \multicolumn{1}{l}{\parbox[t]{0.45\textwidth}}                               & \multicolumn{1}{r}{\parbox[t]{0.45\textwidth}{\raggedright Compute \(A^{S_{1}} S_1^{S_{2}} \pmod{p}\) \\ Verify that it is equal to \(g^{D} \pmod{p}\).}}
        \\ \bottomrule
    \end{tabular}
    \caption{The Elgamal Digital Signature Algorithm}
\end{table}

% \setcounter{chapter}{5}

% \chapter{Elliptic Curves and Cryptography}\label{elliptic curves}
% \vspace*{-0.25in}
% \section{Elliptic Curves}
\newcommand{\op}{\oplus}
\newcommand{\fanO}{\mathcal{O}}
\begin{definition}
    {Elliptic Curve} An \textit{elliptic curve} \(E\) is the set of solutions to an equation of the form \(y^2 = x^3 + ax + b\), together with a point at infinity \(\mathcal{O}\), and the condition that \(4a^{3} + 27b^{2} \ne 0\). The last condition is to prevent singular points that cross. In other words, \(4a^{3} + 27b^{2} = 0 \equiv x^{3} + ax + b\) having 3 distinct roots.
\end{definition}

\begin{center}
    \textbf{Adding Two Elliptic Curve Points} \\

    We define ``\(\oplus\)'' as mapping: \(E \times E \to E\). From this, we get \(P \oplus Q = R\).
\end{center}

Define a line through \(P \op Q\). This line will intersect the curve at a third point, \(R\). Then \(R'\) is the reflection of \(R\) over the \(y\)-axis. \(P \oplus Q = R'\).

\begin{example}
    {Adding Two Elliptic Curve Points} Given the elliptic curve \(E\colon y^{2} = x^{3} - 36x\) with \(P = (-3,9)\), \(Q = (-2,8)\). Find \(P \oplus Q\).
\end{example}

\lesol{
    \begin{enumerate}
        \item Find slope: \(m = \frac{y_{2} - y_{1}}{x_{2} - x_{1}} = \frac{8 - 9}{-2 - (-3)} = -1\).
        \item Solve the equation of line: \(y = -x + b\). Plug in \(P\): \(9 = 3 + b \implies b = 6\). Thus, \(y = -x + 6\).
        \item Plug \(y\) back into given formula:
              \begin{align*}
                  (-x + 6)^{2}                & = x^{3} - 36x \\
                  x^{2} - 12x + 36            & = x^{3} - 36x \\
                  (-x^{3}) + x^{2} + 24x + 36 & = 0           \\
                  x^{3} - x^{2} - 24x - 36    & = 0
              \end{align*}
        \item Find the roots of the equation: \(x = -3, -2, 6\). Thus, \(R = (6,0)\) and \(R' = (6,0)\). Note two important things we did here:
              \begin{itemize}
                  \item We know that two of the roots are 3 and 2 because they are given. We got the third root, \(-6\), by solving for the cubic equation.
                  \item \(R\) and \(R'\) are the same value because to find \(R'\), we reflect \(R\) over the \(y\)-axis.
              \end{itemize}
        \item Conclude: \(P \op Q = R' = (6,0)\).
    \end{enumerate}
}

\begin{theorem}
    {Addition Law Properties} Let \(E\) be an elliptic curve. Then, the addition law on \(E\) has the following properties:\[
        \begin{array}{lrclll}
            \text{(a)} & P \op \mathcal{O} & = & \mathcal{O} \op P = P    & \text{for all } P \in E.       & \text{(Identity)}    \\
            \text{(b)} & P \op (-P)        & = & (-P) \op P = \mathcal{O} & \text{for all } P \in E.       & \text{(Inverse)}     \\
            \text{(c)} & (P \op Q) \op R   & = & P \op (Q \op R)          & \text{for all } P, Q, R \in E. & \text{(Associative)} \\
            \text{(d)} & P \op Q           & = & Q \op P                  & \text{for all } P, Q \in E.    & \text{(Commutative)} \\
        \end{array}\]
    In other words, the addition law makes the points of \(E\) into an Abelian group.
\end{theorem}

\tpf{
    \begin{enumerate}
        \item \textbf{Identity:} True because \(\mathcal{O}\) lies on all vertical lines.
        \item \textbf{Inverse:} Same reason as Identity. (Also, we defined \(\mathcal{O}\) as such.)
        \item \textbf{Associative:} Ignoring because hard.
        \item \textbf{Commutative:} Line through \(P \op Q\) is the same as the line through \(Q \op P\). Hence, \(P \op Q = Q \op P\).
    \end{enumerate}
}

\subsection{Special Cases for Adding Elliptic Curve Points}

\begin{theorem}
    {Elliptic Curve Addition Algorithm} Let
    \[
        E \colon y^{2} = x^{3} + ax + b
    \]
    be an elliptic curve, and let \(P_{1}\) and \(P_{2}\) be points on \(E\).
    \begin{enumerate}[label=(\alph*)]
        \item If \(P_{1} = \mathcal{O}\), then \(P_{1} + P_{2} = P_{2}\).
        \item Otherwise, if \(P_{2} = \mathcal{O}\), then \(P_{1} + P_{2} = P_{1}\).
        \item Otherwise, write \(P_{1} = (x_{1}, y_{1})\) and \(P_{2} =(x_{2},y_{2})\).
        \item If \(x_{1} = x_{2}\) and \(y_{1} = -y_{2}\), then \(P_{1} + P_{2} = \fanO\).
        \item Otherwise, define \(\lambda\) by
              \[\lambda =
                  \begin{cases}
                      \dfrac{y_{2} - y_{1}}{x_{2} - x_{1}} & \text{if } P_{1} \ne P_{2}, \\[0.5cm]
                      \dfrac{3x_{1}^{2} + A}{2y_{1}}       & \text{if } P_{1} = P_{2},
                  \end{cases}
              \]
              and let
              \[
                  x_{3} = \lambda^{2} - x_{1} - x_{2} \qquad \text{ and } \qquad y_{3} = \lambda (x_{1} - x_{3}) - y_{1}.
              \]
              Then, \(P_{1} + P_{2} = (x_{3},y_{3})\).
    \end{enumerate}
\end{theorem}

\newcommand{\dydx}{\frac{dy}{dx}}

\begin{center}
    \textbf{Verbatim From Notes Today In Class}
\end{center}

For \(P = (x_{1}, y_{1})\), \(Q = (x_{2}, y_{2})\).
\begin{enumerate}[label=\arabic*.]
    \item \(P = Q\): This means \(x_{1} = x_{2}\), and there is no slope because \(x_{2} - x_{2} = 0\). Thus, this ``line,'' is actually a point tangent to the curve. Thus, to find the slope of the line, we need to differentiate.
    \item \(P = \mathcal{O}\) or \(Q = \mathcal{O}\): This means that the line is vertical, and the sum is the other point. In other words, \(P \op \mathcal{O} = P\).
\end{enumerate}

For the first case, consider the following example:

\begin{example}
    {Case 1} Solve for \(R'\) with the elliptic curve \(E \colon y^{2} = x^{3} - 36x\).
\end{example}

\lesol{To solve this we first need to differentiate to get the slope for our equation:
    \begin{align*}
        y^{2}   & = x^{3} - 36x                 \\
        2y\dydx & = 3x^{2} - 36                 \\
        \dydx   & = \frac{3x^{2} - 36}{2y}      \\
        \dydx   & = \frac{3(-3)^{2} - 36}{2(9)} \\
        \dydx   & = \frac{-1}{2}
    \end{align*}
    From here, we solve \(y - y_{0} = m(x - x_{0})\):
    \begin{align*}
        y - 9 & = \frac{-1}{2}(x+3)           \\
        y     & = \frac{-1}{2} + \frac{15}{2}
    \end{align*}
    Now, we can substitute our \(x\) and \(y\) values back into the original \(y^{2} = x^{3} + ax + b\):
    \begin{align*}
        \left(\frac{-1}{2}x + \frac{15}{2}\right)^{2}            & = x^{3} - 36x \\
        \frac{1}{4}x^{2} - \frac{15}{2}x + \frac{225}{4}         & = x^{3} - 36x \\
        x^{3} - \frac{1}{4}x^{2} - \frac{57}{2}x - \frac{225}{4} & = 0           \\
        (x + 3)(x+3)(x-\dfrac{25}{4})                            & = 0
    \end{align*}
    Hence, \(x = \frac{25}{4}\) and \(y = \frac{-1}{2}(\frac{25}{4}) + \frac{15}{2} - \frac{35}{8}\).

    Therefore, \(R' = P \op P = (\frac{25}{4},\frac{-35}{8})\). Note that \(\frac{35}{8}\) is negative because we flipped it along the \(y\)-axis.
}

\section{Elliptic Curves over Finite Fields}

\begin{example}
    {Elliptic Addition with Modulo} Given the elliptic curve \(E \colon y^{2} = x^{3} + 2x + 2 \pmod{17}\) with points \(P = (5,1)\) and \(Q = (16,13)\).
    \begin{enumerate}[label=(\alph*)]
        \item Find \(P \op Q\).
        \item Find \(P \op P\).
    \end{enumerate}

\end{example}

\lesol{
    \begin{enumerate}[label=(\alph*)]
        \item First, we need to find lambda. Using the formula for lambda in the \hyperref[thm:Elliptic Curve Addition Algorithm]{Elliptic Curve Addition Algorithm} part (e), first condition, we have: \(\lambda = \frac{12}{11}\), but remember, we are in modulo, so we need to find the modular inverse of 11. This is 14. Thus, \(\lambda = 12 \cdot 14 = 15\). (Note there is a quick trick of subtracting the number by 17 to get a smaller number to work with. For example, 12 and 14 are \(-5\) and \(-3\), respectively. When we multiply these, we get the same answer: 15. Hence, we can use this trick to make our calculations easier.)

              Use the formula for \(x_{3}\) and \(y_{3}\) to find the point \(R\). For \(x_{3}\):
              \begin{align*}
                  x_{3} & \equiv (15)^{2} - 5 - 16 \\
                        & \equiv (-2)^{2} - 5 - 16 \\
                        & \equiv -17               \\
                        & \equiv 0 \pmod{17}       \\
              \end{align*}
              Then, for \(y_{3}\):
              \begin{align*}
                  y_{3} & \equiv 15(5 - 0) - 1 \\
                        & \equiv (-2)(5) - 1   \\
                        & \equiv -11           \\
                        & \equiv 6 \pmod{17}
              \end{align*}
              This gives us the point \(R = (0,6)\).

        \item For \(P \op P\), we have \(P = (5,1)\). Now, we use the \hyperref[thm:Elliptic Curve Addition Algorithm]{Elliptic Curve Addition Algorithm} part (e), second condition, to find lambda. We have
              \[
                  \lambda = \frac{3(5)^{2} + 2}{2(1)} = 9 \cdot 2^{-1} \equiv 9 \cdot 9 \equiv 13 \pmod{17}.
              \]
              Now, we can find \(x_{3}\) and \(y_{3}\):
              \begin{align*}
                  x_{3} & \equiv 13^{2} - 5 - 5 \\
                        & \equiv 169 - 10       \\
                        & \equiv 159            \\
                        & \equiv 6 \pmod{17}
              \end{align*}
              For \(y_{3}\):
              \begin{align*}
                  y_{3} & \equiv 13(5 - 6) - 1 \\
                        & \equiv 13(-1) - 1    \\
                        & \equiv -14           \\
                        & \equiv 3 \pmod{17}
              \end{align*}
              This gives us the point \(R = (6,3)\).
    \end{enumerate}
}

\begin{example}
    {Set of Points \(E(\F_{p})\)} Using the same elliptic curve from the last example, \(E \colon y^{2} = x^{3} + 2x + 2 \pmod{17}\) find the set of points \(E(\F_{17})\).
\end{example}

\lesol{ For this problem, we need to find all the squares modulo 17. We can do this by squaring all the numbers from 0 to 16. \\

    \begin{minipage}{0.48\textwidth}
        \begin{enumerate}
            \item First, find all the squared values:
                  \begin{itemize}
                      \item \(0^{2} = 0\)
                      \item \(1^{2} = 1 = 16^{2}\)
                      \item \(2^{2} = 4 = 15^{2}\)
                      \item \(3^{2} = 9 = 14^{2}\)
                      \item \(4^{2} = 16 = 13^{2}\)
                      \item \(5^{2} = 8 = 12^{2}\)
                      \item \(6^{2} = 2 = 11^{2}\)
                      \item \(7^{2} = 15 = 10^{2}\)
                      \item \(8^{2} = 13 = 9^{2}\)
                  \end{itemize}
        \end{enumerate}
        Notice that the squares are symmetric about 8. This is because the curve is symmetric about the \(y\)-axis.
    \end{minipage}
    \begin{minipage}{0.48\textwidth}
        \begin{enumerate}

            \item[2.] We need to find the \(y\)-values. We need to test each of the \(x\)-values in the equation \(y^{2} = x^{3} + 2x + 2\):
                  \begin{itemize}
                      \item \(0^{3} + 2(0) + 2 = 2\)
                      \item \(1^{3} + 2(1) + 2 = 5\)
                      \item \(2^{3} + 2(4) + 2 = 12\)
                      \item \(3^{3} + 2(9) + 2 = 1\)
                      \item \(4 \to 6\)
                      \item \(5 \to 1\)
                      \item \(6 \to 9\)
                      \item \(7 \to 2\)
                      \item \(8 \to 3\)
                      \item \(9 \to 1\), \(10 \to 2\), \(11 \to 2\), \(12 \to 3\), \(13 \to 15\), \(14 \to 3\), \(15 \to 7\), \(16 \to 16\). \\
                  \end{itemize}
        \end{enumerate}
    \end{minipage}

    \begin{enumerate}
        \item[3.] Now, given a \(y\)-value, we can search for the corresponding \(x\)-value. For example, \(y = 2\) corresponds to \(x = 6\) and \(x = 11\). We find the pairs to be:

              \[
                  \begin{aligned}
                       & \mathcal{O},       \\
                       & (0, 6), (0, 11),   \\
                       & (3, 1), (3, 16),   \\
                       & (5, 1), (5, 16),   \\
                       & (6, 3), (6, 14),   \\
                       & (7, 6), (7, 11),   \\
                       & (9, 1), (9, 16),   \\
                       & (10, 6), (10, 11), \\
                       & (13, 7), (13, 10), \\
                       & (16, 4), (16, 13).
                  \end{aligned}
              \]
              This yields the set of points \(E(\F_{17})\) to be 19 points in total.
    \end{enumerate}
}

\begin{theorem}
    {Hasse} The following formula gives an estimate for the number of points on an elliptic curve over a finite field:
    \[
        p + 1 - 2\sqrt{p} \le \#E(\F_{p}) \le p + 1 + 2\sqrt{p}.
    \]
\end{theorem}

\section{The Elliptic Curve Discrete Logarithm Problem (ECDLP)}


\begin{center}
\textbf{The Double-and-Add Algorithm}
\end{center}

\begin{enumerate}
    \item Write \(n\) in binary.
    \item Repeatedly double the point \(P\) up to the highest multiple of 2 in binary representation of \(n\).
    \item Take points corresponding to binary expansion of \(n\) and add them together.
\end{enumerate}

\begin{example}
    {Double-and-Add Algorithm} Use the double-and-add algorithm to compute \(E \colon y^{2} = x^{3} + 2x + 2 \pmod{17}\) with \(p = (5,1)\) and \(n = 11\).
\end{example}

\lesol{
    \begin{enumerate}
        \item Write \(n = 11\) in binary: \(11 = 1011\).
        \item Double the point \(P\) up to the highest multiple of 2 in the binary representation of \(n\): 
        \begin{align*}
            1P &= (5,1) \\
            2P &= (6,3) \\
            4P &= (3,1) \\
            8P &= (13,7)
        \end{align*}
        \item Solve for \(11P\): 
        \begin{align*}
            11P &= 8P + 2P + P \\
            &= (13,7) + (6,3) + (5,1) \\
            &= (7,11) + (5,1) \\
            &= (13,10).
        \end{align*}
        This algorithm takes \(\leq 2n\) ``steps'' to compute \(nP\).
    \end{enumerate}
}

\begin{center}
\textbf{Ternary Expansion of \(n\)}
\end{center}

\begin{enumerate}
    \item Write \(n\) in binary.
    \item Working from smaller powers of 2 to larger powers when you have 2 or more consecutive powers or 2, we can replace: 
    \begin{align*}
        &(2^{s + 5}) + 1(2^{s + t - 1}) + 1(2^{s + t - 2}) + \cdots + 1(2^{s}) \\
        &= 2^{s + t} - 2^{s}.
    \end{align*}
    This allows us to ``cancel out'' middle terms of consecutive powers of 2. We take the next largest power of 2, for a string of 2s, and subtract the next smallest power of 2.
\end{enumerate}

\begin{example}
    {Ternary Expansion} Find the ternary expansion of 11.
\end{example}

\lesol{
    \begin{enumerate}
        \item Write 11 in binary: 11 = 1011.
        \item Replace the binary expansion with the ternary expansion: 
        \begin{align*}
            11 &= 8 + 2 + 1 \\
            &= 1(8) + 1(4)  + {1(2)} + 1(1) \\
            &= 8 + 4 + 1(2) - 1 \\
            &= 11.
        \end{align*}
    \end{enumerate}
}

\section{Elliptic Curve Cryptography}

\subsection{Elliptic Curve Diffie-Hellman Key Exchange}

\begin{table}[h!]
    \centering
    \begin{tabular}{@{}p{0.9\textwidth}l@{}}
        \toprule
        \multicolumn{2}{c}{\parbox[t]{0.9\textwidth}{\centering \textbf{Public parameter creation}}} \\ \midrule
        \multicolumn{2}{c}{\parbox[t]{0.9\textwidth}{\centering A trusted party chooses and publishes a large prime \(p\), an elliptic curve \(E\) over \(\F_{p}\) and a point \(P\) in \(E(\F_{p})\). }}        \\ \toprule
        \multicolumn{2}{c}{\parbox[t]{0.9\textwidth}{\centering \textbf{Private Computations}}} \\ \midrule
        \multicolumn{1}{c}{\parbox[t]{0.45\textwidth}{\centering \textbf{Alice}}} & \multicolumn{1}{c}{\parbox[t]{0.45\textwidth}{\centering \textbf{Bob}}} \\ \midrule
        \multicolumn{1}{l}{\parbox[t]{0.45\textwidth}{\raggedright Chooses a secret integer \(n_{A}\). \\ Computes the point \(Q_{A} = n_{A}P\)}} & \multicolumn{1}{c}{\parbox[t]{0.45\textwidth}{\raggedright Chooses a secret integer \(n_{B}\). \\ Computes the point \(Q_{B} = n_{B}P\) }} \\ \midrule
        \multicolumn{2}{c}{\parbox[t]{0.9\textwidth}\centering\textbf{Public Exchange of Values}} \\ \midrule
        \multicolumn{1}{l}{\parbox[t]{0.45\textwidth}{\raggedright Alice sends \(Q_{A}\) to Bob}} &  \\ \midrule
        \multicolumn{1}{l}{\parbox[t]{0.45\textwidth}}                               & \multicolumn{1}{r}{\parbox[t]{0.45\textwidth}{\raggedright Bob sends \(Q_{B}\) to Alice}} \\ \toprule
        \multicolumn{2}{c}{\parbox[t]{0.9\textwidth}{\centering\textbf{Further Private Computations}}} \\ \midrule
        \multicolumn{1}{l}{\parbox[t]{0.45\textwidth}{\raggedright Computes the point \(n_{A}Q_{B}\). }} & \multicolumn{1}{c}{\parbox[t]{0.45\textwidth}{\raggedright  Computes the point \(n_{B}Q_{A}\). }} \\\midrule 
        \multicolumn{2}{c}{\parbox[t]{0.9\textwidth}\centering Their shared secret value is \(n_{A}Q_{B} = n_{A}(n_{B}P) = n_{B}(n_{A}P) = n_{B}Q_{A}\).} 
        \\ \bottomrule
    \end{tabular}
    \caption{Diffie-Hellman Key Exchange Using Elliptic Curves} 
\end{table}

\begin{example}
    {Elliptic Curve Diffie-Hellman Key Exchange} Given the elliptic curve \(E \colon y^{2} \equiv x^{3} + x + 6 \pmod{11}\) with point \(p = (5,9)\), Alice's private key \(n_{A} = 4\), and Bob's private key \(n_{B} = 7\), find the shared secret key. Use this website: \href{https://andrea.corbellini.name/ecc/interactive/modk-add.html}{Elliptic Curve Calculator}.
\end{example}

\lesol{ 
    First, we need to find \(Q_{A}\) and \(Q_{B}\). For \(Q_{A}\), we have \(n_{A} = 4\), so we need to find \(4P\). Using the double-and-add algorithm, we have \(n = 4 = 100\). Now, we need to solve for \(2P\) and \(4P\):
    \begin{align*}
        \lambda &= \frac{3(5)^{2} + 1}{2(9)} = \frac{76}{18} = 10 \cdot 18^{-1} \equiv -1 \cdot 7 \equiv -7 \equiv 4 \pmod{11} \\
    \end{align*}
    Now we can find \(x_{3}\) and \(y_{3}\). First, \(x_{3}\):
    \begin{align*}
        x_{3} &\equiv 4^{2} - 5 - 5 \\
        &\equiv 16 - 10 \\
        &\equiv 6 \pmod{11}.
    \end{align*}
    For \(y_{3}\):
    \begin{align*}
        y_{3} &\equiv 4(5 - 6) - 9 \\
        &\equiv 4(-1) - 9 \\
        &\equiv -4 - 9 \\
        &\equiv 9 \pmod{11}.
    \end{align*}
    Thus, \(2P = (6,9)\). Using the same process, we find that \(Q_{A} = 2(2P) = 2(6,9) = (3,6)\). 
    \begin{align*}
        1P &= (5,9) \\
        2P &= (6,9) \\
        4P &= (3,6).
    \end{align*} 

    Similarly, for \(7Q_{B} \Rightarrow 7P = (2,4)\). 
    
    From the image below, we can see that when we take the point \(n_{A}Q_{B} \Rightarrow 4 \cdot (2,4)\), we get the point \((10,9)\). Then, when we take the point \(n_{B}Q_{A} \Rightarrow 7 \cdot (3,6)\), we get the point \((10,9)\). Thus, the shared secret key is \((10,9)\).


\begin{center}
        \includegraphics[width=0.4\textwidth]{images/finding_shared_key.png} \hfill
        \includegraphics[width=0.4\textwidth]{images/finding_shared_key_2.png} \\
\end{center}
}

% \begin{example}
%     {Only Using the \(x\)-Coordinate} (Use the previous problem's values) Alice and Bob decide they only want to use the \(x\)-coordinate of their public key to preserve the amount of information they share. Show that they will get the same shared secret key.
% \end{example}

% \lesol{
%     We have \(Q_{A} = 3\) and \(Q_{B} = 2\). To get \(y_{1}\) from 
%     \[
%     y_{B}^{2} = x_{B}^{3} + x_{B} + 6 = 2^{3} + 2 + 6 = 16 + 2 + 6 = 24 \equiv 2 \pmod{11},
%     \]
%     we need to calculate the square root of 2 modulo 11. We get:
%     \[
%     y_{B}  = 2^{11 + 1} / 4
%     \]
% }
\newpage
\subsection{Elgamal Encryption Using Elliptic Curves}

\begin{table}[h]
    \centering
    \begin{tabular}{@{}p{0.9\textwidth}l@{}}
        \toprule
        \multicolumn{2}{c}{\parbox[t]{0.9\textwidth}{\centering \textbf{Public parameter creation}}}                                                                                  \\ \midrule
        \multicolumn{2}{c}{\parbox[t]{0.9\textwidth}{\centering A trusted party chooses and publishes a large prime \(p\), an elliptic curve \(E\) over \(\F_{p}\), and a point \(P\) in \(E(\F_{p})\).}} \\ \midrule
        \multicolumn{2}{c}{\parbox[t]{0.9\textwidth}{\centering \textbf{Key creation}}}                                                                                               \\ \midrule
        \multicolumn{1}{c}{\parbox[t]{0.45\textwidth}{\centering \textbf{Alice}}} & \multicolumn{1}{c}{\parbox[t]{0.45\textwidth}{\centering \textbf{Bob}}}                           \\ \midrule
        \multicolumn{1}{l}{\parbox[t]{0.45\textwidth}{Choose private key \(n_{A}\). \\ Compute \(Q_{A} = n_{A}P\) in \(E(\F_{p})\). \\ Publish the public key \(Q_{A}\).}} &  \\ \midrule
        \multicolumn{2}{c}{\parbox[t]{0.9\textwidth}\centering\textbf{Encryption}}                                                                                                    \\ \midrule
        \multicolumn{1}{l}{\parbox[t]{0.45\textwidth}}                            & \multicolumn{1}{r}{\parbox[t]{0.45\textwidth}{\raggedright Choose plaintext \(M \in E(\F_{p})\).                \\ Choose random element \(k\). \\ Use Alice's public key \(Q_{A}\) to \\ \quad compute \(C_1 = kP \in E(\F_{P})\) \\ \quad and \(C_2 = M + kQ_{A} \in E(\F_{P})\). \\ Send ciphertext \((C_1,C_2)\) to Alice.}}                                            \\ \midrule
        \multicolumn{2}{c}{\parbox[t]{0.9\textwidth}{\centering\textbf{Decryption}}}                                                                                                  \\ \midrule
        \multicolumn{1}{l}{\parbox[t]{0.45\textwidth}{Compute \(C_{2} - n_{A}C_{1} \in E(\F_{P})\). \\  This quantity is equal to \(M\).}}                     &                                                                                             \\ \bottomrule
    \end{tabular}
    \caption{Elgamal Key Creation, Encryption, and Decryption with Elliptic Curves}
\end{table}

\begin{example}
    {Elgamal Encryption Using Elliptic Curves} Given the elliptic curve \(E \colon y_{2} \equiv x_{3} + 7x + 4 \pmod{17}\) with \(P = (3,1)\), \(n_{A} = 15\), \(M = (16,8)\), and \((K = 5)\), find \(Q_{A}\), and encrypt and decrypt the message.
\end{example}

\lesol{
    We find \(Q_{A}\) to be \((11,16)\)
    \[
    C_{1} = 5(3,1) = (0,15).
    \]
    Then, 
    \[
    C_{2} = (16,8) + 5(11,16) = (3,1).
    \]
    Alice can decrypt the message by computing:
    \[
    (3,1) \ominus 15(0,15) = (3,1) \ominus (2,14) = (3,1) \oplus (2,3) = (16,8).
    \]
    (Note the subtraction is just taking \(-y\).)
}


\renewcommand{\theenumi}{\alph{enumi}}
\renewcommand{\labelenumi}{(\theenumi)}


% \chapter*{Exercises pt. 1} \label{ex 1}
% \vspace*{-0.25in}
% \fancyhead[R]{Exercises 1}
% \begin{exercise}
    {1.1}Build a cipher wheel as illustrated in Figure 1.1, but with an inner wheel that rotates, and use it to complete the following tasks.
    \begin{enumerate}
        \item Encrypt the following plaintext using a rotation of 11 clockwise.
              \begin{center}
                  ``A page of history is worth a volume of logic.''
              \end{center}
        \item Decrypt the following message, which was encrypted with a rotation of 7 clockwise.
              \begin{center}
                  \texttt{AOLYLHYLUVZLJYLAZILAALYAOHUAOLZLJYLALZAOHALCLYFIVKFNBLZZLZ}
              \end{center}
    \end{enumerate}
\end{exercise}

\sol{
    \begin{enumerate}
        \item \texttt{L ALRP ZQ STDEZCJ TD HZCES L GZWFXP ZQ WZRTN}
        \item THERE ARE NO SECRETS BETTER THAN THE SECRETES [sic] THAT EVERY BODY GUESSES \\
              In the encrypted text, ``Secrets'' is \texttt{ZLJYLAZ}. Then, they use an incorrect spelling of the word, \texttt{ZLJYLALZ}, of which has an extra `e' in it. That is what the ``[sic]'' is for.
    \end{enumerate}
}

\begin{exercise}
    {1.2}Decrypt each of the following Caesar encryptions by trying the various possible
    shifts until you obtain readable text.
    \begin{enumerate}
        \item \texttt{LWKLQNWKDWLVKDOOQHYHUVHHDELOOERDUGORYHOBDVDWUHH}
        \item \texttt{UXENRBWXCUXENFQRLQJUCNABFQNWRCJUCNAJCRXWORWMB}
    \end{enumerate}
\end{exercise}

\sol{
    \begin{enumerate}
        \item I THINK THAT I SHALL NEVER SEE A BILLBOARD LOVELY AS A TREE
        \item LOVE IS NOT LOVE WHICH ALTERS WHEN IT ALTERATION FINDS
    \end{enumerate}
}


\begin{exercise}
    {1.3}For this exercise, use the simple substitution table given in Table 1.11.
    \begin{enumerate}
        \item Encrypt the plaintext message:
              \center{The gold is hidden in the garden}
    \end{enumerate}
\end{exercise}

\sol{
    \begin{enumerate}
        \item \texttt{IBX FEPA QL BQAAXW QW IBX FSVAXW}
    \end{enumerate}
}

\begin{exercise}
    {1.4} Each of the following messages has been encrypted using a simple substitution cipher. Decrypt them. For your convenience, we have given you a frequency table and a list of the most common bigrams that appear in the ciphertext.\ (If you do not want to recopy the ciphertexts by hand, they can be downloaded or printed from the web site listed in the preface.)
    \begin{enumerate}
        \item “A Piratical Treasure”
              \begin{flushleft}
                  \texttt{JNRZR BNIGI BJRGZ IZLQR OTDNJ GRIHT USDKR ZZWLG OIBTM NRGJN} \\
                  \texttt{IJTZJ LZISJ NRSBL QVRSI ORIQT QDEKJ JNRQW GLOFN IJTZX QLFQL} \\
                  \texttt{WBIMJ ITQXT HHTBL KUHQL JZKMM LZRNT OBIMI EURLW BLQZJ GKBJT} \\
                  \texttt{QDIQS LWJNR OLGRI EZJGK ZRBGS MJLDG IMNZT OIHRK MOSOT QHIJL} \\
                  \texttt{QBRJN IJJNT ZFIZL WIZTO MURZM RBTRZ ZKBNN LFRVR GIZFL KUHIM} \\
                  \texttt{MRIGJ LJNRB GKHRT QJRUU RBJLW JNRZI TULGI EZLUK JRUST QZLUK} \\
                  \texttt{EURFT JNLKJ JNRXR S}
              \end{flushleft}
    \end{enumerate}
\end{exercise}

\sol{
    \begin{enumerate}
        \item THESE CHARACTERS AS ONE MIGHT READILY GUESS FORM A CIPHER THAT IS TO SAY THEY CONVEY A MEANING BUT THEN FROM WHAT IS KNOWN OF CAPTAIN KIDD I COULD NOT SUPPOSE HIM CAPABLE OF CONSTRUCTING ANY OF THE MORE ABSTRUSE CRYPTOGRAPHS I MADE UP MY MIND AT ONCE THAT THIS WAS OF A SIMPLE SPECIES SUCH HOW EVER AS WOULD APPEAR TO THE CRUDE INTELLECT OF THE SAILOR ABSOLUTELY INSOLUBLE WITHOUT THE KEY
    \end{enumerate}
    \href{https://planetcalc.com/8047/}{Solver}.

}

\begin{exercise}
    {1.5}Suppose that you have an alphabet of 26 letters.
    \begin{enumerate}
        \item How many possible simple substitution ciphers are there?
        \item A letter in the alphabet is said to be fixed if the encryption of the letter is the letter itself. \textit{Show an example of how the pieces work together}
    \end{enumerate}
\end{exercise}

\sol{
    \begin{enumerate}
        \item \(26!\)
        \item This is the formula used for solving for derangements (where \(n\) is the number of elements in the set, and \(!n\) is the number of derangements [Definition: A permutation with no fixed points]): \(!n=n!\sum_{i=0}^{n}{\frac{(-1)^{i}}{i!}}\). From \href{https://en.wikipedia.org/wiki/Derangement}{Wikipedia}. For \(n = 2\). We can run through the following:
              \begin{enumerate}[label=(\arabic*)]
                  \item \(i = 0\): \(\frac{(-1)^0}{0!} = 1\)
                  \item \(i = 1\): \(\frac{(-1)^1}{1!} = -1\)
                  \item \(i = 2\): \(\frac{(-1)^2}{2!} = 0.5\)
              \end{enumerate}
              Sum them together: \(1 - 1 + 0.5 = 0.5\). Now we can get \(!2\): \[!2 = 2! \times (1 - 1 + 0.5) = 2 \times 0.5 = 1\]
    \end{enumerate}
}

\begin{exercise}
    {1.6}Let \(a, b, c \in Z\). Use the definition of divisibility to directly prove the following properties of divisibility.\ (This is \refprop{1.4})
    \begin{enumerate}
        \item If \(a \mid b\) and \(b \mid c\). Then \(a \mid c\).
        \item If \(a \mid b\) and \(b \mid a\). Then \(a = \pm b\).
        \item If \(a \mid b\) and \(a \mid c\). Then \(a \mid (b + c)\) and \(a \mid (b - c)\).
    \end{enumerate}
\end{exercise}

\sol{
    \begin{enumerate}
        \item Let \(a,b,c \in \Z\) such that \(a \mid b\) and \(b \mid c\). We know there exists an \(n \in \Z\) such that \(a \times n = b\). Similarly, \(b \mid c\) means there exists an \(k \in \Z\) such that \(b\times k = c\). We can use the commutative property to show that:
              \begin{align*}
                  k(an) & = (b)k \\
                  ank   & = bk   \\
                  bk    & = c    \\
                  a(nk) & = c    \\
                  a     & \mid c
              \end{align*}
        \item From the problem statement, we can intuitively ascertain that because both \(a,b\) divide each other, then it must be the case that they are the same number. Moreover, because the criteria for dividing is not pertinent to whether the quotient is negative or positive, this number can be positive or negative. Now that we have an idea of what we are trying to accomplish, we can begin the proof: Let \(a,b \in \Z\) such that \(a \mid b\) and \(b \mid a\). We know there exists an \(n \in \Z\) such that \(a \times n = b\). Similarly, \(b \mid a\) means there exists an \(k \in \Z\) such that \(b\times k = a\). Then, we can utilize substitution to get the following:
              \begin{align*}
                  bk    & = a \\
                  (an)k & = a
              \end{align*} From here, we know that because \(a\) is present on both sides of the equation, we should divide by \(a\) to simplify. Thus, consider the following two cases.
              \begin{itemize}
                  \item \textbf{Case 1:} \(a \ne 0\) \\
                        Since \(a\) is not zero, we can divide both sides by \(a\) to get \(nk = 1\). Since, \(n,k \in \Z\). We do not need to worry about fractional reciprocals. Instead, we know from the identity property of multiplication, that \(n,k\) must both be \(\pm 1\).\ \begin{itemize}
                            \item \textbf{\(n,k\) are both \(+1\):} Then, \(a = b \times 1 \text{ and } b = a \times 1\). Which simplifies to \(a = b\) in both cases.
                            \item \textbf{\(n,k\) are both \(-1\):} Then, \(a = b \times (-1) \text{ and } b = a \times (-1)\) Which simplifies to \(a = -b\) in both cases.
                        \end{itemize}
                  \item \textbf{Case 2:} \(a = 0\) \\
                        If \(a = 0\). Then \(b = 0 \times n = 0\) and \(0 = b \times k = 0\). Therefore, \(a = b = 0\). And \(a = \pm b\) is still true.
              \end{itemize}
              We have shown that in either case if \(a \mid b\) and \(b \mid a\). Then \(a = \pm b\).
        \item Because \(a\) needs to be divisible by both \(b\) and \(c\). We know that there must exist an \(n,k \in \Z\) such that \(b = an\) and \(c = ak\). Our goal is to get to the form, \(a \times \text{ some integer } = b + c\) and \(a \times \text{ some integer } = b - c\) so we can use the definition of divides to help us out here. Therefore, let us consider both \(b + c\) and \(b - c\) in two separate cases:
              \begin{itemize}
                  \item \textbf{Case 1:} \(b + c\)
                        \begin{align*}
                            b + c & = (an) + (ak) \\
                            b + c & = a(n + k)    \\
                            a     & \mid (b + c)
                        \end{align*}
                  \item \textbf{Case 2:} \(b - c\)
                        \begin{align*}
                            b - c & = (an) - (ak) \\
                            b - c & = a(n - k)    \\
                            a     & \mid (b - c)
                        \end{align*}
              \end{itemize}
              We have shown that if \(a \mid b\) and \(a \mid c\). Then \(a \mid (b + c)\) and \(a \mid (b - c)\).
    \end{enumerate}
}

\begin{exercise}
    {1.7}Use a calculator and the method described in Remark 1.9 to compute the following quotients and remainders.
    \begin{enumerate}
        \item 34787 divided by 353.
        \item 238792 divided by 7843.
    \end{enumerate}
\end{exercise}

\sol{
    \begin{enumerate}
        \item \(a = 34787\) and \(b = 353\). Then \(a / b \approx 98.54674220\). So \(q = 98\) and \(r = a - b \cdot q = 34787 - 353 \cdot 98 = 193\).
        \item \(a = 238792\) and \(b = 7843\). Then \(a / b \approx 30.446512\). So \(q = 30\) and \(r = a - b \cdot q = 238792 - 7843 \cdot 30 = 3502\).
    \end{enumerate}
}

\begin{exercise}
    {1.9}Use the Euclidean algorithm to compute the following greatest common divisors.
    \begin{enumerate}
        \item \(\gcd(291, 252)\).
        \item \(\gcd(16261, 85652)\).
    \end{enumerate}
\end{exercise}

\sol{
    \begin{enumerate}
        \item \(\gcd(291,252)\)
              \begin{enumerate}[label=(\arabic*)]
                  \item \(r_0 = 291\).\ \(r_1 = 252\).
                  \item \(i = 1\).
                  \item Divide \(r_0\) by \(r_1\) to get a quotient, \(q_1\) and a remainder, \(r_{2}\):
                        \begin{align*}
                            291 / 252            & = 1 = q_1  \\
                            291 - (252 \times 1) & = 39 = r_2
                        \end{align*}
                  \item \(r_2 \ne 0\). So, we continue.
                  \item \(i = 2 + 1 = 3\).
              \end{enumerate}
              \pfs
              \begin{enumerate}[label=(\arabic*)]
                  \setcounter{enumii}{2}
                  \item Divide \(r_1\) by \(r_2\) to get quotient, \(q_2\) and a remainder, \(r_3\):
                        \begin{align*}
                            252 / 39            & = 6 = q_2  \\
                            252 - (39 \times 6) & = 18 = r_3
                        \end{align*}
                  \item \(r_3 \ne 0\). So, we continue.
                  \item \(i = 3 + 1 = 4\).
              \end{enumerate}
              \pfs
              \begin{enumerate}[label=(\arabic*)]
                  \setcounter{enumii}{2}
                  \item Divide \(r_2\) by \(r_3\) to get quotient \(q_3\) and a remainder, \(r_4\):
                        \begin{align*}
                            39 / 18            & = 2 = q_3 \\
                            39 - (18 \times 2) & = 3 = r_4
                        \end{align*}
                  \item \(r_4 \ne 0\). So, we continue.
                  \item \(i = 3 + 1 = 5\).
              \end{enumerate}
              \pfs
              \begin{enumerate}[label=(\arabic*)]
                  \setcounter{enumii}{2}
                  \item Divide \(r_3\) by \(r_4\) to get quotient \(q_4\) and a remainder, \(r_5\):
                        \begin{align*}
                            18 / 3            & = 6 = q_4 \\
                            18 - (3 \times 6) & = 0 = r_5
                        \end{align*}
                  \item \(r_4 = 0\). So, we stop.
              \end{enumerate}
              We have found that the greatest common divisor is \(3\).
        \item \textit{To cut back on paper, I am going to avoid reiterating the steps of 4 and 5. If there is a continuation in the enumeration process, then \(r_i \ne 0\). And the process needs to continue}: \(\gcd(16261,85652) \Rightarrow \gcd(85652,16261)\).
              \begin{enumerate}[label=(\arabic*)]
                  \item
                        \begin{align*}
                            85652 / 16261            & = 5 = q_1    \\
                            85652 - (16261 \times 5) & = 4347 = r_2
                        \end{align*}
                  \item
                        \begin{align*}
                            16261 / 4347            & = 3 = q_2    \\
                            16261 - (4347 \times 3) & = 3220 = r_3
                        \end{align*}
                  \item
                        \begin{align*}
                            4347 / 3220            & = 1 = q_3    \\
                            4347 - (3220 \times 1) & = 1127 = r_4
                        \end{align*}
                  \item
                        \begin{align*}
                            3220 / 1127            & = 2 = q_4   \\
                            3220 - (1127 \times 2) & = 966 = r_5
                        \end{align*}
                  \item
                        \begin{align*}
                            1127 / 966            & = 1 = q_5   \\
                            1127 - (966 \times 1) & = 161 = r_6
                        \end{align*}
                  \item
                        \begin{align*}
                            966 / 161            & = 6 = q_6 \\
                            966 - (161 \times 6) & = 0 = r_7
                        \end{align*}
              \end{enumerate}
              We have found that \(\gcd(85652,16261) = 161\).

    \end{enumerate}
}

\begin{exercise}
    {1.10} For each of the \(\gcd(a, b)\) values in Exercise 1.9, use the extended Euclidean
    algorithm (Theorem 1.11) to find integers \(u\) and \(v\) such that \(au + bv = \gcd(a, b)\).
\end{exercise}

\sol{
    \begin{enumerate}
        \item We need to solve for the various \(u_i\) and \(v_i\). We will start at \(i = 2\).\ \\
              \begin{tabular}{c|c|c|c|c|c|c}
                  \(i\)      & \(u_i\) & \textit{Formula}         & \textit{Evaluation}   & \(v_i\) & \textit{Formula}         & \textit{Evaluation}   \\ \hline
                  \textit{2} & \(1\)   & \(u_0 - q_1 \times u_1\) & \(1 - 1 \times 0\)    & \(-1\)  & \(v_0 - q_1 \times v_1\) & \(0 - 1 \times 1\)    \\ \hline
                  \textit{3} & \(-6\)  & \(u_1 - q_2 \times u_2\) & \(0 - 6 \times 1\)    & \(7\)   & \(v_1 - q_2 \times v_2\) & \(1 - 6 \times (-1)\) \\ \hline
                  \textit{4} & \(13\)  & \(u_2 - q_3 \times u_3\) & \(1 - 2 \times (-6)\) & \(-15\) & \(v_2 - q_3 \times v_3\) & \(-1 - 2 \times 7\)   \\
              \end{tabular} \\

              \pfs

              Thus, we can now fill out the table in full:
              \begin{tabular}{c|c|c|c|c|c}
                  \textit{i} & \textit{\(r_i\)} & \textit{\(q_i\)} & \(r_{i + 1}\) & \textit{\(u_i\)} & \textit{\(v_i\)} \\ \hline
                  \textit{0} & \(291\)          & \(-\)            & \(-\)         & \(1\)            & \(0\)            \\ \hline
                  \textit{1} & \(252\)          & \(1\)            & \(39\)        & \(0\)            & \(1\)            \\ \hline
                  \textit{2} & \(39\)           & \(6\)            & \(18\)        & \(1\)            & \(-1\)           \\ \hline
                  \textit{3} & \(18\)           & \(2\)            & \(3\)         & \(-6\)           & \(7\)            \\ \hline
                  \textit{4} & \(3\)            & \(6\)            & \(0\)         & \(13\)           & \(-15\)          \\
              \end{tabular} \\

              Now, we need to solve: \(au + bv = \gcd(a,b) \Rightarrow 291(13) + 252(-15) = 3\).\ \(3\) matches the gcd that we found in Exercise 1.9, so this is the correct solution.

        \item

              \begin{tabular}{c|c|c|c|c|c}
                  \textit{i} & \textit{\(r_i\)} & \textit{\(q_i\)} & \(r_{i + 1}\) & \textit{\(u_i\)} & \textit{\(v_i\)} \\ \hline
                  \textit{0} & \(85652\)        & \(-\)            & \(-\)         & \(1\)            & \(0\)            \\ \hline
                  \textit{1} & \(16261\)        & \(5\)            & \(4347\)      & \(0\)            & \(1\)            \\ \hline
                  \textit{2} & \(4347\)         & \(3\)            & \(3220\)      & \(1\)            & \(-5\)           \\ \hline
                  \textit{3} & \(3220\)         & \(1\)            & \(1127\)      & \(-3\)           & \(16\)           \\ \hline
                  \textit{4} & \(1127\)         & \(2\)            & \(966\)       & \(4\)            & \(-21\)          \\ \hline
                  \textit{5} & \(966\)          & \(1\)            & \(161\)       & \(-11\)          & \(58\)           \\ \hline
                  \textit{6} & \(161\)          & \(6\)            & \(0\)         & \(15\)           & \(-79\)          \\
              \end{tabular} \\

              \(85652(15) + 16261(-79) = 161\).
    \end{enumerate}
}

\begin{exercise}
    {1.11} Let \(a\) and \(b\) be positive integers.
    \begin{enumerate}
        \item Suppose that there are integers \(u\) and \(v\) satisfying \(au + bv = 1\). Prove that \(\gcd(a, b) = 1\).
    \end{enumerate}
\end{exercise}

\expf{
    \begin{enumerate}
        \item Suppose there are integers \(a,b,u,v\) such that \(av + bv = 1\). Assume \(d \in \Z\) such that \(d = \gcd(a,b)\). Since \(d\) divides both \(a\) and \(b\) by definition of common divisor, it must also divide \(av\) and \(bv\) by definition of divisibility. Moreover, because \(au + bv = 1\) and \(d\) is a common divisor of both \(av\) and \(bv\). It must also divide \(1\) by Proposition 1.4 (c). Then, the only positive integer that divides \(1\) is \(1\) itself, so it must be the case that \(d = 1\). Therefore, since \(d = 1\) and \(\gcd(a,b) = d\). It follows that \(\gcd(a,b) = 1\).\ \qedhere
    \end{enumerate}
}

\begin{exercise}
    {1.14}Let \(m \geq 1\) be an integer and suppose that \[a_1 \equiv a_2 \ (\text{mod } m) \text{ and } b_1 \equiv b_2 \ (\text{mod } m).\]Prove that \[a_1 \pm b_1 \equiv a_2 \pm b_2 \ (\text{mod } m) \text{ and } a_1 \cdot b_1 \equiv a_2 \cdot b_2 \ (\text{mod } m).\](This is Proposition 1.13(a).)
\end{exercise}

\expf{
    Let \(m \geq 1\) be an integer and suppose that \(a_1 \equiv a_2 \ (\text{mod } m) \text{ and } b_1 \equiv b_2 \ (\text{mod } m)\). From the definition of modulo, we know the difference of \(a_1 - a_2\) and \(b_1 - b_2\) is divisible by \(m\).
    \begin{itemize}
        \item \textbf{Addition:} We want to show that \(a_1 + b_1 \equiv a_2 + b_2\). So, our goal is to achieve \(m \mid ((a_1 + b_1) - (a_2 + b_2))\). Thus, consider \((a_1 + b_1) - (a_2 + b_2)\). We can distribute the minus sign to get \((a_1 - a_2) + (b_1 - b_2)\). From Proposition 1.4 (c), because we know that \(m \mid (a_1 - a_2)\) and \(m \mid (b_1 - b_2)\). We can write this as \(m \mid ((a_1 - a_2) + (b_1 - b_2))\) which implies \(m \mid ((a_1 + b_1) - (a_2 + b_2))\). This shows \(a_1 + b_1 \equiv a_2 + b_2 \ (\text{mod } m)\).
        \item \textbf{Subtraction:} Similarly to addition, we want to show \(a_1 - b_1 \equiv a_2 - b_2 \ (\text{mod } m)\). Thus, consider \((a_1 - b_1) - (a_2 - b_2)\) which implies \( (a_1 - a_2) - (b_1 - b_2)\). From Proposition 1.4 (c), because we know that \(m \mid (a_1 - a_2)\) and \(m \mid (b_1 - b_2)\). We can write this as \(m \mid ((a_1 - a_2) - (b_1 - b_2))\) which implies \(m \mid ((a_1 - b_1) - (a_2 - b_2))\). So \(a_1 - b_1 \equiv a_2 - b_2 \ (\text{mod } m)\).
    \end{itemize}
    Therefore we have shown \(a_1 \pm b_1 \equiv a_2 \pm b_2 \ (\text{mod } m)\)
    \begin{itemize}
        \item \textbf{Product} We want to show that \(a_1 \cdot a_2 \equiv a_2 \cdot b_2 \ (\text{mod } m)\). Thus, consider \(a_1 \cdot b_1 - a_2 \cdot b_2\):
              \begin{align*}
                  a_1 \cdot b_1 - a_2 \cdot b_2 & = a_1 \cdot b_1 - a_1 \cdot b_2 + a_1 \cdot b_2 - a_2 \cdot b_2 \\
                                                & = a_1 \cdot (b_1 - b_2) + b_2 \cdot (a_1 - a_2)
              \end{align*}
              Because \(m \mid (b_1 - b_2)\) and \(m \mid (a_1 - a_2)\). We know from the definition of division that when we multiply those numbers by an integer like \(a_1\) and \(b_2\).\ \(m\) still divides the expression. Hence, \(m \mid (a_1 \cdot (b_1 - b_2))\) and \(m \mid (b_2 \cdot (a_1 - a_2))\). Therefore, \(m \mid (a_1 \cdot b_1 - a_2 \cdot b_2)\) and \(a_1 \cdot b_1 \equiv a_2 \cdot b_2 \ (\text{mod } m)\).\ \qedhere
    \end{itemize}
}

\begin{exercise}
    {1.16}Do the following modular computations. In each case, fill in the box with an integer between 0 and \(m - 1\). Where \(m\) is the modulus.
    \begin{enumerate}
        \item \(347 + 513 \equiv \boxed{\phantom{000}} \ (\text{mod } 763)\).
              \setcounter{enumi}{2}
        \item \(153 \cdot 287 \equiv \boxed{\phantom{000}} \ (\text{mod } 353)\)
              \setcounter{enumi}{4}
        \item \(5327 \cdot 6135 \cdot 7139 \cdot 2187 \cdot 5219 \cdot 1873 \equiv \boxed{\phantom{000}} \ (\text{mod } 8157)\) (\textit{Hint:} After each multiplication, reduce modulo \(8157\) before doing the next multiplication.)
              \setcounter{enumi}{6}
        \item \(373^6 \equiv \boxed{\phantom{000}} \ (\text{mod } 581)\).
    \end{enumerate}
\end{exercise}

\sol{
    \begin{enumerate}
        \item \(347 + 513 \ (\text{mod } 763) = 97\)
              \setcounter{enumi}{2}
        \item \(153 \cdot 287 \ (\text{mod } 353) = 139\)
              \setcounter{enumi}{4}
        \item
              \begin{align*}
                  5327 \cdot 6135 \ (\text{mod } 8157) & = 4203        \\
                  4203 \cdot 7139 \ (\text{mod } 8157) & = 3771        \\
                  3771 \cdot 2187 \ (\text{mod } 8157) & = 450         \\
                  450 \cdot 5219 \ (\text{mod } 8157)  & = 7491        \\
                  7491 \cdot 1873 \ (\text{mod } 8157) & = \boxed{603}
              \end{align*}
              \setcounter{enumi}{6}
        \item \(373^6 \ (\text{mod } 581) = 463\)
    \end{enumerate}
}

\begin{exercise}
    {1.17}Find all values of \(x\) between \(0\) and \(m - 1\) that are solutions of the following congruences.\ (\textit{Hint:} If you can't figure out a clever way to find the solution(s), you can just substitute each value \(x = 1\).\ \(x = 2,\dots\).\ \(x = m - 1\) and see which ones work.)
    \begin{enumerate}
        \item \(x + 17 \equiv 23 \ (\text{mod } 37)\).
              \setcounter{enumi}{2}
        \item \(x^2 \equiv 3 \ (\text{mod } 11)\)
              \setcounter{enumi}{6}
        \item \(x \equiv 1 \ (\text{mod } 5)\) and also, \(x \equiv 2 \ (\text{mod } 7)\).\ (Find all solutions modulo 35, that is, find the solutions satisfying \(0 \leq x \leq 34\).
    \end{enumerate}
\end{exercise}

\sol{
    \begin{enumerate}
        \item \(x = 6\)
              \setcounter{enumi}{2}
        \item We know that \(x\) cannot be any number who's square does not exceed 11 because we cannot square a number to get \(3\) (other than \(\sqrt{3}\). But these are only the integers, so we cannot use that). Hence, \(x \ne 1,2,3\) because we know that the results of these, \(1^2 = 1\).\ \(2^2 = 4\).\ \(3^2 = 9\) are not equivalent to 3. Let's try some more values: \(x = 4\): \(4^2 = 16 \ (\text{mod } 11) = 5 \ \slashed{\equiv} \ 3\); \(x = 5\): \(5^2 = 25 \ (\text{mod } 11) = 3 \equiv 3\); \(x = 6\): \(6^2 = 36 \ (\text{mod } 11) = 3 \equiv 3\); \(x = 7\): \(7^2 = 49 \ (\text{mod } 11) = 5 \ \slashed{\equiv} \ 3\); \(x = 8\): \(8^2 = 64 \ (\text{mod } 11) = 9 \ \slashed{\equiv} \ 3\); \(x = 9\): \(9^2 = 81 \ (\text{mod } 11) = 4 \ \slashed{\equiv} \ 3\); \(x = 10\): \(10^2 = 100 \ (\text{mod } 11) = 1 \ \slashed{\equiv} \ 3\).

              Thus, we have it that \(x = 5,6\).
              \setcounter{enumi}{6}
        \item Because we have \(x \equiv 1 \ (\text{mod } 5)\). Verifying correct \(x\)'s are straightforward. All we need to check for is if a multiple of \(5 + 1\) satisfies \(x \equiv 2 \ (\text{mod } 7)\). Thus, the only solution is \(x = 16\)
    \end{enumerate}
}

\begin{exercise}
    {1.19}Prove that if \(a_1\) and \(a_2\) are units modulo \(m\). Then \(a_1a_2\) is a unit modulo \(m\).
\end{exercise}

\expf{
    Suppose \(a_1\) and \(a_2\) are units modulo \(m\). This means \(a \in \Z/m\Z \colon gcd(a,m) = 1\). In other words, \(a_1b_1 \equiv 1 \ (\text{mod } m)\) and \(a_2b_2 \equiv 1 \ (\text{mod } m)\) for some \(b_1, b_2 \in \Z\). When we multiply the equations together, we get \((a_1b_1)(a_2b_2) \equiv 1 \ (\text{mod } m)\) which can be rewritten as \((a_1a_2)(b_1b_2) \equiv 1 \ (\text{mod } m)\). We can multiply \(b_1\) and \(b_2\) to get an integer \(b_3\). Thus, when we multiply \(a_1a_2\) by \(b_3\) and get \(1\). We have shown that \(b_3\) is a multiplicative inverse, and \(a_1a_2\) is a unit modulo \(m\).
}

\begin{exercise}
    {(Additional)}Decide whether each of the following is a group:
    \begin{enumerate}
        \item \texttt{All 2\(\times\)2 matrices with real number entries} with operation matrix addition
        \item \texttt{All 2 \(\times\) 2 matrices with real number entries} with operation matrix multiplication
    \end{enumerate}

\end{exercise}

\sol{
    \begin{enumerate}
        \item \textbf{Matrix Addition:} \cmark
              \begin{enumerate}[label=(\arabic*)]
                  \item \textbf{Closure:} For addition to work between matrices, they must be of dimension \(2 \times 2 + 2 \times 2\). Therefore, the dimensions do not change, and it is closed.
                  \item \textbf{Associativity:} \(2 \times 2\) matrix addition is associative, as it inherits this property from the properties of matrices.
                  \item \textbf{Identity Element:} We can add a matrix \(Z\) that consists of only 0s to a matrix \(A\). And matrix \(A\) will remain unchanged.
                  \item \textbf{Inverse Element:} True. Consider the matrices,
                        \(\begin{bmatrix}
                            a & b \\
                            c & d
                        \end{bmatrix}\). And
                        \(\begin{bmatrix}
                            -a & -b \\
                            -c & -d
                        \end{bmatrix}\). When we add these two together we get
                        \(\begin{bmatrix}
                            0 & 0 \\
                            0 & 0
                        \end{bmatrix}\). This shows that \(2\times 2\) matrices have additive inverses.



              \end{enumerate}
        \item \textbf{Matrix Multiplication:} \xmark
              \begin{enumerate}[label=(\arabic*)]
                  \item \textbf{Closure:} The dimensions will stay the same during multiplication because it is an \(n \times n\) matrix.
                  \item \textbf{Associativity:} \(2 \times 2\) matrix multiplication is associative, as it inherits this property from the properties of matrices.
                  \item \textbf{Identity Element:} True. Consider the identity matrix, \(I_2 =
                        \begin{bmatrix}
                            1 & 0 \\
                            0 & 1
                        \end{bmatrix}\). When we multiply a matrix \(\begin{bmatrix}
                            a & b \\
                            c & d
                        \end{bmatrix}\) by \(I\). We get \[\begin{bmatrix}
                                a & b \\
                                c & d
                            \end{bmatrix}\] \(\cdot
                        \begin{bmatrix}
                            1 & 0 \\
                            0 & 1
                        \end{bmatrix} = \begin{bmatrix}
                            a & b \\
                            c & d
                        \end{bmatrix}\)
                  \item \textbf{Inverse Element:} False. Matrices with a non-zero determinant fail this criteria. Consider the matrix \(B =\)
                        \(\begin{bmatrix}
                            1 & 2 \\
                            3 & 4
                        \end{bmatrix}\). The determinant would be \(\det((1)(4) - (2)(3) = -2\). Therefore, this matrix would not have an inverse.
              \end{enumerate}
    \end{enumerate}
}

\begin{exercise}
    {(Additional)}Is \texttt{All 2x2 matrices with real number entries} a ring with operations matrix addition and matrix multiplication? Justify your answer.
\end{exercise}

\sol{ \cmark
    \begin{enumerate}[label=(\arabic*)]
        \item \textbf{Additive Closure:} True.\ (See the previous exercise (a), (1)).
        \item \textbf{Additive Associativity:} True. Inherited from the properties of matrices.
        \item \textbf{Additive Identity:} True.\ (See the previous exercise (a), (3)).
        \item \textbf{Additive Inverse:} True. You can take the difference between a matrix and its inverted duplicate (e.g., \(-[A]\)) and get \(0\).
        \item \textbf{Multiplicative Closure:} True.\ (See the previous exercise (b), (4)).
        \item \textbf{Distributive Property:} True. While this will pose an error on a calculator, you can do the equivalent: \([A]([B] + [C]) = [A][B] + [A][C]\). This is true because you are still taking the summation of each \(a_{ij}\).\ \(b_{ij}\) and \(c_{ij}\).
    \end{enumerate}
}
% \addcontentsline{toc}{chapter}{Exercises 1}



% \chapter*{Exercises pt. 2} \label{ex 2}
% \vspace*{-0.25in}
% \fancyhead[R]{Exercises 2}
% \input{Exercises/crypt_exercises_2}
% \addcontentsline{toc}{chapter}{Exercises 2}


% \chapter*{Exercises pt. 3}\label{ex 3}
% \vspace*{-0.25in}
% \fancyhead[R]{Exercises 3}
% \input{Exercises/crypt_exercises_3}
% \addcontentsline{toc}{chapter}{Exercises 3}



% \chapter*{Exercises pt. 4}\label{ex 4}
% \vspace*{-0.25in}
% \fancyhead[R]{Exercises 4}
% \input{Exercises/crypt_exercises_4}
% \addcontentsline{toc}{chapter}{Exercises 4}

\chapter*{Exercises pt. 5}\label{ex 5}
\vspace*{-0.25in}
\fancyhead[R]{Exercises 5}
\newcommand{\indco}{\text{IndCo}}
\setcounter{chapter}{5}

\begin{exercise}
    {5.10}Encrypt each of the following Vigen\`ere plaintexts using the given keyword
    and the Vigen\`ere tableau (Table 5.1 in book).
    \begin{enumerate}
        \item Keyword: \texttt{hamlet}

              Plaintext: \texttt{To be, or not to be, that is the question.}
    \end{enumerate}
\end{exercise}

\sol{
    \begin{enumerate}
        \item Hamlet is made up of 6 letters, so we repeat the keyword to match the length of the plaintext:
              \[
                  \texttt{tobeor\quad |\quad nottobe\quad |\quad thatis\quad |\quad theque\quad |\quad stion}
              \]
              Then, find the row that starts with the key-letter. For instance, since \texttt{hamlet} starts with \texttt{h}, we go down to the \(8^{\text{th}}\) row in Table 5.1. Then, we go right until we reach the column of our plaintext. So, for \texttt{t}, that would be the \(20^{\text{th}}\) column.

              See the table below for the full encryption. Note that \(\mathcal{P}, \mathcal{K}\), and \(\mathcal{C}\) indicate the plaintext, keyword, and ciphertext, respectively:
              \begin{center}
                  \begin{tabular}[htbp]{c||c|c|c|c|c}
                      \(\mathcal{P}\) & \texttt{tobeor} & \texttt{nottobe} & \texttt{thatis} & \texttt{theque} & \texttt{stion}  \\
                      \(\mathcal{K}\) & \texttt{hamlet} & \texttt{hamlet}  & \texttt{hamlet} & \texttt{hamlet} & \texttt{hamlet} \\
                      \(\mathcal{C}\) & \texttt{aonpsk} & \texttt{uofesu}  & \texttt{lttlxb} & \texttt{zttpun} & \texttt{lsftsg} \\
                  \end{tabular}
              \end{center}
    \end{enumerate}
}



\begin{exercise}
    {5.11}Decrypt each of the following Vigen\`ere ciphertexts using the given keyword and the Vigen\`ere tableau (Table 5.1 in book).
    \begin{enumerate}
        \item Keyword: \texttt{condiment}
              \vspace*{-0.8em}
              \begin{tabbing}
                  Ciphertext: \= \texttt{r s g h z \ b m c x t \ d v f s q \ h n i g q \ x r n b m} \\
                  \> \texttt{p d n s q \ s m b t r \ k u}
              \end{tabbing}
    \end{enumerate}
\end{exercise}
\sol{
    \begin{enumerate}
        \item Reversing the method from \hyperref[exerc:5.10]{Exercise 5.10}, we have: \\
              \begin{center}
                  \begin{tabular}[htbp]{c||c|c|c|c|c}
                      \(\mathcal{C}\) & \texttt{rsghzbmcx} & \texttt{tdvfsqhni} & \texttt{gpxrnbmpd} & \texttt{nsqsmbtrk} & \texttt{u} \\
                      \(\mathcal{K}\) & \texttt{condiment} & \texttt{condiment} & \texttt{condiment} & \texttt{condiment} & \texttt{c} \\
                      \(\mathcal{P}\) & \texttt{peterpipe} & \texttt{rpickedap} & \texttt{eckofpick} & \texttt{ledpepper} & \texttt{s} \\
                  \end{tabular} \\
              \end{center}

              we find the decrypted ciphertext to be:
              \begin{center}
                  \texttt{peter piper picked a peck of pickled peppers}
              \end{center}
    \end{enumerate}
}

\begin{exercise}
    {5.13}Let
    \begin{align*}
        s & = \text{``I am the very model of a modern major general.''}     \\
        t & = \text{``I have information vegetable, animal, and mineral.''}
    \end{align*}
    \begin{enumerate}
        \item Make frequency tables for \(s\) and \(t\).
        \item Compute \(\indco(s)\) and \(\indco(t)\).
    \end{enumerate}
\end{exercise}

\begin{table}[htbp]
    \centering
    \setlength{\tabcolsep}{4pt}
    \begin{tabular}{@{}ccccccccccccccccccccccccccc@{}}
        \toprule
                   & A & B & C & D & E & F & G & H & I & J & K & L & M & N & O & P & Q & R & S & T & U & V & W & X & Y & Z \\ \midrule
        Freq \(s\) & 4 & 0 & 0 & 2 & 6 & 1 & 1 & 1 & 1 & 1 & 0 & 2 & 4 & 2 & 4 & 0 & 0 & 4 & 0 & 1 & 0 & 1 & 0 & 0 & 1 & 0 \\ \midrule
        Freq \(t\) & 8 & 1 & 0 & 1 & 4 & 1 & 1 & 1 & 5 & 0 & 0 & 3 & 3 & 5 & 2 & 0 & 0 & 2 & 0 & 2 & 0 & 2 & 0 & 0 & 0 & 0 \\
        \bottomrule
    \end{tabular}
    \caption{Frequency Distribution for \(s\) and \(t\) }\label{tab:5.1}
\end{table}


\sol{
    \begin{enumerate}
        \item The frequency table for \(s\) and \(t\) is shown in \hyperref[tab:5.1]{Table 5.1}.
        \item We find the index of coincidence for \(s\) to be:
              \begin{align*}
                  \indco(s) & = \frac{1}{n(n - 1)}\sum_{i = 0}^{25} F_{i}(F_{i} - 1) \\ &= \frac{1}{36(35)}(4 \cdot 3 + 2 \cdot 1 + 6 \cdot 5 + \cdots + 0 \cdot 0) \\
                            & = \frac{84}{1260}                                      \\
                            & \approx 0.0667.
              \end{align*}
              Then, for \(t\), we have:
              \begin{align*}
                  \indco(t) & = \frac{1}{n(n - 1)}\sum_{i = 0}^{25} F_{i}(F_{i} - 1) \\ &= \frac{1}{41(40)}(8 \cdot 7 + 4 \cdot 3 + \cdots + 0 \cdot 0) \\
                            & = \frac{128}{1640}                                     \\
                            & \approx 0.0780.
              \end{align*}
    \end{enumerate}

}


\begin{exercise}
    {5.15}
    \begin{enumerate}
        \item One of the following two strings was encrypted using a simple substitution cipher, while the other is a random string of letters. the index of coincidence of each string and use the results to guess which is which.
              \begin{align*}
                  s_{1} & = \texttt{RCZBWBFHSLPSCPILHBGZJTGBIBJGLYIJIBFHCQQFZBYFP}, \\
                  s_{2} & = \texttt{KHQWGIZMGKPOYRKHUITDUXLXCWZOTWPAHFOHMGFEVUEJJ}.
              \end{align*}
    \end{enumerate}
\end{exercise}

\begin{table}[htbp]
    \centering
    \setlength{\tabcolsep}{4pt}
    \begin{tabular}{@{}ccccccccccccccccccccccccccc@{}}
        \toprule
                       & A & B & C & D & E & F & G & H & I & J & K & L & M & N & O & P & Q & R & S & T & U & V & W & X & Y & Z \\ \midrule
        Freq \(s_{1}\) & 0 & 7 & 3 & 0 & 0 & 4 & 3 & 3 & 4 & 3 & 0 & 3 & 0 & 0 & 0 & 3 & 2 & 1 & 2 & 1 & 0 & 0 & 1 & 0 & 2 & 3 \\ \midrule
        Freq \(s_{2}\) & 1 & 0 & 1 & 1 & 2 & 2 & 3 & 4 & 2 & 2 & 3 & 1 & 2 & 0 & 3 & 2 & 1 & 1 & 0 & 2 & 3 & 1 & 3 & 2 & 1 & 2 \\
        \bottomrule
    \end{tabular}
    \caption{Frequency Distribution for \(s_{1}\) and \(s_{2}\) }\label{tab:5.2}
\end{table}

\sol{
    \begin{enumerate}
        \item See the table below for the frequency distribution of \(s_{1}\) and \(s_{2}\) in \hyperref[tab:5.2]{Table 5.2}. We find the index of coincidence for \(s_{1}\) to be:
              \begin{align*}
                  \indco(s_{1}) & = \frac{114}{45(44)} \\
                                & \approx 0.0576.
              \end{align*}
              Then, for \(s_{2}\), we have:
              \begin{align*}
                  \indco(s_{2}) & = \frac{60}{45(44)} \\
                                & \approx 0.0303.
              \end{align*}

              Since the index of coincidence for \(s_{1}\) is higher than that of \(s_{2}\), we can guess that \(s_{1}\) was encrypted using a simple substitution cipher.
    \end{enumerate}
}




\begin{exercise}
    {5.17}We applied a Kasiski test to the Vigen\`ere ciphertext listed below and found that the key length is probably 5. Use Excel to find the plaintext and the key.
    \[\begin{array}{l}
            \texttt{togmg \ gbymk \ kcqiv \ dmlxk \ kbyif \ vcuek \ cuuis \ vvxqs \ pwwej \ koqgg} \\
            \texttt{phumt \ whlsf \ yovww \ knhhm \ rcqfq \ vvhkw \ psued \ ugrsf \ ctwij \ khvfa} \\
            \texttt{thkef \ fwptj \ ggviv \ cgdra \ pgwvm \ osqxg \ hkdvt \ whuev \ kcwyj \ psgsn} \\
            \texttt{gfwsl \ jsfse \ ooqhw \ tofsh \ aciin \ gfbif \ gabgj \ adwsy \ topml \ ecqzw} \\
            \texttt{asgvs \ fwrqs \ fsfvq \ rhdrs \ nmvmk \ cbhrv \ kblxk \ gzi}                   \\
        \end{array}\]
\end{exercise}


\sol{
    From the problem, I knew the key size was 5. Thus, I used the IndCo keysize 5 sheet. This is my order of operations:
    \begin{enumerate}[label=\arabic*.]
        \item Change the Ciphertext: Capitalize letters and remove spaces.
        \item Paste this into A1.
        \item Confirm the IndCo value is appropriate (\(\indco=0.064\), so it is).
        \item Copy the concatenated string from C122 into first the first Excel sheet we did.
        \item Find the lowest \(\chi^{2}\) value, so I know what value to subtract 26 by.
        \item Go to \href{https://rot13.com/}{rot13} and paste the string in, then rotate it by 26 minus the value we found in the previous step.
        \item Record this string in the Excel sheet.
        \item Repeat step 4-7 for the rest of the concatenated strings.
    \end{enumerate}
    We end up with 5 decrypted strings. Now, we need to read them from top to bottom. To concatenate the strings, I used the following python code in \hyperref[lst:5.1]{Listing 5.1}. Once I ran the code, I got the following plaintext:

    \begin{center}
        Radio, envisioned by its inventor as a great humanitarian contribution, was seized upon by the generals soon after its birth and impressed as an instrument of war. But radio turned over to the commander a copy of every enemy cryptogram it conveyed. Radio made cryptanalysis an end in itself.\footnotemark[1]
    \end{center}

    To find the key, I used the following python code in \hyperref[lst:5.2]{Listing 5.2}. The key was found to be \texttt{CODES}.
    \footnotetext{\(\phantom{0}^{1}\)A Google search with the decrypted words led me to the plaintext in proper grammatical form without capitalization and proper spacing.}
}
\newpage
\begin{lstlisting}[language=Python, caption=Python Code to Concatenate Strings, label={lst:5.1}]
    cipher_text = [
        'REIBITATNINUWIPTNSAIRDEANMFUINEHMRYEEYRCYDDPLAIE',
        'ANOYNOGHIATTAZOHESFTTISSSEWTOEREAAORMPAOEIETYNNL',
        'DVNIVRRUTNRISENEROTSHMSATNARTDTCNCFYYTMNDOCASEIF',
        'IIETEAEMACIOSDBGAOEBAPENRTRAUOOODOEECOIVRMRNINT',
        'OSDSNSAAROBNEUYELNRINRDIUOBDRVTMEPVNRGTEAAYASDS',
    ]
    
    # Number of rows and columns
    num_rows = len(cipher_text)
    num_cols = min(len(row) for row in cipher_text)
    
    # Read the text column by column
    decoded_text = ''
    for col in range(num_cols):
        for row in range(num_rows):
            decoded_text += cipher_text[row][col]
    
    # Print the result
    print('Decoded text (column-by-column):')
    print(decoded_text)
\end{lstlisting}
\vspace*{-0.5cm}
\begin{lstlisting}[language=Python, caption=Python Code to Find Key, label={lst:5.2}]
    # Plaintext and Ciphertext have been omitted because they would not fit in the page.
    plaintext = '...'
    ciphertext = '...'
    
    # Known key length
    key_length = 5
    
    # Helper function to convert letters to alphabetical index (A=0, B=1, ..., Z=25)
    def letter_to_index(letter):
        return ord(letter) - ord('A')
    
    # Helper function to convert index back to a letter
    def index_to_letter(index):
        return chr(index + ord('A'))
    
    # Calculate the key by determining the shift for each character in the key
    key = []
    for i in range(key_length):
        # Compute the shift for each position in the key
        shift = (letter_to_index(ciphertext[i]) - letter_to_index(plaintext[i])) % 26
        key.append(index_to_letter(shift))
    
    # Join the key characters into a string
    key_word = ''.join(key)
    print('Derived key:', key_word)
    
\end{lstlisting}
\addcontentsline{toc}{chapter}{Exercises 5}

\renewcommand{\theenumi}{\arabic{enumi}}
\renewcommand{\labelenumi}{\theenumi.}
\newpage
\fancyhead[R]{Exam Prep}


% \chapter*{Exam Prep}\label{prep}
% \vspace{-0.25in}
% \setcounter{chapter}{99}
\setcounter{ExampleCounter}{50}


\section*{Chapter 2 Review}

\begin{enumerate}
    \item Discrete logarithm
    \item Diffie-Hellman key
    \item El-gamal Encryption
    \item Attacks on DLP
          \begin{itemize}
              \item Babystep/Giantstep
              \item Chinese Remainder Th
              \item Pohlig-Hellman Alg
          \end{itemize}
\end{enumerate}



\begin{center}
    \textbf{Discrete Logarithm}
\end{center}

Solve for \(g^x \equiv h \pmod{p}\) by using logarithms. Specifically, find for an example, if we have \(2^x \equiv 10 \pmod{11}\), we would find \(\log_2(10) \pmod{11}\). Then, we would solve for various powers of 2: \(2^1 = 2 \pmod{11}\)\dots up to \(2^5 = 10 \pmod{11}\). Thus, we have found an \(x\), \(x = 5\) for which \(2^x \equiv 10 \pmod{11}\) is true.


\begin{center}
    \textbf{Diffie-Hellman Keys Exchange}
\end{center}

Straightforward algorithm that really only requires computation of 4 numbers. That is, if we are given \(p, \ g, \ a, \ b\), where \(p\) is the modulus and \(g\) is the agreed primitive root base and \(A\) and \(B\) are respective private keys. (Recall the \hyperlink{thm:Primitive Root Theorem}{Primitive Root Theorem} for how we find \(g\).)

This algorithm involves solving the discrete logarithm problem because if Eve were to obtain \(a\) and \(b\), she would still have to calculate \(a = \log_g(A)\) and \(b = \log_g(B)\) where \(A\) and \(B\) are respective public keys.

Formally, the algorithm goes like this:

\begin{enumerate}
    \item Choose large prime \(p\) and primitive root \(g\) and make a public key, \(k_{\text{pub}} = (p,g)\).
    \item Pick ints \(a\) and \(b\) such that \(k_{\text{priv} A} = a\) and \(k_{\text{priv} B} = b\). Compute \(g^a \pmod{p} = A\) and \(g^b \pmod{p} = B\).
    \item Exchange \(A\) and \(B\).
    \item Compute \(B^a \pmod{p}\) to get \(A'\), and compute \(A^b \pmod{p}\) to get \(B'\).
    \item Now \(A' = B'\) and you have created a shared key.
\end{enumerate}


\begin{center}
    \textbf{Elgamal Public Key Cryptosystem}
\end{center}

This algorithm is a little more convoluted. To set the stage, for Alice and Bob to communicate, only Alice will use a private key \(a\) to send a private message. (In the other instance, Bob would calculate some private key \(b\), and the roles would be swapped. For this instance, we will only be  doing \textit{one} message transfer.) Also notice that we have similar variables from the D-H key exchange, \(p\) and \(q\) which are both publicly known information. Alice will publish her public key so Bob can use it. She calculates it with \(A \equiv g^a \pmod{p}\)---notice the resemblance to D-H.

Now, suppose Bob wants to encrypt a message \(m\) to send to Alice. This message will be between \(2\) and \(p\) (so our message must not be greater than \(p\)). Bob randomly chooses a number \(k \pmod{p}\) to encrypt \textit{one} message, and then he discards it. Then, Bob takes his plaintext message \(m\), his random \(k\) and Alice's public key \(A\), and uses them to compute to quantities \(c_1 \equiv g^k \pmod{p}\) and \(c_2 \equiv mA^k \pmod{p}\). Bob now sends the pair of numbers \(c_1, c_2\) to Alice.

Alice decrypts this message because she knows \(a\). She solves the quantity \(x \equiv (c^a_1)^{-1} \pmod{p}\). She solves this quantity by computing \(c^{p - 1 - a}_1 \pmod{p}\) using the fast power algorithm. Finally, Alice multiplies this \(x\) by \(c_2\) to get \(m\). \textbf{Refer to \hyperlink{Elgamal}{this table} for a more precise algorithm.}

\begin{center}
    \textbf{Shanks's Babystep-Giantstep Algorithm}
\end{center}

Refer to \hyperlink{2.21}{Proposition 2.21} for the algorithm. In this section, we will be going over specific examples for the \hyperlink{Discrete Logarithm Problem}{DLP}.

\begin{example}
    {Babystep-Giantstep Exam Practice} ttt
\end{example}

\begin{center}
    \textbf{Chinese Remainder Theorem}
\end{center}

In general, to solve for CRT such that \(x \equiv a_1 \ (\text{mod } m_1)\dots, x \equiv a_k \ (\text{mod } m_k)\) we follow the algorithm below:
\begin{enumerate}
    \item Let \(m = m_1 \cdot m_2 \cdots m_k\).
    \item Take \(n_i = \frac{m}{m_i}\).
    \item  Check to see if there is a solution, \(y_i\). \(y_i = n_i^{-1} \ (\text{mod } m_i)\). Note that the inverse exists because \(m_i\) and \(n_i\) are relatively prime.
    \item Compute \(x = a_1n_1y_1 + a_2n_2y_2 + \cdots + a_kn_ky_k \ (\text{mod } m)\).
\end{enumerate}

\begin{center}
    \textbf{Pohlig-Hellman Algorithm}
\end{center}

% Suppose that \(N\) factors into a product of prime powers as \(N = q^{e_1}_1 \cdot q^{e_2}_2 \cdots  q^{e_t}_t\). Now, we follow the algorithm as shown below:
% \begin{enumerate}
%     \item For each \(1 \leq i \leq t\), let 
%     \[
%         g_i = g^{N / q^{e_i}_i}\quad \textit{ and }\quad h_i = h^{N / q^{e_i}_i}
%     \]
%     Notice that \(g_i\) has prime power order \(q^{e_i}_i\), so we use the given algorithm to solve the discrete logarithm problem \(g^y_i = h_i\). Let \(y = y_i\) be a solution for the discrete logarithm problem.
%     \item Use the Chinese remainder theorem to solve 
%     \[\begin{array}{cccc}
%         x \equiv y_1 \pmod{q^{e^1}_1}, & x \equiv y_2 \pmod{q^{e^2}_2}, & \dots, & x \equiv y_t \pmod{q^{e^t}_t}
%     \end{array}\]
% \end{enumerate}

\begin{example}
    {Pohlig-Hellman Algorithm} (Continuation of \hyperlink{exerc:2.18}{Exercise 2.18}.) Use the Pohlig-Hellman Algorithm to solve the discrete logarithm problem \(g^x = a\) in \(\F_p\) in each of the following cases:
    \begin{enumerate}[label=(\alph*)]
        \setcounter{enumi}{1}
        \item \(p = 746497\), \(g = 10\), \(a = 243278\)
        \item \(p = 41022299\), \(g = 2\), \(a = 39183497\) (Hint: \(p = 2 \cdot 29^5 + 1\))
        \item \(p = 1291799\), \(g = 17\), \(a = 192988\) (Hint: \(p - 1\) has factor of 709)
    \end{enumerate}
\end{example}

\section*{Chapter 3 Review}

\begin{enumerate}
    \item Semi-primes \((N = pq)\)
    \item RSA-Encryption and Decryption
    \item Roots of \(pq\)
    \item Primality testing
          \begin{itemize}
              \item FLT
              \item M-R
          \end{itemize}
    \item Factoring semi-primes
          \begin{itemize}
              \item Pollard's \(p - 1\) method
                    Difference of 2 squares
          \end{itemize}
\end{enumerate}

\begin{center}
    \textbf{Semi-Primes}
\end{center}

\begin{example}
    {Congruence Ex From Book}Solve the congruence
    \[
        x^{1583} \equiv 4714 \pmod{7919}
    \]
\end{example}

\lesol{
    Because \(p = 7919\) is prime, we need to solve the congruence
    \[
        1583d \equiv 1 \pmod{7918} \quad \Rightarrow \quad d \equiv 1583^{-1} \pmod{7918}
    \]

    (Notice that for \(d\), we use a modulus \(p - 1\)!) We can use the extended \hyperlink{thm:Extended Euclidean Algorithm}{Extended Euclidean Algorithm} for to solve for \(d\):
    For the extended Euclidean algorithm:
    \begin{center}
        \begin{tabular}{ccccccc}
            \(n\) & \(b\) & \(q\) & \(r\) & \(t_1\) & \(t_2\) & \(t_3\)           \\ \toprule
            7918  & 1583  & 5     & 3     & 0       & 1       & \(-5\)            \\ \midrule
            1583  & 3     & 527   & 2     & 1       & \(-5\)  & 2636              \\ \midrule
            3     & 2     & 1     & 1     & \(-5\)  & 2636    & \(\boxed{-2641}\) \\ \midrule
            2     & 1     & 2     & 0     &         &         &                   \\ \midrule
        \end{tabular}
    \end{center} A step by tep break down for how this table got its values:
    \begin{enumerate}[label=\arabic*.]
        \item \textbf{Initial Setup}

              Start with \(t_1 = 0\) and \(t_2 = 1\). Calculate \(t_3\) with the recursive formula \(t_3 = t_1 - q \cdot t_2\) where \(q\) is the quotient from the division of the 2 numbers (\(n\) divided by \(b\)), and the \(t\) values shift as you move to the next row.

        \item \textbf{Calculate \(t_3\) Values}

              \textit{First row:}
              \begin{itemize}
                  \item We start with \(n = 7919\) and \(b = 1583\).
                  \item The quotient \(q = \lfloor \frac{7919}{1583} \rfloor = 5\).
                  \item The values of \(t_1 = 0\) and \(t_2 = 1\) are given by default.
                  \item Now, calculate the \(t_3\) using the formula
                        \[
                            t_3 = t_1 - q \cdot t_2 = 0 - 5 \cdot 1 = -5
                        \]
                  \item This leaves us with \(t_2 = 1, \ t_3 = -5\). These will become \(t_1\) and \(t_2\) in the next row.
              \end{itemize}
              \textit{Second row:}
              \begin{itemize}
                  \item Now, \(n = 1583\) and \(b = 3\).
                  \item The quotient \(q = \lfloor \frac{1583}{3} \rfloor = 527\).
                  \item The values of \(t_1 = 1\) and \(t_2 = -5\) from the previous row.
                  \item Now, calculate the \(t_3\) using the formula
                        \[
                            t_3 = t_1 - q \cdot t_2 = 1 - 527 \cdot -5 = 2636
                        \]
                  \item This leaves us with \(t_2 = -5, \ t_3 = 2636\). These will become \(t_1\) and \(t_2\) in the next row.
              \end{itemize}
              \textit{Third row:}
              \begin{itemize}
                  \item Now, \(n = 3\) and \(b = 2\).
                  \item The quotient \(q = \lfloor \frac{3}{2} \rfloor = 1\).
                  \item The values of \(t_1 = -5\) and \(t_2 = 2636\) from the previous row.
                  \item Now, calculate the \(t_3\) using the formula
                        \[
                            t_3 = t_1 - q \cdot t_2 = -5 - 1 \cdot 2636 = -2641
                        \]
              \end{itemize}
              \textit{Fourth row:}
        \item Now, \(n = 2\) and \(b = 1\).
        \item The quotient \(q = \lfloor \frac{2}{1} \rfloor = 2\).
        \item Because get a remainder of 0 from the previous calculation, we stop, and record the largest \(t\) value.
    \end{enumerate}
    Now we have our \(t = -2641\), but we still need to find the modulo. Now, take the modulo of \(t\), and we get the inverse, 5277. Thus, \(d = 5277\). Then we solve for
    \[
        x^{5277} \equiv 4714 \pmod{7919} \quad \Rightarrow \quad x \equiv 4714^{5277} \pmod{7919} \quad \Rightarrow \quad x \equiv 6059.
    \]
}

\wrpprop{3.5}{Let \(p\) and \(q\) be distinct primes and let \(e \geq 1\)   satisfy
    \[
        \gcd(e, (p - 1)(q - 1)) = 1.
    \]
    Then, \refprop{1.13} tells us that \(e\) has an inverse modulo \((p-1)(q-1)\), say
    \[
        de \equiv 1 \pmod{(p-1)(q-1)}.
    \]
    Then the congruence
    \[
        x^e \equiv c \pmod{pq}
    \]
    has the unique solution \(x \equiv c^d \pmod{pq}\).
}%

\begin{example}
    {Congruence with p and q}Solve the congruence \(x^{17389} \equiv 43927 \pmod{64349}\)
\end{example}

\lesol{
    We factor the number 64349 into its prime factors so we can subtract 1 from both factors to get the value of the modulus for \(d\). We get
    \begin{align*}
        p \cdot q             & = 64349 = 229 \cdot 281 \\
        (p - 1) \cdot (q - 1) & = 63840 = 228 \cdot 280
    \end{align*}
    From here, we need to solve the following equation:
    \[
        17389d \equiv 1 \pmod{63840},
    \]
    We solve this \(d\) to be 53509. Then, \refprop{3.5} tells us that \(x \equiv 43927^{53509} \equiv 14458 \pmod{64349}\) is the solution to \(x^{17389} \equiv 43927 \pmod{64349}\)

    Now, we could have saved ourselves some work by calculating \(\gcd(p-1, q-1)\). This gives \(\gcd(228, 280) = 4\), so \((p-1)(q-1)/g = (228)(280)/4 = 15960\), which means that we can find a value \(d\) by solving the congruence
    \[
    17389d \equiv 1 \pmod{15960}.
    \]
    The solution is \(d \equiv 5629 \pmod{15960}\), and then 
\[
x \equiv 43927^{5629} \equiv 14458 \pmod{64349}.
\]
}

\begin{center}
    \textbf{RSA Encryption and Decryption}
\end{center}

The algorithm for RSA key generation, encryption, and decryption is as follows: \\

\textit{Key Generation:}
\begin{enumerate}
    \item Choose two distinct primes \(p\) and \(q\), and compute \(N = pq\).
    \item Choose encryption exponent \(e\) with \(\gcd(e,(p-1)(q-1)) = 1\). 
    \item Publish \(N = pq\) and \(e\).
\end{enumerate}

\textit{Encryption:}
\begin{enumerate}
    \item Choose plaintext \(m\). 
    \item Use the recipient's public key \((N,e)\) to compute the ciphertext \(c \equiv m^e \pmod{N}\).
    \item Send \(c\) to the recipient.
\end{enumerate}

\textit{Decryption:}
\begin{enumerate}
    \item Compute \(d\) satisfying \(ed \equiv 1 \pmod{(p-1)(q-1)}\).
    \item Compute \(m' \equiv c^d \pmod{N}\).
    \item Then \(m'\) equals the plaintext \(m\).
\end{enumerate}

\begin{example}
    {RSA} (a) \textbf{Key Creation} Use two secret primes \(p = 1223\), \(q = 1987\) to make a public modulus. Also, use the public encryption exponent \(e = 948047\) with the property that \(\gcd(e,(p - 1)(q - 1)) = 1\). (b) \textbf{Encryption} Use the plaintext message \(m = 1070777\) to compute the message to ciphertext. (c) \textbf{Decryption} Use the ciphertext to decrypt the message.
\end{example}

\lesol{
    \begin{enumerate}[label=(\alph*)]
        \item Calculate \(N = p \cdot q = 1223 \cdot 1987 = 2430101\). This is the public modulus that will be sent to person B. Now, we need to test if our \(e = 948047\) works with our \(p - 1\) and \(q - 1\): 
        \[
            \gcd(948047, 1222 \cdot 1986) = 1;
        \]
        it does, so we can move on to encryption.
        \item Now that person A has calculated the public modulus, they can encrypt their message. The message is \(m = 1070777\). Person A uses the public key \((N,e) = (2430101, 948047)\) to encrypt the message:
        \begin{align*}
            c &\equiv m^e \pmod{N} \\
            c &\equiv 1070777^{948047} \pmod{2430101} \\
            c &\equiv 1473513 \pmod{2430101}.
        \end{align*}
        Person A sends the ciphertext \(c = 1473513\) to person B.
        \item Person B receives the ciphertext and decrypts it using their private key. They calculate the decryption exponent \(d\) such that \(ed \equiv 1 \pmod{(p - 1)(q - 1)}\):
        \begin{align*}
            ed &\equiv 1 \pmod{(p - 1)(q - 1)} \\
            948047 \cdot d &\equiv 1 \pmod{2426892} \\
            d &\equiv 948047^{-1} \pmod{2426892} \\
            d &\equiv 1051235 \pmod{2426892}.
        \end{align*} Person B found that \(d = 1051235\). Now, they take the ciphertext \(c = 1473513\) and compute the plaintext message:
        \begin{align*}
            c^d &\equiv 1473513^{1051235} \pmod{2430101} \\
            m &\equiv 1070777 \pmod{2430101}
        \end{align*}
    \end{enumerate}
}

\begin{center}
    \textbf{Primality Testing}
\end{center}

\noindent\textbf{FLT} \\

This is for \hyperlink{thm:Fermat's Little Theorem, Version 2}{Fermat's Little Theorem V2}. This algorithm is pretty straightforward for evaluating if a number is prime or not. Note that it specifically checks for the compositeness of a number, it does not \textit{guarantee} that number is prime. We have the following expression to test if a number is prime:

% \addcontentsline{toc}{chapter}{Exam Prep}


%-%-%-%-%-%-%-%-%-%-%-%-%-%-%-%-%-%-%-%-%-%-%-%-%-%-%-%-%-%-%-%-%-%-%-%-%-%-%

\end{document}
%-%-%-%-%-%-%-%-%-%-%-%-%-%-%-%-%-%-%-%-%-%-%-%-%-%-%-%-%-%-%-%-%-%-%-%-%-%-%

%-%-%-%-%-%-%-%-%-%-%-%-%-%-%-%-%-%-%-%-%-%-%-%-%-%-%-%-%-%-%-%-%-%-%-%-%-%-%