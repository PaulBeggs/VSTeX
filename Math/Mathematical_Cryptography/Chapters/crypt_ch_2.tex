\renewcommand{\theenumi}{\arabic{enumi}}
\renewcommand{\labelenumi}{\theenumi.}
\section{The Birth of Public Key Cryptography}

\wrpdef{One-way Function}{A \textit{one-way function} that is easy to computer, but whose inverse is difficult.}

\wrpdef{Trap-door Function}{A \textit{trap-door function} is a one-way function with an extra piece of information that makes \(f^{-1}\) easy.}

\renewcommand{\theenumi}{\arabic{enumi}}
\renewcommand{\labelenumi}{\theenumi.}
\section{Discrete Logarithm Problem (DLP)}
\label{sec:DLP}

\wrpdef{Discrete Logarithm Problem}{Let \(g\) be a primitive root for \(\F_p\) and let \(h\) be a nonzero element of \(\F_p\). The \textit{Discrete Logarithm Problem} is the problem of finding an exponent \(x\) such that \[g^x \equiv h \ (\text{mod } p).\] The number \(x\) is called the \textit{discrete logarithm} of \(h\) to the base \(g\) and is denoted by \(\log_g(h)\).}

Remember the rules of logarithms: \begin{align*}
    \log_b(a \cdot c) & = \log_b(a) + \log_b(c) \\
    \log_b(a^c)       & = c \cdot \log_b(a)     \\
    \log_b(a / c)     & = \log_b(a) - \log_b(c)
\end{align*}

What is the value of \(x\) such that \(20^x = 21 \ (\text{mod } 23) \overset{\text{Brute-f}}{\Longrightarrow} \log_{20}21 = \boxed{7} \ (\text{mod } 23)\).~(Where \(7\) is from wolfram.)

\begin{example}
    {DLP}Find \(\log_2(10) \ (\text{mod } 11)\). In other words, find the value of \(x\) such that \(2^x \equiv 10 \ (\text{mod } 11)\).
\end{example}

\lesol{
    \begin{align*}
        2^1        & = 2 \ (\text{mod } 11)          \\
        2^2        & = 4 \ (\text{mod } 11)          \\
        2^3        & = 8 \ (\text{mod } 11)          \\
        2^4        & = 5 \ (\text{mod } 11)          \\
        2^5        & = 10 \ (\text{mod } 11)         \\
        \log_2(10) & = \boxed{5} \ (\text{mod } 11).
    \end{align*}
}

\renewcommand{\theenumi}{\alph{enumi}}
\renewcommand{\labelenumi}{(\theenumi)}
\subsection{Exercises}

\begin{exercise}
    {2.3} Let \(g\) be a \hyperref[thm:Primitive Root Theorem]{primitive root} for \(\mathbb{F}_p\).
    \begin{enumerate}
        \setcounter{enumi}{1} % Start enumeration at (b)
        \item Prove that \(\log_g(h_1 h_2) = \log_g(h_1) + \log_g(h_2)\) for all \(h_1, h_2 \in \mathbb{F}_p^*\).
        \item Prove that \(\log_g(h^n) = n \log_g(h)\) for all \(h \in \mathbb{F}_p^*\) and \(n \in \mathbb{Z}\).
    \end{enumerate}
\end{exercise}

\sol{
    \begin{enumerate}
        \setcounter{enumi}{1}
        \item \textit{Proof.} We know that \(g^x \equiv h \pmod{p}\), which means that \(x = \log_g(h)\). Similarly, if \(g^{x_1} \equiv h_1 \pmod{p}\) and \(g^{x_2} \equiv h_2 \pmod{p}\), then \(x_1 = \log_g(h_1)\) and \(x_2 = \log(h_2)\). Now, we can substitute these values into the first equation to get \(g^{x_1 + x_2} \equiv h_1h_2 \pmod{p} \equiv g^{\log_g(h_1) + \log_g(h_2)} \pmod{p}\). Then, from the properties of exponents, we can rewrite this equation as \(h_1 \cdot h_2 \pmod{p} \equiv g^{\log_g(h_1h_2)} \pmod{p}\). Therefore, \(\log_g(h_1 \cdot h_2) = \log_g(h_1) + \log_g(h_2)\).
        \item \textit{Proof} Following a similar process to \((b)\), we start with \(g^{n\log_g(h)}\). Then, by using the properties of logarithms, we can rewrite this as \(g^{\log_g(h^n)}\). Then, because \(g^{\log_g}\) cancel out, we see that \(g^{\log_g(h^n)} = h^n\). Putting everything together, we have \begin{align*}
                  g^{\log_g(h^n)} & \equiv h^n \pmod{p}         \\
                  \log_g(h^n)     & \equiv n\log_g(h) \pmod{p}.
              \end{align*}
    \end{enumerate}
}

\begin{exercise}
    {2.4} Compute the following \hyperref[sec:DLP]{discrete logarithms}.
    \begin{enumerate}
        \item \(\log_2(13)\) for the prime \(23\), i.e., \(p = 23\), \(g = 2\), and you must solve the congruence \(2^x \equiv 13 \pmod{23}\).
        \item \(\log_{10}(22)\) for the prime \(p = 47\).
        \item \(\log_{627}(608)\) for the prime \(p = 941\). (Hint: Look in the second column of Table 2.1 on page 66.)
    \end{enumerate}
\end{exercise}

\sol{
    \begin{enumerate}
        \item We use Wolfram Alpha to solve for \(x\) in the equation \(2^x \equiv 13 \pmod{23}\): \(x = 7\).
        \item Solving for \(x\) we get \(x = 11\).
        \item \(x = 18\).
    \end{enumerate}
}

\renewcommand{\theenumi}{\arabic{enumi}}
\renewcommand{\labelenumi}{\theenumi.}
\section{Diffie-Hellman Key Exchange}
\label{sec:d-h key}

D-H gives a way for Alice and Bob to get a secret shared key in an unsecure environment (i.e., when Eve is listening). Now, we will follow the steps of D-H below:

\begin{enumerate}[label=\arabic*.]
    \item Alice and Bob choose large prime \(p\) and primitive root \(g\) and make public \(k_{\text{pub}} = (p,g)\).
    \item Alice and Bob each pick their own secret integers, \(a,b\) such that \(k_{\text{priv } A} = a\) and \(k_{\text{priv } B} = b\). Compute \(g^a \ (\text{mod } p) = A\) and \(g^b \ (\text{mod } p) = B\).
    \item Exchange \(A\) and \(B\) over an insecure channel.
          \begin{enumerate}[label=\roman*.]
              \item Note that Eve would have to solve \hyperref[Discrete Logarithm Problem]{DLP} if she obtained \(A\) and \(B\) where \(a = \log_g(A)\) and \(b = \log_g(B)\).
              \item Guidelines \(\approx 2^{1000} g \approx p/2\).
          \end{enumerate}
    \item Alice computes \(B^a \ (\text{mod } p) = A'\) and Bob computes \(A^b \ (\text{mod } p) = B'\).
\end{enumerate}

\begin{example}
    {D-H}Let \(p = 23\) and \(g = 5\). Alice chooses \(a = 6\) and Bob chooses \(b = 15\). Compute the shared secret key.
\end{example}

\lesol{
    \begin{align*}
        A  & = 5^6 \ (\text{mod } 23) = 8     \\
        B  & = 5^{15} \ (\text{mod } 23) = 19 \\
        A' & = 19^6 \ (\text{mod } 23) = 2    \\
        B' & = 8^{15} \ (\text{mod } 23) = 2.
    \end{align*}
}

\wrpdef{Diffie-Hellman Problem}{Let \(p\) be a prime number and \(g\) an integer. The \textit{Diffie-Hellman Problem} is the problem of computing the value of \(g^{ab} \ (\text{mod } p)\) from the known values of \(g^a \ (\text{mod } p)\) and \(g^b \ (\text{mod } p)\).}

\renewcommand{\theenumi}{\alph{enumi}}
\renewcommand{\labelenumi}{(\theenumi)}
\subsection{Exercises}



\renewcommand{\theenumi}{\arabic{enumi}}
\renewcommand{\labelenumi}{\theenumi.}
\section{Elgamal Public Key Cryptosystem}
\label{sec:Elgamal}

\begin{table}[h!]
    \centering
    \begin{tabular}{@{}p{0.9\textwidth}l@{}}
        \toprule
        \multicolumn{2}{c}{\parbox[t]{0.9\textwidth}{\centering \textbf{Public parameter creation}}}                                                                                  \\ \midrule
        \multicolumn{2}{c}{\parbox[t]{0.9\textwidth}{\centering A trusted party chooses and publishes a large prime \(p\) and an element \(g\) modulo \(p\) of large (prime) order.}} \\ \midrule
        \multicolumn{2}{c}{\parbox[t]{0.9\textwidth}{\centering \textbf{Key creation}}}                                                                                               \\ \midrule
        \multicolumn{1}{c}{\parbox[t]{0.45\textwidth}{\centering \textbf{Alice}}} & \multicolumn{1}{c}{\parbox[t]{0.45\textwidth}{\centering \textbf{Bob}}}                           \\ \midrule
        \multicolumn{1}{l}{\parbox[t]{0.45\textwidth}{Choose private key \(1 \leq a \leq p -1.\)                                                                                      \\ Compute \(A = g^a \ (\text{mod } p)\). \\ Publish the public key \(A\).}} &  \\ \midrule
        \multicolumn{2}{c}{\parbox[t]{0.9\textwidth}\centering\textbf{Encryption}}                                                                                                    \\ \midrule
        \multicolumn{1}{l}{\parbox[t]{0.45\textwidth}}                            & \multicolumn{1}{r}{\parbox[t]{0.45\textwidth}{\raggedright Choose plaintext \(m\).                \\ Choose random element \(k\). \\ Use Alice's public key \(A\) \\ \quad to compute \(c_1 = g^k \ (\text{mod } p)\) \\ \quad and \(c_2 = mA^k \ (\text{mod } p). \) \\ Send ciphertext \(c_1,c_2\) to Alice.}}                                            \\ \midrule
        \multicolumn{2}{c}{\parbox[t]{0.9\textwidth}{\centering\textbf{Decryption}}}                                                                                                  \\ \midrule
        \multicolumn{1}{l}{\parbox[t]{0.45\textwidth}{Compute \({(c_1^a)}^{-1} \cdot c_2 \ (\text{mod } p)\).                                                                         \\  This quantity is equal to \(m\).}}                     &                                                                                             \\ \bottomrule
    \end{tabular}
    \caption{Elgamal Key Creation, Encryption, and Decryption}
\end{table}

\begin{example}
    {Elgamal}Let \(p = 29\) and \(g = 2\). Alice chooses \(a = 12\) and Bob chooses \(k = 5\) and wants to send secret message \(m = 26\). Compute the shared secret key.
\end{example}

\lesol{First, we need to calculate Alice's \(A\) and Bob's \(B\). Then, we can calculate the ciphertexts \(c_1\) and \(c_2\):
\begin{align*}
    A   & = g^{a} \ (\text{mod } p)  & = & \ 2^{12} \ (\text{mod } 29) = 7     \\
    B   & = g^k \ (\text{mod } p)    & = & \ 2^5 \ (\text{mod } 29) = 3        \\
    c_1 & = g^k \ (\text{mod } p)    & = & \ 2^5 \ (\text{mod } 29) = 3        \\
    c_2 & = m(A^k) \ (\text{mod } p) & = & \ 26(7^{5} \ (\text{mod } 29)) = 10 \\
\end{align*}
Now, for Alice to decrypt the message, she must compute \({(c_1^a)}^{-1} \cdot c_2 \ (\text{mod } p)\):
\[
    {(c_1^a)}^{-1} \cdot c_2 \ (\text{mod } p) = {(3^{12})}^{-1} \cdot 10 \ (\text{mod } 29)
\]
The order of operations to compute this is as follows:
\begin{enumerate}[label=\arabic*.]
    \item \textbf{Compute} \(3^{12} \ (\text{mod } 29) = 16\);
    \item \textbf{Compute} \(16^{-1} \ (\text{mod } 29) = 20\);
    \item \textbf{Finish by multiplying} \(20 \cdot 10 \ (\text{mod } 29) = 26\).
\end{enumerate}
}

\begin{note}
    Be aware: You should only use this encryption scheme once. If you use it more than once, it is possible for an attacker to decrypt the message. For example, Eve knows \(m_1(c_1, c_2) \rightarrow\) Eve finds \(A\) by keeping record of the first message, then by solving for \(d_2\) such that \(c_1, d_2\) (where \(c_1\) is the \textit{same} as the first message) and \(d_2\) is the second message. Then, Eve can solve for \(m_2\) by computing \({(c_1^d)}^{-1} \cdot d_2 \ (\text{mod } p)\).
\end{note}



\renewcommand{\theenumi}{\alph{enumi}}
\renewcommand{\labelenumi}{(\theenumi)}
\subsection{Exercises}

\begin{exercise}
    {2.8} Alice and Bob agree to use the prime \(p = 1373\) and the base \(g = 2\) for communications using the \hyperref[sec:Elgamal]{Elgamal public key cryptosystem}.
    \begin{enumerate}
        \item Alice chooses \(a = 947\) as her private key. What is the value of her public key \(A\)?
        \item Bob chooses \(b = 716\) as his private key, so his public key is
              \[
                  B \equiv 2^{716} \equiv 469 \pmod{1373}.
              \]
              Alice encrypts the message \(m = 583\) using the random element \(k = 877\). What is the ciphertext \((c_1, c_2)\) that Alice sends to Bob?
        \item Alice decides to choose a new private key \(a = 299\) with associated public key
              \[
                  A \equiv 2^{299} \equiv 34 \pmod{1373}.
              \]
              Bob encrypts a message using Alice's public key and sends her the ciphertext \((c_1, c_2) = (661, 1325)\). Decrypt the message.
        \item Now Bob chooses a new private key and publishes the associated public key \(B = 893\). Alice encrypts a message using this public key and sends the ciphertext \((c_1, c_2) = (693, 793)\) to Bob. Eve intercepts the transmission. Help Eve by solving the discrete logarithm problem \(2^b \equiv 893 \pmod{1373}\) and using the value of \(b\) to decrypt the message.
    \end{enumerate}
\end{exercise}

\sol{
    \begin{enumerate}
        \item \(p = 1373, \ g = 2, \ a = 947 \Rightarrow A \equiv 2^{947} \pmod{1373} \equiv 177\).
        \item \(c_1 \equiv 2^{877} \pmod{1373} \equiv 719, \ c_2 \equiv 583 \cdot 469^{877} \pmod{1373} \equiv 623\). Alice sends ``\((719, 623)\)'' to Bob.
        \item To decrypt, we can use the \hyperref[thm:Extended Euclidean Algorithm]{EEA} to find the inverse of \(661^{299} \pmod{1373} \equiv 645^{-1} \equiv 794\). From here, we solve for the message: \(1325 \cdot 794 \pmod{1373} \equiv 332\).
        \item Solving for \(b\) in \(2^b \equiv 893 \pmod{1373}\) gives \(b = 219\). Now we can decrypt:
              \[
                  (c_1^a)^{-1} \cdot c_2 \equiv (693^{219})^{-1} \equiv 431^{-1} \cdot 793 \equiv 532 \cdot 793 \equiv 365 \pmod{1373}.
              \] Alice's private message to Bob is \(m = 365\).
    \end{enumerate}
}



\renewcommand{\theenumi}{\arabic{enumi}}
\renewcommand{\labelenumi}{\theenumi.}
\section{An Overview of the Theory of Groups}
\label{sec:Theory groups}

\wrpdef{Group}{A set \(G\) along with a binary operation (closure) such that for all \(a,b \in G\), \(a\times b \in G\) (closure), and there exists an \(e \in G\) such that \(a \times e = a\) and \(e \times a = a\) (identity), for all \(a \in G\), there exists \(a^{-1} \in G\) such that \(a \times a^{-1} = a^{-1} \times a = e\) (inverse), and for all \(a,b,c \in G\), \((a \times b) \times c = a \times (b \times c)\) (associativity)}

For commutativity, for all \(a,b \in G\), \(a\times b = b \times a\). Some groups have this, some do not. \\

\begin{example}
    {Integer Addition as a Group}Lets check to see addition among the integers are a group: \((\Z, +)\)
\end{example}

\lesol{
    \begin{enumerate}
        \item True. Let \(a,b \in \Z\) \(a + b \in \Z\).
        \item True. \(e = 0 \in \Z\), \(a + 0 = a\) and \(0 + a = a\)
        \item True. For all \(a\in \Z\), \(a^{-1} = -a\) because \(a + (-a) = 0 = -a + a\)
        \item True. For all \(a,b,c \in \Z\), \((a + b) + c = a + (b + c)\)
    \end{enumerate}
    Therefore, the additive property of the integers are a group. In fact, because \(a + b = b + a\) \(\Z\) are a commutative group (abelian group).
}

\begin{example}
    {Integer Multiplication}Lets check to see multiplication among the integers are a group: \((\Z, +)\)
\end{example}
\lesol{
    \begin{enumerate}
        \item True. Let \(a,b \in \Z\) \(ab \in \Z\).
        \item True. \(e = 1 \in \Z\), \(a * 1 = a\) and \(1 * a = a\)
        \item False. Counterexample: consider \(2^{-1} = \frac{1}{2}\) because \(2(\frac{1}{2}) = 1\) but \(\frac{1}{2} \notin \Z\)
    \end{enumerate}
}

\wrpdef{Order}{The \textit{order} of an element \(a \ (\text{mod } p)\) is the smallest exponent \(k \geq 1\) such that \(a^k \equiv 1 \ (\text{mod } p)\).}

\renewcommand{\theenumi}{\alph{enumi}}
\renewcommand{\labelenumi}{(\theenumi)}
\subsection{Exercises}

\begin{exercise}
    {(Additional)}Decide whether each of the following is a group:
    \begin{enumerate}
        \item \texttt{All 2\(\times\)2 matrices with real number entries} with operation matrix addition
        \item \texttt{All 2 \(\times\) 2 matrices with real number entries} with operation matrix multiplication
    \end{enumerate}

\end{exercise}

\sol{
    \begin{enumerate}
        \item \textbf{Matrix Addition:} \cmark
              \begin{enumerate}[label=(\arabic*)]
                  \item \textbf{Closure:} For addition to work between matrices, they must be of dimension \(2 \times 2 + 2 \times 2\). Therefore, the dimensions do not change, and it is closed.
                  \item \textbf{Associativity:} \(2 \times 2\) matrix addition is associative, as it inherits this property from the properties of matrices.
                  \item \textbf{Identity Element:} We can add a matrix \(Z\) that consists of only 0s to a matrix \(A\). And matrix \(A\) will remain unchanged.
                  \item \textbf{Inverse Element:} True. Consider the matrices,
                        \(\begin{bmatrix}
                            a & b \\
                            c & d
                        \end{bmatrix}\). And
                        \(\begin{bmatrix}
                            -a & -b \\
                            -c & -d
                        \end{bmatrix}\). When we add these two together we get
                        \(\begin{bmatrix}
                            0 & 0 \\
                            0 & 0
                        \end{bmatrix}\). This shows that \(2\times 2\) matrices have additive inverses.



              \end{enumerate}
        \item \textbf{Matrix Multiplication:} \xmark
              \begin{enumerate}[label=(\arabic*)]
                  \item \textbf{Closure:} The dimensions will stay the same during multiplication because it is an \(n \times n\) matrix.
                  \item \textbf{Associativity:} \(2 \times 2\) matrix multiplication is associative, as it inherits this property from the properties of matrices.
                  \item \textbf{Identity Element:} True. Consider the identity matrix, \(I_2 =
                        \begin{bmatrix}
                            1 & 0 \\
                            0 & 1
                        \end{bmatrix}\). When we multiply a matrix \(\begin{bmatrix}
                            a & b \\
                            c & d
                        \end{bmatrix}\) by \(I\). We get \[\begin{bmatrix}
                                a & b \\
                                c & d
                            \end{bmatrix}\] \(\cdot
                        \begin{bmatrix}
                            1 & 0 \\
                            0 & 1
                        \end{bmatrix} = \begin{bmatrix}
                            a & b \\
                            c & d
                        \end{bmatrix}\)
                  \item \textbf{Inverse Element:} False. Matrices with a non-zero determinant fail this criteria. Consider the matrix \(B =\)
                        \(\begin{bmatrix}
                            1 & 2 \\
                            3 & 4
                        \end{bmatrix}\). The determinant would be \(\det((1)(4) - (2)(3) = -2\). Therefore, this matrix would not have an inverse.
              \end{enumerate}
    \end{enumerate}
}

\begin{exercise}
    {(Additional)}Is \texttt{All 2x2 matrices with real number entries} a ring with operations matrix addition and matrix multiplication? Justify your answer.
\end{exercise}

\sol{ \cmark
    \begin{enumerate}[label=(\arabic*)]
        \item \textbf{Additive Closure:} True.\ (See the previous exercise (a), (1)).
        \item \textbf{Additive Associativity:} True. Inherited from the properties of matrices.
        \item \textbf{Additive Identity:} True.\ (See the previous exercise (a), (3)).
        \item \textbf{Additive Inverse:} True. You can take the difference between a matrix and its inverted duplicate (e.g., \(-[A]\)) and get \(0\).
        \item \textbf{Multiplicative Closure:} True.\ (See the previous exercise (b), (4)).
        \item \textbf{Distributive Property:} True. While this will pose an error on a calculator, you can do the equivalent: \([A]([B] + [C]) = [A][B] + [A][C]\). This is true because you are still taking the summation of each \(a_{ij}\).\ \(b_{ij}\) and \(c_{ij}\).
    \end{enumerate}
}

\setcounter{section}{6}
\renewcommand{\theenumi}{\arabic{enumi}}
\renewcommand{\labelenumi}{\theenumi.}
\section{A Collision Algorithm for the DLP}

\begin{note}
    Recall that the DLP is the problem of finding \(x\) such that \(g^x \equiv h \ (\text{mod } p)\) for a given value of \(h\).
\end{note}

Remember that to brute force the DLP, it takes \(P - 1\) steps. Recall \(g^{p-1} \ (\text{mod } p) \equiv 1\). In computational complexity, \textbf{we say that the DLP is \(\mathcal{O}(P)\).}

\wrpprop{2.21}{(Shanks Baby-Step Giant-Step Algorithm) Computational time of \(\mathcal{O}(\sqrt{P})\). Below is the algorithm: \begin{enumerate}
        \item Let \(m = \lceil \sqrt{P} \rceil\).
        \item Create two lists:
              \begin{enumerate}
                  \item Baby steps: \(\{g^0, g^1, g^2, \ldots, g^{m}\}\).
                  \item Giant steps: \(\{h, h \cdot g^{-m}, h \cdot g^{-2m}, \ldots, h \cdot g^{-m^2}\}\).
              \end{enumerate}
        \item Find a match between the two lists: \(g^i \equiv hg^{-im} \ (\text{mod } p)\)
        \item \(x = i + jm\) is a solution for \(g^x \equiv h \ (\text{mod } p)\) (another way of saying ``\(x = i + jm\) is a solution for the \hyperref[Discrete Logarithm Problem]{DLP}'').
    \end{enumerate}}

\begin{example}
    {Baby-step, Giant-step}Use the Baby-step, Giant-step algorithm to solve for \(13^x \equiv 5 \ (\text{mod } 47)\).
\end{example}

\lesol{
    \begin{enumerate}
        \item Let \(m = \lceil \sqrt{47} \rceil = 7\).
        \item Create the two lists:
              \begin{enumerate}
                  \item Baby steps: \(\{13^0, 13^1, 13^2, \ldots, 13^6\}\ (\text{mod } 47) \equiv \{1,13,28,35,32,40,3,39\}\).
                  \item Giant steps: \(\{5, 5 \cdot 13^{-7}, 5 \cdot 13^{-14}, \ldots, 5 \cdot 13^{-49}\} \ (\text{mod } 47) \equiv \{5, 17, 39, 1, 48, 36, 19, 27\}\).
              \end{enumerate}
        \item Find a match between the two lists: \(39\) and \(39\), or \(1\) and \(1\).
        \item Substitute the following variables: \(i = 7, \ j = 2,\  n = 7\) for the equation \(x = i + jm \Rightarrow x = 7 + 2(7) = \boxed{21}\). So, \(13^{21} \equiv 5 \ (\text{mod } 47)\).
    \end{enumerate}
}

\begin{example}
    {(From Book) Baby-step, Giant-step} Solve the discrete logarithm problem with these values: \(g = 9704, h = 13896, p = 17389\).
\end{example}

\lesol{
The number \(9704\) has order 1242 in \(\F_{17389}^*\). Set \(n = \lceil \sqrt{1242} \rceil = 36\) and \(u = g^{-n} = 9704{-36} = 2494\). Table 2.4 in the book lists the values of \(g^k\) and \(h\cdot u^{k}\) for \(k = 1,2 \dots\). From the table, we find the collision \[9704^7 = 14567 = 13896 \cdot 2494^{32} \text{ in } \F_{17389}.\] Using the fact that \(2494 = 9704^{-36}\), we compute \[13896 = 9704^7 \cdot 2494^{-32} = 9704^7 \cdot {(9704^{36}})^{32} = 9704^{1159} \text{ in } \F_{17389}.\] Hence, \(x = 1159\) solves the problem \(9704^x = 13896\) in \(\F_{17389}\).
    }

\renewcommand{\theenumi}{\alph{enumi}}
\renewcommand{\labelenumi}{(\theenumi)}
\subsection{Exercises}

\begin{exercise}
    {2.17} Use Shanks's babystep-giantstep method (\refprop{2.21}) to solve the following discrete logarithm problems.
    \begin{enumerate}
        \item \(11^x = 21\) in \(\mathbb{F}_{71}\).
    \end{enumerate}
\end{exercise}

\sol{
    \begin{enumerate}
        \item \begin{enumerate}[label=\arabic*.]
                  \item Let \(m = \lceil \sqrt{70} \rceil = 9\)
                  \item Create two lists: \begin{itemize}
                            \item \textbf{Baby steps:} \[\{11^0, 11^1,\dots,11^9\} \pmod{71} \equiv \{1, \boxed{11}, 50, 53, 15, 23, 40, 14, 12\}\]
                            \item \textbf{Giant steps:} \[\{21, 21 \cdot 11^{-9}, 21 \cdot 11^{-18}, \dots\} \pmod{71} \equiv \{21, 5, 35, 32, \boxed{11}\dots\}\]
                        \end{itemize}
                  \item Find a match between the two lists: \(\boxed{11}\)
                  \item Substitute values for \(i + jm = 1 + 4(9) = 37\). So, \(11^{37} \equiv 21 \pmod{71}\)
              \end{enumerate}
    \end{enumerate}
}

\renewcommand{\theenumi}{\arabic{enumi}}
\renewcommand{\labelenumi}{\theenumi.}
\section{Chinese Remainder Theorem (CRT)}

\begin{theorem}
    {Chinese Remainder} Let \(n_1, n_2, \ldots, n_k\) be pairwise relatively prime integers. This means that \(\gcd(m_i,m_j) = 1 \textit{ for all } i \ne j\). Then, for any integers \(a_1, a_2, \ldots, a_k\), the system of congruences \begin{align*}
        x & \equiv a_1 \ (\text{mod } n_1) \\
        x & \equiv a_2 \ (\text{mod } n_2) \\
          & \vdots                         \\
        x & \equiv a_k \ (\text{mod } n_k)
    \end{align*} has a unique solution \(c \ (\text{mod } n_1n_2 \ldots n_k)\).
\end{theorem}

\begin{example}
    {CRT}Solve the following system of congruences: \begin{align*}
        x & \equiv 6 \ (\text{mod } 7) \\
        x & \equiv 4 \ (\text{mod } 8)
    \end{align*}
\end{example}

\lesol{
    Note that \(x \equiv 6 \ (\text{mod } 7)\) means \begin{align*}
        x      & = 7n + 6                                \\
        7n + 6 & \equiv 4 \ (\text{mod } 8)              \\
        7n     & \equiv 6 \ (\text{mod } 8)              \\
        n      & \equiv 6 \cdot 7^{-1} \ (\text{mod } 8) \\
               & \equiv 6 \cdot 7\ (\text{mod } 8)       \\
               & \equiv 2 \ (\text{mod } 8),
    \end{align*} where \(2\ (\text{mod } 8) = 8m + 2 = n\). We plug this back into the following: \begin{align*}
        x & = 7(8m + 2) + 6 \ (\text{mod } 7 \cdot 8) \\
          & = 56m + 14 + 6 \ (\text{mod } 56)         \\
          & = 56m + 20 \ (\text{mod } 56).
    \end{align*} Note that \(56m\) is a multiple of \(56\) and so it will always be equal to 0. Thus, \(x = 20 \ (\text{mod } 56)\).
}

\begin{note}
    In general, to solve for CRT such that \(x \equiv a_1 \ (\text{mod } m_1)\dots, x \equiv a_k \ (\text{mod } m_k)\) we follow the algorithm below:
    \begin{enumerate}
        \item Let \(m = m_1 \cdot m_2 \cdots m_k\).
        \item Take \(n_i = \frac{m}{m_i}\).
        \item  Check to see if there is a solution, \(y_i\). \(y_i = n_i^{-1} \ (\text{mod } m_i)\). Note that the inverse exists because \(m_i\) and \(n_i\) are relatively prime.
        \item Compute \(x = a_1n_1y_1 + a_2n_2y_2 + \cdots + a_kn_ky_k \ (\text{mod } m)\).
    \end{enumerate}
\end{note}

\begin{example}
    {CRT with New Algorithm}
    Solve the following system of congruences: \(x \equiv a_1 \ (\text{mod } m_1)\) and \(x \equiv a_2 \ (\text{mod } m_2)\) where \(a_1 = 6, m_1 = 7, a_2 = 4, m_2 = 8\).
\end{example}

\lesol{
    \begin{enumerate}
        \item Let \(m = 7 \cdot 8 = 56\).
        \item Compute \(n_1 = 8\) and \(n_2 = 7\).
        \item Compute \(y_1 = 8^{-1} \ (\text{mod } 7) = 1\) and \(y_2 = 7^{-1} \ (\text{mod } 8) = 7\).
        \item Compute \begin{align*}
                  x & = 6 \cdot 8 \cdot 1 + 4 \cdot 7 \cdot 7 \ (\text{mod } 56) \\
                    & = 48 + 196 \ (\text{mod } 56)                              \\
                    & = 244 \ (\text{mod } 56)                                   \\
                    & = \boxed{20}.
              \end{align*}
    \end{enumerate}
}

\renewcommand{\theenumi}{\alph{enumi}}
\renewcommand{\labelenumi}{(\theenumi)}
\subsection{Exercises}



\begin{exercise}
    {2.18} Solve each of the following simultaneous systems of congruences (or explain why no solution exists).
    \begin{enumerate}
        \setcounter{enumi}{1} % Start enumeration at (b)
        \item \(x \equiv 137 \pmod{423}\) and \(x \equiv 87 \pmod{191}\).
              \setcounter{enumi}{3} % Start enumeration at (d)
        \item \(x \equiv 5 \pmod{9}\), \(x \equiv 6 \pmod{10}\), and \(x \equiv 7 \pmod{11}\).
    \end{enumerate}
\end{exercise}

%%
% Variables
% m_1 = 423
% m_2 = 191
% m = 80793
% n_1 = 191
% n_2 = 423
% a_1 = 137
% a_2 = 87
%%


\sol{
    \begin{enumerate}
        \setcounter{enumi}{1}
        \item \begin{enumerate}[label=\arabic*.]
                  \item Let \(m = 423 \cdot 191 = 80793\).
                  \item Compute \(n_1 = \frac{m}{m_1} = \frac{80793}{423} = 191\), and \(n_2 = \frac{80793}{191} = 423\)
                  \item Compute \(y_1 = 191^{-1} \pmod{423} \equiv 392\) and \(y_2 = 423^{-1} \pmod{191} \equiv 14\).
                  \item Compute \(x = (137)(191)(392) + (87)(423)(14) \pmod{80793} \equiv 27209\)
              \end{enumerate}
              \setcounter{enumi}{3} % Start enumeration at (d)
        \item \begin{enumerate}[label=\arabic*.]
                  \item Let \(m = 9 \cdot 10 \cdot 11 = 990\).
                  \item Compute \(n_1 = \frac{990}{9} = 110, \ n_2 = \frac{990}{10} = 99, \text{ and } n_3 = \frac{990}{11} = 90\).
                  \item Compute \(y_1 = 110^{-1} \pmod{9} \equiv 5, \ y_2 = 99^{-1} \pmod{10} \equiv 9\), and \(y_3 = 90^{-1} \pmod{11} \equiv 6\)
                  \item Compute \(x = (5)(110)(5) + (6)(99)(9) + (7)(90)(6) \pmod{990} = 986\)
              \end{enumerate}

    \end{enumerate}
}

\begin{exercise}
    {2.20}Let \( a, b, m, n \) be integers with \( \gcd(m, n) = 1 \). Let
    \[
        c \equiv (b - a) \cdot m^{-1} \pmod{n}.
    \]
    Prove that \( x = a + cm \) is a solution to
    \[
        x \equiv a \pmod{m} \quad \text{and} \quad x \equiv b \pmod{n}, \tag{2.24}
    \]
    and that every solution to (2.24) has the form \( x = a + cm + ymn \) for some \( y \in \mathbb{Z} \).
\end{exercise}

\expf{
Let \(a,b,m,n \in \Z\) with \(\gcd(m,n) = 1\). Let \(c \equiv (b - a)m^{-1} \pmod{n}\). Review the following: \begin{align*}
    x      & \equiv a \pmod{m}       \\
    a + cm & \equiv a \pmod{m}       \\
    a      & \equiv a \pmod{m} - cm.
\end{align*} Then, because \(cm\) is a multiple of \(m\), when we take the mod of \(a - cm \pmod{m}\), we will always get \(a\). Hence, \(a \equiv a \pmod{m}\). For the other equation, we will be using the def of \(c\) in our proof: \begin{align*}
    a + cm & \equiv b \pmod{n}             \\
    cm     & \equiv b - a \pmod{n}         \\
    c      & \equiv (b - a)m^{-1}\pmod{n}.
\end{align*} So, now when we multiply by \(m\) on both sides, we get \(cm \equiv b - a \pmod{n}\). Rearranging, we see \(cm + a \equiv b \pmod{n}\), so \(x\) is a solution.

For the second half of the proof, suppose we have \(x'\) as the solution for \(x' \equiv a \pmod{m}\) and \(x' \equiv b \pmod{n}\). We want to show that \(x' = x\). Thus, we subtract \(x\) from \(x'\) and get the following: \[
    x' - x \equiv a - a = 0 \pmod{m}, \text{ which implies } x' - x = km \text{ for some } k \in \Z.
\] This is the outcome because we know that for anything to be equal to 0 in modulus, the number itself must be a multiple of the modulus, \(m\). We can follow the same logic for \(b\), and see that \(x' - x \equiv b - b = lm\) for some \(l \in \Z\). Since \(\gcd(m,n) = 1\), \(x' - x\) must be a multiple of \(m\) and \(n\), meaning \(x' - x = ymn\) for some \(y \in \Z\). Therefore, \(x' = x + ymn, = a + cm + ymn\).
}

\renewcommand{\theenumi}{\arabic{enumi}}
\renewcommand{\labelenumi}{\theenumi.}
\section{Pohlig-Hellman Algorithm}
\label{sec:Pohlig-Hellman}

This algorithm is used to solve \(g^x \equiv h \ (\text{mod } p)\) for \(p\) prime and \(g\) primitive root. This has computational time of \(\mathcal{O}(\sqrt{p-1})\). Order of \(g\) is \(p-1\) is composite. This is most efficient when \(p - 1\) has small prime factors. Look below for the algorithm:

\begin{enumerate}
    \item Factor \(p - 1 = n_1^{e_1} \cdot n_2^{e_2} \cdots n_k^{e_k}\). (Note that \(\gcd(q_j, q_i) = 1 \  \forall i \ne j\).)
    \item For each \(1 \leq i \leq k\), let \(m_i = \frac{p-1}{n_i^{e_i}}\).
    \item Solve \(g^{x_i} \equiv h^{m_i} \ (\text{mod } p)\) for \(x_i\). (Note this DLP is easier because order of \(g_i\) is way less than the order of \(g\).)
    \item Use CRT to find \(x\) such that \(x \equiv x_1 \ (\text{mod } q_1^{e+1}), \dots, x_i \ (\text{mod } n_i^{e_i})\).
\end{enumerate}

\begin{example}
    {Pohlig-Hellman Algorithm} Solve the following DLP: \(106^x \equiv 12375 \ (\text{mod } 24691)\).
\end{example}

\lesol{
    To use the Pohlig-Hellman algorithm, we need to find the order of the prime number \(p = 24691\). Since it is prime, we know the order to be \(\varphi(24691) = 24690\). We can factor this number to \(30 \cdot 823\). Let \(x = a_0 + 30a_1\):
    \begin{align}
        (106^{a_0 + 30a_1})^{823}                   & \equiv 12375^{823} \pmod{24691} \\
        (106^{823a_0 + 24690a_1})                   & \equiv 24143 \pmod{24691}       \\
        (106^{823a_0} \cdot 106^{24690a_1})         & \equiv 24143 \pmod{24691}       \\
        (106^{823})^{a_0} \cdot (106^{24690})^{a_1} & \equiv 24143 \pmod{24691}       \\
        (1410)^{a_0}                                & \equiv 24143 \pmod{24691}
    \end{align}
    For (1), we got the expression by substituting \(x\) for \(a_0 + 30a_1\) for the exponent in \(g^x \equiv \dots\). From there, (2) --- (4) is simple algebra. Then for (5), because \(106^{a_1}\) is congruent to 1, \((106^{a_1})^{24690} = 1\). Additionally, we take \(106^{823} \pmod{24691}\) to get \(1410^{a_0}\). Then, we do the same thing for the other side of the equation. Now, we can brute force this by setting \(a_0 = \{0,1,2,3,\dots, 823\}\). We find that when \(a_0 = 12\), \(1410^{a_0} \equiv 24143 \pmod{24691}\). Seeing that we have \(a_0\), we need to find \(b_0\):
    \begin{align*}
        (106^{b_0 + 823b_1})^{30} & \equiv 12375^{30} \pmod{24691} \\
        15097^{b_0}               & \equiv 7229 \pmod{24691}
    \end{align*}
    Thus, through brute force, we find that \(b_0 = 171\). We can take our \(a_0\) and \(b_0\) to solve for \(x_1,x_2\):
    \begin{align*}
        x_1 & =  a_0 + 30a_1 \\
        x_1 & = 12 \pmod{30}
    \end{align*} and \begin{align*}
        x_2 & = b_0 + 823b_1   \\
        x_2 & = 171 \pmod{823}
    \end{align*}
    At this point, we can solve the Chinese Remainder Theorem.
    \begin{enumerate}[label=\arabic*.]
        \item Let \(m = 30 \cdot 823 = 24690\).
        \item Compute \(n_1 = \frac{24690}{30} = 823\) and \(n_2 = \frac{24690}{823} = 30\).
        \item Compute \(y_1 = 823^{-1} \pmod{30} \equiv 7\) and \(y_2 = 30^{-1} \pmod{823} \equiv 631\).
        \item Compute \(x = (12)(823)(7) + (171)(30)(631) \pmod{24690} \equiv 22392\).
    \end{enumerate} Therefore, \(a = 22392\)
}

\renewcommand{\theenumi}{\alph{enumi}}
\renewcommand{\labelenumi}{(\theenumi)}
\subsection{Exercise}

\begin{exercise}
    {2.28} Use the \hyperref[sec:Pohlig-Hellman]{Pohlig-Hellman algorithm} to solve the discrete logarithm problem \(g^x = a\) in \(\mathbb{F}_p\) in each of the following cases.
    \begin{enumerate}
        \item \(p = 433\), \(g = 7\), \(a = 166\).
    \end{enumerate}
\end{exercise}

\sol{
    We start by writing the given information into an equation that we can work with. Thus, \(7^x \equiv 166 \pmod{433}\). Since \(433\) is prime, \(\varphi(433) = 432\), which we can factor to \(16 \cdot 27\). Let \(x = a_0 + 16a_1\):
    \begin{align}
        (7^{a_0 + 16a_1})^{27}        & \equiv 166^{27} \pmod{433} \\
        (7^{27a_0 + 432a_1})          & \equiv                     \\
        (7^{27a_0} \cdot 7^{432a_1})  & \equiv                     \\
        (7^{27})^{a_0}(7^{a_1})^{432} & \equiv                     \\
        (265)^{a_0}                   & \equiv 250 \pmod{433}
    \end{align}
    For (3.1), we got the expression by substituting \(x\) for \(a_0 + 16a_1\) for the exponent in \(g^x \equiv \dots\). From there, (3.2) --- (3.4) is simple algebra. Then for (3.5), because \(7^{a_1}\) is congruent to 1, \((7^{a_1})^{432} = 1\). Additionally, we take \(7^{27} \pmod{432}\) to get \(265^{a_0}\). Then, we do the same thing for the other side of the equation. Now, we can brute force this by setting \(a_0 = \{0,1,2,3,\dots, 27\}\). We find that when \(a_0 = 15\), \(265^{a_0} \equiv 250 \pmod{433}\). Seeing that we have \(a_0\), we need to find \(b_0\):
    \begin{align*}
        (7^{b_0 + 27b_1})^{16} & \equiv 166^{16} \pmod{433} \\
        374^{b_0}              & \equiv 335 \pmod{433}
    \end{align*}
    Thus, through brute force, we find that \(b_0 = 20\). We can take our \(a_0\) and \(b_0\) to solve for \(x_1,x_2\):
    \begin{align*}
        x_1 & =  a_0 + 16a_1 \\
        x_1 & = 15 \pmod{16}
    \end{align*} and \begin{align*}
        x_2 & = b_0 + 27b_1  \\
        x_2 & = 20 \pmod{27}
    \end{align*}
    At this point, we can solve the \hyperref[thm:Chinese Remainder]{CRT}.
    \begin{enumerate}[label=\arabic*.]
        \item Let \(m = 16 \cdot 27 = 432\).
        \item Compute \(n_1 = \frac{432}{16} = 27\) and \(n_2 = \frac{432}{27} = 16\).
        \item Compute \(y_1 = 27^{-1} \pmod{16} \equiv 3\) and \(y_2 = 16^{-1} \pmod{27} \equiv 22\).
        \item Compute \(x = (27)(15)(3) + (20)(16)(22) \pmod{432} = 47\)
    \end{enumerate} I relied on \href{https://www.youtube.com/watch?v=CHfP5tBbiAg}{this video} heavily.
}