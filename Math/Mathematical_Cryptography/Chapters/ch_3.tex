\section{Euler's Totient Theorem}

Remember \hyperlink{thm:Fermat's Little Theorem}{FLT} and \hyperlink{Discrete Logarithm Problem}{DLP}? Can we use FLT for non-prime moduli? No, FLT is only true for prime moduli: \(2^5 \ (\text{mod 6}) = 32 \). Thus, we need to look to other ways of solving this problem for non-primes.

\begin{theorem}
    {Euler's Totient Theorem for \text{phi}}Let \(n\) be a positive integer and let \(a\) be an integer such that \(\gcd(a, n) = 1\). Then, \[
        a^{\varphi(n)} \equiv 1 \ (\text{mod }{n}.
    \] where \(\varphi(N) = \#\{0 < x < N \mid \gcd(x,N) = 1\}\)
\end{theorem}

\begin{corollary}
    {1: Euler's Totient Theorem for \textit{phi}}If \(N\) is prime, \(a^{\varphi(N)} \equiv 1 \ (\text{mod } N)\). For \(\varphi(N)\), we can express it as \(N-1\) from \hyperlink{thm:Fermat's Little Theorem}{FLT}.
\end{corollary}

\cpf{
    If \(N\) is prime, then \(\gcd(a,N) = 1\). Thus, for all \(a\), such that \(0 < a < N\), \(\gcd(a,N) = 1\). Thus, \(\varphi(N) = N - 1\).
}

\begin{corollary}
    {2: Euler's Totient Theorem for \textit{phi}}If \(p,q\) are distinct primes, then \[
        a^{(p-1)(q-1)} \equiv 1 \ (\text{mod } pq) \quad \text{for all } a  \text{ such that } \gcd(a, pq) = 1.
    \] From this, we can state that \(N = pq\).
\end{corollary}

\cpf{
    For this proof, we would show that \(\varphi(pq) = \varphi(q) \cdot \varphi(p) = (p - 1)(q - 1)\).
}

\begin{example}
    {ETT for \textit{phi}}Find \(a^x \ (\text{mod } 15)\).
\end{example}

\lesol{
    \(15 = 3 \cdot 5\) (relatively prime); \(\varphi(15) = 2 \cdot 4 = 8\). Thus, \(a^8 \equiv 1 \ (\text{mod } 15)\).
}


\begin{theorem}
    {Euler's Totient Theorem for \textit{pq}}Let \(p\) and \(q\) be distinct primes and let \[
        g = \gcd(p-1, q-1).
    \]Then, \[
        a^{(p-1)(q-1)/g} \equiv 1 \ (\text{mod } pq) \quad \text{for all } a  \text{ such that } \gcd(a, pq) = 1.
    \] In particular, if \(p\) and \(q\) are distinct primes, then \[
        a^{(p-1)(q-1)} \equiv 1 \ (\text{mod } pq) \quad \text{for all } a  \text{ such that } \gcd(a, pq) = 1.
    \]
\end{theorem}

Recall that for Diffie Hellman Elgamal we need to find \(a^x \ (\text{mod } p)\). Now, for RSA, we will be solving \(x^e \ (\text{mod } N)\) where \(N = pq\).

\wrpprop{3.2}{Let \(p\) be prime and let \(e\) be defined as \(\{e \in \Z \geq 1 \ \mid \ \gcd(e,p-q) = 1\}\). (Note this means that \(e^{-1} \text{mod } p - 1\) exists) Then, call \(d = e^{-1}\) i.e, \(d \equiv e^{-1} \ (\text{mod } p-1)\). Then, \(x^e \equiv c \ (\text{mod } p)\) has the unique solution \(x = c^d \ (\text{mod } p)\).}

\begin{example}
    {RSA}Find \(x\): \(x^3 \equiv 2 \ (\text{mod } 17)\).
\end{example}

\lesol{
    \begin{enumerate}
        \item Check with \(\gcd(3, 17-1) = \gcd(3,16)\) and 17 is a prime number. Thus, we can use \hyperlink{3.2}{Proposition 3.2}.
        \item Let \(d \cdot 3 \equiv 1 \ (\text{mod } 16) \equiv d \equiv 3^{-1} \ (\text{mod } 16) \equiv 11\) (Found with \hyperlink{thm:Extended Euclidean Algorithm}{EEA})
        \item Then, \(x^3 \equiv 2 \ (\text{mod } 17) \Rightarrow x \equiv 2^{11} \ (\text{mod } 17) \equiv 8\).
        \item To check: \(8^3 = 512 \ (\text{mod } 17) = 2\) \cmark
    \end{enumerate}
}

\wrpprop{3.5}{Let \(p \ne q\) be primes, and let \(e\) be defined as \(\{e \in \Z \geq 1 \ \mid \ \gcd(e,p-q) = 1\}\) Thus, there exists a \(d = e^{-1}\ (\text{mod } (p - 1)(q - 1))\). Then, \(x^e \equiv c \ (\text{mod } pq)\) has the unique solution \(x \equiv c^d \ (\text{mod } pq)\).}

\begin{example}
    {RSA}Solve \(x^{169} \equiv 1000 \ (\text{mod } 6887)\)
\end{example}

\lesol{
    \begin{enumerate}
        \item Check \(\gcd(169, (70)(96)) = \gcd(169, 6720) = 1\) where 70 and 96 are from the prime factorization of \(6887\) such that \((71 - 1)(97 - 1)\) are from \((p - 1)(q - 1)\).
        \item Solve for \(d\) such that \(d169 \equiv \ (\text{mod } 6720)\). Using \hyperlink{thm:Extended Euclidean Algorithm}{EEA}, we find that \(d \equiv -1511 \ (\text{mod} 6720) \equiv 5209\).
        \item Solve \(x \equiv 1000^{5209} \ (\text{mod } 6887) = 4055\)
    \end{enumerate}
}