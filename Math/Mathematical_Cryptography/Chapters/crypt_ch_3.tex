\section{Euler's Totient Theorem and Roots Modulo pq}

\renewcommand{\theenumi}{\arabic{enumi}}
\renewcommand{\labelenumi}{\theenumi.}

Remember \hyperref[thm:Fermat's Little Theorem]{FLT} and \hyperref[Discrete Logarithm Problem]{DLP}? Can we use FLT for non-prime moduli? No, FLT is only true for prime moduli: \(2^5 \ (\text{mod 6}) = 32 \). Thus, we need to look to other ways of solving this problem for non-primes.

\begin{theorem}
    {Euler's Totient Theorem for phi}Let \(N\) be a positive integer and let \(a\) be an integer such that \(\gcd(a, N) = 1\). Then, \[
        a^{\varphi(N)} \equiv 1 \ (\text{mod }{n}).
    \] where \(\varphi(N) = \#\{0 \leq x < N \mid \gcd(x,N) = 1\}\)
\end{theorem}

\begin{corollary}
    {1: Euler's Totient Theorem for \textit{phi}}If \(N\) is prime, \(a^{\varphi(N)} \equiv 1 \ (\text{mod } N)\). For \(\varphi(N)\), we can express it as \(N-1\) from \hyperref[thm:Fermat's Little Theorem]{FLT}.
\end{corollary}

\cpf{
    If \(N\) is prime, then \(\gcd(a,N) = 1\). Thus, for all \(a\), such that \(0 < a < N\), \(\gcd(a,N) = 1\). Thus, \(\varphi(N) = N - 1\).
}

\begin{corollary}
    {2: Euler's Totient Theorem for \textit{phi}}If \(p,q\) are distinct primes, then \[
        a^{(p-1)(q-1)} \equiv 1 \ (\text{mod } pq) \quad \text{for all } a  \text{ such that } \gcd(a, pq) = 1.
    \] From this, we can state that \(N = pq\).
\end{corollary}

\cpf{
    For this proof, we would show that \(\varphi(pq) = \varphi(q) \cdot \varphi(p) = (p - 1)(q - 1)\).
}

\begin{example}
    {ETT for \textit{phi}}Find \(a^x \ (\text{mod } 15)\).
\end{example}

\lesol{
    \(15 = 3 \cdot 5\) (relatively prime); \(\varphi(15) = 2 \cdot 4 = 8\). Thus, \(a^8 \equiv 1 \ (\text{mod } 15)\).
}


\begin{theorem}
    {Euler's Totient Theorem for pq}Let \(p\) and \(q\) be distinct primes and let \[
        g = \gcd(p-1, q-1).
    \]Then, \[
        a^{(p-1)(q-1)/g} \equiv 1 \ (\text{mod } pq) \quad \text{for all } a  \text{ such that } \gcd(a, pq) = 1.
    \] In particular, if \(p\) and \(q\) are distinct primes, then \[
        a^{(p-1)(q-1)} \equiv 1 \ (\text{mod } pq) \quad \text{for all } a  \text{ such that } \gcd(a, pq) = 1.
    \]
\end{theorem}

Recall that for Diffie Hellman Elgamal we need to find \(a^x \ (\text{mod } p)\). Now, for RSA, we will be solving \(x^e \ (\text{mod } N)\) where \(N = pq\).

\wrpprop{3.2}{Let \(p\) be prime and let \(e\) be defined as \(\{e \in \Z \geq 1 \ \mid \ \gcd(e,p-q) = 1\}\). (Note this means that \(e^{-1} \text{mod } p - 1\) exists) Then, call \(d = e^{-1}\) i.e, \(d \equiv e^{-1} \ (\text{mod } p-1)\). Then, \(x^e \equiv c \ (\text{mod } p)\) has the unique solution \(x = c^d \ (\text{mod } p)\).}

\begin{example}
    {RSA 1}Find \(x\): \(x^3 \equiv 2 \ (\text{mod } 17)\).
\end{example}

\lesol{
    \begin{enumerate}
        \item Check with \(\gcd(3, 17-1) = \gcd(3,16)\) and 17 is a prime number. Thus, we can use \refprop{3.2}.
        \item Let \(d \cdot 3 \equiv 1 \ (\text{mod } 16) \equiv d \equiv 3^{-1} \ (\text{mod } 16) \equiv 11\) (Found with \hyperref[thm:Extended Euclidean Algorithm]{EEA})
        \item Then, \(x^3 \equiv 2 \ (\text{mod } 17) \Rightarrow x \equiv 2^{11} \ (\text{mod } 17) \equiv 8\).
        \item To check: \(8^3 = 512 \ (\text{mod } 17) = 2\) \cmark
    \end{enumerate}
}

\wrpprop{3.5}{Let \(p\) and \(q\) be distinct primes and let \(e \geq 1\) satisfy
    \[
        \gcd(e, (p - 1)(q - 1)) = 1.
    \]
    Then, \refprop{1.13} tells us that \(e\) has an inverse modulo \((p-1)(q-1)\), say
    \[
        de \equiv 1 \pmod{(p-1)(q-1)}.
    \]
    Then the congruence
    \[
        x^e \equiv c \pmod{pq}
    \]
    has the unique solution \(x \equiv c^d \pmod{pq}\).
}%

\begin{example}
    {RSA 2}Solve \(x^{169} \equiv 1000 \ (\text{mod } 6887)\)
\end{example}

\lesol{
    \begin{enumerate}
        \item Check \(\gcd(169, (70)(96)) = \gcd(169, 6720) = 1\) where 70 and 96 are from the prime factorization of \(6887\) such that \((71 - 1)(97 - 1)\) are from \((p - 1)(q - 1)\).
        \item Solve for \(d\) such that \(d169 \equiv 1 \pmod{6720}\). Using \hyperref[thm:Extended Euclidean Algorithm]{EEA}, we find that \(d \equiv -1511 \ (\text{mod} 6720) \equiv 5209\).
        \item Solve \(x \equiv 1000^{5209} \ (\text{mod } 6887) = 4055\)
    \end{enumerate}
}

\renewcommand{\theenumi}{\alph{enumi}}
\renewcommand{\labelenumi}{(\theenumi)}
\subsection{Exercises}

\begin{exercise}
    {3.1} {Solve the following congruences (use \hyperref[ex:3.3]{this example} for help).}
    \begin{enumerate}
        \item \(x^{19} \equiv 36 \pmod{97}\).
        \item \(x^{137} \equiv 428 \pmod{541}\).
        \item \(x^{73} \equiv 614 \pmod{1159}\).
        \item \(x^{751} \equiv 677 \pmod{8023}\).
    \end{enumerate}
\end{exercise}

\sol{
    \begin{enumerate}
        \item Because we know 97 is a prime number, and that \(\gcd(19, 96) = 1\), we can use \refprop{3.2}. Thus, we need to solve for \(d\): \begin{align*}
                  d \cdot 19 & \equiv 1 \pmod{96}       \\
                  d          & \equiv 19^{-1} \pmod{96} \\
                  d          & \equiv 91.
              \end{align*} Then, we can solve for \(x\): \begin{align*}
                  x^{19} & \equiv 36 \pmod{97}      \\
                  x      & \equiv 36^{91} \pmod{97} \\
                  x      & = 36.
              \end{align*}

        \item Using \refprop{3.2}, again:
              \begin{align*}
                  d \cdot 137 & \equiv 1 \pmod{540}        \\
                  d           & \equiv 137^{-1} \pmod{540} \\
                  d           & \equiv 473.
              \end{align*} Then, for \(x\):
              \begin{align*}
                  x^{137} & \equiv 428 \pmod{541}       \\
                  x       & \equiv 428^{473} \pmod{541} \\
                  x       & \equiv 213.
              \end{align*}
        \item \begin{align*}
                  d & \equiv 73^{-1} \pmod{1158} \\
                  d & \equiv 349.
              \end{align*} Then,
              \begin{align*}
                  x & \equiv 614^{349} \pmod{1159} \\
                  x & \equiv 598. 
              \end{align*}
        \item \begin{align*}
                  d & \equiv 751^{-1} \pmod{8022} \\
                  d & \equiv 235.
              \end{align*} Then, \begin{align*}
                  x & \equiv 677^{235} \pmod{8023} \\
                  x & \equiv 4858. 
              \end{align*}
    \end{enumerate}
}

\begin{exercise}
    {Additional Problem 1} Prove that \(\varphi(pq) = (p-1)(q-1)\) when \(p\) and \(q\) are distinct primes (where \(\varphi\) is \hyperref[Euler's Phi Function]{Euler's Phi Function}).
\end{exercise}

\expf{
    Let \(p\) and \(q\) be distinct primes. From the definition of \hyperref[Euler's Phi Function]{Euler's Phi Function}, we know \(\varphi(pq)\) counts the number of integers from 1 to \(pq - 1\) that are co-prime to \(pq\). Since \(p\) and \(q\) are distinct primes, the only numbers that are not co-prime to \(pq\) are multiples of \(p\) and multiples of \(q\).

    The numbers divisible by \(p\) are \(p, 2p, 3p, \dots, (q - 1)p\), for a total of \(q - 1\) multiples of \(p\). Similarly, the numbers divisible by \(q\) are \(q, 2q, 3q, \dots, (p - 1)q\), for a total of \(p - 1\) multiples of \(q\).

    Notice that the number \(pq\) itself is counted twice, once as a multiple of \(p\) and once as a multiple of \(q\). To correct this over-counting, we apply the \href{https://en.wikipedia.org/wiki/Inclusion\%E2\%80\%93exclusion_principle}{Inclusion-Exclusion Principle}. Thus, the total number of integers divisible by either \(p\) or \(q\) is:
    \[
        p + q - 1
    \]

    Since there are \(pq\) integers in total from 1 to \(pq - 1\), the number of integers that are co-prime to \(pq\) is:
    \[
        \varphi(pq) = pq - (p + q - 2) = pq - p - q + 1
    \]

    Simplifying the expression gives:
    \[
        \varphi(pq) = (p - 1)(q - 1) \qedhere
    \]
}

\renewcommand{\theenumi}{\arabic{enumi}}
\renewcommand{\labelenumi}{\theenumi.}
\section{RSA Cryptosystem}

\begin{table}[h!]
    \centering
    \begin{tabular}{@{}p{0.9\textwidth}l@{}}
        \toprule
        \multicolumn{1}{c}{\parbox[t]{0.45\textwidth}{\centering \textbf{Bob}}} & \multicolumn{1}{c}{\parbox[t]{0.45\textwidth}{\centering \textbf{Alice}}}                           \\ \midrule
        \multicolumn{2}{c}{\parbox[t]{0.9\textwidth}{\centering \textbf{Key creation}}}                                                                                               \\ \midrule
        \multicolumn{1}{l}{\parbox[t]{0.45\textwidth}{\raggedright Choose secret primes \(p\) and \(q\).                                                                                      \\ Choose encryption exponent \(e\)  \\ \quad with \(\gcd(e, (p-1)(q-1)) = 1\). \\ 
        Publish \(N = pq\) and \(e\).}} &  \\ \midrule
        \multicolumn{2}{c}{\parbox[t]{0.9\textwidth}\centering\textbf{Encryption}}                                                                                                    \\ \midrule
        \multicolumn{1}{l}{\parbox[t]{0.45\textwidth}}                            & \multicolumn{1}{r}{\parbox[t]{0.45\textwidth}{\raggedright Choose plaintext \(m\).                \\ Use Bob's public key \(N,e\) \\ \quad to compute \(c \equiv m^{e} \pmod{N}\). \\ Send ciphertext \(c\) to Bob.}}                                            \\ \midrule
        \multicolumn{2}{c}{\parbox[t]{0.9\textwidth}{\centering\textbf{Decryption}}}                                                                                                  \\ \midrule
        \multicolumn{1}{l}{\parbox[t]{0.45\textwidth}{\raggedright Compute \(d\) satisfying                                                                          \\ \quad \(ed \equiv 1 \pmod{(p-1)(q-1)}.\) \\ Compute \(m' \equiv c^{d} \pmod{N}\). \\ Then \(m'\) equals the plaintext \(m\). }}                     &                                                                                             \\ \bottomrule
    \end{tabular}
    \caption{Elgamal Key Creation, Encryption, and Decryption}
\end{table}

\label{sec:RSA Algorithm}

\subsection{RSA Algorithm}

Remember, multiplying is easy; we have tools to compute large numbers. However, factoring is hard. The RSA Cryptosystem is based on the difficulty of factoring large numbers. Thus, the RSA Cryptosystem is based on the following steps:
\begin{enumerate}
    \item Alice chooses two distinct primes \(p\) and \(q\) (private key).
    \item With integer \(e\) such that \(\gcd(e, (p-1)(q-1)) = 1\).
    \item Alice computes \(N = pq\) and \(d = e^{-1} \ (\text{mod } (p-1)(q-1))\).
    \item For Bob to send ``\(m\)'' to Alice, he computes \(c = m^e \ (\text{mod } N)\).
    \item For Alice to decrypt, she computes \(m' = c^d \ (\text{mod } N)\).
\end{enumerate}
\pfs

We claim that \(m' = m\). \\

\noindent \textit{Proof.} \begin{align*}
    d  & \equiv e^{-1}\pmod{(p-1)(q-1)}                             \\
    de & \equiv 1                       & \pmod{(p-1)(q-1)}         \\
    de & = 1 + k(p-1)(q-1)              & \text{for some } k \in \Z
\end{align*} Now that we know \(de\), we can compute \(m'\):

\begin{align*}
    m' & = c^d \pmod{N}                                  \\
       & = m^{ed} \pmod{N}                               \\
       & = m^{1 + k(p-1)(q-1)} \pmod{N}                  \\
       & = m \cdot m^{k(p-1)(q-1)} \pmod{N}              \\
       & = m \cdot (m^{(p-1)(q-1)})^k \pmod{N} \tag{3.1} \\
       & = m \cdot 1^k \pmod{N}                          \\
       & = m \pmod{N}.
\end{align*} Note for (3.1), we got 1 because we substituted for \(\varphi\). Hence, \begin{align*}
    m (m^{(p-1)(q-1)}) \pmod{N} & \equiv m (m^{\varphi(pq)} \pmod{N}) \\
                                & \equiv m \cdot 1 \pmod{N}.
\end{align*} \qed

Remember from Chapter 1, for a Cryptosystem to be a \hyperref[Successful Ciphers]{successful cipher}, it must satisfy the following properties:

\begin{enumerate}
    \item \textbf{Encryption:} Must be easy to encrypt: \(c \equiv m^e \pmod{N}\).
    \item \textbf{Decryption:} Must be hard to decrypt:
          \begin{itemize}
              \item Need to find \(d \equiv e^{-1}\pmod{(p-1)(q-1)}\) which requires knowing \(p\) and \(q\) separately. Remember that factoring is hard.
          \end{itemize}
\end{enumerate}

\begin{example}
    {RSA 3}Let \(N = (47)(43) = 2021\) with private key \(47,43\) and \(e = 11\) (Note this means that \(e \ne 2 \cdot 23, \ 6 \cdot 7\), and so on). \begin{enumerate}[label=(\alph*)]
        \item Calculate Bob's public key.
        \item Encrypt the message \(m = 1050\) that Bob sends to Alice.
        \item Decrypt the message.
    \end{enumerate}
\end{example}

\lesol{
    \begin{enumerate}[label=(\alph*)]
        \item Calculate \(c\): \(c = 1050^{11} \pmod{2021} = 435\).
        \item Alice first finds \(d\): \(d \equiv 11^{-1} \pmod{(47 -1) (43 - 1)} \equiv 11^{-1} \pmod{1932}\). Using \hyperref[thm:Extended Euclidean Algorithm]{EEA}, we find that \(d \equiv 527\).
        \item Then, \(m' = 435^{527} \pmod{2021} = 1050\).
    \end{enumerate}
}

\begin{center}
    \textbf{Remarks}:
\end{center}
\begin{itemize}
    \item We often choose small \(e\) to make the encryption process faster.
    \item Most agree that a smaller \(e\) is just as secure. The normal choice for \(e\) is 65537.
    \item What is the smallest possible \(e\) for a semiprime \(pq\)? \(e = 3\)
    \item This cryptosystem may be insecure if \(d < N^{1/4}\).
\end{itemize}

\renewcommand{\theenumi}{\alph{enumi}}
\renewcommand{\labelenumi}{(\theenumi)}
\subsection{Exercises}

\begin{exercise}
    {3.7} {Alice publishes her \hyperref[sec:RSA Algorithm]{RSA} public key: modulus \(N = 2038667\) and exponent \(e = 103\).}
    \begin{enumerate}
        \item Bob wants to send Alice the message \(m = 892383\). What ciphertext does Bob send to Alice?
        \item Alice knows that her modulus factors into a product of two primes, one of which is \(p = 1301\). Find a decryption exponent \(d\) for Alice.
        \item Alice receives the ciphertext \(c = 317730\) from Bob. Decrypt the message.
    \end{enumerate}
\end{exercise}

\sol{
    \begin{enumerate}
        \item Bob needs to calculate \(c \equiv 892383^{103} \pmod{2038667} \equiv 45293\)
        \item To get the other prime factor of the modulus, Alice simply divides 2038667 by 1301. She gets 1567. Then Alice finds: \begin{align*}
                  d & \equiv 103^{-1} \pmod{(1301 - 1)(1567 - 1)} \\
                  d & \equiv 810367
              \end{align*}
        \item To decrypt, Alice solves for \(m\): \begin{align*}
                  m & \equiv 317730^{810367} \pmod{2038667} \\
                  m & \equiv 514407
              \end{align*}
    \end{enumerate}
}

\begin{exercise}
    {3.8} {Bob's RSA public key has modulus \(N = 12191\) and exponent \(e = 37\). Alice sends Bob the ciphertext \(c = 587\). Unfortunately, Bob has chosen too small a modulus. Help Eve \hyperref[sec:Ciphertext Attack]{by factoring} \(N\) and decrypting Alice's message. (Hint: \(N\) has a factor smaller than \(100\).)}
    \begin{enumerate}
        \item Factor \(N\).
        \item Decrypt Alice's message.
    \end{enumerate}
\end{exercise}

\sol{
    \begin{enumerate}
        \item \(N\)'s factors are \(p = 73\) and \(q = 167\). (This was found by simply looping through all integers less than 100 and dividing \(N\) by each. If I got a number that was not a decimal (i.e., \texttt{if m \% a == 0}), then the loop broke, and \(p\) and \(q\) were printed.)
        \item Eve will solve for \(d\): \begin{align*}
                  d & \equiv 37^{-1} \pmod{(73 - 1)(167 - 1)} \\
                  d & \equiv 11629
              \end{align*}


              Then she will solve for \(m\):
              \begin{align*}
                  m & \equiv 587^{11629} \pmod{12191} \\
                  m & \equiv 4894
              \end{align*}
    \end{enumerate}
}

\begin{exercise}
    {3.9} {For each of the given values of \(N = pq\) and \((p - 1)(q - 1)\), use the method described in \hyperref[sec:Ciphertext Attack]{Remark 3.11} to determine \(p\) and \(q\).}
    \begin{enumerate}
        \item \(N = pq = 352717\) and \((p - 1)(q - 1) = 351520\).
    \end{enumerate}
\end{exercise}

\sol{
    \begin{enumerate}
        \item First, we need get \((p + q)\) alone: \begin{align*}
                  (p-1)(q-1) & = pq - (p + q) + 1 \\
                  351520     & = 352718 - (p + q) \\
                  1198       & = p + q.
              \end{align*} Now, we can solve the quadratic equation \(x^2 -1198x + 352717\) to find \(p,q\). The answer is \(p = 521, \ q = 677\). To check: \(521 \cdot 677 = 352717\). \cmark%
    \end{enumerate}
}


\renewcommand{\theenumi}{\arabic{enumi}}
\renewcommand{\labelenumi}{\theenumi.}
\section{Implementation and Security Issues}

\subsubsection{Brute Force Attack}

This involves factoring \(N = pq\). We only consider numbers below \(\sqrt{N}\).


\subsection{Chosen Ciphertext Attack}
\label{sec:Ciphertext Attack}

If Eve can find \((p-1)(q-1)\), then \(p\) and \(q\) are easy.  \\

\noindent \textit{Proof.} \begin{align*}
    (p-1)(q-1) & = pq - p - q + 1   \\
               & = pq - (p +q) + 1.
\end{align*} Thus, Eve just needs to solve the quadratic equation \((x-p)(x-q) = x^2 - (p+q)x + pq = 0\). \qed

\begin{example}
    {CCA}Factor \(N = 2701\) such that \((p-1)(q - 1) = 2592\).
\end{example}

\lesol{
    Find \begin{align*}
        (p-1)(q-1) & = pq - (p + q) + 1   \\
        2592       & = 2701 - (p + q) + 1 \\
        110        & = p + q.
    \end{align*}
    Then, solve the quadratic equation \(x^2 - 110x + 2701 = 0\) to find \(p\) and \(q\). Thus, \(x = 37, 73\).
}

\subsection{Timing Attacks (Ex. of Side Channel Attack)}

This attack is based on the time it takes to decrypt a message. More specifically, it monitors how long it takes a computation to complete. Then, it derives the relative sizes of \(p\) and \(q\).

\renewcommand{\theenumi}{\arabic{enumi}}
\renewcommand{\labelenumi}{\theenumi.}
\section{Primality Testing}

The only primality tests that we know so far is the brute force method. In other words, we brute force finding prime numbers. We do this by dividing \(N\) by primes. If no numbers \hyperref[Divides]{divide} \(N\), then \(N\) is prime. This test is not efficient for large numbers.

Thus, consider a new test that utilizes \hyperref[thm:Fermat's Little Theorem]{FLT}.

\begin{theorem}
    {Fermat's Little Theorem, Version 2}If \(p\) is prime, then \(a^P \equiv a \pmod{p}\). Consider the contrapositive to this statement: if \(a^P \not\equiv 1 \pmod{p}\) for some \(a\), then \(P\) is \textit{not} prime.
\end{theorem}

\begin{example}
    {FLT}Let \(N = 21\). Is \(N\) prime?
\end{example}

\lesol{
    Pick \(a = 3\). Check \(3^{21} \stackrel{?}{\equiv} 3 \pmod{21}\). No, \(3^{21} \equiv 6 \pmod{21}\). Hence, 21 is not prime.
}

\wrpdef{Fermat Witness}{We call \(a\) a \textit{witness} to the fact that \(N\) is composite if \(a^N \not\equiv a \pmod{N}\).}

\wrpdef{Fermat Liar}{A \textit{Fermat liar} is a number \(a\) such that, even though \(N\) is composite, it satisfies \(a^{-1} \equiv 1 \pmod{n}\). This gives a false indication that \(N\) is prime. Simply put, \(a\) ``lies'' by making \(N\) appear prime when it is not.}

However, there are some numbers will force all \(a\)'s to be fermat liars. These are called \textit{Carmichael Numbers}.

\wrpdef{Carmichael Numbers}{A composite number \(N\) is a \textit{Carmichael Number} if \(a^N \equiv a \pmod{N}\) for all \(a\) such that \(\gcd(a,N) = 1\).}
\newpage
\begin{example}
    {Carmichael Numbers}Solve \(37^{561} \equiv \pmod{561}\). What can you conclude about 561? Is it prime?
\end{example}

\lesol{
    \begin{enumerate}
        \item Check if 561 is prime. Pick \(a = 37\). Check \(37^{561} \stackrel{?}{\equiv} 37 \pmod{561}\). Yes, \(37^{561} \equiv 37 \pmod{561}\). Thus, 561 is prime. Right?
        \item If we check for all \(a < 561\), we will see that 561 has no \hyperref[Fermat Witness]{Fermat witnesses}. However, we know that 561 is composite, so that means 561 is a Carmichael Number.
    \end{enumerate}
}
\subsection{Miller-Rabin Primality Test}
\label{sec:Miller-Rabin}

\wrpprop{3.17}{Let \(p\) be an odd prime, and write \(p - 1 = 2^kq\) with \(q\) being odd. Then, let \(a\) be any number not divisible by \(p\). Then, one of the following must hold: \begin{enumerate}[label=(\roman*)]
        \item \(a^q \equiv 1 \pmod{p}\).
        \item One of \(a^q,a^{2q},a^{4q},\dots,a^{2^{k-1}q}\) is congruent to \(-1\) modulo \(p\). 
    \end{enumerate}}

\begin{center}
    \textbf{Miller-Rabin Algorithm }    
\end{center}

\begin{enumerate}
    \item Choose an odd integer \(n > 2\) to test for primality.
    \item Write \(n - 1 = 2^k \cdot q\) where \(q\) is odd and \(k \geq 1\). 
    \item Pick a random integer \(a\) such that \(2 \leq a \leq n - 2\).
    \item compute \(x = a^q \pmod{n}\).
    \begin{itemize}
        \item If \(x = 1\) or \(x = n - 1\), \textbf{continue to the next base because this indicates that the number is probably prime.}
    \end{itemize}
    \item Square \(x\), up to \(k - 1\) times:
    \begin{itemize}
        \item If \(x = n - 1\) at any step, \textbf{continue to the next base for the same reason.}
        \item If \(x \ne n - 1\) for all squarings, \textbf{\(n\) is composite.}
        
        For example, if we have a number and we keep squaring \(x\) up to \(k - 1\), and we never encounter a \(n - 1\), but we encounter other numbers like 1 or anything else, then the number is composite. Remember: ONLY composite if there was NO record of a \(n - 1\) for a given base \(x\).
    \end{itemize}
    \item Repeat the test with different random bases \(a\).
\end{enumerate}

\wrpdef{Miller-Rabin Witness}{Let \(N\) be an odd number. Write \(N - 1 = 2^kq\) with \(q\) odd. If \(a\) is an integer that satisfies \(\gcd(a,N) = 1\), then \(a\) is a \textit{Miller-Rabin Witness} for \(N\) if both of the following hold: \begin{enumerate}[label=(\roman*)]
        \item \(a^q \not\equiv 1 \pmod{N}\).
        \item \(a^{2^iq} \not\equiv -1 \pmod{N}\) for all \(i\) such that \(0 \leq i \leq k-1\).
    \end{enumerate}}

\begin{example}
    {Miller-Rabin Test 1}Let \(N = 561\). Is \(N\) prime?
\end{example}

\lesol{
    \begin{enumerate}
        \item Write \(N - 1 = 2^4 \cdot 35\).
        \item Pick \(a = 7\). Start by solving \(7^{35} \pmod{561} \equiv 241\). If it was congruent to 1 or \(-1\), then \(N\) is a possible prime.
        \item If the number is neither congruent to 1 or \(-1\), keep squaring and checking up to \(k - 1\) times.
              \begin{itemize}
                  \item Like this: \((7^{36})^2 \pmod{561} \equiv 298\). If this number is congruent to 1, then we are done and the number is composite.
              \end{itemize}
        \item If it is congruent to \(-1\), then we choose another \(a\), and note this is a possible prime. If it is not congruent to either \(-1\) or 1, then keep squaring up to \(k - 1\) times.
        \item Thus, \((7^{36})^4 \pmod{561} \equiv 166\), \((7^{36})^8 \pmod{561} \equiv 67\), and \((7^{36})^{16} \pmod{561} \equiv 1\), so 561 is composite and 7 is a \hyperref[Miller-Rabin Witness]{M-R witness} because we found a 1 before we found a \(-1\) in the sequence.
    \end{enumerate}
}

\begin{example}
    {Miller-Rabin Test 2}Let \(N = 563\). Let \(a = 7\). Is \(N\) prime?
\end{example}

\lesol{
    \(N - 1 = 562 = 2(281)\). Start by solving \(7^{281} \pmod{563} \equiv 1\). Thus, 563 is a possible prime. Choose \(a = 8\). Then, \(8^{281} \pmod{563} \equiv -1\), so \(N\) is a possible prime.
}

\begin{example}
    {Miller-Rabin Test 3}Let \(N = 31001\). Determine if \(N\) is prime. 
\end{example}

\lesol{
    Start by finding \(p - 1 = 31000 = 2^3 \cdot 3875\). Then, use the fast powering algorithm to find \(b_0 = 2^{3875} \pmod{31001} \equiv 5799\) for \(b_1\). Since \(b_1\) is not 1 or \(-1\), we will try the other bases up to \(k - 1 = 3 - 1 = 2\). Thus, we use the fast powering algorithm to solve for subsequent \(b\)'s: 
    \[\begin{array}{lccccr}
        b_0^{3875} &=& 2^{3875} &\equiv& 5799 \pmod{31001} &= b_1 \\
        b^2_1 &=& 5799^2 &\equiv& 13026 \pmod{31001} &= b_2 \\
        b^2_2 &=& 13026^2 &\equiv& 8203 \pmod{31001} &= b_3
    \end{array}\] 
    Since no \(n - 1\) appeared in these steps, and there was not a 1 that appeared before a \(-1\), then this number is composite.
}

\begin{note}
    \textbf{Note:} Let \(N\) be an odd positive integer. Then, \(\geq 75\%\) of \(a\)'s are between \(1\) and \(n - 1\) are \hyperref[Miller-Rabin Witness]{Miller-Rabin Witnesses} for \(N\).
\end{note}

\renewcommand{\theenumi}{\alph{enumi}}
\renewcommand{\labelenumi}{(\theenumi)}
\subsection{Exercises}

\begin{exercise}
    {Additional Problem 2} Decide whether each of the following numbers is composite or probably prime by finding a witness or finding 5 liars using \hyperref[thm:Fermat's Little Theorem, Version 2]{Fermat's Little Theorem, V2}:
    \[8911; 13291; 415693\]
    For each of the possible primes left over from the previous part, use the \hyperref[sec:Miller-Rabin]{Miller-Rabin} test to either find a witness for the compositeness of the number or give 5 more liars to show the number is probably prime.
\end{exercise}

\sol{
    \begin{enumerate}
        \item For 8911: pick \(a = 2\). Thus, \(2^{8911} \stackrel{?}{\equiv} 2 \pmod{8911} \equiv 2\) \cmark. Thus, 2 is either a Fermat liar, or 8911 is prime. Let's continue for 4 more \(a\)'s:
              \begin{itemize}
                  \item \(a = 3^{8911} \pmod{8911} \stackrel{?}{\equiv} 3\)\ \cmark%
                  \item \(a = 4^{8911} \pmod{8911} \stackrel{?}{\equiv} 4\)\ \cmark%
                  \item \(a = 5^{8911} \pmod{8911} \stackrel{?}{\equiv} 5\)\ \cmark%
                  \item \(a = 6^{8911} \pmod{8911} \stackrel{?}{\equiv} 6\)\ \cmark%
              \end{itemize}
              Since every \(a\) we tested was true for the equation \(a^{N} \equiv a \pmod{N}\), 8911 is either prime, or all the \(a\)'s are liars. Now, we can use the Miller-Rabin test to further probe the possibility that 8911 is prime. We start by writing \(N - 1 = 2^kq\) with \(q\) odd. Thus, \(8911 = 2^1 \cdot 4455\). Now, we pick random \(a\)'s and solve for \(a^{4455} \pmod{8911} \stackrel{?}{\equiv} 1\):
              \begin{itemize}
                \item \(a = 3^{4455} \pmod{8911} \stackrel{?}{\equiv} 1\). Probably prime because \(\not\equiv 1\) or \(N - 1\).
                \item \(a = 5^{4455} \pmod{8911} \stackrel{?}{\equiv} 1\)\ \xmark. This gives 2813. Now, we need to use the \hyperref[fast powering algorithm]{fast powering algorithm} to check if for any \(k\), \(5^{2^k \cdot q}\) is equal to \(N - 1\) or \(1\):
                \begin{align*}
                    k_0 &= 2813 \pmod{8911} \\
                    k_1 &= 2813^2 \pmod{8911} \equiv 1 \\
                    k_2 &= 1^2 \pmod{8911} \equiv 1 \\
                    &\vdots
                \end{align*} Since we did not have any result of \(a^{2^k \cdot q} = N - 1\), 8911 is not prime with 5 as a Miller-Rabin witness.
              \end{itemize} 
        \item For 13291: pick \(a = 2\): \(2^{13291} \stackrel{?}{\equiv} 2 \pmod{13291} \equiv 2\) \cmark. Thus, 2 is either a Fermat liar, or 13291 is prime. Let's continue for 4 more \(a\)'s:
              \begin{itemize}
                  \item \(a = 3^{13291} \pmod{13291} \stackrel{?}{\equiv} 3\)\ \cmark%
                  \item \(a = 4^{13291} \pmod{13291} \stackrel{?}{\equiv} 4\)\ \cmark%
                  \item \(a = 5^{13291} \pmod{13291} \stackrel{?}{\equiv} 5\)\ \cmark%
                  \item \(a = 6^{13291} \pmod{13291} \stackrel{?}{\equiv} 6\)\ \cmark%
              \end{itemize}
              We can use the Miller-Rabin test to see if 13291 is prime. From the formula \(N - 1 = 2^kq\) with \(q\) odd: \(13290 = 2^1 \cdot 6645\). Pick random \(a\)'s and solve for \(a^{6645} \pmod{13291} \stackrel{?}{\equiv} 1\):
              \begin{itemize}
                  \item \(a = 3^{6645} \pmod{13291} \equiv 13290\)\ \cmark%
                  \item \(a = 7^{6645} \pmod{13291} \equiv 1\)\ \cmark%
                  \item \(a = 11^{6645} \pmod{13291} \equiv 13290\)\ \cmark%
                  \item \(a = 13^{6645} \pmod{13291} \equiv 13290\)\ \cmark%
                  \item \(a = 17^{6645} \pmod{13291} \equiv 13290\)\ \cmark%
              \end{itemize} Either 13290 is prime, or all \(a\)'s that we have chosen are Miller-Rabin liars.
        \item For 415693: pick \(a = 2\): \(2^{415693} \stackrel{?}{\equiv} 2 \pmod{415693} \equiv 116692\) \xmark. Therefore, 2 is a witness that 415693 is composite.
    \end{enumerate}
}

\renewcommand{\theenumi}{\arabic{enumi}}
\renewcommand{\labelenumi}{\theenumi.}
\section{Pollard's Factorization Algorithm}

For \(N = pq\) semiprime, this algorithm is efficient when \(p -1\) only has small prime factors.

This algorithm is based on the following steps:

\begin{itemize}
    \item \textbf{Step 1:} \textit{Choose a smooth bound B:} The smooth bound is a small number such that the prime factors of \(p - 1\) (where \(p\) is a prime factor of \(n\)) are less than or equal to \(B\).
    \item \textbf{Step 2:} \textit{Compute the least common multiple of all integers up to \(B\), call this number \(k\).}
    \item \textbf{Step 3:} \textit{Pick a random base a:} Choose an integer \(a\) such that \(2 \leq a \leq n - 2\).
    \item \textbf{Step 4:} \textit{Compute \(a^{k} \pmod{n}\)}.
    \item \textbf{Step 5:} \textit{Find the GCD of \(a^k - 1\) and \(n\):} The greatest common divisor \(\gcd(a^k - 1, n)\) may give a non-trivial factor of \(n\).
    \item \textbf{Step 6:} If \(\gcd(a^k - 1,n) = n\), try a different \(a\) or increase \(B\).
\end{itemize}

\begin{center}
    \textbf{In Python:}
\end{center}

\label{pollard's p python}
\begin{lstlisting}[language=Python, numbers=left]
def pollards_p_minus_1(base, mod, limit):
    lcm_val = 1
    for i in range(1, limit + 1):
        lcm_val = math.lcm(lcm_val, i)
        pow_val = pow(base, lcm_val, mod) - 1
        gcd_val = math.gcd(pow_val, mod)
        print(f'lcm({i}) = {lcm_val}, gcd: {gcd_val}')
        if 1 < gcd_val < mod:
            return gcd_val
    return None
\end{lstlisting}

\begin{example}
    {Pollard's Factorization Algorithm}Factor 540143 with \(a = 2\).
\end{example}

\lesol{ \vspace{0.5cm} \newline
    \begin{tabular}{l|l|c|r}
        \(B\) & \(\text{lcm}(k = 1,2\dots, 13)\)                                   & \(2^k - 1 \pmod{510143}\)             & \(\gcd(2^k - 1, 510143)\)      \\ \midrule
        5     & \(\text{lcm}(1,2,\underline{3},\underline{4},\underline{5}) = 60\) & \(2^{60 - 1 \pmod{540143}}\)          & \(\gcd(338353, 540143) = 1\)   \\ \midrule
        7     & \(\text{lcm}(1,2,\dots,7) = 420\)                                  & \(2^{420} - 1 \pmod{540143} = 42942\) & \(\boxed{421}\  \text{Stop!}\) \\  \bottomrule
    \end{tabular} \newline

    Thus, we found that 421 is a factor of 540143.
}

\renewcommand{\theenumi}{\alph{enumi}}
\renewcommand{\labelenumi}{(\theenumi)}
\subsection{Exercise}

\begin{exercise}
    {3.22} Use \hyperref[ex:3.10]{Pollard's} \(p - 1\) method to factor each of the following numbers.
    \begin{enumerate}
        \item \(n = 1739\)
        \item \(n = 220459\)
    \end{enumerate}
    Be sure to show your work and indicate which prime factor \(p\) of \(n\) has the property that \(p - 1\) is a product of small primes.
\end{exercise}

\sol{
    \begin{enumerate}
        \item
              \begin{itemize}
                  \item \textbf{Step 1:} Choose a smooth \(B = 5\). We assume that one of the prime factors \(p\) of \(n = 1739\) has the property \(p - 1\) is divisible by small primes, and we will try a small \(B\).
                  \item \textbf{Step 2:} Compute the least common multiple of all integers up to \(B = 5\): \[k = \text{lcm}(1,2,3,4,5) = 60.\]
                  \item \textbf{Step 3:} Pick a random base \(a = 2\).
                  \item \textbf{Step 4:} Compute \(a^{k} - 1 \pmod{n}\): \[2^{60} - 1 \pmod{1739} = 638.\]
                  \item \textbf{Step 5:} Compute the greatest common divisor of \(n\) and \(a^{k} - 1\): \[\gcd(1739, 638) = 1\]
                  \item Since the greatest common divisor is 1, we need to try a larger \(B\). We will try \(B = 7\): \[k = \text{lcm}(1,2,3,4,5,6,7) = 420.\]
                  \item Compute \[2^{420} - 1 \pmod{1739} = 157, \ \gcd(157, 1739) = 1.\]
                  \item Try \(B = 11\), and compute: \[2^{27720} - 1 \pmod{1739} = 407, \ \gcd(407, 1739) = 37\]
              \end{itemize}
              Thus, we have found a factor of 1739: 37. By dividing 1739 by 37, we get 47. Thus, 1739 factors into 37 and 47. Note that \(p - 1 = 36 = 2^2 \cdot 3^2\) and \(q - 1 = 46 = 2 \cdot 23\).
              %   \(B = 13\): \(2^{360360} \pmod{1739} = 1037\), \(\gcd(1037, 1739) = 37\) \\
              %   \(B = 17\): \(2^{12252240} \pmod{1739} = 1555\), \(\gcd(1555, 1739) = 37\) \\
              %   \(B = 19\): \(2^{232792560} \pmod{1739} = 667\), \(\gcd(666, 1739) = 37\) \\
        \item We will still be using the same steps from part (a), except this time we will skip straight to trying different \(B\)'s.
              \begin{itemize}
                  \item \(B = 5\): \(2^{60} - 1 \pmod{220459} = 120157\), \(\gcd(120157, 220459) = 1\).
                  \item \(B = 7\): \(2^{420} - 1 \pmod{220459} = 89023\), \(\gcd(89023, 220459) = 1\).
                  \item \(B = 11\): \(2^{27720} - 1 \pmod{220459} = 155869\), \(\gcd(155869, 220459) = 1\).
                  \item \(B = 13\): \(2^{360360} - 1 \pmod{220459} = 97948\), \(\gcd(97948, 220459) = 1\).
                  \item \(B = 17\): \(2^{12252240} - 1 \pmod{220459} = 144576\), \(\gcd(144576, 220459) = 1\).
                  \item \(B = 19\): \(2^{232792560} - 1 \pmod{220459} = 71838\), \(\gcd(71838, 220459) = 1\).
                  \item \(B = 23\): \(2^{5354228880} - 1 \pmod{220459} = 201150\), \(\gcd(201150, 220459) = 1\).
              \end{itemize}
              If we continue this process, we find a \(B = 5342931457063200\), which gives the factor 449. Divide 220459 by 449, and we get 491. Note that \(p - 1 = 448 = 2^6 \cdot 7\) and \(q - 1 = 490 = 2 \cdot 5 \cdot 7^2\). (See this \hyperref[pollard's p python]{Python code} for how I automated finding the factors of 220459.)
    \end{enumerate}
}



\renewcommand{\theenumi}{\arabic{enumi}}
\renewcommand{\labelenumi}{\theenumi.}
\section{Factorization via Difference of Squares}

We would only use this method if prime numbers, \(p\) and \(q\) are close to each other.

To factor \(N\) we find a value \(k\) such that \(N + k^2 = m^2\) for some \(m \in \Z\). This is equal to \(N = m^2 - k^2 = (m+k)(m-k)\).

\begin{example}
    {Factorization 1} Factor 45.
\end{example}

\lesol{
    \begin{align*}
        45 + 1^2 & = 46 \ \text{\xmark} \\
        45 + 2^2 & = 49 \ \text{\cmark} \\
        45       & = 7^2 - 2^2          \\
        45       & = (7 + 2)(7 - 2)     \\
        45       & = (9)(5)
    \end{align*}
}


\begin{example}
    {Factorization 2} Factor \(N = 25217\) by looking for an integer \(b\) making \(N + b^2\) a perfect square.
\end{example}

\lesol{
    We factor \( N = 25217 \) by looking for an integer \( b \) making \( N + b^2 \)
    a perfect square:


    \begin{align*}
        25217 + 1^2 & = 25218         & \text{not a square},            \\
        25217 + 2^2 & = 25221         & \text{not a square},            \\
        25217 + 3^2 & = 25226         & \text{not a square},            \\
        25217 + 4^2 & = 25233         & \text{not a square},            \\
        25217 + 5^2 & = 25242         & \text{not a square},            \\
        25217 + 6^2 & = 25253         & \text{not a square},            \\
        25217 + 7^2 & = 25266         & \text{not a square},            \\
        25217 + 8^2 & = 25281 = 159^2 & \text{Eureka! \textbf{square}.}
    \end{align*}


    Then we compute

    \[
        25217 = 159^2 - 8^2 = (159 + 8)(159 - 8) = 167 \cdot 151.
    \]

}

\renewcommand{\theenumi}{\alph{enumi}}
\renewcommand{\labelenumi}{(\theenumi)}
\subsection{Exercise}

\begin{exercise}
    {3.24} {For each of the following numbers \(N\), compute the values of \(N + 1^2, N + 2^2, N + 3^2, N + 4^2\),\dots as we did in Example 3.34 (\hyperref[ex:3.11]{Example 3.11 in notes}), until you find a value \(N + b^2\) that is a perfect square \(a^2\). Then use the values of \(a\) and \(b\) to factor \(N\).}
    \begin{enumerate}
        \item \(N = 53357\)
        \item \(N = 34571\)
    \end{enumerate}
\end{exercise}

\sol{
    \begin{enumerate}
        \item     \begin{align*}
                  53357 + 1^2 & = 53358         & \text{not a square},            \\
                  53357 + 2^2 & = 53361 = 231^2 & \text{Eureka! \textbf{square}.}
              \end{align*} \(53357 = 231^2 - 2^2 = (231 + 2)(231 - 2) = 233 \cdot 229\). Therefore, 53357 factors into 233 and 229.
        \item \begin{align*}
                  34571 + 1^2 & = 34572         & \text{not a square},            \\
                  34571 + 2^2 & = 34575         & \text{not a square},            \\
                  34571 + 3^2 & = 34580         & \text{not a square},            \\
                  34571 + 4^2 & = 34587         & \text{not a square},            \\
                  34571 + 5^2 & = 34596 = 186^2 & \text{Eureka! \textbf{square}.}
              \end{align*} \(34571 = 186^2 - 5^2 = (186 + 5)(186 - 5) = 191 \cdot 181\). Therefore, 34571 factors into 191 and 181.
    \end{enumerate}
}

