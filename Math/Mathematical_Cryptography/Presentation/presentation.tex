\documentclass{beamer}
% Option 1: Singapore theme - clean and modern
\usetheme{Singapore}
\usecolortheme{orchid}

% Option 2: Berlin theme - clear hierarchy
% \usetheme{Berlin}
% \usecolortheme{beaver}

% Option 3: Copenhagen theme - professional
% \usetheme{Copenhagen}
% \usecolortheme{rose}

% Option 4: Rochester theme - minimalist
% \usetheme{Rochester}
% \usecolortheme{crane}

% Option 5: Dresden theme - structured
% \usetheme{Dresden}
% \usecolortheme{spruce}

\usepackage{C:/Users/paulb/VSTeX/local/draculatheme}

\usepackage{bold-extra}
\usepackage{graphicx}  
\usepackage{tcolorbox}


\usepackage{pifont}
\usepackage{amsmath,amsthm} 
\usepackage{mathtools}
% \usepackage{arial}
\usepackage{amssymb}
\usepackage[framemethod=tikz]{mdframed}
\usepackage{tabularx}
\usepackage{mathrsfs}
\usepackage{multicol}
\usepackage{slashed}
\usepackage{braket}
\usepackage{enumerate}
\usepackage{enumitem}
\usepackage{lipsum} 
\usepackage{booktabs}   
\usepackage{ragged2e}
\usepackage{multirow}
\usepackage{caption}
\usepackage{listings}
% \usepackage{fourier}
% \usepackage{mathptmx}
\usepackage{helvet}
\definecolor{horange}{HTML}{f58026}
\hypersetup{
	colorlinks=true,
	linkcolor=violet,
	filecolor=violet,      
	urlcolor=violet,
}


%%% New Environments %%%
\newtheorem{exmp}{Example}[section]

%%% Custom Comands %%%

\newcommand{\cmark}{\ding{51}}%
\newcommand{\xmark}{\ding{55}}%

% Natural Numbers 
\newcommand{\N}{\ensuremath{\mathbb{N}}}

% Whole Numbers
\newcommand{\W}{\ensuremath{\mathbb{W}}}

% Integers
\newcommand{\Z}{\ensuremath{\mathbb{Z}}}

% Finite fields
\newcommand{\F}{\ensuremath{\mathbb{F}}}

% Rational Numbers
\newcommand{\Q}{\ensuremath{\mathbb{Q}}}

% Real Numbers
\newcommand{\R}{\ensuremath{\mathbb{R}}}

% Complex Numbers
\newcommand{\C}{\ensuremath{\mathbb{C}}}

\newcommand{\pfs}{\noindent\makebox[\linewidth]{\rule{\textwidth}{0.4pt}}\vspace{0.5cm}}
\definecolor{orangehdx}{rgb}{0.96, 0.51, 0.16}

% Normal colors
\definecolor{xred}{HTML}{BD4242}
\definecolor{xblue}{HTML}{4268BD}
\definecolor{xgreen}{HTML}{52B256}
\definecolor{xpurple}{HTML}{7F52B2}
\definecolor{xorange}{HTML}{FD9337}
\definecolor{xdotted}{HTML}{999999}
\definecolor{xgray}{HTML}{777777}
\definecolor{xcyan}{HTML}{80F5DC}
\definecolor{xpink}{HTML}{F690EA}
\definecolor{xgrayblue}{HTML}{49B095}
\definecolor{xgraycyan}{HTML}{5AA1B9}

% Dark colors
\colorlet{xdarkred}{red!85!black}
\colorlet{xdarkblue}{xblue!85!black}
\colorlet{xdarkgreen}{xgreen!85!black}
\colorlet{xdarkpurple}{xpurple!85!black}
\colorlet{xdarkorange}{xorange!85!black}
\definecolor{xdarkcyan}{HTML}{008B8B}
\colorlet{xdarkgray}{xgray!85!black}

% Very dark colors
\colorlet{xverydarkblue}{xblue!50!black}

% Document-specific colors
\colorlet{normaltextcolor}{black}
\colorlet{figtextcolor}{xblue}

% Light colors
\colorlet{xlightblue}{xblue!85!white}

% Enumerated colors
\colorlet{xcol0}{black}
\colorlet{xcol1}{xred}
\colorlet{xcol2}{xblue}
\colorlet{xcol3}{xgreen}
\colorlet{xcol4}{xpurple}
\colorlet{xcol5}{xorange}
\colorlet{xcol6}{xcyan}
\colorlet{xcol7}{xpink!75!black}

% Blue-Purple (should just used colorbrewer...)
\definecolor{xrainbow0}{HTML}{e41a1c}
\definecolor{xrainbow1}{HTML}{a24057}
\definecolor{xrainbow2}{HTML}{606692}
\definecolor{xrainbow3}{HTML}{3a85a8}
\definecolor{xrainbow4}{HTML}{42977e}
\definecolor{xrainbow5}{HTML}{4aaa54}
\definecolor{xrainbow6}{HTML}{629363}
\definecolor{xrainbow7}{HTML}{7e6e85}
\definecolor{xrainbow8}{HTML}{9c509b}
\definecolor{xrainbow9}{HTML}{c4625d}
\definecolor{xrainbow10}{HTML}{eb751f}
\definecolor{xrainbow11}{HTML}{ff9709}
\usepackage[T1]{fontenc} 







\usepackage{tikz}
\usetikzlibrary{shadows,shapes,backgrounds,calc,patterns}

\title{Shor's Algorithm \& Quantum Cryptography}

\author[Beggs, Hill] % (optional, use only with lots of authors)
{P.~Beggs \and J.~Hill}
% - Give the names in the same order as the appear in the paper.
% - Use the \inst{?} command only if the authors have different
%   affiliation.

\institute%
{
  Department of Mathematics and Computer Science\\
  Hendrix College}



\date{\today}
\setbeamercolor{normal text}{fg=draculafg, bg=draculabg}
% \setbeamercolor{alerted text}{fg=draculared}
% \setbeamercolor{example text}{fg=draculagreen}
\setbeamercolor{structure}{fg=draculapurple}
% \setbeamercolor{background canvas}{bg=draculabg}
% \setbeamercolor{frametitle}{fg=draculafg, bg=draculacl}
% \setbeamercolor{title}{fg=draculafg, bg=draculacl}
\setbeamerfont{title}{size=\huge, series=\bfseries}
% \setbeamerfont{frametitle}{size=\Large, series=\bfseries}




\begin{document}

\begin{frame}
    \titlepage
\end{frame}

%% -----------------------------------------------------------%%

% Section 1: Introduction
% Subsection 1.1: Background

%% -----------------------------------------------------------%%
\begin{frame}{Introduction}
	\framesubtitle{Background}
		Recall that: 
		\begin{itemize}
		\item Discrete Logarithm Problem (DLP): the problem of finding \(x\) given \(g^x\mod p\).
		\pause
		\item Order \(r\): smallest integer \(r \geq 1\) such that \(x^r \equiv 1 \mod n\)
		\pause
		\item Time complexity: DLP = \(\mathcal{O}(2^{n})\).
	\end{itemize}
\end{frame}

\begin{frame}{Introduction}
    \framesubtitle{subtitle}
        
\end{frame}


%% -----------------------------------------------------------%%

% Section 2: Classical Computers
% Subsection 2.1: Time Complexity

%% -----------------------------------------------------------%%
\begin{frame}
	\frametitle{Time Complexity}
	\framesubtitle{Classical Computers}
	The fastest known algorithm (number field sieve) has time complexity \(L_{4096}[\frac{1}{3},c]\) to decrypt a 4096-bit key-size DLP.
	For total steps, solve:
	\[
		L_{p}\left[\frac{1}{3},c\right] = \exp\left(c(\ln p)^{1/3}(\ln\ln p)^{2/3}\right).\footnote{Menezes, A. J., Van Oorschot, P. C., \& Vanstone, S. A. (2018). \textit{Handbook of applied cryptography.} CRC press.
		}
	\]
	\pause
	For \(c = (64/9)^{1/3} \approx 1.923\) and \(p = 4096\), the total steps would be \(10^{155}\). 
\end{frame}


%% -----------------------------------------------------------%%

% Section 3: Quantum Computers
% Subsection 3.1: Time Complexity

%% -----------------------------------------------------------%%
\begin{frame}{Time Complexity}
	\framesubtitle{Quantum Computers}
	Using Shor's Algorithm, the DLP can be solved in polynomial time using a quantum computer. For key size 4096, it would take:
	\pause
	\[
		\mathcal{O}\left((\log p)^{3}\right) = \mathcal{O}\left((\log 4096)^{3}\right) = 6.8 \cdot 10^{10} \text{ steps.}
	\]
\end{frame}


%% -----------------------------------------------------------%%

% Subsection idk: Overview of Classical & Quantum Computing

%% -----------------------------------------------------------%%
\begin{frame}{Bits}
	\framesubtitle{Overview of Classical \& Quantum Computing}
    \begin{itemize}
        \item Classical bits: 0 or 1
        \item[] Bits are manipulated according to \textbf{Boolean logic}, and sequences of bits are manipulated by \textbf{Boolean logic gates}.
        \item Quantum bits (qubits): Simultaneous values between 0 and 1
        \item[]  A quantum computer manipulates \textbf{quantum bits} (qubits) via \textbf{quantum logic gates}, which are supposed to simulate the laws of quantum mechanics \end{itemize}
\end{frame}


%% -----------------------------------------------------------%%

% Subsection 3.2 Understanding Qubits

%% -----------------------------------------------------------%%
\begin{frame}{Quantum Computers}
	\framesubtitle{Understanding Qubits}
    \begin{itemize}
        \item Two-state representation: \(\ket{0}\) and \(\ket{1}\)
        \item Pure states: \(\alpha \ket{0} + \beta \ket{1}\)
        \item Constraint: \(|\alpha|^2 + |\beta|^2 = 1\)
    \end{itemize}
    \pause
    \begin{block}{n-Component System}
        \[ \sum_{i = 0}^{2^n - 1} \alpha_i \ket{s_i}, \quad \text{where } \sum |\alpha_i|^2 = 1 \]
    \end{block}
\end{frame}


%% -----------------------------------------------------------%%

% Section 3: Shor's Algorithm
% Subsection 3.1: Overview

%% -----------------------------------------------------------%%
\begin{frame}{Shor's Algorithm}
	\framesubtitle{Overview}
    \begin{itemize}
        \item Purpose: Find non-trivial factors \(p\) and \(q\) of \(N\)
        \item Applications:
            \begin{itemize}
                \item Integer factorization
                \item Discrete logarithm in \(\mathbb{F}_p^*\)
                \item Elliptic curve discrete logarithm
            \end{itemize}
        \item Runs in polynomial time (quantum)
    \end{itemize}
\end{frame}


%% -----------------------------------------------------------%%

% Subsection 3.2: Overview of Algorithmic Steps

%% -----------------------------------------------------------%%
\begin{frame}{Shor's Algorithm}
	\framesubtitle{Algorithmic Steps}
	\begin{itemize}
		\item[1.] Change problem of factoring into finding the order \(r\).
		\pause
		\item[2.] Use the Quantum Fourier Transform to extract periodicity of \(f(x) = a^x \mod N\).
		\pause
		\item[3.] Once \(r\) is found (and it is even), use it in the computation of \(\gcd(a^{r/2} - 1, N)\).   
	\end{itemize}
	
\end{frame}


%% -----------------------------------------------------------%%

% Subsection 3.3: Quantum Fourier Transform

%% -----------------------------------------------------------%%
\begin{frame}{Shor's Algorithm}
	\framesubtitle{Quantum Fourier Transform 1}
    \begin{block}{Quantum Superposition}
        For \(0 < a < q\):
        \[ \frac{1}{q^{1/2}} \sum_{c = 0}^{q - 1} \ket{c} \exp(2\pi i a c /q) \]
    \end{block}
    \begin{itemize}
        \item Choose \(q\): power of 2 between \(n^2\) and \(2n^2\)
        \item Probability of state \(\ket{c}\) is high when:
        \[ \left| c - \frac{d}{r} \right| < \frac{1}{2q} \]
    \end{itemize}
\end{frame}


%% -----------------------------------------------------------%%


%% -----------------------------------------------------------%%

% Subsection 3.3: Quantum Fourier Transform

%% -----------------------------------------------------------%%
\begin{frame}{Shor's Algorithm}
	\framesubtitle{Quantum Fourier Transform 2}
    
\end{frame}


%% -----------------------------------------------------------%%

% Subsection 3.4: The Algorithm in Detail

%% -----------------------------------------------------------%%
\begin{frame}{Shor's Algorithm}
	\framesubtitle{Algorithmic Steps in Detail}
	\begin{itemize}
		\item[1.] Choose random integer \(a < N\) (to ensure \(a\) and \(N\) are co-prime).
		\pause
		\item[2.] Check if \(a\) is already a factor of \(N\). If so, then the problem is solved.
		\pause
		\item[3.] Otherwise, find the order \(r\) of \(a \mod N\) using quantum super-positioning and interference. (Remember that the order is the smallest integer such that \(a^r \equiv 1 \mod N\).)
		\pause
		\item[4.] Once \(r\) is found, compute the factors of \(N\) using \(r\). If \(r\) is even and 
		\[
			a^{r/2} \not\equiv -1 \mod N,
		\] 
		then the factors are 
		\[
			p = \gcd(a^{r/2} - 1, N) \qquad \text{ and } \qquad q = \gcd(a^{r/2} + 1, N).
		\] 
	\end{itemize}
	
\end{frame}


%% -----------------------------------------------------------%%

% Section 4: Implications

%% -----------------------------------------------------------%%
\begin{frame}{Cryptographic Implications}
    \begin{columns}
        \begin{column}{0.5\textwidth}
            \textbf{Vulnerable}
            \begin{itemize}
                \item RSA
                \item Classical Elgamal
                \item Elliptic curve Elgamal
            \end{itemize}
        \end{column}
		\pause
        \begin{column}{0.5\textwidth}
            \textbf{Still Secure}
            \begin{itemize}
                \item Lattice-based cryptosystems
                \item Shortest vector problems
                \item Closest vector problems
            \end{itemize}
        \end{column}
    \end{columns}
\end{frame}


%% -----------------------------------------------------------%%

% Section 5: Future Considerations

%% -----------------------------------------------------------%%
\begin{frame}{Challenges \& Future}
    \begin{itemize}
        \item Building functioning quantum computers
        \item Decoherence control
        \item Quantum cryptography applications:
            \begin{itemize}
                \item Heisenberg uncertainty principle
                \item Entanglement of quantum states
                \item Secure key exchange
            \end{itemize}
    \end{itemize}
\end{frame}

\end{document}