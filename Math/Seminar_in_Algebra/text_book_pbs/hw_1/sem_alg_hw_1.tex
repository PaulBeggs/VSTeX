%-%-%-%-%-%-%-%-%-%-%-%-%-%-%-%-%-%-%-%-%-%-%-%-%-%-%-%-%-%-%-%-%-%-%-%-%-%-%
%%% MAIN DOCUMENT %%%

%-%-%-%-%-%-%-%-%-%-%-%-%-%-%-%-%-%-%-%-%-%-%-%-%-%-%-%-%-%-%-%-%-%-%-%-%-%-%

\documentclass[12pt, oneside]{book}
\usepackage{xcolor}
\usepackage{minted}

% \usepackage[T1]{fontenc}

% \usepackage{draculatheme}
% \newcommand{\documentTheme}{draculafg}
% \newcommand{\documentLogo}{../../images/small logo_white.png}
% \newcommand{\documentBigLogo}{../../images/logo_white_text.png}
% \usemintedstyle{dracula}

\newcommand{\documentBigLogo}{../../images/Hendrix Logo.png}
\newcommand{\documentLogo}{../../images/small logo.png}
\newcommand{\documentTheme}{black}

%%% AESTHETICS %%%
%-%-%-%-%-%-%-%-%-%-%-%-%-%-%-%-%-%-%-%-%-%-%-%-%-%-%-%-%-%-%-%-%-%-%-%-%-%-%


%%% Dimensions and Spacing %%%
\usepackage[margin=1in]{geometry}
\setlength{\parindent}{0pt}
\usepackage{setspace}
\usepackage{mathtools}
\usepackage{esint}
\usepackage{adjustbox}
\linespread{1}
\usepackage{listings}
\usepackage{tikz}
\usetikzlibrary{shapes,backgrounds,calc,patterns,positioning, arrows.meta, decorations.pathmorphing, decorations.pathreplacing}
\usepgflibrary{shadings}
\usepackage{pgfkeys}
\usepackage{pdftexcmds}
\usepackage{shellesc}
\usepackage{algorithm, algorithmic, xspace}
%%% Define new colors %%%
\definecolor{orangehdx}{rgb}{0.96, 0.51, 0.16}

% Normal colors
\definecolor{xred}{HTML}{BD4242}
\definecolor{xblue}{HTML}{4268BD}
\definecolor{xgreen}{HTML}{52B256}
\definecolor{xpurple}{HTML}{7F52B2}
\definecolor{xorange}{HTML}{FD9337}
\definecolor{xdotted}{HTML}{999999}
\definecolor{xgray}{HTML}{777777}
\definecolor{xcyan}{HTML}{80F5DC}
\definecolor{xpink}{HTML}{F690EA}
\definecolor{xgrayblue}{HTML}{49B095}
\definecolor{xgraycyan}{HTML}{5AA1B9}

% Dark colors
\colorlet{xdarkred}{red!85!black}
\colorlet{xdarkblue}{xblue!85!black}
\colorlet{xdarkgreen}{xgreen!85!black}
\colorlet{xdarkpurple}{xpurple!85!black}
\colorlet{xdarkorange}{xorange!85!black}
\definecolor{xdarkcyan}{HTML}{008B8B}
\colorlet{xdarkgray}{xgray!85!black}

% Very dark colors
\colorlet{xverydarkblue}{xblue!50!black}

% Document-specific colors
\colorlet{normaltextcolor}{black}
\colorlet{figtextcolor}{xblue}

% Enumerated colors
\colorlet{xcol0}{black}
\colorlet{xcol1}{xred}
\colorlet{xcol2}{xblue}
\colorlet{xcol3}{xgreen}
\colorlet{xcol4}{xpurple}
\colorlet{xcol5}{xorange}
\colorlet{xcol6}{xcyan}
\colorlet{xcol7}{xpink!75!black}

% Blue-Purple (should just used colorbrewer...)
\definecolor{xrainbow0}{HTML}{e41a1c}
\definecolor{xrainbow1}{HTML}{a24057}
\definecolor{xrainbow2}{HTML}{606692}
\definecolor{xrainbow3}{HTML}{3a85a8}
\definecolor{xrainbow4}{HTML}{42977e}
\definecolor{xrainbow5}{HTML}{4aaa54}
\definecolor{xrainbow6}{HTML}{629363}
\definecolor{xrainbow7}{HTML}{7e6e85}
\definecolor{xrainbow8}{HTML}{9c509b}
\definecolor{xrainbow9}{HTML}{c4625d}
\definecolor{xrainbow10}{HTML}{eb751f}
\definecolor{xrainbow11}{HTML}{ff9709}

%------- %
% XHFILL %
%------- %



%%% Chapter Headings %%%

\newcommand{\gradientrule}{
    \begin{tikzpicture}
        \shade[left color=orangehdx, right color=black, middle color=gray] (0,0) rectangle (\linewidth,0.4pt);
    \end{tikzpicture}
}

\usepackage[Glenn]{fncychap}
\ChTitleVar{\bfseries\scshape\color{\documentTheme}} % Needed for Dracula theme
\ChNumVar{\large\selectfont\color{\documentTheme}} % Needed for Dracula theme
\ChNameVar{\large\color{\documentTheme}} % Needed for Dracula theme
\usepackage{xpatch}

% \xpatchcmd{\DOCH}
%   {\mghrulefill}{\gradientrule\mghrulefill}
%   {}{\PatchFailed}
% \xpatchcmd{\DOTI}
%   {\mghrulefill}{\gradientrule\mghrulefill}
%   {}{\PatchFailed}
% \xpatchcmd{\DOTIS}
%   {\mghrulefill}{\gradientrule\mghrulefill}
%   {}{\PatchFailed}

\xpatchcmd\DOCH
{\mghrulefill}{\color{orangehdx}\mghrulefill}
{}{\PatchFailed}
\xpatchcmd\DOTI
{\mghrulefill}{\color{orangehdx}\mghrulefill}
{}{\PatchFailed}
\xpatchcmd\DOTIS
{\mghrulefill}{\color{orangehdx}\mghrulefill}
{}{\PatchFailed}


% \usepackage[Bjornstrup]{fncychap}

% \newcommand{\gradient}[1]{
% \begin{tikzpicture}
%     \node (rect) at (0,0) [fill=blue,,path fading=East,minimum width=\linewidth,minimum height=2.5cm] {};
%     \node(title)[above left = 10pt and 10pt of rect.south east, anchor=south east, font=\CTV] {\textcolor{horange}{#1}};
%     \ifnum \thechapter>0\node[left = 10pt of rect.north east,  anchor=center, font=\CNoV] {\textcolor{horange}{\thechapter}};\fi%
% \end{tikzpicture}%
% \vskip 40pt
% }


% \renewcommand{\DOCH}{}
% \renewcommand{\DOTI}[1]{\gradient{#1}}
% \renewcommand{\DOTIS}[1]{\gradient{#1}}

%% Change Chapter Heading Placement %%
\usepackage{etoolbox}
\makeatletter
\patchcmd{\@makechapterhead}{\vspace*{50\p@}}{\vspace*{-20\p@}}{}{}
\patchcmd{\@makeschapterhead}{\vspace*{50\p@}}{\vspace*{-20\p@}}{}{}
\patchcmd{\DOTI}{\vskip 80\p@}{\vskip 40\p@}{}{}
\patchcmd{\DOTIS}{\vskip 40\p@}{\vskip 0\p@}{}{}
\makeatother



% \newcommand*\chapterlabel{}\pmod
% \titleformat{\chapter}
%   {\gdef\chapterlabel{}
%    \normalfont\sffamily\Huge\bfseries\scshape}
%   {\gdef\chapterlabel{\thechapter\ }}{0pt}
%   {\begin{tikzpicture}[remember picture,overlay]
%     \node[yshift=-3cm] at (current page.north west)
%       {\begin{tikzpicture}[remember picture, overlay]
%         \draw[fill=LightSkyBlue] (0,0) rectangle
%           (\paperwidth,3cm);
%         \node[anchor=east,xshift=.9\paperwidth,rectangle,
%               sharp corners=downhill=20pt,inner sep=11pt,
%               fill=MidnightBlue]
%               {\color{white}\chapterlabel#1};
%        \end{tikzpicture}
%       };
%    \end{tikzpicture}
%   }
% \titlespacing*{\chapter}{0pt}{50pt}{-60pt}

% \usepackage[Conny]{fncychap}
% \usepackage[Rejne]{fncychap}
% \ChNameVar{\bfseries}  % Makes the chapter "Chapter #" bold
% \ChNumVar{\bfseries}   % Makes the chapter number bold
% \ChTitleVar{\bfseries} % Makes the chapter title bold
% % Define a custom color, for example:
% \definecolor{mycolor}{RGB}{0,128,255}

% % Redefine the chapter style in Rejne to change the line colors
% \makeatletter
% \ChRuleWidth{2pt}   % Change the thickness of the lines
% \renewcommand{\DOCH}{%
%   \vspace*{-50\p@}% Moves the chapter title up/down if needed
%   {\color{mycolor} \hrule \@chapapp{} \space \thechapter \hrule}% Customizes the chapter header with color
% }
% \renewcommand{\DOTI}[1]{%
%   \vskip 20\p@ % Adjusts the space above the title
%   \bfseries #1\par % Embolden the title
%   \vskip 20\p@ % Adjusts the space below the title
% }
% \makeatother

% %% Change Chapter Heading Placement %%
% \usepackage{etoolbox}
% \makeatletter
% \patchcmd{\@makechapterhead}{\vspace*{50\p@}}{\vspace*{-20\p@}}{}{}
% \patchcmd{\@makeschapterhead}{\vspace*{50\p@}}{\vspace*{-20\p@}}{}{}
% \patchcmd{\DOTI}{\vskip 80\p@}{\vskip 40\p@}{}{}
% \patchcmd{\DOTIS}{\vskip 40\p@}{\vskip 0\p@}{}{}
% \makeatother

\renewcommand{\thesection}{\thechapter.\arabic{section}} %% Chapter.Section Numbering


%%% FIGURES %%%
\usepackage{graphicx}  
% \numberwithin{figure}{section}
\usepackage{float}
\usepackage{caption}

%%% Hyperlinks %%%
\usepackage{hyperref}
\definecolor{horange}{HTML}{f58026}
\hypersetup{
	colorlinks=true,
	linkcolor=horange,
	filecolor=horange,      
	urlcolor=horange,
}


%% Headers and Footers %%
\usepackage{fancyhdr} % This should be set AFTER setting up the page geometry
\pagestyle{fancy} % options: empty , plain , fancy
\fancyhead[R]{\textcolor{\documentTheme}{\assignmentname}}
\fancyhead[L]{\textcolor{\documentTheme}{Hendrix College}}
\fancyhead[C]{\includegraphics[height=.50cm]{\documentLogo}} % Use for Dracula
\usepackage{xpatch}
\xpretocmd\headrule{\color{orangehdx}}{}{\PatchFailed}
\setlength{\footskip}{0.5in}
\setlength{\headheight}{18.5764pt}


%%%%% Colored Boxes %%%%%
\usepackage{tcolorbox}
\tcbuselibrary{skins}
\tcbuselibrary{theorems}
% \tcbuselibrary{minted}
\newcounter{BoxCounter}
% \usepackage{dingbat}

%-%-%-%-%-%-%-%-%-%-%-%-%-%-%-%-%-%-%-%-%-%-%-%-%-%-%-%-%-%-%-%-%-%-%-%-%-%-%

%% MATH PACKAGES, ENVIRONMENTS, COMMANDS %%
%-%-%-%-%-%-%-%-%-%-%-%-%-%-%-%-%-%-%-%-%-%-%-%-%-%-%-%-%-%-%-%-%-%-%-%-%-%-%
%You'll need your own packages, theorem types, and commands.

\usepackage{fix-cm}
\usepackage{amsmath,amsthm, amsfonts, amssymb} 
\usepackage{array, makecell}
\usepackage{colortbl}
\newcommand{\thickvrule}{\vrule width 1pt}
\newcommand{\thickhrule}{\Xhline{1pt}}

\usepackage{mathtools}
\usepackage[framemethod=tikz]{mdframed}
\usepackage{mathrsfs}
\usepackage{changepage}
\usepackage{multicol}
\usepackage{slashed}
\usepackage{enumerate}
\usepackage{braket}
\usepackage{booktabs}
\usepackage{enumitem}
\usepackage{kantlipsum}  %This package lets us generate random text for example purposes.
\usepackage{pgfplots}


%%% Custom Commands %%%
% Natural Numbers 
\newcommand{\N}{\mathbb{N}}

% Whole Numbers
\newcommand{\W}{\mathbb{W}}

% Integers
\newcommand{\Z}{\mathbb{Z}}

% Rational Numbers
\newcommand{\Q}{\mathbb{Q}}

% Real Numbers
\newcommand{\R}{\mathbb{R}}

% Complex Numbers
\newcommand{\C}{\mathbb{C}}

\newcommand{\I}{\mathbb{I}}

\newcommand{\pfs}{\noindent\makebox[\linewidth]{\rule{\textwidth}{0.4pt}}\vspace{0.5cm}}

\newcommand{\mysqrt}[1]{%
	\mathpalette\foo{#1}%
}
\newcommand{\dmysqrt}[1]{%
	\mathpalette\foodisplay{#1}%
}

% !TeX spellcheck = off
\newcommand{\foo}[2]{%
	% #1: math style, #2: content
	\sbox0{$#1\sqrt{#2}$}% Measure the size of the standard sqrt in the current style
	\begin{tikzpicture}[baseline=(sqrt.base)]
		\node[inner sep=0, outer sep=0] (sqrt) {$#1\sqrt{#2}$}; % Use the current math style
		\draw([yshift=-0.045em]sqrt.north east) -- ++(0,-0.5ex); % Draw the tick
	\end{tikzpicture}%
}
% !TeX spellcheck = off
\newcommand{\foodisplay}[2]{%
	% #1: math style, #2: content
	\sbox0{$#1\sqrt{#2}$}% Measure the size of the standard sqrt in the current style
	\begin{tikzpicture}[baseline=(sqrt.base)]
		\node[inner sep=0, outer sep=0] (sqrt) {$\displaystyle\sqrt{#2}$}; % Force displaystyle
		\draw[line width=0.4pt] ([yshift=-0.044em]sqrt.north east) -- ++(0,-0.5ex); % Draw the tick
	\end{tikzpicture}%
}

\renewcommand{\theenumi}{\arabic{enumi}}
\renewcommand{\labelenumi}{\textbf{\theenumi}.}

% \newcommand{}{\marginnote{\includegraphics[width=2em]{caution.png}}}

\newmdenv[
	topline=false,
	bottomline=true,
	rightline=false,
	leftline=true,
	linewidth=1.5pt,
	linecolor=black, % default color, will be overridden in custom commands
	% backgroundcolor=draculabg, % Needed for Dracula theme
	% fontcolor=\documentTheme, % Needed for Dracula theme
	innertopmargin=0pt,
	innerbottommargin=5pt,
	innerrightmargin=10pt,
	innerleftmargin=10pt,
	leftmargin=0pt,
	rightmargin=0pt,
	skipabove=\topsep,
	skipbelow=\topsep,
]{customframedproof}

\newenvironment{proofpart}[2][black]{
    \begin{mdframed}[
        topline=false,
        bottomline=false,
        rightline=false,
        leftline=true,
        linewidth=1pt,
        linecolor=#1!40, % Custom color
        % innertopmargin=10pt,
        % innerbottommargin=10pt,
        innerleftmargin=10pt,
        innerrightmargin=10pt,
        leftmargin=0pt,
        rightmargin=0pt,
        % skipabove=\topsep,
        % skipbelow=\topsep%
    ]
    \noindent
    \begin{minipage}[t]{0.08\textwidth}%
        \textbf{#2}%
    \end{minipage}%
    \begin{minipage}[t]{0.90\textwidth}%
        \begin{adjustwidth}{0pt}{0pt}%
}{
    \end{adjustwidth}
    \end{minipage}
    \end{mdframed}
}

\newenvironment{tipbox}
	{%
		\par\noindent%
		% Top dashed line (using \textwidth)
		\tikz[baseline]{\draw[dash pattern=on 3pt off 2pt, line width=0.5pt, color=draculapurple] (0,0) -- (\textwidth,0);}%
		\par\vspace{0.65em}%
		\noindent\textbf{\textcolor{draculapurple}{TIP:}}\quad%
	}
	{%
		\par%
		% Bottom dashed line
		\noindent%
		\tikz[baseline]{\draw[dash pattern=on 3pt off 2pt, line width=0.5pt, color=draculapurple] (0,0) -- (\textwidth,0);}%
		\par\vspace{0.35em}
	}

\newenvironment{notebox}
	{%
		\par\noindent%
		% Top dashed line (using \textwidth)
		\tikz[baseline]{\draw[dash pattern=on 3pt off 2pt, line width=0.5pt, color=draculagreen] (0,0) -- (\textwidth,0);}%
		\par\vspace{0.65em}%
		\noindent\textbf{\textcolor{draculagreen}{NOTE:}}\quad%
	}
	{%
		\par%
		% Bottom dashed line
		\noindent%
		\tikz[baseline]{\draw[dash pattern=on 3pt off 2pt, line width=0.5pt, color=draculagreen] (0,0) -- (\textwidth,0);}%
		\par\vspace{0.35em}
	}

\newenvironment{warningbox}
	{%
		\par\noindent%
		% Top dashed line (using \textwidth)
		\tikz[baseline]{\draw[dash pattern=on 3pt off 2pt, line width=0.5pt, color=draculared] (0,0) -- (\textwidth,0);}%
		\par\vspace{0.65em}%
		\noindent\textbf{\textcolor{draculared}{WARNING:}}\quad%
	}
	{%
		\par%
		% Bottom dashed line
		\noindent%
		\tikz[baseline]{\draw[dash pattern=on 3pt off 2pt, line width=0.5pt, color=draculared] (0,0) -- (\textwidth,0);}%
		\par\vspace{0.35em}
	}

\newenvironment{solution}{\textit{Solution.}}

\newcommand{\sol}[1]{
    \begin{customframedproof}[linecolor=orangehdx!75,]
        \begin{solution}
        #1
        \end{solution}
    \end{customframedproof}
}

\newcommand{\pf}[1]{
    \begin{customframedproof}[linecolor=orangehdx!75]
        \begin{proof}
        #1
        \end{proof}
    \end{customframedproof}
}

\def \proofDistance {10pt}

\newcommand{\disabs}[1]{\ensuremath{\left|#1\right|}}
\newcommand{\limn}{\lim_{n \rightarrow \infty}}
\newcommand{\limx}[2]{\lim_{x \rightarrow #1}#2}

\newcommand{\limc}[1]{\lim_{x \rightarrow c}#1}

\newcommand{\ninn}{n \in \N}

\newcommand{\mb}[1]{\mathbf{#1}}

\newcommand{\limsupn}[1]{\limsup_{n\rightarrow \infty}#1}
\newcommand{\liminfn}[1]{\liminf_{n\rightarrow \infty}#1}


\newcommand{\proj}{\text{proj}}

\newcommand{\p}{\partial}

\newcommand{\dydx}{\frac{dy}{dx}}
\newcommand{\dxdy}{\frac{dx}{dy}}
\newcommand{\dydt}{\frac{dy}{dt}}
\newcommand{\dxdt}{\frac{dx}{dt}}
\newcommand{\dzdt}{\frac{dz}{dt}}

\newcommand{\barNotationT}[1]{\bigg|_{t = #1}}

\newcommand{\cyanit}[1]{\textit{\textcolor{cyan}{#1}}}

\newcommand{\brackett}[1]{\left\langle #1 \right\rangle}

\newcommand{\norm}[1]{\left\lVert \mathbf{#1}\right\rVert}

\newcommand{\imb}{\mb{i}}
\newcommand{\jmb}{\mb{j}}
\newcommand{\kmb}{\mb{k}}
\newcommand{\rmb}{\mb{r}}
\newcommand{\umb}{\mb{u}}

\newcommand{\vecfuc}[2]{\mb{#1}(#2)}
\newcommand{\dvecfuc}[2]{\mb{#1}'(#2)}
\newcommand{\normdvecfuc}[2]{\|\mb{#1}'(#2)\|}

\newcommand{\funccolon}{\, \colon}

\newcommand{\squigglyline}{%
    \noindent
    \tikz[baseline=-0.5ex]{
        \draw[decorate, decoration={snake, amplitude=0.5mm, segment length=3mm}] 
        (0,0) -- (\dimexpr\linewidth\relax,0);
    }%
}

\newcommand{\gen}[1]{\langle #1 \rangle}
\newcommand{\abs}[1]{\left| #1 \right|}
\newcommand{\generator}[1]{\langle #1 \rangle}


% end of preamble
%-%-%-%-%-%-%-%-%-%-%-%-%-%-%-%-%-%-%-%-%-%-%-%-%-%-%-%-%-%-%-%-%-%-%-%-%-%-%

\begin{document}

\numberwithin{BoxCounter}{section}


%-%-%-%-%-%-%-%-%-%-%-%-%-%-%-%-%-%-%-%-%-%-%-%-%-%-%-%-%-%-%-%-%-%-%-%-%-%-%
%%% COVER PAGE %%%
%-%-%-%-%-%-%-%-%-%-%-%-%-%-%-%-%-%-%-%-%-%-%-%-%-%-%-%-%-%-%-%-%-%-%-%-%-%-%

\newcommand{\assignmentname}{Homework 1: Sections 18 \& 19}

% Do not use all caps.
\newcommand{\cussubtitle}{Seminar in Algebra}
% Date format should be like: "January 1, 2001"
\newcommand{\finaldate}{February 12, 2025}
% Be sure to include degree recognition (e.g., B.S., M.S., Ph.D)
\newcommand{\professor}{Dr. Carol Ann Downes, Ph.D.}

%-%-%-%-%-%-%-%-%-%-%-%-%-%-%-%-%-%-%-%-%-%-%-%-%-%-%-%-%



\begin{titlepage}
    \begin{center}

        \vspace*{-2cm}
        \includegraphics[width=0.8\textwidth]{\documentBigLogo}\\
        \vfill

        % Horizontal line above the title in 'horange' color
        \textcolor{horange}{\rule{\textwidth}{1.0pt}}

        \vspace{2em}

        {\huge \textbf{\assignmentname}}

        \vspace{1em} % Space between the title and the bottom line

        \textcolor{horange}{\rule{\textwidth}{1.0pt}}

        \vspace*{1\baselineskip}

        {\LARGE \textbf{\cussubtitle}}

        \begin{large}
            \vspace*{5\baselineskip}

            \vspace*{1\baselineskip}

            \emph{Author} \\[1ex]
            %Submitted by \\[\baselineskip]
            {\Large Paul Beggs \\ \par} % Editor list
            {\href{mailto:BeggsPA@Hendrix.edu}{{BeggsPA@Hendrix.edu}}}\\ % Editor affiliation

            \vspace*{1\baselineskip}

            \textit{Instructor} \\[1ex] % Tagline(s) or further description
            \professor

            \vspace*{1\baselineskip}

            \textit{Due}\\[1ex]
            {\scshape  \finaldate} \\[0.3\baselineskip] % Year published

            \thispagestyle{empty}

        \end{large}
    \end{center}
\end{titlepage}
% -%-%-%-%-%-%-%-%-%-%-%-%-%-%-%-%-%-%-%-%-%-%-%-%-%-%-%-%-%-%-%-%-%-%-%-%-%-%


% IV.18: 5, 16, 18, 19, 32, (read statement of 42), 43, 44, 46, 50 
% IV.19: 4, 18, 23, 26
\section*{Section 18}

In Exercise 5, compute the product in the given ring.
\begin{enumerate}
    \setcounter{enumi}{4}
    \item \((2,3)(3,5)\) in \(\Z_{5} \times \Z_{9}\)
    \sol{
        \[
            (2,3)(3,5) = (6, 15) = (6 \bmod 5, 15 \bmod 9) = (1, 6).
        \]
    }    
\end{enumerate}

In Exercises 16, 18, and 19, describe all units in the given ring.

\begin{enumerate}
    \setcounter{enumi}{15}
    \item \(\Z_{5}\)
    \sol{
        Every element in \(\Z_{5}\) is a unit (except 0), as each of them are relatively prime to 5.
    }
    \setcounter{enumi}{17}
    \item \(\Z \times \Q \times \Z\)
    \sol{
        The units in \(\Z\) are \(\{1, -1\}\), the units in \(\Q\) are \(\Q \setminus \{0\}\), and the units in the second \(\Z\) are also \(\{1, -1\}\). Thus, the units in \(\Z \times \Q \times \Z\) are of the form \((\pm 1, q, \pm 1)\) where \(q \in \Q \setminus \{0\}\).
    }
    \item \(\Z_{4}\)
    \sol{
        The only relatively prime elements to 4 in \(\Z_{4}\) are 1 and 3, so they are the units.
    }
\end{enumerate}

\textbf{Concepts}

\begin{enumerate}
    \setcounter{enumi}{31}
    \item Given an example of a ring with unity \(1 \ne 0\) that has a subring with nonzero unity \(1' \ne 1\). [Hint: Consider a direct product or a subring of \(\Z_{6}\)]
    \sol{
        Consider the ring \(\Z_{6}\). The element \(1\) is the unity of \(\Z_{6}\). However, the subset \(\{0, 3\}\) forms a subring of \(\Z_{6}\) with unity \(1' = 3\) because \(3 \cdot 3 \equiv 3\) and \(3 \cdot 0 \equiv 0\).
    }
\end{enumerate}

\newpage
\textbf{Theory}

\begin{enumerate}
    \setcounter{enumi}{43}
    \item Show that the multiplicative inverse of a unit in a ring with unity is unique.
    \pf{
        Let \(u\) be a unit in a ring \(R\) with unity, and suppose \(v\) and \(w\) are both multiplicative inverses of \(u\). Then we have \(uv =vu = 1\) and \(uw = wu = 1\). Multiply the equation \(uv = 1\) on the left by \(w\):
        \begin{align*}
            w(uv) &= w \cdot 1 \\
            (wu)v &= w & \text{associativity \& identity property,}\\
            1 \cdot v &= w & \text{since } wu = 1, \\
            v &= w & \text{identity property.}
        \end{align*}
        Thus, the multiplicative inverse of a unit in a ring with unity is unique.
    }
    \item An element of \(a\) of a ring \(R\) is \textbf{idempotent} if \(a^{2} = a\).
    \begin{enumerate}[label=\textbf{\alph*.}]
        \item Show that the set of all idempotent elements of a commutative ring is closed under multiplication.
        \pf{
            Let \(a\) and \(b\) be idempotent elements in a commutative ring \(R\). Then we have \(a^{2} = a\) and \(b^{2} = b\). We want to show that the product \(ab\) is also idempotent, i.e., \((ab)^{2} = ab\).
            \begin{align*}
                (ab)^{2} &= (ab)(ab) \\
                &= a(b(ab)) & \text{associativity,}\\
                &= a((ab)b) & \text{since } R \text{ is commutative,}\\
                &= (a^{2}b)b & \text{associativity,}\\
                &= (ab)b & \text{since } a^{2} = a,\\
                &= a(b^{2}) & \text{associativity,}\\
                &= ab & \text{since } b^{2} = b.
            \end{align*}
            Thus, the set of all idempotent elements of a commutative ring is closed under multiplication.
        }

        \item Find all idempotents in the ring \(\Z_{6} \times \Z_{12}\).
        \sol{
            Using a Python script and exhaustively searching \(\Z_{6}\) and \(\Z_{12}\), we get the individual idempotents of 0, 1, 3, and 4 for \(\Z_{6}\) and 0, 1, 4, and 9 for \(\Z_{12}\). Taking a combination of each element, we get 16 idempotent sets.  
        }
    \end{enumerate}
    \newpage
    \setcounter{enumi}{45}
    \item (Linear algebra) Recall that for an \(m \times n\) matrix \(A\), the \textit{transpose} \(A^{T}\) of \(A\) is the matrix whose \(j\)th column is the \(j\)th row of \(A\). Show that if \(A\) is an \(m \times n\) matrix such that \(A^{T}A\) is invertible, then the \textit{projection matrix} \(P = A(A^{T}A)^{-1}A^{T}\) is an idempotent in the ring of \(n \times n\) matrices.
    \pf{
        We will show that \(P\) is an idempotent by computing \(P^{2}\):
        \begin{align*}
            P^{2} &= [A(A^{T}A)^{-1}A^{T}][A(A^{T}A)^{-1}A^{T}] \\
            &= A[(A^{T}A)^{-1}(A^{T}A)][(A^{T}A)^{-1}A^{T}] & \text{matrix associativity,} \\
            &= A(I_{A})[(A^{T}A)^{-1}A^{T}] & \text{matrix inverses,} \\
            &= A[(A^{T}A)^{-1}A^{T}] & \text{matrix identity,} \\
            &= A(A^{T}A)^{-1}A^{T} & \text{matrix associativity,} \\
            &= P & \text{definition of } P.
        \end{align*}
        Thus, we have shown that \(P^{2} = P\). Therefore, \(P\) is an idempotent.
    }
    \setcounter{enumi}{49}
    \item Let \(R\) be a ring, and let \(a\) be a fixed element of \(R\). Let \(I_{a} = \{x \in R \mid ax = 0\}\). Show that \(I_{a}\) is a subring of \(R\). 
    \pf{
        To prove that \(I_{a}\) is a subring of \(R\), we need to prove the statements given by Exercise 48 (\& from theorem in notes):
        \begin{enumerate}[label=(\arabic*)]
            \item \(0\in I_a\) because \(a\cdot 0=0\).
            \item If \(b,c\in I_a\) then 
            \begin{align*}
                a(b-c) &=ab-ac & \text{left distributive laws from } R, \\
                &=0-0 & \text{definition of } I_{a}, \\
                &=0 & \text{inverse property from } R.
            \end{align*}
            Thus, \(b-c\in I_a\).
            \item If \(b,c\in I_a\) then 
            \begin{align*}
                a(bc) &=(ab)c & \text{associativity from } R, \\
                &=0\cdot c & \text{definition of } I_{a}, \\
                &=0 & \text{definition of } I_{a}.
            \end{align*}
            Thus, \(bc\in I_a\).
        \end{enumerate}
        Therefore, \(I_a\) is a subring of \(R\).
    }
\end{enumerate}

\newpage

\section*{Section 19}

\begin{enumerate}
    \setcounter{enumi}{3}
    \item Find all solutions of \(x^{2} + 2x + 4 = 0\) in \(\Z_{6}\).
    \sol{
        Using a Python script to search through values \(x \in [0,5]\), we find the only solution to the equation is \(x = 2\).
    }

\end{enumerate}

\textbf{Concepts}

\begin{enumerate}
    \setcounter{enumi}{17}
    \item Each of the six numbered regions in Fig. 19.10 corresponds to a certain type of ring. Give an example of a ring in each of the six cells. For example, a ring in the region numbered 3 must be commutative (it is inside the commutative circle), have unity, but not be an integral domain. 
    \sol{
        Starting from 6 and working inward, we have: 6. \(M_{2}(2\Z)\); 5. \(M_{2}(\R)\); 4. \(2\Z\); 3. \(\Z_{6}\); 2. \(\Z_{2}\); 1. \(\R\).  
    }

\end{enumerate}

\textbf{Theory}

\begin{enumerate}
    \setcounter{enumi}{22}
    \item An element \(a\) of a ring \(R\) is \textbf{idempotent} if \(a^{2} = a\). Show that a division ring contains exactly two idempotent elements.
    \pf{
        Let $R$ be a division ring and let $a \in R$ be an idempotent element, so $a^2 = a$. First, note that $0^2 = 0$, so $0$ is an idempotent element. Now, suppose $a \neq 0$. Since $R$ is a division ring, $a$ is a unit, so $a^{-1}$ exists. 
        Rearranging $a^2 = a$ gives $a^2 - a = 0$, which factors as $a(a - 1) = 0$. 
        We can solve for $a - 1$ as follows:
        \begin{align*}
            a(a - 1) &= 0 \\
            a^{-1}(a(a - 1)) &= a^{-1} \cdot 0 & \text{left multiply by } a^{-1}, \\
            (a^{-1}a)(a - 1) &= 0 & \text{associativity,} \\
            1(a - 1) &= 0 & \text{inverse property,} \\
            a - 1 &= 0 & \text{identity property,} \\
            a &= 1 & \text{add 1 to both sides.}
        \end{align*}
        Thus, the only nonzero idempotent element is 1. Therefore, there are exactly two idempotent elements.
    }
\newpage

    \setcounter{enumi}{25}
    \item Let \(R\) be a ring that contains at least two elements. Suppose for each nonzero \(a \in R\), there exists a unique \(b \in R\) such that \(aba = a\).
    \begin{enumerate}[label=\textbf{\alph*.}]
        \item Show that \(R\) has no divisors of 0.
        \sol{
            Let \(a,c \in R\) with \(a \ne 0\) such that \(ac = 0\). Additionally, assume there exists a unique \(b \in R\) such that \(aba = a\). Our goal is to show that \(b + c = b\), implying \(c = 0\). Hence, consider the following computation:
            \begin{align*}
                a(b + c)a &= aba + aca & \text{left and right distribution,} \\
                &= a + aca & aba = a \text{ property,} \\
                &= a + (ac)a & \text{associativity,} \\
                &= a + 0a & ac = 0 \text{ property,} \\
                &= a & \text{Theorem 18.8 (1) \& add. id.}  
            \end{align*}
            Thus, we have shown that \((b + c)\) behaves exactly like \(b\), meaning \(b + c = b\) because \(b\) is unique, so \(c = 0\).
        }
        \item Show that \(bab = b\).
        \sol{
            To show that \(bab = b\), we are going to leverage the theorem in the notes that states, ``The cancellation laws hold in ring \(R\) if, and only if, \(R\) has no zero divisors.'' This allows us to compute the following:
            \begin{align*}
                aba &= a \\
                baba &= ba & \text{left multiplication,} \\
                bab &= b & \text{cancellation laws for } a.
            \end{align*}
        }
        \item Show that \(R\) has unity.
        \sol{
            Given the results from (b) and \(R\)'s \(aba = a\) property, we conjecture that \(ab = 1\) (i.e., \(ab\) is \(R\)'s unity). To prove this conjecture, we must show that for an element \(c \in R\), that \(c(ab) = (ab)c = c\). Thus, consider the following computation:
            \begin{align*}
                aba &= a \\
                caba &= ca & \text{left multiplication by } c, \\
                c(ab) &= c & \text{cancellation laws \& associativity.}
            \end{align*}
            Now, for the other direction:
            \begin{align*}
                bab &= b \\
                babc &= bc & \text{right multiplication by } c, \\
                (ab)c &= c & \text{cancellation laws \& associativity.}
            \end{align*}
            Thus, we have shown that \(R\)'s unity is \(ab\).
        }
        \item Show that \(R\) is a division ring.
        \sol{
            From part (c), we know $R$ has a unity, $1$, and for any nonzero $a$, $ab = 1$.
            From part (b), we have $bab = b$. Multiplying by $a$ on the right gives $baba = ba$. Substituting $aba=a$, we get $ba = 1$.
            Thus, for every $a$, there exists $b \in R$ such that $ab = ba = 1$. This means every nonzero element is a unit.
            Since $R$ is a ring with unity and every nonzero element has an inverse, $R$ is a division ring.
        }
    \end{enumerate}
\end{enumerate}

%-%-%-%-%-%-%-%-%-%-%-%-%-%-%-%-%-%-%-%-%-%-%-%-%-%-%-%-%-%-%-%-%-%-%-%-%-%-%
\end{document}
%-%-%-%-%-%-%-%-%-%-%-%-%-%-%-%-%-%-%-%-%-%-%-%-%-%-%-%-%-%-%-%-%-%-%-%-%-%-%

%-%-%-%-%-%-%-%-%-%-%-%-%-%-%-%-%-%-%-%-%-%-%-%-%-%-%-%-%-%-%-%-%-%-%-%-%-%-%