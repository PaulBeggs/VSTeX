\renewcommand{\theenumi}{\arabic{enumi}}
\renewcommand{\labelenumi}{\theenumi.}
\section{Undefined Terms}

Zermelo Set Axioms (1908) -- Set Theory. 
Set theory is the basis of all mathematics. 

In Mathematics, there are terms that cannot be defined. They should be seen as building blocks. Further, these terms simply are (same as it is in Geometry with terms like ``line,'' ``point,'' etc.).

\begin{center}
    \begin{tabular}{c|c}
        \textbf{Undefined Terms} & \textbf{Synonyms} \\
        \hline
        --Set                    & ``Collection''    \\
        \hline
        --An Element Of          & ``A Member Of''   \\
        \hline
        --Equal To               & ``Equals''        
    \end{tabular}
\end{center}



\begin{note}
    Sets: Uppercase letters, \((A, B)\);
    Elements (elts): lowercase letters, \((a, b)\);
    to say \(x\) is an elt. of set A, we write ``\(x \in A\).''
\end{note}

\pfs

\renewcommand{\theenumi}{\arabic{enumi}}
\renewcommand{\labelenumi}{\theenumi.}
\section{Subsets}

Definitions in mathematics serve as foundational pillars upon which proofs and arguments are constructed, akin to roadmaps guiding the logical progression towards proving statements. Thus, when we want to show something like \( A \subseteq B \), where \( A \) and \( B \) are sets, we have to sate the criterion imposed by the definition. In doing so, we must demonstrate that every element of \( A \) is also an element of \( B \).

Generally, when we try to prove something, we begin by assuming a general statement; such as: \( x \) is an arbitrary element of \( A \). Then, by demonstrating that any such \( x \) necessarily belongs to \( B \), we can confirm that every element of \( A \) indeed resides within \( B \), fulfilling the criteria for \( A \subseteq B \).

To formally prove \( A \subseteq B \), we can use the \defref{Subset}: for all \( x \), if \( x \in A \), then \( x \in B \). This condition ensures that any arbitrary element \( x \) chosen from \( A \) must satisfy \( x \in B \), thereby establishing \( A \subseteq B \).

\begin{definition}
    {Subset}Suppose \(A\) and \(B\) are Sets. The statement: \(\text{``}A \text{ is in a \textit{subset} of } B\text{''}\) 
    Means: If \(x \in A\), \(x \in B\). Notated as \(A \subseteq B\)
\end{definition}



\begin{figure}[htbp]
    \centering
    \begin{tikzpicture}
        % Define the style for the diagonal lines pattern
        \tikzstyle{diagonal lines}=[pattern=north east lines, pattern color=gray];

        % Coordinate for A
        \coordinate (A) at (2,2);

        % Draw A 
        \draw (A) circle (1cm) node {\(A\)};

        % Draw B around A
        \draw (A) circle (2cm) node [below right=2em] {\(B\)};

        % Fill A to emphasize it is a subset of B
        \begin{scope}
            \clip (A) circle (1cm);
            \fill[diagonal lines] (0.5,0.5) rectangle (3.5,3.5);
        \end{scope}

    \end{tikzpicture}
    \caption{\textit{Set \(A\) as a subset of set \(B\).}}
\end{figure}

% ------------------------------------------------------------------------------
% Problem 1
% ------------------------------------------------------------------------------

It is important to note that each definition can be ``denied.'' That is, we can always write what it means to \textit{not} be the defined thing.

When we consider the statement, ``Each definition can be denied,'' this implies that for every mathematical definition, there exists a corresponding negation that describes what it means for an object or number not to meet the criteria specified by that definition. This negation is crucial because it outlines the conditions under which an object falls outside the defined category.

\begin{exercise}
    {Subset Denial}Write the denial of the definition of a subset. In other words, complete the statement: \[\text{``}A \text{ is not a subset of \(B\) means...}\]
\end{exercise}

\sol{
    ...there exists an \(x \in A\) such that \(x \notin B\).''
}

% ------------------------------------------------------------------------------
% Problem 2
% ------------------------------------------------------------------------------

\begin{exercise}
    {Subset Identity}Suppose \(A\) is a set. Then, \(A\subseteq A\).
\end{exercise}

\expf{
    Let \(A\) be a set. 

     If an element \(x\) is in \(A\), then \(x\in A\).

     Therefore, by the \defref{Subset}, \(A\subseteq A\).
}


% ------------------------------------------------------------------------------
% Problem 3
% ------------------------------------------------------------------------------

\section{Introduction to Contraposition}

\refex{3} gives us another ``road map'' (proof strategy) to prove a set \(A\) is a subset of \(B\). We can show this by following the logic of Exercise 3: take an arbitrary element \(y \notin B\) and show \(y \notin A\). This is known as the contrapositive of the conditional statement. Because the conditional statement and the contrapositive are \textit{logically equivalent}, if we can show that the contrapositive is true, then we have proven that the conditional statement must also be true.

\begin{exercise}
    {Contraposition}Suppose \(A\) and \(B\) are sets with \(A\subseteq B\). Also suppose \(y\notin B\). Then, \[y\notin A\]
\end{exercise}

\expf{
    Let \(A\) and \(B\) be sets such that \(A \subseteq B\) and \(y \notin B\). 

     Because \(A \subseteq B\), we know that any element \(y \in A\) implies \(y \in B\) by the \defref{Subset}. Hence, since \(y \notin B\), it must also be the case that \(y \notin A\). If \(y\) did belong to \(A\), it would imply that \(A\) cannot be a subset of \(B\), which contradicts the given information. 

     Therefore, for the given information to hold true, it must be the case that \(y \notin B\) implies \(y \notin A\).
}

% ------------------------------------------------------------------------------
% Problem 4
% ------------------------------------------------------------------------------

\begin{exercise}
    {Contraposition (cont.)}Suppose \(A\) and \(B\) are sets. Also suppose that for each \(y \notin B\) that \(y \notin A\). Then, \[A\subseteq B\]
\end{exercise}

\expf{
    Let \(A\) and \(B\) be sets. Also assume that for all \(y \notin B\) that \(y \notin A\). 

     Because we know \(y \notin B\) implies \(y \notin A\), it must be the case that if \(y \in A\), then \(y \in B\). Any other alternative would contradict the given information as shown: \[y \notin B \text{ implies } y \in A.\]

     Therefore, for the given information to hold true, it must be the case that \(x \in A\) implies \(x \in B\). Hence, \(A\subseteq B\) by the \defref{Subset}.
}

\newpage

% ------------------------------------------------------------------------------
% Problem 5
% ------------------------------------------------------------------------------

\begin{exercise}
    {Transitivity of Subsets}Suppose \(A, B, \text{and } C\) are sets with \(A\subseteq B\) and \(B \subseteq C\). Then, \[A\subseteq C\]
\end{exercise}

\expf{

    Let \(A,B\) and \(C\) be sets such that \(A\subseteq B\) and \(B\subseteq C\). 

     By examining the properties of the \defref{Subset}, we know element \(x \in A\) implies \(x \in B\), and \(y \in B\) implies \(y\in C\). Hence, let any element \(z \in A\). Because \(A \subseteq B\), we know \(z \in B\), and since \(B \subseteq C\), by the same logic, \(z \in C\). 

     Therefore, by the \defref{Subset}, \(A\subseteq C\).
}

\pfs

% ------------------------------------------------------------------------------
% Extensionality
% ------------------------------------------------------------------------------

\renewcommand{\theenumi}{\arabic{enumi}}
\renewcommand{\labelenumi}{\theenumi.}
\section{Implication and Equality}

\vspace*{\fill}

\begin{definition}
    {Extensionality}The symbol \(\implies\) denotes the logical implication. For statements \(P\) and \(Q\), \(P \implies Q\) is true if,
    \begin{align*}
        1. & \quad \text{Whenever } P \text{ is true, } Q \text{ must also be true, and} \\
        2. & \quad \text{If } P \text{ is false, the implication is considered true.}
    \end{align*}
\end{definition}

\begin{axiom}
    {Set Equality}Suppose \(A\) and \(B\) are sets. Then, set \(A\) is equal to set \(B\) if, and only if,
    \begin{align*}
        1. & \quad x \in A \implies x \in B, \\
        2. & \quad y \in B \implies y \in A.
    \end{align*}
\end{axiom}

\begin{note}
    \textbf{Note:} Intuitively, two sets are equal if they have the same elements. We write this as ``\(A = B\)'' for sets \(A\) and \(B\). 

    \hspace{0.5em}\textbf{Careful}! In any proof, make sure to use axioms, theorems, definitions, and previous problems. Do not include intuitive statements.
\end{note}


% ------------------------------------------------------------------------------
% Problem 6
% ------------------------------------------------------------------------------

\begin{exercise}
    {Set Identity Property}Suppose that \(A\) is a set. Then, \[A = A\]
\end{exercise}

\expf{
    Let \(A\) be a set. 

     Let \(x \in A\). From \refex{2}, we know that since \(A \subseteq A\), then \(x \in A\) implies \(x \in A\). 

     Therefore, we have shown that because an element in \(A\) implies that element exists in \(A\) -- and vice versa -- then by the \refax{Set Equality}, \(A = A\).
}

\pfs

% ------------------------------------------------------------------------------
% \namrefthm{Set Equality with Subsets}
% ------------------------------------------------------------------------------

\namrefthm{Set Equality with Subsets} introduces us to an \textit{``if, and only if''} statement (initialized as `iff'). These types of proofs are solved by showing each case of the forward (\(\Rightarrow\)) and backward (\(\Leftarrow\)) implications hold. Thus, in essentially one proof, we need to solve two proofs. We use the given information from the left (\(\Rightarrow\)) side of the ``if, and only if,'' and prove the right (\(\Leftarrow\)) side of the proof is true, and vice versa. Symbolically, we write this as ``\(\iff\).''

\vspace{0.5cm}

\begin{ntheorem}
    {Set Equality with Subsets}Suppose \(A\) and \(B\) are sets. Then, set \(A = B\) if, and only if, \(A\subseteq B\) and \(B\subseteq A\). 

    Forward Implication: If \(A = B\), then \(A\subseteq B\), and \(B\subseteq A\). 
    Backward Implication: If \(A \subseteq B\) and \(B\subseteq A\), then \(A = B\).
\end{ntheorem}

\iffpf
{Suppose \(A\) and \(B\) are sets with \(A = B\). Let \(x\in A\). Since \(A = B\), by the \refax{Set Equality}, we know that \(x\in B\). Therefore, by the \defref{Subset}, \(A\in B\). Let \(y\in B\). Since \(A = B\), by the \refax{Set Equality}, we know \(y\in A\). Therefore, \(B\subseteq A\) the \defref{Subset}.}
{Suppose \(A\) and \(B\) are sets with \(A\subseteq B\) and \(B\subseteq A\). Let \(x\in A\). Then, by the \defref{Subset}, \(x\in B\). Let \(y\in B\). Then, by the \defref{Subset} \(B\subseteq A\), we know \(y\in A\).}
{Therefore, by the \refax{Set Equality}, we know that \(A = B\).}
{xblue}

\newpage

\begin{exercise}
    {Transitivity of Equality}Suppose for the sets \(A, B, \text{ and } C\) such that \(A = B\) and \(B = C\). Then, \[A = C\]
\end{exercise}

\expf{
    Let \(A, B\) be sets such that \(A = B\). Additionally, for set \(C\), \(B = C\). 

     By \namrefthm{Set Equality with Subsets}, we know that these sets must be subsets of each other. That is, \(A\subseteq B\) and \(B\subseteq A\). The same concept from Set Equality with Subsets (\(\Rightarrow\)) applies here: \(C\subseteq B\) and \(B\subseteq C\). Combining the first and second statements, we see that since the sets are equal, then they must be subsets of each respective other. Thus, by the \refax{Set Equality}, \(x\in A\) implies \( x\in B\), \(y\in B\) implies \(y\in A\), and \(x\in B\) implies \(x\in C\), \(y\in C\) implies \(y\in B\). Then, by the \defref{Subset}, since \(x\in A\) implies \(x\in C\), then \(A\subseteq C\); and since \(y\in C\) implies \(y\in A\), then \(C\subseteq A\). 

     Therefore, using the backward implication of Set Equality with Subsets, given the previous statements, \(A\subseteq B, B\subseteq A, B\subseteq C, C\subseteq B\), \(A \subseteq C\) and \(C \subseteq A\) implies \(A = C\).
}


% ------------------------------------------------------------------------------
% Problem 8
% ------------------------------------------------------------------------------

\begin{exercise}
    {}Suppose \(A, B, \text{ and } C\) are sets such that \(A\subseteq B, \ B\subseteq C \text{, and } C\subseteq A\). Then, \[A = B \text{ and } B = C\]
\end{exercise}


\expf{
    Let \(A, B,\) and \(C\) be sets. Also, for each \(A, B \text{, and } C,\) let \(A\subseteq B \text{, and } B\subseteq C \text{, and } C\subseteq A\). 

     By \refex{5}, we know that \(A \subseteq C\), \(B \subseteq A\) and by rearranging the given information, we know that \(C \subseteq B\). 

     Therefore, by using \namrefthm{Set Equality with Subsets}, we can deduce further from this information that since \(A \subseteq C \text{, and } C \subseteq A\), then \(A = C\).
    If you apply the same logic for \(B \text{ and } C\), then, \(B = C\). 
}

\pfs

\begin{note}
    When we say a set such as \(A\) has elements \(a, b, \text{ and } c\), we write this notation wise as: \[A = \{a,b,c\}\]
    \textbf{Note}: when dealing with sets, the arrangement of elements and repetition are not considered. Sets are defined solely by membership; if two sets have the same elements, they are considered equal, regardless of the order in which the elements are listed.
\end{note}





% ------------------------------------------------------------------------------
% Problem 9
% ------------------------------------------------------------------------------

\begin{exercise}
    {Duplicate Property of Elements}Suppose \(A\) is a set, \(x\in A\), and \(A=\{x\}\). Then,
    \[A = \{x,x\}\]
\end{exercise}

\expf{
    Let \(A\) be a set with \(x\in A\) and \(A = \{x\}\). 

     Let any \(x \in A\). 

     Therefore, \(A = \{x\} = \{x,x\}\)
}


% ------------------------------------------------------------------------------
% Problem 10
% ------------------------------------------------------------------------------

\begin{exercise}
    {Commutativity of Elements}Suppose set \(A = \{a,b\}.\) Then,
    \[A = \{b,a\}\]
\end{exercise}

\expf{
    Let \(A\) be a set with \(A = \{a,b\}\). 

     Another way of writing \(A = \{a,b\}\) would be to say \(a,b \in A\). The natural order of these elements being added is arbitrary. Hence, we can also write this as \(b,a \in A\), which would be equivalent to the first set. 

     Therefore, \(A = \{a,b\} = \{b,a\}\)
}