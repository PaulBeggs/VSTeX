\renewcommand{\theenumi}{\arabic{enumi}}
\renewcommand{\labelenumi}{\theenumi.}
\section{Introduction}
 
 The concept of \textit{Specification} is essential for understanding how to define subsets of a given set based on specific criteria. The axiom presented below formally encapsulates this idea by stating that for any set \(A\) and any logical sentence \(s(x)\) applicable to its elements, there exists a subset of \(A\) containing exactly those elements for which \(s(x)\) is true. This subset is denoted using \textit{set-builder notation}, which is a compact and powerful way to specify sets.

\textit{Set-builder notation} can be expressed in two equivalent forms: \(\{x \in A \ | \ s(x)\}\) and \(\{x \in A \colon s(x)\}\). Both notations serve the same purpose but use different symbols to convey the same idea.

    \textbf{Vertical Bar (\(|\))}: When we see the vertical bar (\(|\)) in set-builder notation, it is read as ``such that." Therefore, \(\{x \in A \ | \ s(x)\}\) is read as ``the set of all \(x\) in \(A\) such that \(s(x)\) is true." The vertical bar acts as a separator between the variable \(x \in A\) and the condition \(s(x)\).

    \textbf{Colon (:)}: The colon (:) serves the same function as the vertical bar in set-builder notation. Thus, \(\{x \in A \colon s(x)\}\) is also read as ``the set of all \(x\) in \(A\) such that \(s(x)\) is true." The colon is a stylistic alternative to the vertical bar and is equally valid in expressing the condition that defines the subset.

\begin{axiom}
    {Implication}Suppose \(A\) is a set and \(s\) is any logical sentence which applies to the elements of \(A\). Then, there is a set whose elements are exactly the elements of \(A\) for which \(s(x)\) is true.
\end{axiom}

The following definition is going to be the most important cornerstone of this course. The empty set, denoted by \(\emptyset\), is a set that contains no elements (refer to \refex{11}). The concept of the empty set is fundamental in set theory and will be used extensively throughout the rest of the content.
    
\begin{definition}
    {Empty Set}Suppose \(A\) is a set. Then the \textit{empty set} of \(A\) is the set \(\{x\in A \ | \ x\ne x\}\). We denote this \textit{empty set} as \(\emptyset_A\).
\end{definition}


The empty set is unique in that it is the only set with no elements. It serves as the basis for many important properties and operations in set theory. For example, the empty set is a subset of every set, and the intersection of any set with the empty set is the empty set itself. Additionally, the union of the empty set with any set \(A\) is just \(A\).

% ------------------------------------------------------------------------------
% Placeholder 11
% ------------------------------------------------------------------------------

    \begin{exercise}
        {}Suppose \(A\) is a set. Then, \[x\notin \emptyset_A\]
    \end{exercise}

    \expf{
        Let \(A\) be a set. \\
        
        \indent Assume an element \((x\in \emptyset_A)\). By \defref{Empty Set} of \(A\), \(x\in \emptyset_A\) implies \(\{x\in A \colon x\ne x\}\). However, because of the implication that \(x\in A\) while also satisfying \(x \ne x\) leads to a contradiction, our assumption must be false. \\
        
        \indent Therefore, by Definition 3, \((x \notin \emptyset_A\)).
    }


% ------------------------------------------------------------------------------
% Placeholder 12
% ------------------------------------------------------------------------------

\begin{note}
    \refex{12}shows that all empty sets are equal. So, we can use \(\emptyset\) to refer to any empty set without worrying about which initial set it was built from. \\
    
    \textbf{Proof Strategy}: We want to show: \(x\in \emptyset_A \Rightarrow x\in \emptyset_B\) AND \(y \in \emptyset_B \Rightarrow y \in \emptyset_A\) (proof by Axiom 1.3.2). But this is impossible because \refex{11} showed that there is no \(x\in \emptyset_A\). Hence, if we utilize the contra-positive, we can show that if \(x\notin \emptyset_B\) then \( x\notin \emptyset_A\).
\end{note}

\begin{exercise}
    {Universality}Suppose \(A\) and \(B\) are sets. Then, \[\emptyset_A = \emptyset_B\]
    \end{exercise}


\expf{

    Let \(A\) and \(B\) be sets. \\
    
    \indent For the sake of reason, assume the case that \(\emptyset_A \ne \emptyset_B\). We proved in \refex{11} that for any element \(x\notin \emptyset_A\), and also \(x\notin \emptyset_B\). But if the case is that \(\emptyset_A \ne \emptyset_B\), then there must exist an element \(x\in \emptyset_A\) such that \(x \notin \emptyset_B\). However, as stated earlier, this leads to a contradiction because there cannot be any element x in \(\emptyset_A\). \\
    
    \indent Therefore, by using using vacuous truth, \(\emptyset_A = \emptyset_B\).
}

% ------------------------------------------------------------------------------
% Placeholder 13
% ------------------------------------------------------------------------------

    \begin{exercise}
        {Subset Specification}Suppose \(A\) is a set and \(s\) is a logical sentence which applies to elements of \(A\). Then, \[B = \{x\in A \ | \ s(x)\} \subseteq A\]
    \end{exercise}
        
    \expf{
        Let \(A\) be a set with \(s\) being a logical sentence that applies to the elements of \(A\). \\
        
        \indent Let \(B\) be a set such that its elements are exactly the elements of \(A\) (implicating that \(B\subseteq A\)). Let \(y \in B\). By definition of B, this implies that \(y\) is an element in \(A\). \\
        
        \indent Therefore \(B = \{x\in A \ | \ s(x)\} \subseteq A \).
    }
    
    \begin{corollary}
        {Empty Set Subsets}{\refex{13}}Suppose A is a set. Then, \[\emptyset \subseteq A.\]
    \end{corollary}

% ------------------------------------------------------------------------------

\vspace*{\fill}

\pfs

\renewcommand{\theenumi}{\arabic{enumi}}
\renewcommand{\labelenumi}{\theenumi.}
\section{Existence}

\vspace*{\fill}

    The following Axiom 2.2.1 is original to Zernelo. Modern `axionatization' of set theory does not include it. However, we have only proven the existence of \(\emptyset\) (which hinges on only if other sets exist).

    \begin{axiom}
        {Existence pt. 1}There exists a set.
    \end{axiom}

    \vspace{0.5cm}
    
    \begin{axiom}
        {Existence pt. 2}Suppose \(A\) and \(B\) are sets. Then, \[\text{There exists a set } C = \{A,B\}\]
        \textbf{Careful}! The elements of \(C\) are the sets \(A \text{ and } B\), \textbf{not} the elements of \(A \text{ and } B\) themselves.
    \end{axiom}

% ------------------------------------------------------------------------------
% Placeholder 14
% ------------------------------------------------------------------------------

\newpage

    \refex{14} in combination with \refax{Existence pt. 2}  displays the capability for set production. In essence, we could make an infinite amount of sets. (This will be useful when we get to numbers later in the course!) \\
    
    To show this, consider the following sequence of steps: Since \(\emptyset\) exists, by the \refax{Existence pt. 2}, we know \(\{\emptyset,\emptyset\}\) exists. And by \refex{9}, this set is \(\{\emptyset\}\). Further, by \refex{14}, this set is distinct from \(\emptyset\). Therefore, it is possible to continue to produce distinct new sets like \(\emptyset \text{ and } \{\emptyset\}\).

    \begin{exercise}
        {Empty Set Identity}Show that \(\emptyset \ne \{\emptyset\}\).
    \end{exercise}

    \expf{
        \indent By \defref{Empty Set}, \(\{x\in A \ | \ x\ne x\}\), where \(A\) is an arbitrary set (proven in \refex{12}). \\
        
        \indent Thus, the empty set has no elements (as shown in \refex{11}). However, \(\{\emptyset\}\) \textit{does} have an element in it -- the empty set itself. Hence, because the elements inside of the sets are different, they cannot be equal by the denial of equality. \\
        
        \indent Therefore, \(\emptyset \ne \{\emptyset\}\).
    }

% ------------------------------------------------------------------------------