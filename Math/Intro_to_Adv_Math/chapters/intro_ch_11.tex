    
        \section*{Motivation}
            We want to answer the following questions: \\
            How big are sets? E.g., \(\{a,b,c\}\) has size 3, but what about \(\W\)? Or \(\Z, \Q, \R\). \\
            And what does infinity mean? \\
            We need a way to count/measure the size of a set.

            \begin{definition}
                {Cardinality}Say \(A\) and \(B\) are sets. If there exists a bijection, \(f \ \colon A \rightarrow B\), then we say \(A\) and \(B\) have the same \textit{cardinality}.

                Denote cardinality of set \(A\) by \(|A|\). 
                \begin{itemize}
                    \item Define \(|\emptyset| = 0\) (Cardinality 0) 
                    \item Define for \(n\in \W\), \(|n| = n\). 
                \end{itemize}
            \end{definition}
            \textbf{Think}: \(n = \{0,1,\dots, n-1\}\). \\
            \textbf{Note}: For the take-home exam 2, we proved that having the same cardinality is an equivalence relation on sets. So we can create equivalence classes, each containing sets of the same cardinality.

        \section{Problems}

% ------------------------------------------------------------------------------
% Problem 63
% ------------------------------------------------------------------------------

            \begin{exercise}
                {}Explicitly show that \(|\{a,b,c\}| = 3\) using the definition, and assume \(a,b,c\) are all distinct.
            \end{exercise}

            \expf{ 
                Suppose \(f\) is the relation defined by the ordered pairs \((a,0)\), \((b,1)\), and \((c,2)\) that maps the set \(A = \{a,b,c\}\) to \(B = \{0,1,2\}\). The cardinality of \(B\) corresponds with the whole number three, so to also show that \(A\)'s cardinality is 3, we need to show that a bijection, \(f\), exists. Thus, to show \(f\) is a bijection, we need to show the following: \\
                \begin{itemize}
                    \item \(f\) is a function. \\
                    Since \(f\) is defined such that: 
                    \begin{itemize}
                        \item \(a\) maps to \(0\),
                        \item \(b\) maps to \(1\),
                        \item \(c\) maps to \(2\),
                    \end{itemize}
                    each element in \(A\) is associated with one and only one element in \(B\). Thus, \(f\) is a function by definition.
                    \item \(f \text{'s domain is all of } A\). \\
                    The domain of \(f\) is all of \(A\) because every element \(a,b,c \in A\) has a mapping in \(B\). There is no element in \(A\) that is left unattended to, so \(f\) covers the whole domain.  
                    \item \(f\)'s injective (each \(b\) is mapped to by only one \(a\)). \\
                    As shown in the first point, each element of \(A\) maps to a unique element of \(B\) (no two elements of \(A\) share the same output in \(B\)). This confirms that \(f\) is injective. 
                    \item \(f\)'s surjective (\(\text{range} = B\)). \\
                    Since \(0,1,2 \in B\) are each mapped to by \(a,b,c \in A\) respectively, \(f\) covers all elements of \(B\). Thus confirming that \(f\) is surjective.
                \end{itemize}
                Therefore, because we have shown that \(f\) is a function, its domain is all of \(A\), its injective, and its surjective, \(f\) must be a bijection.
            }

% ------------------------------------------------------------------------------
% Problem 64
% ------------------------------------------------------------------------------

            \begin{exercise}
                {}Suppose \(A\) is a set for which there is \(n\in \W\) where \(|A| = n\). Induct on \(n\) to show that, \[|\mathcal{P}(A)| = 2^n\]
            \end{exercise}
    
            \expf{
                Suppose \(S := \{n\in \W \ | \ \text{if } |A| = n, \text{ then } |\mathcal{P}(A)| = 2^n\}\). We we must show that \(S = \W\), and to do so, we must show that \(S\) is a successor set. 
                \begin{itemize}
                    \item \textbf{Base Case}: \\
                    For a set \(A\) where \(|A| = 0\), (i.e., \(A = \emptyset\) by definition), the power set \(\mathcal{P}(\emptyset)\) is just one subset -- that being the set of the empty set, by Example 1 of Week 5. Thus, \(|P(A)| = |P(\emptyset)| = |\{\emptyset\}| = 1\) by Problem 63 (because the set of the empty set is not equal to the empty set by Problem 14). Since \(2^0 = 1\), \(0 \in S\). \\
                    
                    \item \textbf{Inductive Hypothesis}: \\
                    Suppose for some \(k \in \W\), that \(k \in S\). In other words, if \(|A| = k\), then \(|\mathcal{P}(A)| = 2^k\). \\
                    
                    \item \textbf{Inductive Step}: \\
                    We need to show that \(k + 1 \in S\). \\
                    
                    Suppose \(A\) is a set such that \(|A| = k\). Then, consider the immediate successor of \(A^+ := A \cup \{A\}\). Thus, the set \(A^+\) contains all elements of \(A\) plus the set \(A\) itself as an additional element by the corollary to Axiom 5 (union). Therefore, the cardinality of \(A^+\) is \(k + 1\) by Problem 63.  \\

                    Consider a single subset \(S\) in \(A\). We know that by the definition of subset, an element in that subset implies there is an element in \(A\). Hence, because \(A^+\) contains \(A\) (and therefore its subsets), all subsets of \(A\) form a subset of \(A^+\) by definition of subset. Additionally, for each subset \(S\) of \(A\), there is a corresponding subset \(S \cup \{A\} \subseteq A^+\). Further, by Axiom 6, we know that since \(S \cup \{A\} \subseteq A^+\), then \(S \cup \{A\} \in \mathcal{P}(A^+)\). This means for every subset of \(A\), you get exactly two subsets in \(\mathcal{P}(A^+)\). \\

                    Since every subset \(S\) of \(A\) results in two distinct subsets in \(\mathcal{P}(A^+)\), the total number of subsets in \(\mathcal{P}(A^+)\) doubles. This can be represented as \(|\mathcal{P}(A^+)| = 2 \times |P(A)|\) because each subset of \(A\) contributes exactly two subsets to \(\mathcal{P}(A^+)\). By using the inductive hypothesis, we see that \(|\mathcal{P}(A)| = 2^k\), so \(|\mathcal{P}(A^+)| = 2 \times 2^k = 2^{k+1}\).
                \end{itemize}
                Thus, \(n^+ \in S\). Because \(S\) contains 0, and whenever \(S\) contains \(n\), \(S\) also contains \(n^+\), \(S\) is by definition, a successor set. Thus, by Peano's ``Axioms,'' \(S = \W\).
            }

% ------------------------------------------------------------------------------
% Problem 65
% ------------------------------------------------------------------------------

            \begin{exercise}
                {Swapping Lemma} Suppose \(f \ \colon A \rightarrow B\) is a bijection, and suppose \(a_1,a_2 \in A\) are distinct such that \(f(a_1) = b_1\) and \(f(a_2) = b_2\). Then, \[g \ \colon A \rightarrow B \text{ defined by } 
                \begin{cases}
                    b_2 &\text { if } x = a_1 \\
                    b_1 &\text { if } x = a_2 \\
                    f(x) &\text{ otherwise} 
                \end{cases}, \text{ is also a bijection.}\]
            \end{exercise}
    
            \expf{
                Suppose \(f \ \colon A \rightarrow B\) is a bijection, and suppose \(a_1,a_2\in A\) are distinct such that \(f(a_1) = b_1\) and \(f(a_2) = b_2\). To show \(g\) is a bijection, we need to show the following:
                \begin{itemize}
                    \item \(g\) is a function. \\
                    We know that \(g\) is a function because for every element \(x\in A\), \(g(x)\) corresponds to exactly one other element. Specifically, \((a_1,b_2) \in g\), \((a_2,b_1) \in g\), and \(g(x) = f(x)\) for all \(x \in A \ \backslash \ \{a_1,a_2\}\). \\
                    
                    \item \(g \text{'s domain is all of } A\). \\
                    The function \(g\) is defined for every element \(x \in A\) because we know for the distinct \(a_1\) and \(a_2\), \(g\) assigns the values \(b_2, b_1 \in B\) respectively. This assures that for these specific elements, they are covered. Furthermore, for all other elements in \(A\) (which do not include \(a_1, a_2\)), \(g\) adopts the values assigned by \(f\); and since \(f\) is a bijection, it already provides a unique output in \(B\) for each of those elements. \\
                    
                    \item \(g\)'s injective (each \(b\) is mapped to by only one \(a\)). \\
                    Since \(g\) is a piecewise function, we need to show for each case that for all \(b \in B\), \(g\) maps to only one \(a\). Thus, consider \(x,y\in A\) such that \(g(x) = g(y)\). We need to show that \(x = y\). \\
                    
                    \noindent If \(x\) and \(y\) are neither \(a_1\) nor \(a_2\), then \(g(x) = f(x)\) and \(g(y) = f(y)\). Since \(f\) is injective, it must be the case that \(x = y\). \\

                    \noindent If \(x\) or \(y\) are either \(a_1\) and the other is \(a_2\), then \(g(x)\) and \(g(y)\) would not be equal to each other because one would map to distinct \(b_2\), and the other would map to distinct \(b_1\). Therefore supporting the argument that \(g\) is injective. \\

                    \noindent If \(x = a_1\) and \(y \ne a_2\) (or \(x = a_2\) and \(y \ne a_1\)) the outputs from \(g(x)\) and \(g(y)\) would differ. That is, if \(x = a_1\), then \(g(x) = b_2\), and if \(y \ne a_2\), \(g(y) = f(y)\) which would have the same mapping (see paragraph 2). Since \(b_2 = f(a_2)\) and \(f\) is injective, \(f(y) \ne b_2\) unless \(y = a_2\), which is not the case. \\

                    \noindent Therefore, the only possibility for \(g(x) = g(y)\) is when \(x = y\), thus confirming that \(g\) is injective for all elements in \(A\). \\
                    
                    \item \(g\)'s surjective (\(\text{range} = B\)). \\
                    Similarly to injectivity, we need to show for each case that for every element in \(B\), there exists at least one element in \(A\) such that \(g(x) = b\). \\ 
                    
                    For the third case, since \(f\) is a bijection from \(A\) to \(B\), each element of \(B\) is already paired uniquely with an element of \(A\). Hence, we only show that \(g\) is surjective for the first two cases. \\

                    For these other cases, we know that \(g\) maps an \(x\) to \(b\) because it is defined as such. Hence because we have covered all elements in \(B\) with at least one element in \(A\), \(g\) must be surjective.
                \end{itemize} 
                Therefore, because we have shown that \(g\) is a function, its domain is all of \(A\), its injective, and its surjective, \(g\) must be a bijection.
            }

% ------------------------------------------------------------------------------
% Problem 66
% ------------------------------------------------------------------------------

            \begin{exercise}
                {}Suppose \(a,b \in \W\). Then, there exists a bijection between \(a = \{0,1,\dots,a - 1\}\) and \(b = \{0,1,\dots,b - 1\}\) if, and only if, \(a = b\).
            \end{exercise}
            \textbf{Hint}:\\
            (\(\Rightarrow\)) Use swapping Lemma \\
            (\(\Leftarrow\)) create bijection (can use exam 2 problem)
    
            \expf{
                (\(\Rightarrow\)) \\ 
                \noindent Suppose \(a\) is defined as \(\{0,1,\dots,a-1\}\) and \(b = \{0,1,\dots,b-1\}\). Then suppose there exists a bijection between \(a\) and \(b\). That is, \\
                    
                \noindent \textit{Proof}. (\(\Leftarrow\)) \\
                Because we know \(a\) and \(b\) are whole numbers, and we know there exists a bijection between them, then we know that \(|a|=|b|\). Thus, we know \(|a| = a\), and \(|b|=b\), so \(a = b\).\footnote{This insight is attributed to a remark from Tanvi Keran.}
            }

            \begin{definition}
                {Natural Numbers}The set of \textit{natural numbers}, denoted \(\N\), is the set \(\W \backslash \{0\} = \{1,2,3,\dots\}\). We call the cardinality of \(\N\): \(|\N| := \ale\) (read ``aleph naught'').
            \end{definition}

        \section{Problem}

% ------------------------------------------------------------------------------
% Problem 67
% ------------------------------------------------------------------------------

            \begin{exercise}
                {Cardinality I}Show \(|\W| = \ale\).
            \end{exercise}
            \textbf{Note}: Feels a little bad because adding an element to a set didn't change the size: \(\ale + 1 = \ale\).

            \expf{
                Consider a relation, \(f\) defined as \(f \ \colon \N \rightarrow \W\) by \(f(n) = n - 1\). To show that \(\W\) has the same cardinality as \(\N\) (\(\ale\)), we need to prove that \(f\) is a bijection. Thus, consider the following points: \\
                \begin{itemize}
                    \item \(f\) is a function: \\
                    Let \((a,b_1),(a,b_2) \in f\), then \(b_1 = a - 1 = b_2\) by definition of \(f\).
                    \item \(f\) has \(\N\) as domain: \\
                    Note that for all \(n\in \N\), \(f(n) = n - 1\) so \(n\) is in the domain of \(f\). 
                    \item \(f\) is injective: \\
                    Let \((a_1,b),(a_2,b)\in f\), then \(a_1 - 1 = b = a_2 - 1\) by definition of whole number addition. Thus, \(a_1 = a_2\).
                    \item \(f\) is surjective: \\
                    Let 
                \end{itemize}
            }

        \begin{example}
            Show \(2\N = \ale\) where \(2\N := \{2,4,6,8,\dots\}\)
        \end{example}

        \sol{
            Take out every other element yet of the same size: \(1/2\ale = \ale\). Define \(f \ \colon \N \rightarrow 2\N\) by \(f(n) = 2n\). We'll show \(f\) is a bijection. 
            \begin{enumerate}
                \item \(f\) is a function: \\
                Let \((a,b_1),(a,b_2) \in f\), then \(b_1 = 2a = b_2\) by definition of \(f\).
                \item \(f\) has \(\N\) as domain: \\
                Note that for all \(n\in \N\), \(f(n) = 2n\) so \(n\) is in the domain of \(f\). 
                \item \(f\) is injective: \\
                Let \((a_1,b),(a_2,b)\in f\), then \(2a_1 = b = 2a_2\). Thus, \(a_1 = a_2\), and \(f\) is injective.
                \item \(f\) is surjective: \\
                Let \(b \in 2\N\). Thus, there is \(n\in \N\) such that \(b = 2n\). Then, \(f(n) = b\) with \(n \in \N\), so \(f\) is surjective.
            \end{enumerate}
        }

        \textbf{Weird}: \(2\N \subseteq \N\), yet \(|2\N| = |\N|\). vs. \( 3< 4\), yet, \(|3| < |4|\). \\

            \begin{definition}
                {Infinity} Suppose \(A\) is a set. The statement that \(A\) is \textit{infinite} means there exists a proper subset of \(A\) that has the same cardinality of \(A\) itself. \\
                Similarly, \(A\) is said to be \textit{finite} if \(A\) is not infinite.
            \end{definition}

        \textbf{Analogy}: Hilbert's Hotel -- Hotel room for each number of \(\N\). Asking if \(|A| = \ale\) asks if \(|A|\) guests arrive, can you \underline{exactly} fill the hotel with one guest per room and each guest has a room (a.k.a., bijection).

        \section{Problem}

% ------------------------------------------------------------------------------
% Problem 68
% ------------------------------------------------------------------------------

            \begin{exercise}
                {Cardinality II}Show \(|\Z| = \ale\).
            \end{exercise}
            \textbf{Hint}: Create bijection \(f \ \colon \N \rightarrow \Z\). Feels bad because \(\ale + \ale = \ale\).

            \expf{
                Suppose \(f\) is the relation defined by the following: \[\text{For each }n \in \W\text{, map }n \text{ to:} 
                \begin{cases}
                    \frac{n}{2} & \text{if } n \text{ is even}, \\
                    -\frac{n + 1}{2} & \text{if } n \text{ is odd and } \ne 1.
                \end{cases}\]
                To show that \(\Z\) has the same cardinality as \(\N\) (\(\ale\)), we need to prove that \(f\) is a bijection. Thus, consider the following: \\
                \begin{enumerate}
                    \item \(f\) is a function: \\
                    Let \((a,b_1), (a,b_2) \in f\). If \(n\) is even, then \(b_1\) and \(b_2\) both equal \(\frac{a}{2}\) by definition of \(f\), hence \(b_1 = b_2\). If \(a\) is odd, then \(b_1\) and \(b_2\) both equal \(-\frac{a + 1}{2}\) by definition of \(f\), hence \(b_1 =b_2\). This shows that no matter the input in \(\W\), there is always an output in \(\Z\).
                    \item \(f\) has \(\N\) as domain: \\
                    Note that for all \(n\in \N\), our function covers every possible integer, so the domain of \(f\) is all of \(\W\). 
                    \item \(f\) is injective: \\
                    Let \(f(n) = f(m)\) for some \(n,m \in \W\). If both \(n\) and \(m\) are even, then \(\frac{n}{2} = \frac{m}{2}\), which implies \(n = m\). If both \(n\) and \(m\) are odd, then \(-\frac{n + 1}{2} = -\frac{m + 1}{2}\) implies \(n = m\). As for the last case, it is impossible for \(n\) to be odd and \(m\) to be even (or vice versa), because that would contradict \(f(n) = f(m)\). Hence, \(f\) is injective. 
                    \item \(f\) is surjective: \\
                    For any positive integer, \(z \in \Z\), there exists \(n = 2z \in \W\) such that \(f(n) = z\). For any negative integer, \(z \in \Z\), there exists \(n = -2n - 1 \in \W\) such that \(f(n) = z\). Hence, every integer has a representation in the natural numbers, so \(f\) is surjective.
                \end{enumerate}
                Therefore, because we have shown that \(f\) is a function, its domain is all of \(A\), its injective, and its surjective, \(f\) must be a bijection.
            }

            \begin{ntheorem}
                {Schröder-Bernstein Theorem} Suppose \(A\) and \(B\) are sets for which injections \(f \ \colon A \rightarrow B\) and \(g \ \colon B \rightarrow A\). Each of which uses all of their stated domains. Then, \[|A| = |B|\]
            \end{ntheorem}

            \textbf{Note}: Useful in cases where bijections might be difficult to construct. \\

            Intuitively, injection \(A \rightarrow B\) means \(A\) is not larger than \(B\). Thus, \(|A| = |B| \iff \exists \text{ bijection } A \rightarrow B\). And \( \exists  \text{ bijection } A \rightarrow B \implies |A| \subseteq |B|\).
            
            \expf{
                No proof. 
            }

            \noindent \textbf{Note}: If \(A \subseteq B\), then \(|A| \leq |B|\). \\
            \textbf{Note}: \(\N \subseteq \Q\), so that means \(|\N| \leq |\Q|\) where \(|\N| = \ale\).

        \section{Problems}

% ------------------------------------------------------------------------------
% Problem 69
% ------------------------------------------------------------------------------

            \begin{exercise}
                {Bijection III}Show \(|\Q| = \ale\).
            \end{exercise}
            \textbf{Hint}: We can focus on only the positive rationals: \(\Q^+\). \\
            Use the Schröder-Bernstein Theorem. Then, find injection \(\N \rightarrow \Q^+\) (easy); \(\Q^+ \rightarrow \N\) (weird).

            \expf{
                We need to show that the relation \(f\)---which is defined as \(f \ \colon \N \rightarrow \Q^+\) by \(f(n) = n\)---is injective. Then, we also need to show that the relation \(g\)---which is defined as \(g \ \colon \Q^+ \rightarrow \N\) by \(g(\frac{n}{d}) = 2^n \times 3^d\)---is also injective. Thus, we need to prove that both relations are injective functions. \\

                \noindent Let \((a,b_1), (a,b_2) \in f\), then \(b_1 = a = b_2\) by definition of \(f\). Thus, \(f\) is a function. To show that \(f\) uses its whole domain, let \(n \in \N\). We know that there is an associated \(b \in \Q^+\) for that \(n\) because our function \(f\) maps each natural number to an equivalent number in \(\Q^+\). Hence, the whole domain is used. To show that \(f\) is injective, let \(f(a_1) = f(a_2)\). Because \(f\) is a function, we know that each of these outputs must be unique, and since the whole domain is being used, that means for every number in \(\N\), there must be a unique output such that \(a_1 = a_2\) for \(f(a_1) = f(a_2)\) or else the functions would map to separate numbers and negate the equivalence that we stated. \\

                \noindent For \(g\), we need to use induction to show every \(n > 1 \in \N\) can be expressed as a product of primes. Thus, consider the following: 
                \begin{itemize}
                    \item \textbf{Base Case}: \\
                    When \(n = 2\), \(n\) itself is a prime, and thus is already in its factorized form. 
                    \item \textbf{Inductive Hypothesis}: \\
                    Suppose for every natural number greater than 1 and less than \(n\) can be expressed as a product of primes. 
                    \item \textbf{Inductive Step}: \\
                    We have two cases: If \(n\) is prime, and if \(n\) is not prime. The first case is trivially true, because if \(n\) is prime, then we accomplished our goal. If \(n\) is not prime, then \(n\) can be written as a product of two natural numbers, \(a,b\). Then, by the inductive hypothesis, we know that \(a\) and \(b\) can be factored into primes, so it must be the case that the product of those numbers can also be expressed as primes.
                \end{itemize}
                To show that this factorization is unique, assume to the contrary that it is not unique. Thus, each number \(n \in \N\) has two different prime factorizations. Considering this is the case, we can define a natural number \(n\) as being equal to \(p_1 \times p_2 \times \cdots \times p_k\) and \(q_1 \times q_2 \times \cdots \times q_j\). By substitution, we can arrange these expressions as: \(p_1 \times p_2 \times \cdots \times p_k = q_1 \times q_2 \times \cdots \times q_j\). While trying to solve this expression, we can divide both sides by \(p_1\). Because \(p_1\) is prime, we know it cannot divide the product of \(q_1 \times q_2 \times \cdots \times q_j\) partially---it must completely divide one of them (because primes have only two divisors: 1, and themselves). \\

                \noindent Hence, suppose \(p_1\) divides \(q_1\). The only possible way for \(p_1\) to divide \(q_1\) is if \(p\) is equal to \(q\) (note that we are not counting 1 as a prime number). Therefore, it must be the case that \(p_1 = q_1\). There is an iterative effect here: if \(p_1 = q_1\), then \(p_2 = q_2\), then \(p_3 = q_3 \dots p_k = q_j\). This proves that the prime factorization is unique, and that \(g\) is a function. \\

                \noindent We know that \(g\) uses all of its domain because it is defined as \(g(\frac{n}{d}) = 2^n \times 3^d\), which uses every positive rational number, \(\frac{n}{d} \in \Q^+\). Then, for injectivity, suppose \(g(\frac{n}{d}) = g(\frac{a}{c})\). Because we know that \(g(\frac{n}{d})\) will result in a prime factorization that is unique, and \(g(\frac{a}{c})\) will also result in a prime factorization that is unique, for \(g(\frac{n}{d})\) to be equal to \(g(\frac{a}{c})\), it must be the case that \(\frac{n}{d} = \frac{a}{c}\). \\

                \noindent Therefore, because we have found an injective function \(f \ \colon \N \rightarrow \Q^+\) and injective function \(g \ \colon \Q^+ \rightarrow \N\), by the Schröder-Bernstein Theorem, we know that the cardinality of \(\Q^+\) is \(\ale\). Then, because \(\Q^+\) is a proper subset of \(\Q\) and has cardinality \(\ale\), then it must be the case that \(|\Q| = \ale\) by definition of infinite. 
            }
            
            \textbf{Note}: sometimes people call \(\ale\) ``countably infinite.'' \\

% ------------------------------------------------------------------------------
% Problem 70
% ------------------------------------------------------------------------------

            \begin{exercise}
                {Enumeration}Suppose \(A\) is a set with \(|A| = \ale\). Show there exists an enumeration of \(A\); that is, \[A = \{a_1,a_2,\dots\}\] where \(a_i \ne a_j\) for \(i \ne j\).
            \end{exercise}

            \expf{
                Suppose \(A\) is a set with cardinality equal to \(\ale\). Because \(\N\) also has cardinality equal to \(\ale\), we can posit that there is a bijection, \(f \ \colon \N \rightarrow A\) by \(f(n) = a_n\). Then, suppose \(i,j \in \N\). Because \(f\) is an injective function, the mapping of \(f(i)\) must be unique for both the domain and the co-domain. Hence, it cannot be the case that \(f(i) = f(j)\). It follows that since \(f(i) \ne f(j)\), \(a_i \ne a_j\) for \(i \ne j\). Therefore, there exists an enumeration of \(A\). 
            }

% ------------------------------------------------------------------------------
% Problem 71
% ------------------------------------------------------------------------------

            \begin{exercise}
                {}If \(A\) is a set of cardinality, \(\ale\), then \(A\) is infinite.
            \end{exercise}
            \textbf{Hint}: Need to find a proper subset (i.e., \(C \subseteq A\) and \(C \ne A\). Notation: \(C \subsetneq A\)) \(C\) of \(A\) such that \(|C| = \ale\). \\
            Need to use Problem 70.

            \expf{
                Let \(|A| = \ale\). To demonstrate that \(A\) is infinite, we need to show that for any finite subset of \(A\), there exists at least one element in \(A\) not contained in this subset, implying \(A\) cannot be exhausted by its finite subsets. \\
                
                \noindent Given the enumeration of \(A\) from Problem 70, consider any subset \(C \subseteq A\) where \(C \ne A\), that has cardinality \(\aleph_0\). We can define \(C\) by selecting an element to exclude from \(A\), like \(a_1\). Thus, let \(C\) be defined as \(\{a_2,a_3,\dots\}\). \\
                
                \noindent Hence, \(C\) is a proper subset of \(A\) since it does not include \(a_1\). Despite \(C\) being infinite and having the same cardinality as \(A\), it is still not equal to \(A\) because it lacks at least one element (\(a_1\)) that is in \(A\). That is, because \(A \subsetneq C\) and \(C \subseteq A\), it fails the criterion for Theorem 1.  \\
                
                \noindent If \(A\) were finite, removing any element (like \(a_1\)) would reduce its size, making \(|C|\) less than \(|A|\). However, since both \(A\) and \(C\) share the same cardinality \(\aleph_0\), and \(C\) is a proper subset of \(A\), we have shown that \(A\) is indeed infinite by definition.
            } 
        