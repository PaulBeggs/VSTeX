\begin{exercise}
    {2.5.1}Give an example of each of the following, or argue that such a request is impossible.
    \begin{enumerate}
        \item A sequence that has a subsequence that is bounded but contains no subsequence that converges.
        \item A sequence that does not contain 0 or 1 as a term but contains subsequences converging to each of these values.
    \end{enumerate}
\end{exercise}

\sol{
    \begin{enumerate}
        \item Impossible. This violates the \namrefthm{Bolzano-Weierstrass Theorem}. It assures us that every bounded sequence has a convergent subsequence. If a subsequence is bounded, then it must have a convergent subsequence.
        \item Consider the sequence \((\frac{1}{2},\frac{1}{2},\frac{1}{3},\frac{2}{3},\frac{1}{4},\frac{3}{4}, \cdots \frac{1}{n}, \frac{(n-1)}{n})\). From this, you can have a subsequence \((\frac{1}{2}, \frac{1}{3}, \dots, \frac{1}{n})\) which converges to 0, and also a subsequence \((\frac{1}{2}, \frac{2}{3}, \dots \frac{n - 1}{n})\), which converges to 1.
    \end{enumerate}
}

\begin{exercise}
    {2.5.2} Decide whether the following propositions are true or false, providing a short justification for each conclusion.
    \begin{enumerate}
        \item If every proper subsequence of \((x_n)\) converges, then \((x_n)\) converges as well.
              \setcounter{enumi}{2}
        \item If \((x_n)\) is bounded and diverges, then there exist two subsequences of \((x_n)\) that converge to different limits.
    \end{enumerate}
\end{exercise}

\sol{
    \begin{enumerate}
        \item False. As shown in Example 2.5.4, if we have a sequence like \((1, -\frac{1}{2}, \frac{1}{3}, -\frac{1}{4}, \frac{1}{5}, -\frac{1}{5}, \frac{1}{5}, \cdots)\), we see that it is divergent, and has two proper subsequences \((\frac{1}{5},\frac{1}{5}, \cdots)\) and \((-\frac{1}{5}, -\frac{1}{5}, \cdots)\) that both converge to values \(\frac{1}{5}\) and \(-\frac{1}{5}\), respectively.
              \setcounter{enumi}{2}
        \item True. If we assume that \((x_n)\) is bounded and diverges, then by the \namrefthm{Bolzano-Weierstrass Theorem}, every bounded sequence has a convergent subsequence. Let \((x_{n_k})\) be that convergent subsequence of \((x_n)\), and let \(L_1\) be its limit. Since \((x_n)\) diverges, it cannot converge to \(L_1\). Thus, there exists an \(\epsilon_0 > 0\) such that for all terms of \((x_n)\), they stay outside the \(\epsilon_0\)-neighborhood of \(L_1\). Assume this sub-subsequence \((x_{m_k})\) contains these terms. When we apply the same logic to this sub-subsequence, we see that by the \namrefthm{Bolzano-Weierstrass Theorem}, \((x_{m_k})\) has a convergent subsequence with limit \(L_2\), where \(L_2 \ne L_1\) because the terms of \((x_{m_k})\) stay outside the \(\epsilon_0\)-neighborhood of \(L_1\). Thus \((x_n)\) contains two subsequences that converge to different limits, \(L_1\) and \(L_2\).
    \end{enumerate}
}

\begin{exercise}
    {2.5.5} Assume \((a_n)\) is a bounded sequence with the property that every convergent subsequence of \((a_n)\) converges to the same limit \(a \in \mathbb{R}\). Show that \((a_n)\) must converge to \(a\).
\end{exercise}

\expf{
Suppose that \((a_n)\) does not converge to \(a \in \R\). By the definition of convergence, this means there is a positive real number \(\epsilon_0\) such that no matter how large we choose \(N \in \N\), there will always exist some \(n > N\) where \(|a_n - a| \geq \epsilon_0\). In a formal way, this shows that \((a_n)\) does not converge to \(a\) within the \(\epsilon_0\)-neighborhood.

We aim to demonstrate that this leads to a contradiction by constructing a subsequence of \((a_n)\) that stays outside this neighborhood. Begin by selecting \(n_1\) such that \(|a_{n_1} - a| \geq \epsilon_0\). Next, since the condition holds for all \(N \in \N\), we can find another index \(n_2 > n_1\) such that \(|a_{n_2} - a| \geq \epsilon_0\). Continuing this process, we generate an increasing sequence of indices \(n_1 < n_2 < n_3 < \dots\) such that for each \(i \in \N\), \(|a_{n_i} - a| \geq \epsilon_0\).

Now consider the subsequence \((a_{n_i})\) we have built. Since \((a_n)\) is bounded by assumption, its subsequence \((a_{n_i})\) is also bounded. By the \namrefthm{Bolzano-Weierstrass Theorem}, every bounded sequence has a convergent subsequence. Let \((a_{n_{i_k}})\) denote a convergent subsequence of \((a_{n_i})\). According to our assumption, any convergent subsequence of \((a_n)\) must converge to \(a\).

However, each term of \((a_{n_{i_k}})\) remains outside the \(\epsilon_0\)-neighborhood of \(a\). Thus, it is impossible for \((a_{n_{i_k}})\) to converge to \(a\). This contradiction implies that our initial assumption—that \((a_n)\) does not converge to \(a\)—is false. Therefore, the sequence \((a_n)\) must converge to \(a\).
}



\begin{exercise}
    {2.5.6} Use a similar strategy to the one in \numrefthm{2.5.5} to show
    \[\lim b^{1/n} \text{ exists for all } b \geq 0\] and find the value of the limit. (The results in \hyperref[ex:2.3.1]{Exercise 2.3.1} may be assumed.)
\end{exercise}

\expf{
    For this proof, we will be examining different cases that \(b\) may fall under. These cases are \(b = 0\) and \(b > 0\).
    \begin{enumerate}[label=\arabic*.]
        \item If \(b = 0\), then the sequence \((b^{1/n})\) is simply \((0, 0, 0, \dots)\). Thus, the limit of this sequence is 0.
        \item For \(b > 0\), it is more complicated. We see that this sequence is decreasing because as \(n\) increases, the exponent \(\frac{1}{n}\) trends toward a higher denominator, so \(b^{1/n}\) decreases. We can see this by observing the property that for any \(m < n\), \(\frac{1}{n} < \frac{1}{m}\). This implies that \(b^{1/n} < b^{1/m}\).

              Further, we see that this sequence is bounded below by \(0\) when \(b > 1\) and by 1 when \(b = 1\) (because \(1\) to any exponent is just 1). Therefore, this sequence is monotone and bounded below, so it must converge to some \(L \geq 0\) by the \namrefthm{Monotone Convergence Theorem}.

              To find \(L\), we need to consider another set of cases for \(b\). \begin{enumerate}[label = \roman*.]
                  \item If \(b = 1\), then \(b^{1/n} = 1\) for all \(n\), so its limit is 1.
                  \item If \(b > 1\), then as \(n\) increases, the exponent \(\frac{1}{n}\) approaches 0, so \(b^{1/n}\) approaches \(b^0 = 1\). So, its limit is also 1.
                  \item If \(0 < b < 1\), then we know \(\limn b^{1/n} = 1\) because \(b^{1/n}\) is a decreasing sequence bounded by 0, and when we have smaller and smaller powers of a number less than 1, it pushes it closer to 1.
              \end{enumerate}
    \end{enumerate}
    Through these cases, we see that for all \(b \geq 0\), \(\limn b^{1/n} = 1\).
}
