\begin{exercise}
    {1.2.3}Decide which of the following represent true statements about
    the nature of sets. For any that are false, provide a specific example where the
    statement in question does not hold.
    \begin{enumerate}
        \item If \(A_1 \supseteq A_2 \supseteq A_3 \supseteq A_4\dots\) are all sets containing an infinite number of
              elements, then the intersection \(\bigcap^\infty_{n=1} A_n\) is infinite as well.
        \item If \(A_1 \supseteq A_2 \supseteq A_3 \supseteq A_4\dots\) are all finite, nonempty sets of real numbers,
              then the intersection \(\bigcap^\infty_{n=1} A_n\) is finite and nonempty.
        \item \(A \cap (B \cup C) = (A \cap B) \cup C\)
        \item \(A \cap (B \cap C) = (A \cap B) \cap C\)
        \item \(A \cap (B \cup C) = (A \cap B) \cup (A \cap C)\)
    \end{enumerate}
\end{exercise}

\sol{
    \begin{enumerate}
        \item This is false. Consider the following as a counterexample: If we define \(A_1\) as \(A_n = \{n, n+1, n+2,\dots\} = \{k \in \N \ | \ k \geq n\}\), we can see why the intersection of these sets of infinite numbers are actually empty. Consider a number \(m\) that actually satisfies \(m \in A_n\) for every \(A_n\) in our collection of sets. Because \(m\) is not an element of \(A_{m + 1}\), no such \(m\) exists and the intersection is empty.
        \item This is true.
        \item False. Consider sets \(A = \{1, 2, 3\}\), \(B = \{3, 6, 7\}\) and \(C = \{5\}\). Note that
              \(A \cap (B \cup C) = \{3\}\) is not equal to \((A \cap B) \cup C = \{3, 5\}\).
        \item This is true.
        \item This is true.
    \end{enumerate}
}

\begin{exercise}
    {1.2.5 (De Morgan's Laws)}(\hyperref[def:1.2.1]{This definition} may prove useful.) Let \(A\) and \(B\) be subsets of \(\R\).
    \begin{enumerate}
        \item  If \(x \in (A \cap B)^c\), explain why \(x \in A^c \cup B^c\). This shows that \((A \cap B)^c \subseteq A^c \cup B^c\).
        \item Prove the reverse inclusion \((A \cap B)^c \supseteq A^c \cup B^c\), and conclude that \((A \cap B)^c = A^c \cup B^c\).
        \item Show \((A \cup B)^c = A^c \cap B^c\) by demonstrating inclusion both ways.
    \end{enumerate}
\end{exercise}

\sol{
    \begin{enumerate}
        \item If \(x \in (A \cap B)^c\), and we know that \(A^c = \{x \in \R \ \colon x \notin A\}\), then we know \(x\) must cannot exist in \(A^c\) and \(B^c\) because \((A \cap B)^c = \{x \in \R \ \colon x \notin (A \cap B)\}\). Thus, \(x\) is in either \(A^c\) or \(B^c\). Put another way \(x \in A^c \cup B^c\). Since we have shown that an element that started in \((A \cap B)^c\) ended up in \(A^c \cup B^c\), then we know \((A \cap B)^c \subseteq A^c \cup B^c\).
        \item Assume  \(y \in A^c \cup B^c\). Thus, it must be the case that \(y \notin A\) or \(y \notin B\). Hence, \(y\) cannot be exist in both sets at the same time, so \(y \in (A \cap B)^c\). Because we have taken an element that started in \(A^c \cup B^c\) and have shown that it exists in \((A \cap B)^c\), we have proven \(A^c \cup B^c \subseteq (A \cap B)^c\).

              Therefore, since both inclusions hold, \((A \cap B)^{c} = A^{c} \cup B^{c}\).
        \item \sbseteqpf{
                  Let \(x \in (A \cup B)^c\).

                  By definition of the complement, this means \(x \notin A \cup B\).
                  Therefore, by definition of union, \(x \notin A\) and \(x \notin B\).
                  Thus, \(x \in A^c\) and \(x \in B^c\).
                  Hence, \(x \in (A^c \cap B^c)\).
                  So, \((A \cup B)^c \subseteq A^c \cap B^c\).}
              {Let \(x \in (A^c \cap B^c)\).

                  By definition of intersection, \(x \in A^c\) and \(x \in B^c\).
                  So, by definition of the complement \(x \notin A\) and \(x \notin B\).
                  Therefore, \(x \notin A \cup B\).
                  Hence, \(x \in (A \cup B)^c\).
                  Thus, \(A^c \cap B^c \subseteq (A \cup B)^c\).}
              {Since both inclusions hold, we have:

                  \[(A \cup B)^c = A^c \cap B^c.\qedhere\] }
              {xgray}
    \end{enumerate}
}



\begin{exercise}
    {1.2.7}Given a function \(f\) and a subset \(A\) of its domain, let \(f(A)\)
    represent the range of \(f\) over the set \(A\); that is, \(f(a) = \{f(x) \colon x \in A\}\).
    \begin{enumerate}
        \item Let \(f(x) = x^2\). If \(A = [0,2]\) (the closed interval \(\{x \in \R \colon 0 \leq x \leq 2\}\)) and \(B = [1,4]\), find \(f(A)\) and \(f(B)\). Does \(f(A\cap B) = f(A) \cap f(B)\) in this case? Does \(f(A \cup B) = f(A) \cup f(B)\)?
        \item Find two sets \(A\) and \(B\) for which \(f(A\cap B) \ne f(A) \cap f(B)\).
        \item Show that, for an arbitrary function \(g \ \colon \ \R \rightarrow \R\), it is always true that \(g(A \cap B) \subseteq g(A) \cap g(B)\) for all sets \(A,B\subseteq \R\).
        \item Form and prove a conjecture about the relationship between \(g(A \cup B)\) and \(g(A) \cup g(B)\) for an arbitrary function \(g\).
    \end{enumerate}
\end{exercise}

\sol{
    \begin{enumerate}
        \item Since \(f(x) = x^2\), the intervals of \(f(A)\) would be \([0,4]\) and \(f(B)\) would be \([1,16]\). The interval of the intersection of \(A \cap B\) is \([1,2]\). Take this through our function, we get \(f(A \cap B) = [1,4]\). On the other side of the equation, we already know the intervals of \(f(A)\) and \(f(B)\), and the intersection of theirs would be \([1,4]\). So they do equal each other. We know \(f(A \cup B)\) and \(f(A) \cup f(B)\) will be equivalent because \(f(A \cup B)\) has an interval of \([0,16]\), and \(f(A) \cup f(B)\) also has an interval of \([0,16]\) because taking the union of \([0,4] \cup [1,16]\) is \([0,16]\).
        \item Two sets could be \(A = [-1,0]\) and \(B = [0,1]\). For \(f(A \cap B)\), we have \(\{0\}\) but \(f(A) \cap f(B) = [0,1]\).
        \item \innerpf{
                  Let \(x \in g(A \cap B)\). Using the definition of function, we know there exists a \(y \in A \cap B\) to which that \(y\) is mapped to as \(g(y) = x\). From the definition of intersection, we know \(y \in A\) and \(y \in B\) such that \(x = g(y) \in g(A)\) and \(x = g(y) \in g(B)\) because \(y \in A \cap B\). Putting it together, we have \(x \in g(A) \cap g(B)\) thus proving \(g(A \cap B) \subseteq g(A) \cap g(B)\)
              }
        \item Conjecture: For any function \(g\) defined as \(g \ \colon \R \rightarrow \R\) and for any subsets \(A,B \subseteq \R\), it is always that case that 
        \[
            g(A \cup B) = g(A) \cup g(B).
        \]

              \sbseteqpf{
                  Let \(y \in g(A \cup B)\). 
                  
                  By definition of function, there exists \(x \in A \cup B\) such that \(g(x) = y\). From the definition of union, \(y \in A\) or \(y \in B\) and thus, \(g(x) \in g(A)\) or \(g(x) \in g(B)\). Together, we have \(g(x) = y \in g(A) \cup g(B)\). Therefore, \(g(A \cup B) \subseteq g(A) \cup g(B)\).
              }{
                  Let \(y \in g(A) \cup g(B)\).

                  By definition of union, \(y \in g(A)\) or \(y \in g(B)\). By definition of function, this means there exists an \(x \in A\) or \(x \in B\) such that \(g(x) = y\). Hence, \(x \in A \cup B\) by the definition of union, and \(g(x) \in g(A \cup B)\). Therefore, \(g(A) \cup g(B) \subseteq g(A \cup B)\).
              }{
                  Since we have shown the inclusion for both directions, this proves that \(g(A \cup B) = g(A) \cup g(B)\).
              }%
              {xgray}
    \end{enumerate}
}

\begin{exercise}
    {1.2.8}Given a function \(f \ \colon A \rightarrow B\) can be defined as either \hyperref[def:1.2.5]{injective} or \hyperref[def:1.2.6]{surjective}, give an example of each or state that the request is impossible:
    \begin{enumerate}
        \item \(f \ \colon \N \rightarrow \N\) that is 1-1 but not onto.
        \item \(f \ \colon \N \rightarrow \N\) that is onto but not 1-1.
        \item \(f \ \colon \N \rightarrow \Z\) that is 1-1 and onto.
    \end{enumerate}
\end{exercise}

\sol{
    \begin{enumerate}
        \item \textbf{Possible.} Define \(f \colon \N \rightarrow \N\) as \(f(a) = a + 1\). This function is injective because when \(f(a_{1}) = f(a_{2})\), we have:
        \begin{align*}
            a_{1} + 1 &= a_{2} + 1 \\
            a_{1} &= a_{2}.
        \end{align*}
        However, this function is not surjective because the entire co-domain is not covered; that being 1.
        \item \textbf{Possible.} Consider the example: 
        \[
            f(1) = 1, f(2) = 1, f(3) = 2, f(4) = 3, \ldots.
        \] 
        This function is surjective because for every \(b\) in the co-domain there is an \(a\) in the domain. However, it is not injective because we have two distinct values, 1 and 2, that map to the same \(b\), 1.
        \item \textbf{Possible.} Consider the example:
        \[
            f(1) = 0, f(2) = 1, f(3) = -1, f(4) = 2, f(5) = -2, \ldots.
        \]
        Since every value \(b\) in the co-domain is mapped to an element in  the domain, this function is surjective. Additionally, because that mapping is unqiue (i.e., \(a_{1} \ne a_{2} \implies f(a_{1}) \ne f(a_{2})\)), this function is injective by definition.
    \end{enumerate}
}