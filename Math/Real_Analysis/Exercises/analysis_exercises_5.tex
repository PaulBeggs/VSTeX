\begin{exercise}
    {3.2.4}Let \(A\) be a nonempty and bounded above set so that \(s = \sup(A)\) exists. 
    \begin{enumerate}
        \item Show that \(s \in \overline{A}\).
        \item Can an open set contain its supremum?
    \end{enumerate}
\end{exercise}

\sol{
    \begin{enumerate}
        \item We will show that \(s \in \overline{A}\). \\
        
        \innerpf{
            Let \(s = \sup(A)\). We know \(s\) can be in either two conditions:
            \begin{enumerate}[label=\arabic*.]
                \item If \(s \in A\), then \(s \in \overline{A}\) by definition of closure.
                \item For the sake of contradiction, suppose \(s \notin A\). We will show that \(s = \sup(A)\) is a limit point so \(s \in L \subseteq \overline{A}\). By definition, \(x\) is a limit point of \(A\) if for all \(\epsilon > 0\), there exists \(a \in A\) such that \(a \in V_\epsilon(x)\) and \(a \ne x\). Pick any \(\epsilon > 0\). Since \(s = \sup(A)\), for this \(\epsilon\), \(s - \epsilon\) is not an upper bound of \(A\). Therefore, there exists \(a \in A\) such that
                \[
                    s - \epsilon < a \leq s.
                \]
                Since \(a \leq s\) and \(a > s - \epsilon\), we have \(\disabs{a - s} < \epsilon\), so \(a \in V_\epsilon(s)\) and \(a \ne s\). Therefore, \(s\) is a limit point of \(A\), and hence \(s \in \overline{A}\). \qedhere
            \end{enumerate}
        }
        \item A set cannot contain its supremum if the set is open. For the sake of contradiction, assume \(A \subseteq \R\) is a non-empty open set containing its suprema \(s\). Since \(A\) is an open set, then for every \(x \in A\), there must exist an \(\epsilon > 0\) such that \(V_\epsilon(x) \subseteq A\) such that 
        \[
        V_\epsilon(s) = \{x \in \R \mid \disabs{x - s} < \epsilon\} \subseteq A.
        \]
        However, this implies there exists point \(y \in V_\epsilon(s)\) such that \(y > s\) because \(V_\epsilon(s)\) includes values both less than and great than \(s\). This contradicts the assumption that \(s\) is an upper bound for \(A\), as it would imply that \(A\) contains points greater than \(s\). Therefore, an open set \(A\) cannot contain its supremum.
    \end{enumerate}
}

\begin{exercise}
    {3.2.6}Decide whether the following statements are true or false. Provide counter examples for those that are false, and supply proofs for those that are true. (Choose 2.)
    \begin{enumerate}
        \item An open set that contains every rational number must necessarily be all of \(\R\).
        \item The \namrefthm{Nested Interval Property} remains true if the term ``closed interval'' is replaced by ``closed set.''
        \item Every nonempty open set contains a rational number.
        \item Every bounded infinite closed set contains a rational number.
        \item The Cantor set (\defref{3.1.1}) is closed. 
    \end{enumerate}
\end{exercise}

\sol{
    \begin{enumerate}
        \setcounter{enumi}{1}
        \item False. Consider the counter example of a sequence of closed sets \(C_n = [n, \infty)\) for \(n \in \N\). These sets satisfy \(C_1 \supseteq C_2 \supseteq \cdots\), but their intersection is 
        \[
            \bigcap_{n=1}^\infty C_n = \emptyset,
        \]
        which contradicts the \namrefthm{Nested Interval Property}. Therefore, replacing ``closed interval'' with ``closed set'' invalidates this property.
        \setcounter{enumi}{4}
        \item This statement is true: \\

        \innerpf{
            The \hyperref[def:3.1.1]{Cantor set} \(C\) is constructed as the intersection of a decreasing sequence of closed sets (finite unions of closed intervals). Since each of these sets is closed, and by \numrefthm{3.2.8}, the intersection of any collection of closed sets is closed, so \(C\) is closed.
        }
    \end{enumerate}
}

\begin{exercise}
    {3.2.8} Assume \(A\) is an open set and \(B\) is a closed set. Determine if the following sets are definitely open, definitely closed, both or neither.
    \begin{enumerate}
        \item \(\overline{A \cup B}\)
        \item \(A \setminus B = \{x \in A \mid x \notin B\}\)
        \item \((A^c \cup B)^c\)
        \item \((A \cap B) \cup (A^c \cap B)\)
        \item \(\overline{A}^c \cap \overline{A^c}\)
    \end{enumerate}
\end{exercise}

\textbf{Note:} Theorems~\ref{thm:3.2.3},~\ref{thm:3.2.6},~\ref{thm:3.2.8}, and~\ref{thm:3.2.9} are going to be needed to solve this problem.

\sol{
    \begin{enumerate}
        \item The closure of any set is closed by \numrefthm{3.2.9}. Therefore, \(\overline{A \cup B}\) is closed.
        \item We need to find a way to transform \(A \setminus B\) into something we can work with. Thus, if \(A \setminus B = \{x \in A \mid x \notin B\}\), another way of saying \(x \notin B\) is by the complement of \(B\). Specifically, \(x \in B^c\) implies \(x \notin B\). Because we want to preserve the set that is created by \(A \setminus B\), we can take the intersection of \(A\) and \(B^c\) to include only the parts that are not in \(B\) and in \(A\). Thus, \(A \setminus B = A \cap B^c\). Since the complement of a closed set is open by \numrefthm{3.2.6}, then \(B^c\) is open. Since the intersection of two open sets is open by \numrefthm{3.2.3}, \(A \cap B^c\) is open, and so is \(A \setminus B\).
        \item Taking the complement of \((A^c \cup B)^c\) gives \(A \cap B^c\) by De Morgan's Laws. Thus, since \(A\) is open and \(B^c\) is open because of \numrefthm{3.2.6}, \((A^c \cup B)^c\) is open by \numrefthm{3.2.3}.
        \item Because of the shared \(B\) in the expression \((A \cap B) \cup (A^c \cap B)\), we can factor out \(B\) to get 
        \[
            (A \cap B) \cup (A^c \cap B) = [(A \cup A^c) \cap B] = \R \cap B = B.
        \]
        Therefore, the set equals \(B\), which is closed.
        \item Since \(A\) is open, its closure \(\overline{A}\) is a closed set containing all limit points of \(A\). Therefore, the complement \(\overline{A}^c\) is open. Similarly, \(A^c\) is closed (being the complement of an open set), so its closure is \(\overline{A^c} = A^c\), which is closed.

        Now, consider the intersection:
        \[
        \overline{A}^c \cap \overline{A^c} = \overline{A}^c \cap A^c.
        \]
        Since \(\overline{A} \supseteq A\), we have \(\overline{A}^c \subseteq A^c\). Thus, the intersection simplifies to \(\overline{A}^c\). However, any point not in \(\overline{A}\) cannot be a limit point of \(A\) or belong to \(A\). In \(\mathbb{R}\), this set is empty unless \(A\) is either \(\emptyset\) or \(\mathbb{R}\). Therefore,
        \[
        \overline{A}^c \cap \overline{A^c} = \emptyset.
        \]
        The empty set is both open and closed.
    \end{enumerate}
}
\newpage
\begin{exercise}
    {3.2.11} \begin{enumerate}
        \item Prove that \(\overline{A \cup B} = \overline{A} \cup \overline{B}\).
        \item Does this result about closures extend to infinite unions of sets?
    \end{enumerate}
\end{exercise}

\sol{
    \begin{enumerate}
        \item We will prove that \(\overline{A \cup B} = \overline{A} \cup \overline{B}\): \\
        
        \sbseteqpf{
            Let \(x \in \overline{A \cup B}\). Then \(x \in A\) or \(x\) is a limit point of \(A \cup B\).
            \begin{itemize}
                \item If \(x \in \overline{A \cup B}\), then \(x \in A\) or \(x \in B\), so \(x \in \overline{A}\) or \(x \in \overline{B}\), thus \(x \in \overline{A} \cup \overline{B}\).
                \item If \(x\) is a limit point of \(A \cup B\), then every \(\epsilon\)-neighborhood \(V_\epsilon(x)\) contains a point \(y \ne x\) such that \(y \in A \cup B\). Therefore, \(y \in A\) or \(y \in B\), so \(x\) is a limit point of \(A\) or \(B\). Hence, \(x \in \overline{A}\) or \(x \in \overline{B}\), so \(x \in \overline{A} \cup \overline{B}\).
            \end{itemize}
            Therefore, \(\overline{A \cup B} \subseteq \overline{A} \cup \overline{B}\).
        }{
            Let \(x \in \overline{A} \cup \overline{B}\). Then \(x \in \overline{A}\) or \(x \in \overline{B}\).
            \begin{itemize}
                \item If \(x \in \overline{A}\), then \(x \in A\) or \(x\) is a limit point of \(A\). Since \(A \subseteq A \cup B\), \(x \in A \cup B\) or \(x\) is a limit point of \(A \cup B\). Thus, \(x \in \overline{A \cup B}\).
                \item Similarly, if \(x \in \overline{B}\), then \(x \in \overline{A \cup B}\).
            \end{itemize}
            Therefore, \(\overline{A} \cup \overline{B} \subseteq \overline{A \cup B}\).
        }{Hence, \(\overline{A \cup B} = \overline{A} \cup \overline{B}\).}{xgray}
        \item The result does not necessarily extend to infinite unions of sets.Consider the sets \(A_n = \left( \frac{1}{n}, 1 - \frac{1}{n} \right)\) for \(n \in \N\). Then \(\overline{A_n} = \left[ \frac{1}{n}, 1 - \frac{1}{n} \right]\). The infinite union is \(A = \bigcup_{n=1}^\infty A_n = (0,1)\), so \(\overline{A} = [0,1]\). The union of the closures is \(\bigcup_{n=1}^\infty \overline{A_n} = (0,1)\), since none of the closed intervals \(\left[ \frac{1}{n}, 1 - \frac{1}{n} \right]\) include the endpoints 0 or 1. Therefore, \(\overline{A} \ne \bigcup_{n=1}^\infty \overline{A_n}\). Hence, the equality does not hold for infinite unions.
    \end{enumerate}
}

\begin{exercise}
    {3.3.4}Assume \(K\) is \hyperref[def:3.3.1]{compact} and \(F\) is \hyperref[def:3.2.5]{closed}. Decide if the following sets are definitely compact, definitely closed, both, or neither.  
    \begin{enumerate}
        \item \(K \cap F\)
        \item \(\overline{F^c \cup K^c}\)
        \item \(K \setminus F = \{x \in K \mid x \notin F\}\)
        \item \(\overline{K \cap F^c}\)
    \end{enumerate}
\end{exercise}

\sol{
    \begin{enumerate}
        \item The intersection of two closed sets is \hyperref[def:3.2.5]{closed}. Since \(K\) is \hyperref[def:3.3.1]{compact} (and thus closed) and \(F\) is closed, \(K \cap F\) is closed. Moreover, \(K \cap F\) is a closed subset of the compact set \(K\). Because it is a closed subset of \(K\), it must also be bounded because \(K\) is bounded. Thus, this subset is compact. Therefore, \(K \cap F\) is both compact and closed.

        \item Since \(F\) is closed, its complement \(F^c\) is open. Similarly, \(K\) is compact and thus closed, so \(K^c\) is open. The union \(F^c \cup K^c\) is open, and its closure \(\overline{F^c \cup K^c}\) is closed by definition. However, this set may not be bounded and therefore not necessarily compact. Thus, \(\overline{F^c \cup K^c}\) is definitely closed but not necessarily compact.

        \item The set \(K \setminus F = K \cap F^c\) is the intersection of a compact set \(K\) and an open set \(F^c\). The result may not be closed or compact. Although \(K \setminus F\) is bounded (since \(K\) is bounded), it is not necessarily closed and therefore not necessarily compact. Hence, \(K \setminus F\) is neither compact nor closed.

        (Consider the example of \(K = [0,1]\) and \(F = \{0\}\). Then \(K \setminus F = (0,1]\) which is not closed (or open).)

        \item The closure \(\overline{K \cap F^c}\) is closed by definition. Since \(K\) is compact and \(\overline{K \cap F^c} \subseteq K\), the set \(\overline{K \cap F^c}\) is a closed subset of a compact set. By the same logic in (a), it is compact. Therefore, \(\overline{K \cap F^c}\) is both compact and closed.
    \end{enumerate}
}