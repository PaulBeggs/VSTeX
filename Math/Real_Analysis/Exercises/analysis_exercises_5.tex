\begin{exercise}
    {3.2.4}Let \(A\) be a nonempty and bounded above set so that \(s = \sup(A)\) exists. (See \defref{1.3.2} and \defref{3.2.7})
    \begin{enumerate}
        \item Show that \(s \in \bar{A}\).
        \item Can an open set contain its supremum?
    \end{enumerate}
\end{exercise}

\sol{
    \begin{enumerate}
        \item We need to show that \(s = \sup(A) \in \bar{A}\), where \(\bar{A} = A \cup L\), and \(L\) is the set of limit points of \(A\).

        Since \(A\) is nonempty and bounded above, \(s = \sup(A)\) exists.

        If \(s \in A\), then \(s \in \bar{A}\) trivially.

        Suppose \(s \notin A\). We will show that \(s\) is a limit point of \(A\), so \(s \in L \subseteq \bar{A}\).

        By definition, \(x\) is a limit point of \(A\) if for all \(\epsilon > 0\), there exists \(a \in A\) such that \(a \in V_\epsilon(x)\) and \(a \ne x\).

        Fix any \(\epsilon > 0\). Since \(s = \sup(A)\), for this \(\epsilon\), \(s - \epsilon\) is not an upper bound of \(A\). Therefore, there exists \(a \in A\) such that
        \[
        s - \epsilon < a \leq s.
        \]
        Since \(a \leq s\) and \(a > s - \epsilon\), we have \(|a - s| < \epsilon\), so \(a \in V_\epsilon(s)\) and \(a \ne s\).

        Therefore, \(s\) is a limit point of \(A\), and hence \(s \in \bar{A}\).

        \item An open set cannot contain its supremum if the supremum is finite.

        Assume \(A\) is an open set containing its supremum \(s\).

        Since \(A\) is open and \(s \in A\), there exists \(\epsilon > 0\) such that
        \[
        V_\epsilon(s) = \{ x \in \mathbb{R} \mid |x - s| < \epsilon \} \subseteq A.
        \]
        This means \(s + \frac{\epsilon}{2} \in A\).

        However, \(s\) is an upper bound of \(A\), so no element of \(A\) can be greater than \(s\).

        This is a contradiction.

        Therefore, an open set cannot contain its supremum.
    \end{enumerate}
}

\begin{exercise}
    {3.2.6}Decide whether the following statements are true or false. Provide counterexamples for those that are false, and supply proofs for those that are true.
    \begin{enumerate}
        \item An \hyperref[def:3.2.2]{open set} that contains every rational number must necessarily be all of \(\R\).
        \item The \namrefthm{Nested Interval Property} remains true if the term ``closed interval'' is replaced by ``\hyperref[def:3.2.5]{closed set}.''
        \item Every nonempty open set contains a rational number.
        \item Every bounded infinite closed set contains a rational number.
        \item The \hyperref[def:3.1.1]{Cantor set} is closed.
    \end{enumerate}
\end{exercise}

\sol{
    \begin{enumerate}
        \item \textbf{False.}

        \textit{Counterexample:} Consider the set \(U = \bigcup_{n=1}^\infty \left( q_n - \frac{1}{n},\ q_n + \frac{1}{n} \right)\), where \((q_n)\) is an enumeration of all rational numbers.

        Each interval \(\left( q_n - \frac{1}{n},\ q_n + \frac{1}{n} \right)\) is open, and the union \(U\) is open. Since every rational number is included in some interval, \(U\) contains all rationals. However, \(U \ne \mathbb{R}\) because there are irrational numbers not covered by these intervals. Therefore, an open set can contain all rational numbers without being all of \(\mathbb{R}\).

        \item \textbf{True.}

        \innerpf{The Nested Interval Property holds for any nested sequence of nonempty closed and bounded sets in \(\mathbb{R}\). If \(\{F_n\}\) is such a sequence with \(F_{n+1} \subseteq F_n\) for all \(n\), then the intersection \(\bigcap_{n=1}^\infty F_n\) is nonempty. This follows from the completeness of \(\mathbb{R}\), as every decreasing sequence of nonempty closed and bounded sets has a nonempty intersection. Therefore, replacing ``closed interval'' with ``closed set'' does not invalidate the property.}

        \item \textbf{True.}

        \innerpf{Let \(U\) be a nonempty open set. Then there exists \(x \in U\) and \(\epsilon > 0\) such that \((x - \epsilon,\ x + \epsilon) \subseteq U\). Since the rationals are dense in \(\mathbb{R}\), there exists a rational number \(q \in (x - \epsilon,\ x + \epsilon)\). Therefore, \(U\) contains a rational number.}

        \item \textbf{True.}

        \innerpf{Let \(F\) be a bounded infinite closed set. Since \(F\) is infinite and bounded, it must contain limit points. As \(F\) is closed, it contains its limit points. The real numbers are densely ordered with rationals between any two real numbers. Therefore, \(F\) must contain a rational number.}

        \item \textbf{True.}

        \innerpf{The \hyperref[def:3.1.1]{Cantor set} \(C\) is constructed as the intersection of a decreasing sequence of closed sets (finite unions of closed intervals). Since each of these sets is closed and the intersection of closed sets is closed, \(C\) is closed.}
    \end{enumerate}
}
\begin{exercise}
    {3.2.8} Assume \(A\) is an open set and \(B\) is a closed set. Determine if the following sets are definitely \hyperref[def:3.2.2]{open}, definitely \hyperref[def:3.2.5]{closed}, both, or neither.
    \begin{enumerate}
        \item \(\overline{A \cup B}\)
        \item \(A \setminus B = \{x \in A \mid x \notin B\}\)
        \item \((A^c \cup B)^c\)
        \item \((A \cap B) \cup (A^c \cap B)\)
        \item \(\overline{A}^c \cap \overline{A^c}\)
    \end{enumerate}
\end{exercise}

\sol{
    \begin{enumerate}
        \item \(\overline{A \cup B}\)

        The closure of any set is closed by definition. Therefore, \(\overline{A \cup B}\) is definitely closed.

        \textbf{Conclusion: Closed.}

        \item \(A \setminus B = \{x \in A \mid x \notin B\}\)

        Since \(B\) is closed, its complement \(B^c\) is open. Since \(A\) is open, the intersection \(A \cap B^c = A \setminus B\) is the intersection of two open sets, which is open.

        \textbf{Conclusion: Open.}

        \item \((A^c \cup B)^c\)

        Applying De Morgan's Law:
        \[
        (A^c \cup B)^c = A \cap B^c
        \]
        Since \(A\) is open and \(B^c\) is open (because \(B\) is closed), their intersection \(A \cap B^c\) is open.

        \textbf{Conclusion: Open.}

        \item \((A \cap B) \cup (A^c \cap B)\)

        Simplify the expression:
        \[
        (A \cap B) \cup (A^c \cap B) = [A \cup A^c] \cap B = \R \cap B = B
        \]
        Thus, the set equals \(B\), which is closed.

        \textbf{Conclusion: Closed.}

        \item \(\overline{A}^c \cap \overline{A^c}\)

        Since \(A\) is open, its closure \(\overline{A}\) is closed, so \(\overline{A}^c\) is open.

        Since \(A^c\) is closed (being the complement of an open set), \(\overline{A^c} = A^c\) is closed.

        Therefore, \(\overline{A}^c \cap \overline{A^c}\) is the intersection of an open set and a closed set, which is generally open but not necessarily closed.

        For example, let \(A = (0,1)\). Then:
        \[
        \overline{A} = [0,1], \quad \overline{A}^c = (-\infty, 0) \cup (1,\infty)
        \]
        and
        \[
        \overline{A^c} = A^c = (-\infty, 0] \cup [1,\infty)
        \]
        Then:
        \[
        \overline{A}^c \cap \overline{A^c} = [(-\infty, 0) \cup (1,\infty)] \cap [(-\infty, 0] \cup [1,\infty)] = (-\infty, 0) \cup (1,\infty)
        \]
        Which is an open set.

        \textbf{Conclusion: Open.}

    \end{enumerate}
}

\begin{exercise}
    {3.2.11} \begin{enumerate}
        \item Prove that \(\overline{A \cup B} = \overline{A} \cup \overline{B}\).
        \item Does this result about closures extend to infinite unions of sets?
    \end{enumerate}
\end{exercise}

\sol{
    \begin{enumerate}
        \item We will prove that \(\overline{A \cup B} = \overline{A} \cup \overline{B}\).
        
        \sbseteqpf[
            Recall that the closure of a set \(A\) is defined as \(\overline{A} = A \cup L_A\), where \(L_A\) is the set of limit points of \(A\).
        ]{
            Let \(x \in \overline{A \cup B}\). Then \(x \in A \cup B\) or \(x\) is a limit point of \(A \cup B\).

                \begin{itemize}
                            \item If \(x \in A \cup B\), then \(x \in A\) or \(x \in B\), so \(x \in \overline{A}\) or \(x \in \overline{B}\), thus \(x \in \overline{A} \cup \overline{B}\).
                
                            \item If \(x\) is a limit point of \(A \cup B\), then every neighborhood \(V_\epsilon(x)\) contains a point \(y \ne x\) such that \(y \in A \cup B\). Therefore, \(y \in A\) or \(y \in B\), so \(x\) is a limit point of \(A\) or \(B\). Hence, \(x \in \overline{A}\) or \(x \in \overline{B}\), so \(x \in \overline{A} \cup \overline{B}\).
                \end{itemize}
                
                            Therefore, \(\overline{A \cup B} \subseteq \overline{A} \cup \overline{B}\).
        }{
            Let \(x \in \overline{A} \cup \overline{B}\). Then \(x \in \overline{A}\) or \(x \in \overline{B}\).

                \begin{itemize}
                            \item If \(x \in \overline{A}\), then \(x \in A\) or \(x\) is a limit point of \(A\). Since \(A \subseteq A \cup B\), \(x \in A \cup B\) or \(x\) is a limit point of \(A \cup B\). Thus, \(x \in \overline{A \cup B}\).
                
                            \item Similarly, if \(x \in \overline{B}\), then \(x \in \overline{A \cup B}\).
                \end{itemize}
                
                            Therefore, \(\overline{A} \cup \overline{B} \subseteq \overline{A \cup B}\).
        }{
            Hence, \(\overline{A \cup B} = \overline{A} \cup \overline{B}\).
        }{xgray}
        \item The result does not necessarily extend to infinite unions of sets.

        Consider the sets \(A_n = \left( \frac{1}{n}, 1 - \frac{1}{n} \right)\) for \(n \in \N\). Then \(\overline{A_n} = \left[ \frac{1}{n}, 1 - \frac{1}{n} \right]\).

        The infinite union is \(A = \bigcup_{n=1}^\infty A_n = (0,1)\), so \(\overline{A} = [0,1]\).

        The union of the closures is \(\bigcup_{n=1}^\infty \overline{A_n} = (0,1)\), since none of the closed intervals \(\left[ \frac{1}{n}, 1 - \frac{1}{n} \right]\) include the endpoints 0 or 1.

        Therefore, \(\overline{A} \ne \bigcup_{n=1}^\infty \overline{A_n}\).

        Hence, the equality does not hold for infinite unions.

    \end{enumerate}
}

\begin{exercise}
    {3.3.4}Assume \(K\) is \hyperref[def:3.3.1]{compact} and \(F\) is \hyperref[def:3.2.5]{closed}. Decide if the following sets are definitely compact, definitely closed, both, or neither. 
    \begin{enumerate}
        \item \(K \cap F\)
        \item \(\overline{F^c \cup K^c}\)
        \item \(K \setminus F = \{x \in K \mid x \notin F\}\)
        \item \(\overline{K \cap F^c}\)
    \end{enumerate}
\end{exercise}

\sol{
\begin{enumerate}
    \item Since \(K\) and \(F\) are closed, their intersection \(K \cap F\) is closed (\numrefthm{3.2.8}).

    To show that \(K \cap F\) is compact, let \(\mathcal{U}\) be any open cover of \(K \cap F\). Our goal is to extract a finite subcover from \(\mathcal{U}\). We can then use the \namrefthm{Bolzano-Weierstrass Theorem} (iii) to show that \(K \cap F\) is compact.

    Since \(F\) is closed, its complement \(F^c\) is open (\numrefthm{3.2.6}). Then \(K \setminus F = K \cap F^c\) is open as the intersection of an open set and \(K\).

    Consider the open cover \(\mathcal{U}' = \mathcal{U} \cup \{K \setminus F\}\) of \(K\). Every point in \(K\) is either in \(K \cap F\) (covered by \(\mathcal{U}\)) or in \(K \setminus F\) (covered by \(K \setminus F\)).

    Since \(K\) is compact, there exists a finite subcover \(\mathcal{U}'' \subseteq \mathcal{U}'\) that covers \(K\).

    If \(K \setminus F\) is in \(\mathcal{U}''\), remove it to obtain a finite subcollection of \(\mathcal{U}\) that still covers \(K \cap F\). If \(K \setminus F\) is not in \(\mathcal{U}''\), then \(\mathcal{U}'' \subseteq \mathcal{U}\) already covers \(K \cap F\).

    Therefore, \(K \cap F\) is compact.

    \textbf{Conclusion:} Both compact and closed.

    \item Since \(F\) and \(K\) are closed, \(F^c\) and \(K^c\) are open. The union \(F^c \cup K^c\) is open (\numrefthm{3.2.3}), so its closure \(\overline{F^c \cup K^c}\) is closed by \numrefthm{3.2.9}.

    This set may not be bounded, so it's not necessarily compact.

    \textbf{Conclusion:} Definitely closed.

    \item The set \(K \setminus F = K \cap F^c\) is the intersection of a compact set \(K\) and an open set \(F^c\). This set is open in \(K\) but not necessarily open or closed in \(\mathbb{R}\).

    Since \(K \setminus F\) is not necessarily closed, it may not be compact.

    \textbf{Conclusion:} Neither compact nor closed.

    \item The set \(K \cap F^c\) is open in \(K\), so its closure \(\overline{K \cap F^c}\) is closed by \numrefthm{3.2.9}.

    To show that \(\overline{K \cap F^c}\) is compact, let \(\mathcal{U}\) be any open cover of \(\overline{K \cap F^c}\).

    Since \(\overline{K \cap F^c} \subseteq K\) and \(K\) is compact, we can consider \(\mathcal{U}\) as an open cover of a subset of \(K\).

    By the definition of \hyperref[def:3.3.3]{open cover}, there exists a finite subcover of \(\mathcal{U}\) that covers \(\overline{K \cap F^c}\).

    Therefore, \(\overline{K \cap F^c}\) is compact.

    \textbf{Conclusion:} Both compact and closed.

\end{enumerate}
}


