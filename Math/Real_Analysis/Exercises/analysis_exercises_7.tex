\begin{exercise}
    {4.3.11 (Contraction Mapping Theorem)} Let \(f\) be a function defined on all of \(\mathbb{R}\), and assume there is a constant \(c\) such that \(0 < c < 1\) and
    \[
        |f(x) - f(y)| \leq c|x - y|
    \]
    for all \(x, y \in \mathbb{R}\).
    \begin{enumerate}
        \item Show that \(f\) is continuous on \(\mathbb{R}\).
        \item Pick some point \(y_1 \in \mathbb{R}\) and construct the sequence
              \[
                  (y_1, f(y_1), f(f(y_1)), \ldots).
              \]
              In general, if \(y_{n+1} = f(y_n)\), show that the resulting sequence \((y_n)\) is a Cauchy sequence (\defref{2.6.1}). Hence, we may let \(y = \lim y_n\).
        \item Prove that \(y\) is a fixed point of \(f\) (i.e., \(f(y) = y\)) and that it is unique in this regard.
        \item Finally, prove that if \(x\) is \textit{any} arbitrary point in \(\mathbb{R}\), then the sequence \((x,f(x),f(f(x)),\dots)\) converges to \(y\) defined in (b).
    \end{enumerate}
\end{exercise}

\sol{
    \begin{enumerate}
        \item Let \(\epsilon > 0\). Choose \(\delta = \dfrac{\epsilon}{c}\). Then, for all \(x, y \in \mathbb{R}\) with \(|x - y| < \delta\), we have
              \[
                  |f(x) - f(y)| \leq c|x - y| < c \left( \dfrac{\epsilon}{c} \right) = \epsilon.
              \]
              Thus, \(f\) is continuous at every point in \(\mathbb{R}\).

        \item
              \innerpf{

              Since \(y_{n+1} = f(y_{n})\), we have \(y_{2} = f(y_{1})\), \(y_{3} = f(y_{2}) = f(f(y_{1}))\), and so on. Thus, the difference between consequent terms in the sequence is:
              \[
                  |y_{n + 1} - y_{n}| = |f(y_{n}) - f(y_{n - 1})| \leq c|y_{n} - y_{n - 1}|.
              \]
              Substituting, we see that:
              \[
                  |y_{n + 1} - y_{n}| \leq c|y_{n} - y_{n - 1}|.
              \]
              This inequality shows that our sequence is contracting (\defref{2.6.3}) with \(0 < c < 1\). Thus, by \numrefthm{2.6.4}, the sequence is Cauchy.

              % To show that \((y_{n})\) is Cauchy, we need to prove that for every \(\epsilon > 0\), there exists \(N \in \N\) such that for all \(m, n \geq N\), \(|y_{m} - y_{n}| < \epsilon\). (This works because of the \namrefthm{Cauchy Criterion}.)

              % Notice that for any two terms in the sequence, we have:
              % \[
              %     |y_{k} - y_{k + 1}| = |f(y_{k - 1}) - f(y_{k})| \leq c|y_{k - 1} - y_{k}|.
              % \]
              % This inequality can be repeated recursively to get:
              % \[
              %     |y_{k} - y_{k + 1}| \leq c^{k-1}|y_{1} - y_{2}|, \tag{1}
              % \]
              % where \(|y_{1} - y_{2}|\) is the difference between the first two terms of the sequence. 

              % Now, consider the difference \(|y_{m} - y_{n}\) for \(m < n\). We can write this as a telescoping sum:
              % \[
              %     |y_{m} - y_{n}| = |(y_{m} - y_{m + 1}) + (y_{m + 1} - y_{m + 2}) + \cdots + (y_{n - 1} - y_{n})|.
              % \]
              % Using the \hyperref[def:1.2.4]{triangle inequality}, we have:
              % \[
              %     |y_{m} - y_{n}| \leq |y_{m} - y_{m + 1}| + |y_{m + 1} - y_{m + 2}| + \cdots + |y_{n - 1} - y_{n}|.
              % \]
              % From (1), each term \(|y_{k} - y_{k + 1}|\) is bounded by \(c^{k - 1}|y_{1} - y_{2}|\). Thus, we have:
              % \[
              %     |y_{m} - y_{n}| \leq |y_{1} - y_{2}|(c^{m - 1} + c^{m} + \cdots + c^{n - 1}).
              % \] 
              % This is a geometric series, and we can factor out the 
              }
              \newpage
        \item Taking the limit as \(n \to \infty\), and using the continuity of \(f\), we have:
              \[
                  f(y) = f\left( \lim_{n \to \infty} y_n \right) = \lim_{n \to \infty} f(y_n) = \lim_{n \to \infty} y_{n+1} = y.
              \]
              Suppose there is another fixed point \(z\) such that \(f(z) = z\). Then
              \[
                  |y - z| = |f(y) - f(z)| \leq c|y - z|.
              \]
              Since \(0 < c < 1\), this implies \(y - z = 0\), so \(y = z\). Thus, the fixed point is unique.
        \item From the work we did in (b), we know the fixed point must be unique. Therefore, for any \(x \in \R\), it must be the case that the sequence \((x, f(x), f(f(x)), \ldots)\) converges to the fixed point \(y\).
    \end{enumerate}
}

\begin{exercise}
    {4.3.13} Let \(f\) be a function defined on all of \(\mathbb{R}\) that satisfies the additive condition \(f(x + y) = f(x) + f(y)\) for all \(x, y \in \mathbb{R}\).
    \begin{enumerate}
        \item Show that \(f(0) = 0\) and that \(f(-x) = -f(x)\) for all \(x \in \mathbb{R}\).
        \item Let \(k = f(1)\). Show that \(f(n) = kn\) for all \(n \in \mathbb{N}\), and then prove that \(f(z) = kz\) for all \(z \in \mathbb{Z}\). Now, prove that \(f(r) = kr\) for any rational number \(r\).
        \item Show that if \(f\) is continuous at \(x = 0\), then \(f\) is continuous at every point in \(\mathbb{R}\) and conclude that \(f(x) = kx\) for all \(x \in \mathbb{R}\). Thus, any additive function that is continuous at \(x = 0\) must necessarily be a linear function through the origin.
    \end{enumerate}
\end{exercise}

\sol{
    \begin{enumerate}
        \item For \(f(0)\), we have:
              \begin{align*}
                  f(0)        & = f(0 + 0)     \\
                  f(0)        & = f(0) + f(0)  \\
                  f(0)        & = 2f(0)        \\
                  f(0) - f(0) & = 2f(0) - f(0) \\
                  0           & = f(0).
              \end{align*}

              For all \(x \in \R\), we have:
              \begin{align*}
                  f(0)  & = f(x + (-x))  \\
                  0     & = f(x) + f(-x) \\
                  -f(x) & = f(-x).
              \end{align*}
        \item Let \(k = f(1)\).

              First, we will show that \(f(n) = kn\) for all \(n \in \mathbb{N}\) by induction.

              \textit{Base case:} For \(n = 1\),
              \[
                  f(1) = k = k \cdot 1.
              \]

              \textit{Inductive step:} Assume \(f(n) = kn\) for some arbitrary \(n \in \mathbb{N}\). Then,
              \begin{align*}
                  f(n + 1) & = f(n) + f(1) \\
                           & = kn + k      \\
                           & = k(n + 1).
              \end{align*}
              Thus, by induction, \(f(n) = kn\) for all \(n \in \mathbb{N}\).

              Next, from (a), we have \(f(-x) = -f(x)\). Hence, for \(z \in \mathbb{Z}\), if \(z = -n\) where \(n \in \mathbb{N}\),
              \[
                  f(z) = f(-n) = -f(n) = -kn = k(-n) = kz. \tag{1}
              \]
              Therefore, \(f(z) = kz\) for all integers \(z\).

              Now, observe that for \(p,q \in \Z\) with \(q \ne 0\), we have:
              \[
                  f\left(\underbrace{\dfrac{p}{q} + \dfrac{p}{q} + \cdots + \dfrac{p}{q}}_{q\text{ times}}\right) = f(p).
              \]
              Similarly, by the additive condition (\(q\) times on each side):
              \[
                  f\left(\dfrac{p}{q} + \dfrac{p}{q} + \cdots + \dfrac{p}{q}\right) = f\left(\dfrac{p}{q}\right) + f\left(\dfrac{p}{q}\right) + \cdots + f\left(\dfrac{p}{q}\right).
              \]
              This is equivalent to:
              \[
                  q \cdot f\left(\dfrac{p}{q}\right) = f(p).
              \]
              Therefore,
              \[
                  f\left(\dfrac{p}{q}\right) = \dfrac{1}{q} f(p)
              \]
              Putting everything together, we let \(r = \dfrac{p}{q}\). Then,
              \begin{align*}
                  f\left( \dfrac{p}{q} \right) & = \dfrac{1}{q} f(p)                         \\
                                               & = \dfrac{1}{q} (kp) \quad (\text{from (1)}) \\
                                               & = k \left( \dfrac{p}{q} \right)             \\
                                               & = kr.
              \end{align*}
              Thus, \(f(r) = kr\) for any rational number \(r\).
        \item Since \(f\) is continuous at \(x = 0\), we will show \(f\) is continuous everywhere.

              Let \(c \in \mathbb{R}\) and let \(\epsilon > 0\). Since \(f\) is continuous at \(0\), there exists \(\delta > 0\) such that if \(|h| < \delta\), then \(|f(h)| < \epsilon\).

              For \(x = c + h\), we have:
              \begin{align*}
                  |f(x) - f(c)| & = |f(c + h) - f(c)|          \\
                                & = |f(c) + f(h) - f(c)| \quad \\
                                & = |f(h)|                     \\
                                & < \epsilon.
              \end{align*}
              Thus, \(f\) is continuous at \(c\).

              Since \(f\) agrees with the continuous function \(kx\) on all rational numbers and \(f\) is continuous on \(\mathbb{R}\), it follows that \(f(x) = kx\) for all \(x \in \mathbb{R}\).

              Therefore, any additive function that is continuous at \(x = 0\) must be linear, \(f(x) = kx\).
    \end{enumerate}
}
\newpage
\begin{exercise}
    {4.4.3} Show that \(f(x) = \dfrac{1}{x^2}\) is uniformly continuous on the set \([1, \infty)\) but not on the set \((0, 1]\).
\end{exercise}

\sol{
    \begin{enumerate}
        \item \textbf{Uniform Continuity on \([1, \infty)\):}

              Observe that for all \(x, y \in [1, \infty)\),
              \[
                  |f(x) - f(y)| = \left| \dfrac{1}{x^2} - \dfrac{1}{y^2} \right| = \left| \dfrac{y^2 - x^2}{x^2 y^2} \right| = |y - x| \left( \dfrac{y + x}{x^2 y^2} \right).
              \]
              Since \(x, y \geq 1\), we have \(x^2 y^2 \geq 1\). Therefore,
              \[
                  \frac{y + x}{x^2 y^2} = \frac{y}{x^{2}y^{2}} + \frac{x}{x^{2}y^{2}} \leq \frac{1}{x^{2}y} + \frac{1}{xy^{2}} \leq 1 + 1 = 2.
              \]
              Now, let \(\epsilon > 0\). Choose \(\delta = \dfrac{\epsilon}{2}\). Then, when \(|x - y| < \delta\) for \(x, y \in [1, \infty)\), we have:
              \[
                  |f(x) - f(y)| = |y - x| \left( \dfrac{y + x}{x^2 y^2} \right) < \delta \cdot 2 = \epsilon.
              \]

        \item \textbf{Not Uniformly Continuous on \((0, 1]\):}

              We will show that \(f\) is not uniformly continuous on \((0, 1]\) by demonstrating that no matter how small we choose \(\delta > 0\), there exist \(x, y \in (0, 1]\) such that \(|x - y| < \delta\) but \(|f(x) - f(y)|\) is arbitrarily large.

              Let \(\epsilon = 1\). For the sake of contradiction, assume that \(f\) is uniformly continuous on \((0, 1]\); then, there exists \(\delta > 0\) such that for all \(x, y \in (0, 1]\), if \(|x - y| < \delta\), then \(|f(x) - f(y)| < \epsilon\).

              Choose \(x = \dfrac{\delta}{2}\) and \(y = \dfrac{\delta}{2} + h\) for some \(h\) with \(0 < h < \dfrac{\delta}{2}\). Then \(|x - y| = h < \delta\).

              Compute \(|f(x) - f(y)|\):
              \[
                  |f(x) - f(y)| = \left| \dfrac{1}{x^2} - \dfrac{1}{y^2} \right| = \left| \dfrac{y^2 - x^2}{x^2 y^2} \right| = \dfrac{|y^2 - x^2|}{x^2 y^2}.
              \]
              As \(x \to 0^+\), \(x^2 y^2 \to 0\), making the denominator very small and the entire expression very large.

              Specifically, take \(x_n = \dfrac{1}{n}\) and \(y_n = \dfrac{1}{n + 1}\) for \(n \in \mathbb{N}\). Then,
              \[
                  |x_n - y_n| = \left| \dfrac{1}{n} - \dfrac{1}{n + 1} \right| = \dfrac{1}{n(n + 1)} < \delta \quad \text{for large } n.
              \]
              However,
              \[
                  |f(x_n) - f(y_n)| = \left| n^2 - (n + 1)^2 \right| = \left| n^2 - n^2 - 2n - 1 \right| = 2n + 1.
              \]
              Thus, as \(|x_{n} - y_{n}| \rightarrow 0\), \(|f(x_{n}) - f(y_{n}) \rightarrow \infty\). Therefore, by the \namrefthm{Sequential Criterion for Absence of Uniform Continuity} theorem, \(f\) is not uniformly continuous on \((0, 1]\).
    \end{enumerate}
}

\begin{exercise}
    {4.4.8} Give an example of each of the following, or provide a short argument for why the request is impossible.
    \begin{enumerate}
        \item A continuous function defined on \([0, 1]\) with range \((0, 1)\).
        \item A continuous function defined on \((0, 1)\) with range \([0, 1]\).
        \item A continuous function defined on \((0, 1]\) with range \((0, 1)\).
    \end{enumerate}
\end{exercise}

\sol{
    \begin{enumerate}
        \item \textbf{Impossible}. There does not exist a continuous function defined on \([0, 1]\) with range \((0, 1)\). This is because \([0, 1]\) is a compact set, and the continuous image of a compact set is also compact by the \namrefthm{Preservation of Compact Sets} theorem. However, \((0, 1)\) is not a compact set since it is not closed in \(\mathbb{R}\). Therefore, a continuous function cannot map \([0, 1]\) onto \((0, 1)\).

        \item \textbf{Possible}. An example of a continuous function defined on \( (0, 1) \) with range \( [0, 1] \) is:
              \[
                  f(x) = \frac{1}{2}\sin(4\pi x) + \frac{1}{2}.
              \]
              This is because \(\sin(4\pi x)\) oscillates between \(-1\) and \(1\). Therefore, \(\frac{1}{2}\sin(4\pi x)\) oscillates between \(-\frac{1}{2}\) and \(\frac{1}{2}\). Adding \(\frac{1}{2}\) shifts the range to \([0, 1]\).
        \item \textbf{Impossible}. There does not exist a continuous function defined on \((0, 1]\) with range \((0, 1)\). Since \(x = 1\) is in the domain \((0, 1]\) and \(f\) is continuous at \(x = 1\), the value \(f(1)\) exists and must be in the range \((0, 1)\). However, because \(f\) approaches \(f(1)\) as \(x \to 1^-\), the function attains its supremum at \(x = 1\), meaning \(f(1)\) should be included in the range. But the range \((0, 1)\) excludes its endpoints, leading to a contradiction. Therefore, such a function cannot exist.
    \end{enumerate}
}
