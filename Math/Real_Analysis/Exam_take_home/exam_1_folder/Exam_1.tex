\documentclass[12pt]{article}

\usepackage{amsmath, amssymb, amsthm}
%%% Dimensions and Spacing %%%
\usepackage[margin=1in]{geometry}
\usepackage{setspace}
\linespread{1}


%%% Define new colors %%%
\usepackage{xcolor}
\definecolor{orangehdx}{rgb}{0.96, 0.51, 0.16}

% Normal colors
\definecolor{xred}{HTML}{BD4242}
\definecolor{xblue}{HTML}{4268BD}
\definecolor{xgreen}{HTML}{52B256}
\definecolor{xpurple}{HTML}{7F52B2}
\definecolor{xorange}{HTML}{FD9337}
\definecolor{xdotted}{HTML}{999999}
\definecolor{xgray}{HTML}{777777}
\definecolor{xcyan}{HTML}{80F5DC}
\definecolor{xpink}{HTML}{F690EA}
\definecolor{xgrayblue}{HTML}{49B095}
\definecolor{xgraycyan}{HTML}{5AA1B9}

% Dark colors
\colorlet{xdarkred}{red!85!black}
\colorlet{xdarkblue}{xblue!85!black}
\colorlet{xdarkgreen}{xgreen!85!black}
\colorlet{xdarkpurple}{xpurple!85!black}
\colorlet{xdarkorange}{xorange!85!black}
\definecolor{xdarkcyan}{HTML}{008B8B}
\colorlet{xdarkgray}{xgray!85!black}

% Very dark colors
\colorlet{xverydarkblue}{xblue!50!black}

% Document-specific colors
\colorlet{normaltextcolor}{black}
\colorlet{figtextcolor}{xblue}

% Enumerated colors
\colorlet{xcol0}{black}
\colorlet{xcol1}{xred}
\colorlet{xcol2}{xblue}
\colorlet{xcol3}{xgreen}
\colorlet{xcol4}{xpurple}
\colorlet{xcol5}{xorange}
\colorlet{xcol6}{xcyan}
\colorlet{xcol7}{xpink!75!black}

% Blue-Purple (should just used colorbrewer...)
\definecolor{xrainbow0}{HTML}{e41a1c}
\definecolor{xrainbow1}{HTML}{a24057}
\definecolor{xrainbow2}{HTML}{606692}
\definecolor{xrainbow3}{HTML}{3a85a8}
\definecolor{xrainbow4}{HTML}{42977e}
\definecolor{xrainbow5}{HTML}{4aaa54}
\definecolor{xrainbow6}{HTML}{629363}
\definecolor{xrainbow7}{HTML}{7e6e85}
\definecolor{xrainbow8}{HTML}{9c509b}
\definecolor{xrainbow9}{HTML}{c4625d}
\definecolor{xrainbow10}{HTML}{eb751f}
\definecolor{xrainbow11}{HTML}{ff9709}

%------- %
% XHFILL %
%------- %


%%% FIGURES %%%
\usepackage{graphicx}  
\graphicspath{ {images/} }  
% \numberwithin{figure}{section}
\usepackage{float}
\usepackage{caption}

%%% Hyperlinks %%%
\usepackage{hyperref}
\definecolor{horange}{HTML}{f58026}
\hypersetup{
	colorlinks=true,
	linkcolor=horange,
	filecolor=horange,      
	urlcolor=horange,
}



%-%-%-%-%-%-%-%-%-%-%-%-%-%-%-%-%-%-%-%-%-%-%-%-%-%-%-%-%-%-%-%-%-%-%-%-%-%-%

%% MATH PACKAGES, ENVIRONMENTS, COMMANDS %%
%-%-%-%-%-%-%-%-%-%-%-%-%-%-%-%-%-%-%-%-%-%-%-%-%-%-%-%-%-%-%-%-%-%-%-%-%-%-%
%You'll need your own packages, theorem types, and commands.

\usepackage{tikz}
\usetikzlibrary{shapes,backgrounds,calc,patterns}
\usepackage{amsmath,amsthm} 
\usepackage{mathtools}
\usepackage{amssymb}
\usepackage{mdframed}
\usepackage{mathrsfs}
\usepackage{changepage}
\usepackage{multicol}
\usepackage{slashed}
\usepackage{enumerate}
\usepackage{booktabs}
\usepackage{enumitem}
\usepackage{lipsum}  %This package lets us generate random text for example purposes.


%%% Custom Comands %%%
% Natural Numbers 
\newcommand{\N}{\ensuremath{\mathbb{N}}}

% Whole Numbers
\newcommand{\W}{\ensuremath{\mathbb{W}}}

% Integers
\newcommand{\Z}{\ensuremath{\mathbb{Z}}}

% Rational Numbers
\newcommand{\Q}{\ensuremath{\mathbb{Q}}}

% Real Numbers
\newcommand{\R}{\ensuremath{\mathbb{R}}}

% Complex Numbers
\newcommand{\C}{\ensuremath{\mathbb{C}}}

\newcommand{\I}{\ensuremath{\mathbb{I}}}

\newcommand{\pfs}{\noindent\makebox[\linewidth]{\rule{\textwidth}{0.4pt}}\vspace{0.5cm}}

\newcommand{}{\marginnote{\includegraphics[width=2em]{caution.png}}}

\def \proofDistance {10pt}
\newenvironment{solution}
  {\vspace{10pt}\\ \textit{Solution.} \vspace{10pt}}

\newenvironment{specialsolution}{\vspace{10pt}\\ \textit{Solution.}}

% % Command for theorems proofs with a predefined color
% \newcommand{\epf}[1]{
%     \begin{customframedproof}[linecolor=teal]
%         \begin{proof}
%         #1
%         \end{proof}
%     \end{customframedproof}
% }


\newcommand{\disabs}[1]{\ensuremath{\displaystyle\left|#1\right|}}
\newcommand{\limn}[1]{\ensuremath{\displaystyle \lim_{n \rightarrow \infty}}#1}

\newcommand{\supu}[1]{\sup_{x \in [0,1]} #1}



% end of preamble
%-%-%-%-%-%-%-%-%-%-%-%-%-%-%-%-%-%-%-%-%-%-%-%-%-%-%-%-%-%-%-%-%-%-%-%-%-%-%


	
%-%-%-%-%-%-%-%-%-%-%-%-%-%-%-%-%-%-%-%-%-%-%-%-%-%-%-%-%-%-%-%-%-%-%-%-%-%-%
 %%% COVER PAGE %%%
 %-%-%-%-%-%-%-%-%-%-%-%-%-%-%-%-%-%-%-%-%-%-%-%-%-%-%-%-%-%-%-%-%-%-%-%-%-%-%

\title{Real Analysis: Exam 1}
\author{Paul Beggs}
\date{\today}

%-%-%-%-%-%-%-%-%-%-%-%-%-%-%-%-%-%-%-%-%-%-%-%-%-%-%-%-%


\begin{document}

\maketitle

``All work on this take-home exam is my own.'' \hfill \textbf{Signature:} \underline{\hspace{5cm}}

\vspace{0.5cm}
\begin{enumerate}
    \item For a function \(f \colon D \rightarrow \R\), we define \[\displaystyle \sup_{x \in D} f(x) = \sup\{f(x) \mid x \in D\}.\] This is just the supremum of the image, or the smallest number \(M\) such that \(f(x) \leq M\) for all \(x \in D\). Think of a least upper bound on the graph. \\

          Consider two functions,
          \[f \colon D \rightarrow \R \text{ and } g \colon D \rightarrow \R.\]
          \begin{enumerate}
              \item Prove that \[\displaystyle \sup_{x \in D} (f(x) + g(x)) \leq \sup_{x \in D} f(x) + \sup_{x \in D} g(x).\]
                    \begin{proof}
                        We are given that the function \(\ \sup_{x \in D} f(x)\) is the smallest number \(M\) such that \(f(x) \leq M\) for all \(x \in D\). Similarly, the function \(\ \sup_{x \in D} g(x)\) is the smallest number \(N\) such that \(g(x) \leq N\) for all \(x \in D\). From the definitions of \(M\) and \(N\), it follows that \(f(x) + g(x) \leq M + N\) for each \(x \in D\). Thus, since this holds for every \(x \in D\), \(f(x) + g(x)\) is bounded above by \(M + N\) for all \(x \in D\). Therefore, the supremum of \(f(x) + g(x)\) is also less than or equal to \(M + N\). Then, by substituting the definitions of \(M\) and \(N\), we find that
                        \begin{align*}
                            \displaystyle \sup_{x \in D} \left(f(x) + g(x)\right) & \leq M + N                                               \\
                                                                                  & \leq \sup_{x \in D} f(x) + \sup_{x \in D} g(x). \qedhere
                        \end{align*}
                    \end{proof}
              \item Give an example of two functions \(f\) and \(g\) on the domain \([0,1]\) where the above inequality is strict (that is, the left is \textit{less than but not equal to} the right). \\

                    \textit{Goal: of the computations inside \((f(x) + g(x))\), the largest sum (of those functions) must be less than the largest \(x \in D\) for both \(f(x)\) and \(g(x)\)} \\

                    \textit{Idea: because we proved \(\sup(A + B) = \sup(A) + \sup(B)\), we can not simply substitute \(f(x)\) and \(g(x)\) to be numbers alone. If we were to do so, then we can not find an answer other than that they are equal. Therefore, we need to specify that these functions have variables that can cancel each other out in a way that is unique to the parenthesis, but not for the individual functions.}
                    \begin{specialsolution}
                        If we set the function \(f(x) = x\) and the function \(g(x) = 1 - x\), then we can see that \begin{align*}
                            \sup_{x \in [0,1]}(f(x) + g(x)) & = \sup_{x \in [0,1]}(x + (1 - x)) \\
                                                            & = \sup_{x \in [0,1]}(1)           \\
                                                            & = 1.
                        \end{align*}
                        Now, we can calculate \(\sup_{x \in [0,1]}f(x)\) to be \(1\), when \(x = 1\). Then for \(\supu g(x)\), we see that it is also \(1\), when \(x = 0\). Therefore, we have that \[\sup_{x \in [0,1]}(f(x) + g(x)) = 1 < 2 = \sup_{x \in [0,1]}f(x) + \sup_{x \in [0,1]}g(x).\]
                    \end{specialsolution}
          \end{enumerate}
    \item For each \(j,k \in \N\), define \( a_{j,k} = \begin{cases}
              1 & \text{if } j < k,    \\
              0 & \text{if } j \geq k.
          \end{cases} \)
          \begin{enumerate}
              \item For any fixed \(j \in \N\), we can look at the sequence \( (a_{j,k})_{k=1}^\infty \).
                    \begin{enumerate}
                        \item Describe (by showing enough terms to establish the pattern) the sequences \( (a_{1,k}) \), \( (a_{2,k}) \), \( (a_{3,k}) \), and \( (a_{4,k}) \).
                              \begin{solution}
                                  \begin{align*}
                                      (a_{1,k}) & = (0, 1, 1, 1, 1, \ldots) \\
                                      (a_{2,k}) & = (0, 0, 1, 1, 1, \ldots) \\
                                      (a_{3,k}) & = (0, 0, 0, 1, 1, \ldots) \\
                                      (a_{4,k}) & = (0, 0, 0, 0, 1, \ldots)
                                  \end{align*}
                              \end{solution}
                        \item For any fixed \( j \in \N \), find the limit \( \lim_{k \rightarrow \infty} a_{j,k} \).
                              \begin{solution}
                                  For any fixed \( j \in \N \), the limit of the sequence \( (a_{j,k}) \) is 1.
                              \end{solution}
                        \item Now that you know the limit for each \(j\), find the iterated limit below. \[ \lim_{j \rightarrow \infty} \left(\lim_{k \rightarrow \infty} a_{j,k}\right). \]
                              \begin{solution}
                                  \(\lim_{j \rightarrow \infty} \left(\lim_{k \rightarrow \infty} a_{j,k}\right) = \lim_{j \rightarrow \infty} (1) = 1\).
                              \end{solution}
                    \end{enumerate}
              \item For any fixed \(k \in \N\), we can look at the sequence \( (a_{j,k})_{j=1}^\infty \).
                    \begin{enumerate}
                        \item Describe (by showing enough terms to establish the pattern) the sequences \( (a_{j,1}) \), \( (a_{j,2}) \), \( (a_{j,3}) \), and \( (a_{j,4}) \).
                              \begin{solution}
                                  \begin{align*}
                                      (a_{j, 1}) & = (1, 0, 0, 0, 0, \ldots) \\
                                      (a_{j, 2}) & = (1, 1, 0, 0, 0, \ldots) \\
                                      (a_{j, 3}) & = (1, 1, 1, 0, 0, \ldots) \\
                                      (a_{j, 4}) & = (1, 1, 1, 1, 0, \ldots)
                                  \end{align*}
                              \end{solution}
                        \item For any fixed \( k \in \N \), find the limit \( \lim_{j \rightarrow \infty} a_{j,k} \).
                              \begin{solution}
                                  For any fixed \( k \in \N \), the limit of the sequence \( (a_{j,k}) \) is 0.
                              \end{solution}
                        \item Now that you know the limit for each \(k\), find the iterated limit below. \[ \lim_{j \rightarrow \infty} \left(\lim_{k \rightarrow \infty} a_{j,k}\right). \]
                              \begin{solution}
                                  \(\lim_{j \rightarrow \infty} \left(\lim_{k \rightarrow \infty} a_{j,k}\right) = \lim_{j \rightarrow \infty} (0) = 0\).
                              \end{solution}
                    \end{enumerate}
              \item What is the lesson to learn here about iterated limits?
                    \begin{solution}
                        From these examples, we can see that the order of the limits matters. Moreover, we can observe that, in certain cases \[\lim_{k \rightarrow \infty} \lim_{j \rightarrow \infty} a_{j,k} \ne \lim_{j \rightarrow \infty} \lim_{k \rightarrow \infty} a_{j,k}.\]
                    \end{solution}
          \end{enumerate}

\end{enumerate}





\end{document}
