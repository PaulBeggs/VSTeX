Able to have 3 pages front and back for notes. Focus on definitions, key theorems, and basic proof techniques.

Need to know vocabulary. Like, bounded, continuous, supremum, infinum, etc.

Know if a sequence is Cauchy, convergent, or divergent. Remember, if we can prove a sequence is convergent, then it is Cauchy.

Any theorem that has a name, know it. Like, Bolzano-Weierstrass, Heine-Borel, etc.

Know the hypotheses of the theorems. Like, if a function is continuous on a closed interval, then it is bounded and attains its maximum and minimum.

Proof rule of thumb for sequences: Know if and when to pick a sequence (there exists) vs.~an arbitrary sequence (for all).

Know when a convergent subsequence exists (Bolzano-Weierstrass).

Proof techniques:
\begin{enumerate}
    \item Be able to use the \(N-\epsilon\) definitions of a convergent sequence. 
    \item Be able to use the \(\delta-\epsilon\) definition for functional limits.

    Remember the specific rules for the range of \(\delta\). We do not want to have a fraction that has a 0 in the denominator.
    \item Prove a sequence is both monotone and bounded \(\Rightarrow\) convergent (Monotone Convergence Theorem).
\end{enumerate}
Will include true-false statements. Do you know this is false? Provide a counterexample. Is it true? Prove it.

On take home: part (4) will use part 2.

On part (7), you can assume that \(f(x) = e^{x}\) is differentiable to \(e^{x}\)