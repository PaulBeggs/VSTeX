\section{Discussion: Examples of Dirichlet and Thomae}

\begin{definition}
    The \textit{Dirichlet function} \(\limx{c}{g(x)}\) does not exist for any \(c \in \R\). 
    \[
        g(x) = \begin{cases}
            1 & \text{if } x \in \Q, \\
            0 & \text{if } x \in \R \setminus \Q.
        \end{cases}
    \]
\end{definition}
\begin{center}

\begin{tikzpicture}
        
    \begin{axis}[
        axis lines = middle,
        xlabel = \(x\),
        ylabel = \(g(x)\),
        xmin = -1, xmax = 1,
        ymin = -0.5, ymax = 1.5,
        xtick = {-1, -0.5, 0, 0.5, 1},
        ytick = {0, 1},
    ]
    \addplot [
        domain=-1:1, 
        samples=25, 
        color=blue,
    ]
    {1};
    \addplot [
        domain=-1:1, 
        samples=50, 
        color=red,
    ]
    {0};

\end{axis}

\end{tikzpicture}
\end{center}

\begin{definition}
    The \textit{Thomae function} is defined as 
    \[
        f(x) = \begin{cases}
            \frac{1}{q} & \text{if } x = \frac{p}{q} \text{ is in lowest terms}, \\
            0 & \text{if } x \in \R \setminus \Q.
        \end{cases}
    \]
\end{definition}

\textit{Thomae's function}, \(t(x)\) is continuous at all \(x \notin \Q\). It is not continuous at any \(x \in \Q\).

\newpage

\section{Functional Limits}

Recall from calculus I, that a function \(f(x)\) is continuous at \(x = c\) if \(\lim_{x\rightarrow c}f(x) = f(c)\).

\begin{definition}
    Let \(f \colon A \rightarrow \R\) be a function and let \(c\) be a limit point of \(A\). We say \(\lim_{x \rightarrow c}f(x) = L \), if for all \(\epsilon > 0\), there exists \(\delta > 0\) such that if \(0 < \disabs{x - c} < \delta\), then \(\disabs{f(x) - L} < \epsilon\).
\end{definition}


\begin{example}
    {Functional Limit (From book) 1}Let \(f(x) = 3x + 1\). Claim: \(\limx{2}{f(x)} = 7\).
\end{example}

\pfscratch{
    Let \(\epsilon > 0\). After we have done our scratch work, we can choose \(\delta = \epsilon/ 3\), then \(0 < \disabs{x - 2} < \delta\) implies \(\disabs{f(x) - 7} < 3(\epsilon/3) = 3\).
    }%
{
\defref{4.2.1} requires that we produce a \(\delta > 0\) so that \(0 < \disabs{ x - 2} < \delta\) leads to the conclusion that \(\disabs{f(x) - 7} < \epsilon\). Notice that 
\[
\disabs{f(x) - 7} = \disabs{3x + 1 - 7} = \disabs{3x - 6} = 3\disabs{x - 2}.
\]
}{teal}


\begin{example}
    {Functional Limit (From book) 2}Let \(g(x) = x^{2}\). Claim: \(\limx{2}{g(x)} = 4\).
\end{example}

\pfscratch{
    Let \(\epsilon > 0\). Choose \(\delta = \min\{1,\epsilon/5\}\). If \(0 < \disabs{x - 2} < \delta\), then 
    \begin{align*}
        \disabs{g(x) - 4} &= \disabs{x^{2} - 4} \\
        &= \disabs{x - 2}\disabs{x + 2} \\
        &< 5\delta \\
        &= (5)\frac{\epsilon}{5} \\
        &= \epsilon.
    \end{align*}
}{
    Our goal this time is to make \(\disabs{g(x) - 4} < \epsilon\) by restricting \(\disabs{x - 2}\) to be smaller than some carefully chosen \(\delta\). As in the previous example, a little algebra reveals
    \[
    \disabs{g(x) - 4} = \disabs{x^{2} - 4} = \disabs{x - 2}\disabs{x + 2}.
    \]
    We can make \(\disabs{x + 2}\) as small as we like, but we need an upper bound on \(\disabs{x + 2}\) in order to know how small to choose \(\delta\). The presence of the variable \(x\) causes some initial confusion, but keep in mind that we are discussing the limt as \(x\) approaches 2. If we agree that our \(\delta-\)neighborhood around \(c = 2\) must have radius no bigger than \(\delta = 1\), then we get the upper bound \(\disabs{x + 2} < \disabs{3 + 2} = 5\) for all \(x \in V_{\delta}(c)\).
}{teal}


\begin{example}
    {Functional Limit 1}Let \(f(x) = 3x + 1\). Show that \(\lim_{x \rightarrow 2}f(x) = 7\).
\end{example}

\lepf{
    Let \(\epsilon > 0\). Set \(\delta = \frac{\epsilon}{3}\). Assume \(0 < \disabs{x - 2} < \delta\). Since \(\delta > 0\), \(2 - \delta < x < 2 + \delta\). Then,
    \begin{align*}
        \disabs{x - 2} &< \delta, \\
        \disabs{f(x) - 7} &= \disabs{3x + 1 - 7} \\
        &= \disabs{3x - 6} \\
        &= 3\disabs{x - 2} \\
        &< 3\delta \\
        &= \epsilon.
    \end{align*}
    Therefore, \(\lim_{x \rightarrow 2} f(x) = 7\).
}



\begin{example}
    {Functional Limit 3}Let \(f(x) = x^{2}\). Claim: \(\limx{7}{f(x)} = 49\)
\end{example}

\pfscratch{
    Let \(\epsilon > 0\). Set \(\delta = \min\{\frac{\epsilon}{8}, 1\}\). If \(0 < \disabs{x - 7} < \delta\), then \begin{align*}
        \disabs{f(x) - 49} &= \disabs{x^{2} - 49} \\
        &= \disabs{x - 7}\disabs{x + 7} \\
        &< 8\delta \\
        &= 8\left(\frac{\epsilon}{8}\right) \\
        &= \epsilon.
    \end{align*}
    }%
    {
        Always start with the goal statement: \(\disabs{f(x) - 49} = \disabs{x^{2} - 49}\). This factors into \(\disabs{x - 7} \disabs{x + 7}\). Then, if \(\delta < 1\), \(\disabs{x - 7} < \delta\) and \(\disabs{x + 7} < 8\). All together, we have \(8\delta < \epsilon < \frac{\epsilon}{8}\).
    }%
    {teal}

    \begin{example}
        {Functional Limit 4}Claim: \(\limx{3}{\frac{1}{x + 1}} = \frac{1}{4}\).
    \end{example}

    \pfscratch{
        Let \(\epsilon > 0\). Set \(\delta = \min\{12\epsilon,1\}\). If \(0 < \disabs{x - 3} < \delta\), then \begin{align*}
            \disabs{\frac{1}{x + 1} - \frac{1}{4}} &= \disabs{\frac{4 - (x + 1)}{4(x + 1)}} \\
            &= \disabs{\frac{3 - x}{4(x + 1)}} \\
            &< \frac{\delta}{4(3)} \\
            &= \frac{12\epsilon}{12} \\
            &= \epsilon.
        \end{align*}
        Therefore, \(\limx{3}{\frac{1}{x + 1}} = \frac{1}{4}\)
        }{
        Goal: \(\disabs{\frac{1}{x + 1} - \frac{1}{4}}\). Hence,
        \begin{align*}
            \disabs{\frac{1}{x + 1} - \frac{1}{4}} &= \disabs{\frac{4 - (x + 1)}{4(x + 1)}} \\
            &= \disabs{\frac{3 - x}{4(x + 1)}} \\
            &< \frac{\delta}{4\disabs{x + 1}} \\
            &< \frac{\delta}{4(3)} \\
            &= \frac{\delta}{12} \\
            &< \epsilon.
        \end{align*}
        Thus, we need a \(\delta < 1\), and we can choose \(\delta = \min\{12\epsilon,1\}\). Note: When we are determining the value for \(\disabs{x + 2}\), we solve for \(\delta = 3 \pm 1 \Rightarrow x \in (2,4)\). Then, we find \(x + 1 = (3,5)\). We choose 3 rather than 5 because of division. We want to be as close as possible.
    }{teal}

\begin{example}
    {Functional Limit 5}Claim: \(\limx{3}{(x^{2} + 7x)} = 30\).
\end{example}

\lepf{
    Let \(\epsilon > 0\) and set \(\delta = \min\{\frac{\epsilon}{14}, 1\}\). If \(0 < \disabs{x - 3} < \delta\), then
    \begin{align*}
        \disabs{x^{2} + 7x - 30} &= \disabs{x - 3}\disabs{x + 10} \\
        &< 14\delta \\
        &= 14\left(\frac{\epsilon}{14}\right) \\
        &= \epsilon.
    \end{align*}
}

\begin{example}
    {Functional Limit 6}Claim: \(\limx{3}{\frac{2x + 3}{4x - 9}} = 3\).
\end{example}

\lepfscratch{
    Let \(\epsilon > 0\). Set \(\delta = \min\{\frac{\epsilon}{10}, \frac{1}{2}\}\). (Note: We are choosing \(\frac{1}{2}\) because we want to avoid having 0 anywhere in the interval.) Assume \(0 < \disabs{x - 3} < \delta\). Since \(\delta < \frac{1}{2}\), \(\frac{5}{2} < x < \frac{7}{2}\), then \(1 < \disabs{4x - 9} < 5\). (Thus, 0 can not possibly be in the denominator.)
}{
    \begin{align*}
        \disabs{\frac{2x + 3}{4x + 9} - 3} &= \disabs{\frac{2x + 3 - 3(4x + 9)}{4x + 9}} \\
        &= \disabs{\frac{2x + 3 - 12x - 27}{4x + 9}} \\
        &= 10\disabs{\frac{x-3}{4x - 4}} \\
        &< 10 \frac{\epsilon / 10}{1} \\ 
        &= \epsilon.
\end{align*}}

\begin{example}
    {Functional Limit 7}Claim: \(\limx{4}{\sqrt{x}} = 2\).
\end{example}

\lepfscratch{
    Let \(\epsilon > 0\). Set \(\delta = \min\{1,3\epsilon\}\). Assume \(0 < \disabs{x - 4} < \delta\). Then (refer to scratch work).
}{
    \begin{align*}
        \disabs{\sqrt{x} - 2} &= \disabs{\sqrt{x} - 2} \\
        &= \disabs{\frac{(\sqrt{x} - 2) \cdot (\sqrt{x} + 2)}{\sqrt{x} + 2}} \\
        &= \disabs{\frac{x - 4}{\sqrt{x} + 2}} \\
        &< \frac{\delta}{3} \\
        &< \frac{3\epsilon}{3} \\
        &= \epsilon.
    \end{align*}
    Notice that we picked \(\delta < 1\) such that \(3 < x < 4\) so \(1 < \sqrt{x} < 2\) and \(3 < \sqrt{x} + 2 < 4\).
}

