\section{Discussion: Examples of Dirichlet and Thomae}

\begin{definition}
    The \textit{Dirichlet function} \(\limx{c}{g(x)}\) does not exist for any \(c \in \R\). 
    \[
        g(x) = \begin{cases}
            1 & \text{if } x \in \Q, \\
            0 & \text{if } x \in \R \setminus \Q.
        \end{cases}
    \]
\end{definition}
\begin{center}

\begin{tikzpicture}
        
    \begin{axis}[
        axis lines = middle,
        xlabel = \(x\),
        ylabel = \(g(x)\),
        xmin = -1, xmax = 1,
        ymin = -0.5, ymax = 1.5,
        xtick = {-1, -0.5, 0, 0.5, 1},
        ytick = {0, 1},
    ]
    \addplot [
        domain=-1:1, 
        samples=25, 
        color=blue,
    ]
    {1};
    \addplot [
        domain=-1:1, 
        samples=50, 
        color=red,
    ]
    {0};

\end{axis}

\end{tikzpicture}
\end{center}

\begin{definition}
    The \textit{Thomae function} is defined as 
    \[
        f(x) = \begin{cases}
            \frac{1}{q} & \text{if } x = \frac{p}{q} \text{ is in lowest terms}, \\
            0 & \text{if } x \in \R \setminus \Q.
        \end{cases}
    \]
\end{definition}

\textit{Thomae's function}, \(t(x)\) is continuous at all \(x \notin \Q\). It is not continuous at any \(x \in \Q\).

\newpage

\section{Functional Limits}

Recall from calculus I, that a function \(f(x)\) is continuous at \(x = c\) if \(\lim_{x\rightarrow c}f(x) = f(c)\).

\begin{definition}
    Let \(f \colon A \rightarrow \R\) be a function and let \(c\) be a limit point of \(A\). We say \(\lim_{x \rightarrow c}f(x) = L \), if for all \(\epsilon > 0\), there exists \(\delta > 0\) such that if \(0 < \disabs{x - c} < \delta\), then \(\disabs{f(x) - L} < \epsilon\).
\end{definition}


\begin{example}
    {Functional Limit (From book) 1}Let \(f(x) = 3x + 1\). Claim: \(\limx{2}{f(x)} = 7\).
\end{example}

\pfscratch{
    Let \(\epsilon > 0\). After we have done our scratch work, we can choose \(\delta = \epsilon/ 3\), then \(0 < \disabs{x - 2} < \delta\) implies \(\disabs{f(x) - 7} < 3(\epsilon/3) = 3\).
    }%
{
\defref{4.2.1} requires that we produce a \(\delta > 0\) so that \(0 < \disabs{ x - 2} < \delta\) leads to the conclusion that \(\disabs{f(x) - 7} < \epsilon\). Notice that 
\[
\disabs{f(x) - 7} = \disabs{3x + 1 - 7} = \disabs{3x - 6} = 3\disabs{x - 2}.
\]
}{teal}


\begin{example}
    {Functional Limit (From book) 2}Let \(g(x) = x^{2}\). Claim: \(\limx{2}{g(x)} = 4\).
\end{example}

\pfscratch{
    Let \(\epsilon > 0\). Choose \(\delta = \min\{1,\epsilon/5\}\). If \(0 < \disabs{x - 2} < \delta\), then 
    \begin{align*}
        \disabs{g(x) - 4} &= \disabs{x^{2} - 4} \\
        &= \disabs{x - 2}\disabs{x + 2} \\
        &< 5\delta \\
        &= (5)\frac{\epsilon}{5} \\
        &= \epsilon.
    \end{align*}
}{
    Our goal this time is to make \(\disabs{g(x) - 4} < \epsilon\) by restricting \(\disabs{x - 2}\) to be smaller than some carefully chosen \(\delta\). As in the previous example, a little algebra reveals
    \[
    \disabs{g(x) - 4} = \disabs{x^{2} - 4} = \disabs{x - 2}\disabs{x + 2}.
    \]
    We can make \(\disabs{x + 2}\) as small as we like, but we need an upper bound on \(\disabs{x + 2}\) in order to know how small to choose \(\delta\). The presence of the variable \(x\) causes some initial confusion, but keep in mind that we are discussing the limt as \(x\) approaches 2. If we agree that our \(\delta-\)neighborhood around \(c = 2\) must have radius no bigger than \(\delta = 1\), then we get the upper bound \(\disabs{x + 2} < \disabs{3 + 2} = 5\) for all \(x \in V_{\delta}(c)\).
}{teal}


\begin{example}
    {Functional Limit 1}Let \(f(x) = 3x + 1\). Show that \(\lim_{x \rightarrow 2}f(x) = 7\).
\end{example}

\lepf{
    Let \(\epsilon > 0\). Set \(\delta = \frac{\epsilon}{3}\). Assume \(0 < \disabs{x - 2} < \delta\). Since \(\delta > 0\), \(2 - \delta < x < 2 + \delta\). Then,
    \begin{align*}
        \disabs{x - 2} &< \delta, \\
        \disabs{f(x) - 7} &= \disabs{3x + 1 - 7} \\
        &= \disabs{3x - 6} \\
        &= 3\disabs{x - 2} \\
        &< 3\delta \\
        &= \epsilon.
    \end{align*}
    Therefore, \(\lim_{x \rightarrow 2} f(x) = 7\).
}



\begin{example}
    {Functional Limit 3}Let \(f(x) = x^{2}\). Claim: \(\limx{7}{f(x)} = 49\)
\end{example}

\pfscratch{
    Let \(\epsilon > 0\). Set \(\delta = \min\{\frac{\epsilon}{8}, 1\}\). If \(0 < \disabs{x - 7} < \delta\), then \begin{align*}
        \disabs{f(x) - 49} &= \disabs{x^{2} - 49} \\
        &= \disabs{x - 7}\disabs{x + 7} \\
        &< 8\delta \\
        &< 8\left(\frac{\epsilon}{8}\right) \\
        &= \epsilon.
    \end{align*}
    }%
    {
        Always start with the goal statement: \(\disabs{f(x) - 49} = \disabs{x^{2} - 49}\). This factors into \(\disabs{x - 7} \disabs{x + 7}\). Then, if \(\delta < 1\), \(\disabs{x - 7} < \delta\) and \(\disabs{x + 7} < 8\). All together, we have \(8\delta < \epsilon < \frac{\epsilon}{8}\).
    }%
    {teal}

    \begin{example}
        {Functional Limit 4}Claim: \(\limx{3}{\frac{1}{x + 1}} = \frac{1}{4}\).
    \end{example}

    \pfscratch{
        Let \(\epsilon > 0\). Set \(\delta = \min\{12\epsilon,1\}\). If \(0 < \disabs{x - 3} < \delta\), then \begin{align*}
            \disabs{\frac{1}{x + 1} - \frac{1}{4}} &= \disabs{\frac{4 - (x + 1)}{4(x + 1)}} \\
            &= \disabs{\frac{3 - x}{4(x + 1)}} \\
            &< \frac{\delta}{4(3)} \\
            &= \frac{12\epsilon}{12} \\
            &= \epsilon.
        \end{align*}
        Therefore, \(\limx{3}{\frac{1}{x + 1}} = \frac{1}{4}\)
        }{
        Goal: \(\disabs{\frac{1}{x + 1} - \frac{1}{4}}\). Hence,
        \begin{align*}
            \disabs{\frac{1}{x + 1} - \frac{1}{4}} &= \disabs{\frac{4 - (x + 1)}{4(x + 1)}} \\
            &= \disabs{\frac{3 - x}{4(x + 1)}} \\
            &< \frac{\delta}{4\disabs{x + 1}} \\
            &< \frac{\delta}{4(3)} \\
            &= \frac{\delta}{12} \\
            &< \epsilon.
        \end{align*}
        Thus, we need a \(\delta < 1\), and we can choose \(\delta = \min\{12\epsilon,1\}\). Note: When we are determining the value for \(\disabs{x + 2}\), we solve for \(\delta = 3 \pm 1 \Rightarrow x \in (2,4)\). Then, we find \(x + 1 = (3,5)\). We choose 3 rather than 5 because of division. We want to be as close as possible.
    }{teal}

\begin{example}
    {Functional Limit 5}Claim: \(\limx{3}{(x^{2} + 7x)} = 30\).
\end{example}

\lepf{
    Let \(\epsilon > 0\) and set \(\delta = \min\{\frac{\epsilon}{14}, 1\}\). If \(0 < \disabs{x - 3} < \delta\), then
    \begin{align*}
        \disabs{x^{2} + 7x - 30} &= \disabs{x - 3}\disabs{x + 10} \\
        &< 14\delta \\
        &= 14\left(\frac{\epsilon}{14}\right) \\
        &= \epsilon.
    \end{align*}
}

\begin{example}
    {Functional Limit 6}Claim: \(\limx{3}{\frac{2x + 3}{4x - 9}} = 3\).
\end{example}

\lepfscratch{
    Let \(\epsilon > 0\). Set \(\delta = \min\{\frac{\epsilon}{10}, \frac{1}{2}\}\). (Note: We are choosing \(\frac{1}{2}\) because we want to avoid having 0 anywhere in the interval.) Assume \(0 < \disabs{x - 3} < \delta\). Since \(\delta < \frac{1}{2}\), \(\frac{5}{2} < x < \frac{7}{2}\), then \(1 < \disabs{4x - 9} < 5\). (Thus, 0 can not possibly be in the denominator.)
}{
    \begin{align*}
        \disabs{\frac{2x + 3}{4x + 9} - 3} &= \disabs{\frac{2x + 3 - 3(4x + 9)}{4x + 9}} \\
        &= \disabs{\frac{2x + 3 - 12x - 27}{4x + 9}} \\
        &= 10\disabs{\frac{x-3}{4x - 4}} \\
        &< 10 \frac{\epsilon / 10}{1} \\ 
        &= \epsilon.
\end{align*}}

\begin{example}
    {Functional Limit 7}Claim: \(\limx{4}{\sqrt{x}} = 2\).
\end{example}

\lepfscratch{
    Let \(\epsilon > 0\). Set \(\delta = \min\{1,3\epsilon\}\). Assume \(0 < \disabs{x - 4} < \delta\). Then (refer to scratch work).
}{
    \begin{align*}
        \disabs{\sqrt{x} - 2} &= \disabs{\sqrt{x} - 2} \\
        &= \disabs{\frac{(\sqrt{x} - 2) \cdot (\sqrt{x} + 2)}{\sqrt{x} + 2}} \\
        &= \disabs{\frac{x - 4}{\sqrt{x} + 2}} \\
        &< \frac{\delta}{3} \\
        &< \frac{3\epsilon}{3} \\
        &= \epsilon.
    \end{align*}
    Notice that we picked \(\delta < 1\) such that \(3 < x < 4\) so \(1 < \sqrt{x} < 2\) and \(3 < \sqrt{x} + 2 < 4\).
}


\begin{ntheorem}
    {Sequential Criterion for Functional Limits}
    The following statements are equivalent:

    \begin{enumerate}
        \item \(\lim_{x \rightarrow c} f(x) = L\).
        \item For all sequences \((x_n)\) where \(x_n \ne c\) and \((x_n) \rightarrow c\), we have \(\limn{f(x_n)} = L\).
    \end{enumerate}
\end{ntheorem}

\tpf{
    \((1) \rightarrow (2)\) \\

    Assume \(\limc{f(x)} = L\).

    Let \((x_n) \rightarrow c\) with \(x_n \ne c\)

    Let \(\epsilon > 0\).
    \begin{itemize}
        \item Since \(\limc{f(x)} = L\), there exists \(\delta > 0\) such that if \(0 < \disabs{x - c} < \delta\), then \(\disabs{f(x) - L} < \epsilon\).

        \item Since \(x_n \rightarrow c\), there exists \(N \in \N\) such that for all \(n \geq N\), \(\disabs{x_n - c} < \delta\).

        \item Now, for all \(n \geq N\), it follows that \(x_n - c < \delta\) and thus \(\disabs{f(x) - L} < \epsilon\).
    \end{itemize}

    Thus, \(\limn{f(x_n)} = L\). \\

    \((2) \rightarrow (1)\) \\

    Proof by contrapositive.

    Assume (1) is not true. Thus,
    \[
        \limc{f(x)} \ne L.
    \]

    There exists \(\epsilon_0 > 0\) such that for all \(\delta > 0\), there exists an \(x\) with \(0 < \disabs{x - c} < \delta\) and \(\disabs{f(x) - L} \geq \epsilon_0\).

    For each \(n \in \N\), consider \(\delta = \frac{1}{n}\). There exists \(x_n \in (c - \frac{1}{n}, c + \frac{1}{n})\) with \(x_n \ne c\) such that \(\disabs{f(x) - L} \geq \epsilon_0\).

    \begin{itemize}
        \item Since \(\disabs{x_n - c} < \frac{1}{n}\), we see that \((x_n) \rightarrow c\).
        \item Since for all \(\ninn\), \(\disabs{f(x) - L} \geq \epsilon_0\). Then, \(\limn{f(x)} \ne L\).
    \end{itemize}
    Thus, \(\neg(1) \rightarrow \neg(2)\). So \((2) \rightarrow (1)\) and \((1) \rightarrow (2)\).
}

If functional limits and sequential limits are the same thing, then everything we know about sequential limits is also true about functional limits.

Recall \namrefthm{Algebraic Limit Theorem}. From this, we can write the functional equivalent:

Assume \(\limc{f(x)} = L\) and \(\limc{g(x)} = M\). Then,
\begin{itemize}
    \item \(\limc{(f(x) + g(x))} = L + M\)
    \item \(\limc{(f(x) - g(x))} = L - M\)
    \item \(\limc{(f(x)g(x))} = LM\)
    \item \(\limc{(\frac{f(x)}{g(x)})} = \frac{L}{M}\) unless \(M = 0\).
\end{itemize}

\begin{ntheorem}
    {Divergence Criterion} Let \(f \colon A \rightarrow \R\) with \(c\) as a limit point of \(A\). If there exists two sequences \((x_n)\) and \((y_n)\) in \(A \setminus \{c\}\) (that both converge to \(c\)) such that \(\limn{f(x_n)} \ne \limn{f(y_n)}\), then \(\limc{f(x)}\) does not exist.
\end{ntheorem}

\begin{example}
    {Divergence Criterion 1} \(f(x) = \frac{x}{|x|} = \begin{cases}
        1, \text{ if } x > 0 \\
        -1, \text{ if } x < 0
    \end{cases}\)
    Our goal is to show that \(\lim_{x \rightarrow 0} f(x)\) does not exist.
\end{example}

\lepf{
    Let \((x_n) = (\frac{1}{n})\) and let \((y_n) = (\frac{-1}{n})\). We will see that as \(n \rightarrow \infty\), \(\limn{f(x_n)} = 1\) and \(\limn{f(y_n)} = -1\). Thus, \(\lim_{x \rightarrow 0} f(x)\) does not exist.
}

\begin{example}
    {Divergence Criterion 2} \(g(x) = \sin(\frac{1}{x})\). Show that \(\lim_{x \rightarrow 0} g(x)\) does not exist.
\end{example}

\lepf{
    Let \((x_n) = (\frac{1}{2\pi n})\) and let \((y_n) = (\frac{1}{2\pi n + \frac{\pi}{2}})\). We will see that as \(n \rightarrow \infty\), \(\limn{g(x_n)} = 1\) and \(\limn{g(y_n)} = -1\). Thus, \(\lim_{x \rightarrow 0} g(x)\) does not exist.
}

(section?) Infinite limits

We say \(\limn{x_n} = \infty\) if for all \(M > 0\), there exists \(N \in \N\) such that for all \(n \geq N\), \(x_n > M\).

We say \(\limc{f(x)} = \infty\) if for all \(M > 0\), there exists \(\delta > 0\) such that if \(0 < \disabs{x - c} < \delta\), then \(f(x) > M\). Think of vertical asymptotes.


\begin{ntheorem}
    {Infinite Limits Cauchy Criterion} If \((x_n) \rightarrow \infty\), \((x_n)\) will not be Cauchy. It is possible to have \(x_{n+1} - x_n\) approach 0, but \((x_n)\) converges to \(\infty\).
\end{ntheorem}

\section{Continuous Functions}

\begin{definition}
    We say a function \(f\) is \textit{continuous} at \(c\) if
    \[
        \limc{f(x)} = f(c).
    \]

    Equivalent definition:

    For all \(\epsilon > 0\), there exists \(\delta > 0\) such that if \(\disabs{x - c} < \delta\), then
    \[
        \disabs{f(x) - f(c)} < \epsilon.
    \]
\end{definition}

Idea: When \(x\) is close to \(c\), \(f(x)\) is close to \(f(c)\). Then, for the topological definition, we can say if \(x \in V_\delta(c)\) then \(f(x) \in V_\epsilon(f(c))\).

\begin{definition}
    We say function \(f\) is \textit{continuous} on a set \(D\) if \(f\) is continuous at every point in \(D\). 
\end{definition}

The following are equivalent (TFAE):

\begin{enumerate}
    \item \(\limc{f(x)} = L\)
    \item For all sequences \((x_n)\) such that \((x_n) \rightarrow c\), we have \(\limn{f(x_n)} = L\).
\end{enumerate}

\begin{center}
    \underline{Continuous Functions} (Thm 4.3.2 in book)
\end{center}

Claim: Let \(a \in \R\). Then \(f(x) = a\) is continuous.

\begin{customframedproof}[linecolor=xgray]
    \begin{proof}
        Let \(c \in \R\). Let \(\epsilon > 0\). Set \(\delta = 8\). Now, if \(x - c < 8\), then \(\disabs{f(x) - f(c)} = a - a = 0 < \epsilon\). Thus, constant functions are continuous.
    \end{proof}
\end{customframedproof}

Claim: \(f(x) = x\) is continuous.

\begin{customframedproof}[linecolor=xgray]
    \begin{proof}
        Let \(c \in \R\). Let \(\epsilon > 0\). Set \(\delta = \epsilon\). If \(\disabs{x - c} < \delta\), then \(\disabs{f(x) - f(c)} = \disabs{x - c} < \delta = \epsilon\). Thus, the identity function is continuous.
    \end{proof}
\end{customframedproof}

% \namrefthm{Algebraic Limit Theorem} tells us that if \(f\) and \(g\) are continuous at \(c\), then \(f + g\), \(f - g\), \(fg\), and \(\frac{f}{g}\) (so long as \(g \ne 0\)) are continuous at \(c\). From this, we know that polynomials are continuous. We'll show that in the next theorem.

% \begin{ntheorem}
%     {Thm. 4.3.2 in book}
% \end{ntheorem}

Claim: \(g(x) = \sqrt{x}\) is continuous on \([0,\infty)\).

\begin{customframedproof}[linecolor=xgray]
    \begin{proof}
        \begin{itemize}
            \item\hfill \textbf{Case 1:} \(c \ne 0\) \\

            Let \(c \in [0,\infty)\). Let \(\epsilon > 0\). Set \(\delta < \epsilon\). If \(\disabs{x - c} < \delta\), then \(\disabs{g(x) - g(c)} = \disabs{\sqrt{x} - \sqrt{c}} = \frac{\disabs{x - c}}{\sqrt{x} + \sqrt{c}} < \frac{\delta}{\sqrt{c}} < \epsilon\). Thus, \(g(x) = \sqrt{x}\) is continuous on \([0,\infty)\).

            \item \textbf{Case 2:} \(c = 0\) \\

            Let \(c \in [0,\infty)\). Let \(\epsilon > 0\) Set \(\delta = \epsilon^2\). If \(\disabs{x - 0} < \delta\), then \(\disabs{g(x) - g(0)} = \disabs{\sqrt{x} - 0} = \sqrt{x} < \sqrt{\delta} = \epsilon\). Thus, \(g(x) = \sqrt{x}\) is continuous on \([0,\infty)\).
        \end{itemize}
        
    \end{proof}
\end{customframedproof}

