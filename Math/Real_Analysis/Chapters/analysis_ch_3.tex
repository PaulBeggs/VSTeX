\renewcommand{\theenumi}{\arabic{enumi}}
\renewcommand{\labelenumi}{\theenumi.}
\section{The Cantor Set}

We will build this set through an iterative process. Start with a number line \(C_0\) that stretches from 0 to 1. Remove the middle third of the interval, leaving two intervals of length \(\frac{1}{3}\). We will call the set of points removed from \(C_0\) \(C_1\). Next, remove the middle third of each of the two intervals, leaving four intervals of length \(\frac{1}{9}\). We will call the set of points removed from \(C_1\) \(C_2\). Continue this process indefinitely. 

% Put a tikz picture here that shows the cantor set at each step

\begin{definition}
    The \textit{Cantor set}, \(C\), is defined as \(C = \bigcap_{n=0}^\infty C_n\). This set is 
    \begin{enumerate}
        \item Non-empty. All end points stay within the interval.
        \item \(C\) is uncountable.
    \end{enumerate}
\end{definition}

The second part of that definition is a bit tricky to prove, but a visual will do for now. We can put all elements of the Cantor set in a one-to-one correspondence with the set of all 0s and 1s. This shows that not only is it uncountable, but it also has the same cardinality as \([0,1]\). % Put a tikz picture here from the notes.

The total length of removed elements, \(\frac{1}{3} + \frac{2}{9} + \frac{4}{27} + \frac{8}{81} + \cdots = \frac{1}{3}(1 + \frac{2}{3} + \frac{4}{9})\). Notice the resemblance to the geometric series? We can write this as \[\frac{1}{3} \sum_{n=0}^\infty \left(\frac{2}{3}\right)^n = \frac{1}{3} \left(\frac{1}{1 - \frac{2}{3}}\right) = 1.\] 

In summary, \begin{enumerate}
    \item Start with intercal \([0,1]\).
    \item Remove countably disjoint intervals.
    \item Uncountably many points between these intervals, all isolated from each other.
    \item The space taken up by the leftover points has ``length'' 0.
\end{enumerate}

When we review the properties of a fractal, we see that the Cantor set is a fractal. It is self-similar, and the dimension of the Cantor set is \(\log_3 2\).

\begin{note}
    For a cool look at the cantor set as a fractal, check out \url{https://en.wikipedia.org/wiki/File:Cantor_Set_Expansion.gif}.
\end{note}

\renewcommand{\theenumi}{\arabic{enumi}}
\renewcommand{\labelenumi}{\theenumi.}
\section{Open and closed Sets}

\begin{definition}
    For a point \(x \in \R\), and \(\epsilon > 0\), we define the \textit{epsilon-neighborhood} of \(x\) to be \(V_\epsilon(x) = \{y \in \R \mid |x - y| < \epsilon\}\).
\end{definition}

\subsection{Open Sets}

\begin{definition}
    A set \(A \subseteq \R\), is called an \textit{open set} if for every \(x \in A\), there exists an \(\epsilon > 0\) such that \(V_\epsilon(x) \subset A\).
\end{definition}

\subsubsection{Some Examples of Open Sets}

\begin{multicols}{2}
    \begin{itemize}
        \item All open intervals are also open sets.
        \item \(\R\) is open.
        \item \(\emptyset\) is open.
        \item \(\{1\}\) is not open.
        \item \([0,2]\) is not open.
        \item \(\Q\) is not open.
        \item \([4,6)\) is not open.
        \item \((0,1) \cup (1,3) \cup (5,10)\) is open.
        \item \((0,3] \cap [2,4)\) is open.
        \item Canter set is not open.
    \end{itemize}
\end{multicols}

\begin{theorem}
    \begin{enumerate}[label=(\roman*)]
        \item The union of an arbitrary collection of open sets is open.
        \item The intersection of a finite collection of open sets is open. 
    \end{enumerate}
\end{theorem}

\tpf{
    \textit{(ii)}. Let \(A_1, A_2, \dots, A_n\) be open sets. Let \(x \in \bigcap_{n = 1}^n A_i\). Then for each \(i \leq n\), \(x \in A_i\). Since each \(A_i\) is open, there exists \(\epsilon > 0\) such that \(V_\epsilon(x) \subset A\). Let \(\epsilon = \min\{\epsilon, \epsilon_i, \epsilon_n\} > 0\). Then, for all \(i \leq n, \ V_\epsilon(x) \subseteq V_\epsilon(x) \subseteq A_i\).
}

Note that we cannot use this for cases with infinity. For example, consider \(A_n = (-\frac{1}{n}, \frac{1}{n})\). This is open, but \(\bigcap_{n = 1}^\infty\ A_n = \{0\}\), which is not open. 

\subsection{Closed Sets}

\begin{definition}
    Let \(A \subseteq \R\). We say \(x\) is a \textit{limit point} of \(A\) if for all \(\epsilon > 0\), there exists \(a \in A\) such that \(a \in V_\epsilon(x)\) that is not \(x\). Additionally, a point \(x \in \R\) is a \textit{limit point} if, and only if, there exists a sequence \((a_n)\) of points from \(A\) that are not \(x\). But \(\limn(a_n) = x\). 
\end{definition}

\begin{definition}
    A set \(B \subseteq \R\) is called a \textit{closed set} if \(B\) contains all its limit points. 
\end{definition}

\begin{note}
    \textbf{Important note:} Limit points could be outside a set. Consider \((0,1)\). Even though 0 and 1 do not belong to the set, they are considered limit points that are outside the set.
\end{note}

\subsubsection{Some Examples of Closed Sets}
\begin{multicols}{2}
    \begin{itemize}
        \item \([0,1]\) is closed.
        \item \((0,1)\) is not closed.
        \item \(\R\) is closed.
        \item \(\emptyset\) is closed.
        \item \(\Q\) is not closed.
        \item \([3, \infty)\) is closed.
        \item \(\frac{1}{n} \mid n \in \N\) not closed. (Because of 0)
        \item \([1,4] \cup \{8\}\) is closed. 
        \item \(\{1\}\) is closed. 
        \item \([1,2)\) is not closed. Note that this set is neither open or closed. 
    \end{itemize}
\end{multicols}

\begin{theorem}
    A set \(B \subseteq \R\) is closed if, and only if, its complement is open. Similarly, a set \(A \in \R\) is open if, and only if, its complement is closed.
\end{theorem}

\iffpf{Assume \(B \subseteq \R\) is a closed set. We will show that \(B^c\) is open. Let \(x \in B^c\). So, \(x \notin B\). This means \(x\) is not a limit point. (From the negated definition of limit point:) There must exist \(\epsilon >0\) such that no elements of \(B\) belong to \(V_\epsilon(x)\). Then, \(V_\epsilon(x) \subseteq B^c\). Therefore, \(B^c\) is open.}{Assume \(B^c\) is open. We will show that \(B\) is closed. Let \(x\) be a limit point of \(B\). For all \(\epsilon > 0\), there exists a \(b \in B\) such that \(b \subseteq V_\epsilon(x)\). So, \(V_\epsilon(x)\) is not a subset of \(B^c\). This is true for every \(\epsilon\). Since \(B^c\) is open, it must be that \(x \notin B^c\). Thus, \(x \in B\). So, \(B\) contains all its limit points. Therefore, \(B\) is closed.\qedhere}{}{xblue}

\begin{definition}
    Let \(A \subseteq \R\) and let \(L\) be the set of limit points of \(A\). The \textit{closure} of \(A\) is defined as \(\bar{A} = A \cup L\).
\end{definition}

\begin{theorem}
    \begin{enumerate}[label=(\roman*)]
        \item The intersection of any collection of closed sets is closed. 
        \item The union of finitely many closed sets is closed.
    \end{enumerate}
\end{theorem}

\begin{theorem}
    The closure of a set is a closed set.
\end{theorem}

\textbf{Note:} This theorem may seem trivial, but it answers the question of ``Are there limit points in \(L\) that are not accounted for?''

\tpf{
    We need to show that \(\bar{A}\) contains all the limit points of \(\bar{A}\). Let \(L\) be the limit points of \(A\). Thus, \(\bar{A} = A \cup L\). Let \(x\) be a limit point of \(\bar{A}\). There exists a sequence of points \((x_n)\) coming from \(\bar{A}\) such that \((x_n) \rightarrow x\). Then, for all \(n \in \N\), either \(x_n \in A\) or \(x_n \in L\).
    \begin{itemize}
        \item \textbf{Case 1:} \(x_n \in A\) \\
        There exists a subsequence \((x_{n_k})\) where each \(x_{n_k} \in A\). This subsequence also converges to \(x\), and we know the limit belongs to \(L\), so \(x \in L \subseteq \bar{A}\).

        \item \textbf{Case 2:} \(x_n \in L\) \\
        \(x_n\) belongs to \(A\) for only finitely many \(n \in \N\). Thus, a tail-end of the sequence is comprised entirely of points from \(L\). To simplify things, we will assume the entire sequence \((x_n)\) comes from \(L\). (We know that \((x_n)\) converges to \(x\), but we cannot assume those limit points converge as well.) Let \(n \in \N\). Since \(x_n \in L\), there exists \(a_n \in A\) such that \(\disabs{x_n - a_n} < \frac{1}{n}\). We now have \((x_n) \rightarrow x\) and \((x_n - a_n) \rightarrow 0\). Then, \((a_n) \rightarrow x\). Thus, \(x \in L \subseteq \bar{A}\).
    \end{itemize}
    Now that we have shown that either cases leads to the same conclusion, we know that \(x \in L \subseteq \bar{A}\), and therefore \(\bar{A}\) contains all its limit points.
}

\begin{theorem}
    The closure set \(A\) is the \textit{smallest} closed set containing \(A\). (Where ``smallest'' refers to a subset of any other closed set containing \(A\).)
\end{theorem}

\tpf{
    If \(B\) is a closed set containing \(A\), then \(A \subseteq B\) and \(L \subseteq B\). Thus, \(\bar{A} = A \cup L \subseteq B\).
}

\begin{example}
    {Closed Sets 1}Generate countably many closed sets where the union is not closed.
\end{example}

\lesol{
    \(B_n = [\frac{1}{n}, 1 - \frac{1}{n}]\). Therefore, \(\bigcup_{n = 3}^\infty B_n = (0,1)\). For example, that would look like: \(\{\frac{1}{2}\} \cup \{\frac{1}{3}\} \cup \dots\)
}

\begin{example}
    {Closed Sets 2}What is the closure of the following sets? 
    \[
    (\text{a}) \ (0,1), \quad \text{(b)} \ \R, \quad (\text{c}) \ \left\{\frac{1}{n} \mid n \in \N\right\}, \quad (\text{d}) \ [0,1) \cup (1,3], \quad (\text{e}) \ \Q
    \]
\end{example}

\lesol{
    (a) \(\bar{A} = [0,1]\), \quad (b) \(\bar{A} = \R\), \quad (c) \( \bar{A} = A \cup \{0\}\), \quad (d) \(\bar{A} = [0,3]\), \quad (e) \(\bar{A} = \R\).
}

\renewcommand{\theenumi}{\alph{enumi}}
\renewcommand{\labelenumi}{(\theenumi)}
\subsection{Exercises}

\begin{exercise}
    {3.2.4}Let \(A\) be a nonempty and bounded above set so that \(s = \sup(A)\) exists. (See \defref{1.3.2} and \defref{3.2.7})
    \begin{enumerate}
        \item Show that \(s \in \bar{A}\).
        \item Can an open set contain its supremum?
    \end{enumerate}
\end{exercise}

\sol{
    \begin{enumerate}
        \item Let \(s\) be the suprema of \(A\). Because \(A\) is bounded above, \(s\) is contained in \(A\). Since \(\bar{A} = A \cup L\), \(s \in \bar{A}\) by definition of union.
        \item No. For an open set to have a supremum, the supremum does not need to reside within the set. Thus, the supremum is not contained in the set. 
    \end{enumerate}
}

\begin{exercise}
    {3.2.6}Decide whether the following statements are true or false. Provide counter examples for those that are false, and supply proofs for those that are true. (Choose 2.)
    \begin{enumerate}
        \item An \hyperref[def:3.2.2]{open set} that contains every rational number must necessarily be all of \(\R\).
        \item The \namrefthm{Nested Interval Property} remains true if the term ``closed interval'' is replaced by ``\hyperref[def:3.2.5]{closed set}.''
        \item Every nonempty open set contains a rational number.
        \item Every bounded infinite closed set contains a rational number.
        \item The \hyperref[def:3.1.1]{Cantor set} is closed. 
    \end{enumerate}
\end{exercise}

\sol{
    \begin{enumerate}
        \item True. 
        \item False. Consider the closed set counter example of \(K_n = [n,\infty)\). This set is closed, but is not a closed interval.   
    \end{enumerate}
}

\begin{exercise}
    {3.2.8} Assume \(A\) is an open set and \(B\) is a closed set. Determine if the following sets are definitely \hyperref[def:3.2.2]{open}, definitely \hyperref[def:3.2.5]{closed}, both or neither.
    \begin{enumerate}
        \item \(\overline{A \cup B}\)
        \item \(A \setminus B = \{x \in A \mid x \notin B\}\)
        \item \((A^c \cup B)^c\)
        \item \((A \cap B) \cup (A^c \cap B)\)
        \item \(\overline{A}^c \cap \overline{A^c}\)
    \end{enumerate}
\end{exercise}

\sol{
    \begin{enumerate}
        \item e
        \item e
        \item e
        \item e
        \item ee
    \end{enumerate}
}

\begin{exercise}
    {3.2.11} \begin{enumerate}
        \item Prove that \(\overline{A \cup B} = \overline{A} \cup \overline{B}\).
        \item Does this result about closures extend to infinite unions of sets?
    \end{enumerate}
\end{exercise}

\sol{
    \begin{enumerate}
        \item \hfill \innerpf{Let me break it down for you req. : lol!
        }
        \item NO (Refuses to elaborate).
    \end{enumerate}
}

\renewcommand{\theenumi}{\arabic{enumi}}
\renewcommand{\labelenumi}{\theenumi.}
\section{Compact Sets}

\begin{definition}
    A set \(K \subseteq \R\) is a \textit{compact set} if every sequence from \(K\) has a convergent subsequence where the limit is also \(K\).
\end{definition}

\begin{theorem}
    A set \(K\) is compact if, and only if, it is closed and bounded.
\end{theorem}

\iffpf{Assume a set \(A \subseteq \R\) is closed and bounded. Thus, there exists a convergent subsequence by \namrefthm{Bolzano-Weierstrass Theorem}. Because \(A\) is closed, the limit is in the set by \defref{3.2.5}.}{Assume a set \(A\) is compact. If it is not bounded, then there exists an \((a_n)\) that heads toward infinity. This contradicts \defref{3.3.1}, so it must be bounded. Then, by the same definition, the limit points belong in the set, so it is closed.\qedhere}{}{xblue}

\begin{definition}
    Let \(A \in \R\). An \textit{open cover} for \(A\) is a collection of open sets \(\{O_\lambda \mid \lambda \in A\}\) whose union contains the set \(A\); that is \(A \subseteq \bigcup_{\lambda \in A}O_\lambda\). Given an open cover for \(A\), a \textit{finite subcover} is a finite sub-collection of open sets from the original open cover whose union still manages to completely contain \(A\).
\end{definition}

\begin{ntheorem}
    {Heine-Borel Theorem}A set \(K \subseteq \R\) is compact if, and only if, every open cover of \(K\) has a finite subcover. 
\end{ntheorem}

\iffpf{Assume \(K\) is compact. }{Assume every open cover of \(K\) has a finite subcover. We need to prove \(K\) is closed and bounded. To show \(\mathcal{U}\) is bounded, consider the open cover \(\mathcal{U} = \{(-n, n) \mid n \in \N\} \). \(\mathcal{U}\) covers all of \(\R\), so it certainly covers \(K\). Thus, there must exist a finite subcover. Consider the longest interval in the subcover. Then, \(K\) is a subset of this interval, so \(K\) is bounded. To show \(K\) is closed, we need to show every limit point belongs to \(K\). Assume \(x\) is a limit point of \(K\). From \defref{3.2.4} for every \(a \in K\), let \(\epsilon_a = \frac{1}{2}\disabs{a - x}\). Consider the open cover \(\mathcal{U} = \{V_{\epsilon_a}(a) \mid a \in K\} \). This covers every point on \(K\).}{}{xblue}

\begin{note}
    This allows us to take an infinite amount of \(\epsilon\)-neighborhoods and turn them into finite subcovers.
\end{note}
\begin{example}
    {Compactness}Let \(A = (0,1)\). Construct a set \(\mathcal{U}\) that is an open cover of \((0,1)\), but does not have a finite subcover.
\end{example}

\lesol{
    Consider \(\mathcal{U} = \{(0,t) \mid 0 < t < 1\}\). Thus, \(\mathcal{U}\) is an open cover, but does not contain a finite amount of subcovers because there will always be a point not covered. 
}

\renewcommand{\theenumi}{\alph{enumi}}
\renewcommand{\labelenumi}{(\theenumi)}
\subsection{Exercises}

\begin{exercise}
    {3.3.4}Assume \(K\) is \hyperref[def:3.3.1]{compact} and \(F\) is \hyperref[def:3.2.5]{closed}. Decide if the following sets are definitely compact, definitely closed, both, or neither. 
    \begin{enumerate}
        \item \(K \cap F\)
        \item \(\overline{F^c \cup K^c}\)
        \item \(K \setminus F = \{x \in K \mid x \notin F\}\)
        \item \(\overline{K \cap F^c}\)
    \end{enumerate}
\end{exercise}