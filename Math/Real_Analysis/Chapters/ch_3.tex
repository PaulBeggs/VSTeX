\section{The Cantor Set}

We will build this set through an iterative process. Start with a number line \(C_0\) that stretches from 0 to 1. Remove the middle third of the interval, leaving two intervals of length \(\frac{1}{3}\). We will call the set of points removed from \(C_0\) \(C_1\). Next, remove the middle third of each of the two intervals, leaving four intervals of length \(\frac{1}{9}\). We will call the set of points removed from \(C_1\) \(C_2\). Continue this process indefinitely. 

% Put a tikz picture here that shows the cantor set at each step

\begin{definition}
    The \textit{Cantor set}, \(C\), is defined as \(C = \bigcap_{n=0}^\infty C_n\). This set is 
    \begin{enumerate}
        \item Non-empty. All end points stay within the interval.
        \item \(C\) is uncountable.
    \end{enumerate}
\end{definition}

The second part of that definition is a bit tricky to prove, but a visual will do for now. We can put all elements of the Cantor set in a one-to-one correspondence with the set of all 0s and 1s. This shows that not only is it uncountable, but it also has the same cardinality as \([0,1]\). % Put a tikz picture here from the notes.

The total length of removed elements, \(\frac{1}{3} + \frac{2}{9} + \frac{4}{27} + \frac{8}{81} + \cdots = \frac{1}{3}(1 + \frac{2}{3} + \frac{4}{9})\). Notice the resemblance to the geometric series? We can write this as \[\frac{1}{3} \sum_{n=0}^\infty \left(\frac{2}{3}\right)^n = \frac{1}{3} \left(\frac{1}{1 - \frac{2}{3}}\right) = 1.\] 

In summary, \begin{enumerate}
    \item Start with intercal \([0,1]\).
    \item Remove countably disjoint intervals.
    \item Uncountably many points between these intervals, all isolated from each other.
    \item The space taken up by the leftover points has ``length'' 0.
\end{enumerate}

When we review the properties of a fractal, we see that the Cantor set is a fractal. It is self-similar, and the dimension of the Cantor set is \(\log_3 2\).

\begin{note}
    For a cool look at the cantor set as a fractal, check out \url{https://en.wikipedia.org/wiki/File:Cantor_Set_Expansion.gif}.
\end{note}

\section{Open and closed Sets}

\begin{definition}
    For a point \(x \in \R\), and \(\epsilon > 0\), we define the \textit{epsilon-neighborhood} of \(x\) to be \(V_\epsilon(x) = \{y \in \R \mid |x - y| < \epsilon\}\).
\end{definition}

\subsection{Open Sets}

\begin{definition}
    A set \(A \subseteq \R\), is called an \textit{open set} if for every \(x \in A\), there exists an \(\epsilon > 0\) such that \(V_\epsilon(x) \subset A\).
\end{definition}

\subsubsection{Some Examples of Open Sets}

\begin{multicols}{2}
    \begin{itemize}
        \item All open intervals are also open sets.
        \item \(\R\) is open.
        \item \(\emptyset\) is open.
        \item \(\{1\}\) is not open.
        \item \([0,2]\) is not open.
        \item \(\Q\) is not open.
        \item \([4,6)\) is not open.
        \item \((0,1) \cup (1,3) \cup (5,10)\) is open.
        \item \((0,3] \cap [2,4)\) is open.
        \item Canter set is not open.
    \end{itemize}
\end{multicols}

\begin{theorem}
    \begin{enumerate}[label=(\roman*)]
        \item The union of an arbitrary collection of open sets is open.
        \item The intersection of a finite collection of open sets is open. 
    \end{enumerate}
\end{theorem}

\tpf{
    \textit{(ii)}. Let \(A_1, A_2, \dots, A_n\) be open sets. Let \(x \in \bigcap_{n = 1}^n A_i\). Then for each \(i \leq n\), \(x \in A_i\). Since each \(A_i\) is open, there exists \(\epsilon > 0\) such that \(V_\epsilon(x) \subset A\). Let \(\epsilon = \min\{\epsilon, \epsilon_i, \epsilon_n\} > 0\). Then, for all \(i \leq n, \ V_\epsilon(x) \subseteq V_\epsilon(x) \subseteq A_i\).
}

Note that we cannot use this for cases with infinity. For example, consider \(A_n = (-\frac{1}{n}, \frac{1}{n})\). This is open, but \(\bigcap_{n = 1}^\infty\ A_n = \{0\}\), which is not open. 

\subsection{Closed Sets}

\begin{definition}
    Let \(A \subseteq \R\). We say \(x\) is a \textit{limit point} of \(A\) if for all \(\epsilon > 0\), there exists \(a \in A\) such that \(a \in V_\epsilon(x)\) that is not \(x\). Additionally, a point \(x \in \R\) is a \textit{limit point} if, and only if, there exists a sequence \((a_n)\) of points from \(A\) that are not \(x\). But \(\limn(a_n) = x\). 
\end{definition}

\begin{definition}
    Let \(B \subseteq \R\) is called a \textit{closed set} if \(B\) contains all its limit points. 
\end{definition}

\begin{note}
    \textbf{Important note:} Limit points could be outside a set. Consider \((0,1)\). Even though 0 and 1 do not belong to the set, they are considered limit points that are outside the set.
\end{note}

\subsubsection{Some Examples of Closed Sets}
\begin{multicols}{2}
    \begin{itemize}
        \item \([0,1]\) is closed.
        \item \((0,1)\) is not closed.
        \item \(\R\) is closed.
        \item \(\emptyset\) is closed.
        \item \(\Q\) is not closed.
        \item \([3, infty)\) is closed.
        \item \(\frac{1}{n} \mid n \in \N\) not closed. (Because of 0)
        \item \([1,4] \cup \{8\}\) is closed. 
        \item \(\{1\}\) is closed. 
        \item \([1,2)\) is not closed. Note that this set is neither open or closed. 
    \end{itemize}
\end{multicols}

\begin{theorem}
    A set \(B \subseteq \R\) is closed if, and only if, its complement is open. Similarly, a set \(A \in \R\) is open if, and only if, its complement is closed.
\end{theorem}

\iffpf{Assume \(B \subseteq \R\) is a closed set. We will show that \(B^c\) is open. Let \(x \in B^c\). So, \(x \notin B\). This means \(x\) is not a limit point. (From the negated definition of limit point:) There must exist \(\epsilon >0\) such that no elements of \(B\) belong to \(V_\epsilon(x)\). Then, \(V_\epsilon(x) \subseteq B^c\). Therefore, \(B^c\) is open.}{Assume \(B^c\) is open. We will show that \(B\) is closed. Let \(x\) be a limit point of \(B\). For all \(\epsilon > 0\), there exists a \(b \in B\) such that \(b \subseteq V_\epsilon(x)\). So, \(V_\epsilon(x)\) is not a subset of \(B^c\). This is true for every \(\epsilon\). Since \(B^c\) is open, it must be that \(x \notin B^c\). Thus, \(x \in B\). So, \(B\) contains all its limit points. Therefore, \(B\) is closed.}{}{xblue}

\begin{definition}
    Let \(A \subseteq \R\) and let \(L\) be the set of limit points of \(A\). The \textit{closure} of \(A\) is defined as \(\bar{A} = \bar{A} = A \cup L\).
\end{definition}

\begin{theorem}
    \begin{enumerate}[label=(\roman*)]
        \item The intersection of any collection of closed sets is closed. 
        \item The union of finitely many closed sets is closed.
    \end{enumerate}
\end{theorem}

\begin{theorem}
    The closure of a set is a closed set.
\end{theorem}

\textbf{Note:} This theorem may seem trivial, but it answers the question of ``Are there limit points in \(L\) that are not accounted for?''

\begin{theorem}
    The closure set \(A\) is the \textit{smallest} closed set containing \(A\). (Where ``smallest'' refers to a subset of any other closed set containing \(A\).)
\end{theorem}

\begin{example}
    {Closed Sets}Generate countably many closed sets where the union is not closed.
\end{example}

\lesol{
    \(B_n = [\frac{1}{n}, 1 - \frac{1}{n}]\). Therefore, \(\bigcup_{n = 3}^\infty B_n = (0,1)\). For example, that would look like: \(\{\frac{1}{2}\} \cup \{\frac{1}{3}\} \cup \dots\)
}