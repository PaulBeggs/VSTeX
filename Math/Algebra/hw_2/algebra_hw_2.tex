%-%-%-%-%-%-%-%-%-%-%-%-%-%-%-%-%-%-%-%-%-%-%-%-%-%-%-%-%-%-%-%-%-%-%-%-%-%-%
%%% MAIN DOCUMENT %%%

%-%-%-%-%-%-%-%-%-%-%-%-%-%-%-%-%-%-%-%-%-%-%-%-%-%-%-%-%-%-%-%-%-%-%-%-%-%-%

\documentclass[12pt, oneside]{book}
\usepackage{xcolor}
\usepackage{minted}

% \usepackage[T1]{fontenc}

\usepackage{draculatheme}
\newcommand{\documentTheme}{draculafg}
\newcommand{\documentLogo}{../images/small logo_white.png}
\newcommand{\documentBigLogo}{../images/logo_white_text.png}
\usemintedstyle{dracula}

% \newcommand{\documentBigLogo}{images/Hendrix Logo.png}
% \newcommand{\documentLogo}{images/small logo.png}
% \newcommand{\documentTheme}{black}

%%% AESTHETICS %%%
%-%-%-%-%-%-%-%-%-%-%-%-%-%-%-%-%-%-%-%-%-%-%-%-%-%-%-%-%-%-%-%-%-%-%-%-%-%-%


%%% Dimensions and Spacing %%%
\usepackage[margin=1in]{geometry}
\setlength{\parindent}{0pt}
\usepackage{setspace}
\usepackage{mathtools}
\usepackage{esint}
\usepackage{adjustbox}
\linespread{1}
\usepackage{listings}
\usepackage{tikz}
\usetikzlibrary{shapes,backgrounds,calc,patterns,positioning}
\usepgflibrary{shadings}
\usepackage{pgfkeys}
\usepackage{pdftexcmds}
\usepackage{shellesc}
\usepackage{algorithm, algorithmic, xspace}
%%% Define new colors %%%
\definecolor{orangehdx}{rgb}{0.96, 0.51, 0.16}

% Normal colors
\definecolor{xred}{HTML}{BD4242}
\definecolor{xblue}{HTML}{4268BD}
\definecolor{xgreen}{HTML}{52B256}
\definecolor{xpurple}{HTML}{7F52B2}
\definecolor{xorange}{HTML}{FD9337}
\definecolor{xdotted}{HTML}{999999}
\definecolor{xgray}{HTML}{777777}
\definecolor{xcyan}{HTML}{80F5DC}
\definecolor{xpink}{HTML}{F690EA}
\definecolor{xgrayblue}{HTML}{49B095}
\definecolor{xgraycyan}{HTML}{5AA1B9}

% Dark colors
\colorlet{xdarkred}{red!85!black}
\colorlet{xdarkblue}{xblue!85!black}
\colorlet{xdarkgreen}{xgreen!85!black}
\colorlet{xdarkpurple}{xpurple!85!black}
\colorlet{xdarkorange}{xorange!85!black}
\definecolor{xdarkcyan}{HTML}{008B8B}
\colorlet{xdarkgray}{xgray!85!black}

% Very dark colors
\colorlet{xverydarkblue}{xblue!50!black}

% Document-specific colors
\colorlet{normaltextcolor}{black}
\colorlet{figtextcolor}{xblue}

% Enumerated colors
\colorlet{xcol0}{black}
\colorlet{xcol1}{xred}
\colorlet{xcol2}{xblue}
\colorlet{xcol3}{xgreen}
\colorlet{xcol4}{xpurple}
\colorlet{xcol5}{xorange}
\colorlet{xcol6}{xcyan}
\colorlet{xcol7}{xpink!75!black}

% Blue-Purple (should just used colorbrewer...)
\definecolor{xrainbow0}{HTML}{e41a1c}
\definecolor{xrainbow1}{HTML}{a24057}
\definecolor{xrainbow2}{HTML}{606692}
\definecolor{xrainbow3}{HTML}{3a85a8}
\definecolor{xrainbow4}{HTML}{42977e}
\definecolor{xrainbow5}{HTML}{4aaa54}
\definecolor{xrainbow6}{HTML}{629363}
\definecolor{xrainbow7}{HTML}{7e6e85}
\definecolor{xrainbow8}{HTML}{9c509b}
\definecolor{xrainbow9}{HTML}{c4625d}
\definecolor{xrainbow10}{HTML}{eb751f}
\definecolor{xrainbow11}{HTML}{ff9709}

%------- %
% XHFILL %
%------- %



%%% Chapter Headings %%%

\newcommand{\gradientrule}{
    \begin{tikzpicture}
        \shade[left color=orangehdx, right color=black, middle color=gray] (0,0) rectangle (\linewidth,0.4pt);
    \end{tikzpicture}
}

\usepackage[Glenn]{fncychap}
\ChTitleVar{\bfseries\scshape\color{\documentTheme}} % Needed for Dracula theme
\ChNumVar{\large\selectfont\color{\documentTheme}} % Needed for Dracula theme
\ChNameVar{\large\color{\documentTheme}} % Needed for Dracula theme
\usepackage{xpatch}

% \xpatchcmd{\DOCH}
%   {\mghrulefill}{\gradientrule\mghrulefill}
%   {}{\PatchFailed}
% \xpatchcmd{\DOTI}
%   {\mghrulefill}{\gradientrule\mghrulefill}
%   {}{\PatchFailed}
% \xpatchcmd{\DOTIS}
%   {\mghrulefill}{\gradientrule\mghrulefill}
%   {}{\PatchFailed}

\xpatchcmd\DOCH
{\mghrulefill}{\color{orangehdx}\mghrulefill}
{}{\PatchFailed}
\xpatchcmd\DOTI
{\mghrulefill}{\color{orangehdx}\mghrulefill}
{}{\PatchFailed}
\xpatchcmd\DOTIS
{\mghrulefill}{\color{orangehdx}\mghrulefill}
{}{\PatchFailed}


% \usepackage[Bjornstrup]{fncychap}

% \newcommand{\gradient}[1]{
% \begin{tikzpicture}
%     \node (rect) at (0,0) [fill=blue,,path fading=East,minimum width=\linewidth,minimum height=2.5cm] {};
%     \node(title)[above left = 10pt and 10pt of rect.south east, anchor=south east, font=\CTV] {\textcolor{horange}{#1}};
%     \ifnum \thechapter>0\node[left = 10pt of rect.north east,  anchor=center, font=\CNoV] {\textcolor{horange}{\thechapter}};\fi%
% \end{tikzpicture}%
% \vskip 40pt
% }


% \renewcommand{\DOCH}{}
% \renewcommand{\DOTI}[1]{\gradient{#1}}
% \renewcommand{\DOTIS}[1]{\gradient{#1}}

%% Change Chapter Heading Placement %%
\usepackage{etoolbox}
\makeatletter
\patchcmd{\@makechapterhead}{\vspace*{50\p@}}{\vspace*{-20\p@}}{}{}
\patchcmd{\@makeschapterhead}{\vspace*{50\p@}}{\vspace*{-20\p@}}{}{}
\patchcmd{\DOTI}{\vskip 80\p@}{\vskip 40\p@}{}{}
\patchcmd{\DOTIS}{\vskip 40\p@}{\vskip 0\p@}{}{}
\makeatother



% \newcommand*\chapterlabel{}\pmod
% \titleformat{\chapter}
%   {\gdef\chapterlabel{}
%    \normalfont\sffamily\Huge\bfseries\scshape}
%   {\gdef\chapterlabel{\thechapter\ }}{0pt}
%   {\begin{tikzpicture}[remember picture,overlay]
%     \node[yshift=-3cm] at (current page.north west)
%       {\begin{tikzpicture}[remember picture, overlay]
%         \draw[fill=LightSkyBlue] (0,0) rectangle
%           (\paperwidth,3cm);
%         \node[anchor=east,xshift=.9\paperwidth,rectangle,
%               sharp corners=downhill=20pt,inner sep=11pt,
%               fill=MidnightBlue]
%               {\color{white}\chapterlabel#1};
%        \end{tikzpicture}
%       };
%    \end{tikzpicture}
%   }
% \titlespacing*{\chapter}{0pt}{50pt}{-60pt}

% \usepackage[Conny]{fncychap}
% \usepackage[Rejne]{fncychap}
% \ChNameVar{\bfseries}  % Makes the chapter "Chapter #" bold
% \ChNumVar{\bfseries}   % Makes the chapter number bold
% \ChTitleVar{\bfseries} % Makes the chapter title bold
% % Define a custom color, for example:
% \definecolor{mycolor}{RGB}{0,128,255}

% % Redefine the chapter style in Rejne to change the line colors
% \makeatletter
% \ChRuleWidth{2pt}   % Change the thickness of the lines
% \renewcommand{\DOCH}{%
%   \vspace*{-50\p@}% Moves the chapter title up/down if needed
%   {\color{mycolor} \hrule \@chapapp{} \space \thechapter \hrule}% Customizes the chapter header with color
% }
% \renewcommand{\DOTI}[1]{%
%   \vskip 20\p@ % Adjusts the space above the title
%   \bfseries #1\par % Embolden the title
%   \vskip 20\p@ % Adjusts the space below the title
% }
% \makeatother

% %% Change Chapter Heading Placement %%
% \usepackage{etoolbox}
% \makeatletter
% \patchcmd{\@makechapterhead}{\vspace*{50\p@}}{\vspace*{-20\p@}}{}{}
% \patchcmd{\@makeschapterhead}{\vspace*{50\p@}}{\vspace*{-20\p@}}{}{}
% \patchcmd{\DOTI}{\vskip 80\p@}{\vskip 40\p@}{}{}
% \patchcmd{\DOTIS}{\vskip 40\p@}{\vskip 0\p@}{}{}
% \makeatother

\renewcommand{\thesection}{\thechapter.\arabic{section}} %% Chapter.Section Numbering


%%% FIGURES %%%
\usepackage{graphicx}  
% \numberwithin{figure}{section}
\usepackage{float}
\usepackage{caption}

%%% Hyperlinks %%%
\usepackage{hyperref}
\definecolor{horange}{HTML}{f58026}
\hypersetup{
	colorlinks=true,
	linkcolor=horange,
	filecolor=horange,      
	urlcolor=horange,
}


%% Headers and Footers %%
\usepackage{fancyhdr} % This should be set AFTER setting up the page geometry
\pagestyle{fancy} % options: empty , plain , fancy
\fancyhead[R]{\textcolor{draculafg}{\assignmentname}}
\fancyhead[L]{\textcolor{draculafg}{Hendrix College}}
\fancyhead[C]{\includegraphics[height=.50cm]{\documentLogo}} % Use for Dracula
\usepackage{xpatch}
\xpretocmd\headrule{\color{orangehdx}}{}{\PatchFailed}
\setlength{\footskip}{0.5in}
\setlength{\headheight}{18.5764pt}


%%%%% Colored Boxes %%%%%
\usepackage{tcolorbox}
\tcbuselibrary{skins}
\tcbuselibrary{theorems}
% \tcbuselibrary{minted}
\newcounter{BoxCounter}
\usepackage{dingbat}

%-%-%-%-%-%-%-%-%-%-%-%-%-%-%-%-%-%-%-%-%-%-%-%-%-%-%-%-%-%-%-%-%-%-%-%-%-%-%

%% MATH PACKAGES, ENVIRONMENTS, COMMANDS %%
%-%-%-%-%-%-%-%-%-%-%-%-%-%-%-%-%-%-%-%-%-%-%-%-%-%-%-%-%-%-%-%-%-%-%-%-%-%-%
%You'll need your own packages, theorem types, and commands.

\usepackage{fix-cm}
\usepackage{amsmath,amsthm} 
\usepackage{array, makecell}
\usepackage{colortbl}
\newcommand{\thickvrule}{\vrule width 1pt}

\newcommand{\thickhrule}{\Xhline{1pt}}

\usepackage{mathtools}
\usepackage{amssymb}
\usepackage[framemethod=tikz]{mdframed}
\usepackage{mathrsfs}
\usepackage{changepage}
\usepackage{multicol}
\usepackage{slashed}
\usepackage{enumerate}
\usepackage{braket}
\usepackage{booktabs}
\usepackage{enumitem}
\usepackage{kantlipsum}  %This package lets us generate random text for example purposes.
\usepackage{pgfplots}


%%% Custom Commands %%%
% Natural Numbers 
\newcommand{\N}{\mathbb{N}}

% Whole Numbers
\newcommand{\W}{\mathbb{W}}

% Integers
\newcommand{\Z}{\mathbb{Z}}

% Rational Numbers
\newcommand{\Q}{\mathbb{Q}}

% Real Numbers
\newcommand{\R}{\mathbb{R}}

% Complex Numbers
\newcommand{\C}{\mathbb{C}}

\newcommand{\I}{\mathbb{I}}

\newcommand{\pfs}{\noindent\makebox[\linewidth]{\rule{\textwidth}{0.4pt}}\vspace{0.5cm}}

\newcommand{\mysqrt}[1]{%
	\mathpalette\foo{#1}%
}
\newcommand{\dmysqrt}[1]{%
	\mathpalette\foodisplay{#1}%
}

% !TeX spellcheck = off
\newcommand{\foo}[2]{%
	% #1: math style, #2: content
	\sbox0{$#1\sqrt{#2}$}% Measure the size of the standard sqrt in the current style
	\begin{tikzpicture}[baseline=(sqrt.base)]
		\node[inner sep=0, outer sep=0] (sqrt) {$#1\sqrt{#2}$}; % Use the current math style
		\draw([yshift=-0.045em]sqrt.north east) -- ++(0,-0.5ex); % Draw the tick
	\end{tikzpicture}%
}
% !TeX spellcheck = off
\newcommand{\foodisplay}[2]{%
	% #1: math style, #2: content
	\sbox0{$#1\sqrt{#2}$}% Measure the size of the standard sqrt in the current style
	\begin{tikzpicture}[baseline=(sqrt.base)]
		\node[inner sep=0, outer sep=0] (sqrt) {$\displaystyle\sqrt{#2}$}; % Force displaystyle
		\draw[line width=0.4pt] ([yshift=-0.044em]sqrt.north east) -- ++(0,-0.5ex); % Draw the tick
	\end{tikzpicture}%
}

\renewcommand{\theenumi}{\arabic{enumi}}
\renewcommand{\labelenumi}{\textbf{\theenumi}.}

% \newcommand{}{\marginnote{\includegraphics[width=2em]{caution.png}}}

\newmdenv[
	topline=false,
	bottomline=true,
	rightline=false,
	leftline=true,
	linewidth=1.5pt,
	linecolor=black, % default color, will be overridden in custom commands
	backgroundcolor=draculabg, % Needed for Dracula theme
	fontcolor=\documentTheme, % Needed for Dracula theme
	innertopmargin=0pt,
	innerbottommargin=5pt,
	innerrightmargin=10pt,
	innerleftmargin=10pt,
	leftmargin=0pt,
	rightmargin=0pt,
	skipabove=\topsep,
	skipbelow=\topsep,
]{customframedproof}

\newenvironment{proofpart}[2][black]{
    \begin{mdframed}[
        topline=false,
        bottomline=false,
        rightline=false,
        leftline=true,
        linewidth=1pt,
        linecolor=#1!40, % Custom color
        % innertopmargin=10pt,
        % innerbottommargin=10pt,
        innerleftmargin=10pt,
        innerrightmargin=10pt,
        leftmargin=0pt,
        rightmargin=0pt,
        % skipabove=\topsep,
        % skipbelow=\topsep%
    ]
    \noindent
    \begin{minipage}[t]{0.08\textwidth}%
        \textbf{#2}%
    \end{minipage}%
    \begin{minipage}[t]{0.90\textwidth}%
        \begin{adjustwidth}{0pt}{0pt}%
}{
    \end{adjustwidth}
    \end{minipage}
    \end{mdframed}
}

% Define a new environment 'proofscratch' for Proof and Scratch Paper
\newenvironment{proofscratch}[2]{%
    \begin{mdframed}[
        topline=false,
        bottomline=false,
        rightline=false,
        leftline=false,
        backgroundcolor=draculabg, % Background color for Dracula theme
        fontcolor=\documentTheme,       % Font color for Dracula theme
        innerleftmargin=-6pt,
        innertopmargin=0pt,
        leftmargin=0pt,
        rightmargin=0pt,
        skipabove=\topsep,
        skipbelow=\topsep,
    ]
    % % Begin TikZ picture for the vertical dotted line
    % \begin{tikzpicture}[overlay, remember picture]
    %     % Draw a vertical dotted line at approximately the center of the mdframed width
    %     \draw[dotted, thick] ([xshift=0.005\linewidth]current bounding box.north west) -- ([xshift=0.005\linewidth]current bounding box.south west);
    % \end{tikzpicture}%
    % % Create the left minipage for Proof
    \begin{minipage}[t]{0.47\textwidth}%
        \noindent\textit{Proof.} #1 \hfill \(\qed\)
    \end{minipage}%
    \hfill
    % Create the right minipage for Scratch Paper
    \begin{minipage}[t]{0.47\textwidth}%
        \textit{Scratch Paper.} #2
    \end{minipage}%
}{
    \end{mdframed}
}

\newenvironment{tipbox}
	{%
		\par\noindent%
		% Top dashed line (using \textwidth)
		\tikz[baseline]{\draw[dash pattern=on 3pt off 2pt, line width=0.5pt, color=draculapurple] (0,0) -- (\textwidth,0);}%
		\par\vspace{0.65em}%
		\noindent\textbf{\textcolor{draculapurple}{TIP:}}\quad%
	}
	{%
		\par%
		% Bottom dashed line
		\noindent%
		\tikz[baseline]{\draw[dash pattern=on 3pt off 2pt, line width=0.5pt, color=draculapurple] (0,0) -- (\textwidth,0);}%
		\par\vspace{0.35em}
	}

\newenvironment{notebox}
	{%
		\par\noindent%
		% Top dashed line (using \textwidth)
		\tikz[baseline]{\draw[dash pattern=on 3pt off 2pt, line width=0.5pt, color=draculagreen] (0,0) -- (\textwidth,0);}%
		\par\vspace{0.65em}%
		\noindent\textbf{\textcolor{draculagreen}{NOTE:}}\quad%
	}
	{%
		\par%
		% Bottom dashed line
		\noindent%
		\tikz[baseline]{\draw[dash pattern=on 3pt off 2pt, line width=0.5pt, color=draculagreen] (0,0) -- (\textwidth,0);}%
		\par\vspace{0.35em}
	}

\newenvironment{warningbox}
	{%
		\par\noindent%
		% Top dashed line (using \textwidth)
		\tikz[baseline]{\draw[dash pattern=on 3pt off 2pt, line width=0.5pt, color=draculared] (0,0) -- (\textwidth,0);}%
		\par\vspace{0.65em}%
		\noindent\textbf{\textcolor{draculared}{WARNING:}}\quad%
	}
	{%
		\par%
		% Bottom dashed line
		\noindent%
		\tikz[baseline]{\draw[dash pattern=on 3pt off 2pt, line width=0.5pt, color=draculared] (0,0) -- (\textwidth,0);}%
		\par\vspace{0.35em}
	}

\newenvironment{solution}{\textit{Solution.}}

\newcommand{\sol}[1]{
    \begin{customframedproof}[linecolor=orangehdx!75,]
        \begin{solution}
        #1
        \end{solution}
    \end{customframedproof}
}

\newcommand{\pf}[1]{
    \begin{customframedproof}[linecolor=orangehdx!75]
        \begin{proof}
        #1
        \end{proof}
    \end{customframedproof}
}

\def \proofDistance {10pt}

\newcommand{\disabs}[1]{\ensuremath{\left|#1\right|}}
\newcommand{\limn}{\lim_{n \rightarrow \infty}}
\newcommand{\limx}[2]{\lim_{x \rightarrow #1}#2}

\newcommand{\limc}[1]{\lim_{x \rightarrow c}#1}

\newcommand{\ninn}{n \in \N}

\newcommand{\mb}[1]{\mathbf{#1}}

\newcommand{\limsupn}[1]{\limsup_{n\rightarrow \infty}#1}
\newcommand{\liminfn}[1]{\liminf_{n\rightarrow \infty}#1}


\newcommand{\proj}{\text{proj}}

\newcommand{\p}{\partial}

\newcommand{\dydx}{\frac{dy}{dx}}
\newcommand{\dxdy}{\frac{dx}{dy}}
\newcommand{\dydt}{\frac{dy}{dt}}
\newcommand{\dxdt}{\frac{dx}{dt}}
\newcommand{\dzdt}{\frac{dz}{dt}}

\newcommand{\barNotationT}[1]{\bigg|_{t = #1}}

\newcommand{\cyanit}[1]{\textit{\textcolor{cyan}{#1}}}

\newcommand{\brackett}[1]{\left\langle #1 \right\rangle}

\newcommand{\norm}[1]{\left\lVert \mathbf{#1}\right\rVert}

\newcommand{\imb}{\mb{i}}
\newcommand{\jmb}{\mb{j}}
\newcommand{\kmb}{\mb{k}}
\newcommand{\rmb}{\mb{r}}
\newcommand{\umb}{\mb{u}}

\newcommand{\vecfuc}[2]{\mb{#1}(#2)}
\newcommand{\dvecfuc}[2]{\mb{#1}'(#2)}
\newcommand{\normdvecfuc}[2]{\|\mb{#1}'(#2)\|}

\newcommand{\squigglyline}{%
    \noindent
    \tikz[baseline=-0.5ex]{
        \draw[decorate, decoration={snake, amplitude=0.5mm, segment length=3mm}] 
        (0,0) -- (\dimexpr\linewidth\relax,0);
    }%
}


% end of preamble
%-%-%-%-%-%-%-%-%-%-%-%-%-%-%-%-%-%-%-%-%-%-%-%-%-%-%-%-%-%-%-%-%-%-%-%-%-%-%

\begin{document}

\numberwithin{BoxCounter}{section}


%-%-%-%-%-%-%-%-%-%-%-%-%-%-%-%-%-%-%-%-%-%-%-%-%-%-%-%-%-%-%-%-%-%-%-%-%-%-%
%%% COVER PAGE %%%
%-%-%-%-%-%-%-%-%-%-%-%-%-%-%-%-%-%-%-%-%-%-%-%-%-%-%-%-%-%-%-%-%-%-%-%-%-%-%

\newcommand{\assignmentname}{Homework 2: Sections 3 \& 4}

% Do not use all caps.
\newcommand{\cussubtitle}{Algebra}
% Date format should be like: "January 1, 2001"
\newcommand{\finaldate}{Septemeber 11, 2025}
% Be sure to include degree recognition (e.g., B.S., M.S., Ph.D)
\newcommand{\professor}{Dr. Christopher Camfield, Ph.D.}

%-%-%-%-%-%-%-%-%-%-%-%-%-%-%-%-%-%-%-%-%-%-%-%-%-%-%-%-%



\begin{titlepage}
    \begin{center}

        \vspace*{-2cm}
        \includegraphics[width=0.8\textwidth]{\documentBigLogo}\\
        \vfill

        % Horizontal line above the title in 'horange' color
        \textcolor{horange}{\rule{\textwidth}{1.0pt}}

        \vspace{2em}

        {\huge \textbf{\assignmentname}}

        \vspace{1em} % Space between the title and the bottom line

        \textcolor{horange}{\rule{\textwidth}{1.0pt}}

        \vspace*{1\baselineskip}

        {\LARGE \textbf{\cussubtitle}}

        \begin{large}
            \vspace*{5\baselineskip}

            \vspace*{1\baselineskip}

            \emph{Author} \\[1ex]
            %Submitted by \\[\baselineskip]
            {\Large Paul Beggs \\ \par} % Editor list
            {\href{mailto:BeggsPA@Hendrix.edu}{{BeggsPA@Hendrix.edu}}}\\ % Editor affiliation

            \vspace*{1\baselineskip}

            \textit{Instructor} \\[1ex] % Tagline(s) or further description
            \professor

            \vspace*{1\baselineskip}

            \textit{Due}\\[1ex]
            {\scshape  \finaldate} \\[0.3\baselineskip] % Year published

            \thispagestyle{empty}

        \end{large}
    \end{center}
\end{titlepage}
% -%-%-%-%-%-%-%-%-%-%-%-%-%-%-%-%-%-%-%-%-%-%-%-%-%-%-%-%-%-%-%-%-%-%-%-%-%-%
\section*{Section 3}
% In Exercises 6 and 7, determine whether the given map is an isomorphism of the first binary structure with the second. If it is not an isomorphism, why not?
\begin{enumerate}
    \setcounter{enumi}{5}
    \item \(\langle \Q, \cdot \rangle\) with \(\langle \Q, \cdot \rangle\) where \(\phi(x) = x^{2}\) for \(x \in \Q\).
          \sol{
              The binary operation \(\phi\) is not an isomorphism because it is not onto. A counterexample is that there is no rational number \(a\) such that \(\phi(a) = -1\).
          }

    \item \(\langle \R, \cdot \rangle\) with \(\langle \R, \cdot \rangle\) where \(\phi(x) = x^{3}\) for \(x \in \R\).
          \sol{
              The binary operation \(\phi\) is an isomorphism because it has the following properties:
              \begin{itemize}
                  \item \textbf{One-to-one:} If \(\phi(x) = \phi(y)\), then \(x^{3} = y^{3}\). Taking the cube root of both sides, we have \(x = y\). Thus, \(\phi\) is one-to-one.
                  \item \textbf{Onto:} For any \(z \in \R\), we can find an \(x \in \R\) such that \(\phi(x) = z\). Specifically, we can choose \(x = \sqrt[3]{z}\). Therefore, \(\phi\) is onto.
                  \item \textbf{Homomorphism:} For any \(x, y \in \R\), we have
                        \[
                            \phi(x \cdot y) = (x \cdot y)^{3} = x^{3} \cdot y^{3} = \phi(x) \cdot \phi(y).
                        \]
                        Therefore, \(\phi\) preserves the binary operation.
              \end{itemize}
          }

\end{enumerate}
\squigglyline \\

% In Exercise 11 and 12, let \(F\) be the set of all functions mapping \(\R\) into \(\R\) that have derivatives of all orders. Follow the instructions for Exercises 6 and 7.
\begin{enumerate}
    \setcounter{enumi}{10}
    \item \(\langle F, + \rangle\) with \(\langle F, + \rangle\) where \(\phi(f) = f'\), the derivative of \(f\).
          \sol{
              The binary operation \(\phi\) is not an isomorphism because it is not one-to-one. A counterexample is that \(\phi(f) = \phi(g)\) for \(f(x) = x^{2}\) and \(g(x) = x^{2} + 1\), but \(f \ne g\).
          }

    \item \(\langle F, + \rangle\) with \(\langle F, + \rangle\) where \(\phi(f) = f'(0)\).
          \sol{
              Similarly to the previous problem, the binary operation \(\phi\) is not an isomorphism because it is not one-to-one. A counterexample is that \(\phi(f) = \phi(g)\) for \(f(x) = \frac{x}{2}\) and \(g(x) = x^{2} + \frac{x}{2}\), but \(f \ne g\).
          }
\end{enumerate}

\newpage
\section*{Section 4}

% In Exercise 3, determine whether the binary operation \(*\) gives a group structure on the given set. If no group results, give the first axiom in the order \(\mathcal{G}_{1},\ \mathcal{G}_{2},\ \mathcal{G}_{3}\) from Definition 4.1 that does not hold.

\begin{enumerate}
    \setcounter{enumi}{2}
    \item Let \(*\) be defined on \(\R^{+}\) by letting \(a * b = \mysqrt{ab}\).
          \sol{
              The binary operation \(*\) does not give a group structure on \(\R^{+}\) because it fails \(\mathcal{G}_{1}\). A counterexample is that \((1 * 2) * 3 = \sqrt{2} * 3 = \sqrt{3\sqrt{2}}\), but \(1 * (2 * 3) = 1 * \mysqrt{6} = \sqrt{\sqrt{6}}\). Since \(\sqrt{3\sqrt{2}} \ne \sqrt{\sqrt{6}}\), the operation is not associative.
          }

\end{enumerate}
\squigglyline \\

% In Exercise 12, determine whether the given set of matrices under matrix multiplication is a group. Recall that a \textbf{diagonal matrix} is a square matrix whose only nonzero entries lie on the \textbf{main diagonal}, from the upper left to the lower right comer. Associated with each \(n \times n\) matrix \(A\) is a number called the determinant of \(A\), denoted by \(\det(A)\). If \(A\) and \(B\) are both \(n \times n\) matrices, then \(\det(AB) = \det(A) \det(B)\). Also, \(\det(I_{n}) = 1\) and \(A\) is invertible if and only if \(\det(A) \ne 0\).

\begin{enumerate}
    \setcounter{enumi}{11}
    \item All \(n \times n\) diagonal matrices under matrix multiplication.
          \sol{
              The set of all \(n \times n\) diagonal matrices under matrix multiplication does not form a group because it fails \(\mathcal{G}_{3}\). A counterexample is the diagonal matrix
              \[
                  A = \begin{bmatrix}
                      1 & 0 \\
                      0 & 0
                  \end{bmatrix},
              \]
              which is not invertible since \(\det(A) = 0\).
          }

\end{enumerate}
\squigglyline \\
\begin{enumerate}
    \setcounter{enumi}{18}
    \item Let \(S\) be the set of all real numbers except -1. Define \(*\) on \(S\) by \(a * b = a + b + ab\).
          \begin{enumerate}[label=\textbf{\alph*.}]
              \item Show that \(*\) gives a binary operation on \(S\).
                    \sol{
                        Let \(a, b \in S\). Assume, for the sake of contradiction, that \(a * b = -1\). Then,
                        \begin{align*}
                            a * b          & = -1 \\
                            a + b + ab     & = -1 \\
                            ab + a + b + 1 & = 0  \\
                            (a + 1)(b + 1) & = 0.
                        \end{align*}
                        Notice this equation is only true if \(a = -1\) and \(b = -1\), which contradicts our assumption that \(a, b \in S\). Thus, \(a * b \ne -1\) and so \(a * b \in S\). Therefore, \(*\) is a binary operation on \(S\).
                    }
              \item Show that \((S, *)\) is a group.
                    \sol{
                        We will verify that \((S, *)\) satisfies the group axioms:
                        \begin{itemize}
                            \item \(\mathcal{G}_{1}\) (Associativity): Let \(a, b, c \in S\). Then,
                                  \begin{align*}
                                      (a * b) * c & = (a + b + ab) * c                 \\
                                                  & = (a + b + ab) + c + (a + b + ab)c \\
                                                  & = a + b + ab + c + ac + bc + abc.
                                  \end{align*}
                                  Similarly,
                                  \begin{align*}
                                      a * (b * c) & = a * (b + c + bc)                 \\
                                                  & = a + (b + c + bc) + a(b + c + bc) \\
                                                  & = a + b + c + bc + ab + ac + abc.
                                  \end{align*}
                                  Since both expressions are equal, we have \((a * b) * c = a * (b * c)\), confirming associativity.

                            \item \(\mathcal{G}_{2}\) (Identity Element): We need to find an element \(e \in S\) such that for all \(a \in S\), \(a * e = e * a = a\). Let \(e = 0\). Then,
                                  \[
                                      a * 0 = a + 0 + a(0) = a,
                                  \]
                                  and
                                  \[
                                      0 * a = 0 + a + 0(a) = a.
                                  \]
                                  Since \(0 \in S\), it serves as the identity element.

                            \item \(\mathcal{G}_{3}\) (Inverse Element): For each \(a \in S\), we need to find an element \(a'\in S\) such that \(a * a'= a'* a = e\), where \(e\) is the identity element found above. We want to solve for \(a'\) in the equation:
                                  \begin{align*}
                                      a * a'      & = 0   \\
                                      \intertext{This gives us:}
                                      a + a'+ aa' & = 0.
                                      \intertext{Rearranging, we have:}
                                      aa'+ a'     & = -a,
                                      \intertext{or}
                                      a'(a + 1)   & = -a.
                                  \end{align*}
                                  Since \(a \in S\), we have \(a + 1 \ne 0\), and thus we can divide by \(a + 1\) to find:
                                  \[
                                      a'= \frac{-a}{a + 1}.
                                  \]
                                  With candidate \(a'\) identified, we can plug in to verify:
                                  \[
                                      a * a' = a + \frac{-a}{a + 1} + a\left(\frac{-a}{a + 1}\right) = 0,
                                  \]
                                  and
                                  \[
                                      a' * a = \frac{-a}{a + 1} + a + \left(\frac{-a}{a + 1}\right)a = 0.
                                  \]
                                  Since \(a' \ne -1\), we have \(a' \in S\). Therefore, every element in \(S\) has an inverse in \(S\).
                        \end{itemize}
                    }
              \item Find the solution of the equation \(2* x* 3 = 7\) in \(S\).
                    \sol{
                        Since we are working with an expression in \(S\), we have to be mindful of the order of operations. Since we don't have a nice way to ``decouple'' \(x\) from its coefficient, we'll have to work from the outside in. First, we must use the inverses (and Theorem 4.15) to get rid of the 2 and 3 from the left side:
                        \[
                            2 * (x * 3) = 7 \quad \Rightarrow \quad 2' * (2 * x * 3) * 3' = 2' * 7 * 3'.
                        \]
                        Using the associative property, we can rewrite the left side as:
                        \[
                            (2' * 2) * x * (3 * 3') = 2' * 7 * 3'.
                        \]
                        Since \(2' * 2 = 0\) and \(3 * 3' = 0\):
                        \[
                            0 * x * 0 = 2' * 7 * 3'.
                        \]
                        The identity property tells us that \(0 * x = x\) and \(x * 0 = x\), so:
                        \[
                            x = 2' * 7 * 3'.
                        \]
                        Using the formula for inverses found in part (b), we have:
                        \[
                            2' = \frac{-2}{2 + 1} = -\frac{2}{3}, \quad 3' = \frac{-3}{3 + 1} = -\frac{3}{4}.
                        \]
                        Substituting these values back into our equation for \(x\), we have:
                        \[
                            x = \left(-\frac{2}{3}\right) * 7 * \left(-\frac{3}{4}\right).
                        \]
                        Thus, we can expand the right side:
                        \begin{align*}
                            x &= \left[\left(-\frac{2}{3}\right) + 7 + \left(-\frac{2}{3}\right)(7)\right] * \left(-\frac{3}{4}\right) \\
                              &= \frac{5}{3} * \left(-\frac{3}{4}\right) \\
                              &= \frac{5}{3} + \left(-\frac{3}{4}\right) + \left(\frac{5}{3}\right)\left(-\frac{3}{4}\right) \\
                              &= -\frac{1}{3}.
                        \end{align*}

                    }
          \end{enumerate}


\end{enumerate}
\squigglyline \\
\begin{enumerate}
    \setcounter{enumi}{40}
    \item Let \(G\) be a group and let \(g\) be one fixed element of \(G\). Show that the map \(i_g\), such that \(i_g(x) = gxg^{-1}\) for \(x \in G\), is an isomorphism of \(G\) with itself.
          \sol{

          }

\end{enumerate}

%-%-%-%-%-%-%-%-%-%-%-%-%-%-%-%-%-%-%-%-%-%-%-%-%-%-%-%-%-%-%-%-%-%-%-%-%-%-%
\end{document}
%-%-%-%-%-%-%-%-%-%-%-%-%-%-%-%-%-%-%-%-%-%-%-%-%-%-%-%-%-%-%-%-%-%-%-%-%-%-%

%-%-%-%-%-%-%-%-%-%-%-%-%-%-%-%-%-%-%-%-%-%-%-%-%-%-%-%-%-%-%-%-%-%-%-%-%-%-%