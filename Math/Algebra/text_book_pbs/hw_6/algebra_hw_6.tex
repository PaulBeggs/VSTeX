%-%-%-%-%-%-%-%-%-%-%-%-%-%-%-%-%-%-%-%-%-%-%-%-%-%-%-%-%-%-%-%-%-%-%-%-%-%-%
%%% MAIN DOCUMENT %%%

%-%-%-%-%-%-%-%-%-%-%-%-%-%-%-%-%-%-%-%-%-%-%-%-%-%-%-%-%-%-%-%-%-%-%-%-%-%-%

\documentclass[12pt, oneside]{book}
\usepackage{xcolor}
\usepackage{minted}

% \usepackage[T1]{fontenc}

\usepackage{draculatheme}
\newcommand{\documentTheme}{draculafg}
\newcommand{\documentLogo}{../../images/small logo_white.png}
\newcommand{\documentBigLogo}{../../images/logo_white_text.png}
\usemintedstyle{dracula}

% \newcommand{\documentBigLogo}{../../images/Hendrix Logo.png}
% \newcommand{\documentLogo}{../../images/small logo.png}
% \newcommand{\documentTheme}{black}

%%% AESTHETICS %%%
%-%-%-%-%-%-%-%-%-%-%-%-%-%-%-%-%-%-%-%-%-%-%-%-%-%-%-%-%-%-%-%-%-%-%-%-%-%-%


%%% Dimensions and Spacing %%%
\usepackage[margin=1in]{geometry}
\setlength{\parindent}{0pt}
\usepackage{setspace}
\usepackage{mathtools}
\usepackage{esint}
\usepackage{adjustbox}
\linespread{1}
\usepackage{listings}
\usepackage{tikz}
\usetikzlibrary{shapes,backgrounds,calc,patterns,positioning, arrows.meta, decorations.pathmorphing, decorations.pathreplacing}
\usepgflibrary{shadings}
\usepackage{pgfkeys}
\usepackage{pdftexcmds}
\usepackage{shellesc}
\usepackage{algorithm, algorithmic, xspace}
%%% Define new colors %%%
\definecolor{orangehdx}{rgb}{0.96, 0.51, 0.16}

% Normal colors
\definecolor{xred}{HTML}{BD4242}
\definecolor{xblue}{HTML}{4268BD}
\definecolor{xgreen}{HTML}{52B256}
\definecolor{xpurple}{HTML}{7F52B2}
\definecolor{xorange}{HTML}{FD9337}
\definecolor{xdotted}{HTML}{999999}
\definecolor{xgray}{HTML}{777777}
\definecolor{xcyan}{HTML}{80F5DC}
\definecolor{xpink}{HTML}{F690EA}
\definecolor{xgrayblue}{HTML}{49B095}
\definecolor{xgraycyan}{HTML}{5AA1B9}

% Dark colors
\colorlet{xdarkred}{red!85!black}
\colorlet{xdarkblue}{xblue!85!black}
\colorlet{xdarkgreen}{xgreen!85!black}
\colorlet{xdarkpurple}{xpurple!85!black}
\colorlet{xdarkorange}{xorange!85!black}
\definecolor{xdarkcyan}{HTML}{008B8B}
\colorlet{xdarkgray}{xgray!85!black}

% Very dark colors
\colorlet{xverydarkblue}{xblue!50!black}

% Document-specific colors
\colorlet{normaltextcolor}{black}
\colorlet{figtextcolor}{xblue}

% Enumerated colors
\colorlet{xcol0}{black}
\colorlet{xcol1}{xred}
\colorlet{xcol2}{xblue}
\colorlet{xcol3}{xgreen}
\colorlet{xcol4}{xpurple}
\colorlet{xcol5}{xorange}
\colorlet{xcol6}{xcyan}
\colorlet{xcol7}{xpink!75!black}

% Blue-Purple (should just used colorbrewer...)
\definecolor{xrainbow0}{HTML}{e41a1c}
\definecolor{xrainbow1}{HTML}{a24057}
\definecolor{xrainbow2}{HTML}{606692}
\definecolor{xrainbow3}{HTML}{3a85a8}
\definecolor{xrainbow4}{HTML}{42977e}
\definecolor{xrainbow5}{HTML}{4aaa54}
\definecolor{xrainbow6}{HTML}{629363}
\definecolor{xrainbow7}{HTML}{7e6e85}
\definecolor{xrainbow8}{HTML}{9c509b}
\definecolor{xrainbow9}{HTML}{c4625d}
\definecolor{xrainbow10}{HTML}{eb751f}
\definecolor{xrainbow11}{HTML}{ff9709}

%------- %
% XHFILL %
%------- %



%%% Chapter Headings %%%

\newcommand{\gradientrule}{
    \begin{tikzpicture}
        \shade[left color=orangehdx, right color=black, middle color=gray] (0,0) rectangle (\linewidth,0.4pt);
    \end{tikzpicture}
}

\usepackage[Glenn]{fncychap}
\ChTitleVar{\bfseries\scshape\color{\documentTheme}} % Needed for Dracula theme
\ChNumVar{\large\selectfont\color{\documentTheme}} % Needed for Dracula theme
\ChNameVar{\large\color{\documentTheme}} % Needed for Dracula theme
\usepackage{xpatch}

% \xpatchcmd{\DOCH}
%   {\mghrulefill}{\gradientrule\mghrulefill}
%   {}{\PatchFailed}
% \xpatchcmd{\DOTI}
%   {\mghrulefill}{\gradientrule\mghrulefill}
%   {}{\PatchFailed}
% \xpatchcmd{\DOTIS}
%   {\mghrulefill}{\gradientrule\mghrulefill}
%   {}{\PatchFailed}

\xpatchcmd\DOCH
{\mghrulefill}{\color{orangehdx}\mghrulefill}
{}{\PatchFailed}
\xpatchcmd\DOTI
{\mghrulefill}{\color{orangehdx}\mghrulefill}
{}{\PatchFailed}
\xpatchcmd\DOTIS
{\mghrulefill}{\color{orangehdx}\mghrulefill}
{}{\PatchFailed}


% \usepackage[Bjornstrup]{fncychap}

% \newcommand{\gradient}[1]{
% \begin{tikzpicture}
%     \node (rect) at (0,0) [fill=blue,,path fading=East,minimum width=\linewidth,minimum height=2.5cm] {};
%     \node(title)[above left = 10pt and 10pt of rect.south east, anchor=south east, font=\CTV] {\textcolor{horange}{#1}};
%     \ifnum \thechapter>0\node[left = 10pt of rect.north east,  anchor=center, font=\CNoV] {\textcolor{horange}{\thechapter}};\fi%
% \end{tikzpicture}%
% \vskip 40pt
% }


% \renewcommand{\DOCH}{}
% \renewcommand{\DOTI}[1]{\gradient{#1}}
% \renewcommand{\DOTIS}[1]{\gradient{#1}}

%% Change Chapter Heading Placement %%
\usepackage{etoolbox}
\makeatletter
\patchcmd{\@makechapterhead}{\vspace*{50\p@}}{\vspace*{-20\p@}}{}{}
\patchcmd{\@makeschapterhead}{\vspace*{50\p@}}{\vspace*{-20\p@}}{}{}
\patchcmd{\DOTI}{\vskip 80\p@}{\vskip 40\p@}{}{}
\patchcmd{\DOTIS}{\vskip 40\p@}{\vskip 0\p@}{}{}
\makeatother



% \newcommand*\chapterlabel{}\pmod
% \titleformat{\chapter}
%   {\gdef\chapterlabel{}
%    \normalfont\sffamily\Huge\bfseries\scshape}
%   {\gdef\chapterlabel{\thechapter\ }}{0pt}
%   {\begin{tikzpicture}[remember picture,overlay]
%     \node[yshift=-3cm] at (current page.north west)
%       {\begin{tikzpicture}[remember picture, overlay]
%         \draw[fill=LightSkyBlue] (0,0) rectangle
%           (\paperwidth,3cm);
%         \node[anchor=east,xshift=.9\paperwidth,rectangle,
%               sharp corners=downhill=20pt,inner sep=11pt,
%               fill=MidnightBlue]
%               {\color{white}\chapterlabel#1};
%        \end{tikzpicture}
%       };
%    \end{tikzpicture}
%   }
% \titlespacing*{\chapter}{0pt}{50pt}{-60pt}

% \usepackage[Conny]{fncychap}
% \usepackage[Rejne]{fncychap}
% \ChNameVar{\bfseries}  % Makes the chapter "Chapter #" bold
% \ChNumVar{\bfseries}   % Makes the chapter number bold
% \ChTitleVar{\bfseries} % Makes the chapter title bold
% % Define a custom color, for example:
% \definecolor{mycolor}{RGB}{0,128,255}

% % Redefine the chapter style in Rejne to change the line colors
% \makeatletter
% \ChRuleWidth{2pt}   % Change the thickness of the lines
% \renewcommand{\DOCH}{%
%   \vspace*{-50\p@}% Moves the chapter title up/down if needed
%   {\color{mycolor} \hrule \@chapapp{} \space \thechapter \hrule}% Customizes the chapter header with color
% }
% \renewcommand{\DOTI}[1]{%
%   \vskip 20\p@ % Adjusts the space above the title
%   \bfseries #1\par % Embolden the title
%   \vskip 20\p@ % Adjusts the space below the title
% }
% \makeatother

% %% Change Chapter Heading Placement %%
% \usepackage{etoolbox}
% \makeatletter
% \patchcmd{\@makechapterhead}{\vspace*{50\p@}}{\vspace*{-20\p@}}{}{}
% \patchcmd{\@makeschapterhead}{\vspace*{50\p@}}{\vspace*{-20\p@}}{}{}
% \patchcmd{\DOTI}{\vskip 80\p@}{\vskip 40\p@}{}{}
% \patchcmd{\DOTIS}{\vskip 40\p@}{\vskip 0\p@}{}{}
% \makeatother

\renewcommand{\thesection}{\thechapter.\arabic{section}} %% Chapter.Section Numbering


%%% FIGURES %%%
\usepackage{graphicx}  
% \numberwithin{figure}{section}
\usepackage{float}
\usepackage{caption}

%%% Hyperlinks %%%
\usepackage{hyperref}
\definecolor{horange}{HTML}{f58026}
\hypersetup{
	colorlinks=true,
	linkcolor=horange,
	filecolor=horange,      
	urlcolor=horange,
}


%% Headers and Footers %%
\usepackage{fancyhdr} % This should be set AFTER setting up the page geometry
\pagestyle{fancy} % options: empty , plain , fancy
\fancyhead[R]{\textcolor{draculafg}{\assignmentname}}
\fancyhead[L]{\textcolor{draculafg}{Hendrix College}}
\fancyhead[C]{\includegraphics[height=.50cm]{\documentLogo}} % Use for Dracula
\usepackage{xpatch}
\xpretocmd\headrule{\color{orangehdx}}{}{\PatchFailed}
\setlength{\footskip}{0.5in}
\setlength{\headheight}{18.5764pt}


%%%%% Colored Boxes %%%%%
\usepackage{tcolorbox}
\tcbuselibrary{skins}
\tcbuselibrary{theorems}
% \tcbuselibrary{minted}
\newcounter{BoxCounter}
% \usepackage{dingbat}

%-%-%-%-%-%-%-%-%-%-%-%-%-%-%-%-%-%-%-%-%-%-%-%-%-%-%-%-%-%-%-%-%-%-%-%-%-%-%

%% MATH PACKAGES, ENVIRONMENTS, COMMANDS %%
%-%-%-%-%-%-%-%-%-%-%-%-%-%-%-%-%-%-%-%-%-%-%-%-%-%-%-%-%-%-%-%-%-%-%-%-%-%-%
%You'll need your own packages, theorem types, and commands.

\usepackage{fix-cm}
\usepackage{amsmath,amsthm, amsfonts, amssymb} 
\usepackage{array, makecell}
\usepackage{colortbl}
\newcommand{\thickvrule}{\vrule width 1pt}
\newcommand{\thickhrule}{\Xhline{1pt}}

\usepackage[framemethod=tikz]{mdframed}
\usepackage{mathrsfs}
\usepackage{changepage}
\usepackage{multicol}
\usepackage{slashed}
\usepackage{enumerate}
\usepackage{braket}
\usepackage{booktabs}
\usepackage{enumitem}
\usepackage{kantlipsum}  %This package lets us generate random text for example purposes.
\usepackage{pgfplots}


%%% Custom Commands %%%
% Natural Numbers 
\newcommand{\N}{\mathbb{N}}

% Whole Numbers
\newcommand{\W}{\mathbb{W}}

% Integers
\newcommand{\Z}{\mathbb{Z}}

% Rational Numbers
\newcommand{\Q}{\mathbb{Q}}

% Real Numbers
\newcommand{\R}{\mathbb{R}}

% Complex Numbers
\newcommand{\C}{\mathbb{C}}

\newcommand{\I}{\mathbb{I}}

\newcommand{\pfs}{\noindent\makebox[\linewidth]{\rule{\textwidth}{0.4pt}}\vspace{0.5cm}}

\newcommand{\mysqrt}[1]{%
	\mathpalette\foo{#1}%
}
\newcommand{\dmysqrt}[1]{%
	\mathpalette\foodisplay{#1}%
}

% !TeX spellcheck = off
\newcommand{\foo}[2]{%
	% #1: math style, #2: content
	\sbox0{$#1\sqrt{#2}$}% Measure the size of the standard sqrt in the current style
	\begin{tikzpicture}[baseline=(sqrt.base)]
		\node[inner sep=0, outer sep=0] (sqrt) {$#1\sqrt{#2}$}; % Use the current math style
		\draw([yshift=-0.045em]sqrt.north east) -- ++(0,-0.5ex); % Draw the tick
	\end{tikzpicture}%
}
% !TeX spellcheck = off
\newcommand{\foodisplay}[2]{%
	% #1: math style, #2: content
	\sbox0{$#1\sqrt{#2}$}% Measure the size of the standard sqrt in the current style
	\begin{tikzpicture}[baseline=(sqrt.base)]
		\node[inner sep=0, outer sep=0] (sqrt) {$\displaystyle\sqrt{#2}$}; % Force displaystyle
		\draw[line width=0.4pt] ([yshift=-0.044em]sqrt.north east) -- ++(0,-0.5ex); % Draw the tick
	\end{tikzpicture}%
}

\renewcommand{\theenumi}{\arabic{enumi}}
\renewcommand{\labelenumi}{\textbf{\theenumi}.}

% \newcommand{}{\marginnote{\includegraphics[width=2em]{caution.png}}}

\newmdenv[
	topline=false,
	bottomline=true,
	rightline=false,
	leftline=true,
	linewidth=1.5pt,
	linecolor=black, % default color, will be overridden in custom commands
	backgroundcolor=draculabg, % Needed for Dracula theme
	fontcolor=\documentTheme, % Needed for Dracula theme
	innertopmargin=0pt,
	innerbottommargin=5pt,
	innerrightmargin=10pt,
	innerleftmargin=10pt,
	leftmargin=0pt,
	rightmargin=0pt,
	skipabove=\topsep,
	skipbelow=\topsep,
]{customframedproof}

\newenvironment{proofpart}[2][black]{
    \begin{mdframed}[
        topline=false,
        bottomline=false,
        rightline=false,
        leftline=true,
        linewidth=1pt,
        linecolor=#1!40, % Custom color
        % innertopmargin=10pt,
        % innerbottommargin=10pt,
        innerleftmargin=10pt,
        innerrightmargin=10pt,
        leftmargin=0pt,
        rightmargin=0pt,
        % skipabove=\topsep,
        % skipbelow=\topsep%
    ]
    \noindent
    \begin{minipage}[t]{0.08\textwidth}%
        \textbf{#2}%
    \end{minipage}%
    \begin{minipage}[t]{0.90\textwidth}%
        \begin{adjustwidth}{0pt}{0pt}%
}{
    \end{adjustwidth}
    \end{minipage}
    \end{mdframed}
}

\newenvironment{tipbox}
	{%
		\par\noindent%
		% Top dashed line (using \textwidth)
		\tikz[baseline]{\draw[dash pattern=on 3pt off 2pt, line width=0.5pt, color=draculapurple] (0,0) -- (\textwidth,0);}%
		\par\vspace{0.65em}%
		\noindent\textbf{\textcolor{draculapurple}{TIP:}}\quad%
	}
	{%
		\par%
		% Bottom dashed line
		\noindent%
		\tikz[baseline]{\draw[dash pattern=on 3pt off 2pt, line width=0.5pt, color=draculapurple] (0,0) -- (\textwidth,0);}%
		\par\vspace{0.35em}
	}

\newenvironment{notebox}
	{%
		\par\noindent%
		% Top dashed line (using \textwidth)
		\tikz[baseline]{\draw[dash pattern=on 3pt off 2pt, line width=0.5pt, color=draculagreen] (0,0) -- (\textwidth,0);}%
		\par\vspace{0.65em}%
		\noindent\textbf{\textcolor{draculagreen}{NOTE:}}\quad%
	}
	{%
		\par%
		% Bottom dashed line
		\noindent%
		\tikz[baseline]{\draw[dash pattern=on 3pt off 2pt, line width=0.5pt, color=draculagreen] (0,0) -- (\textwidth,0);}%
		\par\vspace{0.35em}
	}

\newenvironment{warningbox}
	{%
		\par\noindent%
		% Top dashed line (using \textwidth)
		\tikz[baseline]{\draw[dash pattern=on 3pt off 2pt, line width=0.5pt, color=draculared] (0,0) -- (\textwidth,0);}%
		\par\vspace{0.65em}%
		\noindent\textbf{\textcolor{draculared}{WARNING:}}\quad%
	}
	{%
		\par%
		% Bottom dashed line
		\noindent%
		\tikz[baseline]{\draw[dash pattern=on 3pt off 2pt, line width=0.5pt, color=draculared] (0,0) -- (\textwidth,0);}%
		\par\vspace{0.35em}
	}

\newenvironment{solution}{\textit{Solution.}}

\newcommand{\sol}[1]{
    \begin{customframedproof}[linecolor=orangehdx!75,]
        \begin{solution}
        #1
        \end{solution}
    \end{customframedproof}
}

\newcommand{\pf}[1]{
    \begin{customframedproof}[linecolor=orangehdx!75]
        \begin{proof}
        #1
        \end{proof}
    \end{customframedproof}
}

\def \proofDistance {10pt}

\newcommand{\disabs}[1]{\ensuremath{\left|#1\right|}}
\newcommand{\limn}{\lim_{n \rightarrow \infty}}
\newcommand{\limx}[2]{\lim_{x \rightarrow #1}#2}

\newcommand{\limc}[1]{\lim_{x \rightarrow c}#1}

\newcommand{\ninn}{n \in \N}

\newcommand{\mb}[1]{\mathbf{#1}}

\newcommand{\limsupn}[1]{\limsup_{n\rightarrow \infty}#1}
\newcommand{\liminfn}[1]{\liminf_{n\rightarrow \infty}#1}


\newcommand{\proj}{\text{proj}}

\newcommand{\p}{\partial}

\newcommand{\dydx}{\frac{dy}{dx}}
\newcommand{\dxdy}{\frac{dx}{dy}}
\newcommand{\dydt}{\frac{dy}{dt}}
\newcommand{\dxdt}{\frac{dx}{dt}}
\newcommand{\dzdt}{\frac{dz}{dt}}

\newcommand{\barNotationT}[1]{\bigg|_{t = #1}}

\newcommand{\cyanit}[1]{\textit{\textcolor{cyan}{#1}}}

\newcommand{\brackett}[1]{\left\langle #1 \right\rangle}

\newcommand{\norm}[1]{\left\lVert \mathbf{#1}\right\rVert}

\newcommand{\imb}{\mb{i}}
\newcommand{\jmb}{\mb{j}}
\newcommand{\kmb}{\mb{k}}
\newcommand{\rmb}{\mb{r}}
\newcommand{\umb}{\mb{u}}

\newcommand{\vecfuc}[2]{\mb{#1}(#2)}
\newcommand{\dvecfuc}[2]{\mb{#1}'(#2)}
\newcommand{\normdvecfuc}[2]{\|\mb{#1}'(#2)\|}

\newcommand{\funccolon}{\, \colon}

\newcommand{\squigglyline}{%
    \noindent
    \tikz[baseline=-0.5ex]{
        \draw[decorate, decoration={snake, amplitude=0.5mm, segment length=3mm}] 
        (0,0) -- (\dimexpr\linewidth\relax,0);
    }%
}

\newcommand{\gen}[1]{\langle #1 \rangle}
\newcommand{\abs}[1]{\left| #1 \right|}
\newcommand{\generator}[1]{\langle #1 \rangle}


% end of preamble
%-%-%-%-%-%-%-%-%-%-%-%-%-%-%-%-%-%-%-%-%-%-%-%-%-%-%-%-%-%-%-%-%-%-%-%-%-%-%

\begin{document}

\numberwithin{BoxCounter}{section}


%-%-%-%-%-%-%-%-%-%-%-%-%-%-%-%-%-%-%-%-%-%-%-%-%-%-%-%-%-%-%-%-%-%-%-%-%-%-%
%%% COVER PAGE %%%
%-%-%-%-%-%-%-%-%-%-%-%-%-%-%-%-%-%-%-%-%-%-%-%-%-%-%-%-%-%-%-%-%-%-%-%-%-%-%

\newcommand{\assignmentname}{Homework 6: Sections 13 \& 14}

% Do not use all caps.
\newcommand{\cussubtitle}{Algebra}
% Date format should be like: "January 1, 2001"
\newcommand{\finaldate}{November 18, 2025}
% Be sure to include degree recognition (e.g., B.S., M.S., Ph.D)
\newcommand{\professor}{Dr. Christopher Camfield, Ph.D.}

%-%-%-%-%-%-%-%-%-%-%-%-%-%-%-%-%-%-%-%-%-%-%-%-%-%-%-%-%



\begin{titlepage}
    \begin{center}

        \vspace*{-2cm}
        \includegraphics[width=0.8\textwidth]{\documentBigLogo}\\
        \vfill

        % Horizontal line above the title in 'horange' color
        \textcolor{horange}{\rule{\textwidth}{1.0pt}}

        \vspace{2em}

        {\huge \textbf{\assignmentname}}

        \vspace{1em} % Space between the title and the bottom line

        \textcolor{horange}{\rule{\textwidth}{1.0pt}}

        \vspace*{1\baselineskip}

        {\LARGE \textbf{\cussubtitle}}

        \begin{large}
            \vspace*{5\baselineskip}

            \vspace*{1\baselineskip}

            \emph{Author} \\[1ex]
            %Submitted by \\[\baselineskip]
            {\Large Paul Beggs \\ \par} % Editor list
            {\href{mailto:BeggsPA@Hendrix.edu}{{BeggsPA@Hendrix.edu}}}\\ % Editor affiliation

            \vspace*{1\baselineskip}

            \textit{Instructor} \\[1ex] % Tagline(s) or further description
            \professor

            \vspace*{1\baselineskip}

            \textit{Due}\\[1ex]
            {\scshape  \finaldate} \\[0.3\baselineskip] % Year published

            \thispagestyle{empty}

        \end{large}
    \end{center}
\end{titlepage}
% -%-%-%-%-%-%-%-%-%-%-%-%-%-%-%-%-%-%-%-%-%-%-%-%-%-%-%-%-%-%-%-%-%-%-%-%-%-%
\section*{Section 13}

% Section 13: Exercises 4, 5, 19, 23, 44, 45, 49

In Exercises 4 and 5, determine whether the given map \(\varphi\) is a homomorphism. [\textit{Hint:} The straightforward way to proceed is to check whether \(\varphi(ab) = \varphi(a)\varphi(b)\) for all \(a\) and \(b\) in the domain of \(\varphi\). However, if we should happen to notice that \(\varphi^{-1}[\{ e' \}]\) is not a subgroup whose left and right cosets coincide, or that \(\varphi\) does not satisfy the properties given in Exercise 44 or 45 for finite groups, then we can say at once that \(\varphi\) is not a homomorphism.]

\begin{enumerate}
    \setcounter{enumi}{3}
    \item Let \(\varphi \funccolon \Z_{6} \to \Z_{2}\) be given by \(\varphi(x) = \text{the remainder of } x\) when divided by 2, as in the division algorithm.
    \sol{
        We know \(\varphi(x) = x \bmod 2\), and we have binary operators \((a + b) \bmod 6\) for \(\Z_{6}\) and \((a + b) \bmod 2\) for \(\Z_{2}\). This leaves us with the following equation to check: 
        \[
            \varphi((a + b) \bmod 6) \stackrel{?}{=} (\varphi(a)  + \varphi(b)) \bmod 2
        \]
        For the left-hand side:
        \[
            \varphi((a + b) \bmod 6) = ((a + b) \bmod 6) \bmod 2.
        \]
        This simplifies down to \((a + b) \bmod 2\) since \(6 \equiv 0 \pmod{2}\). For the right-hand side:
        \[
            (\varphi(a) + \varphi(b)) \bmod 2 = (a \bmod 2 + b \bmod 2) \bmod 2.
        \]
        Since addition is commutative and associative in both groups, we can rewrite this as:
        \[
            (a + b) \bmod 2.
        \]
        Therefore, the equation holds, and the map is a homomorphism.
    }
    
    \item Let \(\varphi \funccolon \Z_{9} \to \Z_{2}\) be given by \(\varphi(x) = \text{the remainder of } x\) when divided by 2, as in the division algorithm.
    \sol{
        This is not a homomorphism because the two sides of the equation are not equal:
        \[
            \varphi((3 + 7) \bmod 9) = \varphi(1) = 1,
        \]
        but 
        \[
            (\varphi(3) + \varphi(7)) \bmod 2 = (1 + 1) \bmod 2 = 0.
        \]
    }

\end{enumerate}

\newpage

In Exercises 19 and 23, compute the indicated quantities for the given homomorphism \(\varphi\). (See Exercise 46.)

\begin{enumerate}
    \setcounter{enumi}{18}
    \item \(\ker(\varphi)\) and \(\varphi(20)\) for \(\varphi \funccolon \Z \to S_{8}\) such that \(\varphi(1) = (1,4,2,6)(2,5,7)\).
    \sol{
        We know \(\varphi(1) = (1,4,2,6)(2,5,7) = (1,4,2,5,7,6)\) has order 6, so \(\ker(\varphi) = 6\Z\). Then, we know from the homomorphism property:
        \[
            \varphi(20) = (\varphi(1))^{20} = (\varphi(1))^{18} (\varphi(1))^{2} = (\varphi(1))^{2} = (1,2,7)(4,5,6). 
        \]
    }

    \setcounter{enumi}{22}
    \item \(\ker(\varphi)\) and \(\varphi(4, 6)\) for \(\varphi \funccolon \Z \times \Z \to \Z \times \Z\) where \(\varphi(1, 0) = (2, -3)\) and \(\varphi(0, 1) = (-1, 5)\).
    \sol{
        We have:
        \[
            \varphi(x, y) = \varphi(x, 0) + \varphi(0, y) = x\varphi(1, 0) + y\varphi(0, 1) = (2x - y, -3x + 5y).
        \]
        To find the kernel, we set \(\varphi(x, y) = (0, 0)\):
        \[
                2x - y = 0 \quad \text{and} \quad -3x + 5y = 0
        \]
        From the first equation, we have \(y = 2x\). Substituting into the second equation gives:
        \[
            -3x + 5(2x) = -3x + 10x = 7x = 0 \implies x = 0.
        \]
        Thus, \(y = 0\) as well. Therefore, \(\ker(\varphi) = \{(0, 0)\}\). Now, we find \(\varphi(4, 6)\):
        \[
            \varphi(4, 6) = (2(4) - 6, -3(4) + 5(6)) = (8 - 6, -12 + 30) = (2, 18).
        \]
    }

\end{enumerate}



\begin{enumerate}

    \setcounter{enumi}{43}
    \item Let \(\varphi \funccolon G \to G'\) be a group homomorphism. Show that if \(|G|\) is finite, then \(|\varphi[G]|\) is finite and is a divisor of \(|G|\).
    \pf{
        Let \(\varphi \funccolon G \to G'\) be a group homomorphism, and let \(H = \ker(\varphi)\). Our goal is to show that the order of the preimage of \(G\), \(|\varphi[G]|\), is finite, and a divisor of \(|G|\), given \(G\) is finite. The theorem \(aH = \varphi^{-1}[\{\varphi(a)\}]\) maps a single coset to the same single element \(\varphi(a)\) in the image. This establishes a one-to-one correspondence between the set of all cosets (\(G/H\)) and the set of all images, \(\varphi[G]\). Because there is a one-to-one correspondence, the two sets must have equal size: \(|G/H| = |\varphi[G]|\). Then, by Lagrange's Theorem, since \(G\) is finite, the number of cosets (\(G/H\)) must be a finite number that divides the order of the group, \(|G|\).  Since \(|G/H| = |\varphi[G]|\), it follows that \(|\varphi[G]|\) must also be a finite number that divides \(|G|\).
    }    

\newpage

    \item Let \(\varphi \funccolon G \to G'\) be a group homomorphism. Show that if \(|G'|\) is finite, then \(|\varphi[G]|\) is finite and is a divisor of \(|G'|\).
    \pf{
        From Theorem 13.12 (3) (the fundamental properties of homomorphisms), if \(H\) is a subgroup of \(G\), then \(\varphi[H]\) is a subgroup of \(G'\). Now, let \(H = G\). It is certainly true that \(G\) is a subgroup of \(G\), so \(\varphi[G]\) is a subgroup of \(G'\). Then, by Lagrange's Theorem, since \(|G'|\) is finite, the order of its subgroup, \(|\varphi[G]|\) is also a finite number that divides \(|G'|\). 
    }

    \setcounter{enumi}{48}
    \item Show that if \(G\), \(G'\), and \(G''\) are groups and if \(\varphi \funccolon G \to G'\) and \(\gamma \funccolon G' \to G''\) are homomorphisms, then the composite map \(\gamma \varphi \funccolon G \to G''\) is a homomorphism.
    \pf{
        Let \(a, b \in G\). Then,
        \begin{align}
            (\gamma \varphi)(ab) &= \gamma(\varphi(ab)) \\
            &= \gamma(\varphi(a)\varphi(b)) \\
            &= \gamma(\varphi(a))\gamma(\varphi(b)) \\
            &= (\gamma \varphi)(a)(\gamma \varphi)(b).
        \end{align}
        Equations (1) and (4) are from the definition of a composite map, and (2) and (3) are from the homomorphic properties of \(\varphi\) and \(\gamma\), respectively. Therefore, we have shown that the composite map is a homomorphism.
    }
\end{enumerate}

\section*{Section 14}
% Section 14: Exercises 2, 6, 11, 15, 27, 30

In Exercises 2 and 6, find the order of the given factor group.

\begin{enumerate}
    \setcounter{enumi}{1}
    \item \((\Z_4 \times \Z_{12}) / (\gen{2} \times \gen{2})\)
    \sol{
        In \(\Z_{4}\), the subgroup generated by \(\gen{2}\) is \(\{0,2\}\). Then, in \(\Z_{12}\), the subgroup generated by \(\gen{2}\) is \(\{0,2,4,6,8,10\}\). Thus, the subgroup \(\gen{2} \times \gen{2}\) has order \(2 \times 6 = 12\), and \(\Z_{4} \times \Z_{12}\) has order \(4 \times 12 = 48\). By Lagrange's Theorem, the order of the factor group is \(48 / 12 = 4\).
    }

    \setcounter{enumi}{5}
    \item \((\Z_{12} \times \Z_{18}) / \gen{(4,3)}\)
    \sol{
        The order of \(\Z_{12} \times \Z_{18}\) is \(12 \times 18 = 216\). In \(\Z_{12} \times \Z_{18}\), the subgroup generated by \(\gen{(4,3)}\) has order 6. So, by Lagrange's Theorem, the order of the factor group is \(216 / 6 = 36\).
    }

\end{enumerate}

\newpage

In Exercises 11 and 15, give the order of the element in the factor group.

\begin{enumerate}

    \setcounter{enumi}{10}
    \item \((2, 1) + \gen{(1, 1)}\) in \((\Z_3 \times \Z_6) / \gen{(1, 1)}\)
    \sol{
        In \(\Z_3 \times \Z_6\), \(\gen{(1,1)} = \{(1,1), (2,2), (0,3), (1,4), (2,5), (0,0)\}\). Then, we test the powers of \((2,1)\) until we find one that is in \(\gen{(1,1)}\):
        \[
            (2,1)(2,1) = (1,2), \quad (1,2)(2,1) = (0,3).
        \]
        So, the order of \((2, 1) + \gen{(1, 1)}\) in \((\Z_3 \times \Z_6) / \gen{(1, 1)}\) is 3. 
    }

    \setcounter{enumi}{14}
    \item \((2, 0) + \gen{(4, 4)}\) in \((\Z_6 \times \Z_8) / \gen{(4, 4)}\)
    \sol{
        Since \((2,0)\) is in \(\gen{(4, 4)}\), \((2,0) + \gen{(4,4)}\) has order 1.
    }

    \setcounter{enumi}{26}
    \item A subgroup \(H\) is \textbf{conjugate to a subgroup} \(K\) of a group \(G\) if there exists an inner automorphism \(i_g\) of \(G\) such that \(i_g[H] = K\). Show that conjugacy is an equivalence relation on the collection of subgroups of \(G\).
    \pf{
        Let \(H\), \(K\), and \(L\) be subgroups of \(G\). To show that conjugacy is an equivalence relation on the collection of subgroups of \(G\), we must show the following:
        \begin{itemize}
            \item \textbf{Reflexive:} We must show that \(H\) is conjugate to itself. That is, we must find an element \(g \in G\) such that \(H = i_{g}[H] = gHg^{-1}\). The obvious choice is the identity element \(e \in G\):
            \[
                H = i_{e}[H] = eHe^{-1} = \{ehe^{-1} \mid h \in H\} = \{h \mid h \in H\} = H.
            \]
            Thus, we have found an element \(g = e\) such that \(i_{g}[H] = H\), so the subgroup is conjugate to itself. Therefore, the relation is reflexive.
            \item \textbf{Symmetric:} Assume \(H\) to be congruent to \(K\) (i.e., \(K = i_{g}[H] = gHg^{-1}\) for some \(g \in G\)). We must show \(K\) is congruent to \(H\). That is, for some \(g' \in G\), \(H = i_{g'}[K] = g'Kg'^{-1}\). If we multiply \(K = gHg^{-1}\) by \(g^{-1}\) and \(g\) on the left and right, respectively, we get \(g^{-1}Kg = H\). Thus, we have found our elements \(g' = g^{-1}\), and \(g'^{-1} = g\). Therefore, the relation is symmetric.
            \item \textbf{Transitive:} Assume \(H\) to be congruent to \(K\) and \(K\) to be congruent to \(L\). Then, \(K = gHg^{-1}\) and \(L = pKp^{-1}\) for some \(p \in G\). We want to show \(L = g''Hg''^{-1}\). We can do that by using substitution:
            \[
                L = pKp^{-1} = pgHg^{-1}p^{-1}.
            \]
            Notice \(pg = g''\) and \(g^{-1}p^{-1} = (pg)^{-1} = g''\). The product \(pg\) and \((pg)^{-1}\) are both in \(G\) because \(G\) is a group. Thus, we have shown that \(H\) is conjugate to \(L\). Therefore, the relation is transitive. \qedhere
        \end{itemize}
    }
\newpage
    \setcounter{enumi}{29}
    \item Let \(H\) be a normal subgroup of a group \(G\), and let \(m = (G \funccolon H)\). Show that \(a^m \in H\) for every \(a \in G\).
    \pf{
        Let \(H\) be a normal subgroup of \(G\) and let \(m = (G \funccolon H)\). Because \(H\) is normal, we can form the factor group \(G/H\). The order of this group, \(|G/H|\), is equal to the index of \(H\) in \(G\), so \(|G/H| = m\). Now, let \(a\) be any element in \(G\). The corresponding element in the factor group \(G/H\) is the coset \(aH\). Let \(k\) be the order of this element \(aH\) in the factor group. By the definition of order, \((aH)^{k} = H\) (where \(H\) is the identity element of the factor group). By Lagrange's Theorem, the order of an element must divide the order of the group. Therefore, \(k\) divides \(m\). This implies that \(m = kq\) for some integer \(q\). Now, we can compute \((aH)^{m}\):
        \[
            (aH)^{m} = (aH)^{kq} = \left((aH)^{k}\right)^{q}.
        \]
        Since \((aH)^{k} = H\), we can substitute this back in:
        \[
            \left((aH)^{k}\right)^{q} = H^{q} = H.
        \]
        Thus, we have shown that \((aH)^{m} = a^{m}H = H\). By the properties of cosets, \(gH = H\) implies \(g \in H\). Consequently, \(a^{m} \in H\).
    }


\end{enumerate}


%-%-%-%-%-%-%-%-%-%-%-%-%-%-%-%-%-%-%-%-%-%-%-%-%-%-%-%-%-%-%-%-%-%-%-%-%-%-%
\end{document}
%-%-%-%-%-%-%-%-%-%-%-%-%-%-%-%-%-%-%-%-%-%-%-%-%-%-%-%-%-%-%-%-%-%-%-%-%-%-%

%-%-%-%-%-%-%-%-%-%-%-%-%-%-%-%-%-%-%-%-%-%-%-%-%-%-%-%-%-%-%-%-%-%-%-%-%-%-%