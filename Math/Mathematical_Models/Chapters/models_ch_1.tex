\section{The Five-Step Method}

\textbf{Step 1: Ask the Question.}

\begin{itemize}
    \item Make a list of all the variables in the problem, including appropriate units.
    \item Be careful not to confuse variables and constants.
    \item State any assumptions you are making about these variables, including equations and inequalities.
    \item Check units to make sure that your assumptions make sense.
    \item State the objective of the problem in precise mathematical terms.
\end{itemize}

\textbf{Step 2: Select the modeling approach.}


\begin{itemize}
    	\item Choose a general solution procedure to be followed in solving this problem.
    	\item Generally speaking, success in this step requires experience, skill, and
    familiarity with the relevant literature.
    	\item In this book we will usually specify the modeling approach to be used.
\end{itemize}

\textbf{Step 3. Formulate the model.}

\begin{itemize}
    	\item Restate the question posed in step 1 in the terms of the modeling approach
    specified in step 2.
    	\item You may need to relabel some variables specified in step 1 in order
    to agree with the notation used in step 2.
    	\item Note any additional assumptions made in order to fit the problem described in step 1 into the mathematical structure specified in step 2.
\end{itemize}

\textbf{Step 4. Solve the model.}

\begin{itemize}
    	\item Apply the general solution procedure specified in step 2 to the specific
    problem formulated in step 3.
    	\item Be careful in your mathematics. Check your work for math errors. Does
    your answer make sense?
    	\item Use appropriate technology. Computer algebra systems, graphics, and
    numerical software will increase the range of problems within your grasp,
    and they also help reduce math errors.
\end{itemize}

\textbf{Step 5. Answer the question.}

\begin{itemize}
    	\item Rephrase the results of step 4 in nontechnical terms.
    	\item Avoid mathematical symbols and jargon.
    	\item Anyone who can understand the statement of the question as it was presented to you should be able to understand your answer.
\end{itemize}

\begin{example}
    {Pigs 1} A pig weighing 200 pounds gains 5 pounds per day and costs 45 cents per day to keep. The market price for pigs is 65 cents per pound, but is falling at 1 cent per day. When should the pig be sold to maximize profits?
\end{example}

\lesol{
    Begin by labeling \textcolor{draculared}{variables} and \textcolor{draculablue}{parameters}, and relate them with \textcolor{draculagreen}{equations}: \\
    
    \begin{tabular}{lp{10cm} l}
        \(t\) & \textcolor{draculared}{\textit{time (days)}}, \\
        \(m_{0} = 0.65\) & \textcolor{draculablue}{\textit{initial market price (\$)}}, \\
        \(r_{m} = -0.01\) & \textcolor{draculablue}{\textit{rate of change of market price (\$)}}, \\
        \(w_{0} = 200\) & \textcolor{draculablue}{\textit{initial weight (lbs)}}, \\
        \(r_{w} = 5\) & \textcolor{draculablue}{\textit{rate of change of weight (lbs)}}, \\
        \(d = 0.45\) & \textcolor{draculablue}{\textit{daily costs}}, \\
        \(m(t) = m_{0} + r_{m}t\) & \textcolor{draculagreen}{\textit{market price after \(t\) days (\$)}}, \\
        \(w(t) = w_{0} + r_{0}t\) & \textcolor{draculagreen}{\textit{pig's weight after \(t\) days (lbs)}}, \\
        \(c(t) = dt\) & \textcolor{draculagreen}{\textit{total cost after \(t\) days (\$)}},  \\
        \(p(t) = r(t) - c(t)\) & \textcolor{draculagreen}{\textit{profit after \(t\) days (\$)}}. \\[1em]
    \end{tabular}

    We take the derivative of \(p(t)\) and set it equal to 0 to get the critical points, which we can then evaluate. 
}


\begin{figure}[ht]
    \begin{minted}{python}
    # Example 1.1 Pig Problem
    t = var('t')  # time in days

    # Parameters
    w0 = 200   # init weight of pig
    rw = 5     # growth rate of pig (lb/day)
    m0 = 0.65  # init market price ($/pound)
    rm = -0.01 # market price rate of change ($/lbs/day)
    d  = 0.45  # daily cost to keep the pig

    # Functions
    w(t) = w0 + rw*t   # weight of pig
    m(t) = m0 + rm*t   # market price
    r(t) = w(t)*m(t)   # revenue
    c(t) = d*t         # cost
    p(t) = r(t) - c(t) # profit

    # Solve the Model
    pprime    = p.derivative(t)        # derivative of profit
    critpts   = solve(pprime(t)==0,t)  # critical points
    optimum_t = critpts[0].rhs()       # best time to sell the pig
    maxprofit = p(optimum_t)           # maximum profit

    print(f'The best time to sell the pig is in {optimum_t} days.')  # optimum_t=8
    print(f'The maximum profit is ${round(maxprofit,3)}.')           # maxprofit=133.20
    \end{minted}
    \caption{Python Code for the Pig Problem}
\end{figure}

