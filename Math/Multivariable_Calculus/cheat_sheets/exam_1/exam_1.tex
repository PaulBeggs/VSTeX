\documentclass{article}
\usepackage[landscape]{geometry}
\usepackage{url}
\usepackage{multicol}
\usepackage{amsmath}
\usepackage{esint}
\usepackage{amsthm}
\usepackage{amsfonts}
\usepackage{booktabs}
\usepackage{tikz}
\usetikzlibrary{decorations.pathmorphing}
\usetikzlibrary{graphs}
\usetikzlibrary{shapes}
\usetikzlibrary{backgrounds}
\usetikzlibrary{calc}
\usetikzlibrary{patterns}
\usetikzlibrary{positioning, fit, arrows.meta}
\usepackage{amsmath,amssymb}
\usepackage{subcaption}
\usepackage[english]{babel}
\usepackage{enumitem}


\usepackage{colortbl}
\usepackage{xcolor}
\usepackage{mathtools}
\usepackage{amsmath,amssymb}
\usepackage{enumitem}
\makeatletter

\newcommand*\bigcdot{\mathpalette\bigcdot@{.5}}
\newcommand*\bigcdot@[2]{\mathbin{\vcenter{\hbox{\scalebox{#2}{\(\m@th#1\bullet\)}}}}}
\makeatother

\title{Discrete Exam 2 Note Sheets}
\usepackage[utf8]{inputenc}

\advance\topmargin-.8in
\advance\textheight
3in
\advance\textwidth3in
\advance\oddsidemargin-1.40in
\advance\evensidemargin-1.45in
\parindent0pt
\parskip1pt
\newcommand{\hr}{\centerline{\rule{3.5in}{1pt}}}

% Natural Numbers 
\newcommand{\N}{\ensuremath{\mathbb{N}}}

% Whole Numbers
\newcommand{\W}{\ensuremath{\mathbb{W}}}

% Integers
\newcommand{\Z}{\ensuremath{\mathbb{Z}}}

% Rational Numbers
\newcommand{\Q}{\ensuremath{\mathbb{Q}}}

% Real Numbers
\newcommand{\R}{\ensuremath{\mathbb{R}}}

% Complex Numbers
\newcommand{\C}{\ensuremath{\mathbb{C}}}

\newcommand{\cyanit}[1]{\textbf{\textit{#1}}}

\newcommand{\brackett}[1]{\left\langle #1 \right\rangle}

\newcommand{\norm}[1]{\left\lVert \mathbf{#1}\right\rVert}

\newcommand{\dydx}{\frac{dy}{dx}}
\newcommand{\dxdy}{\frac{dx}{dy}}
\newcommand{\dydt}{\frac{dy}{dt}}
\newcommand{\dxdt}{\frac{dx}{dt}}
\newcommand{\dzdt}{\frac{dz}{dt}}


\newcommand{\cost}{\cos(t)}
\newcommand{\sint}{\sin(t)}
\newcommand{\tant}{\tan(t)}

% 
\newcommand{\barNotationT}[1]{\bigg|_{t = #1}}

\newcommand{\mb}[1]{\mathbf{#1}}

\newcommand{\imb}{\mb{i}}
\newcommand{\jmb}{\mb{j}}
\newcommand{\kmb}{\mb{k}}
\newcommand{\rmb}{\mb{r}}
\newcommand{\umb}{\mb{u}}

\newcommand{\vecfuc}[2]{\mb{#1}(#2)}
\newcommand{\dvecfuc}[2]{\mb{#1}'(#2)}
\newcommand{\normdvecfuc}[2]{||\mb{#1}'(#2)||}

\DeclareMathOperator{\arccot}{arccot}
\DeclareMathOperator{\arcsec}{arcsec}
\DeclareMathOperator{\arccsc}{arccsc}

\usepackage{titlesec}

\newcommand{\mysqrt}[1]{%
  \mathpalette\foo{#1}%
}
\newcommand{\dmysqrt}[1]{%
  \mathpalette\foodisplay{#1}%
}

% !TeX spellcheck = off
\newcommand{\foo}[2]{%
  % #1: math style, #2: content
  \sbox0{$#1\sqrt{#2}$}% Measure the size of the standard sqrt in the current style
  \begin{tikzpicture}[baseline=(sqrt.base)]
    \node[inner sep=0, outer sep=0] (sqrt) {$#1\sqrt{#2}$}; % Use the current math style
    \draw([yshift=-0.045em]sqrt.north east) -- ++(0,-0.5ex); % Draw the tick
  \end{tikzpicture}%
}
% !TeX spellcheck = off
\newcommand{\foodisplay}[2]{%
  % #1: math style, #2: content
  \sbox0{$#1\sqrt{#2}$}% Measure the size of the standard sqrt in the current style
  \begin{tikzpicture}[baseline=(sqrt.base)]
    \node[inner sep=0, outer sep=0] (sqrt) {$\displaystyle\sqrt{#2}$}; % Force displaystyle
    \draw[line width=0.4pt] ([yshift=-0.044em]sqrt.north east) -- ++(0,-0.5ex); % Draw the tick
  \end{tikzpicture}%
}

\setlist[enumerate,1]{left=0pt}
\setlist[enumerate,2]{left=-10pt}


\begin{document}





















\begin{center}{\huge{\textbf{Multivariable Calculus Exam 1}}}\\
\end{center}
\begin{multicols*}{2}

    \tikzstyle{mybox} = [draw=black, fill=white, very thick,
    rectangle, rounded corners, inner sep=5pt, inner ysep=10pt]
    \tikzstyle{innerbox} = [draw=black, fill=gray!20, thick,
    rectangle, rounded corners, inner sep=5pt, inner ysep=10pt]
    \tikzstyle{fancytitle} =[fill=black, text=white, font=\bfseries]

    %------------ Measures of Center ---------------
    \begin{tikzpicture}
        \node [mybox] (box){%
            \begin{minipage}{0.46\textwidth}

                \begin{small}
                    \begin{multicols}{2}
                        \subsubsection*{Basic Derivatives}
                        \begin{align*}
                             & \frac{d}{dx} e^{f(x)} =            & f'(x)e^{f(x)}                    \\
                             & \frac{d}{dx} \sin f(x) =           & \cos f(x) \cdot f'(x)            \\
                             & \frac{d}{dx} \cos f(x) =           & -\sin f(x) \cdot f'(x)           \\
                             & \frac{d}{dx} \tan f(x) =           & \sec^2 f(x) \cdot f'(x)          \\
                             & \frac{d}{dx} \cot f(x) =           & -\csc^2 f(x) \cdot f'(x)         \\
                             & \frac{d}{dx} \sec f(x) =           & \sec f(x) \tan f(x) \cdot f'(x)  \\
                             & \frac{d}{dx} \csc f(x) =           & -\csc f(x) \cot f(x) \cdot f'(x) \\
                             & \frac{d}{dx} \ln f(x) =            & \frac{f'(x)}{f(x)}               \\
                             & \frac{d}{dx} \log_a f(x) =         & \frac{f'(x)}{f(x) \ln a}         \\
                             & \frac{d}{dx} \left(f(x)\right)^n = & n \left(f(x)\right)^{n-1} f'(x)  \\
                             & \frac{d}{dx} \dmysqrt{f(x)} =      & \frac{f'(x)}{2\dmysqrt{f(x)}}    \\
                             & \frac{d}{dx} a^x =                 & a^x \ln a                        \\
                             & \frac{d}{dx} b^{g(x)} =            & b^{g(x)} \ln b \cdot g'(x)       \\
                        \end{align*}
                        \subsubsection*{Product and Quotient}

                        \begin{align*}
                             & \frac{d}{dx} [u \cdot v] =                & u' \cdot v + u \cdot v'             \\
                             & \frac{d}{dx} \left( \frac{u}{v} \right) = & \frac{u' \cdot v - u \cdot v'}{v^2} \\
                        \end{align*}

                        \subsubsection*{Inverse Trigonometric}

                        \begin{align*}
                             & \frac{d}{dx} \arcsin f(x) = & \frac{f'(x)}{\dmysqrt{1 - \left(f(x)\right)^2}}        \\
                             & \frac{d}{dx} \arccos f(x) = & -\frac{f'(x)}{\dmysqrt{1 - \left(f(x)\right)^2}}       \\
                             & \frac{d}{dx} \arctan f(x) = & \frac{f'(x)}{1 + \left(f(x)\right)^2}                  \\
                             & \frac{d}{dx} \arccot f(x) = & -\frac{f'(x)}{1 + \left(f(x)\right)^2}                 \\
                             & \frac{d}{dx} \arcsec f(x) = & \frac{f'(x)}{|f(x)|\dmysqrt{\left(f(x)\right)^2 - 1}}  \\
                             & \frac{d}{dx} \arccsc f(x) = & -\frac{f'(x)}{|f(x)|\dmysqrt{\left(f(x)\right)^2 - 1}} \\
                        \end{align*}

                        \subsubsection*{Chain Rule}

                        \begin{align*}
                             & \frac{d}{dx} f(g(x)) = & f'(g(x)) \cdot g'(x) \\
                        \end{align*}

                        \subsubsection*{Higher-Order Derivatives}

                        \begin{align*}
                             & \frac{d^2}{dx^2} e^x =    & e^x     \\
                             & \frac{d^3}{dx^3} \sin x = & -\cos x \\
                             & \frac{d^4}{dx^4} \cos x = & \cos x  \\
                        \end{align*}
                    \end{multicols}
                \end{small}

            \end{minipage}
        };
        %------------ Measures of Center Header ---------------------
        \node[fancytitle, right=10pt] at (box.north west) {Derivation};
    \end{tikzpicture}
    \begin{tabular}{|c||c|c|c|c|c|c|}
        \hline
        \(n\) & \(\displaystyle \int_{0}^{\pi/2} \sin^{n} x\, dx\) & \(\displaystyle \int_{0}^{\pi/2} \cos^{n} x\, dx\) & \(\displaystyle \int_{0}^{\pi} \sin^{n} x\, dx\) & \(\displaystyle \int_{0}^{\pi} \cos^{n} x\, dx\) & \(\displaystyle \int_{0}^{2\pi} \sin^{n}x \, dx\) & \(\displaystyle \int_{0}^{2\pi} \cos^{n}x \, dx\) \\ \hline
        1     & 1                                                  & 1                                                  & 2                                                & 0                                                & 0                                                 & 0                                                 \\
        2     & \(\pi/4\)                                          & \(\pi/4\)                                          & \(\pi/2\)                                        & \(\pi/2\)                                        & \(\pi\)                                           & \(\pi\)                                           \\
        3     & 2/3                                                & 2/3                                                & 4/3                                              & 0                                                & 0                                                 & 0                                                 \\
        4     & \(3\pi/16\)                                        & \(3\pi/16\)                                        & \(3\pi/8\)                                       & \(3\pi/8\)                                       & \(3\pi/4\)                                        & \(3\pi/4\)                                        \\
        5     & 8/15                                               & 8/15                                               & 16/15                                            & 0                                                & 0                                                 & 0                                                 \\
        6     & \(5\pi/32\)                                        & \(5\pi/32\)                                        & \(5\pi/16\)                                      & \(5\pi/16\)                                      & \(5\pi/8\)                                        & \(5\pi/8\)                                        \\
        \hline
    \end{tabular}



    %------------ Discrete Random Variable and Distributions ---------------
    \begin{tikzpicture}
        \node [mybox] (box){%
            \begin{minipage}{0.46\textwidth}

                \begin{small}
                    \begin{multicols}{2}
                        \subsubsection*{Trigonometric Integrals}

                        \begin{align*}
                             & \int \sin x \, dx =        & -\cos x + C                        \\
                             & \int \cos x \, dx =        & \sin x + C                         \\
                             & \int\sin^{2}x =            & \frac{1}{2}(x - \sin x \cos x) + C \\
                             & \int\cos^{2}x =            & \frac{1}{2}(x + \sin x \cos x) + C \\
                             & \int \tan x \, dx =        & -\ln|\cos x| + C                   \\
                             & \int \cot x \, dx =        & \ln|\sin x| + C                    \\
                             & \int \sec x \, dx =        & \ln|\sec x + \tan x| + C           \\
                             & \int \csc x \, dx =        & -\ln|\csc x + \cot x| + C          \\
                             & \int \sec^2 x \, dx =      & \tan x + C                         \\
                             & \int \csc^2 x \, dx =      & -\cot x + C                        \\
                             & \int \sec x \tan x \, dx = & \sec x + C                         \\
                             & \int \csc x \cot x \, dx = & -\csc x + C
                        \end{align*}

                        \subsubsection*{Reduction Formulas for Sine and Cosine}

                        \begin{align*}
                             & \int \sin^{n}x \, dx = -\frac{1}{n}\sin^{n-1}x\cos x + \frac{n-1}{n}\int \sin^{n-2}x \, dx \\
                             & \int \cos^{n}x \, dx = \frac{1}{n}\cos^{n-1}x\sin x + \frac{n-1}{n}\int \cos^{n-2}x \, dx
                        \end{align*}

                        \subsubsection*{Inverse Trigonometric Integrals}

                        \begin{align*}
                             & \int \frac{1}{\mysqrt{a^{2} - x^{2}}} \, dx =  & \arcsin\left(\frac{x}{a}\right) + C            \\
                             & \int \frac{1}{a^{2} + x^{2}} \, dx =           & \frac{1}{a}\arctan\left(\frac{x}{a}\right) + C \\
                             & \int \frac{1}{x\mysqrt{x^{2} - a^{2}}} \, dx = & \frac{1}{a}\arcsec\left(\frac{x}{a}\right) + C
                        \end{align*}

                        \subsubsection*{Regular Integrals and \(e\)}

                        \begin{align*}
                             & \int x^{n} \, dx =         & \frac{1}{n+1}x^{n+1} + C \\
                             & \int \frac{1}{x} \, dx =   & \ln|x| + C               \\
                             & \int e^{x} \, dx =         & e^{x} + C                \\
                             & \int e^{ax} \, dx =        & \frac{1}{a}e^{ax} + C    \\
                             & \int e^{f(x)}f'(x) \, dx = & e^{f(x)} + C             \\
                        \end{align*}

                        \subsubsection*{Exponetials}

                        \begin{align*}
                             & \int_{0}^{{b}} e^{x} \, dx = e^{b} - 1     \\
                             & \int_{0}^{{b}} e^{-x} \, dx = 1 - e^{-b}   \\
                             & \int_{0}^{{\infty}} e^{-x} \, dx = 1       \\
                             & \int_{0}^{{\infty}} x^{n}e^{-x} \, dx = n!
                        \end{align*}

                    \end{multicols}
                \end{small}

            \end{minipage}
        };
        %------------ Discrete Random Variable and Distributions Header ---------------------
        \node[fancytitle, right=10pt] at (box.north west) {Integration};
    \end{tikzpicture}
    \vspace*{-0.25cm}
    \begin{center}
        \begin{tabular}{cccc}
            \hline
            Radians    & \(\sin(\theta)\) & \(\cos(\theta)\) & \(\tan(\theta)\) \\  \hline
            \(0\)      & \(0\)            & \(1\)            & \(0\)            \\  \hline
            \(\pi/6\)  & \(1/2\)          & \(\mysqrt{3}/2\) & \(\mysqrt{3}/3\) \\  \hline
            \(\pi/4\)  & \(\mysqrt{2}/2\) & \(\mysqrt{2}/2\) & \(1\)            \\  \hline
            \(\pi/3\)  & \(\mysqrt{3}/2\) & \(1/2\)          & \(\mysqrt{3}\)   \\ \hline
            \(\pi/2\)  & \(1\)            & \(0\)            & \(-\)            \\  \hline
            \(\pi\)    & \(0\)            & \(-1\)           & \(0\)            \\ \hline
            \(3\pi/2\) & \(-1\)           & \(0\)            & \(-\)            \\
            \hline
        \end{tabular}
    \end{center}

    \newpage

    %------------ Sets and Probability ---------------
    \begin{tikzpicture}
        \node [mybox] (box){%
            \begin{minipage}{0.46\textwidth}

                \begin{small}
                    \begin{multicols}{2}
                        \subsubsection*{Pythagorean}
                        \begin{align*}
                             & \sin^{2}\theta + \cos^{2}\theta =  1              \\
                             & \tan^{2}\theta + 1 =               \sec^{2}\theta \\
                             & 1 + \cot^{2}\theta =               \csc^{2}\theta
                        \end{align*}
                        \subsubsection*{Half Angle}
                        \begin{align*}
                             & \sin^{2}\left(\frac{x}{2}\right) =  \frac{1 - \cos x}{2}          \\
                             & \cos^{2}\left(\frac{x}{2}\right) =  \frac{1 + \cos x}{2}          \\
                             & \tan^{2}\left(\frac{x}{2}\right) =  \frac{1 - \cos x}{1 + \cos x}
                        \end{align*}
                        \subsubsection*{Double Angle}
                        \begin{align*}
                             & \sin 2x =  2\sin x \cos x                \\
                             & \cos 2x =  \cos^{2}x - \sin^{2}x         \\
                             & \cos 2x = 2\cos^{2}x - 1                 \\
                             & \cos 2x =  1 - 2\sin^{2}x                \\
                             & \tan 2x =  \frac{2\tan x}{1 - \tan^{2}x}
                        \end{align*}
                        \subsubsection*{Product to Sum}
                        \begin{align*}
                             & \sin x \sin y =  \frac{1}{2}\bigl[\cos(x-y) - \cos(x+y)\bigr] \\
                             & \cos x \cos y =  \frac{1}{2}\bigl[\cos(x-y) + \cos(x+y)\bigr] \\
                             & \sin x \cos y =  \frac{1}{2}\bigl[\sin(x+y) + \sin(x-y)\bigr] \\
                             & \cos x \sin y =  \frac{1}{2}\bigl[\sin(x+y) - \sin(x-y)\bigr]
                        \end{align*}
                        \subsubsection*{Sum to Product}
                        \begin{align*}
                             & \sin x + \sin y =  2\sin\left(\frac{x+y}{2}\right)\cos\left(\frac{x-y}{2}\right)  \\
                             & \sin x - \sin y =  2\cos\left(\frac{x+y}{2}\right)\sin\left(\frac{x-y}{2}\right)  \\
                             & \cos x + \cos y =  2\cos\left(\frac{x+y}{2}\right)\cos\left(\frac{x-y}{2}\right)  \\
                             & \cos x - \cos y =  -2\sin\left(\frac{x+y}{2}\right)\sin\left(\frac{x-y}{2}\right)
                        \end{align*}


                    \end{multicols}
                \end{small}
            \end{minipage}
        };
        %------------ Sets and Probability Header ---------------------
        \node[fancytitle, right=10pt] at (box.north west) {Trigonometric Identities};
    \end{tikzpicture}


    %------------ Continuous Random Variable and Distributions ---------------
    \begin{tikzpicture}
        \node [mybox] (box){%
            \begin{minipage}{0.46\textwidth}

                \begin{itemize}
                    \item \textbf{Slope:} \(\displaystyle \dydx\barNotationT{t_{0}} = \frac{dy/dt}{dx/dt}\barNotationT{t_{0}}.\)

                          \noindent The \cyanit{tangent line} at \(t_{0}\) is given by
                          \[
                              y = \left(\dydx\barNotationT{t_{0}}\right)\bigl(x - x(t_{0})\bigr) + y(t_{0}).
                          \]
                    \item \textbf{Concavity:} \(\displaystyle\dfrac{d^{2}y}{dx^{2}}\barNotationT{t_{0}} = \frac{d}{dt}\left(\dydx\right)\barNotationT{t_{0}} = \frac{d}{dt}\left(\frac{dy/dt}{dx/dt}\right)\barNotationT{t_{0}}.\)
                    \item \textbf{Area Under a Curve:} \(\displaystyle\int_{t_{a}}^{t_{b}} y(t) \frac{dx}{dt} \, dt\).
                    \item \textbf{Arc Length:} \(\displaystyle\int_{t_{a}}^{t_{b}} \mysqrt{\left(\dfrac{dx}{dt}\right)^{2} + \left(\dfrac{dy}{dt}\right)^{2}} \, dt\).
                    \item \textbf{Surface Area:} \(\displaystyle\int_{t_{a}}^{t_{b}} 2\pi y(t) \dmysqrt{\left(\frac{dx}{dt}\right)^{2} + \left(\frac{dy}{dt}\right)^{2}} \, dt\).
                \end{itemize}

            \end{minipage}
        };
        %------------ Continuous Random Variable and Distributions Header ---------------------
        \node[fancytitle, right=10pt] at (box.north west) {Chapter 1: Parametric Equations and Polar Coordinates};
    \end{tikzpicture}


    %------------ Summarizing Main Features of f(x) ---------------
    \begin{tikzpicture}
        \node [mybox] (box){%
            \begin{minipage}{0.46\textwidth}
                \begin{itemize}
                    \item \cyanit{Direction:} \(P = (x_{1},y_{1})\) and \(Q = (x_{2},y_{2})\): \(\mathbf{PQ} = \langle x_{2}-x_{1},y_{2}-y_{1}\rangle\).
                    \item \cyanit{Vector Sum:} \(\mathbf{u} + \mathbf{v} = \langle u_{1} + v_{1}, u_{2} + v_{2}\rangle\).
                    \item \cyanit{Magnitude:} \(\norm{u} = \mysqrt{u_{1}^{2} + u_{2}^{2}} = \mysqrt{u} \cdot u\).
                    \item \cyanit{Dot Product:} \(\mathbf{u} \cdot \mathbf{v} = u_{1}v_{1} + u_{2}v_{2}\).
                          \begin{itemize}
                              \item \cyanit{Angle:} \(\mathbf{u} \cdot \mathbf{v} = \norm{u}\norm{v}\cos(\theta)\), where \(0 \leq \theta \leq \pi\) is between \(\mathbf{u}\) \& \(\mathbf{v}\).
                              \item \cyanit{Self-Product:} \(\mathbf{u} \cdot \mathbf{u} = \norm{u}^{2}\).
                              \item \cyanit{Work:} \(W = \mathbf{F} \cdot \mb{PQ} = (\norm{F})\norm{PQ}\cos(\theta).\)
                          \end{itemize}
                    \item To \cyanit{Normalize} a vector, divide it by its magnitude \(\mb{v} = \brackett{x,y,z}\), then \(\mb{u} = \frac{1}{\norm{\mb{v}}}\mb{v} = \brackett{\frac{x}{\norm{v}},\frac{y}{\norm{v}}, \frac{z}{\norm{v}}}\). \(\therefore\mb{u} :=\) \cyanit{Unit Vector} in direction of \(\mb{v}\).
                    \item \cyanit{Projection:} \(\text{proj}_{\mathbf{b}}\mathbf{a} = \left(\frac{\mathbf{a} \cdot \mathbf{b}}{\norm{b}^{2}}\right)\mathbf{b}.\)
                    \item \cyanit{Cross product:} \(\mathbf{u} \times \mathbf{v} = \brackett{u_{2}v_{3} - u_{3}v_{2}, \ u_{3}v_{1} - u_{1}v_{3}, \ u_{1}v_{2} - u_{2}v_{1}}.\)
                          \begin{itemize}
                              \item \cyanit{Angle:} \(\norm{u \times v} = \norm{u}\norm{v}\sin(\theta)\), where \(0 \leq \theta \leq \pi\) is between \(\mathbf{u}\) \& \(\mathbf{v}\).
                              \item \cyanit{Torque:} \(\mb{\tau} = \mb{r} \times \mb{F} \qquad \text{or} \qquad \norm{\mb{\tau}} = \norm{r}\norm{F}\sin(\theta)\)
                          \end{itemize}

                \end{itemize}
                \subsubsection*{Parametric Equations Revisted}
                \begin{itemize}
                    \item \cyanit{Vector Equation:} \(\mb{r}(t) = \mb{r_{0} + t\mb{V}}\).
                    \item \cyanit{Parametric Equation:} \(x = x_{0} + at, \ y = y_{0} + bt, \ z = z_{0} + ct\).
                    \item \cyanit{Symmetric Equation:} \(\frac{x - x_{0}}{a} = \frac{y - y_{0}}{b} = \frac{z - z_{0}}{c}\).
                    \item The \cyanit{Line Segment} from \(P\) to \(Q\): \(\mb{r}(t) = (1 - t)\mb{p} + t\mb{q}\) (where \(\mb{p},\mb{q}\) are the vector forms of \(P,Q\) and \(0 \leq t \leq 1\)).
                    \item \cyanit{Direction Vector:} \(d = \left|\frac{\mb{PM} \times \mb{v}}{\norm{v}}\right|\).
                          \begin{itemize}
                              \item \cyanit{Equal}: Same direction vector, share a point.
                              \item \cyanit{Parallel}: Same direction vector, do not share a point.
                              \item \cyanit{Intersecting}: Different direction vectors, share a point.
                              \item \cyanit{Skew}: Different direction vectors, do not share a point.
                          \end{itemize}
                    \item If \((x_{0},y_{0},z_{0})\) is a point on a plane, the \cyanit{Scalar Equation} would be: \(\brackett{x - x_{0}, y - y_{0}, z - z_{0}} \cdot \brackett{a,b,c} = 0 \Longrightarrow a(x - x_{0}) + b(y - y_{0}) + c(z - z_{0}) = 0\).
                \end{itemize}
            \end{minipage}
        };
        %------------ Summarizing Main Features of f(x) ---------------------
        \node[fancytitle, right=10pt] at (box.north west) {Chapter 2: Vectors in Space};
    \end{tikzpicture}


    %------------ Sum and Average of Independent Random Variables ---------------
    \begin{tikzpicture}
        \node [mybox] (box){%
            \begin{minipage}{0.46\textwidth}
                If each of \(f_{1},f_{2},\ldots,f_{n} \colon \R \to \R\) is a function we can then define the \cyanit{vector-valued function} \(\mathbf{r} \colon \R \to \R^{n}\) by \(\mathbf{r}(t) = \langle f_{1}(t),f_{2}(t),\ldots,f_{n}(t) \rangle\)
                \begin{itemize}
                    \item When \(n = 2\), we might write \(\mb{r} = \brackett{f(t),g(t)} = f(t)\hat{\imath} + g(t)\hat{\jmath}\),
                    \item and when \(n = 3\), we might write \(\mb{r} = \brackett{f(t),g(t),h(t)} = f(t)\hat{\imath} + g(t)\hat{\jmath} + h(t)\hat{k}\).
                \end{itemize}
                \textbf{Note:} Deriving and integrating vector-valued functions follow the same rules as regular derivates.
                \begin{itemize}
                    \item \cyanit{Principle unit tangent vector} \(\mb{T}(t)\): \(\mb{T}(t) = \frac{\mb{r}'(t)}{||\dvecfuc{r}{t}||}\).
                          \begin{itemize}
                              \item This vector, of length 1, points in the tangent direction of the curve.
                          \end{itemize}
                    \item \cyanit{Unit Normal Vector} \(\mb{N}\): \(\vecfuc{N}{t} = \frac{\dvecfuc{T}{t}}{||\dvecfuc{T}{t}||}.\)
                          \begin{itemize}
                              \item This vector points in the direction the curve is turning.
                          \end{itemize}
                    \item \cyanit{Binormal Vector} \(\mb{B}\): \(\vecfuc{B}{t} = \vecfuc{T}{t} \times \vecfuc{N}{t}.\)
                \end{itemize}
                \subsubsection*{Arc Length Parameterization}
                We can define the \cyanit{arc length parameterization} of a curve \(C\) by:
                \begin{itemize}
                    \item Define the arc length \(s(t) = \int_{0}^{t} ||\dvecfuc{r}{\tau}||\,d\tau = \int_{0}^{t} \mysqrt{\left[f'(\tau)\right]^{2} + \left[g'(\tau)\right]^{2}}\,d\tau.\) (Where \(f(\tau), g(\tau)\) correspond to the \(x,y\) components of \(\mb{r}(t)\)).
                    \item Solving, if possible, the resulting expression for \(t\) as a function of \(s\).
                    \item Rewriting \(\mb{r}(t) = \mb{r}(t(s)) = \mb{r}\), so that the curve is written as a function of its length, from a given starting point.
                \end{itemize}
                \subsubsection*{Curvature (\& Oscillating Circle)}
                \textbf{Steps for Finding Oscillating Circle Equation}
                \begin{enumerate}
                    \item Get values for your curvature formulas.
                    \item Find the radius (\(R = \frac{1}{\kappa}\)).
                    \item Find the center by first finding the normal vector at \(x = \ldots\).  
                \end{enumerate}
                For all \(\mb{r} \colon \ \kappa = \frac{||\mb{T}'(t)||}{||\mb{r}'(t)||}\); for \(\R^{3}\): \(\kappa = \frac{||\mb{r}'(t) \times \mb{r}''(t)||}{||\mb{r}'(t)||^{3}}\); if \(y = f(x)\): \(\kappa = \frac{|y''(x)|}{[1 + (y'(x)^{2})]^{3/2}}\) \\

            \end{minipage}
        };
        %------------ Sum and Average of Independent Random Variables ---------------------
        \node[fancytitle, right=10pt] at (box.north west) {Chapter 3: Vector-Valued Functions};
    \end{tikzpicture}


    %------------ Maximum and Minimum of Independent Variables ---------------
    \begin{tikzpicture}
        \node [mybox] (box){%
            \begin{minipage}{0.46\textwidth}

                \underline{\textbf{Definitions:}}
                \begin{center} \textbf{Graphs} \\
                \end{center}
                \begin{tabular}{lp{10cm} l}
                    \textit{Graph:}
                     & \(G = (V,E)\), is a pair of sets \(V\), the \textit{vertex set}, and \textit{E} the \textit{edge set}, so that each element of \textit{E} has the form \(\{v_i,v_j\}, v_1, v_j \in V\). \\
                    \textit{Degree:}
                     & The number of edges which include \(v\). Granted that \(v\in V\).                                                                                                                       \\
                    \textit{Adjacent:}
                     & Vertices \(u, v\) belong to \(\{u,v\} \in E\).                                                                                                                                          \\
                    \textit{Path:}
                     & From vertex \(v_0\) to vertex \(v_n\) is a sequence \(v_0, v_1, v_2 \dots, v_n\), where each \(v_i \in V\) and \(\{v_i,v_{i+1}\} \in E\).                                               \\
                    \textit{Simple:}
                     & No edge occurs twice in a path.                                                                                                                                                         \\
                    \textit{Connected:}
                     & If each pair of vertices are adjoined by an edge.                                                                                                                                       \\
                \end{tabular}

                \begin{center} \textbf{Circuits} \\
                \end{center}

                \begin{tabular}{lp{10cm} l}
                    \textit{Circuit:}
                     & A path with the same starting and ending vertex.                                                \\
                    \textit{Complete:}
                     & On \(n\) vertices, \(K_n\), is the connected graph where each vertex is adjacent to each other. \\
                \end{tabular}

                \begin{center}\textbf{Bipartite Graphs} \\
                \end{center}

                \begin{tabular}{lp{10cm} l}

                    \textit{Bipartite:}
                     & The vertex set \(V = v_1 \cup v_2, v_1 \cap v_2 = \emptyset\), and no vertex in \(V_1\) is adjacent to any other in \(V_1\), and no vertex in \(V_2\) is adjacent to any other in \(V_2\).
                \end{tabular}

                \begin{center}\textbf{Trees} \\
                \end{center}

                \begin{tabular}{lp{10cm} l}
                    \textit{Tree:}
                     & A connected graph that has no circuit.                                                                                              \\
                    \textit{BiSTree:}
                     & For every node, all elements in the left subtree are less than the node's value, and all elements in the right subtree are greater. \\
                    \textbf{\textit{Lemma:}}
                     & If \(G\) is a tree, \(G\) has at least one vertex of degree 1.
                \end{tabular}
                \begin{proof}
                    For the sake of contradiction, suppose each vertex has degree \(\geq 2\). Pick a vertex, \(v_0\). Since, \(\deg(v_0) \geq 2\), it is adjacent to some \(v_j\). Because \(\deg(v_1) \geq 2\), it has an edge distinct from \(\{v_0,v_1\}\), follow it to \(v_2\). Then, \(v_2\) has edge distinct from \(\{v_1,v_2\}\), follow it to \(v-3\). If \(v_3=v_0 (v_1,v_2)\) we have a circuit. \(v_3\) has edge distinct from \(\{v_2,v_3\}\). Go to \(v_4\). Continue \dots. Either some \(v_j\) is visited again, or \(v_0,v_1,v_2,\dots v_n\). Therefore, it must be a circuit. \\

                    Hence, \(G\) must have a vertex with degree of at least 1 such that \(1\leq\).
                \end{proof}
                \begin{tabular}{lp{10cm} l}
                    \textbf{\textit{Theorem 1:}}
                     & A tree with \textit{n} vertices always has \(n - 1\) edges.
                \end{tabular}
                \begin{proof}
                    By the Lemma, there exists a vertex of degree 1. Remove it and its edge. We still have a tree. This new tree has vertex of degree 1. Remove it and its edge. Continue until you get to \(k_2\), then \(k_1\). We stop with 1 vertex, 0 edges, we have removed \(n-1\) vertices. Each edge was removed. Thus, we threw out \textit{all} \(n-1\) edges.
                \end{proof}

            \end{minipage}
        };
        %------------ Maximum and Minimum of Independent Variables ---------------------
        \node[fancytitle, right=10pt] at (box.north west) {2.6 Graph Theory};
    \end{tikzpicture}


    %------------ Some Continuous Distributions ---------------
    \begin{tikzpicture}
        \node [mybox] (box){%
            \begin{minipage}{0.46\textwidth}

                \begin{center}\textbf{Euler Circuits} \\
                \end{center}

                \begin{tabular}{lp{9cm} l}
                    \textit{Euler circuit:}
                     & A circuit graph which uses each edge only once.                     \\
                    \textit{Euler path:}
                     & A path which uses each edge -- start and end vertices are distinct.
                \end{tabular} \\

                \textbf{Note:} \(G\) has an Euler circuit if, and only if, it is connected and each vertex has an even degree. Intuitively, if each vertex has an even degree, then if you come into the vertex through the entrance (first edge), and you leave through the exit (second edge) you have used up both openings. \\

                \(G\) has an Euler path if, and only if, it is connected and has exactly two vertices of odd degree.

                \begin{center}\textbf{Isomorphisms} \\
                \end{center}
                Two graphs, \(G = (V,E)\) and \(H = (W,f)\) are \textit{isomorphic} if there is an \(f\colon V\rightarrow W\) which is one-to-one, onto, and \(\{v_i, v_j\} \in E \iff \{f(v_i), f(v_j)\} \in F\).

                \begin{center}\textbf{Vertex Colorings} \\
                \end{center}
                If \(G\) contains a triangle (i.e., if it has a copy of \(k_3\), we need at least 3 colors. If \(G\) contains a copy of \(K_n\), we need at least \(n\). \\

                If a graph has no overlapping paths, the graph requires no more than 4 colors.

                \begin{center}\textbf{Hamilton Graphs} \\
                \end{center}
                A graph has a \textit{Hamilton Circuit} if there is a circuit that uses each vertex once. \\

                \textbf{Note:} This is different from Euler, as Euler uses edges. This is specifically for vertices. If it has a vertex of degree 1, it cannot have a Hamilton circuit.





            \end{minipage}
        };
        %------------ Some Continuous Distributions Header ---------------------
        \node[fancytitle, right=10pt] at (box.north west) {2.6 Graph Theory (cont.)};
    \end{tikzpicture}


    %------------ Normal Distribution ---------------
    \begin{tikzpicture}
        \node [mybox] (box){%
            \begin{minipage}{0.46\textwidth}

                \begin{center}\textbf{Sets}
                \end{center}

                \begin{enumerate}
                    \item Suppose that the universal set \(U = \{1,2,3,4,5,6,7,8,9,10\}\) and the particular sets \(A = \{2,3,5\}\) and \(B = \{1,3,5,7,9\}\). Find each of the following:
                          \begin{enumerate}
                              \item \(A \cup B = \{1,2,3,5,7,9\}\)
                              \item \(A \cap B = \{3,5\}\)
                              \item \(\{x\in U \colon x^2 < 10 \land x \in A\} = \{2,3\}\)
                              \item \(B' \cap A = \{2\}\)
                              \item \(\{x \in A \colon x \geq 7\} = \emptyset\)
                              \item \(\mathcal{P}(A) = \{\emptyset, \{2\}, \{3\}, \{5\}, \{2,3\}, \{2,5\}, \{3,5\}, \{2,3,5\}\}\)
                          \end{enumerate}
                \end{enumerate}

            \end{minipage}
        };

        %------------ Normal Distribution Header ---------------------
        \node[fancytitle, right=10pt] at (box.north west) {Estimating Big \(\Theta\)};
    \end{tikzpicture}

\end{multicols*}



\end{document}
% ------------------------------------------- Additional content that can be added -------------------------------------------

\begin{center}
    \textbf{The Cycloid}
\end{center}


We can apply each of the above to the cycloid:
\begin{itemize}
    \item \cyanit{Derivative:} \(\dydx = \frac{dy}{dx} = \frac{\sin t}{1 - \cos t}\). Note that the slope is then independent of the radius of the wheel and
          that the slope is undefined at each of \(t = \dots, -4\pi, -2\pi, 0, 2\pi, 4\pi, \dots\).
    \item \cyanit{Cartesian Equation:} With radius of \(3\) and when \(t = \frac{\pi}{3}\), the point is found by solving for \(x(\frac{\pi}{3})\) and \(y(\frac{\pi}{3})\):
          \begin{align*}
              x\left(\frac{\pi}{3}\right) & = 3\left(\frac{\pi}{3} - \sin\left(\frac{\pi}{3}\right)\right) = \pi - \frac{3\mysqrt{3}}{2} \\
              y\left(\frac{\pi}{3}\right) & = 3\left(1 - \cos\left(\frac{\pi}{3}\right)\right) = \frac{3}{2}                             \\
              (x,y)                       & = \left(\pi - \frac{3\mysqrt{3}}{2},\frac{3}{2}\right)
          \end{align*} Plugging in our \(t\) value into our derivative, we get a slope of
          \[
              \frac{\sin(\pi/3)}{1 - \cos(\pi/3)} = \frac{\mysqrt{3}/2}{1/2} = \mysqrt{3}.
          \]
          Now, we can write the equation of the tangent line as
          \[
              y = \mysqrt{3}\left(x - \pi + \frac{3\mysqrt{3}}{2}\right) + \frac{3}{2}.
          \]
          \newpage
    \item \cyanit{Concavity:} \(\frac{d^{2}y}{dx^{2}} = \frac{d}{dt}\left(\frac{dy}{dx}\right) = \frac{d}{dt}\left(\frac{\sin t}{1 - \cos t}\right)\).

          \begin{align*}
              \frac{d^{2}y}{dx^{2}} & = \frac{d/dt(dy/dx)}{dx/dt}                                                        \\
                                    & = \frac{\frac{d}{dt}\left(\frac{\sin t}{1 - \cos t}\right)}{a - a \cos t}          \\
                                    & = \frac{\frac{\cos t(1 - \cos t) - \sin t \sin t}{(1 - \cos t)^{2}}}{a - a \cos t} \\
                                    & = \frac{\cos t - \cos^{2} - \sin^{2}(t)}{(1 - \cos t)^{2}a(1 - \cos t)}            \\
                                    & = \frac{\cos t - 1}{a(1 - \cos t)^{2}}                                             \\
                                    & = -\frac{1}{a(1 - \cos t)^{2}}                                                     \\
                                    & = -\frac{a}{a^{2}(1 - \cos t)^{2}}                                                 \\
                                    & = -\frac{a}{y^{2}}
          \end{align*}

          After some work, we find that \(\frac{d^{2}y}{dx^{2}} = -\frac{a}{y^{2}}\), which shows that the cycloid is always concave down.
    \item \cyanit{Area:} The area of one period of the cycloid \(A = 3\pi a^{2}\), after some work:
          \begin{align*}
              A & = \int_{t_{a}}^{t_{b}} y(t)x'(t) dt                                                                                                 \\
                & = \int_{0}^{2\pi} (a - a \cos t)(a - a\cos t) dt                                                                                    \\
                & = a^{2} \int_{0}^{2\pi} (1 - 2\cos t + \cos^{2} t) dt                                                                               \\
                & = a^{2} \left(t + \frac{t}{2} + \frac{1}{4}\sin(2t)\right)\bigg|_{0}^{2\pi}                                                         \\
                & = a^{2} \left[\left(2\pi + \frac{2\pi}{2} + \frac{1}{4}\sin(2\pi)\right) - \left(0 + \frac{0}{2} + \frac{1}{4}\sin(0)\right)\right] \\
                & = a^{2} [2\pi + \pi]                                                                                                                \\
                & = 3\pi a^{2}.
          \end{align*}
          \newpage
    \item \cyanit{Arc Length:} The arc length of one period of the cycloid is \(s = 8a\), again after some work:

          \begin{align*}
              s & = \int_{t_{1}}^{t_{2}} \dmysqrt{\left(\frac{dx}{dt}\right)^{2} + \left(\frac{dy}{dt}\right)^{2}} \ dt \\
                & = \int_{0}^{2\pi} \dmysqrt{(a- a \cos t)^{2} + (a \sin t)^{2}} \ dt                                   \\
                & = a \int_{0}^{2\pi} \dmysqrt{1 - 2\cos t + \cos^{2} t + \sin^{2} t} \ dt                              \\
                & = a \int_{0}^{2\pi} \dmysqrt{2 - 2\cos t} \ dt                                                        \\
                & = \dmysqrt{2}a \int_{0}^{2\pi} \dmysqrt{1 - \cos t} \ dt                                              \\
                & = \dmysqrt{2}a \int_{0}^{2\pi} \dmysqrt{2\sin^{2}\left(\frac{t}{2}\right)} \ dt                       \\
                & = \dmysqrt{2}a \cdot \dmysqrt{2} \int_{0}^{2\pi} \sin\left(\frac{t}{2}\right) \ dt                    \\
                & = 2a\left(-2 \cos\left(\frac{t}{2}\right)\right)\bigg|_{0}^{2\pi}                                     \\
                & = 8a.
          \end{align*}

    \item \cyanit{Surface Area:} The surface area of the solid obtained by rotating one period of the cycloid around the \(x\)-axis is \(S = \frac{64\pi a^{2}}{3}\), after a lot of tedious work.

          \begin{align*}
              S & = \int_{0}^{2\pi} 2\pi y(t) \dmysqrt{\left(\frac{dx}{dt}\right)^{2} + \left(\frac{dy}{dt}\right)^{2}} \ dt \\
          \end{align*}

\end{itemize}