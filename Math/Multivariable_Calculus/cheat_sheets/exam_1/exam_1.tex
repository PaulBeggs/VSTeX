\documentclass{article}
\usepackage[landscape]{geometry}
\usepackage{url}
\usepackage{multicol}
\usepackage{amsmath}
\usepackage{esint}
\usepackage{amsthm}
\usepackage{amsfonts}
\usepackage{booktabs}
\usepackage{tikz}
\usetikzlibrary{decorations.pathmorphing}
\usetikzlibrary{graphs}
\usetikzlibrary{shapes}
\usetikzlibrary{backgrounds}
\usetikzlibrary{calc}
\usetikzlibrary{patterns}
\usetikzlibrary{positioning, fit, arrows.meta}
\usepackage{amsmath,amssymb}
\usepackage{subcaption}
\usepackage[english]{babel}
\usepackage{enumitem}


\usepackage{colortbl}
\usepackage{xcolor}
\usepackage{mathtools}
\usepackage{amsmath,amssymb}
\usepackage{enumitem}
\makeatletter

\newcommand*\bigcdot{\mathpalette\bigcdot@{.5}}
\newcommand*\bigcdot@[2]{\mathbin{\vcenter{\hbox{\scalebox{#2}{\(\m@th#1\bullet\)}}}}}
\makeatother

\newenvironment{solution}%
    {\textit{Solution.}}

    \newcommand{\sol}[1]{
        \begin{solution}
            #1
        \end{solution}
    }

\title{Discrete Exam 2 Note Sheets}
\usepackage[utf8]{inputenc}

\advance\topmargin-.8in
\advance\textheight
3in
\advance\textwidth3in
\advance\oddsidemargin-1.40in
\advance\evensidemargin-1.45in
\parindent0pt
\parskip1pt
\newcommand{\hr}{\centerline{\rule{3.5in}{1pt}}}

% Natural Numbers 
\newcommand{\N}{\ensuremath{\mathbb{N}}}

% Whole Numbers
\newcommand{\W}{\ensuremath{\mathbb{W}}}

% Integers
\newcommand{\Z}{\ensuremath{\mathbb{Z}}}

% Rational Numbers
\newcommand{\Q}{\ensuremath{\mathbb{Q}}}

% Real Numbers
\newcommand{\R}{\ensuremath{\mathbb{R}}}

% Complex Numbers
\newcommand{\C}{\ensuremath{\mathbb{C}}}

\newcommand{\cyanit}[1]{\textbf{\textit{#1}}}

\newcommand{\brackett}[1]{\left\langle #1 \right\rangle}

\newcommand{\norm}[1]{\left\lVert \mathbf{#1}\right\rVert}

\newcommand{\dydx}{\frac{dy}{dx}}
\newcommand{\dxdy}{\frac{dx}{dy}}
\newcommand{\dydt}{\frac{dy}{dt}}
\newcommand{\dxdt}{\frac{dx}{dt}}
\newcommand{\dzdt}{\frac{dz}{dt}}


\newcommand{\cost}{\cos(t)}
\newcommand{\sint}{\sin(t)}
\newcommand{\tant}{\tan(t)}

% 
\newcommand{\barNotationT}[1]{\bigg|_{t = #1}}

\newcommand{\mb}[1]{\mathbf{#1}}

\newcommand{\imb}{\mb{i}}
\newcommand{\jmb}{\mb{j}}
\newcommand{\kmb}{\mb{k}}
\newcommand{\rmb}{\mb{r}}
\newcommand{\umb}{\mb{u}}

\newcommand{\vecfuc}[2]{\mb{#1}(#2)}
\newcommand{\dvecfuc}[2]{\mb{#1}'(#2)}
\newcommand{\normdvecfuc}[2]{||\mb{#1}'(#2)||}

\DeclareMathOperator{\arccot}{arccot}
\DeclareMathOperator{\arcsec}{arcsec}
\DeclareMathOperator{\arccsc}{arccsc}

\usepackage{titlesec}

\newcommand{\mysqrt}[1]{%
  \mathpalette\foo{#1}%
}
\newcommand{\dmysqrt}[1]{%
  \mathpalette\foodisplay{#1}%
}

% !TeX spellcheck = off
\newcommand{\foo}[2]{%
  % #1: math style, #2: content
  \sbox0{$#1\sqrt{#2}$}% Measure the size of the standard sqrt in the current style
  \begin{tikzpicture}[baseline=(sqrt.base)]
    \node[inner sep=0, outer sep=0] (sqrt) {$#1\sqrt{#2}$}; % Use the current math style
    \draw([yshift=-0.045em]sqrt.north east) -- ++(0,-0.5ex); % Draw the tick
  \end{tikzpicture}%
}
% !TeX spellcheck = off
\newcommand{\foodisplay}[2]{%
  % #1: math style, #2: content
  \sbox0{$#1\sqrt{#2}$}% Measure the size of the standard sqrt in the current style
  \begin{tikzpicture}[baseline=(sqrt.base)]
    \node[inner sep=0, outer sep=0] (sqrt) {$\displaystyle\sqrt{#2}$}; % Force displaystyle
    \draw[line width=0.4pt] ([yshift=-0.044em]sqrt.north east) -- ++(0,-0.5ex); % Draw the tick
  \end{tikzpicture}%
}

\setlist[enumerate,1]{left=0pt}
\setlist[enumerate,2]{left=-10pt}


\begin{document}





















\begin{center}{\huge{\textbf{Multivariable Calculus Exam 1}}}\\
\end{center}
\begin{multicols*}{2}

    \tikzstyle{mybox} = [draw=black, fill=white, very thick,
    rectangle, rounded corners, inner sep=5pt, inner ysep=10pt]
    \tikzstyle{innerbox} = [draw=black, fill=gray!20, thick,
    rectangle, rounded corners, inner sep=5pt, inner ysep=10pt]
    \tikzstyle{fancytitle} =[fill=black, text=white, font=\bfseries]

    %------------ Measures of Center ---------------
    \begin{tikzpicture}
        \node [mybox] (box){%
            \begin{minipage}{0.46\textwidth}

                \begin{small}
                    \begin{multicols}{2}
                        \subsubsection*{Basic Derivatives}
                        \begin{align*}
                             & \frac{d}{dx} e^{f(x)} =            & f'(x)e^{f(x)}                    \\
                             & \frac{d}{dx} \sin f(x) =           & \cos f(x) \cdot f'(x)            \\
                             & \frac{d}{dx} \cos f(x) =           & -\sin f(x) \cdot f'(x)           \\
                             & \frac{d}{dx} \tan f(x) =           & \sec^2 f(x) \cdot f'(x)          \\
                             & \frac{d}{dx} \cot f(x) =           & -\csc^2 f(x) \cdot f'(x)         \\
                             & \frac{d}{dx} \sec f(x) =           & \sec f(x) \tan f(x) \cdot f'(x)  \\
                             & \frac{d}{dx} \csc f(x) =           & -\csc f(x) \cot f(x) \cdot f'(x) \\
                             & \frac{d}{dx} \ln f(x) =            & \frac{f'(x)}{f(x)}               \\
                             & \frac{d}{dx} \log_a f(x) =         & \frac{f'(x)}{f(x) \ln a}         \\
                             & \frac{d}{dx} \left(f(x)\right)^n = & n \left(f(x)\right)^{n-1} f'(x)  \\
                             & \frac{d}{dx} \dmysqrt{f(x)} =      & \frac{f'(x)}{2\dmysqrt{f(x)}}    \\
                             & \frac{d}{dx} a^x =                 & a^x \ln a                        \\
                             & \frac{d}{dx} b^{g(x)} =            & b^{g(x)} \ln b \cdot g'(x)       \\
                        \end{align*}
                        \subsubsection*{Product and Quotient}

                        \begin{align*}
                             & \frac{d}{dx} [u \cdot v] =                & u' \cdot v + u \cdot v'             \\
                             & \frac{d}{dx} \left( \frac{u}{v} \right) = & \frac{u' \cdot v - u \cdot v'}{v^2} \\
                        \end{align*}

                        \subsubsection*{Inverse Trigonometric}

                        \begin{align*}
                             & \frac{d}{dx} \arcsin f(x) = & \frac{f'(x)}{\dmysqrt{1 - \left(f(x)\right)^2}}        \\
                             & \frac{d}{dx} \arccos f(x) = & -\frac{f'(x)}{\dmysqrt{1 - \left(f(x)\right)^2}}       \\
                             & \frac{d}{dx} \arctan f(x) = & \frac{f'(x)}{1 + \left(f(x)\right)^2}                  \\
                             & \frac{d}{dx} \arccot f(x) = & -\frac{f'(x)}{1 + \left(f(x)\right)^2}                 \\
                             & \frac{d}{dx} \arcsec f(x) = & \frac{f'(x)}{|f(x)|\dmysqrt{\left(f(x)\right)^2 - 1}}  \\
                             & \frac{d}{dx} \arccsc f(x) = & -\frac{f'(x)}{|f(x)|\dmysqrt{\left(f(x)\right)^2 - 1}} \\
                        \end{align*}

                        \subsubsection*{Chain Rule}

                        \begin{align*}
                             & \frac{d}{dx} f(g(x)) = & f'(g(x)) \cdot g'(x) \\
                        \end{align*}

                        \subsubsection*{Higher-Order Derivatives}

                        \begin{align*}
                             & \frac{d^2}{dx^2} e^x =    & e^x     \\
                             & \frac{d^3}{dx^3} \sin x = & -\cos x \\
                             & \frac{d^4}{dx^4} \cos x = & \cos x  \\
                        \end{align*}
                    \end{multicols}
                \end{small}

            \end{minipage}
        };
        %------------ Measures of Center Header ---------------------
        \node[fancytitle, right=10pt] at (box.north west) {Derivation};
    \end{tikzpicture}
    \begin{tabular}{|c||c|c|c|c|c|c|}
        \hline
        \(n\) & \(\displaystyle \int_{0}^{\pi/2} \sin^{n} x\, dx\) & \(\displaystyle \int_{0}^{\pi/2} \cos^{n} x\, dx\) & \(\displaystyle \int_{0}^{\pi} \sin^{n} x\, dx\) & \(\displaystyle \int_{0}^{\pi} \cos^{n} x\, dx\) & \(\displaystyle \int_{0}^{2\pi} \sin^{n}x \, dx\) & \(\displaystyle \int_{0}^{2\pi} \cos^{n}x \, dx\) \\ \hline
        1     & 1                                                  & 1                                                  & 2                                                & 0                                                & 0                                                 & 0                                                 \\
        2     & \(\pi/4\)                                          & \(\pi/4\)                                          & \(\pi/2\)                                        & \(\pi/2\)                                        & \(\pi\)                                           & \(\pi\)                                           \\
        3     & 2/3                                                & 2/3                                                & 4/3                                              & 0                                                & 0                                                 & 0                                                 \\
        4     & \(3\pi/16\)                                        & \(3\pi/16\)                                        & \(3\pi/8\)                                       & \(3\pi/8\)                                       & \(3\pi/4\)                                        & \(3\pi/4\)                                        \\
        5     & 8/15                                               & 8/15                                               & 16/15                                            & 0                                                & 0                                                 & 0                                                 \\
        6     & \(5\pi/32\)                                        & \(5\pi/32\)                                        & \(5\pi/16\)                                      & \(5\pi/16\)                                      & \(5\pi/8\)                                        & \(5\pi/8\)                                        \\
        \hline
    \end{tabular}



    %------------ Discrete Random Variable and Distributions ---------------
    \begin{tikzpicture}
        \node [mybox] (box){%
            \begin{minipage}{0.46\textwidth}

                \begin{small}
                    \begin{multicols}{2}
                        \subsubsection*{Trigonometric Integrals}

                        \begin{align*}
                             & \int \sin x \, dx =        & -\cos x + C                        \\
                             & \int \cos x \, dx =        & \sin x + C                         \\
                             & \int\sin^{2}x =            & \frac{1}{2}(x - \sin x \cos x) + C \\
                             & \int\cos^{2}x =            & \frac{1}{2}(x + \sin x \cos x) + C \\
                             & \int \tan x \, dx =        & -\ln|\cos x| + C                   \\
                             & \int \cot x \, dx =        & \ln|\sin x| + C                    \\
                             & \int \sec x \, dx =        & \ln|\sec x + \tan x| + C           \\
                             & \int \csc x \, dx =        & -\ln|\csc x + \cot x| + C          \\
                             & \int \sec^2 x \, dx =      & \tan x + C                         \\
                             & \int \csc^2 x \, dx =      & -\cot x + C                        \\
                             & \int \sec x \tan x \, dx = & \sec x + C                         \\
                             & \int \csc x \cot x \, dx = & -\csc x + C
                        \end{align*}

                        \subsubsection*{Reduction Formulas for Sine and Cosine}

                        \begin{align*}
                             & \int \sin^{n}x \, dx = -\frac{1}{n}\sin^{n-1}x\cos x + \frac{n-1}{n}\int \sin^{n-2}x \, dx \\
                             & \int \cos^{n}x \, dx = \frac{1}{n}\cos^{n-1}x\sin x + \frac{n-1}{n}\int \cos^{n-2}x \, dx
                        \end{align*}

                        \subsubsection*{Inverse Trigonometric Integrals}

                        \begin{align*}
                             & \int \frac{1}{\mysqrt{a^{2} - x^{2}}} \, dx =  & \arcsin\left(\frac{x}{a}\right) + C            \\
                             & \int \frac{1}{a^{2} + x^{2}} \, dx =           & \frac{1}{a}\arctan\left(\frac{x}{a}\right) + C \\
                             & \int \frac{1}{x\mysqrt{x^{2} - a^{2}}} \, dx = & \frac{1}{a}\arcsec\left(\frac{x}{a}\right) + C
                        \end{align*}

                        \subsubsection*{Regular Integrals and \(e\)}

                        \begin{align*}
                             & \int x^{n} \, dx =         & \frac{1}{n+1}x^{n+1} + C \\
                             & \int \frac{1}{x} \, dx =   & \ln|x| + C               \\
                             & \int e^{x} \, dx =         & e^{x} + C                \\
                             & \int e^{ax} \, dx =        & \frac{1}{a}e^{ax} + C    \\
                             & \int e^{f(x)}f'(x) \, dx = & e^{f(x)} + C             \\
                        \end{align*}

                        \subsubsection*{Exponetials}

                        \begin{align*}
                             & \int_{0}^{{b}} e^{x} \, dx = e^{b} - 1     \\
                             & \int_{0}^{{b}} e^{-x} \, dx = 1 - e^{-b}   \\
                             & \int_{0}^{{\infty}} e^{-x} \, dx = 1       \\
                             & \int_{0}^{{\infty}} x^{n}e^{-x} \, dx = n!
                        \end{align*}

                    \end{multicols}
                \end{small}

            \end{minipage}
        };
        %------------ Discrete Random Variable and Distributions Header ---------------------
        \node[fancytitle, right=10pt] at (box.north west) {Integration};
    \end{tikzpicture}
    \vspace*{-0.25cm}
    \begin{center}
        \begin{tabular}{cccc}
            \hline
            Radians    & \(\sin(\theta)\) & \(\cos(\theta)\) & \(\tan(\theta)\) \\  \hline
            \(0\)      & \(0\)            & \(1\)            & \(0\)            \\  \hline
            \(\pi/6\)  & \(1/2\)          & \(\mysqrt{3}/2\) & \(\mysqrt{3}/3\) \\  \hline
            \(\pi/4\)  & \(\mysqrt{2}/2\) & \(\mysqrt{2}/2\) & \(1\)            \\  \hline
            \(\pi/3\)  & \(\mysqrt{3}/2\) & \(1/2\)          & \(\mysqrt{3}\)   \\ \hline
            \(\pi/2\)  & \(1\)            & \(0\)            & \(-\)            \\  \hline
            \(\pi\)    & \(0\)            & \(-1\)           & \(0\)            \\ \hline
            \(3\pi/2\) & \(-1\)           & \(0\)            & \(-\)            \\
            \hline
        \end{tabular}
    \end{center}

    \newpage

    %------------ Sets and Probability ---------------
    \begin{tikzpicture}
        \node [mybox] (box){%
            \begin{minipage}{0.46\textwidth}

                \begin{small}
                    \begin{multicols}{2}
                        \subsubsection*{Pythagorean}
                        \begin{align*}
                             & \sin^{2}\theta + \cos^{2}\theta =  1              \\
                             & \tan^{2}\theta + 1 =               \sec^{2}\theta \\
                             & 1 + \cot^{2}\theta =               \csc^{2}\theta
                        \end{align*}
                        \subsubsection*{Half Angle}
                        \begin{align*}
                             & \sin^{2}\left(\frac{x}{2}\right) =  \frac{1 - \cos x}{2}          \\
                             & \cos^{2}\left(\frac{x}{2}\right) =  \frac{1 + \cos x}{2}          \\
                             & \tan^{2}\left(\frac{x}{2}\right) =  \frac{1 - \cos x}{1 + \cos x}
                        \end{align*}
                        \subsubsection*{Double Angle}
                        \begin{align*}
                             & \sin 2x =  2\sin x \cos x                \\
                             & \cos 2x =  \cos^{2}x - \sin^{2}x         \\
                             & \cos 2x = 2\cos^{2}x - 1                 \\
                             & \cos 2x =  1 - 2\sin^{2}x                \\
                             & \tan 2x =  \frac{2\tan x}{1 - \tan^{2}x}
                        \end{align*}
                        \subsubsection*{Product to Sum}
                        \begin{align*}
                             & \sin x \sin y =  \frac{1}{2}\bigl[\cos(x-y) - \cos(x+y)\bigr] \\
                             & \cos x \cos y =  \frac{1}{2}\bigl[\cos(x-y) + \cos(x+y)\bigr] \\
                             & \sin x \cos y =  \frac{1}{2}\bigl[\sin(x+y) + \sin(x-y)\bigr] \\
                             & \cos x \sin y =  \frac{1}{2}\bigl[\sin(x+y) - \sin(x-y)\bigr]
                        \end{align*}
                        \subsubsection*{Sum to Product}
                        \begin{align*}
                             & \sin x + \sin y =  2\sin\left(\frac{x+y}{2}\right)\cos\left(\frac{x-y}{2}\right)  \\
                             & \sin x - \sin y =  2\cos\left(\frac{x+y}{2}\right)\sin\left(\frac{x-y}{2}\right)  \\
                             & \cos x + \cos y =  2\cos\left(\frac{x+y}{2}\right)\cos\left(\frac{x-y}{2}\right)  \\
                             & \cos x - \cos y =  -2\sin\left(\frac{x+y}{2}\right)\sin\left(\frac{x-y}{2}\right)
                        \end{align*}


                    \end{multicols}
                \end{small}
            \end{minipage}
        };
        %------------ Sets and Probability Header ---------------------
        \node[fancytitle, right=10pt] at (box.north west) {Trigonometric Identities};
    \end{tikzpicture}


    %------------ Continuous Random Variable and Distributions ---------------
    \begin{tikzpicture}
        \node [mybox] (box){%
            \begin{minipage}{0.46\textwidth}

                \begin{itemize}
                    \item \textbf{Slope:} \(\displaystyle \dydx\barNotationT{t_{0}} = \frac{dy/dt}{dx/dt}\barNotationT{t_{0}}.\)

                          \noindent The \cyanit{tangent line} at \(t_{0}\) is given by
                          \[
                              y = \left(\dydx\barNotationT{t_{0}}\right)\bigl(x - x(t_{0})\bigr) + y(t_{0}).
                          \]
                    \item \textbf{Concavity:} \(\displaystyle\dfrac{d^{2}y}{dx^{2}}\barNotationT{t_{0}} = \frac{\frac{d}{dt}(dy/dx)}{dx/dt}\barNotationT{t_{0}}\)
                    \item \textbf{Area Under a Curve:} \(\displaystyle\int_{t_{a}}^{t_{b}} y(t) \frac{dx}{dt} \, dt\).
                    \item \textbf{Arc Length:} \(\displaystyle\int_{t_{a}}^{t_{b}} \mysqrt{\left(\dfrac{dx}{dt}\right)^{2} + \left(\dfrac{dy}{dt}\right)^{2}} \, dt\).
                    \item \textbf{Surface Area:} \(\displaystyle\int_{t_{a}}^{t_{b}} 2\pi y(t) \dmysqrt{\left(\frac{dx}{dt}\right)^{2} + \left(\frac{dy}{dt}\right)^{2}} \, dt\).
                \end{itemize}

            \end{minipage}
        };
        %------------ Continuous Random Variable and Distributions Header ---------------------
        \node[fancytitle, right=10pt] at (box.north west) {Chapter 1: Parametric Equations and Polar Coordinates};
    \end{tikzpicture}


    %------------ Summarizing Main Features of f(x) ---------------
    \begin{tikzpicture}
        \node [mybox] (box){%
            \begin{minipage}{0.46\textwidth}
                \begin{itemize}
                    \item \cyanit{Direction:} \(P = (x_{1},y_{1})\) and \(Q = (x_{2},y_{2})\): \(\mathbf{PQ} = \langle x_{2}-x_{1},y_{2}-y_{1}\rangle\).
                    \item \cyanit{Vector Sum:} \(\mathbf{u} + \mathbf{v} = \langle u_{1} + v_{1}, u_{2} + v_{2}\rangle\).
                    \item \cyanit{Magnitude:} \(\norm{u} = \mysqrt{u_{1}^{2} + u_{2}^{2}} = \mysqrt{u} \cdot u\).
                    \item \cyanit{Dot Product:} \(\mathbf{u} \cdot \mathbf{v} = u_{1}v_{1} + u_{2}v_{2}\).
                          \begin{itemize}
                              \item \cyanit{Angle:} \(\mathbf{u} \cdot \mathbf{v} = \norm{u}\norm{v}\cos(\theta)\), where \(0 \leq \theta \leq \pi\) is between \(\mathbf{u}\) \& \(\mathbf{v}\).
                              \item \cyanit{Self-Product:} \(\mathbf{u} \cdot \mathbf{u} = \norm{u}^{2}\).
                              \item \cyanit{Work:} \(W = \mathbf{F} \cdot \mb{PQ} = (\norm{F})\norm{PQ}\cos(\theta).\)
                          \end{itemize}
                    \item To \cyanit{Normalize} a vector, divide it by its magnitude \(\mb{v} = \brackett{x,y,z}\), then \(\mb{u} = \frac{1}{\norm{\mb{v}}}\mb{v} = \brackett{\frac{x}{\norm{v}},\frac{y}{\norm{v}}, \frac{z}{\norm{v}}}\). \(\therefore\mb{u} :=\) \cyanit{Unit Vector} in direction of \(\mb{v}\).
                    \item \cyanit{Projection:} \(\text{proj}_{\mathbf{b}}\mathbf{a} = \left(\frac{\mathbf{a} \cdot \mathbf{b}}{\norm{b}^{2}}\right)\mathbf{b}.\)
                    \item \cyanit{Cross product:} \(\mathbf{u} \times \mathbf{v} = \brackett{u_{2}v_{3} - u_{3}v_{2}, \ u_{3}v_{1} - u_{1}v_{3}, \ u_{1}v_{2} - u_{2}v_{1}}.\)
                          \begin{itemize}
                              \item \cyanit{Angle:} \(\norm{u \times v} = \norm{u}\norm{v}\sin(\theta)\), where \(0 \leq \theta \leq \pi\) is between \(\mathbf{u}\) \& \(\mathbf{v}\).
                              \item \cyanit{Torque:} \(\mb{\tau} = \mb{r} \times \mb{F} \qquad \text{or} \qquad \norm{\mb{\tau}} = \norm{r}\norm{F}\sin(\theta)\)
                          \end{itemize}

                \end{itemize}
                \subsubsection*{Parametric Equations Revisted}
                \begin{itemize}
                    \item To \cyanit{Parameterize} an equation such as \(y = x^{3} - 4x + 1\) we can let \(x = t\) and \(y = t^{3} - 4t + 1\). This allows us to write the equation as \(\mb{r}(t) = \brackett{t,t^{3} - 4t + 1}\).
                    \item \cyanit{Vector Equation:} \(\mathbf{r}(t) = \mb{r}_{0} + t \mb{v}\).
                    \item \cyanit{Parametric Equation:} \(x = x_{0} + at, \ y = y_{0} + bt, \ z = z_{0} + ct\).
                    \item \cyanit{Symmetric Equation:} \(\frac{x - x_{0}}{a} = \frac{y - y_{0}}{b} = \frac{z - z_{0}}{c}\).
                    \item The \cyanit{Line Segment} from \(P\) to \(Q\): \(\mb{r}(t) = (1 - t)\mb{p} + t\mb{q}\) (where \(\mb{p},\mb{q}\) are the vector forms of \(P,Q\) and \(0 \leq t \leq 1\)).
                    \item \cyanit{Shortest Distance:} \(d = \frac{||\mb{PM} \times \mb{v}||}{\norm{v}}\).
                          \begin{itemize}
                              \item \cyanit{Equal}: Same direction vector, share a point.
                              \item \cyanit{Parallel}: Same direction vector, do not share a point.
                              \item \cyanit{Intersecting}: Different direction vectors, share a point.
                              \item \cyanit{Skew}: Different direction vectors, do not share a point.
                          \end{itemize}
                    \item If \((x_{0},y_{0},z_{0})\) is a point on a plane, the \cyanit{Scalar Equation} would be: \(\brackett{x - x_{0}, y - y_{0}, z - z_{0}} \cdot \brackett{a,b,c} = 0 \Longrightarrow a(x - x_{0}) + b(y - y_{0}) + c(z - z_{0}) = 0\).
                \end{itemize}
            \end{minipage}
        };
        %------------ Summarizing Main Features of f(x) ---------------------
        \node[fancytitle, right=10pt] at (box.north west) {Chapter 2: Vectors in Space};
    \end{tikzpicture}


    %------------ Sum and Average of Independent Random Variables ---------------
    \begin{tikzpicture}
        \node [mybox] (box){%
            \begin{minipage}{0.46\textwidth}
                If each of \(f_{1},f_{2},\ldots,f_{n} \colon \R \to \R\) is a function we can then define the \cyanit{vector-valued function} \(\mathbf{r} \colon \R \to \R^{n}\) by \(\mathbf{r}(t) = \langle f_{1}(t),f_{2}(t),\ldots,f_{n}(t) \rangle\)
                \begin{itemize}
                    \item When \(n = 2\), we might write \(\mb{r} = \brackett{f(t),g(t)} = f(t)\hat{\imath} + g(t)\hat{\jmath}\),
                    \item and when \(n = 3\), we might write \(\mb{r} = \brackett{f(t),g(t),h(t)} = f(t)\hat{\imath} + g(t)\hat{\jmath} + h(t)\hat{k}\).
                \end{itemize}
                \textbf{Note:} Deriving and integrating vector-valued functions follow the same rules as regular derivates.
                \begin{itemize}
                    \item \cyanit{Principle unit tangent vector} \(\mb{T}(t)\): \(\mb{T}(t) = \frac{\mb{r}'(t)}{||\dvecfuc{r}{t}||}\).
                          \begin{itemize}
                              \item This vector, of length 1, points in the tangent direction of the curve.
                          \end{itemize}
                    \item \cyanit{Unit Normal Vector} \(\mb{N}\): \(\vecfuc{N}{t} = \frac{\dvecfuc{T}{t}}{||\dvecfuc{T}{t}||}.\)
                          \begin{itemize}
                              \item This vector points in the direction the curve is turning.
                          \end{itemize}
                    \item \cyanit{Binormal Vector} \(\mb{B}\): \(\vecfuc{B}{t} = \vecfuc{T}{t} \times \vecfuc{N}{t}.\)
                \end{itemize}
                \vspace*{-0.5cm}
                \subsubsection*{Arc Length Parameterization}
                We can define the \cyanit{arc length parameterization} of a curve \(C\) by:
                \begin{itemize}
                    \item Define the arc length \(s(t) = \int_{0}^{t} \| \dvecfuc{r}{\tau} \|\,d\tau = \int_{0}^{t} \mysqrt{\left[f'(\tau)\right]^{2} + \left[g'(\tau)\right]^{2}}\,d\tau.\) (Where \(f(\tau), g(\tau)\) correspond to the \(x,y\) components of \(\mb{r}(t)\)).
                    \item Solving, if possible, the resulting expression for \(t\) as a function of \(s\).
                    \item Rewriting \(\mb{r}(t) = \mb{r}(t(s)) = \mb{r}\), so that the curve is written as a function of its length, from a given starting point.
                \end{itemize}
                \vspace*{-0.5cm}
                \subsubsection*{Curvature}
                For all \(\mb{r} \colon \ \kappa = \frac{||\mb{T}'(t)\| }{ \|\mb{r}'(t)\| }\); for \(\R^{3}\): \(\kappa = \frac{ \|\mb{r}'(t) \times \mb{r}''(t)\| }{ \|\mb{r}'(t)||^{3}}\); if \(y = f(x)\): \(\kappa = \frac{|y''(x)|}{[1 + (y'(x)^{2})]^{3/2}}\)
                \vspace*{-0.25cm}
                \subsubsection*{Motion}

                \begin{itemize}
                    \item \cyanit{Velocity:} \(\mb{v}(t) = \dvecfuc{r}{t}\).
                    \item \cyanit{Speed:} \(\| \mb{v}(t) \| = \| \dvecfuc{r}{t} \|\).
                    \item \cyanit{Acceleration:} \(\mb{a}(t) = \dvecfuc{v}{t} = \mb{r}''(t)\).
                \end{itemize}
                The motion of an object -- in 2-dimensions, typically, acted on only by gravity \(\mb{F}_{g} = -mg\mb{j}\), where \(g \approx 9.8\) m/s\(^{2}\) and \(m\) is the mass of the object. By Newton's second law, \(\mb{F} = m\mb{a}\), so we have \(\vecfuc{a}{t} = - g\mb{j}.\)
Thus, \(\vecfuc{v}{t} = -gt\mb{j} + \mb{v}_{0}\), where \(\mb{v}_{0}\) is the initial velocity vector, and \(\vecfuc{s}{t} = -\frac{1}{2}gt^{2}\mb{j} + \mb{v}_{0}t + \mb{s}_{0},\) where \(\mb{s}\) is the position, and \(\mb{s}_{0}\) is the initial position vector. Often, we have an object starting at the origin (so \(\mb{s}_{0} = \mb{0}\)) and fired at a velocity of \(v_{0}\) at an angle \(\theta\) above the horizon. Then, \(\vecfuc{s}{t} = v_{0}t\cos(\theta)\mb{i} + \left(v_{0}t\sin(\theta) - \frac{1}{2}gt^{2}\right)\mb{j}.\)

            \end{minipage}
        };
        %------------ Sum and Average of Independent Random Variables ---------------------
        \node[fancytitle, right=10pt] at (box.north west) {Chapter 3: Vector-Valued Functions};
    \end{tikzpicture}


    %------------ Maximum and Minimum of Independent Variables ---------------
    \begin{tikzpicture}
        \node [mybox] (box){%
            \begin{minipage}{0.46\textwidth}
                \begin{enumerate}
                    \item Consider the curve defined by the parametric equations \(x(t) = \sin(2t)\), \(y(t) = \cos(t)\), for \(0 \leq t \leq 2\pi\).
                    \begin{enumerate}
                        \item Find the equation of the tangent line to the curve at the point where \(t = \pi/3\).

                        \sol{
                            Solve for \(\frac{dy}{dt}\) and \(\frac{dx}{dt}\):
                        
                            \(y(t) = \cos t \Rightarrow \frac{dy}{dt} = -\sin t.\)
                        
                            \(x(t)           = \sin 2t   \Rightarrow
                            \frac{dx}{dt}  = 2\cos 2t.\)

                        Now, we have our slope, which we evaluate at \(t = \pi/3\):
                        \begin{small}
                            \[
                                \frac{dy}{dx} = \frac{dy/dt}{dx/dt} = \frac{-\sin t}{2\cos 2t} \quad \Rightarrow \quad \frac{-\sin t}{2\cos 2t} = \frac{-\mysqrt{3}/2}{-1} = \frac{\mysqrt{3}}{2}.
                            \]
                        \end{small}
                        With our slope, we need the points:
                        \begin{small}
                            \[
                                x(\pi / 3) = \sin(2(\pi / 3)) =  \mysqrt{3}/2 \quad \text{and} \quad y(\pi / 3) = \cos(\pi / 3) = 1/2.
                            \]
                        \end{small}
                        Putting it all together, we have \(y = \frac{\mysqrt{3}}{2}(x - \frac{\mysqrt{3}}{2}) + \frac{1}{2}\). (Simplify.)
                        }
                    \item Determine geometric area enclosed by the curve.

                    \sol{
                        Our equation has 4. We find 1 quadrant and multiply it by 4, we can get the total geometric area for the whole shape.
                        \begin{small}
                            \begin{align*}
                                4\int_{0}^{\pi / 2} y(t)\frac{dx}{dt} \ dt & = 4\int_{0}^{\pi / 2}\cos t(2\cos 2t) \ dt                                                   \\
                                                                           & = 8\int_{0}^{\pi / 2}\cos t(1 - 2 \sin^{2}t) \ dt                                            \\
                                                                           & = 8\int_{0}^{\pi / 2}\cos t - 2\cos t \sin^{2}t \ dt                                         \\
                                                                           & = 8\left[\int_{0}^{\pi / 2} \cos t \ dt - 2\int_{0}^{\pi / 2} \cos t \sin^{2}t \ dt \right]. 
                            \end{align*}
                             Thus, let \(u = \sin t\) such that \(\dfrac{du}{\cos t} = dt\). Hence,
                            \begin{align*}
                                8\left[\int_{0}^{\pi / 2} \cos t \ dt - 2\int_{0}^{\pi / 2} \cos t \sin^{2}t \ dt \right] & = 8\left[\int_{0}^{\pi / 2} \cos t \ dt - 2\int_{0}^{\pi / 2} u^{2} \ du \right] \\
                                                                                                                          & = 8\left[\sin t - \frac{2}{3}\sin^{3}t\right]_{0}^{\pi/2}   
                            \end{align*}
                        \end{small}
                        }
                        
                    \end{enumerate}
                    \item Find angle between \(\mathbf{u} = 6\mathbf{i} + 2\mathbf{j} - 5\mathbf{k}\) and \(\mathbf{v} = -4\mathbf{i} + \mathbf{j} - 7\mathbf{k}\).

                    \sol{
                        Find the magnitudes of \(\mathbf{u}\) and \(\mathbf{v}\). Then solve:
                        \begin{small}
                            \[
                            \cos \theta = \frac{\mathbf{u} \cdot \mathbf{v}}{\norm{u}\norm{v}} = \frac{13}{\mysqrt{65}\mysqrt{66}} \approx 0.198.
                        \]
                    \end{small}

                        Thus, \(\theta \approx \cos^{-1}(0.198) \approx \boxed{1.371}\) radians.
                    }
                \end{enumerate}
            \end{minipage}
        };
        %------------ Maximum and Minimum of Independent Variables ---------------------
        \node[fancytitle, right=10pt] at (box.north west) {Chapter 1 Examples};
    \end{tikzpicture}


    %------------ Some Continuous Distributions ---------------
    \begin{tikzpicture}
        \node [mybox] (box){%
            \begin{minipage}{0.46\textwidth}

                \begin{enumerate}
                    \setcounter{enumi}{2}
                    \item Determine a parametric equation for the line \textit{segment} that goes from the point \(P = (6,1,-2)\) to \(Q = (-2, 0, 5)\).

                    \sol{
                         \(x(t) = 6 - 8t; \,y(t) = 1 - t \,z(t) = -2 + 7t\), for \(0 \leq t \leq 1\).
                    }
                    \item Find a symmetric equation for the line which contains the points \(R = (4,-6,1)\) and \(S = (1,2,3)\).
                    
                    \sol{
                        \(\frac{x - 4}{-3} = \frac{y + 6}{8} = \frac{z - 1}{-2}.\)
                    }
                    \item Find the general form of an equation of the plane which contain the three points \(P = (3,1,-4)\), \(Q = (-2,0,5)\) and \(R = (4,-6,1)\).

                    \sol{
                     Let \(\mathbf{PQ} = \brackett{-5,-1,9}\) and \(\mathbf{PR} = \brackett{1,-7,5}\). Thus, \(\mb{n} = \mathbf{PQ} \times \mathbf{QR} =  \brackett{58,34,36}.\) General equation:
                    \[
                        58(x - 3) + 34(y - 1) + 36(z + 4) = 0 \quad \Rightarrow \quad 58x + 34y + 36z - 64 = 0.
                    \]
                    }
                    \item Find an equation, in symmetric form, of the line of intersection between the planes \(2x + y - z + 4 = 0\) and \(x - y + 3z = 1\).

                    \sol{
                        Add the plane equations to eliminate \(y\) so that \(3x + 2z = -3\).
                        Thus, \(x = -1 - \frac{2}{3}z\). Substitute this equation into the first equation to express \(y\) in terms of \(z\), giving \(y = -2 +\frac{7}{3}z.\)
                        Define \(z\) in terms of \(t\). Choose parameter \(t\) as \(t = -\frac{1}{3}z\). This gives \(z = -3t\). When we substitute our value \(t\) back into the previous two equations, we see that the parametric equations for the line of intersection are \(x = -1 + 2t\), \(y = -2 - 7t\), and \(z = -3t\). Therefore, the symmetric equations for the line are \(\boxed{\frac{x + 1}{2} = \frac{y + 2}{-7} = \frac{z}{-3}.}\)
                    }
                \end{enumerate}
            \end{minipage}
        };
        %------------ Some Continuous Distributions Header ---------------------
        \node[fancytitle, right=10pt] at (box.north west) {Chapter 1 \& 2 Examples (cont.)};
    \end{tikzpicture}


    %------------ Normal Distribution ---------------
    \begin{tikzpicture}
        \node [mybox] (box){%
            \begin{minipage}{0.46\textwidth}

                \begin{enumerate}
                    \item Write, in scalar form, an equation of the plane which contains the point \((5,2,1)\) and the line given by \(x + 2 = \dfrac{y}{4} = \dfrac{z - 5}{2}\).

                    \sol{
                      We start by parametrizing the line with common parameter \(t\): 
                      \(x + 2 = t \quad\Rightarrow \quad x = t - 2\),
                        \(\frac{y}{4} = t \quad\Rightarrow \quad y = 4t\), and
                         \(\frac{z - 5}{2} = t \quad \Rightarrow \quad z = 2t + 5\).
                      This gives us the parametric form: \((x,y,z) = (-2,0,5) + t(1,4,2)\)
                      Thus, the line passes through the point \((-2,0,5)\) and has the direction vector
                        \(\mb{v}_{1} = \brackett{1,4,2}.\)
                      Form a second vector \(\mb{v}_{2}\) by taking the difference between the given point and a point on the line:
                        \(\mb{v}_{2} = (5,2,1) - (-2,0,5) = \brackett{7,2,-4}.\)
                      With \(\mb{v}_{1}\) and \(\mb{v}_{2}\), we find the normal vector:
                      \[
                        \mb{n} = \mb{v}_{1} \times \mb{v}_{2} = \begin{vmatrix}
                          \mb{i} & \mb{j} & \mb{k} \\
                          1 & 4 & 2 \\
                          7 & 2 & -4
                        \end{vmatrix} = (-16 - 4)\mb{i} - (-4 - 14)\mb{j} + (2 - 28)\mb{k} = -20\mb{i} + 20\mb{j} - 26\mb{k}
                      \]
                      Therefore, we find the scalar form to be:
                      \[
                        \boxed{-20(x + 2) + 18y - 26(z - 5) = 0.}
                      \]
                    }
                \end{enumerate}

            \end{minipage}
        };

        %------------ Normal Distribution Header ---------------------
        \node[fancytitle, right=10pt] at (box.north west) {Chapter 2 Examples};
    \end{tikzpicture}
    %------------ Normal Distribution ---------------
    \begin{tikzpicture}
        \node [mybox] (box){%
            \begin{minipage}{0.46\textwidth}

                \begin{enumerate}
                    \setcounter{enumi}{1}
                    \item Determine the arc length parametrization for the curve \(\vecfuc{r}{t} = 3e^{t} \sin(t)\imb + 3e^{t}\cos(t)\jmb\), where you start from \(t = 0\). 

                    \sol{
                      Rewrite the arc length parametrization as:
                        \(s = \int_{0}^{t} \| \dvecfuc{r}{\tau} \|\,d\tau = \int_{0}^{t} \mysqrt{\left[f'(\tau)\right]^{2} + \left[g'(\tau)\right]^{2}}\,d\tau.\) Thus, 
                        \(f'(\tau) = 3e^{\tau}(\sin(\tau) + \cos(\tau)) \quad \text{and} \quad
                        g'(\tau) = 3e^{\tau}(\cos(\tau) - \sin(\tau)).\)
                      Thus, we have:
                      \begin{align*}
                        s &= \int_{0}^{t} \mysqrt{\left[3e^{\tau}\bigl(\sin(\tau) + \cos(\tau)\bigr)\right]^{2} + \left[3e^{\tau}\bigl(\cos(\tau) - \sin(\tau)\bigr)\right]^{2}}\,d\tau \\
      &= \int_{0}^{t} \mysqrt{9e^{2\tau} \left[2\bigl(\sin^{2}(\tau) + \cos^{2}(\tau)\bigr) + \bigl(2\sin(\tau)\cos(\tau) - 2\sin(\tau)\cos(\tau)\bigr) \right]} \\
      &= \int_{0}^{t} \mysqrt{9e^{2\tau} \cdot \left[2(1 + 0)\right]} 
                        = \int_{0}^{t} 3e^{\tau}\mysqrt{2}\,d\tau 
                        = 3\mysqrt{2}\int_{0}^{t} e^{\tau}\,d\tau 
                        = 3\mysqrt{2}(e^{t} - 1).
                      \end{align*}
                      With \(s\), we know that \(\mb{r}(t) = \mb{r}(t(s)) = \mb{r}\), so we need to find \(t\) in terms of \(s\):
                        \(s = 3\mysqrt{2}(e^{t} - 1) \Rightarrow
                        e^{t} = \dfrac{s}{3\mysqrt{2}} + 1 \Rightarrow
                        t = \ln\left(\dfrac{s}{3\mysqrt{2}} + 1\right).\)
                      Finally, by replacing \(t\) with \(t(s)\) in the original equation, we can get the arc length parametrization:
                      \[
                        \boxed{\vecfuc{r}{s} = \left(\dfrac{s}{\mysqrt{2}} + 3\right)\sin\left(\ln\left(\dfrac{s}{\mysqrt{2}} + 3\right)\right)\imb + \left(\dfrac{s}{\mysqrt{2}} + 3\right)\cos\left(\ln\left(\dfrac{s}{\mysqrt{2}} + 3\right)\right)\jmb.}
                      \]
                    }
                    \item Use curvature to find the equation of the osculating circle at the planar curve \(y = x^{3} - 4x + 1\) at \(x = 1\).

  \sol{
    First, we need to find the curvature of the curve at \(x = 1\). We start by finding the first and second derivatives of the function: \(y(x) = x^{3} - 4x + 1 \Rightarrow y'(x) = 3x^{2} - 4 \Rightarrow y''(x) = 6x.\)

    Then, we evaluate the point and the first and second derivatives at \(x = 1\): \(y(1)= -2; \, y'(1) = -1; \, y''(1)  = 6.\)

    Find the curvature: 
      \(\kappa = \frac{|y''(x)|}{\bigl(1 + y'(x)^{2}\bigr)^{3/2}} = \frac{6}{\bigl(1 + (-1)^{2}\bigr)^{3/2}} = \frac{3\mysqrt{2}}{2}.\)

    Find radius: \(R = \frac{1}{\kappa} = \frac{1}{\frac{3\mysqrt{2}}{2}} = \frac{2\mysqrt{2}}{6} = \frac{\mysqrt{2}}{3}.\)

    For the center, find \(\mb{N}\) at \(x = 1\):
      \(\mb{N} = \frac{(-y',1)}{\mysqrt{1 + (y')^{2}}} = \frac{\bigl(-(-1),1\bigr)}{\mysqrt{1 + (-1)^{2}}} = \frac{(1,1)}{\mysqrt{2}}\).

    The center \(C\) can be found by moving our point \(P(1,-2)\) the distance \(R\) along the unit normal vector:
      \(C = P + R\mb{N} = \left(\frac{4}{3},-\frac{5}{3}\right)\).
    This gives the equation for the osculating circle:
    \[
      \boxed{\left(x - \frac{4}{3}\right)^{2} + \left(y + \frac{5}{3}\right)^{2} = \frac{2}{9}.}
    \]
  }
                \end{enumerate}

            \end{minipage}
        };

        %------------ Normal Distribution Header ---------------------
        \node[fancytitle, right=10pt] at (box.north west) {Chapter 3 Examples};
    \end{tikzpicture}

\end{multicols*}



\end{document}