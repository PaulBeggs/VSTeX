\documentclass[11pt]{article}

\usepackage{fix-cm}
\usepackage{tikz}
\usetikzlibrary{shapes,backgrounds,calc,patterns}
\usepackage{amsmath,amsthm} 
\usepackage{amssymb}
\usepackage[framemethod=tikz]{mdframed}
% \usepackage{draculatheme}
\usepackage{mathrsfs}
\usepackage{changepage}
\usepackage{multicol}
\usepackage{mathtools}
\usepackage{hyperref}
\usepackage{slashed}
\usepackage{enumerate}
\usepackage{booktabs}
\usepackage{enumitem}
\usepackage{kantlipsum} 
\usepackage{pgfplots}
\pgfplotsset{compat=1.18}
\usetikzlibrary{decorations.markings}

\setlength{\parindent}{0pt}

\newmdenv[
  topline=false,
  bottomline=true,
  rightline=false,
  leftline=true,
  linewidth=1.5pt,
  linecolor=black, % default color, will be overridden in custom commands
  % backgroundcolor=draculabg, % Needed for Dracula theme
  % fontcolor=draculafg, % Needed for Dracula theme
  innertopmargin=0pt,
  innerbottommargin=5pt,
  innerrightmargin=10pt,
  innerleftmargin=10pt,
  leftmargin=0pt,
  rightmargin=0pt,
  skipabove=\topsep,
  skipbelow=\topsep,
]{customframedproof}

\newenvironment{proofpart}[2][black]{
    \begin{mdframed}[
        topline=false,
        bottomline=false,
        rightline=false,
        leftline=true,
        linewidth=1pt,
        linecolor=#1!40, % Custom color
        % innertopmargin=10pt,
        % innerbottommargin=10pt,
        innerleftmargin=10pt,
        innerrightmargin=10pt,
        leftmargin=0pt,
        rightmargin=0pt,
        % skipabove=\topsep,
        % skipbelow=\topsep%
    ]
    \noindent
    \begin{minipage}[t]{0.08\textwidth}%
        \textbf{#2}%
    \end{minipage}%
    \begin{minipage}[t]{0.90\textwidth}%
        \begin{adjustwidth}{0pt}{0pt}%
}{
    \end{adjustwidth}
    \end{minipage}
    \end{mdframed}
}

\newenvironment{solution}
  {\textit{Solution.}}



%%% AESTHETICS %%%
%-%-%-%-%-%-%-%-%-%-%-%-%-%-%-%-%-%-%-%-%-%-%-%-%-%-%-%-%-%-%-%-%-%-%-%-%-%-%


%%% Dimensions and Spacing %%%
\usepackage[left=0.5in,right=0.5in,top=1in,bottom=1in]{geometry}
% \usepackage{setspace}
% \linespread{1}
\usepackage{listings}
% \usepackage{minted}

%%% Define new colors %%%
\usepackage{xcolor}
\definecolor{orangehdx}{rgb}{0.96, 0.51, 0.16}

% Normal colors
\definecolor{xred}{HTML}{BD4242}
\definecolor{xblue}{HTML}{4268BD}
\definecolor{xgreen}{HTML}{52B256}
\definecolor{xpurple}{HTML}{7F52B2}
\definecolor{xorange}{HTML}{FD9337}
\definecolor{xdotted}{HTML}{999999}
\definecolor{xgray}{HTML}{777777}
\definecolor{xcyan}{HTML}{80F5DC}
\definecolor{xpink}{HTML}{F690EA}
\definecolor{xgrayblue}{HTML}{49B095}
\definecolor{xgraycyan}{HTML}{5AA1B9}

% Dark colors
\colorlet{xdarkred}{red!85!black}
\colorlet{xdarkblue}{xblue!85!black}
\colorlet{xdarkgreen}{xgreen!85!black}
\colorlet{xdarkpurple}{xpurple!85!black}
\colorlet{xdarkorange}{xorange!85!black}
\definecolor{xdarkcyan}{HTML}{008B8B}
\colorlet{xdarkgray}{xgray!85!black}

% Very dark colors
\colorlet{xverydarkblue}{xblue!50!black}

% Document-specific colors
\colorlet{normaltextcolor}{black}
\colorlet{figtextcolor}{xblue}

% Enumerated colors
\colorlet{xcol0}{black}
\colorlet{xcol1}{xred}
\colorlet{xcol2}{xblue}
\colorlet{xcol3}{xgreen}
\colorlet{xcol4}{xpurple}
\colorlet{xcol5}{xorange}
\colorlet{xcol6}{xcyan}
\colorlet{xcol7}{xpink!75!black}

% Blue-Purple (should just used colorbrewer...)
\definecolor{xrainbow0}{HTML}{e41a1c}
\definecolor{xrainbow1}{HTML}{a24057}
\definecolor{xrainbow2}{HTML}{606692}
\definecolor{xrainbow3}{HTML}{3a85a8}
\definecolor{xrainbow4}{HTML}{42977e}
\definecolor{xrainbow5}{HTML}{4aaa54}
\definecolor{xrainbow6}{HTML}{629363}
\definecolor{xrainbow7}{HTML}{7e6e85}
\definecolor{xrainbow8}{HTML}{9c509b}
\definecolor{xrainbow9}{HTML}{c4625d}
\definecolor{xrainbow10}{HTML}{eb751f}
\definecolor{xrainbow11}{HTML}{ff9709}

%%% FIGURES %%%
\usepackage{graphicx}  
\graphicspath{ {images/} }  
% \numberwithin{figure}{section}
\usepackage{float}
\usepackage{caption}

%%% Hyperlinks %%%
\usepackage{hyperref}
\definecolor{horange}{HTML}{f58026}
\hypersetup{
	colorlinks=true,
	linkcolor=horange,
	filecolor=horange,      
	urlcolor=horange,
}

\newcommand{\mysqrt}[1]{%
  \mathpalette\foo{#1}%
}
\newcommand{\dmysqrt}[1]{%
  \mathpalette\foodisplay{#1}%
}

\newcommand{\sol}[1]{
    \begin{customframedproof}[linecolor=orangehdx!75,]
        \begin{solution}
        #1
        \end{solution}
    \end{customframedproof}
}
\allowdisplaybreaks

% !TeX spellcheck = off
\newcommand{\foo}[2]{%
  % #1: math style, #2: content
  \sbox0{$#1\sqrt{#2}$}% Measure the size of the standard sqrt in the current style
  \begin{tikzpicture}[baseline=(sqrt.base)]
    \node[inner sep=0, outer sep=0] (sqrt) {$#1\sqrt{#2}$}; % Use the current math style
    \draw([yshift=-0.045em]sqrt.north east) -- ++(0,-0.5ex); % Draw the tick
  \end{tikzpicture}%
}
% !TeX spellcheck = off
\newcommand{\foodisplay}[2]{%
  % #1: math style, #2: content
  \sbox0{$#1\sqrt{#2}$}% Measure the size of the standard sqrt in the current style
  \begin{tikzpicture}[baseline=(sqrt.base)]
    \node[inner sep=0, outer sep=0] (sqrt) {$\displaystyle\sqrt{#2}$}; % Force displaystyle
    \draw[line width=0.4pt] ([yshift=-0.044em]sqrt.north east) -- ++(0,-0.5ex); % Draw the tick
  \end{tikzpicture}%
}

\newcommand{\barNotationT}[1]{\bigg|_{t = #1}}

\newcommand{\cyanit}[1]{\textit{\textcolor{cyan}{#1}}}

\newcommand{\brackett}[1]{\left\langle #1 \right\rangle}

\newcommand{\norm}[1]{\left\lVert \mathbf{#1}\right\rVert}

\newcommand{\mbi}{\mb{i}}
\newcommand{\mbj}{\mb{j}}
\newcommand{\mbk}{\mb{k}}
\newcommand{\mbr}{\mb{r}}
\newcommand{\mbu}{\mb{u}}
\newcommand{\mbv}{\mb{v}}

\newcommand{\vecfuc}[2]{\mb{#1}(#2)}
\newcommand{\dvecfuc}[2]{\mb{#1}'(#2)}
\newcommand{\normdvecfuc}[2]{||\mb{#1}'(#2)||}

\newcommand{\proj}{\text{proj}}

\newcommand{\mb}[1]{\mathbf{#1}}


% \renewcommand{\theenumi}{\arabic{enumi}} 
% \renewcommand{\labelenumi}{\theenumi.}

\title{Multivariable Calculus Exam III Corrections}
\author{Paul Beggs}
\date{\today}

%%% Custom Comands %%%
% Natural Numbers 
\newcommand{\N}{\ensuremath{\mathbb{N}}}

% Whole Numbers
\newcommand{\W}{\ensuremath{\mathbb{W}}}

% Integers
\newcommand{\Z}{\ensuremath{\mathbb{Z}}}

% Rational Numbers
\newcommand{\Q}{\ensuremath{\mathbb{Q}}}

% Real Numbers
\newcommand{\R}{\ensuremath{\mathbb{R}}}

% Complex Numbers
\newcommand{\C}{\ensuremath{\mathbb{C}}}

\newcommand{\I}{\ensuremath{\mathbb{I}}}

\newcommand{\p}{\partial}


\begin{document}

\maketitle

\section*{In-Class Portion}

\begin{enumerate}
    \item (6 points) Set up, but do not evaluate the integral, including appropriate limits, to find the circulation of the vector field \(\mb{F} = \cos(x\mb{i}) + xe^{y}\mb{j}\) along the curve \(\mb{r}(t) = t^{2}\mb{i} + \mysqrt{t}\mb{j}\), for \(0 \leq t \leq 4\). [You should write down a definite integral with only \(t\) as a variable and with \(dt\) as the differential.]

    \sol{
        Given the formula:
        \[
            \int_{C} \mb{F}(\mb{r}(t)) \cdot d\mb{r},
        \]
        we can write the integral as:
        \[
            \int_{0}^{4} \brackett{\cos(t^{2}), t^{2}e^{\mysqrt{t}}} \cdot \brackett{2t, \frac{1}{2\mysqrt{t}}} dt = \int_{0}^{4} \left(2t\cos(t^{2}) + t^{2}e^{\mysqrt{t}}\frac{1}{2\mysqrt{t}}\right) dt.
        \]
    }

    \item (6 points) Set up, but do not evaluate the integral, including appropriate limits, to find the flux of the vector field \(\mb{F} = \cos(x\mb{i}) + xe^{y}\mb{j}\) across the curve \(\mb{r}(t) = t^{2}\mb{i} + \mysqrt{t}\mb{j}\), for \(0 \leq t \leq 4\). [You should write down a definite integral with only \(t\) as a variable and with \(dt\) as the differential.] \\
    
    \sol{
        Given the formula:
        \[
            \int_{C} \mb{F}(\mb{r}(t)) \cdot \brackett{y'(t), -x'(t)} dt,
        \]
        we can write the integral as:
        \[
            \int_{0}^{4} \brackett{\cos(t^{2}), t^{2}e^{\mysqrt{t}}} \cdot \brackett{\frac{1}{2\mysqrt{t}}, -2t} dt = \int_{0}^{4} \left(\cos(t^{2})\frac{1}{2\mysqrt{t}} - t^{2}e^{\mysqrt{t}}(2t)\right) dt 
        \]
    }
    \newpage
    \setcounter{enumi}{5}
    \item (10 points) Set up the \textbf{line integral} that Stokes' Theorem would use to evaluate 
    \[
        \iint_{S} \nabla \times (x^{2}z\mb{i} + xy^{2}\mb{j} + xy\mb{k}) \cdot d\mb{S},
    \]
    where \(S\) is the part of the paraboloid \(z = 1 - x^{2} - y^{2}\) that lies above the \(xy\)-plane, oriented upward. \textbf{DO NOT} worry about working the integral - but you should write the eventual integral as a \(dt\) (or maybe \(d\theta\)) integral.

    \sol{
        This paraboloid is bounded by the circle \(x^{2} + y^{2} = 1\) in the \(xy\)-plane. This leaves us with the following bounding circle:
        \[
            z = 0, \quad x^{2} + y^{2} = 1,
        \]
        parameterized using polar coordinates:
        \[
            \mb{r}(\theta) = \brackett{\cos(\theta),\sin(\theta), 0} \quad \text{for } 0 \leq \theta \leq 2\pi.
        \]
        Then, for our integral, we need to find:
        \[
            d\mb{r} = \brackett{-\sin(\theta),\cos(\theta), 0} \, d\theta.
        \]
        We can then write the integral as:
        \begin{align*}
            \oint_{S} \mb{F}(\mb{r}(\theta)) \cdot d\mb{r} &= \int_{0}^{2\pi} \brackett{0,\cos(\theta) \sin^{2}(\theta), \cos(\theta) \sin(\theta)} \cdot \brackett{(-\sin(\theta),\cos(\theta), 0)} \, d\theta \\
            &= \int_{0}^{2\pi} \cos^{2}(\theta)\sin^{2}(\theta) \, d\theta.
        \end{align*}
    }
\end{enumerate}

\newpage

\section*{Half-Point Redo}

\begin{enumerate}
    \setcounter{enumi}{4}
    \item (6 points) Consider the vector field \(\mb{F}(x,y) = (2x + y\cos(xy))\mb{i} + (-3 + x\cos(xy))\mb{j}\).
    \begin{enumerate}
        \item Show that \(\mb{F}\) is conservative.  \\

        \sol{
            A vector field is conservative if the mixed partial derivatives are equal. That is:
            \[
                \frac{\p}{\p y}(2x + y\cos(xy)) = \cos(xy) - xy\sin(xy) = \frac{\p}{\p x}(-3 + x\cos(xy)).
            \]
            Therefore, \(\mb{F}\) is conservative.
        }
        \item Find a potential function \(f\) for \(\mb{F}\). \\

        \sol{
            We can find a potential function by following the following algorithm:
            \begin{itemize}
                \item Integrate \(P\) with respect to \(x\) to get \(g(x,y) + h(y)\):
                \begin{align*}
                    g(x,y) &= \int (2x + y\cos(xy)) \, dx \\
                    &= \int 2x \, dx + \int y\cos(xy) \, dx \\
                    \intertext{Make a substitution \(u = xy\), then \(du = y \, dx\) and \(dx = \frac{du}{y}\):}
                    g(x,y) &= x^{2} + y\int \cos(u) \, \frac{du}{y} \\
                    \intertext{The \(y\)'s cancel out, and we substitute back \(u = xy\):}
                    g(x,y) &= x^{2} + \sin(xy) \\
                    \intertext{Therefore,}
                    g(x,y) + h(y) &= x^{2} + \sin(xy) + h(y).
                \end{align*}
                \item Take the partial derivative of \(g(x,y) + h(y)\) with respect to \(y\), which results in function \(g_{y}(x,y) + h'(y)\):
                \[
                    g_{y}(x,y) = \frac{\p}{\p y}\left(x^{2} + \sin(xy) + h(y)\right) = x\cos(xy) + h'(y).
                \]
                \item Use the equation \(g_{y}(x,y) = Q\) to find \(h'(y)\):
                \[
                    x\cos(xy) + h'(y) = -3 + x\cos(xy) \implies h'(y) = -3.
                \]
                \item Integrate \(h'(y)\) with respect to \(y\) to find \(h(y)\):
                \[
                    h(y) = \int -3 \, dy = -3y + C.
                \]
                \item Substitute \(h(y)\) back into \(g(x,y)\) to find the potential function:
                \[
                    f(x,y) = g(x,y) + h(y) = x^{2} + \sin(xy) - 3y + C.
                \]
            \end{itemize}
            Therefore, the potential function is:
            \[
                f(x,y) = x^{2} + \sin(xy) - 3y + C.
            \]
            We can check this work by taking the gradient of \(f\) and checking that it is equal to \(\mb{F}\):
            \[
                \nabla f(x,y) = \brackett{\frac{\p}{\p x}f(x,y), \frac{\p}{\p y}f(x,y)} = \brackett{2x + y\cos(xy), -3 + x\cos(xy)} = \mb{F}(x,y).
            \]
        }
        \item Determine the value of \(\displaystyle \int_{C} \mb{F} \cdot d\mb{r}\), where \(C\) is a plane curve which starts at \((0,0)\) and ends at \((2,1)\).

        \sol{
            Since \(\mb{F}\) is conservative, we can use the Fundamental Theorem of Line Integrals to evaluate the integral:
            \[
                \int_{C} \mb{F} \cdot d\mb{r} = f(2,1) - f(0,0).
            \]
            We can find \(f(2,1)\) and \(f(0,0)\):
            \begin{align*}
                f(2,1) &= 2^{2} + \sin(2) - 3(1) + C = 4 + \sin(2) - 3 + C = 1 + \sin(2) + C \\
                f(0,0) &= 0^{2} + \sin(0) - 3(0) + C = 0 + 0 - 0 + C = C.
            \end{align*}
            Therefore:
            \[
                \int_{C} \mb{F} \cdot d\mb{r} = (1 + \sin(2) + C) - C = 1 + \sin(2).
            \]
        }
    \end{enumerate}
\end{enumerate}


\end{document}