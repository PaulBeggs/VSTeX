\documentclass[11pt]{article}

\usepackage{fix-cm}
\usepackage{tikz}
\usetikzlibrary{shapes,backgrounds,calc,patterns}
\usepackage{amsmath,amsthm} 
\usepackage{amssymb}
\usepackage[framemethod=tikz]{mdframed}
% \usepackage{C:/Users/paulb/VSTeX/local/draculatheme}
\usepackage{mathrsfs}
\usepackage{changepage}
\usepackage{multicol}
\usepackage{mathtools}
\usepackage{hyperref}
\usepackage{slashed}
\usepackage{enumerate}
\usepackage{booktabs}
\usepackage{enumitem}
\usepackage{kantlipsum} 
\usepackage{pgfplots}
\pgfplotsset{compat=1.18}
\usetikzlibrary{decorations.markings}

\setlength{\parindent}{0pt}

\newmdenv[
  topline=false,
  bottomline=true,
  rightline=false,
  leftline=true,
  linewidth=1.5pt,
  linecolor=black, % default color, will be overridden in custom commands
  % backgroundcolor=draculabg, % Needed for Dracula theme
  % fontcolor=draculafg, % Needed for Dracula theme
  innertopmargin=0pt,
  innerbottommargin=5pt,
  innerrightmargin=10pt,
  innerleftmargin=10pt,
  leftmargin=0pt,
  rightmargin=0pt,
  skipabove=\topsep,
  skipbelow=\topsep,
]{customframedproof}

\newenvironment{proofpart}[2][black]{
    \begin{mdframed}[
        topline=false,
        bottomline=false,
        rightline=false,
        leftline=true,
        linewidth=1pt,
        linecolor=#1!40, % Custom color
        % innertopmargin=10pt,
        % innerbottommargin=10pt,
        innerleftmargin=10pt,
        innerrightmargin=10pt,
        leftmargin=0pt,
        rightmargin=0pt,
        % skipabove=\topsep,
        % skipbelow=\topsep%
    ]
    \noindent
    \begin{minipage}[t]{0.08\textwidth}%
        \textbf{#2}%
    \end{minipage}%
    \begin{minipage}[t]{0.90\textwidth}%
        \begin{adjustwidth}{0pt}{0pt}%
}{
    \end{adjustwidth}
    \end{minipage}
    \end{mdframed}
}

\newenvironment{solution}
  {\textit{Solution.}}



%%% AESTHETICS %%%
%-%-%-%-%-%-%-%-%-%-%-%-%-%-%-%-%-%-%-%-%-%-%-%-%-%-%-%-%-%-%-%-%-%-%-%-%-%-%


%%% Dimensions and Spacing %%%
\usepackage[left=0.5in,right=0.5in,top=1in,bottom=1in]{geometry}
% \usepackage{setspace}
% \linespread{1}
\usepackage{listings}
\usepackage{minted}

%%% Define new colors %%%
\usepackage{xcolor}
\definecolor{orangehdx}{rgb}{0.96, 0.51, 0.16}

% Normal colors
\definecolor{xred}{HTML}{BD4242}
\definecolor{xblue}{HTML}{4268BD}
\definecolor{xgreen}{HTML}{52B256}
\definecolor{xpurple}{HTML}{7F52B2}
\definecolor{xorange}{HTML}{FD9337}
\definecolor{xdotted}{HTML}{999999}
\definecolor{xgray}{HTML}{777777}
\definecolor{xcyan}{HTML}{80F5DC}
\definecolor{xpink}{HTML}{F690EA}
\definecolor{xgrayblue}{HTML}{49B095}
\definecolor{xgraycyan}{HTML}{5AA1B9}

% Dark colors
\colorlet{xdarkred}{red!85!black}
\colorlet{xdarkblue}{xblue!85!black}
\colorlet{xdarkgreen}{xgreen!85!black}
\colorlet{xdarkpurple}{xpurple!85!black}
\colorlet{xdarkorange}{xorange!85!black}
\definecolor{xdarkcyan}{HTML}{008B8B}
\colorlet{xdarkgray}{xgray!85!black}

% Very dark colors
\colorlet{xverydarkblue}{xblue!50!black}

% Document-specific colors
\colorlet{normaltextcolor}{black}
\colorlet{figtextcolor}{xblue}

% Enumerated colors
\colorlet{xcol0}{black}
\colorlet{xcol1}{xred}
\colorlet{xcol2}{xblue}
\colorlet{xcol3}{xgreen}
\colorlet{xcol4}{xpurple}
\colorlet{xcol5}{xorange}
\colorlet{xcol6}{xcyan}
\colorlet{xcol7}{xpink!75!black}

% Blue-Purple (should just used colorbrewer...)
\definecolor{xrainbow0}{HTML}{e41a1c}
\definecolor{xrainbow1}{HTML}{a24057}
\definecolor{xrainbow2}{HTML}{606692}
\definecolor{xrainbow3}{HTML}{3a85a8}
\definecolor{xrainbow4}{HTML}{42977e}
\definecolor{xrainbow5}{HTML}{4aaa54}
\definecolor{xrainbow6}{HTML}{629363}
\definecolor{xrainbow7}{HTML}{7e6e85}
\definecolor{xrainbow8}{HTML}{9c509b}
\definecolor{xrainbow9}{HTML}{c4625d}
\definecolor{xrainbow10}{HTML}{eb751f}
\definecolor{xrainbow11}{HTML}{ff9709}

%%% FIGURES %%%
\usepackage{graphicx}  
\graphicspath{ {images/} }  
% \numberwithin{figure}{section}
\usepackage{float}
\usepackage{caption}

%%% Hyperlinks %%%
\usepackage{hyperref}
\definecolor{horange}{HTML}{f58026}
\hypersetup{
	colorlinks=true,
	linkcolor=horange,
	filecolor=horange,      
	urlcolor=horange,
}

\newcommand{\mysqrt}[1]{%
  \mathpalette\foo{#1}%
}
\newcommand{\dmysqrt}[1]{%
  \mathpalette\foodisplay{#1}%
}

\newcommand{\sol}[1]{
    \begin{customframedproof}[linecolor=orangehdx!75,]
        \begin{solution}
        #1
        \end{solution}
    \end{customframedproof}
}
\allowdisplaybreaks

% !TeX spellcheck = off
\newcommand{\foo}[2]{%
  % #1: math style, #2: content
  \sbox0{$#1\sqrt{#2}$}% Measure the size of the standard sqrt in the current style
  \begin{tikzpicture}[baseline=(sqrt.base)]
    \node[inner sep=0, outer sep=0] (sqrt) {$#1\sqrt{#2}$}; % Use the current math style
    \draw([yshift=-0.045em]sqrt.north east) -- ++(0,-0.5ex); % Draw the tick
  \end{tikzpicture}%
}
% !TeX spellcheck = off
\newcommand{\foodisplay}[2]{%
  % #1: math style, #2: content
  \sbox0{$#1\sqrt{#2}$}% Measure the size of the standard sqrt in the current style
  \begin{tikzpicture}[baseline=(sqrt.base)]
    \node[inner sep=0, outer sep=0] (sqrt) {$\displaystyle\sqrt{#2}$}; % Force displaystyle
    \draw[line width=0.4pt] ([yshift=-0.044em]sqrt.north east) -- ++(0,-0.5ex); % Draw the tick
  \end{tikzpicture}%
}

\newcommand{\barNotationT}[1]{\bigg|_{t = #1}}

\newcommand{\cyanit}[1]{\textit{\textcolor{cyan}{#1}}}

\newcommand{\brackett}[1]{\left\langle #1 \right\rangle}

\newcommand{\norm}[1]{\left\lVert \mathbf{#1}\right\rVert}

\newcommand{\mbi}{\mb{i}}
\newcommand{\mbj}{\mb{j}}
\newcommand{\mbk}{\mb{k}}
\newcommand{\mbr}{\mb{r}}
\newcommand{\mbu}{\mb{u}}
\newcommand{\mbv}{\mb{v}}

\newcommand{\vecfuc}[2]{\mb{#1}(#2)}
\newcommand{\dvecfuc}[2]{\mb{#1}'(#2)}
\newcommand{\normdvecfuc}[2]{||\mb{#1}'(#2)||}

\newcommand{\proj}{\text{proj}}

\newcommand{\mb}[1]{\mathbf{#1}}


% \renewcommand{\theenumi}{\arabic{enumi}} 
% \renewcommand{\labelenumi}{\theenumi.}

\title{Multivariable Calculus Exam I Corrections}
\author{Paul Beggs}
\date{\today}

%%% Custom Comands %%%
% Natural Numbers 
\newcommand{\N}{\ensuremath{\mathbb{N}}}

% Whole Numbers
\newcommand{\W}{\ensuremath{\mathbb{W}}}

% Integers
\newcommand{\Z}{\ensuremath{\mathbb{Z}}}

% Rational Numbers
\newcommand{\Q}{\ensuremath{\mathbb{Q}}}

% Real Numbers
\newcommand{\R}{\ensuremath{\mathbb{R}}}

% Complex Numbers
\newcommand{\C}{\ensuremath{\mathbb{C}}}

\newcommand{\I}{\ensuremath{\mathbb{I}}}


\begin{document}

\maketitle

\section*{In Class Portion}

\begin{enumerate}
    \item Consider the parametric curve defined by \(x(t) = 3t^{2} - 8t + 1\), \(y(t) = e^{-t^{2}}\), for \(0 \leq t \leq 2\).
    \begin{enumerate}
        \item (4 points) Find the equation, in regular Cartesian coordinates, of the tangent line to this curve at \(t = 1\). Please use exact values here!
        \sol{
            First, we compute the derivatives of \(x(t)\) and \(y(t)\) with respect to \(t\):
            \[
                \tfrac{dx}{dt} = 6t - 8 \quad \text{and} \quad \tfrac{dy}{dt} = -2te^{-t^{2}}.
            \]
            Plugging this into the formula for slope, we see that:
            \[
                \frac{dx}{dy} = \frac{dy/dt}{dx/dt} = \frac{-2te^{-t^{2}}}{6t - 8}.
            \]
            To get our points, we plug in \(t = 1\):
            \[
                x(1) = 3(1)^{2} - 8(1) + 1 = -4 \quad \text{and} \quad y(1) = e^{-1}.
            \]
            Thus, our point is \((-4, e^{-1})\). Plugging in \(t = 1\) into the slope formula, we get:
            \[
                \frac{dy}{dx}\bigg|_{t = 1} = \frac{-2e^{-1}}{6 - 8} = e^{-1}.
            \]
            Thus, the equation of the tangent line is:
            \[
                \boxed{y = e^{-1}(x + 4) + e^{-1}.}
            \]
        }
        \item (4 points) Is this curve concave up, down, or neither when \(t = 1\)? Justify this answer.
        \sol{
            To determine concavity, we must solve the following equation:
            \[
                \dfrac{d^{2}y}{dx^{2}} = \frac{\frac{d}{dt}(dy/dx)}{dx/dt} = \frac{\frac{d}{dt}(-2te^{-t^{2}})}{6t - 8} = \frac{-2e^{-t^{2}} + 4t^{2}e^{-t^{2}}}{6t - 8} \quad \Rightarrow \quad  t = 1 \quad \Rightarrow \quad -e^{-1}.
            \]
            Since \(-e^{-1} < 0\), the curve is concave down at \(t = 1\). (\textit{Correction Explained}: My equation for concavity was incorrect on my test sheet.)
        }
    \end{enumerate}
    \newpage
    \item (4 points each) Let \(\mb{u} = 5\mbi + 2\mbj - 3\mbk\) and \(\mbv = - \mbj + 2\mbk\).
    \begin{enumerate}
        \setcounter{enumii}{2}
        \item Determine \(\proj_{\mbv}\mb{u}\). Leave all components as exact values.
        \sol{
            We know that \(\proj_{\mbv}\mb{u} = \frac{\mbu \cdot \mbv}{\norm{\mbv}^{2}}\mbv\). Thus, we compute:
            \[
                \mbu \cdot \mbv = 5(0) + 2(-1) + (-3)(2) = -8 \quad \text{and} \quad \norm{\mbv}^{2} = \bigl(\mysqrt{5}\bigr)^{2} = 5.
            \]
            Thus, \(\proj_{\mbv}\mb{u} = \frac{-8}{5}(-\mbj + 2\mbk) = \boxed{\frac{8}{5}\mbj - \frac{16}{5}\mbk.}\)
        }
    \end{enumerate}
    \item (4 points) Let \(\mbu = \brackett{5,-1,2}\) and \(\mbv = \brackett{-2,y,z}\). What is the relationship between \(y\) and \(z\) which makes \(\mbu\) orthogonal to \(\mbv\)?
    \sol{
        We know that two vectors are orthogonal if their dot product is zero. Thus:
        \[
            \mbu \cdot \mbv = 5(-2) + (-1)y + 2z = -10 - y + 2z.
        \]
        Therefore, the relationship between \(y\) and \(z\) which makes \(\mbu\) orthogonal to \(\mbv\) is:
        \[
            \boxed{-10 - y + 2z = 0 \quad \Rightarrow \quad y = 2z - 10.}
        \]
    }
    \setcounter{enumi}{4}
    \item (6 points) Find an equation in scalar form of the plane which passes through \((-2,7,1)\) and is perpendicular to the planes \(3x + y - z = 0 \) and \(-2x - y + 5z + 1 = 0\) [Hint: Think about what the relationship among the various normal vectors must be.]
    \sol{
        For the plane to be perpendicular to a given plane, its normal vector must lie in that given plane. Hence, our normal vector must be orthogonal to both \(\brackett{3,1,-1}\) and \(\brackett{-2,-1,5}\). Thus, we can take the cross product of these two vectors to get our normal vector:
        \[
            \brackett{3,1,-1} \times \brackett{-2,-1,5} = \begin{vmatrix}
                \mbi & \mbj & \mbk \\
                3 & 1 & -1 \\
                -2 & -1 & 5
            \end{vmatrix} = \brackett{4,-13,-1}.
        \]
        With our normal vector found, we can plug in our point to get our scalar equation:
        \[
            \boxed{4(x + 2) - 13(y - 7) - (z - 1) = 0 \quad \Rightarrow \quad 4x - 13y - z + 100 = 0}
        \]
    }
    \newpage
    \item (6 points) Find the exact value of curvature \(\kappa\) for the curve defined by \(\vecfuc{r}{t} = (t^{2} - t)\mbi + (t^{3} - 7t + 1)\mbj + t^{3}\mbk\) at the point \(t = 1\). [Hint: Since this is defined in \(\R^{3}\), it is \textit{significantly} easier to use the version of \(\kappa\) which uses a cross product!] Numerical approximations, rounded to 4 decimal places, are appropriate here.
    \sol{
        For this problem, we will use the following formula for \(\kappa\):
        \[
            \frac{||\dvecfuc{r}{t} \times \mb{r}''(t)\| }{ \|\dvecfuc{r}{t}||^{3}}.
        \]
        Thus, we need to find \(\dvecfuc{r}{t}\) and \(\mb{r}''(t)\). We find:
        \[
            \dvecfuc{r}{t} = (2t - 1)\mbi + (3t^{2} - 7)\mbj + 3t^{2}\mbk \quad \text{and} \quad \mb{r}''(t) = 2\mbi + 6t\mbj + 6t\mbk.
        \]
        Plugging in \(t = 1\), we get:
        \[
            \dvecfuc{r}{1} = \mbi - 4\mbj + 3\mbk \quad \text{and} \quad \mb{r}''(1) = 2\mbi + 6\mbj + 6\mbk.
        \]
        Next, we need to find \(\| \mb{r}'(t) \|^{3}\):
        \[
            \| \dvecfuc{r}{t} \|^{3} = \bigl(\mysqrt{1 + 16 + 9}\bigr)^{3} = \frac{\left(\mysqrt{\mysqrt{\mysqrt{\left({\sum_{n = 0}^{\infty}\frac{\left(\tfrac{3}{2}\ln 26\right)^{n}}{n!}}\right)^{5}}}}\right)^{\frac{729}{8}}}{\int_{0}^{2\pi} \dmysqrt{(\tfrac{1}{8} - \tfrac{1}{8} \cos \beta)^{2} + (\tfrac{1}{8} \sin \beta)^{2}} \ d\beta} = 26^{3/2}.
        \]
        Taking the cross product of \(\dvecfuc{r}{1}\) and \(\mb{r}''(1)\), we get:
        \[
            \dvecfuc{r}{1} \times \mb{r}''(1) = \begin{vmatrix}
                \mbi & \mbj & \mbk \\
                1 & -4 & 3 \\
                2 & 6 & 6
            \end{vmatrix} =  \bigl((-24) - 18\bigr)\mbi - (6 - 6)\mbj + \bigl(6 - (-8)\bigr)\mbk = -42\mbi + 14\mbk.
        \]
        Then, we calculate the magnitude of this cross product:
        \[
            \| \dvecfuc{r}{1} \times \mb{r}''(1) \| = \mysqrt{42^{2} + 0 + 14^{2}} = \mysqrt{42^{2} + 14^{2}} = \mysqrt{1764 + 196} = 14\mysqrt{10}.
        \]
        Putting everything together:
        \[
            \kappa = \frac{14\mysqrt{10}}{26^{3/2}} = \boxed{0.3339.}
        \]
    }
\end{enumerate}



\end{document}