\documentclass[11pt]{article}

\usepackage{fix-cm}
\usepackage{tikz}
\usetikzlibrary{shapes,backgrounds,calc,patterns}
\usepackage{amsmath,amsthm} 
\usepackage{amssymb}
\usepackage[framemethod=tikz]{mdframed}
\usepackage{C:/Users/paulb/VSTeX/local/draculatheme}
\usepackage{mathrsfs}
\usepackage{changepage}
\usepackage{multicol}
\usepackage{mathtools}
\usepackage{hyperref}
\usepackage{slashed}
\usepackage{enumerate}
\usepackage{booktabs}
\usepackage{enumitem}
\usepackage{kantlipsum} 
\usepackage{pgfplots}
\pgfplotsset{compat=1.18}
\usetikzlibrary{decorations.markings}

\setlength{\parindent}{0pt}

\newmdenv[
  topline=false,
  bottomline=true,
  rightline=false,
  leftline=true,
  linewidth=1.5pt,
  linecolor=black, % default color, will be overridden in custom commands
  backgroundcolor=draculabg, % Needed for Dracula theme
  fontcolor=draculafg, % Needed for Dracula theme
  innertopmargin=0pt,
  innerbottommargin=5pt,
  innerrightmargin=10pt,
  innerleftmargin=10pt,
  leftmargin=0pt,
  rightmargin=0pt,
  skipabove=\topsep,
  skipbelow=\topsep,
]{customframedproof}

\newenvironment{proofpart}[2][black]{
    \begin{mdframed}[
        topline=false,
        bottomline=false,
        rightline=false,
        leftline=true,
        linewidth=1pt,
        linecolor=#1!40, % Custom color
        % innertopmargin=10pt,
        % innerbottommargin=10pt,
        innerleftmargin=10pt,
        innerrightmargin=10pt,
        leftmargin=0pt,
        rightmargin=0pt,
        % skipabove=\topsep,
        % skipbelow=\topsep%
    ]
    \noindent
    \begin{minipage}[t]{0.08\textwidth}%
        \textbf{#2}%
    \end{minipage}%
    \begin{minipage}[t]{0.90\textwidth}%
        \begin{adjustwidth}{0pt}{0pt}%
}{
    \end{adjustwidth}
    \end{minipage}
    \end{mdframed}
}

\newenvironment{solution}
  {\textit{Solution.}}



%%% AESTHETICS %%%
%-%-%-%-%-%-%-%-%-%-%-%-%-%-%-%-%-%-%-%-%-%-%-%-%-%-%-%-%-%-%-%-%-%-%-%-%-%-%


%%% Dimensions and Spacing %%%
\usepackage[left=0.5in,right=0.5in,top=1in,bottom=1in]{geometry}
% \usepackage{setspace}
% \linespread{1}
\usepackage{listings}
\usepackage{minted}

%%% Define new colors %%%
\usepackage{xcolor}
\definecolor{orangehdx}{rgb}{0.96, 0.51, 0.16}

% Normal colors
\definecolor{xred}{HTML}{BD4242}
\definecolor{xblue}{HTML}{4268BD}
\definecolor{xgreen}{HTML}{52B256}
\definecolor{xpurple}{HTML}{7F52B2}
\definecolor{xorange}{HTML}{FD9337}
\definecolor{xdotted}{HTML}{999999}
\definecolor{xgray}{HTML}{777777}
\definecolor{xcyan}{HTML}{80F5DC}
\definecolor{xpink}{HTML}{F690EA}
\definecolor{xgrayblue}{HTML}{49B095}
\definecolor{xgraycyan}{HTML}{5AA1B9}

% Dark colors
\colorlet{xdarkred}{red!85!black}
\colorlet{xdarkblue}{xblue!85!black}
\colorlet{xdarkgreen}{xgreen!85!black}
\colorlet{xdarkpurple}{xpurple!85!black}
\colorlet{xdarkorange}{xorange!85!black}
\definecolor{xdarkcyan}{HTML}{008B8B}
\colorlet{xdarkgray}{xgray!85!black}

% Very dark colors
\colorlet{xverydarkblue}{xblue!50!black}

% Document-specific colors
\colorlet{normaltextcolor}{black}
\colorlet{figtextcolor}{xblue}

% Enumerated colors
\colorlet{xcol0}{black}
\colorlet{xcol1}{xred}
\colorlet{xcol2}{xblue}
\colorlet{xcol3}{xgreen}
\colorlet{xcol4}{xpurple}
\colorlet{xcol5}{xorange}
\colorlet{xcol6}{xcyan}
\colorlet{xcol7}{xpink!75!black}

% Blue-Purple (should just used colorbrewer...)
\definecolor{xrainbow0}{HTML}{e41a1c}
\definecolor{xrainbow1}{HTML}{a24057}
\definecolor{xrainbow2}{HTML}{606692}
\definecolor{xrainbow3}{HTML}{3a85a8}
\definecolor{xrainbow4}{HTML}{42977e}
\definecolor{xrainbow5}{HTML}{4aaa54}
\definecolor{xrainbow6}{HTML}{629363}
\definecolor{xrainbow7}{HTML}{7e6e85}
\definecolor{xrainbow8}{HTML}{9c509b}
\definecolor{xrainbow9}{HTML}{c4625d}
\definecolor{xrainbow10}{HTML}{eb751f}
\definecolor{xrainbow11}{HTML}{ff9709}

%%% FIGURES %%%
\usepackage{graphicx}  
\graphicspath{ {images/} }  
% \numberwithin{figure}{section}
\usepackage{float}
\usepackage{caption}

%%% Hyperlinks %%%
\usepackage{hyperref}
\definecolor{horange}{HTML}{f58026}
\hypersetup{
	colorlinks=true,
	linkcolor=horange,
	filecolor=horange,      
	urlcolor=horange,
}

\newcommand{\mysqrt}[1]{%
  \mathpalette\foo{#1}%
}
\newcommand{\dmysqrt}[1]{%
  \mathpalette\foodisplay{#1}%
}

\newcommand{\sol}[1]{
    \begin{customframedproof}[linecolor=orangehdx!75,]
        \begin{solution}
        #1
        \end{solution}
    \end{customframedproof}
}

% !TeX spellcheck = off
\newcommand{\foo}[2]{%
  % #1: math style, #2: content
  \sbox0{$#1\sqrt{#2}$}% Measure the size of the standard sqrt in the current style
  \begin{tikzpicture}[baseline=(sqrt.base)]
    \node[inner sep=0, outer sep=0] (sqrt) {$#1\sqrt{#2}$}; % Use the current math style
    \draw([yshift=-0.045em]sqrt.north east) -- ++(0,-0.5ex); % Draw the tick
  \end{tikzpicture}%
}
% !TeX spellcheck = off
\newcommand{\foodisplay}[2]{%
  % #1: math style, #2: content
  \sbox0{$#1\sqrt{#2}$}% Measure the size of the standard sqrt in the current style
  \begin{tikzpicture}[baseline=(sqrt.base)]
    \node[inner sep=0, outer sep=0] (sqrt) {$\displaystyle\sqrt{#2}$}; % Force displaystyle
    \draw[line width=0.4pt] ([yshift=-0.044em]sqrt.north east) -- ++(0,-0.5ex); % Draw the tick
  \end{tikzpicture}%
}

\newcommand{\barNotationT}[1]{\bigg|_{t = #1}}

\newcommand{\cyanit}[1]{\textit{\textcolor{cyan}{#1}}}

\newcommand{\brackett}[1]{\left\langle #1 \right\rangle}

\newcommand{\norm}[1]{\left\lVert \mathbf{#1}\right\rVert}

\newcommand{\imb}{\mb{i}}
\newcommand{\jmb}{\mb{j}}
\newcommand{\kmb}{\mb{k}}
\newcommand{\rmb}{\mb{r}}
\newcommand{\umb}{\mb{u}}

\newcommand{\vecfuc}[2]{\mb{#1}(#2)}
\newcommand{\dvecfuc}[2]{\mb{#1}'(#2)}
\newcommand{\normdvecfuc}[2]{||\mb{#1}'(#2)||}

\newcommand{\proj}{\text{proj}}

\newcommand{\mb}[1]{\mathbf{#1}}

% \renewcommand{\theenumi}{\arabic{enumi}} 
% \renewcommand{\labelenumi}{\theenumi.}

\title{Multivariable Calculus Practice Set II}
\author{Paul Beggs}
\date{\today}

%%% Custom Comands %%%
% Natural Numbers 
\newcommand{\N}{\ensuremath{\mathbb{N}}}

% Whole Numbers
\newcommand{\W}{\ensuremath{\mathbb{W}}}

% Integers
\newcommand{\Z}{\ensuremath{\mathbb{Z}}}

% Rational Numbers
\newcommand{\Q}{\ensuremath{\mathbb{Q}}}

% Real Numbers
\newcommand{\R}{\ensuremath{\mathbb{R}}}

% Complex Numbers
\newcommand{\C}{\ensuremath{\mathbb{C}}}

\newcommand{\I}{\ensuremath{\mathbb{I}}}


\begin{document}

\maketitle

\begin{enumerate}
  \item (2 points) Write, in general equation form, an equation of the plane which contains the three points \(P = (2,7,3)\), \(Q = (-5,0,1)\), and \(R = (-3,1,2)\).

  \sol{
    First, we find \(\mb{PQ}\) and \(\mb{PR}\):
    \[
      \mb{PQ} = \brackett{-7,-7,-2} \qquad \text{and} \qquad \mb{PR} = \brackett{-5,-6,-1}.
    \]  
    With \(\mb{PQ}\) and \(\mb{PR}\), we can find \(\mb{n}\) by solving for the cross product:
    \[
      \mb{n} =  \mb{PQ} \times \mb{PR} = \begin{vmatrix}
        \mb{i} & \mb{j} & \mb{k} \\
        -7 & -7 & -2 \\
        -5 & -6, & -1 
      \end{vmatrix} = (7 -12)\mb{i} - (7 - 10)\mb{j} + (42 - 35)\mb{k} = -5\mb{i} + 3\mb{j} + 7\mb{k}.
    \]
    With \(\mb{n}\), we get the general formula:
    \[
      \boxed{-5(x - 2) + 3(y - 7) + 7(z - 3) = 0.}
    \]
  }

  \item (2 points) Write, in scalar form, an equation of the plane which contains the point \((5,2,1)\) and the line given by \(x + 2 = \dfrac{y}{4} = \dfrac{z - 5}{2}\).

  \sol{
    We start by parametrizing the line with common parameter \(t\): 
    \begin{itemize}
      \item \(x + 2 = t \quad\Rightarrow \quad x = t - 2\),
      \item \(\frac{y}{4} = t \quad\Rightarrow \quad y = 4t\), and
      \item \(\frac{z - 5}{2} = t \quad \Rightarrow \quad z = 2t + 5\).
    \end{itemize}
    This gives us the parametric form:
    \[
      (x,y,z) = (-2,0,5) + t(1,4,2)
    \]
    Thus, the line passes through the point \((-2,0,5)\) and has the direction vector
    \[
      \mb{v}_{1} = \brackett{1,4,2}.
    \]
    Since the plane is two-dimensional, we need 2 independent directions within it. We got the first through our line, but we need another because there are infinitely many planes that contain the same line. Thus, we can form a second vector \(\mb{v}_{2}\) by taking the difference between the given point and a point on the line:
    \[
      \mb{v}_{2} = (5,2,1) - (-2,0,5) = \brackett{7,2,-4}.
    \]
    With \(\mb{v}_{1}\) and \(\mb{v}_{2}\), we find the normal vector:
    \[
      \mb{n} = \mb{v}_{1} \times \mb{v}_{2} = \begin{vmatrix}
        \mb{i} & \mb{j} & \mb{k} \\
        1 & 4 & 2 \\
        7 & 2 & -4
      \end{vmatrix} = (-16 - 4)\mb{i} - (-4 - 14)\mb{j} + (2 - 28)\mb{k} = -20\mb{i} + 20\mb{j} - 26\mb{k}
    \]
    Therefore, we find the scalar form to be:
    \[
      \boxed{-20(x + 2) + 18y - 26(z - 5) = 0.}
    \]
  }

  \newpage
  \item (3 points) Determine the arc length parametrization for the curve \(\vecfuc{r}{t} = 3e^{t} \sin(t)\imb + 3e^{t}\cos(t)\jmb\), where you start from \(t = 0\). 

  \sol{
    From equation 3.11 from Theorem 3.4 in the book, (and \href{https://calcworkshop.com/vector-functions/arc-length-parameterization/}{this website}) we know that we can rewrite the arc length parametrization as:
    \[
      s = \int_{0}^{t} ||\dvecfuc{r}{\tau}||\,d\tau = \int_{0}^{t} \mysqrt{\left[f'(\tau)\right]^{2} + \left[g'(\tau)\right]^{2}}\,d\tau,
    \]
    where \(f(\tau) = 3e^{\tau}\sin(\tau)\) and \(g(\tau) = 3e^{\tau}\cos(\tau)\). Thus, we find:
    \begin{align*}
      f'(\tau) &= 3e^{\tau}(\sin(\tau) + \cos(\tau)) \\
      g'(\tau) &= 3e^{\tau}(\cos(\tau) - \sin(\tau)).
    \end{align*}
    Thus, we have:
    \begin{align*}
      s &= \int_{0}^{t} \mysqrt{\left[3e^{\tau}\bigl(\sin(\tau) + \cos(\tau)\bigr)\right]^{2} + \left[3e^{\tau}\bigl(\cos(\tau) - \sin(\tau)\bigr)\right]^{2}}\,d\tau \\
      &= \int_{0}^{t} \mysqrt{9e^{2\tau}\bigl(\sin^{2}(\tau) + 2\sin(\tau)\cos(\tau) + \cos^{2}(\tau)\bigr) + 9e^{2\tau}\bigl(\cos^{2}(\tau) - 2\sin(\tau)\cos(\tau) + \sin^{2}(\tau)\bigr)}\,d\tau \\
      &= \int_{0}^{t} \mysqrt{9e^{2\tau} \left[2\bigl(\sin^{2}(\tau) + \cos^{2}(\tau)\bigr) + \bigl(2\sin(\tau)\cos(\tau) - 2\sin(\tau)\cos(\tau)\bigr) \right]} \\
      &= \int_{0}^{t} \mysqrt{9e^{2\tau} \cdot \left[2(1 + 0)\right]} \\
      &= \int_{0}^{t} 3e^{\tau}\mysqrt{2}\,d\tau \\
      &= 3\mysqrt{2}\int_{0}^{t} e^{\tau}\,d\tau \\
    \end{align*}
    With our integral in the form \(3e^{\tau}\int_{0}^{t} \mysqrt{2}\,d\tau\), we can solve it to get:
    \[
      s = 3\mysqrt{2}(e^{t} - 1).
    \]
    With \(s\), we know that \(\mb{r}(t) = \mb{r}(t(s)) = \mb{r}\), so we need to find \(t\) in terms of \(s\): 
    \begin{align*}
      s = 3\mysqrt{2}(e^{t} - 1) \\
      e^{t} = \dfrac{s}{3\mysqrt{2}} + 1 \\
      t = \ln\left(\dfrac{s}{3\mysqrt{2}} + 1\right).
    \end{align*}
    Finally, by replacing \(t\) with \(t(s)\) in the original equation, we can get the arc length parametrization:
    \[
      \boxed{\vecfuc{r}{s} = \left(\dfrac{s}{\mysqrt{2}} + 3\right)\sin\left(\ln\left(\dfrac{s}{\mysqrt{2}} + 3\right)\right)\imb + \left(\dfrac{s}{\mysqrt{2}} + 3\right)\cos\left(\ln\left(\dfrac{s}{\mysqrt{2}} + 3\right)\right)\jmb.}
    \]
  }

  \newpage
  \item (3 points) Use curvature to find the equation of the osculating circle at the planar curve \(y = x^{3} - 4x + 1\) at \(x = 1\). Then, check your answer by graphing both the curve and its circle on the same axes. [you do not need to include the graph in your work turned in -- but you should be able to tell if your work is correct.]

  \sol{
    First, we need to find the curvature of the curve at \(x = 1\). We start by finding the first and second derivatives of the function:
    \begin{align*}
      y(x) &= x^{3} - 4x + 1 \\
      y'(x) &= 3x^{2} - 4 \\
      y''(x) &= 6x.
    \end{align*}
    Then, we evaluate the point and the first and second derivatives at \(x = 1\):
    \begin{align*}
      y(1) &= 1 - 4 + 1 = -2 \\
      y'(1) &= 3(1)^{2} - 4 = -1 \\
      y''(1) &= 6(1) = 6.
    \end{align*}
    With these values, we can find the curvature:
    \[
      \kappa = \dfrac{|y''(x)|}{\bigl(1 + y'(x)^{2}\bigr)^{3/2}} = \dfrac{6}{\bigl(1 + (-1)^{2}\bigr)^{3/2}} = \dfrac{6}{2^{3/2}} = \frac{3\mysqrt{2}}{2}.
    \]
    With the curvature, we can find the radius of the osculating circle:
    \[
      R = \dfrac{1}{\kappa} = \dfrac{1}{\frac{3\mysqrt{2}}{2}} = \dfrac{2\mysqrt{2}}{6} = \dfrac{\mysqrt{2}}{3}.
    \]
    To find the center, we need the unit normal vector at \(x = 1\):
    \[
      \mb{N} = \frac{(-y,1)}{\mysqrt{1 + (y')^{2}}} = \frac{\bigl(-(-1),1\bigr)}{\mysqrt{1 + (-1)^{2}}} = \frac{(1,1)}{\mysqrt{2}}
    \]
    The center \(C\) can be found by moving our point \(P(1,-2)\) the distance \(R\) along the unit normal vector:
    \begin{align*}
      C &= P + R\mb{N} \\
      &= (1,-2) + \frac{\mysqrt{2}}{3}\frac{(1,1)}{\mysqrt{2}} \\
      &= (1,-2) + \left(\frac{1}{3},\frac{1}{3}\right) \\
      &= \left(\frac{4}{3},-\frac{5}{3}\right).
    \end{align*}
    This gives the equation for the osculating circle:
    \[
      \boxed{\left(x - \frac{4}{3}\right)^{2} + \left(y + \frac{5}{3}\right)^{2} = \frac{2}{9}.}
    \]
  }
  \newpage
  \item (3 points each) Suppose the position of some particle is given by \(\vecfuc{r}{t} = \sin(t)\imb + t\jmb + 3t\kmb\).
  \begin{enumerate}
    \item Find the velocity vector, \(\vecfuc{v}{t}\).\\

    \sol{
      \begin{align*}
        \vecfuc{v}{t} &= \dvecfuc{r}{t} = \cos(t)\imb + \jmb + 3\kmb
      \end{align*}
    }
    \item What total distance is travelled by the particle over the time period \([0, 3\pi]\)? (You can set up the necessary internal, and calculate it using your calculator up to 3 decimal places.)\\

    \sol{
      \[
        \int_{0}^{3\pi} ||\dvecfuc{r}{t}||\,dt = \int_{0}^{3\pi} \mysqrt{\cos^{2}(t) + 1 + 9}\,dt = 9.709 
      \]
    }
    \item Find the unit tangent vector \(\vecfuc{T}{t}\).\\
    
    \sol{
      \begin{align*}
        \vecfuc{T}{t} &= \dfrac{\vecfuc{v}{t}}{||\vecfuc{v}{t}||} = \dfrac{\cos(t)\imb + \jmb + 3\kmb}{\mysqrt{\cos^{2}(t) + 10}}
      \end{align*}
    }
    \item Find unit normal vector \(\vecfuc{N}{t}\).\\
    
    \sol{
      To find the unit normal vector, we need to find the derivative of the unit tangent vector. To avoid making mistakes (and making differentiating easier), lets break \(\vecfuc{T}{t}\) into separate functions \(u(t)\) and \(v(t)\):
      \[
        \vecfuc{T}{t} = \underbrace{\bigl(\cos(t)\imb + \jmb + 3\kmb\bigr)}_{u(t)} \cdot \underbrace{\left(\cos^{2}(t) + 10\right)^{-1/2}}_{v(t)}
      \]
      Differentiating \(u(t)\):
      \[
        u'(t) = -\sin(t)\imb,
      \]
      and differentiating \(v(t)\) with the chain rule:
      \[
        v'(t) = -\frac{1}{2}(\cos^{2} + 10)^{-3/2} \cdot 2\cos(t)(-\sin(t)) = \cos(t)\sin(t)(\cos^{2}(t) + 10)^{-3/2}.
      \]
      Thus, we apply the product rule for \(\dvecfuc{T}{t}\):
      \[
        \dvecfuc{T}{t} = \Bigl[-\sin(t)\,\mathbf{i}\Bigr]\bigl(\cos^2(t)+10\bigr)^{-1/2} + \Bigl[\cos(t)\,\mathbf{i} + \mathbf{j} + 3\,\mathbf{k}\Bigr] \Bigl[\cos(t)\sin(t)\,\bigl(\cos^2(t)+10\bigr)^{-3/2}\Bigr].
      \]
      Notice that both terms contain a factor of \(\sin(t)\) and a power of \(\cos^{2}(t) + 10\), so we can factor them out:
      \[
        \dvecfuc{T}{t} = \sin(t)\,\bigl(\cos^2(t)+10\bigr)^{-3/2}\left\{ -\Bigl[\cos^2(t)+10\Bigr]\mathbf{i} + \cos(t)\Bigl[\cos(t)\,\mathbf{i} + \mathbf{j} + 3\,\mathbf{k}\Bigr]\right\}.
      \]
      Inside the braces, multiply out the terms:
      \begin{align*}
        \{...\} &= -\cos^2(t)\,\mathbf{i} - 10\,\mathbf{i} + \cos^2(t)\,\mathbf{i} + \cos(t)\,\mathbf{j} + 3\cos(t)\,\mathbf{k} \\
        &= -\cos^{2}(t)\imb - 10\imb + \cos^{2}(t)\imb + \cos(t)\jmb + 3\cos(t)\kmb \\
        &= -10\imb + \cos(t)\jmb + 3\cos(t)\kmb.
      \end{align*}
      This gives us:
      \[
        \dvecfuc{T}{t} = \frac{\sin(t)}{\bigl(\cos^2(t)+10\bigr)^{3/2}} \Bigl[-10\,\mathbf{i} + \cos(t)\,\mathbf{j} + 3\cos(t)\,\mathbf{k}\Bigr].
      \]
      Now we need to find the magnitude of \(\dvecfuc{T}{t}\):
      \[
        ||\dvecfuc{T}{t}|| = \frac{|\sin(t)|}{\bigl(\cos^2(t)+10\bigr)^{3/2}} \mysqrt{(-10)^2 + \bigl(\cos(t)\bigr)^2 + \bigl(3\cos(t)\bigr)^2}.
      \]
      Simplify inside the square root and factor:
      \[
        ||\dvecfuc{T}{t}|| = \frac{|\sin(t)|}{\bigl(\cos^2(t)+10\bigr)^{3/2}} \cdot \mysqrt{10}\,\mysqrt{\cos^2(t)+10}.
      \]
      This simplifies to:
      \[
        ||\dvecfuc{T}{t}|| = \frac{|\sin(t)|\,\mysqrt{10}}{\cos^2(t)+10}.
      \]
      Finally, we can find the unit normal vector:
      \[
        \mb{N}(t) = \frac{\frac{\sin(t)}{\bigl(\cos^2(t)+10\bigr)^{3/2}} \Bigl[-10\,\mathbf{i} + \cos(t)\,\mathbf{j} + 3\cos(t)\,\mathbf{k}\Bigr]}{\frac{|\sin(t)|\,\mysqrt{10}}{\cos^2(t)+10}}.
      \]
      Notice that \(\sin(t)\) cancels out with the top-most numerator and numerator in the bottom-most denominator, and we can move the \((\cos^2(t)+10)\) term from the top numerator in the denominator to the bottom-most denominator (I hope you were able to follow that). This gives us:
      \[
        \boxed{\mb{N}(t) = \frac{-10\mb{i} + \cos(t)\jmb + 3\cos(t)\kmb}{\mysqrt{10}\mysqrt{\cos^{2}(t) + 10}}}
      \]
    }
    \newpage
    \item Find binormal vector, \(\vecfuc{B}{t}\).\\

    \sol{
      The binomial vector is the cross product of the unit tangent and unit normal vectors:
      \[
        \vecfuc{T}{t} \times \vecfuc{N}{t} = \begin{vmatrix}
          \imb & \jmb & \kmb \\
          \dfrac{\cos(t)}{\mysqrt{\cos^{2}(t) + 10}} & \dfrac{1}{\mysqrt{\cos^{2}(t) + 10}} & \dfrac{3}{\mysqrt{\cos^{2}(t) + 10}} \\
          \frac{-10}{\mysqrt{10}\mysqrt{\cos^{2}(t) + 10}} & \frac{\cos(t)}{\mysqrt{10}\mysqrt{\cos^{2}(t) + 10}} & \frac{3\cos(t)}{\mysqrt{10}\mysqrt{\cos^{2}(t) + 10}}
        \end{vmatrix}
      \]
      Thankfully, we can factor out the common term \(\frac{1}{\mysqrt{\cos^{2}(t) + 10}}\) from each vector in the cross product:
      \[
        \vecfuc{T}{t} \times \vecfuc{N}{t} = \frac{1}{\mysqrt{\cos^{2}(t) + 10}} \cdot \frac{1}{\mysqrt{\cos^{2}(t) + 10}} \begin{vmatrix}
          \imb & \jmb & \kmb \\
          \cos(t) & 1 & 3 \\
          \frac{-10}{\mysqrt{10}} & \frac{\cos(t)}{\mysqrt{10}} & \frac{3\cos(t)}{\mysqrt{10}}
        \end{vmatrix}
      \]
      Now we can find the cross product:
      \[
      \begin{aligned}
        \frac{1}{\cos^{2}(t) + 10}\bigg[&\left(1 \cdot \frac{3\cos(t)}{\mysqrt{10}}\right) - \left(\frac{\cos(t)}{\mysqrt{10}} \cdot 3\right), \\
        &\left(3 \cdot \frac{-10}{\mysqrt{10}}\right) - \left(\cos(t) \cdot \frac{3\cos(t)}{\mysqrt{10}}\right), \  \cos(t) \cdot \frac{\cos(t)}{\mysqrt{10}} - \left(1 \cdot \frac{-10}{\mysqrt{10}}\right) \bigg] 
      \end{aligned}
      \]
      We can further simply this expression to:
      \[
        \boxed{\frac{1}{\cos^{2}(t) + 10}\brackett{0, \ \frac{-30 - 3\cos^{2}(t)}{\mysqrt{10}}, \ \frac{\cos^{2}(t) + 10}{\mysqrt{10}}}}
      \]

    }
  \end{enumerate}
\end{enumerate}


\end{document}