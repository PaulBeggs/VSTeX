\documentclass[11pt]{article}

\usepackage{fix-cm}
\usepackage{tikz}
\usetikzlibrary{shapes,backgrounds,calc,patterns}
\usepackage{amsmath,amsthm} 
\usepackage{amssymb}
\usepackage[framemethod=tikz]{mdframed}
\usepackage{C:/Users/paulb/VSTeX/local/draculatheme}
\usepackage{mathrsfs}
\usepackage{changepage}
\usepackage{multicol}
\usepackage{mathtools}
\usepackage{hyperref}
\usepackage{slashed}
\usepackage{enumerate}
\usepackage{booktabs}
\usepackage{enumitem}
\usepackage{kantlipsum} 
\usepackage{tabularx}
\usepackage{array}
\usepackage{makecell}
\usepackage{pgfplots}
\pgfplotsset{compat=1.18}
\usetikzlibrary{decorations.markings}

\setlength{\parindent}{0pt}

\newmdenv[
  topline=false,
  bottomline=true,
  rightline=false,
  leftline=true,
  linewidth=1.5pt,
  linecolor=black, % default color, will be overridden in custom commands
  backgroundcolor=draculabg, % Needed for Dracula theme
  fontcolor=draculafg, % Needed for Dracula theme
  innertopmargin=0pt,
  innerbottommargin=5pt,
  innerrightmargin=10pt,
  innerleftmargin=10pt,
  leftmargin=0pt,
  rightmargin=0pt,
  skipabove=\topsep,
  skipbelow=\topsep,
]{customframedproof}

\newenvironment{proofpart}[2][black]{
    \begin{mdframed}[
        topline=false,
        bottomline=false,
        rightline=false,
        leftline=true,
        linewidth=1pt,
        linecolor=#1!40, % Custom color
        % innertopmargin=10pt,
        % innerbottommargin=10pt,
        innerleftmargin=10pt,
        innerrightmargin=10pt,
        leftmargin=0pt,
        rightmargin=0pt,
        % skipabove=\topsep,
        % skipbelow=\topsep%
    ]
    \noindent
    \begin{minipage}[t]{0.08\textwidth}%
        \textbf{#2}%
    \end{minipage}%
    \begin{minipage}[t]{0.90\textwidth}%
        \begin{adjustwidth}{0pt}{0pt}%
}{
    \end{adjustwidth}
    \end{minipage}
    \end{mdframed}
}

\newenvironment{solution}
  {\textit{Solution.}}



%%% AESTHETICS %%%
%-%-%-%-%-%-%-%-%-%-%-%-%-%-%-%-%-%-%-%-%-%-%-%-%-%-%-%-%-%-%-%-%-%-%-%-%-%-%


%%% Dimensions and Spacing %%%
\usepackage[left=0.5in,right=0.5in,top=1in,bottom=1in]{geometry}
% \usepackage{setspace}
% \linespread{1}
\usepackage{listings}

%%% Define new colors %%%
\usepackage{xcolor}
\definecolor{orangehdx}{rgb}{0.96, 0.51, 0.16}

% Normal colors
\definecolor{xred}{HTML}{BD4242}
\definecolor{xblue}{HTML}{4268BD}
\definecolor{xgreen}{HTML}{52B256}
\definecolor{xpurple}{HTML}{7F52B2}
\definecolor{xorange}{HTML}{FD9337}
\definecolor{xdotted}{HTML}{999999}
\definecolor{xgray}{HTML}{777777}
\definecolor{xcyan}{HTML}{80F5DC}
\definecolor{xpink}{HTML}{F690EA}
\definecolor{xgrayblue}{HTML}{49B095}
\definecolor{xgraycyan}{HTML}{5AA1B9}

% Dark colors
\colorlet{xdarkred}{red!85!black}
\colorlet{xdarkblue}{xblue!85!black}
\colorlet{xdarkgreen}{xgreen!85!black}
\colorlet{xdarkpurple}{xpurple!85!black}
\colorlet{xdarkorange}{xorange!85!black}
\definecolor{xdarkcyan}{HTML}{008B8B}
\colorlet{xdarkgray}{xgray!85!black}

% Very dark colors
\colorlet{xverydarkblue}{xblue!50!black}

% Document-specific colors
\colorlet{normaltextcolor}{black}
\colorlet{figtextcolor}{xblue}

% Enumerated colors
\colorlet{xcol0}{black}
\colorlet{xcol1}{xred}
\colorlet{xcol2}{xblue}
\colorlet{xcol3}{xgreen}
\colorlet{xcol4}{xpurple}
\colorlet{xcol5}{xorange}
\colorlet{xcol6}{xcyan}
\colorlet{xcol7}{xpink!75!black}

% Blue-Purple (should just used colorbrewer...)
\definecolor{xrainbow0}{HTML}{e41a1c}
\definecolor{xrainbow1}{HTML}{a24057}
\definecolor{xrainbow2}{HTML}{606692}
\definecolor{xrainbow3}{HTML}{3a85a8}
\definecolor{xrainbow4}{HTML}{42977e}
\definecolor{xrainbow5}{HTML}{4aaa54}
\definecolor{xrainbow6}{HTML}{629363}
\definecolor{xrainbow7}{HTML}{7e6e85}
\definecolor{xrainbow8}{HTML}{9c509b}
\definecolor{xrainbow9}{HTML}{c4625d}
\definecolor{xrainbow10}{HTML}{eb751f}
\definecolor{xrainbow11}{HTML}{ff9709}

%%% FIGURES %%%
\usepackage{graphicx}  
% \graphicspath{ {images/} }  
% \numberwithin{figure}{section}
\usepackage{float}
\usepackage{caption}

%%% Hyperlinks %%%
\usepackage{hyperref}
\definecolor{horange}{HTML}{f58026}
\hypersetup{
	colorlinks=true,
	linkcolor=horange,
	filecolor=horange,      
	urlcolor=horange,
}

\newcommand{\mysqrt}[1]{%
  \mathpalette\foo{#1}%
}
\newcommand{\dmysqrt}[1]{%
  \mathpalette\foodisplay{#1}%
}

\newcommand{\sol}[1]{
    \begin{customframedproof}[linecolor=orangehdx!75,]
        \begin{solution}
        #1
        \end{solution}
    \end{customframedproof}
}

% !TeX spellcheck = off
\newcommand{\foo}[2]{%
  % #1: math style, #2: content
  \sbox0{$#1\sqrt{#2}$}% Measure the size of the standard sqrt in the current style
  \begin{tikzpicture}[baseline=(sqrt.base)]
    \node[inner sep=0, outer sep=0] (sqrt) {$#1\sqrt{#2}$}; % Use the current math style
    \draw([yshift=-0.045em]sqrt.north east) -- ++(0,-0.5ex); % Draw the tick
  \end{tikzpicture}%
}
% !TeX spellcheck = off
\newcommand{\foodisplay}[2]{%
  % #1: math style, #2: content
  \sbox0{$#1\sqrt{#2}$}% Measure the size of the standard sqrt in the current style
  \begin{tikzpicture}[baseline=(sqrt.base)]
    \node[inner sep=0, outer sep=0] (sqrt) {$\displaystyle\sqrt{#2}$}; % Force displaystyle
    \draw[line width=0.4pt] ([yshift=-0.044em]sqrt.north east) -- ++(0,-0.5ex); % Draw the tick
  \end{tikzpicture}%
}

\newcommand{\barNotationT}[1]{\bigg|_{t = #1}}

\newcommand{\cyanit}[1]{\textit{\textcolor{cyan}{#1}}}

\newcommand{\brackett}[1]{\left\langle #1 \right\rangle}

\newcommand{\norm}[1]{\left\lVert \mathbf{#1}\right\rVert}

\newcommand{\imb}{\mb{i}}
\newcommand{\jmb}{\mb{j}}
\newcommand{\kmb}{\mb{k}}
\newcommand{\rmb}{\mb{r}}
\newcommand{\umb}{\mb{u}}

\newcommand{\vecfuc}[2]{\mb{#1}(#2)}
\newcommand{\dvecfuc}[2]{\mb{#1}'(#2)}
\newcommand{\normdvecfuc}[2]{||\mb{#1}'(#2)||}

\newcommand{\proj}{\text{proj}}

\newcommand{\mb}[1]{\mathbf{#1}}

% \renewcommand{\theenumi}{\arabic{enumi}} 
% \renewcommand{\labelenumi}{\theenumi.}

\title{Multivariable Calculus Practice Set V}
\author{Paul Beggs}
\date{\today}

%%% Custom Comands %%%
% Natural Numbers 
\newcommand{\N}{\ensuremath{\mathbb{N}}}

% Whole Numbers
\newcommand{\W}{\ensuremath{\mathbb{W}}}

% Integers
\newcommand{\Z}{\ensuremath{\mathbb{Z}}}

% Rational Numbers
\newcommand{\Q}{\ensuremath{\mathbb{Q}}}

% Real Numbers
\newcommand{\R}{\ensuremath{\mathbb{R}}}

% Complex Numbers
\newcommand{\C}{\ensuremath{\mathbb{C}}}

\newcommand{\I}{\ensuremath{\mathbb{I}}}

\newcommand{\p}{\partial}

\newcommand{\mbi}{\mathbf{i}}
\newcommand{\mbj}{\mathbf{j}}
\newcommand{\mbk}{\mathbf{k}}
\newcommand{\mbr}{\mathbf{r}}

\begin{document}

% \maketitle

\begin{enumerate}
    \setcounter{enumi}{1}
    \item (3 points) Determine \(\displaystyle \int_{C} y^{2} \, dx + z \, dy + x \, dz\), where \(C\) is the line segment which connects \((2,0,0)\) to \((3,4,5)\).  

    \sol{
        We can parameterize this line segment as follows:
        \[
            \rmb(t) = (2+t)\mbi + (4t)\mbj + (5t)\mbk, \quad 0 \leq t \leq 1
        \]
        We get this from taking each component of the vector function and setting \(t = 0\), to get the point \((2,0,0)\). Then, by setting \(t = 1\), we get the point \((3,4,5)\). Now, we can differentiate \(\mbr(t)\) to get \(\dvecfuc{\rmb}{t}\):
        \[
            \dvecfuc{\rmb}{t} = \left\langle \tfrac{d}{dt}\left[(2+t)\right] + \tfrac{d}{dt}\left[(4t)\right] + \tfrac{d}{dt}\left[(5t)\right] \right\rangle = \left\langle 1, 4, 5 \right\rangle.
        \]
        This gives us the following:
        \[
          dx = dt, \quad dy = 4dt, \quad dz = 5dt.
        \]
        Expressing the integral in terms of \(t\), we can solve the integral as follows:
        \begin{align*}
            \int_{C} y^{2} \, dx + z \, dy + x \, dz &= \int_{0}^{1} (4t)^{2} dt + (5t)(4dt) + (2+t)(5dt) \\
            &= \int_{0}^{1} (16t^{2} + 20t + 10 + 5t) dt \\
            &= \int_{0}^{1} (16t^{2} + 25t + 10) dt \\
            &= \left[\frac{16}{3}t^{3} + \frac{25}{2}t^{2} + 10t\right]_{0}^{1} \\
            &= \boxed{\frac{167}{6}}.
        \end{align*}
    }

    \item (3 points) Determine \(\displaystyle \int_{C} \frac{1}{x^{2} + y^{2} + z^{2}} \, ds\), where \(C\) is given by \(\langle \cos t, \sin t, t \rangle\), \(0 \leq t \leq \pi\).  

    \sol{
        First, we need to find \(ds\), which is given by finding the derivative of the vector function and taking the norm:
        \[
            ds = \normdvecfuc{\rmb}{t} \, dt = \sqrt{(-\sin t)^{2} + (\cos t)^{2} + 1^{2}} \, dt = \sqrt{1 + 1} \, dt = \sqrt{2} dt.
        \]
        Rewriting the integral in terms of \(t\), we have:
        \[
            \int_{C} \frac{1}{x^{2} + y^{2} + z^{2}} \, ds = \int_{0}^{\pi} \frac{1}{\cos^{2} t + \sin^{2} t + t^{2}} \sqrt{2} \, dt = \sqrt{2} \int_{0}^{\pi} \frac{1}{1 + t^{2}} \, dt.
        \]
        We know that \(\int \frac{1}{1 + t^{2}} dt = \tan^{-1}(t)\), so we can evaluate the integral as follows:
        \begin{align*}
            \sqrt{2} \int_{0}^{\pi} \frac{1}{1 + t^{2}} \, dt &= \sqrt{2} \left[\tan^{-1}(t)\right]_{0}^{\pi} \\
            &= \sqrt{2} \left(\tan^{-1}(\pi) - \tan^{-1}(0)\right) \\
            &= \boxed{\sqrt{2} \tan^{-1}(\pi).}
        \end{align*}
    }
    \newpage

    \item (3 points) Let \(\mb{F}(x,y) = 3x^{2}y^{2}\mbi + (2x^{3}y + 5)\mbj\). Find a scalar function \(f\) such that \(\nabla f = \mb{F}\) and use this to determine \(\displaystyle \int_{C} \mb{F} \cdot d\mb{r}\) where \(C\) is given by \(\mb{r}(t) = (t^{3} - 2t)\mbi + (t^{3} + 2t)\mbj\) for \(0 \leq t \leq 1\).  

    \sol{
        First, we need to check if \(\mb{F}\) is conservative. We can do this by checking if the mixed partials are equal:
        \[
            \frac{\partial}{\partial y}(3x^{2}y^{2}) = 6x^{2}y, \quad \frac{\partial}{\partial x}(2x^{3}y + 5) = 6x^{2}y.
        \]
        Since these are equal, we can conclude that \(\mb{F}\) is conservative. Now, we need to find a scalar function \(f\) such that \(\nabla f = \mb{F}\). We can do this by integrating the components of \(\mb{F}\):
        \begin{align*}
            f(x,y) &= \int 3x^{2}y^{2} \, dx + h(y) \\
            &= x^{3}y^{2} + h(y).
        \end{align*}
        Now, we can differentiate \(f\) with respect to \(y\) and set it equal to the second component of \(\mb{F}\):
        \begin{align*}
            \frac{\partial}{\partial y}(x^{3}y^{2} + h(y)) &= 2x^{3}y + h'(y) \\
            &= 2x^{3}y + 5.
        \end{align*}
        This gives us \(h'(y) = 5\), so we can integrate to find \(h(y)\):
        \[
            h(y) = 5y + K.
        \]
        Thus, we have:
        \[
            f(x,y) = x^{3}y^{2} + 5y + K.
        \]
        Now, we can use the Fundamental Theorem of Line Integrals to evaluate the integral:
        \begin{align*}
            \int_{C} \mb{F} \cdot d\mb{r} &= f(\mb{r}(1)) - f(\mb{r}(0)) \\
            &= f(-1,3) - f(0,0) \\
            &= \left[(-1)^{3}(3)^{2} + 5(3) + K\right] - \left[(0)^{3}(0)^{2} + 5(0) + K\right] \\
            &= \left[-9 + 15 + K\right] - \left[0 + 0 + K\right] \\
            &= -9 + 15 + K - K \\
            &= \boxed{6}.
        \end{align*}
    }

    \item (2 points) Set up, but do not evaluate the ``direct'' integral for the previous problem. Then, use your calculator to determine a numerical approximation for the integral. Did you get the same answer?  

    \sol{
        We can set up the integral as follows:
        
    }

    \item (3 points) Find the work done by the force field \(\mb{F} = x^{2}\mbi + y^{3}\mbj\) in moving an object from \((1,0)\) to \((2,2)\).   

    \sol{
        
    }

    \item (3 points) For what value(s), if any, of \(a\) is \((3x^{2}y + az)\mbi + x^{3}j + (3x + 3z^{2})\mbk\) conservative?  

    \sol{
        
    }

    \item (3 points) Find the circulation of \(\mb{F} = xy\mbi + x^{2}y^{3}\mbj\) along \(C\), where \(C\) is the counter-clockwise oriented triangle with vertices \((0,0)\), \((1,0)\), and \((1,2)\). Determine the value of this integral by working three separate line integrals.

    \sol{
        
    }

    \item (3 points) Find the flux of \(\mb{F} = xy\mbi + x^{2}y^{3}\mbj\) over \(C\), the same counter-clockwise oriented triangle with vertices \((0,0)\), \((1,0)\), and \((1,2)\) as in the previous problem (notice that the vector field is the same as well). Determine this by working three separate line integrals.  

    \sol{
        
    }
\end{enumerate}
\end{document}
