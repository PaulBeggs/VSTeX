\documentclass[11pt]{article}

\usepackage{fix-cm}
\usepackage{tikz}
\usetikzlibrary{shapes,backgrounds,calc,patterns}
\usepackage{amsmath,amsthm} 
\usepackage{amssymb}
\usepackage[framemethod=tikz]{mdframed}
% \usepackage{C:/Users/paulb/VSTeX/local/draculatheme}
\usepackage{mathrsfs}
\usepackage{changepage}
\usepackage{multicol}
\usepackage{mathtools}
\usepackage{hyperref}
\usepackage{slashed}
\usepackage{enumerate}
\usepackage{booktabs}
\usepackage{enumitem}
\usepackage{kantlipsum} 
\usepackage{tabularx}
\usepackage{array}
\usepackage{makecell}
\usepackage{pgfplots}
\pgfplotsset{compat=1.18}
\usetikzlibrary{decorations.markings}

\setlength{\parindent}{0pt}

\newmdenv[
  topline=false,
  bottomline=true,
  rightline=false,
  leftline=true,
  linewidth=1.5pt,
  linecolor=black, % default color, will be overridden in custom commands
%   backgroundcolor=draculabg, % Needed for Dracula theme
%   fontcolor=draculafg, % Needed for Dracula theme
  innertopmargin=0pt,
  innerbottommargin=5pt,
  innerrightmargin=10pt,
  innerleftmargin=10pt,
  leftmargin=0pt,
  rightmargin=0pt,
  skipabove=\topsep,
  skipbelow=\topsep,
]{customframedproof}

\newenvironment{proofpart}[2][black]{
    \begin{mdframed}[
        topline=false,
        bottomline=false,
        rightline=false,
        leftline=true,
        linewidth=1pt,
        linecolor=#1!40, % Custom color
        % innertopmargin=10pt,
        % innerbottommargin=10pt,
        innerleftmargin=10pt,
        innerrightmargin=10pt,
        leftmargin=0pt,
        rightmargin=0pt,
        % skipabove=\topsep,
        % skipbelow=\topsep%
    ]
    \noindent
    \begin{minipage}[t]{0.08\textwidth}%
        \textbf{#2}%
    \end{minipage}%
    \begin{minipage}[t]{0.90\textwidth}%
        \begin{adjustwidth}{0pt}{0pt}%
}{
    \end{adjustwidth}
    \end{minipage}
    \end{mdframed}
}

\newenvironment{solution}
  {\textit{Solution.}}



%%% AESTHETICS %%%
%-%-%-%-%-%-%-%-%-%-%-%-%-%-%-%-%-%-%-%-%-%-%-%-%-%-%-%-%-%-%-%-%-%-%-%-%-%-%


%%% Dimensions and Spacing %%%
\usepackage[left=0.5in,right=0.5in,top=1in,bottom=1in]{geometry}
% \usepackage{setspace}
% \linespread{1}
\usepackage{listings}

%%% Define new colors %%%
\usepackage{xcolor}
\definecolor{orangehdx}{rgb}{0.96, 0.51, 0.16}

% Normal colors
\definecolor{xred}{HTML}{BD4242}
\definecolor{xblue}{HTML}{4268BD}
\definecolor{xgreen}{HTML}{52B256}
\definecolor{xpurple}{HTML}{7F52B2}
\definecolor{xorange}{HTML}{FD9337}
\definecolor{xdotted}{HTML}{999999}
\definecolor{xgray}{HTML}{777777}
\definecolor{xcyan}{HTML}{80F5DC}
\definecolor{xpink}{HTML}{F690EA}
\definecolor{xgrayblue}{HTML}{49B095}
\definecolor{xgraycyan}{HTML}{5AA1B9}

% Dark colors
\colorlet{xdarkred}{red!85!black}
\colorlet{xdarkblue}{xblue!85!black}
\colorlet{xdarkgreen}{xgreen!85!black}
\colorlet{xdarkpurple}{xpurple!85!black}
\colorlet{xdarkorange}{xorange!85!black}
\definecolor{xdarkcyan}{HTML}{008B8B}
\colorlet{xdarkgray}{xgray!85!black}

% Very dark colors
\colorlet{xverydarkblue}{xblue!50!black}

% Document-specific colors
\colorlet{normaltextcolor}{black}
\colorlet{figtextcolor}{xblue}

% Enumerated colors
\colorlet{xcol0}{black}
\colorlet{xcol1}{xred}
\colorlet{xcol2}{xblue}
\colorlet{xcol3}{xgreen}
\colorlet{xcol4}{xpurple}
\colorlet{xcol5}{xorange}
\colorlet{xcol6}{xcyan}
\colorlet{xcol7}{xpink!75!black}

% Blue-Purple (should just used colorbrewer...)
\definecolor{xrainbow0}{HTML}{e41a1c}
\definecolor{xrainbow1}{HTML}{a24057}
\definecolor{xrainbow2}{HTML}{606692}
\definecolor{xrainbow3}{HTML}{3a85a8}
\definecolor{xrainbow4}{HTML}{42977e}
\definecolor{xrainbow5}{HTML}{4aaa54}
\definecolor{xrainbow6}{HTML}{629363}
\definecolor{xrainbow7}{HTML}{7e6e85}
\definecolor{xrainbow8}{HTML}{9c509b}
\definecolor{xrainbow9}{HTML}{c4625d}
\definecolor{xrainbow10}{HTML}{eb751f}
\definecolor{xrainbow11}{HTML}{ff9709}

%%% FIGURES %%%
\usepackage{graphicx}  
% \graphicspath{ {images/} }  
% \numberwithin{figure}{section}
\usepackage{float}
\usepackage{caption}

%%% Hyperlinks %%%
\usepackage{hyperref}
\definecolor{horange}{HTML}{f58026}
\hypersetup{
	colorlinks=true,
	linkcolor=horange,
	filecolor=horange,      
	urlcolor=horange,
}

\newcommand{\mysqrt}[1]{%
  \mathpalette\foo{#1}%
}
\newcommand{\dmysqrt}[1]{%
  \mathpalette\foodisplay{#1}%
}

\newcommand{\sol}[1]{
    \begin{customframedproof}[linecolor=orangehdx!75,]
        \begin{solution}
        #1
        \end{solution}
    \end{customframedproof}
}

% !TeX spellcheck = off
\newcommand{\foo}[2]{%
  % #1: math style, #2: content
  \sbox0{$#1\sqrt{#2}$}% Measure the size of the standard sqrt in the current style
  \begin{tikzpicture}[baseline=(sqrt.base)]
    \node[inner sep=0, outer sep=0] (sqrt) {$#1\sqrt{#2}$}; % Use the current math style
    \draw([yshift=-0.045em]sqrt.north east) -- ++(0,-0.5ex); % Draw the tick
  \end{tikzpicture}%
}
% !TeX spellcheck = off
\newcommand{\foodisplay}[2]{%
  % #1: math style, #2: content
  \sbox0{$#1\sqrt{#2}$}% Measure the size of the standard sqrt in the current style
  \begin{tikzpicture}[baseline=(sqrt.base)]
    \node[inner sep=0, outer sep=0] (sqrt) {$\displaystyle\sqrt{#2}$}; % Force displaystyle
    \draw[line width=0.4pt] ([yshift=-0.044em]sqrt.north east) -- ++(0,-0.5ex); % Draw the tick
  \end{tikzpicture}%
}

\newcommand{\barNotationT}[1]{\bigg|_{t = #1}}

\newcommand{\cyanit}[1]{\textit{\textcolor{cyan}{#1}}}

\newcommand{\brackett}[1]{\left\langle #1 \right\rangle}

\newcommand{\norm}[1]{\left\lVert \mathbf{#1}\right\rVert}

\newcommand{\imb}{\mb{i}}
\newcommand{\jmb}{\mb{j}}
\newcommand{\kmb}{\mb{k}}
\newcommand{\rmb}{\mb{r}}
\newcommand{\umb}{\mb{u}}

\newcommand{\vecfuc}[2]{\mb{#1}(#2)}
\newcommand{\dvecfuc}[2]{\mb{#1}'(#2)}
\newcommand{\normdvecfuc}[2]{||\mb{#1}'(#2)||}

\newcommand{\proj}{\text{proj}}

\newcommand{\mb}[1]{\mathbf{#1}}

% \renewcommand{\theenumi}{\arabic{enumi}} 
% \renewcommand{\labelenumi}{\theenumi.}

\title{Multivariable Calculus Practice Set V}
\author{Paul Beggs}
\date{\today}

%%% Custom Comands %%%
% Natural Numbers 
\newcommand{\N}{\ensuremath{\mathbb{N}}}

% Whole Numbers
\newcommand{\W}{\ensuremath{\mathbb{W}}}

% Integers
\newcommand{\Z}{\ensuremath{\mathbb{Z}}}

% Rational Numbers
\newcommand{\Q}{\ensuremath{\mathbb{Q}}}

% Real Numbers
\newcommand{\R}{\ensuremath{\mathbb{R}}}

% Complex Numbers
\newcommand{\C}{\ensuremath{\mathbb{C}}}

\newcommand{\I}{\ensuremath{\mathbb{I}}}

\newcommand{\p}{\partial}

\newcommand{\mbi}{\mathbf{i}}
\newcommand{\mbj}{\mathbf{j}}
\newcommand{\mbk}{\mathbf{k}}
\newcommand{\mbr}{\mathbf{r}}

\begin{document}

% \maketitle
\setcounter{page}{2}
\begin{enumerate}
    \setcounter{enumi}{1}
    \item (3 points) Determine \(\displaystyle \int_{C} y^{2} \, dx + z \, dy + x \, dz\), where \(C\) is the line segment which connects \((2,0,0)\) to \((3,4,5)\).  

    \sol{
        We can parameterize this line segment as follows:
        \[
            \rmb(t) = (2+t)\mbi + (4t)\mbj + (5t)\mbk, \quad 0 \leq t \leq 1
        \]
        We get this from taking each component of the vector function and setting \(t = 0\), to get the point \((2,0,0)\). Then, by setting \(t = 1\), we get the point \((3,4,5)\). Now, we can differentiate \(\mbr(t)\) to get \(\dvecfuc{\rmb}{t}\):
        \[
            \dvecfuc{\rmb}{t} = \left\langle \tfrac{d}{dt}\left[(2+t)\right] + \tfrac{d}{dt}\left[(4t)\right] + \tfrac{d}{dt}\left[(5t)\right] \right\rangle = \left\langle 1, 4, 5 \right\rangle.
        \]
        This gives us the following:
        \[
          dx = dt, \quad dy = 4dt, \quad dz = 5dt.
        \]
        Expressing the integral in terms of \(t\), we can solve the integral as follows:
        \begin{align*}
            \int_{C} y^{2} \, dx + z \, dy + x \, dz &= \int_{0}^{1} (4t)^{2} dt + (5t)(4dt) + (2+t)(5dt) \\
            &= \int_{0}^{1} (16t^{2} + 20t + 10 + 5t) dt \\
            &= \int_{0}^{1} (16t^{2} + 25t + 10) dt \\
            &= \left[\frac{16}{3}t^{3} + \frac{25}{2}t^{2} + 10t\right]_{0}^{1} \\
            &= \boxed{\frac{167}{6}}.
        \end{align*}
    }

    \item (3 points) Determine \(\displaystyle \int_{C} \frac{1}{x^{2} + y^{2} + z^{2}} \, ds\), where \(C\) is given by \(\left \langle \cos t, \sin t, t \right \rangle\), \(0 \leq t \leq \pi\).  

    \sol{
        First, we need to find \(ds\), which is given by finding the derivative of the vector function and taking the norm:
        \[
            ds = \normdvecfuc{\rmb}{t} \, dt = \sqrt{(-\sin t)^{2} + (\cos t)^{2} + 1^{2}} \, dt = \sqrt{1 + 1} \, dt = \sqrt{2} dt.
        \]
        Rewriting the integral in terms of \(t\), we have:
        \[
            \int_{C} \frac{1}{x^{2} + y^{2} + z^{2}} \, ds = \int_{0}^{\pi} \frac{1}{\cos^{2} t + \sin^{2} t + t^{2}} \sqrt{2} \, dt = \sqrt{2} \int_{0}^{\pi} \frac{1}{1 + t^{2}} \, dt.
        \]
        We know that \(\int \frac{1}{1 + t^{2}} dt = \tan^{-1}(t)\), so we can evaluate the integral as follows:
        \begin{align*}
            \sqrt{2} \int_{0}^{\pi} \frac{1}{1 + t^{2}} \, dt &= \sqrt{2} \left[\tan^{-1}(t)\right]_{0}^{\pi} \\
            &= \sqrt{2} \left(\tan^{-1}(\pi) - \tan^{-1}(0)\right) \\
            &= \boxed{\sqrt{2} \tan^{-1}(\pi).}
        \end{align*}
    }
    \newpage

    \item (3 points) Let \(\mb{F}(x,y) = 3x^{2}y^{2}\mbi + (2x^{3}y + 5)\mbj\). Find a scalar function \(f\) such that \(\nabla f = \mb{F}\) and use this to determine \(\displaystyle \int_{C} \mb{F} \cdot d\mb{r}\) where \(C\) is given by \(\mb{r}(t) = (t^{3} - 2t)\mbi + (t^{3} + 2t)\mbj\) for \(0 \leq t \leq 1\).  

    \sol{
        First, we need to check if \(\mb{F}\) is conservative. We can do this by checking if the mixed partials are equal:
        \[
            \frac{\partial}{\partial y}(3x^{2}y^{2}) = 6x^{2}y, \quad \frac{\partial}{\partial x}(2x^{3}y + 5) = 6x^{2}y.
        \]
        Since these are equal, we can conclude that \(\mb{F}\) is conservative. Now, we need to find a scalar function \(f\) such that \(\nabla f = \mb{F}\). We can do this by integrating the components of \(\mb{F}\):
        \begin{align*}
            f(x,y) &= \int 3x^{2}y^{2} \, dx + h(y) \\
            &= x^{3}y^{2} + h(y).
        \end{align*}
        Then, we can differentiate \(f\) with respect to \(y\) and set it equal to the second component of \(\mb{F}\):
        \[
            \tfrac{\partial}{\partial y}(x^{3}y^{2} + h(y)) = 2x^{3}y + h'(y) \quad \Rightarrow \quad 2x^{3}y + h'(y) = 2x^{3}y + 5 \quad \Rightarrow \quad h'(y) = 5. 
        \]
        This gives us \(h'(y) = 5\), so we can integrate to find \(h(y)\):
        \[
            h(y) = 5y + K.
        \]
        Thus, we have:
        \[
            f(x,y) = x^{3}y^{2} + 5y + K.
        \]
        Finally, we can use the Fundamental Theorem of Line Integrals to evaluate the integral:
        \begin{align*}
            \int_{C} \mb{F} \cdot d\mb{r} &= f(\mb{r}(1)) - f(\mb{r}(0)) \\
            &= f(-1,3) - f(0,0) \\
            &= \left[(-1)^{3}(3)^{2} + 5(3) + K\right] - \left[(0)^{3}(0)^{2} + 5(0) + K\right] \\
            &= \left[-9 + 15 + K\right] - \left[0 + 0 + K\right] \\
            &= -9 + 15 + K - K \\
            &= \boxed{6}.
        \end{align*}
    }

    \item (2 points) Set up, but do not evaluate the ``direct'' integral for the previous problem. Then, use your calculator to determine a numerical approximation for the integral. Did you get the same answer?  

    \sol{
        To set up the integral, we must find 2 values: \(\mb{r}'(t)\) and \(\mb{F}(\mb{r}(t))\). First, we can find \(\mb{r}'(t)\):
        \[
            \mb{r}'(t) = \tfrac{d}{dt}\left\langle t^{3} - 2t, t^{3} + 2t \right\rangle = \left\langle 3t^{2} - 2, 3t^{2} + 2 \right\rangle.
        \]
        Then, we can find \(\mb{F}(\mb{r}(t))\): 
        \[
            \mb{F}(\mb{r}(t)) = \mb{F}(t^{3} - 2t, t^{3} + 2t) = \left( 3(t^{3} - 2t)^{2}(t^{3} + 2t)^{2}, \, 2(t^{3} - 2t)^{3}(t^{3} + 2t) \right).
        \]
        Taking the dot product of \(\mb{F}(\mb{r}(t))\) and \(\mb{r}'(t)\), and integrating from \(0\) to \(1\), we have:
        \[
            \int_{0}^{1} \mb{F}(\mb{r}(t)) \cdot \mb{r}'(t) \, dt = \int_{0}^{1} \bigl[ 3(t^{3} - 2t)^{2}(t^{3} + 2t)^{2} \bigr](3t^{2} - 2) + \bigl[ 2(t^{3} - 2t)^{3}(t^{3} + 2t) \bigr](3t^{2} + 2) \, dt = \boxed{6},
        \]
        which is the same answer we got in the previous problem. Thus, we can conclude that the two methods yield the same result.
    }

    \item (3 points) Find the work done by the force field \(\mb{F} = x^{2}\mbi + y^{3}\mbj\) in moving an object from \((1,0)\) to \((2,2)\).   

    \sol{
        From \((1,0)\) to \((2,2)\), we can parameterize and find its derivative for the line segment as follows:
        \[
            \mb{r}(t) = \langle 1 + t, 2t \rangle \quad 0 \leq t \leq 1 \quad \Rightarrow \quad \mb{r}'(t) = \langle 1, 2 \rangle.
        \]
        Now, we can find \(\mb{F}(\mb{r}(t))\) and its dot product with \(\mb{r}'(t)\) as follows:
        \[
            \mb{F}(\mb{r}(t)) = \mb{F}(1 + t, 2t) = \langle (1 + t)^{2}, 8t^{3} \rangle \quad \Rightarrow \quad \mb{F}(\mb{r}(t)) \cdot \mb{r}'(t) = (1 + t)^{2} + 16t^{3}.
        \]
        Integrating, we have:
        \begin{align*}
            W &= \int_{0}^{1} \mb{F}(\mb{r}(t)) \cdot \mb{r}'(t) \, dt \\
            &= \int_{0}^{1} ((1 + t)^{2} + 16t^{3}) \, dt \\
            &= \left[ \tfrac{1}{3}(1 + t)^{3} + 4t^{4} \right]_{0}^{1} \\
            &= \boxed{\frac{19}{3}}.
        \end{align*}
    }

    \item (3 points) For what value(s), if any, of \(a\) is \((3x^{2}y + az)\mbi + x^{3}j + (3x + 3z^{2})\mbk\) conservative?  

    \sol{
        From Section 1, we know that for a 3-dimensional vector field to be conservative, the mixed partials must be equal. Thus, we must find each of the following:
        \[
            \frac{\partial P}{\partial y} = \frac{\partial Q}{\partial x}, \quad \frac{\partial Q}{\partial z} = \frac{\partial R}{\partial y}, \quad \frac{\partial R}{\partial x} = \frac{\partial P}{\partial z},
        \]
        where \(P = 3x^{2}y + az\), \(Q = x^{3}\), and \(R = 3x + 3z^{2}\). Starting with the first equality, we have:
        \[
            \tfrac{\partial}{\partial y}(3x^{2}y + az) = 3x^{2}, \quad \tfrac{\partial}{\partial x}(x^{3}) = 3x^{2}.
        \]
        Since these are equal, we can move on to the second equality:
        \[
            \tfrac{\partial}{\partial z}(x^{3}) = 0, \quad \tfrac{\partial}{\partial y}(3x + 3z^{2}) = 0.
        \]
        Since these are equal, we can move on to the last equality:
        \[
            \tfrac{\partial}{\partial x}(3x + 3z^{2}) = 3, \quad \tfrac{\partial}{\partial z}(3x^{2}y + az) = a.
        \]
        Thus, the only value of \(a\) for which the vector field is conservative is \(\boxed{a = 3}\).
    }
\newpage
    \item (3 points) Find the circulation of \(\mb{F} = xy\mbi + x^{2}y^{3}\mbj\) along \(C\), where \(C\) is the counter-clockwise oriented triangle with vertices \((0,0)\), \((1,0)\), and \((1,2)\). Determine the value of this integral by working three separate line integrals.

    \sol{
        \begin{center}
        
          \begin{tikzpicture}
            \begin{axis}
                [
                    axis lines=middle, 
                    xlabel=$x$,
                    ylabel=$y$, 
                    xmin=-0.5, 
                    xmax=1.5, 
                    ymin=-0.5, 
                    ymax=2.5, 
                    xtick={0,1}, 
                    ytick={0,1,2}, 
                    minor tick num=1, 
                    every axis label/.style={font=\small}, 
                    every axis label/.append style={font=\normalsize}
                ]
                \node[below] at (axis cs:0.5,0) {\(C_{1}\)};
                \node[right] at (axis cs:1,1) {\(C_{2}\)};
                \node[above] at (axis cs:0.4,1) {\(C_{3}\)};

                % Arrows
                \plot[
                    smooth,
                    color=horange,
                    thick,
                    decoration={
                        markings,
                        mark=at position 0.5 with {\arrow[scale=1.5,horange]{stealth}},
                    },
                    postaction=decorate
                ] coordinates {(0,0) (1,0)};
                \plot[
                    smooth,
                    color=horange,
                    thick,
                    decoration={
                        markings,
                        mark=at position 0.5 with {\arrow[scale=1.5,horange]{stealth}},
                    },
                    postaction=decorate
                ] coordinates {(1,0) (1,2)};
                \plot[
                    color=horange,
                    thick,
                    decoration={
                        markings,
                        mark=at position 0.5 with {\arrow[scale=1.5,horange]{stealth}},
                    },
                    postaction=decorate
                ] coordinates {(1,2) (0,0)};
            \end{axis}
          \end{tikzpicture}
        \end{center}

        We can parameterize the three line segments of the triangle as follows:
        \begin{align*}
            C_{1} &: \mb{r}_{1}(t) = \left \langle t, 0 \right \rangle, \quad 0 \leq t \leq 1 \\
            C_{2} &: \mb{r}_{2}(t) = \left \langle 1, 2t \right \rangle, \quad 0 \leq t \leq 1 \\
            C_{3} &: \mb{r}_{3}(t) = \left \langle 1 - t, 2 - 2t \right \rangle, \quad 0 \leq t \leq 1.
        \end{align*}
        Now that we have our segments parameterized, we can find the line integrals for each segment. Remember from Problem 5 that for each integral, we need to find \(\mb{F}(\mb{r}(t))\) and \(\mb{r}'(t)\). We can do this as follows:
        \begin{itemize}
            \item \textbf{For \(\displaystyle \int_{C_{3}} \mb{F} \cdot d\mb{r}\):}
            \[
                \mb{F}(\mb{r}_{1}(t)) = \mb{F}(t,0) = \left \langle 0, 0 \right \rangle \quad \text{and} \quad \mb{r}'_{1}(t) = \left \langle 1, 0 \right \rangle.
            \]
            Thus, we have:
            \[
                \int_{C_{1}} \mb{F} \cdot d\mb{r} = \int_{0}^{1} \mb{F}(\mb{r}_{1}(t)) \cdot \mb{r}'_{1}(t) = \int_{0}^{1} \left \langle 0, 0 \right \rangle \cdot \left \langle 1, 0 \right \rangle \, dt = 0.
            \]
            \item \textbf{For \(\displaystyle \int_{C_{2}} \mb{F} \cdot d\mb{r}\):}
            \[
                \mb{F}(\mb{r}_{2}(t)) = \mb{F}(1,2t) = \left \langle 2t, 8t^{3} \right \rangle \quad \text{and} \quad \mb{r}'_{2}(t) = \left \langle 0, 2 \right \rangle.
            \]
            Substituting these into the integral, we have:
            \begin{align*}
                \int_{C_{2}} \mb{F} \cdot d\mb{r} &= \int_{0}^{1} (2t, 8t^{3}) \cdot \langle 0, 2 \rangle \, dt \\
                &= \int_{0}^{1} 16t^{3} \, dt \\
                &= \left[4t^{4}\right]_{0}^{1} \\
                &= 4.
            \end{align*}
            \item \textbf{For \(\displaystyle \int_{C_{1}} \mb{F} \cdot d\mb{r}\):} \\
            
            First, for \(\mb{F}(\mb{r}_{3}(t))\), we could find \(\left \langle (1 - t)(2 - 2t), \, (1 - t)^{2}(2 - 2t)^{3} \right \rangle\) by expanding the polynomial, but that would be a bit tedious. Instead, notice that we can factor out a 2 from the first component and an \(8\) from the second component. This gives us \(2(1 - t)^{2}\) and \((1 - t)^{2}8(1 - t)^{3} = 8(1 - t)^{5}\). Thus, we can continue below:
            \[
                \mb{F}(\mb{r}_{3}(t)) = \left \langle 2(1 - t)^{2}, 8(1 - t)^{5} \right \rangle \quad \text{and} \quad \mb{r}'_{3}(t) = \left \langle -1, -2 \right \rangle.
            \]
            Now, substituting these into the integral, we have:
            \begin{align*}
                \int_{C_{3}} \mb{F} \cdot d\mb{r} &= \int_{0}^{1} \left \langle 2(1 - t)^{2}, 8(1 - t)^{5} \right \rangle \cdot \langle -1, -2 \rangle \, dt \\
                &= \int_{0}^{1} \left[-2(1 - t)^{2} - 16(1 - t)^{5}\right] \, dt. 
            \intertext{Using the substitution \(u = 1 - t\) (so that \(u = -dt\) and our bounds change to the interval 1 to 0), we get:}
                &= \int_{1}^{0} \left[-2u^{2} - 16u^{5}\right] (-du) \\
                &= \int_{1}^{0} \left[2u^{2} + 16u^{5}\right] \, du. 
            \intertext{By splitting this into two integrals, we have:}
                &= \int_{1}^{0} 2u^{2} \, du + \int_{1}^{0} 16u^{5} \, du \\
                &= \left[\frac{2}{3}u^{3}\right]_{1}^{0} + \left[\frac{16}{6}u^{6}\right]_{1}^{0} \\
                &= \left(0 - \frac{2}{3}\right) + \left(0 - \frac{8}{3}\right) \\
                &= -\frac{10}{3}.
            \end{align*}
            \item \textbf{Putting it all together:} \\
            Now that we have all three integrals, we can add them together to find the total circulation:
            \begin{align*}
                \oint_{C} \mb{F} \cdot d\mb{r} &= \int_{C_{1}} \mb{F} \cdot d\mb{r} + \int_{C_{2}} \mb{F} \cdot d\mb{r} + \int_{C_{3}} \mb{F} \cdot d\mb{r} \\
                &= \boxed{\frac{2}{3}}.
            \end{align*}
        \end{itemize}
    }
\newpage
    \item (3 points) Find the flux of \(\mb{F} = xy\mbi + x^{2}y^{3}\mbj\) over \(C\), the same counter-clockwise oriented triangle with vertices \((0,0)\), \((1,0)\), and \((1,2)\) as in the previous problem (notice that the vector field is the same as well). Determine this by working three separate line integrals.  

    \sol{
        To find the flux of \(\mb{F}\) over \(C\), we can use the following formula:
        \[
            \int_{C} \mb{F} \cdot \mb{N} \, ds = \int_{C} \mb{F} \cdot \mb{n}(t) \, dt = \int_{a}^{b} \mb{F}(\mb{r}(t)) \cdot \langle y'(t), -x'(t) \rangle \, dt.
        \]
        We can reuse the parameterizations from the previous problem, but we need to find \(\mb{n}(t)\) for each segment and then integrate each one. Thus:
        \begin{itemize}
            \item \textbf{For \(C_{1}\):}
            Our tangent vector is \(\mb{r}'_{1}(t) = \left \langle 1, 0 \right \rangle\). Using the formula above, we see that \(\mb{n}_{1}(t) = \left \langle 0, -1 \right \rangle\).
            Thus, we can evaluate the integral:
            \[
                \int_{C_{1}} \mb{F}(\mb{r}_{1}(t)) \cdot \mb{n}_{1}(t) \, dt = \int_{0}^{1} \left \langle 0, 0 \right \rangle \cdot \left \langle 0, -1 \right \rangle \, dt = 0.
            \]            
            \item \textbf{For \(C_{2}\):}
            The tangent vector is \(\mb{r}'_{2}(t) = \left \langle 0, 2 \right \rangle\). Hence, the normal vector is \(\mb{n}_{2}(t) = \left \langle 2, 0 \right \rangle\).
            Now, we can evaluate the integral as follows:
            \[
                \int_{C_{2}} \mb{F}(\mb{r}_{2}(t)) \cdot \mb{n}_{2}(t) \, dt = \int_{0}^{1} \left \langle 2t, 8t^{3} \right \rangle \cdot \left \langle 2, 0 \right \rangle \, dt = \int_{0}^{1} 4t \, dt = \left[2t^{2}\right]_{0}^{1} = 2.
            \]
            \item \textbf{For \(C_{3}\):}
            From the tangent vector, we know the normal vector is \(\mb{n}_{3}(t) = \left \langle -2, 1 \right \rangle\).
            Evaluating the integral, we see:
            \begin{align*}
                \int_{C_{3}} \mb{F}(\mb{r}_{3}(t)) \cdot \mb{n}_{3}(t) \, dt &= \int_{0}^{1} \left \langle 2(1 - t)^{2}, 8(1 - t)^{5} \right \rangle \cdot \left \langle -2, 1 \right \rangle \\
                &= \int_{0}^{1} (-4(1 - t)^{2} + 8(1 - t)^{5}) \, dt \\
                &= -4\int_{0}^{1} (1 - t)^{2} \, dt + 8\int_{0}^{1} (1 - t)^{5} \, dt \\
                &= -4\left[\frac{(1 - t)^{3}}{3}\right]_{0}^{1} + 8\left[\frac{(1 - t)^{6}}{6}\right]_{0}^{1} \\
                &= -4\left[0 - \frac{1}{3}\right] + 8\left[0 - \frac{1}{6}\right] \\
                &= \frac{4}{3} - \frac{4}{3} = 0.
            \end{align*}
            \item \textbf{Putting it all together:} \\
            Now that we have all three integrals, we can add them together to find the total flux:
            \begin{align*}
                \oint_{C} \mb{F} \cdot \mb{N} \, ds &= \int_{C_{1}} \mb{F}(\mb{r}_{1}(t)) \cdot \mb{n}_{1}(t) \, dt + \int_{C_{2}} \mb{F}(\mb{r}_{2}(t)) \cdot \mb{n}_{2}(t) \, dt + \int_{C_{3}} \mb{F}(\mb{r}_{3}(t)) \cdot \mb{n}_{3}(t) \, dt \\
                &= \boxed{2}.
            \end{align*}
        \end{itemize}
    }
\end{enumerate}
\end{document}
