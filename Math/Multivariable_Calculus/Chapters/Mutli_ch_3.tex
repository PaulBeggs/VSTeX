\section{Vector Valued Functions and Space Curves}

\subsection{Vector-Valued Functions}

Recall that a \cyanit{function} \(f\) from a domain \(D\) to codomain \(E\) is a rule which assigns a single element of \(E\) to each element of \(D\). \\

If each of \(f_{1},f_{2},\ldots,f_{n} \colon \R \to \R\) is a function we can then define the \cyanit{vector-valued function} \(\mathbf{r} \colon \R \to \R^{n}\) by
\[
    \mathbf{r}(t) = \langle f_{1}(t),f_{2}(t),\ldots,f_{n}(t) \rangle
\]\
\begin{itemize}
    \item When \(n = 2\), we might write \(\mb{r} = \brackett{f(t),g(t)} = f(t)\hat{\imath} + g(t)\hat{\jmath}\),
    \item and when \(n = 3\), we might write \(\mb{r} = \brackett{f(t),g(t),h(t)} = f(t)\hat{\imath} + g(t)\hat{\jmath} + h(t)\hat{k}\).
\end{itemize}

\subsection{Curves}

A \cyanit{plane curve} is the set of points satisfying the parameterized curve \(\mb{r} \colon \R \to \R^{2}\) over a given domain and for a given function \(\mb{r}\). \\

A \cyanit{space curve} is the set of points satisfying the parameterized curve \(\mb{r} \colon \R \to \R^{3}\) over a given domain and for a given function \(\mb{r}\). \\

We will call these \cyanit{vector parameterizations} of the curve.

\subsection{Limits}

\subsubsection{Formal Definition of a Limit}

Suppose that \(\mb{r} \colon \R \to \R^{n}\) is a vector-valued function and that \(a \in \R\), though perhaps nto in the domain of \(\mb{r}\). \\

If there exists a vector \(\mb{L} \in \R^{n}\) such that for each choice of \(\epsilon > 0\) there exists \(\delta >0\) such that whenever \(t \in \R\), \(t \ne a\) and \(|t - a| < \delta\), then 
\[
    |\mb{r}(t) - \mb{L}| < \epsilon,
\]
then we say that \(\mb{r}\) has \cyanit{limit} \(\mb{L}\) as \(t\) approaches \(a\), and we write
\[
    \lim_{t \to a} \mb{r}(t) = \mb{L}.
\]
That is, if we want the output from \(\mb{r}\) to be close to \(\mb{L}\), we can always choose inputs close to \(a\) to make that occur.
