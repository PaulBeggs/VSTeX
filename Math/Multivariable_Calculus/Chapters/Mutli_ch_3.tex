\section{Vector Valued Functions and Space Curves}

\subsection{Vector-Valued Functions}

Recall that a \cyanit{function} \(f\) from a domain \(D\) to codomain \(E\) is a rule which assigns a single element of \(E\) to each element of \(D\). \\

If each of \(f_{1},f_{2},\ldots,f_{n} \colon \R \to \R\) is a function we can then define the \cyanit{vector-valued function} \(\mathbf{r} \colon \R \to \R^{n}\) by
\[
    \mathbf{r}(t) = \langle f_{1}(t),f_{2}(t),\ldots,f_{n}(t) \rangle
\]\
\begin{itemize}
    \item When \(n = 2\), we might write \(\mb{r} = \brackett{f(t),g(t)} = f(t)\hat{\imath} + g(t)\hat{\jmath}\),
    \item and when \(n = 3\), we might write \(\mb{r} = \brackett{f(t),g(t),h(t)} = f(t)\hat{\imath} + g(t)\hat{\jmath} + h(t)\hat{k}\).
\end{itemize}

\subsection{Curves}

A \cyanit{plane curve} is the set of points satisfying the parameterized curve \(\mb{r} \colon \R \to \R^{2}\) over a given domain and for a given function \(\mb{r}\). \\

A \cyanit{space curve} is the set of points satisfying the parameterized curve \(\mb{r} \colon \R \to \R^{3}\) over a given domain and for a given function \(\mb{r}\). \\

We will call these \cyanit{vector parameterizations} of the curve.

\subsection{Limits}

\subsubsection{Formal Definition of a Limit}

Suppose that \(\mb{r} \colon \R \to \R^{n}\) is a vector-valued function and that \(a \in \R\), though perhaps nto in the domain of \(\mb{r}\). \\

If there exists a vector \(\mb{L} \in \R^{n}\) such that for each choice of \(\epsilon > 0\) there exists \(\delta >0\) such that whenever \(t \in \R\), \(t \ne a\) and \(|t - a| < \delta\), then 
\[
    |\mb{r}(t) - \mb{L}| < \epsilon,
\]
then we say that \(\mb{r}\) has \cyanit{limit} \(\mb{L}\) as \(t\) approaches \(a\), and we write
\[
    \lim_{t \to a} \mb{r}(t) = \mb{L}.
\]
That is, if we want the output from \(\mb{r}\) to be close to \(\mb{L}\), we can always choose inputs close to \(a\) to make that occur.


\newcommand{\vecfuc}[2]{\mb{#1}(#2)}
\newcommand{\dvecfuc}[2]{\mb{#1}'(#2)}
\newcommand{\normdvecfuc}[2]{||\mb{#1}'(#2)||}

\section{Arc Length and Curvature}

\subsection{The Unit Normal Vector}

For a curve \(C\) defined by \(\mb{r}\) in \(\R^{3}\), we have the \cyanit{unit normal vector}, \(\mb{N}(t)\), which is defined by 
\[
    \vecfuc{N}{t} = \frac{\dvecfuc{T}{t}}{||\dvecfuc{T}{t}||}.
\]
The normal vector points orthogonally to the tangent vector; it points in the direction the curve is turning. \\

We show that \(\mb{T}\) and \(\mb{N}\) are orthogonal:
\begin{align*}
    \mb{T} \cdot \mb{N} &= \vecfuc{T}{t} \cdot \frac{\dvecfuc{T}{t}}{||\dvecfuc{T}{t}||} \\
    &= \frac{1}{\normdvecfuc{T}{t}} \vecfuc{T}{t} \cdot \dvecfuc{T}{t} \\
    &= \frac{1}{\normdvecfuc{T}{t}} \cdot 0 \\
    &= 0.
\end{align*}
The last step is true since \(\mb{T}\) is a \textit{unit} vector, and so that \(||\mb{T}||\), and from our rules for differentiation, if a vector \(\mb{T}\) has a constant magnitude, then \(\mb{T} \cdot \mb{T}' = 0\).

\section{The Binormal Vector}

The \cyanit{binormal vector}, denoted by \(\mb{B}\), is defined by
\[
    \vecfuc{B}{t} = \vecfuc{T}{t} \times \vecfuc{N}{t}.
\]
Since \(\mb{T}\) and \(\mb{N}\) are each unit vectors which are orthogonal, by its definition, \(\mb{B}\) is also a unit vector which is orthogonal to both \(\mb{T}\) and \(\mb{N}\). \\

You can think of \(\mb{T}\) and \(\mb{N}\) as defining the instantaneous plane of motion of a particle. Then, \(\mb{B}\) is the noraml to that plane. If \(\mb{B}\) is constant, then the particle stays in a single plane, otherwise, 


((FINISH THE UNIT NORMAL VECTOR HERE FIRST))


\subsection{Arc Length}

If \(\mb{r}(t)\) defines a smooth curve in \(\R^{n}\), then the \cyanit{arc length} of the curve from \(t = a\) to \(t = b\) is given by
\[
    \int_{a}^{b} |\mb{r}'(t)| \, dt.
\]
That is, we are just adding up the ``speed'' of the curve to find its length (i.e., its distance).

\subsubsection{Arc Length Parameterization} 

We can define the \cyanit{arc length parameterization} of a curve \(C\) by:
\begin{itemize}
    \item Define the arc length \(s(t) = \int_{a}^{t}||\mb{r}'(t)|| \ dt\).
    \item Solving, if possible, the resulting expression for \(t\) as a function of \(s\).
    \item Rewriting \(\mb{r}(t) = \mb{r}(t(s)) = \mb{r}\), so that the curve is written as a function of its length, from a given starting point.
    \item This is useful since there are many possible parameterizations of a given curve, but only a single arc length parameterization.
\end{itemize}
As you may recall from Calculus II, arc length can only be explicitly worked out for carefully selected problems. The same applies here.

\section{Curvature}

We define the \cyanit{curvature}, denoted by \(\kappa\), for a smooth curve given by \(\mb{r}(s)\) as 
\[
    \kappa = ||\frac{d\mb{T}}{ds}||,
\]
that is, how fast is the unit tangent vector changing, relative to the length of the curve itself. We use \(s\) because we want our answer to be independent of the parameterization. \\

However, though this formula is the easiest to reason with, it is in practice not useful in most cases. You would have to find the arc-length parameterization, and then find the unit tangent vector from that.  

\subsection{Calculating Curvature}

\begin{itemize}
    \item For all \(\mb{r} \colon \ \kappa = \frac{||\mb{T}'(t)||}{||\mb{r}'(t)||}\).
    \item For \(\R^{3}\): \(\kappa = \frac{||\mb{r}'(t) \times \mb{r}''(t)||}{||\mb{r}'(t)||^{3}}\).
    \item If \(y = f(x)\): \(\kappa = \frac{|y''(x)|}{[1 + (y'(x)^{2})]^{3/2}}\)
\end{itemize}

\subsection{Curvature of a Circle}

We use each of the three equations to determine the curvature of a circle of radius \(a\).

\subsubsection{Parameterized Circle}

A circle of radius \(a\) is parameterized by \(\mb{r}(t) = a \cos(t)\mb{i} + a \sin(t)\mb{j}\). Thus, \(\mb{r}'(5) = -a \sin(t)\mb{i} + a \sin(t)\mb{j}\), so that \(||\mb{r}'(t)|| = a\). So, using the definition above, we get:
\[
    \mb{T}(t) = \frac{\mb{r}'(t)}{||\mb{r}'(t)||} = \frac{-a \sin(t)\mb{i} + a \sin(t)\mb{j}}{a} = -\sin(t)\mb{i} + \cos(t)\mb{j}
\]
and therefore
\[
    \mb{T}'(t) = -\cos(t)\mb{i} - \sin(t)\mb{j}
\]
so that \(||\mb{T}(t)|| = 1\). Finally, we see that 
\[
    \kappa = \frac{||\mb{T}'(t)||}{||\mb{r}'(t)||} = \frac{1}{a}.
\]
Therefore, a circle has constant curvature, and in fact the curvature is just the reciprocal of the radius. This should make some intuitive sense, as the radius gets smaller, the curvature will increase and vice-versa. Notice also that the curvature of a line is 0 -- in some sense, a line is a circle, but with infinite radius!

\subsubsection{Curvature of a Circle in Three Dimensions}

We can also use the second formula, again with a circle, noting that a circle of radius \(a\) is parameterized by \(\mb{r} = a \cos(t)\mb{i} + a \sin(t)\mb{j} + 0\mb{k}\) so that \(\mb{r}'(t) = -a \sin(t)\mb{i} + a \cos(t)\mb{j} + 0\mb{k}\) so that \(||\mb{r}'(t)|| = a\). Thus, \(\mb{r}'' = -a \cos(t)\mb{i} - a \sin(t)\mb{j}\), so that 
\[
    \mb{r}'(t) \times \mb{r}''(t) = \begin{vmatrix}
        \mb{i} & \mb{j} & \mb{k} \\
        -a \sin(t) & a \cos(t) & 0 \\
        -a \cos(t) & -a \sin(t) & 0
    \end{vmatrix} = a^{2}\mb{k}.
\] 
Then, 
\[
    \kappa = \frac{||\mb{r}'(t) \times \mb{r}''(t)||}{||\mb{r}'(t)||^{3}} = \frac{a^{2}}{a^{3}} = \frac{1}{a}.
\]
\subsubsection{Curvature as a Function}

We can also write (the top half)