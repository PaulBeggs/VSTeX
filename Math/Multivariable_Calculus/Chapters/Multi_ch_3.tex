\section{Vector Valued Functions and Space Curves}

\subsection{Vector-Valued Functions}

Recall that a \cyanit{function} \(f\) from a domain \(D\) to codomain \(E\) is a rule which assigns a single element of \(E\) to each element of \(D\). \\

If each of \(f_{1},f_{2},\ldots,f_{n} \colon \R \to \R\) is a function we can then define the \cyanit{vector-valued function} \(\mathbf{r} \colon \R \to \R^{n}\) by
\[
    \mathbf{r}(t) = \langle f_{1}(t),f_{2}(t),\ldots,f_{n}(t) \rangle
\]\
\begin{itemize}
    \item When \(n = 2\), we might write \(\mb{r} = \brackett{f(t),g(t)} = f(t)\hat{\imath} + g(t)\hat{\jmath}\),
    \item and when \(n = 3\), we might write \(\mb{r} = \brackett{f(t),g(t),h(t)} = f(t)\hat{\imath} + g(t)\hat{\jmath} + h(t)\hat{k}\).
\end{itemize}

\subsection{Curves}

A \cyanit{plane curve} is the set of points satisfying the parameterized curve \(\mb{r} \colon \R \to \R^{2}\) over a given domain and for a given function \(\mb{r}\). \\

A \cyanit{space curve} is the set of points satisfying the parameterized curve \(\mb{r} \colon \R \to \R^{3}\) over a given domain and for a given function \(\mb{r}\). \\

We will call these \cyanit{vector parameterizations} of the curve.

\subsection{Limits}

\subsubsection{Formal Definition of a Limit}

Suppose that \(\mb{r} \colon \R \to \R^{n}\) is a vector-valued function and that \(a \in \R\), though perhaps into in the domain of \(\mb{r}\). \\

If there exists a vector \(\mb{L} \in \R^{n}\) such that for each choice of \(\epsilon > 0\) there exists \(\delta >0\) such that whenever \(t \in \R\), \(t \ne a\) and \(|t - a| < \delta\), then 
\[
    |\mb{r}(t) - \mb{L}| < \epsilon,
\]
then we say that \(\mb{r}\) has \cyanit{limit} \(\mb{L}\) as \(t\) approaches \(a\), and we write
\[
    \lim_{t \to a} \mb{r}(t) = \mb{L}.
\]
That is, if we want the output from \(\mb{r}\) to be close to \(\mb{L}\), we can always choose inputs close to \(a\) to make that occur.

\subsubsection{Limit Properties}

Suppose that \(f,g,h \colon \R \to \R\) are functions and that each of \(\lim_{t \to a} f(t)\), \(\lim_{t \to a} g(t)\), and \(\lim_{t \to a} h(t)\) exist. Then, if \(\vecfuc{r}{t} = f(t)\mb{i} + g(t)\jmb + h(t)\kmb\), then \(\lim_{t \to a} \vecfuc{r}{t}\) exists as well, and has value 
\[
    \lim_{t \to a} \vecfuc{r}{t} = \left(\lim_{t \to a} f(t)\right)\imb + \left(\lim_{t \to a} g(t)\right)\jmb + \left(\lim_{t \to a} h(t)\right)\kmb.
\]
That is, we can essentially do limits component-wise.

Also, there is nothing special here about \(\R^{3}\). This works equally well for \(\R^{n}\) for any \(n\).

\subsection{Continuity}

Suppose that \(\mb{r} \colon \R \to \R^{n}\) is a vector-valued function and that \(a \in \R\). The statement that \(\mb{r}\) is \cyanit{continuous} at \(t = a\) means that
\begin{itemize}
    \item \(\lim_{t \to a} \mb{r}(t)\) exists,
    \item \(\mb{r}(a)\) is defined, and
    \item \(\vecfuc{r}{a} = \lim_{t \to a} \vecfuc{r}{t}\).
\end{itemize}

This is of course essentially the same as our Calculus I definition of continuity, but now in \(\R^{n}\). \\

Like that definition, informally, it means that the curve generated by the function is in ``one piece.''

\newpage

\section{Calculus of Vector Valued Functions}

Last time, we saw that limits and continuity for vector valued functions work component-wise. This is mostly the case for derivatives and integrals as well.

\subsection{Derivatives}

Suppose that \(\rmb \colon \R \to \R^{n}\) is a vector valued function. We define the \cyanit{derivative} of \(\rmb\) as 
\[
    \rmb'(t) = \lim_{h \to 0} \frac{\rmb(t + h) - \rmb(t)}{h},
\]
when this limit exists. We can of course also define the left- or right- hand derivatives as needed.

\subsection{Properties of Derivatives}

Suppose that \(f, \rmb, \umb\) are differentiable and \(c\) is a scalar. Then, we have the following properties:

\begin{itemize}
    \item \(\frac{d}{dt}(c\rmb) = c\frac{d}{dt}\rmb\)
    \item \(\frac{d}{dt}(\rmb \pm \umb) = \frac{d}{dt}\rmb \pm \frac{d}{dt}\umb\)
    \item \(\frac{d}{dt}(f(t)\rmb(t)) = f'(t)\rmb(t) + f(t)\rmb'(t)\)
    \item \(\frac{d}{dt}(\rmb \cdot \umb) = \rmb' \cdot \umb + \rmb \cdot \umb'\)
    \item \(\frac{d}{dt}(\rmb \times \umb) = \rmb' \times \umb + \rmb \times \umb'\)
    \item \(\frac{d}{dt}(\rmb(f(t))) = \rmb'(f(t))f'(t)\)
\end{itemize}

From the fourth above, note that if \(\rmb \cdot \rmb\) is constant, then \(\rmb\) is orthogonal to \(\rmb'\).

\subsection{Tangent Vectors}

Suppose that \(\vecfuc{r}{t}\) is a differentiable vector-valued function and \(t_{0}\) is the domain. This means that \(\vecfuc{r}{t_{0}}\) is tangenet to the curve generated by \(\vecfuc{r}{t}\) at \(t = t_{0}\). In particular, it is a tangent vector in the sense that if we interpret \(\vecfuc{r}{t}\) to refer to the position of some particle as a function of time, then \(\dvecfuc{r}{t_{0}}\) is the velocity vector, and \(\normdvecfuc{r}{t_{0}}\) describes the speed of the particle.

\subsubsection{Unit Tangent Vector}

Of importance is the \cyanit{principle unit tangent vector}, \(\mb{T}\), which is defined by
\[
    \vecfuc{T}{t} = \frac{\dvecfuc{r}{t}}{\|\dvecfuc{r}{t}\|}.
\]
Note that this is only defined when \(\dvecfuc{r}{t} \ne \mb{0}\). \\

The unit tangent vector is simply a vector, of length 1, which points in the tangent direction of the curve. It is useful for describing the direction of motion of a particle along a curve.

\subsection{Integrals}

Like derivatives, we do integrals component-wise:
\begin{align*}
    \int \rmb(t) \, dt &= \int \langle f_{1}(t),f_{2}(t),\ldots,f_{n}(t) \rangle \, dt \\
    &= \left\langle \int f_{1}(t) \, dt, \int f_{2}(t) \, dt, \ldots, \int f_{n}(t) \, dt \right\rangle,
\end{align*}
where since these are indefinite, each produce a constant \(C = \brackett{C_{1},C_{2},\ldots,C_{n}}\). Definite intergrals can be defined in the same way.





\newpage

\section{Arc Length and Curvature}

\subsection{The Unit Normal Vector}

For a curve \(C\) defined by \(\mb{r}\) in \(\R^{3}\), we have the \cyanit{unit normal vector}, \(\mb{N}(t)\), which is defined by 
\[
    \vecfuc{N}{t} = \frac{\dvecfuc{T}{t}}{\|\dvecfuc{T}{t}\|}.
\]
The normal vector points orthogonally to the tangent vector; it points in the direction the curve is turning. \\

We show that \(\mb{T}\) and \(\mb{N}\) are orthogonal:
\begin{align*}
    \mb{T} \cdot \mb{N} &= \vecfuc{T}{t} \cdot \frac{\dvecfuc{T}{t}}{\|\dvecfuc{T}{t}\|} \\
    &= \frac{1}{\normdvecfuc{T}{t}} \vecfuc{T}{t} \cdot \dvecfuc{T}{t} \\
    &= \frac{1}{\normdvecfuc{T}{t}} \cdot 0 \\
    &= 0.
\end{align*}
The last step is true since \(\mb{T}\) is a \textit{unit} vector, and so that \(\|\mb{T}\|\), and from our rules for differentiation, if a vector \(\mb{T}\) has a constant magnitude, then \(\mb{T} \cdot \mb{T}' = 0\).

\subsection{The Binormal Vector}

The \cyanit{binormal vector}, denoted by \(\mb{B}\), is defined by
\[
    \vecfuc{B}{t} = \vecfuc{T}{t} \times \vecfuc{N}{t}.
\]
Since \(\mb{T}\) and \(\mb{N}\) are each unit vectors which are orthogonal, by its definition, \(\mb{B}\) is also a unit vector which is orthogonal to both \(\mb{T}\) and \(\mb{N}\). \\

You can think of \(\mb{T}\) and \(\mb{N}\) as defining the instantaneous plane of motion of a particle. Then, \(\mb{B}\) is the normal to that plane. If \(\mb{B}\) is constant, then the particle stays in a single plane, otherwise, \(\frac{d\mb{B}}{dt}\) measures your ``twisting'' or torsion of motion.

\subsection{Arc Length}

If \(\mb{r}(t)\) defines a smooth curve in \(\R^{n}\), then the \cyanit{arc length} of the curve from \(t = a\) to \(t = b\) is given by
\[
    \int_{a}^{b}\|\mb{r}'(t)\|\, dt.
\]
That is, we are just adding up the ``speed'' of the curve to find its length (i.e., its distance).

\subsubsection{Arc Length Parameterization} 

We can define the \cyanit{arc length parameterization} of a curve \(C\) by:
\begin{itemize}
    \item Define the arc length \(s(t) = \int_{a}^{t}\|\mb{r}'(t)\|\ dt\).
    \item Solving, if possible, the resulting expression for \(t\) as a function of \(s\).
    \item Rewriting \(\mb{r}(t) = \mb{r}(t(s)) = \mb{r}\), so that the curve is written as a function of its length, from a given starting point.
    \item This is useful since there are many possible parameterizations of a given curve, but only a single arc length parameterization.
\end{itemize}
As you may recall from Calculus II, arc length can only be explicitly worked out for carefully selected problems. The same applies here.

\subsection{Curvature}

We define the \cyanit{curvature}, denoted by \(\kappa\), for a smooth curve given by \(\mb{r}(s)\) as 
\[
    \kappa = \left|\left|\frac{d\mb{T}}{ds}\right|\right|.
\]
that is, how fast is the unit tangent vector changing, relative to the length of the curve itself. We use \(s\) because we want our answer to be independent of the parameterization. \\

However, though this formula is the easiest to reason with, it is in practice not useful in most cases. You would have to find the arc-length parameterization, and then find the unit tangent vector from that.  

\subsection{Calculating Curvature}

\begin{itemize}
    \item For all \(\mb{r} \colon \ \kappa = \frac{\|\mb{T}'(t)\|}{\|\mb{r}'(t)\|}\).
    \item For \(\R^{3}\): \(\kappa = \frac{\|\mb{r}'(t) \times \mb{r}''(t)\|}{\|\mb{r}'(t)\|^{3}}\).
    \item If \(y = f(x)\): \(\kappa = \frac{|y''(x)|}{[1 + (y'(x)^{2})]^{3/2}}\)
\end{itemize}

\subsection{Curvature of a Circle}

We use each of the three equations to determine the curvature of a circle of radius \(a\).

\subsubsection{Parameterized Circle}

A circle of radius \(a\) is parameterized by \(\mb{r}(t) = a \cos(t)\mb{i} + a \sin(t)\mb{j}\). Thus, \(\mb{r}'(5) = -a \sin(t)\mb{i} + a \sin(t)\mb{j}\), so that \(\|\mb{r}'(t)\|= a\). So, using the definition above, we get:
\[
    \mb{T}(t) = \frac{\mb{r}'(t)}{\|\mb{r}'(t)\|} = \frac{-a \sin(t)\mb{i} + a \sin(t)\mb{j}}{a} = -\sin(t)\mb{i} + \cos(t)\mb{j}
\]
and therefore
\[
    \mb{T}'(t) = -\cos(t)\mb{i} - \sin(t)\mb{j}
\]
so that \(\|\mb{T}(t)\|= 1\). Finally, we see that 
\[
    \kappa = \frac{\|\mb{T}'(t)\|}{\|\mb{r}'(t)\|} = \frac{1}{a}.
\]
Therefore, a circle has constant curvature, and in fact the curvature is just the reciprocal of the radius. This should make some intuitive sense, as the radius gets smaller, the curvature will increase and vice-versa. Notice also that the curvature of a line is 0 -- in some sense, a line is a circle, but with infinite radius!

\subsubsection{Curvature of a Circle in Three Dimensions}

We can also use the second formula, again with a circle, noting that a circle of radius \(a\) is parameterized by \(\mb{r} = a \cos(t)\mb{i} + a \sin(t)\mb{j} + 0\mb{k}\) so that \(\mb{r}'(t) = -a \sin(t)\mb{i} + a \cos(t)\mb{j} + 0\mb{k}\) so that \(\|\mb{r}'(t)\|= a\). Thus, \(\mb{r}'' = -a \cos(t)\mb{i} - a \sin(t)\mb{j}\), so that 
\[
    \mb{r}'(t) \times \mb{r}''(t) = \begin{vmatrix}
        \mb{i} & \mb{j} & \mb{k} \\
        -a \sin(t) & a \cos(t) & 0 \\
        -a \cos(t) & -a \sin(t) & 0
    \end{vmatrix} = a^{2}\mb{k}.
\] 
Then, 
\[
    \kappa = \frac{\|\mb{r}'(t) \times \mb{r}''(t)\|}{\|\mb{r}'(t)\|^{3}} = \frac{a^{2}}{a^{3}} = \frac{1}{a}.
\]

\subsubsection{Curvature as a Function}

We can also write (the top half) of a circle as \(y(x) = \mysqrt{a_{2} - x_{2}}\). Then, \(y' = \dfrac{-x}{\mysqrt{a^{2} - x^{2}}}\) and 
\begin{align*}
    y'' &= \frac{-\mysqrt{a^{2} - x^{2}} - \frac{1}{2}(a^{2} - x^{2})^{-1/2}(-2x)(-x)}{a^{2} - x^{2}} \\
    &= \frac{-\mysqrt{a^{2} - x^{2}} - x^{2}/\mysqrt{a^{2} - x^{2}}}{a^{2}-x^{2}} \\
    &= \frac{-(a^{2} - x^{2}) - x^{2}}{(a^{2} - x^{2})^{3/2}} \\
    &= \frac{-a^{2}}{(a^{2} - x^{2})^{3/2}}.
\end{align*}
Thus, 
\[
    \kappa = \frac{|y''(x)|}{[1 + (y'(x)^{2})]^{3/2}} = \frac{1}{a},
\]
after some algebra.

\subsubsection{Osculating Circle}

In general, of course, curvature is not constant. At a particular point of interest, the curvature finds the reciprocal of the radius of the \cyanit{osculating circle}. That is, the ``tangent circle''--the circle that best matches the curve at the point of interest.

\newpage

\section{Motion in Space}

\subsection{Motion}

\newcommand{\rt}{\mb{r}(t)}

Let \(\mb{r}(t)\) represent the position of some object as a function of time in \(\R^{3}\). Then, as in Calculus I:
\begin{itemize}
    \item Velocity vector: \(\mb{v}(t) = \dvecfuc{r}{t}\)
    \item Speed: \(\|\mb{v}(t)\|=\|\dvecfuc{r}{t}\|\)
    \item Acceleration vector: \(\mb{a}(t) = \dvecfuc{v}{t} = \mb{r}''(t)\)
\end{itemize}
Note that \(\vecfuc{v}{t} \ne \vecfuc{T}{t}\), though they do point in the same direction. More importantly, the relationship between \(\vecfuc{a}{t}\) and \(\vecfuc{N}{t}\) is complicated. \(\vecfuc{N}{t}\) is not acceleration, or even the direction of acceleration, but is the direction of ``turning.''

\subsubsection{Acceleration}

In fact, \(\vecfuc{a}{t} = v'(t)\vecfuc{T}{t} + (v(t))^{2}\kappa\vecfuc{N}{t}\).

\begin{proof}
    Notice that \(\vecfuc{v}{t} = v(t)\vecfuc{T}{t}\). Then,
    \begin{align*}
        \vecfuc{a}{t} &= v'(t)\vecfuc{T}{t} + v(t)\dvecfuc{T}{t} \\
        &= v'(t)\vecfuc{T}{t} + v(t)\|\dvecfuc{T}{t}\|\vecfuc{N}{t} \\
        &= v'(t)\vecfuc{T}{t} + v(t)\kappa\|\dvecfuc{r}{t}\|\vecfuc{N}{t} \\
        &= v'(t)\vecfuc{T}{t} + (v(t))^{2}\kappa\vecfuc{N}{t}.
    \end{align*}
    Since \(\mb{T}\) and \(\mb{T}'\) are orthogonal, this decomposes \(\mb{a}\) into two pieces -- the first is the change in speed, and the second the change in direction.
\end{proof}

\subsection{Projectile Motion}

The motion of an object -- in 2-dimensions, typically, acted on only by gravity \(\mb{F}_{g} = -mg\mb{j}\), where \(g \approx 9.8\) m/s\(^{2}\) and \(m\) is the mass of the object. By Newton's second law, \(\mb{F} = m\mb{a}\), so we have 
\[
    \vecfuc{a}{t} = - g\mb{j}.
\]
Thus, \(\vecfuc{v}{t} = -gt\mb{j} + \mb{v}_{0}\), where \(\mb{v}_{0}\) is the initial velocity vector, and
\[
    \vecfuc{s}{t} = -\frac{1}{2}gt^{2}\mb{j} + \mb{v}_{0}t + \mb{s}_{0},
\]
where \(\mb{s}\) is the position, and \(\mb{s}_{0}\) is the initial position vector. \\

Often, we have an object starting at the origin (so \(\mb{s}_{0} = \mb{0}\)) and fired at a velocity of \(v_{0}\) at an angle \(\theta\) above the horizon. Then,
\[  
    \vecfuc{s}{t} = v_{0}t\cos(\theta)\mb{i} + \left(v_{0}t\sin(\theta) - \frac{1}{2}gt^{2}\right)\mb{j}.
\]
What is the horizontal distance travelled by the object before it hits the ground again? \\

Solving \(\bigl(v_{0}t\sin(\theta) - \frac{1}{2}gt^{2}\bigr) = 0\) gives \(t_{1} = \slashed{0}, \ \frac{2v_{0}\sin(\theta)}{g}\). We plug this time into the horizontal component and find 
\[
    s(t_{1}) = \tfrac{v_{0}^{2}}{g}\sin(2\theta).
\]
We note the unsurprising result that this is maximized when \(\theta = \frac{\pi}{4}\).

\subsection{Planetary Motion}

There is a handout (on teams) that works through the mathematics of planetary motion --specifically:
\begin{itemize}
    \item Starting from Newton's Law of Gravity and Newton's Second Law, we determine the position of a planet circling a (fixed) star.
    \item Describe the conditions under which the planet is bound to the star, or separately, flung off into space!
\end{itemize}
These are not exam questions, but are the sort of things that this material is used for in the ``real world.'' \\