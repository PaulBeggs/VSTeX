\section{Vector Fields}

\subsection{Introduction}
So far, we have talked about:

\begin{itemize}
    \item Parametric Functions: \(f \colon \mathbb{R} \to \mathbb{R}^n\) (also called vector-valued functions)
    \item Scalar Functions: \(f \colon \mathbb{R}^n \to \mathbb{R}\)
\end{itemize}

We conclude the course by considering:

\begin{itemize}
    \item \(f \colon \mathbb{R}^n \to \mathbb{R}^n\), specifically:
    \begin{itemize}
        \item \(f \colon \mathbb{R}^2 \to \mathbb{R}^2\)
        \item \(f \colon \mathbb{R}^3 \to \mathbb{R}^3\)
    \end{itemize}
\end{itemize}

\subsection{Vector Field}
A vector field is an assignment to each point in \(\mathbb{R}^n\) a vector in \(\mathbb{R}^n\). For example:

\begin{itemize}
    \item At each point on the Earth's surface, assign the current wind speed and direction.
    \item A force field – gravity, electromagnetism, etc.
    \item Fluid flow – in a pipe, the rate and direction of flow of water.
    \item A gradient field – given a surface, its gradient at each point.
\end{itemize}

Formally, a vector field is a function \(\mathbf{F} : \mathbb{R}^n \to \mathbb{R}^n\) with some domain \(D \subseteq \mathbb{R}^n\). We will only be concerned with \(n = 2\) and \(n = 3\). \\

\textbf{Common notation:}
\[
    \mathbf{F}(x, y) = \langle P(x, y), Q(x, y) \rangle = P(x, y) \mathbf{i} + Q(x, y) \mathbf{j} 
\]
and
\[
    \mathbf{F}(x, y, z) = \langle P(x, y, z), Q(x, y, z), \mb{R}(x, y, z) \rangle = P(x, y, z) \mathbf{i} + Q(x, y, z) \mathbf{j} + \mb{R}(x, y, z) \mathbf{k} 
\]

\subsection{Graph}
To truly graph a vector field, you need 4 or 6 dimensions, which is somewhat inconvenient. Instead, we plot representative vectors in either \(\mathbb{R}^2\) or \(\mathbb{R}^3\).

The applet at \url{https://www.geogebra.org/m/QPE4PaDZ} is one way to graph a 2-D vector field.

\subsection{Special Vector Fields}
There are a few commonly seen special types of vector fields:

\begin{itemize}
    \item \textbf{Radial} – each vector points either directly toward or away from the origin. For example, \(\langle x, y \rangle\).
    \item \textbf{Rotational} – each vector points tangent to a circle centered at the origin. For example, \(\langle -y, x \rangle\).
    \begin{itemize}
        \item \textbf{Think:} At the point \((2,1)\), we have the vector \(\langle -1, 2 \rangle\). This vector is tangent to the circle of radius \(\mysqrt{5}\) centered at the origin. The vector is perpendicular to the radius of the circle at that point.
    \end{itemize}
    \item \textbf{Unit} – each output vector has magnitude 1. For example, \(\langle \sin(x + y), \cos(x + y) \rangle\).
    \item \textbf{Velocity} – the vectors represent the velocity of some particle.
    \item \textbf{Gradient} – the vectors are the gradient of some function \(\mb{F} \colon \mathbb{R}^n \to \mathbb{R}\).
\end{itemize}

\subsection{Gradient Field}
A vector field \(\mathbf{F}\) is called a gradient vector field if there exists some scalar function \(\mb{F} \colon \mathbb{R}^n \to \mathbb{R}\) such that \(\nabla f = \mathbf{F}\). That is, if \(\mathbf{F}\) is the gradient of some scalar function \(f\).

\subsection{Conservative Vector Fields}
A vector field \(\mathbf{F}\) is a conservative vector field when it is the gradient field for some scalar function \(f\). We will call \(f\) the \textbf{potential function} of \(\mathbf{F}\). \\

\textbf{Theorem.} Potential functions are unique, except for a constant. That is, if \(f\) and \(g\) are each potential functions for \(\mathbf{F}\), then there exists some constant \(c\) such that \(f - g = c\).

\subsection{Cross Partial Property}
\textbf{Theorem.} Suppose that \(\mathbf{F}\) is a conservative vector field with continuous mixed second partials. Then:

\begin{itemize}
    \item In two dimensions, if \(\mathbf{F} = \langle P,Q \rangle\), then \(\frac{\partial P}{\partial y} = \frac{\partial Q}{\partial x}\).
    \item In three dimensions, if \(\mathbf{F} = \langle P, Q, R \rangle\), then:
    \begin{itemize}
        \item \(\frac{\partial P}{\partial y} = \frac{\partial Q}{\partial x}\)
        \item \(\frac{\partial Q}{\partial z} = \frac{\partial R}{\partial y}\)
        \item \(\frac{\partial R}{\partial x} = \frac{\partial P}{\partial z}\)
    \end{itemize}
\end{itemize}

\subsection{Examples}

\subsubsection{Example 1: Gradient Field}

\(f(x,y) = x^{2}y + y^{3}\) is a potential function for the vector field \(\nabla f = \mathbf{F}(x,y) = \langle 2xy, x^{2} + 3y^{2} \rangle = \langle P, Q \rangle\). \\

If we find the second derivative of these functions, we see that \(\frac{\p P}{\p y} = 2x; \, \frac{\p Q}{\p x} = 2x\). Thus, the cross partial property holds. \\

\newpage

\section{Line Integrals}

\subsection{Scalar Line Integral}
Suppose that \(\mb{F} \colon \mathbb{R}^n \to \mathbb{R}\) is a continuous scalar function and that \(C\) is a continuous curve in \(\mathbb{R}^n\) included in the domain of \(f\). We want to find the area of the sheet which extends down from \(f\) to the domain along the curve \(C\). We parameterize \(C\) as \(\mb{r}(t)\), \(a \leq t \leq b\) and break \(C\) into small subintervals of length \(\Delta t = \frac{b-a}{n}\). Then, we have:
\begin{align*}
    \Delta s_1 &= [\mb{r}(a), \mb{r}(a + \Delta t)] \\
    \Delta s_2 &= [\mb{r}(a + \Delta t), \mb{r}(a + 2\Delta t)] \\
    &\vdots \\
    \Delta s_n &= [\mb{r}(a + (n - 1)\Delta t), \mb{r}(a + n\Delta t)]
\end{align*}
In each \(\Delta s_i\), choose \(t^*_i\) and find \(f(\mb{r}(t^*_i ))\). Then, we have a Riemann sum:
\[
    \sum_{i=1}^{n} f(\mb{r}(t^*_i ))\Delta s_i, 
\]
which, if the limit as \(n \to \infty\) exists, we will write as one of:
\[
    \int_C f(x, y) \, ds \quad \text{or} \quad \int_C f(x, y, z) \, ds. 
\]

The term \(ds\) represents the infinitesimal step taken along the path \(C\), analogous to \(dx\) in single-variable integration. Since working with \(ds\) directly is cumbersome, we use:
\[
    \int_C f(x, y) \, ds = \int_C f(x(t), y(t)) \mysqrt{(x'(t))^2 + (y'(t))^2} \, dt = \int_a^b f(\mb{r}(t)) \underbrace{\| \mb{r}'(t) \|}_{\text{*}} \, dt, \| \mb{r} \|
\]
(*: Area of sheet under surface above curve.) \\
which simplifies computations and ensures that parametrization does not affect the integral's value, though orientation does. If \(C\) and \(D\) are the same curve with opposite orientations:
\[
    \int_C f \, ds = - \int_D f \, ds, 
\]
analogous to:
\[
    \int_a^b f(x) \, dx = - \int_b^a f(x) \, dx. 
\]

\subsection{Line Integrals over Vector Fields}
Suppose that \(\mb{F} \colon \mathbb{R}^n \to \mathbb{R}^n\) is a vector field, and \(C\) is a continuous curve in \(\mathbb{R}^n\) within the domain of \(\mb{F}\). We wish to evaluate:
\[
    \int_C \mb{F} \cdot d\mb{s}, 
\]
which essentially ``multiplies'' \(\mb{F}\) by a small infinitesimal along the curve \(C\) (denoted by \(d\mb{s}\)). \\

We choose the points \(P_{0}, P_{1}, P_{2}, \ldots, P_{i},\ldots, P_{n}\) to divide up our curve \(C\) and then approximate our line integral as a Riemann sum. We can write the line integral as:
\[
    \int_C \mb{F} \cdot d\mb{s} = \lim_{n\to\infty} \sum_{i=1}^{n} F(P^*_i) \cdot \Delta \mb{s}_i. 
\]
Since \(d\mb{s} = \mb{r}'(t) dt\), the integral simplifies to:
\[
    \int_C \mb{F} \cdot d\mb{s} = \int_a^b F(\mb{r}(t)) \cdot \mb{r}'(t) \, dt, 
\]

Where \(\mb{r}(t)\), \(a \leq t \leq b\) is any parametrization of \(C\). \\ 

Equivalent notations include:
\[
    \int_C \mb{F} \cdot d\mb{r}, \quad \int P \, dx + Q \, dy, \quad \text{or} \quad \int P \, dx + Q \, dy + R \, dz. 
\]
These alternate notations come from expanding the dot product: \(\int\langle P,Q \rangle \cdot \langle x', y' \rangle \, dt\). Solving, we get the notation above.

\subsection{Properties of Line Integrals over Vector Fields}
\begin{itemize}
    \item \(\int_C (\mb{F} + G) \cdot d\mb{r} = \int_C \mb{F} \cdot d\mb{r} + \int_C G \cdot d\mb{r}\)
    \item \(\int_C k\mb{F} \cdot d\mb{r} = k \int_C \mb{F} \cdot d\mb{r}\), for constant \(k\)
    \item \(\int_C \mb{F} \cdot d\mb{r} = - \int_{-C} \mb{F} \cdot d\mb{r}\), where \(-C\) denotes \(C\) with reversed orientation
    \item If \(C = C_1 + C_2\) with intersection only at endpoints, then:
    \[
    \int_C \mb{F} \cdot d\mb{r} = \int_{C_1} \mb{F} \cdot d\mb{r} + \int_{C_2} \mb{F} \cdot d\mb{r}. 
\]
\end{itemize}

\subsection{Circulation Along a Curve}
The circulation of \(\mb{F}\) along \(C\) is defined as:
\[
    \int_C \mb{F} \cdot dr. 
\]
If \(\mb{F}\) represents a force field, this integral gives the total work done by \(\mb{F}\) in moving an object along \(C\).

\subsection{Flux Across a Curve}
Given a vector field \(\mb{F}\) and curve \(C\) in \(R^{2}\), the \cyanit{flux} of \(\mb{F}\) across \(C\) is the total amount of \(\mb{F}\) that crosses orthogonally across \(C\). We note that there is an orientation issue here. We define positive flux as moving from left to right across the curve where forward is the direction of the curve’s orientation. In two dimensions, this is actually opposite of \(N\) from chapter 2, but we’ll fix that easily by defining \(n\) to be the unit vector which points left-to-right across the curve. We can see that \(n(t) = \langle y'(t), -x'(t) \rangle\).
Thus, the flux integral is given by:
\[
    \int_C \mb{F} \cdot \mb{N} \, ds = \int_a^b \mb{F}(\mb{r}(t)) \cdot \langle y'(t), -x'(t) \rangle \, dt, 
\]
or equivalently:
\[
    \int_C -Q \, dx + P \, dy. 
\]

\subsection{Examples}
\subsubsection{Example 1: Scalar Line Integral}
Evaluate \(\int_C (x^2 + y^2) ds\) where \(C\) is the quarter-circle \(\mb{r}(t) = (\cos t, \sin t)\), \(0 \leq t \leq \frac{\pi}{2}\).

\textbf{Solution:}
\begin{align*}
    ds &= \|\mb{r}'(t)\| dt = \mysqrt{(-\sin t)^2 + (\cos t)^2} dt = dt. \\
    f(\mb{r}(t)) &= \cos^2 t + \sin^2 t = 1. \\
    \int_C (x^2 + y^2) ds &= \int_0^{\pi/2} 1 \, dt = \frac{\pi}{2}.
\end{align*}

\subsubsection{Example 2: Scalar Line Integral}
Let \(f(x,y) = x^{2}y\) and \(C\) by the curve from \((0,0)\) to \((2,4)\) to \((0,4)\). Call lines \(C_1\) and \(C_2\). Find the value of \(\int_C f(x,y) ds\).

\textbf{Solution:}

First, we find \(C_{1}\): \(\mb{r}(t) = \langle t,2t\rangle, \, 0 \leq t \leq 2\). This turns into the first integral:
\begin{align*}
    \int_{C_{1}} f(x,y) \, ds &= \int_{0}^{2} t^{2}(2t) \sqrt{(1)^{2} + (2)^{2}} \, dt \\
    &= 2\sqrt{5} \int_{0}^{2} t^{3}  \, dt \\
    &= 2\sqrt{5} \left[ \frac{t^{4}}{4} \right]_{0}^{2} \\
    &= 2\sqrt{5} \left[ \frac{16}{4} - 0 \right] \\
    &= 8\sqrt{5}.
\end{align*}

Now, for \(C_{2}\): \(\mb{r}(t) = \langle 2 - t, 4\rangle, \, 0 \leq t \leq 2\). This turns into the second integral:
\begin{align*}
    \int_{C_{2}} f(x,y) \, ds &= \int_{0}^{2} (2 - t)^{2}(4) \sqrt{(-1)^{2} + (0)^{2}} \, dt \\
    &= 4\int_{0}^{2} (2 - t)^{2} \, dt \\
    &= \frac{-4}{3}\left[ (2 - t)^{3} \right]_{0}^{2} \\
    &= \frac{-4}{3}\left[ 0 - \frac{8}{3} \right] \\
    &= \frac{32}{3}.
\end{align*}

\subsubsection{Example 3: Vector Line Integral}
Evaluate \( \int_C \mb{F} \cdot dr \) where \( \mb{F} = \langle y, x \rangle \) and \( C \) is the line from \( (0,0) \) to \( (1,1) \).

\textbf{Solution:}
\begin{align*}
    \mb{r}(t) &= (t, t), \quad 0 \leq t \leq 1. \\
    \mb{r}'(t) &= (1,1). \\
    F(\mb{r}(t)) &= \langle t, t \rangle. \\
    \int_C \mb{F} \cdot dr &= \int_0^1 \langle t, t \rangle \cdot \langle 1, 1 \rangle \, dt = \int_0^1 (t + t) \, dt = \int_0^1 2t \, dt = 2.
\end{align*}

\newpage

\section{Conservative Vector Fields}

\subsection{Conservative Vector Fields}
Previously, we saw that a conservative vector field, \(\mathbf{F}\), is one for which there exists a scalar potential function \(f\) such that \(\nabla f = \mathbf{F}\). That is, \(\mathbf{F}\) is the gradient of some potential function \(f\).

\subsection{Fundamental Theorem of Line Integrals}
Suppose that \(C\) is piecewise smooth, parameterized by \(\mathbf{r}\) on \(a \leq t \leq b\), and \(f\) is differentiable in two or three variables. If \(\mathbf{F} = \nabla f\) (i.e., \(f\) is conservative) is continuous on \(C\), then
\[
    \int_C \mathbf{F} \cdot d\mathbf{r} = \int_C \nabla f \cdot d\mathbf{r} = f(\mathbf{r}(b)) - f(\mathbf{r}(a)).
\]

\begin{proof}
    This is an immediate result of the chain rule and the ``regular'' Fundamental Theorem of Calculus:
\begin{align*}
    \int_C \nabla f \cdot d\mathbf{r} &= \int_a^b \nabla f(\mathbf{r}(t)) \cdot \mathbf{r}'(t) \, dt \\
    &= \int_a^b \frac{d}{dt}f(\mathbf{r}(t)) \, dt \\
    &= f(\mathbf{r}(b)) - f(\mathbf{r}(a)). \qedhere
\end{align*}
\end{proof} 


Thus, when \(\mathbf{F}\) is the gradient of some potential function – i.e., when \(\mathbf{F}\) is conservative – we can evaluate a line integral by:
\begin{itemize}
    \item determining a potential function for \(\mathbf{F}\)
    \item evaluating the potential function at each of \(t = a\) and \(t = b\) and finding the difference.
\end{itemize}

The line integral of a conservative vector field \(\mathbf{F}\) is just the change in potential between the ending and starting points on the curve. \\

Therefore, for a conservative vector field, the circulation along a curve from the point \(P\) to \(Q\) is independent of the path to get from \(P\) to \(Q\). In fact, we call this the \cyanit{path independence property} of \(\mathbf{F}\). Moreover, if you have 2 curves \(C_{1}\) and \(C_{2}\), and \(\mb{F}\) is conservative, then
\[
    \int_{C_{1}} \mb{F} \cdot d\mb{r} = \int_{C_{2}} \mb{F} \cdot d\mb{r}.
\]
This is very much like the Calculus I version of the Fundamental Theorem of Calculus – in that case, to determine \(\int_a^b f(x) \, dx\), we find an antiderivative, \(F(x)\), for \(f\) and evaluate that antiderivative at each of \(x = a\) and \(x = b\) and subtract. We did not notice ``path independence'' since there is (essentially) only one path in \(\mathbb{R}\) to go from \(x = a\) to \(x = b\).

\subsubsection{Line Integrals on a Closed Curve}

If \(C\) is a closed curve, we often emphasize this for a line integral by writing 
\[
    \oint_C \mathbf{F} \cdot d\mathbf{r}.
\]
Notice that the Fundamental Theorem of Line Integrals implies that whenever \(\mathbf{F}\) is conservative, then
\[
    \oint_C \mathbf{F} \cdot d\mathbf{r} = 0.
\]

\subsubsection{Path Independence}

The Fundamental Theorem of Line Integral implies that any \(\int_C \mathbf{F} \cdot d\mathbf{r}\) is path independent whenever \(\mathbf{F}\) is conservative. The reverse is almost true: \\

\textbf{Theorem.} If \(\mathbf{F}\) is a continuous vector field that is path independent with open, connected domain, then \(\mathbf{F}\) is conservative.

\subsubsection{Cross Partial Property}

\textbf{Theorem.} Suppose that \(\mathbf{F}\) is a continuous vector field on an open, simply connected region \(D\) in \(\mathbb{R}^2\) or \(\mathbb{R}^3\). If the cross partial property holds in \(D\), then \(\mathbf{F}\) is conservative.

\subsection{Finding a Potential Function}

To find a potential function for a conservative vector field, we can use the following algorithm: \\

Given a conservative vector field \(\mathbf{F}(x,y) = P(x,y)\mathbf{i} + Q(x,y)\mathbf{j}\), we can find a potential function \(f\) such that \(\nabla f = \mathbf{F}\). The algorithm is as follows:

\begin{itemize}
    \item We can find \(\int P(x,y) \, dx\) (i.e. hold \(y\) constant) = \(g(x,y) + h(y)\), for some currently unknown function \(h\) – that is, \(g\) is “just” the antiderivative of \(P\), only with respect to \(x\).
    \item Then \(\frac{\partial}{\partial y} (g(x,y) + h(y)) = g_y(x,y) + h'(y)\).
    \item This should be \(Q(x,y) = g_y(x,y) + h'(y)\), so we know \(h'(y)\).
    \item Then, \(h(y) = \int h'(y) \, dy\).
    \item Finally, \(f(x,y) = g(x,y) + h(y) + K\), where \(K\) is a constant of integration.
\end{itemize}

\subsection{Examples}

\subsubsection{Example 1}

Let \(\mb{F} = \langle y^{2}, 2xy + 2 \rangle\) and \(C\) be the curve parameterized by \(\mb{r}(t) = \langle t\cos(t), t^{2} + e^{t} - \sin(t) \rangle\), \(0 \leq t \leq 2\). Find \(\int_C \mb{F} \cdot d\mb{r}\). \\

\textbf{Solution:} \\
We first check if \(\mb{F}\) is conservative. We find the cross partials:
\begin{align*}
    \frac{\partial P}{\partial y} = 2y, \quad \text{and} \quad 
    \frac{\partial Q}{\partial x} = 2y.
\end{align*}
Since these are equal, we know that \(\mb{F}\) is conservative. \\

We can find a potential function for \(\mb{F}\) by using the algorithm above. We have:
\begin{align*}
    \int P(x,y) \, dx &= \int y^{2} \, dx = xy^{2} + h(y), \\
    \frac{\partial}{\partial y} (xy^{2} + h(y)) &= 2xy + h'(y) = Q(x,y) = 2xy + 2. \\
    h'(y) &= 2, \\
    h(y) &= 2y + K. \\
    f(x,y) &= xy^{2} + 2y + K.
\end{align*}
Now, we can evaluate the line integral:
\begin{align*}
    \int_C \mb{F} \cdot d\mb{r} &= f(\mb{r}(2)) - f(\mb{r}(0)) \\
    &= f(2\cos(2), 4 + e^{2} - \sin(2)) - f(0, e^{2}) \\
    &= (2\cos(2))(4 + e^{2} - \sin(2)) + 2(4 + e^{2} - \sin(2)) - (0) - (e^{2}) \\ 
    &= 8\cos(2) + 3e^{2} - 2\sin(2).
\end{align*}

\newpage

\section{Green's Theorem}

\subsection{Circulation over Closed Curves}

Let \(C\) be a closed curve and \(\mathbf{F} : \mathbb{R}^2 \to \mathbb{R}^2\) be a vector field. Then, the circulation of \(\mathbf{F}\) along \(C\) is given by:
\[
    \oint_C \mathbf{F} \cdot \mb{T} \, ds = \oint_C \mathbf{F} \cdot d\mb{r} = \oint P \, dx + Q \, dy.
\]
You can think of this as measuring the total amount that \(\mathbf{F}\) pushes along the curve \(C\). \\

We will define the closed curve \(C\) to have positive orientation if the interior of the region enclosed by the curve is always to the left as you traverse the curve – we often call this counter-clockwise orientation. \\

Notice that if \(\mathbf{F}\) happens to be conservative, then \(P_y = Q_x\) by the cross partial property. Thus, \(Q_x - P_y = 0\). Then, if \(D\) is the region defined by the interior of the curve \(C\), we have:
\[
    \iint_D (Q_x - P_y) \, dA = 0,
\]
which happens to match:
\[
    \oint_C \mathbf{F} \cdot d\mb{r}.
\]
But what if \(\mathbf{F}\) is not conservative?

\subsection{Green's Theorem – Circulation Form}

Suppose that \(C\) is a positively oriented closed curve and that \(D\) is the region bounded by \(C\). If \(\mathbf{F} = \langle P,Q \rangle\) is any vector field with continuous partial derivatives, then:
\[
    \oint_C \mathbf{F} \cdot d\mb{r} = \iint_D \left( \frac{\partial Q}{\partial x} - \frac{\partial P}{\partial y} \right) dA.
\]
We can write each of these as:
\[
    \oint_C \mathbf{F} \cdot d\mb{r} = \oint_C P \, dx + Q \, dy = \iint_D \left( \frac{\partial Q}{\partial x} - \frac{\partial P}{\partial y} \right) dA = \iint_D (Q_x - P_y) dA.
\]
Thus, the total circulation of \(\mathbf{F}\) along the closed curve \(C\) is equal to the sum (i.e. integral) of the difference between the cross partials over the entire region \(D\). The difference of the partials measures how non-conservative the vector field is – and measures the total microscopic circulation at each point in \(D\). But, in the interior all of the microscopic circulations cancel out, leaving only the circulation along the boundary. \\

\subsection{Interpretation as a Fundamental Theorem}

We can interpret the quantity \(Q_x - P_y\) as a type of derivative of \(\mathbf{F}\). The total double sum of the quantity \(Q_x - P_y\) over some region matches the sum of the original function \(\mathbf{F}\) over the boundary of that region. \\

\subsection{Flux over Closed Curves}

The flux of \(\mathbf{F} : \mathbb{R}^2 \to \mathbb{R}^2\), where we write \(\mathbf{F} = \langle P,Q \rangle\) over a closed curve \(C\) is given by:
\[
    \oint_C \mathbf{F} \cdot \mb{N} \, ds = \oint_C \mathbf{F} \cdot \mb{n} \, dt = \oint_C \mathbf{F} \cdot \langle y'(t), -x'(t) \rangle \, dt = \oint_C -Q \, dx + P \, dy.
\]

\subsubsection{Green's Theorem – Flux Form}

Suppose that \(C\) is a positively oriented closed curve and that \(D\) is the region bounded by \(C\). If \(\mathbf{F} = \langle P,Q \rangle\) is any vector field with continuous partial derivatives, then:
\[
    \oint_C \mathbf{F} \cdot \mb{n} \, dt = \iint_D \left( \frac{\partial P}{\partial x} + \frac{\partial Q}{\partial y} \right) dA.
\]
We can write each of these as:
\[
    \oint_C \mathbf{F} \cdot \mb{n} \, dt = \oint_C -Q \, dx + P \, dy = \iint_D \left( \frac{\partial P}{\partial x} + \frac{\partial Q}{\partial y} \right) dA = \iint_D (P_x + Q_y) dA.
\]
Thus, the total flux of \(\mathbf{F}\) across the closed curve \(C\) is equal to the sum (i.e. integral) of the sum of the regular partials over the entire region \(D\). The sum of the partials measures how much the vector field is running away (expanding away) from a point. But, as the animations in the slides show, in the interior all of the expansions cancel out, leaving only the expansion (i.e. flux) across the boundary.

\subsection{Interpretation as a Fundamental Theorem}

We can interpret the quantity \(P_x + Q_y\) as a type of derivative of \(\mathbf{F}\). The total double sum of the quantity \(P_x + Q_y\) over some region matches the sum of the original function \(\mathbf{F}\) across the boundary of that region.
