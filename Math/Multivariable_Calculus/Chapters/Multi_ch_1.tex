\section{Parametric Equations}


\subsection{Introduction}
Most of your calculus experience has been single variable, so that the functions under consideration were typically \(f : \mathbb{R} \to \mathbb{R}\). Our course is divided into roughly 3 sections:
\begin{itemize}
    \item Parametric Equations/Functions: Functions of the form \(f : \mathbb{R} \to \mathbb{R}^n\) (Chapters 1 - 3)
    \item Scalar Functions: Functions of the form \(f : \mathbb{R}^n \to \mathbb{R}\) (Chapters 4 - 5)
    \item Vector Fields: Functions of the form \(f : \mathbb{R}^n \to \mathbb{R}^n\) (Chapter 6)
\end{itemize}

\subsection{Parametric Equations}

A \cyanit{parametric equation} (or, \cyanit{sometimes parametric function} or \cyanit{ vector-valued function}) is a function of the form \(f \colon \mathbb{R} \to \mathbb{R}^{n}\). We will typically consider \(n = 2\) or \(n = 3\) and call the input variable the parameter, usually denoted by \(t\). We write them as
A \cyanit{parametric equation} (or, \cyanit{sometimes parametric function} or \cyanit{ vector-valued function}) is a function of the form \(f \colon \mathbb{R} \to \mathbb{R}^{n}\). We will typically consider \(n = 2\) or \(n = 3\) and call the input variable the parameter, usually denoted by \(t\). We write them as

\[
    f(t) =
    \begin{cases}
        x(t) \\
        y(t)
    \end{cases}
    \quad \text{or} \quad
    f(t) =
    \begin{cases}
        x(t) \\
        y(t) \\
        z(t)
    \end{cases}.
\]

A \cyanit{parametric curve} is the set of points \((x(t), y(t))\) in \(\R^{2}\) or \((x(t), y(t), z(t))\) in \(\R^{3}\) traced out. Note that in general, the curve may not be a function for \(y\) in terms of \(x\), but is a function of the parameter \(t\).

\subsection{Graphing Parametric Curves in the Second Dimension}

\subsubsection{Elimination of the Parameter}

In some cases, we can explicitly solve for \(t\) in terms of one of \(x\) or \(y\). When this is possible, you can write \(y(x)\) or \(x(y)\) and use your “regular” algebraic knowledge. We call this process \cyanit{eliminating the parameter}.

\subsubsection{Using Technology}

\begin{itemize}
    \item Your TI-84 can graph this if you switch to \texttt{par} mode.
    \item Likewise, GeoGebra can do this, using the \texttt{curve} function.
          \begin{itemize}
              \item In general, the syntax is: \texttt{curve(x(t), y(t), t, min, max)}
          \end{itemize}
\end{itemize}

\subsection{The Cycloid}
A wheel of radius \(a\) is rolling along a flat road at a constant velocity. The curve generated by a point along the edge of the wheel traces out a shape called a \cyanit{cycloid}. Let \(t\) represent the angle - in radians!!!! - rotated through, and that the point of interest starts at the origin. Before we find the equations for the point, let's find the location of the center of the circle:
\[
    f_{\text{center}}(t) = \begin{cases}
        x(t) = a t \\
        y(t) = a
    \end{cases}
\]

Then, relative to the center, our point along the edge has equations
\[
    f(t) = \begin{cases}
        x(t) = -a \sin(t) \\
        y(t) = -a \cos(t)
    \end{cases}
\]

Thus, our point has parametric equations
\[
    f(t) = \begin{cases}
        x(t) = a(t - \sin(t)) \\
        y(t) = a(1 - \cos(t))
    \end{cases}
\]

\subsection{Final Notes}

Next time, we'll start asking Calculus-y questions: What are the velocities in the \(x\), \(y\), and total directions? What total distance does it travel? What is the area of the region under one period of the cycloid?
\begin{itemize}
    \item The syllabus has a number of practice problems to work on. These are not required, and not to be
          turned in, but are for you to work before class next time.
    \item We will talk about them at the start of the next class. You should try them beforehand.
    \item The most common reason for a lack of success in this class is not spending time working problems on
          your own.
\end{itemize}

\section{Calculus of Parametric Curves}

For this section, we will have a parametric curve in R2, defined by \(f(t) = \begin{cases}
    x(t) \\
    y(t)
\end{cases}.\) 
In many cases, the curve does not describe \(y\) as a function of \(x\). However, we can still carry over many ideas from single variable calculus.

\subsection{Slope for a Parametric Curve}

Given a point \(t_{0}\), the \cyanit{slope of the curve} in the \(xy\)-plane is given by
\[
    \dydx\barNotationT{t_{0}} = \frac{dy/dt}{dx/dt}\barNotationT{t_{0}}. \\
\]
Note that this is undefined when \(x'(t_{0}) = 0\). \\

\noindent The \cyanit{tangent line} at \(t_{0}\) is given by
\[
    y = \left(\dydx\barNotationT{t_{0}}\right)(x - x(t_{0})) + y(t_{0}).
\]

\subsection{Second Derivative}

The value of the second derivative for the curve at \(t_{0}\) is given by
\[
    \dfrac{d^{2}y}{dx^{2}}\barNotationT{t_{0}} = \frac{d}{dt}\left(\dydx\right)\barNotationT{t_{0}} = \frac{d}{dt}\left(\frac{dy/dt}{dx/dt}\right)\barNotationT{t_{0}}. \\
\]
Note the benefit of Leibnitz notation for each of these two derivatives!

\subsection{Area Under a Curve}

Suppose that a parametric curve is non-self intersecting. Then, the signed area of the region between the curve and the \(x\)-axis on the \(t\) interval \([t_{a},t_{b}]\) is given by 

\[
    A = \int_{t_{a}}^{t_{b}} y(t) \frac{dx}{dt} \, dt = \int_{t_{a}}^{t_{b}} y(t) \frac{dx}{dt} \, dt.
\]

\subsection{Arc Length}

The \cyanit{arc length} of a parametric curve over the \(t\) interval \([t_{a},t_{b}]\) is given by

\[
    s = \int_{t_{a}}^{t_{b}} \mysqrt{\left(\dfrac{dx}{dt}\right)^{2} + \left(\dfrac{dy}{dt}\right)^{2}} \, dt.
\]

\subsection{Surface Area}

The \cyanit{surface area} of the region obtained by rotating a non-self intersecting parametric curve is given by

\[
    S = \int_{t_{a}}^{t_{b}} 2\pi y(t) \dmysqrt{\left(\frac{dx}{dt}\right)^{2} + \left(\frac{dy}{dt}\right)^{2}} \, dt.
\]

\subsection{The Cycloid}

We can apply each of the above to the cycloid:
\begin{itemize}
    \item \cyanit{Derivative}: \(\dydx = \frac{dy}{dx} = \frac{\sin(t)}{1 - \cos(t)}\). Note that the slope is then independent of the radius of the wheel and
    that the slope is undefined at each of \(t = \dots, -4\pi, -2\pi, 0, 2\pi, 4\pi, \dots\). 
    \item \cynit{Concavity}: \(\dfrac{d^{2}y}{dx^{2}} = \frac{d}{dt}\left(\frac{dy}{dx}\right) = \frac{d}{dt}\left(\frac{\sin(t)}{1 - \cos(t)}\right)\). After some work, we find that \(\frac{d^{2}y}{dx^{2}} = -\frac{a}{y^{2}}\), which shows that the cycloid is always concave down.
    \item \cynit{Area}: The area of one period of the cycloid \(A = 3\pi a2\), after some work.

    \begin{align*}
        \frac{d^{2}y}{dx^{2}} &= \frac{d/dt(dy/dx)}{dx/dt} \\
        &= \frac{\frac{d}{dt}\left(\frac{\sin(t)}{1 - \cos(t)}\right)}{a - a \cos(t)} \\
        &= \frac{\frac{\cos(t)(1 - \cos(t)) - \sin(t) \sin(t)}{(1 - \cos t)^{2}}}{a - a \cos(t)} \\
        &= \frac{\cos(t) - \cos^{2} - \sin^{2}(t)}{(1 - \cos(t))^{2}a(1 - \cos(t))} \\
        &= \frac{\cos(t) - 1}{a(1 - \cos(t))^{2}} \\
        &= -\frac{1}{a(1 - \cos(t))^{2}} \\
        &= -\frac{a}{a^{2}(1 - \cos(t))^{2}} \\
        &= -\frac{a}{y^{2}}
    \end{align*}
    
    After some work, we find that \(\frac{d^{2}y}{dx^{2}} = -\frac{a}{y^{2}}\), which shows that the cycloid is always concave down.
    \item \cyanit{Area}: The area of one period of the cycloid \(A = 3\pi a^{2}\), after some work:
    \begin{align*}
        A &= \int_{0}^{2\pi} y(t)x'(t) dt \\
        &= \int_{0}^{2\pi} (a - a \cos t)(a - a\cos t) dt \\
        &= a^{2} \int_{0}^{2\pi} (1 - 2\cos t + \cos^{2} t) dt \\
        &= a^{2} \int_{0}^{2\pi} (1 - 2\cos t + \cos^{2} t) dt \\
        &= a^{2} \left(t + \frac{t}{2} + \frac{1}{4}\sin(2t)\right)\bigg|_{0}^{2\pi} \\
        &= a^{2} \left[\left(2\pi + \frac{2\pi}{2} + \frac{1}{4}\sin(2\pi)\right) - \left(0 + \frac{0}{2} + \frac{1}{4}\sin(0)\right)\right] \\
        &= a^{2} [2\pi + \pi] \\
        &= 3\pi a^{2}.
    \end{align*}

    \item \cynit{Arc Length}: The arc length of one period of the cycloid is \(s = 8a\), again after some work.

    \begin{align*}
        s &= \int_{t_{1}}^{t_{2}} \dmysqrt{\left(\frac{dx}{dt}\right)^{2} + \left(\frac{dy}{dt}\right)^{2}} \ dt \\
        &= \int_{0}^{2\pi} \dmysqrt{(a- a \cos t)^{2} + (a \sin t)^{2}} \ dt \\
        &= a \int_{0}^{2\pi} \dmysqrt{1 - 2\cos t + \cos^{2} t + \sin^{2} t} \ dt \\
        &= a \int_{0}^{2\pi} \dmysqrt{2 - 2\cos t} \ dt \\
        &= \dmysqrt{2}a \int_{0}^{2\pi} \dmysqrt{1 - \cos t} \ dt \\
        &= \dmysqrt{2}a \int_{0}^{2\pi} \dmysqrt{2\sin^{2}\left(\frac{t}{2}\right)} \ dt \\
        &= \dmysqrt{2}a \cdot \dmysqrt{2} \int_{0}^{2\pi} \sin\left(\frac{t}{2}\right) \ dt \\
        &= 2a\left(-2 \cos\left(\frac{t}{2}\right)\right)\bigg|_{0}^{2\pi} \\ 
        &= 8a.
    \end{align*}

    \item \cynit{Surface Area}: The surface area of the solid obtained by rotating one period of the cycloid around the \(x\)-axis is \(S = \frac{64\pi a^{2}}{3}\), after a lot of tedious work.
\end{itemize}
