\section*{4.1-4.3 Functions of Several Variables}

A \textit{scalar function} is a function \(f \colon \R^{n} \to \R\). This is in some sense a reverse of a parametric function. Parametric functions take in a single real variable \(t\) and output an answer of multiple variable (which we typically interpreted as a vector). A scalar function takes in multiple independent variables (typically interpreted as a point in \(\R^{n}\)) and outputs a single real variable.

\subsection{Notation and Graphs}


A scalar function is a function f : Rn → R. This is in some sense a reverse of a parametric function.
Parametric functions take in a single real variable t and output an answer of multiple variable (which
we typically interpreted as a vector). A scalar function takes in multiple independent variables (typically
interpreted as a point in Rn) and outputs a single real variable.
Notation and Graphs
When f : R2 → R, we will often write z = f(x, y). For each point (x, y) in the domain of f, we can interpret
z as a height, so that the set of points (x, y, z) which satisfy z = f(x, y) form a surface in R3. You can graph
this in GeoGebra using geogebra.com/3d.
When f : R3 → R, we often write w = f(x, y, z). Then, the point (x, y, z) is in the domain and w is still a
height, though since the graph lives we are in R4, we cannot draw it in a simple way.
Contour Plots
Let f : R2 → R be a function. The graph is a surface in R3, but often it can be difficult to show this on
paper. For a given z0 ∈ R a level curve consists of all points (x, y) ∈ R2 for which f(x, y) = z0. If we think
of the function f as describing a height at a function of two-dimension position, a level curve is the set of
all points with a particular given height.
A contour plot consists of a collection of level curves – typically, we select some number of equally spaced
{z1, z2, ..., zn} and build the level curves for each. This gives a “topographical map” of the function f.
Limits
Formal Definition of lim(x,y)→(a,b) f(x, y)
Let f : R2 → R, (a, b) ∈ R2, and L ∈ R. The statement that f has limit L as (x, y) approaches (a, b) means
that for each choice of ϵ > 0, there exists δ > 0 such that for all (x, y) ̸= (a, b) within a disk of radius δ,
centered at (a, b), then
|f(x, y) − L| < ϵ.
Essentially, this says that the limit exists whenever we can make the output from f as close to L as we’d
like by choosing (x, y) sufficiently close to (a, b).
In Calculus I, you showed that a limit limx→a f(x) exists if and only if the left and right hand limits both
exist and match. Basically, you showed that no matter if you approach a from the left or from the right,
1
you always get the same answer. The same idea applies here, but there is an important complication: We
need to check every possible path into the point (a, b) – and rather than just two, there are infinitely many
paths!
In general, there is not a simple recipe that will work. You can try various combinations of paths until
something goes wrong. For practice set/exam purposes, I will always tell you that the limit does not exist,
and ask you to prove it. Showing that a limit does exist is a completely different procedure – just because
two paths happen to produce the same answer does not prove that all do.
Problem #4 on Practice Set #3 will ask you to show a limit exists. It provides step-by-step instructions and
this is the only time I will ask you to do this.
Limit Laws
The basic rules are all the same. Suppose that a, b, c ∈ R and f, g : R2 → R and that lim(x,y)→(a,b) f(x, y) = L
and lim(x,y)→(a,b) g(x, y) = M.
ˆ lim(x,y)→(a,b) c = c
ˆ lim(x,y)→(a,b) x = a
ˆ lim(x,y)→(a,b) y = b
ˆ lim(x,y)→(a,b) [f(x, y) + g(x, y)] = L +M.
ˆ lim(x,y)→(a,b) [cf(x, y)] = cL
ˆ if f has a single, simple algebraic definition and (a, b) is in the domain, then lim(x,y)→(a,b) f(x, y) =
f(a, b).
Continuity
A function f : R2 → R is continuous at (a, b) ∈ R2 provided that lim(x,y)→(a,b) f(x, y) exists and is equal
in value to f(a, b). This is exactly our usual definition of continuity from Calculus I again – and there is
nothing special here about being in two dimensions.
All of the usual algebraic functions (polynomials, root functions, trig functions, exponentials, and logarithms)
and their combinations by addition, subtraction, multiplication, division, exponentiation, and composition
are continuous everywhere in their domains.
Partial Derivatives
Suppose f : Rn → R is a scalar function. Then, the partial derivative of f with respect to xi is
∂
∂xi
f = lim
h→0
f(x1, x2, ..., xi + h, ...xn) − f(x1, x2, ..., xi, ...xn)
h
,
2
if this limit exists.
That is, we hold all other variables constant and take the derivative as though xi is the only variable.
Just like in Calculus I, the partial derivative is a rate of change – it measures the ratio of how much the
output of f changes relative to a small change in xi.
Notation
Suppose that f is a function and x is one of its variables. Then, we denote the partial derivative of f with
respect to x by any of:
ˆ Leibnitz/Condorcet Notation:
∂f
∂x
ˆ Jacobi Notation: fx
ˆ Euler Operation Notation: Dxf
Higher order derivatives are notated as:
ˆ Leibnitz/Condorcet Notation:
∂nf
∂xn
ˆ Jacobi Notation: fxxx···x
ˆ Euler Operation Notation: Dn
xf
Mixed Partials
Given a function f : R2 → R we can determine the mixed partial derivative of f with respect to x and then
y. In Leibnitz notation, this would be
∂
∂y
∂f
∂x
[notice the order!] or in Jacobi notation
fxy.
Clairaut’s Theorem
Suppose that f : R2 → R is defined on an open disk D containing the point (a, b). If each of fxy and fyx are
continuous on D, then fxy(a, b) = fyx(a, b).
That is, for pretty much any function you’ll even encounter, order does not matter for the mixed partial
derivatives! This is especially nice, since notationally, ∂2
∂x∂y f = fyx and ∂2
∂y∂xf = fxy, and who wants to
worry about keeping up with that!