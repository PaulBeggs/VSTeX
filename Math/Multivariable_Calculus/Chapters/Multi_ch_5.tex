\section{Double Integrals}

\subsection{Integration in \(\R\)}

Recall that for a continuous function \(f : \R \to \R\), the \cyanit{signed area} between the graph of \(y = f(x)\) and the
\(x\)-axis on the interval \([a, b]\) is given by
\[
  \int_a^b f(x) \, dx.
\]
Recall also that
\[
  \int_a^b f(x) \, dx \approx \sum_{i=1}^n f(x^*_i) \Delta x,
\]
where \(\Delta x = \frac{b-a}{n}\) and \(x^*_i\) is any convenient point in the \(i\)th subinterval. The sum above is called a \cyanit{Riemann Sum}.

\subsection{Integration in \(\R^2\)}

Let \(R\) be a rectangle defined by

\[
  R = [a, b] \times [c, d] = \{(x, y) \in \R^2 : a \leq x \leq b, \, c \leq y \leq d\}.
\]
Suppose that \(f : R \to \R\) be continuous. If we interpret \(f(x, y)\) as a height above or below the \(xy\)-plane, we can find the \cyanit{signed volume} of the solid
\[
  S = \{(x, y, z) \in \R^3 : (x, y) \in R, z \text{ lives between } 0 \text{ and } f(x, y)\}.
\]
We can divide \(R\) into \(m \cdot n\) rectangles by
\[
  \Delta x = \frac{b-a}{m} \quad \text{and} \quad \Delta y = \frac{d-c}{n}
\]
and then write
\[
  \Delta A = \Delta x \Delta y.
\]
Then,
\[
  V \approx \sum_{i=1}^m \sum_{j=1}^n f(x^*_{ij}, y^*_{ij}) \Delta A,
\]
where \((x^*_{ij}, y^*_{ij})\) is a point selected from the \(ij\)th rectangle.
\newpage
If \(f\) is continuous, then
\[
  \lim_{m,n \to \infty} \sum_{i=1}^m \sum_{j=1}^n f(x^*_{ij}, y^*_{ij}) \Delta A
\]
exists thus we define the \cyanit{double integral} of \(f\) over the rectangle \(R\) as
\[
  \iint_R f(x, y) \, dA = \lim_{m,n \to \infty} \sum_{i=1}^m \sum_{j=1}^n f(x^*_{ij}, y^*_{ij}) \Delta A.
\]
\subsubsection{Double Integral – Properties}
Basically, the properties are exactly what you’d expect – we’ll assume \(R\) is a rectangle, \(c\) is constant, and \(f, g\) are continuous:
\begin{itemize}
  \item \(\iint_R (f + g) \, dA = \iint_R f \, dA + \iint_R g \, dA\)
  \item \(\iint_R cf \, dA = c \iint_R f \, dA\)
  \item \(\iint_R f \, dA = \iint_S f \, dA + \iint_T f \, dA\), if \(S\) and \(T\) are rectangles, \(R = S \cup T\) and \(S \cap T\) is one-dimensional
  \item if \(f \geq g\) on \(R\), then \(\iint_R f \, dA \geq \iint_R g \, dA\)
  \item if \(m \leq f(x, y) \leq M\) on \(R\) then \(m \cdot \text{area}(R) \leq \iint_R f \, dA \leq M \cdot \text{area}(R)\)
\end{itemize}
\subsubsection{Factorization Property}
Suppose that \(f(x, y) = g(x) \cdot h(y)\) – that is, suppose we can factor \(f\) into a product of one function involving only \(x\) and another involving only \(y\). Then,
\[
  \iint_R f(x, y) \, dA = \left( \int_a^b g(x) \, dx \right) \left( \int_c^d h(y) \, dy \right).
\]
\subsubsection{Iterated Integrals}
For a fixed \(x\), define
\[
  A(x) = \int_c^d f(x, y) \, dy.
\]
This can be called a \cyanit{partial integral}, and is a function of \(x\). It is the area of one slice of the solid. Then,
\[
  \int_a^b A(x) \, dx = \int_a^b \left( \int_c^d f(x, y) \, dy \right) dx = \int_a^b \int_c^d f(x, y) \, dy dx,
\]

where we call this an \cyanit{iterated integral}. (There is, of course, nothing special about doing \(y\) first and then \(x\) here\ldots.)

\subsubsection{Fubini’s Theorem}

If \(f\) is continuous on \(R = [a, b] \times [c, d]\) then
\[
  \iint_R f(x, y) \, dA = \int_b^a \int_d^c f(x, y) \, dy dx = \int_d^c \int_b^a f(x, y) \, dx dy.
\]
This is just like Clariaut’s Theorem, but for integrals, rather than partial derivatives. You can work the integral in whatever order is most convenient.

\newpage

\section{Double Integrals Over General Regions}

\subsection{General Regions}

Suppose that \(D\) is a general region of finite size in \(\R^2\). Then, there exists some rectangle \(R\) which contains \(D\). We can take a function \(f(x, y)\) and define
\[
  F(x, y) =
  \begin{cases}
    f(x, y), & \text{if } (x, y) \in D, \\ 0, & \text{if otherwise}.
  \end{cases}
\]
This does likely introduce discontinuities into our function, but is not a problem – trust me. Then,
\[
  \iint_R F(x, y) \, dA = \iint_D f(x, y) \, dA.
\]
So, we can always actually integrate over a rectangle!

\subsubsection{Two Special Cases}

Suppose we have a general region \(D\). Then,
\begin{itemize}
  \item \textbf{Type I Region} – we say that \(D\) is a Type I region provided there exists constants \(a\), \(b\) and continuous functions \(g_1\), \(g_2 : \R \to \R\) so that
        \[
          D = \{(x, y) : a \leq x \leq b, \text{ and } g_1(x) \leq y \leq g_2(x)\}.
        \]
  \item \textbf{Type II Region} – we say that \(D\) is a Type II region provided there exists constants \(c\), \(d\) and continuous functions \(h_1\), \(h_2 : \R \to \R\) so that
        \[
          D = \{(x, y) : h_1(y) \leq x \leq h_2(y), \text{ and } c \leq y \leq d\}.
        \]
\end{itemize}
\subsubsection{Type I Regions}

Suppose that \(D\) is a type I region:
\begin{align*}
  \iint_D f(x, y) \, dA & = \iint_R F(x, y) \, dA                             \\
                        & = \int_b^a \int_{g_1(x)}^{g_2(x)} f(x, y) \, dy dx.
\end{align*}
The ``see below'' line is true since \(F(x, y) = 0\) if \(y > g_2(x)\) or \(y < g_1(x)\).

\subsubsection{Type II Regions}

In the same way, if \(D\) is type II, we have
\[
  \iint_D f(x, y) \, dA = \int_d^c \int_{h_1(y)}^{h_2(y)} f(x, y) \, dx dy.
\]
How do you tell? DRAW A PICTURE! (In practice, you don’t typically explicitly note what type an integral is.)

\subsubsection{Area}

Suppose that \(D\) is a region. Then, the \cyanit{area} of \(D\) is given by
\[
  \text{area}(D) = \iint_D 1 \, dA.
\]

\subsubsection{Average Value}

The \cyanit{average value} of \(f\) over \(D\) is given by
\[
  \text{ave}(f) = \frac{1}{\text{area}(D)} \iint_D f(x, y) \, dA.
\]

\newpage

\section{Double Integrals and Polar Coordinates}

\subsection{Polar Coordinates}

Let \((x, y) \in \R^2\). Then, in polar coordinates
\begin{align*}
  x & = r \cos(\theta) \\
  y & = r \sin(\theta)  \\
  r & = \sqrt{x^2 + y^2} \\
  \theta & = \arctan\left(\frac{y}{x}\right),
\end{align*}
and recall that \(\theta\) is not uniquely determined by the last identity.

\subsection{Polar Rectangle}

A region \(R \subseteq \R^2\) is called a polar rectangle provided it can be written as
\[
  R = \{(r, \theta) : a \leq r \leq b, \alpha \leq \theta \leq \beta\},
\]
for choices of \(a\), \(b\), \(\alpha\), \(\beta\). \\

We can divide each polar rectangle into smaller ones by
\[
  \Delta r = \frac{b-a}{m} \quad \text{and} \quad \Delta \theta = \frac{\beta-\alpha}{n}.
\]
(This should look very familiar!!) Then, a little polar rectangle \(R_{ij} = \{(r, \theta) : r_{i-1} \leq r \leq r_i, \theta_{j-1} \leq \theta \leq \theta_j\}\) has its center at
\begin{align*}
  r^*_i & = \frac{1}{2}(r_{i-1} + r_i) \\
  \theta^*_i & = \frac{1}{2}(\theta_{j-1} + \theta_j)
\end{align*}

The area of this polar rectangle is
\begin{align*}
  \Delta A & = \frac{1}{2}r_i^2 \Delta \theta - \frac{1}{2}r_{i-1}^2 \Delta \theta \\
           & = \frac{1}{2}\left(r_i^2 - r_{i-1}^2\right) \Delta \theta \\
           & = \frac{1}{2}(r_i + r_{i-1})(r_i - r_{i-1}) \Delta \theta \\
           & = r^*_i \Delta r \Delta \theta
\end{align*}

\subsection{Polar Integrals}

\begin{align*}
  \iint_{R}f(x,y) \, dA & = \lim_{m,n \to \infty} \sum_{i=1}^{m} \sum_{j=1}^{n} f(x^*_{ij}, y^*_{ij}) \Delta A \\
                        & = \lim_{m,n \to \infty} \sum_{i=1}^{m} \sum_{j=1}^{n} f\bigl(r^*_i \cos\left(\theta^*_j\right), r^*_i \sin\left(\theta^*_j\right)\bigr) r^*_i \Delta r \Delta \theta \\
                        & = \int_{\alpha}^{\beta} \int_{a}^{b} f\bigl(r \cos(\theta), r \sin(\theta \bigr) r \, dr \, d\theta
\end{align*}

Thus, we can convert each \(x\) to \(r \cos(\theta)\), each \(y\) to \(r \sin(\theta)\) and replace \(dA = dxdy\) with \(dA = r \, dr d\theta\). You have made a change in coordinates and are now integrating over a polar region, so the results of sections 5.1 \& 5.2 now apply.

\subsection{Change of Coordinates}

Suppose that we wish to change from one two-dimensional coordinate system to another.

\begin{itemize}
  \item Initial: \((x, y)\)
  \item New: \((\alpha, \beta)\): That is, we have \(x(\alpha, \beta)\) and \(y(\alpha, \beta)\).
  \item Then, 
  \[
    dA = dxdy = \left(\frac{\p x}{\p \alpha} \cdot \frac{\p y}{\p \beta} - \frac{\p y}{\p \alpha} \cdot \frac{\p y}{\p \beta}\right) d\alpha d\beta
  \]
\end{itemize}

This idea is called the \cyanit{Jacobian}. We can think of this as being the determinant of the Jacobian Matrix:

\[
  J = \begin{bmatrix} \frac{\partial x}{\partial \alpha} & \frac{\partial x}{\partial \beta} \\ \frac{\partial y}{\partial \alpha} & \frac{\partial y}{\partial \beta} \end{bmatrix}
\]