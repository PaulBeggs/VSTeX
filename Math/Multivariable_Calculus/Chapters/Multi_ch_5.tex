\section{Double Integrals}

\subsection{Integration in \(\R\)}

Recall that for a continuous function \(f : \R \to \R\), the \cyanit{signed area} between the graph of \(y = f(x)\) and the
\(x\)-axis on the interval \([a, b]\) is given by 
\[
\int_a^b f(x) \, dx.
\]
Recall also that
\[
\int_a^b f(x) \, dx \approx \sum_{i=1}^n f(x^*_i) \Delta x,
\]
where \(\Delta x = \frac{b-a}{n}\) and \(x^*_i\) is any convenient point in the \(i\)th subinterval. The sum above is called a \cyanit{Riemann Sum}.

\subsection{Integration in \(\R^2\)}

Let \(R\) be a rectangle defined by

\[
R = [a, b] \times [c, d] = \{(x, y) \in \R^2 : a \leq x \leq b, \, c \leq y \leq d\}.
\]
Suppose that \(f : R \to \R\) be continuous. If we interpret \(f(x, y)\) as a height above or below the \(xy\)-plane, we can find the \cyanit{signed volume} of the solid
\[
S = \{(x, y, z) \in \R^3 : (x, y) \in R, z \text{ lives between } 0 \text{ and } f(x, y)\}.
\]
We can divide \(R\) into \(m \cdot n\) rectangles by
\[
\Delta x = \frac{b-a}{m} \quad \text{and} \quad \Delta y = \frac{d-c}{n}
\]
and then write 
\[
\Delta A = \Delta x \Delta y.
\]
Then,
\[
V \approx \sum_{i=1}^m \sum_{j=1}^n f(x^*_{ij}, y^*_{ij}) \Delta A,
\]
where \((x^*_{ij}, y^*_{ij})\) is a point selected from the \(ij\)th rectangle.
\newpage
If \(f\) is continuous, then
\[
\lim_{m,n \to \infty} \sum_{i=1}^m \sum_{j=1}^n f(x^*_{ij}, y^*_{ij}) \Delta A
\]
exists thus we define the \cyanit{double integral} of \(f\) over the rectangle \(R\) as
\[
\iint_R f(x, y) \, dA = \lim_{m,n \to \infty} \sum_{i=1}^m \sum_{j=1}^n f(x^*_{ij}, y^*_{ij}) \Delta A.
\]
\subsubsection{Double Integral – Properties}
Basically, the properties are exactly what you’d expect – we’ll assume \(R\) is a rectangle, \(c\) is constant, and \(f, g\) are continuous:
\begin{itemize}
    \item \(\iint_R (f + g) \, dA = \iint_R f \, dA + \iint_R g \, dA\)
    \item \(\iint_R cf \, dA = c \iint_R f \, dA\)
    \item \(\iint_R f \, dA = \iint_S f \, dA + \iint_T f \, dA\), if \(S\) and \(T\) are rectangles, \(R = S \cup T\) and \(S \cap T\) is one-dimensional
    \item if \(f \geq g\) on \(R\), then \(\iint_R f \, dA \geq \iint_R g \, dA\)
    \item if \(m \leq f(x, y) \leq M\) on \(R\) then \(m \cdot \text{area}(R) \leq \iint_R f \, dA \leq M \cdot \text{area}(R)\)
\end{itemize}
\subsubsection{Factorization Property}  
Suppose that \(f(x, y) = g(x) \cdot h(y)\) – that is, suppose we can factor \(f\) into a product of one function involving only \(x\) and another involving only \(y\). Then,
\[
\iint_R f(x, y) \, dA = \left( \int_a^b g(x) \, dx \right) \left( \int_c^d h(y) \, dy \right).
\]
\subsubsection{Iterated Integrals}
For a fixed \(x\), define
\[
A(x) = \int_c^d f(x, y) \, dy.
\]
This can be called a \cyanit{partial integral}, and is a function of \(x\). It is the area of one slice of the solid. Then,
\[
\int_a^b A(x) \, dx = \int_a^b \left( \int_c^d f(x, y) \, dy \right) dx = \int_a^b \int_c^d f(x, y) \, dy dx,
\]

where we call this an \cyanit{iterated integral}. (There is, of course, nothing special about doing \(y\) first and then \(x\) here\ldots.)

\subsubsection{Fubini’s Theorem}

If \(f\) is continuous on \(R = [a, b] \times [c, d]\) then
\[
\iint_R f(x, y) \, dA = \int_b^a \int_d^c f(x, y) \, dy dx = \int_d^c \int_b^a f(x, y) \, dx dy.
\]
This is just like Clariaut’s Theorem, but for integrals, rather than partial derivatives. You can work the integral in whatever order is most convenient.
\subsubsection{Area}
Suppose that \(R\) is a rectangle. Then, the area of \(R\) is given by
\[
\text{area}(R) = \iint_R 1 \, dA.
\]

\subsubsection{Average Value}

The \cyanit{average value} of \(f\) over \(R\) is given by
\[
\text{ave}(f) = \frac{1}{\text{area}(R)} \iint_R f(x, y) \, dA.
\]

\begin{tikzcd}
    Y \arrow[r] \arrow[d] 
      & X \arrow[d] \\
    T \arrow[r]
      & S
\end{tikzcd}

\(\frac{d}{dx}(x^{3} + x^{2} + 1)\)