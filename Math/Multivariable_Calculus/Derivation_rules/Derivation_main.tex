\documentclass[12pt]{article}
\usepackage{multicol}
\usepackage{C:/Users/paulb/VSTeX/local/draculatheme}
\usepackage{amsmath}
\usepackage{xcolor}
\usepackage{booktabs}
\usepackage{tikz}
\usepackage[left=0.5in, right=0.5in, top=1in, bottom=1in]{geometry}
\DeclareMathOperator{\arccot}{arccot}
\DeclareMathOperator{\arcsec}{arcsec}
\DeclareMathOperator{\arccsc}{arccsc}

\usepackage{titlesec}

\newcommand{\mysqrt}[1]{%
  \mathpalette\foo{#1}%
}
\newcommand{\dmysqrt}[1]{%
  \mathpalette\foodisplay{#1}%
}

% !TeX spellcheck = off
\newcommand{\foo}[2]{%
  % #1: math style, #2: content
  \sbox0{$#1\sqrt{#2}$}% Measure the size of the standard sqrt in the current style
  \begin{tikzpicture}[baseline=(sqrt.base)]
    \node[inner sep=0, outer sep=0] (sqrt) {$#1\sqrt{#2}$}; % Use the current math style
    \draw([yshift=-0.045em]sqrt.north east) -- ++(0,-0.5ex); % Draw the tick
  \end{tikzpicture}%
}
% !TeX spellcheck = off
\newcommand{\foodisplay}[2]{%
  % #1: math style, #2: content
  \sbox0{$#1\sqrt{#2}$}% Measure the size of the standard sqrt in the current style
  \begin{tikzpicture}[baseline=(sqrt.base)]
    \node[inner sep=0, outer sep=0] (sqrt) {$\displaystyle\sqrt{#2}$}; % Force displaystyle
    \draw[line width=0.4pt] ([yshift=-0.044em]sqrt.north east) -- ++(0,-0.5ex); % Draw the tick
  \end{tikzpicture}%
}

% Adjust spacing before and after section headers
\titlespacing*{\section}{0pt}{0.5ex}{0.5ex}

% Optionally, adjust section spacing similarly
\titlespacing*{\section}{0pt}{0.5ex}{0.5ex}
\titlespacing*{\subsection}{0pt}{0.5ex}{0.5ex}

\begin{document}

\begin{multicols}{2}

  \subsection*{Basic Derivatives}

  \begin{align*}
     & \frac{d}{dx} e^{f(x)} =            & f'(x)e^{f(x)}                    \\
     & \frac{d}{dx} \sin f(x) =           & \cos f(x) \cdot f'(x)            \\
     & \frac{d}{dx} \cos f(x) =           & -\sin f(x) \cdot f'(x)           \\
     & \frac{d}{dx} \tan f(x) =           & \sec^2 f(x) \cdot f'(x)          \\
     & \frac{d}{dx} \cot f(x) =           & -\csc^2 f(x) \cdot f'(x)         \\
     & \frac{d}{dx} \sec f(x) =           & \sec f(x) \tan f(x) \cdot f'(x)  \\
     & \frac{d}{dx} \csc f(x) =           & -\csc f(x) \cot f(x) \cdot f'(x) \\
     & \frac{d}{dx} \ln f(x) =            & \frac{f'(x)}{f(x)}               \\
     & \frac{d}{dx} \log_a f(x) =         & \frac{f'(x)}{f(x) \ln a}         \\
     & \frac{d}{dx} \left(f(x)\right)^n = & n \left(f(x)\right)^{n-1} f'(x)  \\
     & \frac{d}{dx} \dmysqrt{f(x)} =      & \frac{f'(x)}{2\dmysqrt{f(x)}}    \\
     & \frac{d}{dx} a^x =                 & a^x \ln a                        \\
     & \frac{d}{dx} b^{g(x)} =            & b^{g(x)} \ln b \cdot g'(x)       \\
  \end{align*}

  \vfill

  \subsection*{Chain Rule}

  \begin{align*}
     & \frac{d}{dx} f(g(x)) = & f'(g(x)) \cdot g'(x) \\
  \end{align*}

  \subsection*{Higher-Order Derivatives}

  \begin{align*}
     & \frac{d^2}{dx^2} e^x =    & e^x     \\
     & \frac{d^3}{dx^3} \sin x = & -\cos x \\
     & \frac{d^4}{dx^4} \cos x = & \cos x  \\
  \end{align*}

  \subsection*{Inverse Trigonometric}

  \begin{align*}
     & \frac{d}{dx} \arcsin f(x) = & \frac{f'(x)}{\dmysqrt{1 - \left(f(x)\right)^2}}        \\
     & \frac{d}{dx} \arccos f(x) = & -\frac{f'(x)}{\dmysqrt{1 - \left(f(x)\right)^2}}       \\
     & \frac{d}{dx} \arctan f(x) = & \frac{f'(x)}{1 + \left(f(x)\right)^2}                  \\
     & \frac{d}{dx} \arccot f(x) = & -\frac{f'(x)}{1 + \left(f(x)\right)^2}                 \\
     & \frac{d}{dx} \arcsec f(x) = & \frac{f'(x)}{|f(x)|\dmysqrt{\left(f(x)\right)^2 - 1}}  \\
     & \frac{d}{dx} \arccsc f(x) = & -\frac{f'(x)}{|f(x)|\dmysqrt{\left(f(x)\right)^2 - 1}} \\
  \end{align*}

  \subsection*{Hyperbolic Function}

  \begin{align*}
     & \frac{d}{dx} \sinh f(x) =         & \cosh f(x) \cdot f'(x)                     \\
     & \frac{d}{dx} \cosh f(x) =         & \sinh f(x) \cdot f'(x)                     \\
     & \frac{d}{dx} \tanh f(x) =         & \text{sech}^2 f(x) \cdot f'(x)             \\
     & \frac{d}{dx} \coth f(x) =         & -\text{csch}^2 f(x) \cdot f'(x)            \\
     & \frac{d}{dx} \text{sech}\, f(x) = & -\text{sech}\, f(x) \tanh f(x) \cdot f'(x) \\
     & \frac{d}{dx} \text{csch}\, f(x) = & -\text{csch}\, f(x) \coth f(x) \cdot f'(x) \\
  \end{align*}

  \subsection*{Product and Quotient}

  \begin{align*}
     & \frac{d}{dx} [u \cdot v] =                & u' \cdot v + u \cdot v'             \\
     & \frac{d}{dx} \left( \frac{u}{v} \right) = & \frac{u' \cdot v - u \cdot v'}{v^2} \\
  \end{align*}

\end{multicols}

\newpage

\begin{table}[htbp]
  \centering
  \begin{tabular}{ccccc}
    \toprule
    \(\theta\)      & Radians    & \(\sin(\theta)\) & \(\cos(\theta)\) & \(\tan(\theta)\) \\  \midrule
    \(0^{\circ}\)   & \(0\)      & \(0\)            & \(1\)            & \(0\)            \\  \midrule
    \(30^{\circ}\)  & \(\pi/6\)  & \(1/2\)          & \(\mysqrt{3}/2\) & \(\mysqrt{3}/3\) \\  \midrule
    \(45^{\circ}\)  & \(\pi/4\)  & \(\mysqrt{2}/2\) & \(\mysqrt{2}/2\) & \(1\)            \\  \midrule
    \(60^{\circ}\)  & \(\pi/3\)  & \(\mysqrt{3}/2\) & \(1/2\)          & \(\mysqrt{3}\)   \\  \midrule
    \(90^{\circ}\)  & \(\pi/2\)  & \(1\)            & \(0\)            & \(-\)            \\  \midrule
    \(180^{\circ}\) & \(\pi\)    & \(0\)            & \(-1\)           & \(0\)            \\  \midrule
    \(270^{\circ}\) & \(3\pi/2\) & \(-1\)           & \(0\)            & \(-\)            \\
    \bottomrule
  \end{tabular}
  \caption{Important Trigonometric Angles}\label{tab:angles}
\end{table}

\section*{Trigonometric Identities}

\begin{multicols}{2}
  \subsection*{Pythagorean}
  \begin{align*}
    & \sin^{2}\theta + \cos^{2}\theta = & 1 \\
    & \tan^{2}\theta + 1 = & \sec^{2}\theta \\
    & 1 + \cot^{2}\theta = & \csc^{2}\theta
  \end{align*}
  \subsection*{Reciprocal}
  \begin{align*}
    & \csc x = & \frac{1}{\sin x} \\
    & \sec x = & \frac{1}{\cos x} \\
    & \cot x = & \frac{1}{\tan x}
  \end{align*}
  \subsection*{Even and Odd}
  \begin{align*}
    & \sin(-x) = & -\sin x \\
    & \cos(-x) = & \cos x \\
    & \tan(-x) = & -\tan x
  \end{align*}
  \subsection*{Product to Sum}
  \begin{align*}
    & \sin x \sin y = & \frac{1}{2}[\cos(x-y) - \cos(x+y)] \\
    & \cos x \cos y = & \frac{1}{2}[\cos(x-y) + \cos(x+y)] \\
    & \sin x \cos y = & \frac{1}{2}[\sin(x+y) + \sin(x-y)] \\
    & \cos x \sin y = & \frac{1}{2}[\sin(x+y) - \sin(x-y)]
  \end{align*}
  \subsection*{Sum to Product}
  \begin{align*}
    & \sin x + \sin y = & 2\sin\left(\frac{x+y}{2}\right)\cos\left(\frac{x-y}{2}\right) \\
    & \sin x - \sin y = & 2\cos\left(\frac{x+y}{2}\right)\sin\left(\frac{x-y}{2}\right) \\
    & \cos x + \cos y = & 2\cos\left(\frac{x+y}{2}\right)\cos\left(\frac{x-y}{2}\right) \\
    & \cos x - \cos y = & -2\sin\left(\frac{x+y}{2}\right)\sin\left(\frac{x-y}{2}\right)
  \end{align*}
  \subsection*{Double Angle}
\end{multicols}

\end{document}