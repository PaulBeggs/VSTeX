%-%-%-%-%-%-%-%-%-%-%-%-%-%-%-%-%-%-%-%-%-%-%-%-%-%-%-%-%-%-%-%-%-%-%-%-%-%-%
%%% MAIN DOCUMENT %%%

%-%-%-%-%-%-%-%-%-%-%-%-%-%-%-%-%-%-%-%-%-%-%-%-%-%-%-%-%-%-%-%-%-%-%-%-%-%-%

\documentclass[12pt, oneside]{book}
\usepackage{xcolor}
% \usepackage{minted}

% \usepackage[T1]{fontenc}

\usepackage{draculatheme}
\newcommand{\documentTheme}{draculafg}
\newcommand{\documentLogo}{images/small logo_white.png}
\newcommand{\documentBigLogo}{images/logo_white_text.png}
% \usemintedstyle{dracula}

% \newcommand{\documentBigLogo}{images/Hendrix Logo.png}
% \newcommand{\documentLogo}{images/small logo.png}
% \newcommand{\documentTheme}{black}

%%% AESTHETICS %%%
%-%-%-%-%-%-%-%-%-%-%-%-%-%-%-%-%-%-%-%-%-%-%-%-%-%-%-%-%-%-%-%-%-%-%-%-%-%-%


%%% Dimensions and Spacing %%%
\usepackage[margin=1in]{geometry}
\setlength{\parindent}{0pt}
\usepackage{setspace}
\usepackage{mathtools}
\usepackage{esint}
\linespread{1}
\usepackage{listings}
\usepackage{tikz}
\usetikzlibrary{shapes,backgrounds,calc,patterns,positioning}
\usepgflibrary{shadings}
\usepackage{pgfkeys}
\usepackage{pdftexcmds}
\usepackage{shellesc}
\usepackage{algorithm, algorithmic, xspace}
%%% Define new colors %%%
\definecolor{orangehdx}{rgb}{0.96, 0.51, 0.16}

% Normal colors
\definecolor{xred}{HTML}{BD4242}
\definecolor{xblue}{HTML}{4268BD}
\definecolor{xgreen}{HTML}{52B256}
\definecolor{xpurple}{HTML}{7F52B2}
\definecolor{xorange}{HTML}{FD9337}
\definecolor{xdotted}{HTML}{999999}
\definecolor{xgray}{HTML}{777777}
\definecolor{xcyan}{HTML}{80F5DC}
\definecolor{xpink}{HTML}{F690EA}
\definecolor{xgrayblue}{HTML}{49B095}
\definecolor{xgraycyan}{HTML}{5AA1B9}

% Dark colors
\colorlet{xdarkred}{red!85!black}
\colorlet{xdarkblue}{xblue!85!black}
\colorlet{xdarkgreen}{xgreen!85!black}
\colorlet{xdarkpurple}{xpurple!85!black}
\colorlet{xdarkorange}{xorange!85!black}
\definecolor{xdarkcyan}{HTML}{008B8B}
\colorlet{xdarkgray}{xgray!85!black}

% Very dark colors
\colorlet{xverydarkblue}{xblue!50!black}

% Document-specific colors
\colorlet{normaltextcolor}{black}
\colorlet{figtextcolor}{xblue}

% Enumerated colors
\colorlet{xcol0}{black}
\colorlet{xcol1}{xred}
\colorlet{xcol2}{xblue}
\colorlet{xcol3}{xgreen}
\colorlet{xcol4}{xpurple}
\colorlet{xcol5}{xorange}
\colorlet{xcol6}{xcyan}
\colorlet{xcol7}{xpink!75!black}

% Blue-Purple (should just used colorbrewer...)
\definecolor{xrainbow0}{HTML}{e41a1c}
\definecolor{xrainbow1}{HTML}{a24057}
\definecolor{xrainbow2}{HTML}{606692}
\definecolor{xrainbow3}{HTML}{3a85a8}
\definecolor{xrainbow4}{HTML}{42977e}
\definecolor{xrainbow5}{HTML}{4aaa54}
\definecolor{xrainbow6}{HTML}{629363}
\definecolor{xrainbow7}{HTML}{7e6e85}
\definecolor{xrainbow8}{HTML}{9c509b}
\definecolor{xrainbow9}{HTML}{c4625d}
\definecolor{xrainbow10}{HTML}{eb751f}
\definecolor{xrainbow11}{HTML}{ff9709}

%------- %
% XHFILL %
%------- %



%%% Chapter Headings %%%

\newcommand{\gradientrule}{
    \begin{tikzpicture}
        \shade[left color=orangehdx, right color=black, middle color=gray] (0,0) rectangle (\linewidth,0.4pt);
    \end{tikzpicture}
}

\usepackage[Glenn]{fncychap}
\ChTitleVar{\bfseries\scshape\color{\documentTheme}} % Needed for Dracula theme
\ChNumVar{\large\selectfont\color{\documentTheme}} % Needed for Dracula theme
\ChNameVar{\large\color{\documentTheme}} % Needed for Dracula theme
\usepackage{xpatch}

% \xpatchcmd{\DOCH}
%   {\mghrulefill}{\gradientrule\mghrulefill}
%   {}{\PatchFailed}
% \xpatchcmd{\DOTI}
%   {\mghrulefill}{\gradientrule\mghrulefill}
%   {}{\PatchFailed}
% \xpatchcmd{\DOTIS}
%   {\mghrulefill}{\gradientrule\mghrulefill}
%   {}{\PatchFailed}

\xpatchcmd\DOCH
{\mghrulefill}{\color{orangehdx}\mghrulefill}
{}{\PatchFailed}
\xpatchcmd\DOTI
{\mghrulefill}{\color{orangehdx}\mghrulefill}
{}{\PatchFailed}
\xpatchcmd\DOTIS
{\mghrulefill}{\color{orangehdx}\mghrulefill}
{}{\PatchFailed}


% \usepackage[Bjornstrup]{fncychap}

% \newcommand{\gradient}[1]{
% \begin{tikzpicture}
%     \node (rect) at (0,0) [fill=blue,,path fading=East,minimum width=\linewidth,minimum height=2.5cm] {};
%     \node(title)[above left = 10pt and 10pt of rect.south east, anchor=south east, font=\CTV] {\textcolor{horange}{#1}};
%     \ifnum \thechapter>0\node[left = 10pt of rect.north east,  anchor=center, font=\CNoV] {\textcolor{horange}{\thechapter}};\fi%
% \end{tikzpicture}%
% \vskip 40pt
% }


% \renewcommand{\DOCH}{}
% \renewcommand{\DOTI}[1]{\gradient{#1}}
% \renewcommand{\DOTIS}[1]{\gradient{#1}}

%% Change Chapter Heading Placement %%
\usepackage{etoolbox}
\makeatletter
\patchcmd{\@makechapterhead}{\vspace*{50\p@}}{\vspace*{-20\p@}}{}{}
\patchcmd{\@makeschapterhead}{\vspace*{50\p@}}{\vspace*{-20\p@}}{}{}
\patchcmd{\DOTI}{\vskip 80\p@}{\vskip 40\p@}{}{}
\patchcmd{\DOTIS}{\vskip 40\p@}{\vskip 0\p@}{}{}
\makeatother

\usepackage{titlesec}

% Section title: 90% opacity
\titleformat{\section}
  {\normalfont\Large\bfseries\color{draculapink!100}}
  {\thesection}{1em}{}

% Subsection title: 80% opacity
\titleformat{\subsection}
  {\normalfont\large\bfseries\color{draculapink!75}}
  {\thesubsection}{1em}{}

% Subsubsection title: 70% opacity
\titleformat{\subsubsection}
  {\normalfont\normalsize\bfseries\color{draculapink!50}}
  {\thesubsubsection}{1em}{}


% \newcommand*\chapterlabel{}\pmod
% \titleformat{\chapter}
%   {\gdef\chapterlabel{}
%    \normalfont\sffamily\Huge\bfseries\scshape}
%   {\gdef\chapterlabel{\thechapter\ }}{0pt}
%   {\begin{tikzpicture}[remember picture,overlay]
%     \node[yshift=-3cm] at (current page.north west)
%       {\begin{tikzpicture}[remember picture, overlay]
%         \draw[fill=LightSkyBlue] (0,0) rectangle
%           (\paperwidth,3cm);
%         \node[anchor=east,xshift=.9\paperwidth,rectangle,
%               sharp corners=downhill=20pt,inner sep=11pt,
%               fill=MidnightBlue]
%               {\color{white}\chapterlabel#1};
%        \end{tikzpicture}
%       };
%    \end{tikzpicture}
%   }
% \titlespacing*{\chapter}{0pt}{50pt}{-60pt}

% \usepackage[Conny]{fncychap}
% \usepackage[Rejne]{fncychap}
% \ChNameVar{\bfseries}  % Makes the chapter "Chapter #" bold
% \ChNumVar{\bfseries}   % Makes the chapter number bold
% \ChTitleVar{\bfseries} % Makes the chapter title bold
% % Define a custom color, for example:
% \definecolor{mycolor}{RGB}{0,128,255}

% % Redefine the chapter style in Rejne to change the line colors
% \makeatletter
% \ChRuleWidth{2pt}   % Change the thickness of the lines
% \renewcommand{\DOCH}{%
%   \vspace*{-50\p@}% Moves the chapter title up/down if needed
%   {\color{mycolor} \hrule \@chapapp{} \space \thechapter \hrule}% Customizes the chapter header with color
% }
% \renewcommand{\DOTI}[1]{%
%   \vskip 20\p@ % Adjusts the space above the title
%   \bfseries #1\par % Embolden the title
%   \vskip 20\p@ % Adjusts the space below the title
% }
% \makeatother

% %% Change Chapter Heading Placement %%
% \usepackage{etoolbox}
% \makeatletter
% \patchcmd{\@makechapterhead}{\vspace*{50\p@}}{\vspace*{-20\p@}}{}{}
% \patchcmd{\@makeschapterhead}{\vspace*{50\p@}}{\vspace*{-20\p@}}{}{}
% \patchcmd{\DOTI}{\vskip 80\p@}{\vskip 40\p@}{}{}
% \patchcmd{\DOTIS}{\vskip 40\p@}{\vskip 0\p@}{}{}
% \makeatother

\renewcommand{\thesection}{\thechapter.\arabic{section}} %% Chapter.Section Numbering


%%% FIGURES %%%
\usepackage{graphicx}  
% \numberwithin{figure}{section}
\usepackage{float}
\usepackage{caption}

%%% Hyperlinks %%%
\usepackage{hyperref}
\definecolor{horange}{HTML}{f58026}
\hypersetup{
	colorlinks=true,
	linkcolor=horange,
	filecolor=horange,      
	urlcolor=horange,
}


%% Headers and Footers %%
\usepackage{fancyhdr} % This should be set AFTER setting up the page geometry
\pagestyle{fancy} % options: empty , plain , fancy
\renewcommand{\chaptermark}[1]{\markboth{\textcolor{draculafg}{\thechapter.\ #1}}{}}
\fancyhead[R]{\textcolor{draculafg}{\leftmark}}
\fancyhead[L]{\textcolor{draculafg}{Hendrix College}}
\fancyhead[C]{\includegraphics[height=.50cm]{\documentLogo}} % Use for Dracula
\usepackage{xpatch}
\xpretocmd\headrule{\color{orangehdx}}{}{\PatchFailed}
\setlength{\footskip}{0.5in}
\setlength{\headheight}{18.5764pt}


%%%%% Colored Boxes %%%%%
\usepackage{tcolorbox}
\tcbuselibrary{skins}
\tcbuselibrary{theorems}
% \tcbuselibrary{minted}
\newcounter{BoxCounter}
\usepackage{dingbat}

%-%-%-%-%-%-%-%-%-%-%-%-%-%-%-%-%-%-%-%-%-%-%-%-%-%-%-%-%-%-%-%-%-%-%-%-%-%-%

%% MATH PACKAGES, ENVIRONMENTS, COMMANDS %%
%-%-%-%-%-%-%-%-%-%-%-%-%-%-%-%-%-%-%-%-%-%-%-%-%-%-%-%-%-%-%-%-%-%-%-%-%-%-%
%You'll need your own packages, theorem types, and commands.

\usepackage{fix-cm}
\usepackage{amsmath,amsthm} 
\usepackage{mathtools}
\usepackage{amssymb}
\usepackage[framemethod=tikz]{mdframed}
\usepackage{mathrsfs}
\usepackage{changepage}
\usepackage{multicol}
\usepackage{slashed}
\usepackage{enumerate}
\usepackage{braket}
\usepackage{booktabs}
\usepackage{enumitem}
\usepackage{kantlipsum}  %This package lets us generate random text for example purposes.
\usepackage{pgfplots}


%%% Custom Commands %%%
% Natural Numbers 
\newcommand{\N}{\mathbb{N}}

% Whole Numbers
\newcommand{\W}{\mathbb{W}}

% Integers
\newcommand{\Z}{\mathbb{Z}}

% Rational Numbers
\newcommand{\Q}{\mathbb{Q}}

% Real Numbers
\newcommand{\R}{\mathbb{R}}

% Complex Numbers
\newcommand{\C}{\mathbb{C}}

\newcommand{\I}{\mathbb{I}}

\newcommand{\pfs}{\noindent\makebox[\linewidth]{\rule{\textwidth}{0.4pt}}\vspace{0.5cm}}

\newcommand{\mysqrt}[1]{%
  \mathpalette\foo{#1}%
}
\newcommand{\dmysqrt}[1]{%
  \mathpalette\foodisplay{#1}%
}

% !TeX spellcheck = off
\newcommand{\foo}[2]{%
  % #1: math style, #2: content
  \sbox0{$#1\sqrt{#2}$}% Measure the size of the standard sqrt in the current style
  \begin{tikzpicture}[baseline=(sqrt.base)]
    \node[inner sep=0, outer sep=0] (sqrt) {$#1\sqrt{#2}$}; % Use the current math style
    \draw([yshift=-0.045em]sqrt.north east) -- ++(0,-0.5ex); % Draw the tick
  \end{tikzpicture}%
}
% !TeX spellcheck = off
\newcommand{\foodisplay}[2]{%
  % #1: math style, #2: content
  \sbox0{$#1\sqrt{#2}$}% Measure the size of the standard sqrt in the current style
  \begin{tikzpicture}[baseline=(sqrt.base)]
    \node[inner sep=0, outer sep=0] (sqrt) {$\displaystyle\sqrt{#2}$}; % Force displaystyle
    \draw[line width=0.4pt] ([yshift=-0.044em]sqrt.north east) -- ++(0,-0.5ex); % Draw the tick
  \end{tikzpicture}%
}


% \newcommand{}{\marginnote{\includegraphics[width=2em]{caution.png}}}

\newmdenv[
  topline=false,
  bottomline=true,
  rightline=false,
  leftline=true,
  linewidth=1.5pt,
  linecolor=black, % default color, will be overridden in custom commands
  backgroundcolor=draculabg, % Needed for Dracula theme
  fontcolor=\documentTheme, % Needed for Dracula theme
  innertopmargin=0pt,
  innerbottommargin=5pt,
  innerrightmargin=10pt,
  innerleftmargin=10pt,
  leftmargin=0pt,
  rightmargin=0pt,
  skipabove=\topsep,
  skipbelow=\topsep,
]{customframedproof}

\newenvironment{proofpart}[2][black]{
    \begin{mdframed}[
        topline=false,
        bottomline=false,
        rightline=false,
        leftline=true,
        linewidth=1pt,
        linecolor=#1!40, % Custom color
        % innertopmargin=10pt,
        % innerbottommargin=10pt,
        innerleftmargin=10pt,
        innerrightmargin=10pt,
        leftmargin=0pt,
        rightmargin=0pt,
        % skipabove=\topsep,
        % skipbelow=\topsep%
    ]
    \noindent
    \begin{minipage}[t]{0.08\textwidth}%
        \textbf{#2}%
    \end{minipage}%
    \begin{minipage}[t]{0.90\textwidth}%
        \begin{adjustwidth}{0pt}{0pt}%
}{
    \end{adjustwidth}
    \end{minipage}
    \end{mdframed}
}

% Define a new environment 'proofscratch' for Proof and Scratch Paper
\newenvironment{proofscratch}[2]{%
    \begin{mdframed}[
        topline=false,
        bottomline=false,
        rightline=false,
        leftline=false,
        backgroundcolor=draculabg, % Background color for Dracula theme
        fontcolor=\documentTheme,       % Font color for Dracula theme
        innerleftmargin=-6pt,
        innertopmargin=0pt,
        leftmargin=0pt,
        rightmargin=0pt,
        skipabove=\topsep,
        skipbelow=\topsep,
    ]
    % % Begin TikZ picture for the vertical dotted line
    % \begin{tikzpicture}[overlay, remember picture]
    %     % Draw a vertical dotted line at approximately the center of the mdframed width
    %     \draw[dotted, thick] ([xshift=0.005\linewidth]current bounding box.north west) -- ([xshift=0.005\linewidth]current bounding box.south west);
    % \end{tikzpicture}%
    % % Create the left minipage for Proof
    \begin{minipage}[t]{0.47\textwidth}%
        \noindent\textit{Proof.} #1 \hfill \(\qed\)
    \end{minipage}%
    \hfill
    % Create the right minipage for Scratch Paper
    \begin{minipage}[t]{0.47\textwidth}%
        \textit{Scratch Paper.} #2
    \end{minipage}%
}{
    \end{mdframed}
}



\newcommand{\sol}[1]{
    \begin{customframedproof}[linecolor=orangehdx!75,]
        \begin{solution}
        #1
        \end{solution}
    \end{customframedproof}
}

\def \proofDistance {10pt}
\newenvironment{solution}
  {\textit{Solution.}}

\newcommand{\disabs}[1]{\ensuremath{\left|#1\right|}}
\newcommand{\limn}{\lim_{n \rightarrow \infty}}
\newcommand{\limx}[2]{\lim_{x \rightarrow #1}#2}

\newcommand{\limc}[1]{\lim_{x \rightarrow c}#1}

\newcommand{\ninn}{n \in \N}

\newcommand{\mb}[1]{\mathbf{#1}}

\newcommand{\limsupn}[1]{\limsup_{n\rightarrow \infty}#1}
\newcommand{\liminfn}[1]{\liminf_{n\rightarrow \infty}#1}

\renewcommand{\theenumi}{\alph{enumi}} 
\renewcommand{\labelenumi}{(\theenumi)}

\newcommand{\proj}{\text{proj}}

\newcommand{\p}{\partial}

\newcommand{\dydx}{\frac{dy}{dx}}
\newcommand{\dxdy}{\frac{dx}{dy}}
\newcommand{\dydt}{\frac{dy}{dt}}
\newcommand{\dxdt}{\frac{dx}{dt}}
\newcommand{\dzdt}{\frac{dz}{dt}}

\newcommand{\barNotationT}[1]{\bigg|_{t = #1}}

\newcommand{\cyanit}[1]{\textit{\textcolor{cyan}{#1}}}

\newcommand{\brackett}[1]{\left\langle #1 \right\rangle}

\newcommand{\norm}[1]{\left\lVert \mathbf{#1}\right\rVert}

\newcommand{\imb}{\mb{i}}
\newcommand{\jmb}{\mb{j}}
\newcommand{\kmb}{\mb{k}}
\newcommand{\rmb}{\mb{r}}
\newcommand{\umb}{\mb{u}}

\newcommand{\vecfuc}[2]{\mb{#1}(#2)}
\newcommand{\dvecfuc}[2]{\mb{#1}'(#2)}
\newcommand{\normdvecfuc}[2]{\|\mb{#1}'(#2)\|}


% end of preamble
%-%-%-%-%-%-%-%-%-%-%-%-%-%-%-%-%-%-%-%-%-%-%-%-%-%-%-%-%-%-%-%-%-%-%-%-%-%-%

\begin{document}

\numberwithin{BoxCounter}{section}


%-%-%-%-%-%-%-%-%-%-%-%-%-%-%-%-%-%-%-%-%-%-%-%-%-%-%-%-%-%-%-%-%-%-%-%-%-%-%
%%% COVER PAGE %%%
%-%-%-%-%-%-%-%-%-%-%-%-%-%-%-%-%-%-%-%-%-%-%-%-%-%-%-%-%-%-%-%-%-%-%-%-%-%-%

% Do not use all caps.
\newcommand{\titlestandin}{Multivariable Calculus Notes}
\newcommand{\cussubtitle}{MATH 230}
% Date format should be like: "January 1, 2001"
\newcommand{\startdate}{January 22, 2025}
\newcommand{\customenddate}{May 13, 2025}
% Be sure to include degree recognition (e.g., B.S., M.S., Ph.D)
\newcommand{\professor}{Prof. Lars Seme, M.S.}

%-%-%-%-%-%-%-%-%-%-%-%-%-%-%-%-%-%-%-%-%-%-%-%-%-%-%-%-%



\begin{titlepage}
    \begin{center}

        \vspace*{-2cm}
        \includegraphics[width=0.8\textwidth]{\documentBigLogo}\\ % Use for Dracula. Its the all white version of the logo.
        \vfill

        % Horizontal line above the title in 'horange' color
        \textcolor{horange}{\rule{\textwidth}{1.0pt}}

        \vspace{2em}

        {\huge \textbf{\titlestandin}}

        \vspace{1em} % Space between the title and the bottom line

        \textcolor{horange}{\rule{\textwidth}{1.0pt}}

        \vspace*{1\baselineskip}

        {\LARGE \textbf{\cussubtitle}}

        \begin{large}
            \vspace*{2\baselineskip}

            \textit{Start}  \\[1ex]
            {\scshape \startdate} \\[0.3\baselineskip] % Year published

            \vspace*{1\baselineskip}

            \emph{Author} \\[1ex]
            %Submitted by \\[\baselineskip]
            {\Large Paul Beggs \\ \par} % Editor list
            {\href{mailto:BeggsPA@Hendrix.edu}{{BeggsPA@Hendrix.edu}}}\\ % Editor affiliation

            \vspace*{1\baselineskip}

            \textit{Instructor} \\[1ex] % Tagline(s) or further description
            \professor

            \vspace*{1\baselineskip}

            \textit{End}\\[1ex]
            {\scshape  \customenddate} \\[0.3\baselineskip] % Year published

            \thispagestyle{empty}

        \end{large}
    \end{center}
\end{titlepage}
\pagebreak


%-%-%-%-%-%-%-%-%-%-%-%-%-%-%-%-%-%-%-%-%-%-%-%-%-%-%-%-%-%-%-%-%-%-%-%-%-%-%

%%% Table of Contents %%%


% -%-%-%-%-%-%-%-%-%-%-%-%-%-%-%-%-%-%-%-%-%-%-%-%-%-%-%-%-%-%-%-%-%-%-%-%-%-%

% Helper command to fix header for double paged ToC.
% Temporarily adjust section marks for Table of Contents
\begin{spacing}{1} % Can change spacing between entries in TOC
    \renewcommand{\contentsname}{\Large\textbf{Table of Contents}} % Can rename here
    \markboth{}{} % Clear the header marks for TOC
    \pagestyle{fancy} % Ensure fancy style is active
    \fancyhead[R]{Table of Contents} % Set TOC-specific header manually
    \tableofcontents
    \addtocontents{toc}{\protect\enlargethispage{\baselineskip}}
\end{spacing}

% Reset the headers to default for the rest of the document
\fancyhead[R]{\leftmark}




% -%-%-%-%-%-%-%-%-%-%-%-%-%-%-%-%-%-%-%-%-%-%-%-%-%-%-%-%-%-%-%-%-%-%-%-%-%-%
\chapter{Parametric Eqs and Polar Coords}\label{para eqs and polar coords}
\vspace*{-0.25in}
\section{Parametric Equations}

\subsection{Introduction}
Most of your calculus experience has been single variable, so that the functions under consideration were typically \(f : \mathbb{R} \to \mathbb{R}\). Our course is divided into roughly 3 sections:
\begin{itemize}
    \item Parametric Equations/Functions: Functions of the form \(f : \mathbb{R} \to \mathbb{R}^n\) (Chapters 1 - 3)
    \item Scalar Functions: Functions of the form \(f : \mathbb{R}^n \to \mathbb{R}\) (Chapters 4 - 5)
    \item Vector Fields: Functions of the form \(f : \mathbb{R}^n \to \mathbb{R}^n\) (Chapter 6)
\end{itemize}

\subsection{Parametric Equations}

A \cynit{parametric equation} (or, \cynit{sometimes parametric function} or \cynit{ vector-valued function}) is a function of the form \(f \colon \mathbb{R} \to \mathbb{R}^{n}\). We will typically consider \(n = 2\) or \(n = 3\) and call the input variable the parameter, usually denoted by \(t\). We write them as

\[
    f(t) =
    \begin{cases}
        x(t) \\
        y(t)
    \end{cases}
    \quad \text{or} \quad
    f(t) =
    \begin{cases}
        x(t) \\
        y(t) \\
        z(t)
    \end{cases}.
\]

A \cynit{parametric curve} is the set of points \((x(t), y(t))\) in \(\R^{2}\) or \((x(t), y(t), z(t))\) in \(\R^{3}\) traced out. Note that in general, the curve may not be a function for \(y\) in terms of \(x\), but is a function of the parameter \(t\).

\subsection{Graphing Parametric Curves in the Second Dimension}

\subsubsection{Elimination of the Parameter}

In some cases, we can explicitly solve for \(t\) in terms of one of \(x\) or \(y\). When this is possible, you can write \(y(x)\) or \(x(y)\) and use your “regular” algebraic knowledge. We call this process \cynit{eliminating the parameter}.

\subsubsection{Using Technology}

\begin{itemize}
    \item Your TI-84 can graph this if you switch to \texttt{par} mode.
    \item Likewise, GeoGebra can do this, using the \texttt{curve} function.
          \begin{itemize}
              \item In general, the syntax is: \texttt{curve(x(t), y(t), t, min, max)}
          \end{itemize}
\end{itemize}

\subsection{The Cycloid}
A wheel of radius \(a\) is rolling along a flat road at a constant velocity. The curve generated by a point along the edge of the wheel traces out a shape called a \cynit{cycloid}. Let \(t\) represent the angle - in radians!!!! - rotated through, and that the point of interest starts at the origin. Before we find the equations for the point, let's find the location of the center of the circle:
\[
    f_{\text{center}}(t) = \begin{cases}
        x(t) = a t \\
        y(t) = a
    \end{cases}
\]

Then, relative to the center, our point along the edge has equations
\[
    f(t) = \begin{cases}
        x(t) = -a \sin(t) \\
        y(t) = -a \cos(t)
    \end{cases}
\]

Thus, our point has parametric equations
\[
    f(t) = \begin{cases}
        x(t) = a(t - \sin(t)) \\
        y(t) = a(1 - \cos(t))
    \end{cases}
\]

\subsection{Final Notes}

Next time, we'll start asking Calculus-y questions: What are the velocities in the \(x\), \(y\), and total directions? What total distance does it travel? What is the area of the region under one period of the cycloid?
\begin{itemize}
    \item The syllabus has a number of practice problems to work on. These are not required, and not to be
          turned in, but are for you to work before class next time.
    \item We will talk about them at the start of the next class. You should try them beforehand.
    \item The most common reason for a lack of success in this class is not spending time working problems on
          your own.
\end{itemize}

\section{Calculus of Parametric Curves}

For this section, we will have a parametric curve in R2, defined by \(f(t) = \begin{cases}
    x(t) \\
    y(t)
\end{cases}.\) 
In many cases, the curve does not describe \(y\) as a function of \(x\). However, we can still carry over many ideas from single variable calculus.

\subsection{Slope for a Parametric Curve}

Given a point \(t_{0}\), the \cynit{slope of the curve} in the \(xy\)-plane is given by
\[
    \dydx\barNotationT{t_{0}} = \frac{dy/dt}{dx/dt}\barNotationT{t_{0}}. \\
\]
Note that this is undefined when \(x'(t_{0}) = 0\). \\

\noindent The \cynit{tangent line} at \(t_{0}\) is given by
\[
    y = \left(\dydx\barNotationT{t_{0}}\right)(x - x(t_{0})) + y(t_{0}).
\]

\subsection{Second Derivative}

The value of the second derivative for the curve at \(t_{0}\) is given by
\[
    \dfrac{d^{2}y}{dx^{2}}\barNotationT{t_{0}} = \frac{d}{dt}\left(\dydx\right)\barNotationT{t_{0}} = \frac{d}{dt}\left(\frac{dy/dt}{dx/dt}\right)\barNotationT{t_{0}}. \\
\]
Note the benefit of Leibnitz notation for each of these two derivatives!

\subsection{Area Under a Curve}

Suppose that a parametric curve is non-self intersecting. Then, the signed area of the region between the curve and the \(x\)-axis on the \(t\) interval \([t_{a},t_{b}]\) is given by 

\[
    A = \int_{t_{a}}^{t_{b}} y(t) \frac{dx}{dt} \, dt = \int_{t_{a}}^{t_{b}} y(t) \frac{dx}{dt} \, dt.
\]

\subsection{Arc Length}

The \cynit{arc length} of a parametric curve over the \(t\) interval \([t_{a},t_{b}]\) is given by

\[
    s = \int_{t_{a}}^{t_{b}} \sqrt{\left(\frac{dx}{dt}\right)^{2} + \left(\frac{dy}{dt}\right)^{2}} \, dt.
\]

\subsection{Surface Area}

The \cynit{surface area} of the region obtained by rotating a non-self intersecting parametric curve is given by

\[
    S = \int_{t_{a}}^{t_{b}} 2\pi y(t) \sqrt{\left(\frac{dx}{dt}\right)^{2} + \left(\frac{dy}{dt}\right)^{2}} \, dt.
\]

\subsection{The Cycloid}

We can apply each of the above to the cycloid:
\begin{itemize}
    \item \cynit{Derivative}: \(\dydx = \frac{dy}{dx} = \frac{\sin(t)}{1 - \cos(t)}\). Note that the slope is then independent of the radius of the wheel and
    that the slope is undefined at each of \(t = \dots, -4\pi, -2\pi, 0, 2\pi, 4\pi, \dots\). 
    \item \cynit{Concavity}: \(\dfrac{d^{2}y}{dx^{2}} = \frac{d}{dt}\left(\frac{dy}{dx}\right) = \frac{d}{dt}\left(\frac{\sin(t)}{1 - \cos(t)}\right)\). After some work, we find that \(\frac{d^{2}y}{dx^{2}} = -\frac{a}{y^{2}}\), which shows that the cycloid is always concave down.
    \item \cynit{Area}: The area of one period of the cycloid \(A = 3\pi a2\), after some work.

    \begin{align*}
        A &= \int_{0}^{2\pi} (a - a \cos t)(a - a\cos t) dt \\
        &= a^{2} \int_{0}^{2\pi} (1 - 2\cos t + \cos^{2} t) dt \\
        &= a^{2} \left(2\pi + \int_{0}^{2\pi} 1 - 2\cos t + \cos^{2} t \right) dt \\
        &= a^{2}\left(2\pi + \left(\frac{t}{2} + \frac{1}{4}\sin(2t)\right)\right)\bigg|_{0}^{2\pi} \\
        &= 3\pi a^{2}.
    \end{align*}

    \item \cynit{Arc Length}: The arc length of one period of the cycloid is \(s = 8a\), again after some work.

    \begin{align*}
        S &= \int_{0}^{2\pi} \sqrt{(a- a \cos t)^{2} + (a \sin t)^{2}} \ dt \\
        &= a \int_{0}^{2\pi} \sqrt{1 - 2\cos t + \cos^{2} t + \sin^{2} t} \ dt \\
        &= a \int_{0}^{2\pi} \sqrt{2 - 2\cos t} \ dt \\
        &= \sqrt{2}a \int_{0}^{2\pi} \sqrt{1 - cost} \ dt \\
        &= \sqrt{2}a \int_{0}^{2\pi} \sqrt{2\sin^{2}\left(\frac{t}{2}\right)} \ dt \\
        &= \sqrt{2}a \cdot \sqrt{2} \int_{0}^{2\pi} \sin\left(\frac{t}{2}\right) \ dt \\
        &= 2a\left(-2 \cos\left(\frac{t}{2}\right)\right)\bigg|_{0}^{2\pi} \\ 
        &= 8a.
    \end{align*}

    \item \cynit{Surface Area}: The surface area of the solid obtained by rotating one period of the cycloid around the \(x\)-axis is \(S = \frac{64\pi a^{2}}{3}\), after a lot of tedious work.
\end{itemize}

\setcounter{chapter}{1}

\chapter{Vectors in Space}\label{vectors}
\vspace*{-0.25in}
\section{Introduction}

\subsection{Vectors}

A \cyanit{vector} is a quantity with both \cyanit{magnitude} (size, length, strength, \dots) and \cyanit{direction}.  

\subsection{Notation}

In print, we write vectors in bold like: $\mathbf{v}$, $\mathbf{w}$, $\mathbf{u}$, \dots.  In handwriting, we often write vectors with an arrow over the top: $\vec{v}$, $\vec{w}$, $\vec{u}$, \dots.

\section{Vectors in the Plane}

Given two points in the plane \(P = (x_{1},y_{1})\) and \(Q = (x_{2},y_{2})\), the vector from \(P\) to \(Q\), denoted \(\overrightarrow{PQ} = \mathbf{PQ} = \langle x_{2}-x_{1},y_{2}-y_{1}\rangle\). \\

We can also simply state components: \(\mathbf{v} = \langle x, y\rangle\). \\

The \cyanit{zero vector}, denoted \(\mathbf{0}\), is \(\mathbf{0} = \langle 0,0 \rangle\). Note that \(\mathbf{0} \ne 0\). \\

A \cyanit{scalar} is a real number (or a magnitude), without direction. \\

If \(c\) is a scalar and \(\mathbf{v} = \langle x,y\rangle\), then
\[
    c\mathbf{v} = c\langle x,y\rangle = \langle cx,cy\rangle.
\]
This operation is called \cyanit{scalar multiplication}. Scalar multiplication changes the magnitude of a vector, but not its direction. \\

Note that the individual components of a vector are themselves \cyanit{scalars}. You need to keep track of which is which. \\

If \(\mathbf{v} = \langle x_{1},y_{1} \rangle\) and \(\mathbf{w} = \brackett{x_{2},y_{2}}\), then the \cyanit{vector sum} 
\[
    \mathbf{v} + \mathbf{w} = \langle x_{1}+x_{2},y_{1}+y_{2}\rangle.
\] 
That is, we add component wise. \\

If \(\mathbf{v} = \langle x_{1},y_{1} \rangle\), then the \cyanit{magnitude} of \(\mathbf{v}\) is given by
\[
    \norm{v} = \sqrt{x_{1}^{2}+y_{1}^{2}}.
\]
This is really just the Pythagorean theorem. \\

\section{Vectors in Space}

In \(\R^{3}\), we have three axes, \(x\), \(y\), and \(z\), which follow the \cyanit{right-hand rule}: point the fingers of the right hand in the direction of the positive \(x\)-axis, curl them towards the positive \(y\)-axis, and the thumb points in the direction of the positive \(z\)-axis. \\

Since the distance formula in \(\R^{3}\) is \(d = \sqrt{(x_{2} - x_{1})^{2} + (y_{2} - y_{1})^{2} + (z_{2} - z_{1})^{2}}\), then \(\mathbf{u} = \brackett{x,y,z}\) we have \(\norm{u} = \sqrt{x^{2} + y^{2} + z^{2}}\). \\

To \cyanit{normalize} a vector, we divide by its magnitude: \(\mathbf{v} = \brackett{x,y,z}\), then \(\mathbf{u} = \frac{1}{\norm{v}}\mathbf{v} = \brackett{\frac{x}{\norm{v}},\frac{y}{\norm{v}},\frac{z}{\norm{v}}}\). This gives us a \cyanit{unit vector} in the direction of \(\mathbf{v}\). \\

Everything else is basically the same.

\subsection{Vector Properties}

Suppose that each of \(\mathbf{u}\), \(\mathbf{v}\), and \(\mathbf{w}\) are vectors and \(r\) and \(s\) are scalars. Then the following properties hold:
\begin{itemize}
    \item \cyanit{Additive Commutativity}: \(\mathbf{v} + \mathbf{w} = \mathbf{w} + \mathbf{v}\).
    \item \cyanit{Additive Associativity}: \(\mathbf{u} + (\mathbf{v} + \mathbf{w}) = (\mathbf{u} + \mathbf{v}) + \mathbf{w}\).
    \item \cyanit{Additive Identity}: \(\mathbf{v} + \mathbf{0} = \mathbf{v}\).
    \item \cyanit{Additive Inverse}: \(-\mathbf{v} = (-1)\mathbf{v}\) and \(\mathbf{v} + (-\mathbf{v}) = \mathbf{0}\).
    \item \cyanit{Scalar Associativity}: \(r(s\mathbf{u})= (rs)\mathbf{u}\).
    \item \cyanit{Scalars Distributive over Vectors}: \(r(\mathbf{u} + \mathbf{v}) = r\mathbf{u} + r\mathbf{v}\).
    \item \cyanit{Vectors Distributive over Scalars}: \((r+s)\mathbf{u} = r\mathbf{u} + s\mathbf{u}\).
    \item \cyanit{Multiplicative Identity}: \(1\mathbf{u} = \mathbf{u}\).
    \item \cyanit{Zero Scalar}: \(0\mathbf{u} = \mathbf{0}\).
\end{itemize}

\subsection{Special Vectors}

A \cyanit{unit vector} is a vector \(\mathbf{u}\) such that \(\norm{u} = 1\). \\

In \(\R^{2}\) the \cyanit{standard unit vectors} are \(\hat{\imath} = \mathbf{i} = \brackett{1,0}\) and \(\hat{\jmath} = \mathbf{j} = \brackett{0,1}\). This allows us to write \(\mathbf{v} = \brackett{2,3} = 2\mathbf{i} + 3\mathbf{j}\), for example. \\

In \(\R^{3}\), we have three stand unit vectors, \(\hat{\imath} = \mathbf{i} = \brackett{1,0,0}\), \(\hat{\jmath} = \mathbf{j} = \brackett{0,1,0}\), and \(\hat{k} = \mathbf{k} = \brackett{0,0,1}\).  \\

It is a picky detail, but \(\mathbf{i} \in \R^{2} \ne \mathbf{i} \in \R^{3}\). 

\setcounter{chapter}{2}

\chapter{Vector-Valued Functions}\label{vector-valued functions}
\vspace*{-0.25in}
\input{Chapters/Multi_ch_3.tex}

\chapter{Diff. of Fun. of Several Variables}\label{diff of fun of several variables}
\vspace*{-0.25in}
\section*{4.1-4.3 Functions of Several Variables}
\setcounter{section}{1}
A \textit{scalar function} is a function \(f \colon \R^{n} \to \R\). This is in some sense a reverse of a parametric function. Parametric functions take in a single real variable \(t\) and output an answer of multiple variable (which we typically interpreted as a vector). A scalar function takes in multiple independent variables (typically interpreted as a point in \(\R^{n}\)) and outputs a single real variable.

\subsubsection{Notation and Graphs}

When \(f \colon \R^{2} \to \R\), we can write \(z = f(x, y)\). For each point \((x, y)\) in the domain of \(f\), we can interpret \(z\) as a height, so that the set of points \((x, y, z)\) which satisfy \(z = f(x, y)\) form a surface in \(\R^{3}\). You can graph this in GeoGebra using \url{geogebra.com/3d}. \\

When \(f \colon \R^{3} \to \R\), we often write \(w = f(x, y, z)\). Then, the point \((x, y, z)\) is in the domain and \(w\) is still a height, though since the graph lives we are in \(\R^{4}\), we cannot draw it in a simple way.

\subsubsection{Contour Plots}

Let \(f \colon \R^{2} \to \R\) be a function. The graph is a surface in \(\R^{3}\), but often it can be difficult to show this on paper. For a given \(z_{0} \in \R\) a \textit{level curve} consists of all points \((x, y) \in \R^{2}\) for which \(f(x, y) = z_{0}\). If we think of the function \(f\) as describing a height at a function of two-dimension position, a level curve is the set of all points with a particular given height. \\

A \textit{contour plot} consists of a collection of level curves – typically, we select some number of equally spaced \(\{z_{1}, z_{2}, \ldots, z_{n}\}\) and build the level curves for each. This gives a “topographical map” of the function \(f\).

\subsection{Limits}

\subsubsection{Formal Definition of \(\lim_{(x, y) \to (a, b)} f(x, y)\)}

Let \(f \colon \R^{2} \to \R\), \((a, b) \in \R^{2}\), and \(L \in \R\). The statement that \(f\) has limit \(L\) as \((x, y)\) approaches \((a, b)\) means that for each choice of \(\epsilon > 0\), there exists \(\delta > 0\) such that for all \((x, y) \neq (a, b)\) within a disk of radius \(\delta\), centered at \((a, b)\), then

\[
|f(x, y) - L| < \epsilon.
\]

Essentially, this says that the limit exists whenever we can make the output from \(f\) as close to \(L\) as we’d like by choosing \((x, y)\) sufficiently close to \((a, b)\). \\

In Calculus I, you showed that a limit \(\lim_{x \to a} f(x)\) exists if and only if the left and right hand limits both exist and match. Basically, you showed that no matter if you approach \(a\) from the left or from the right, you always get the same answer. The same idea applies here, but there is an important complication: We need to check every possible path into the point \((a, b)\) – and rather than just two, there are infinitely many paths! \\

In general, there is not a simple recipe that will work. You can try various combinations of paths until something goes wrong. For practice set/exam purposes, I will always tell you that the limit does not exist, and ask you to prove it. Showing that a limit does exist is a completely different procedure – just because two paths happen to produce the same answer does not prove that all do. \\

Problem \#4 on Practice Set \#3 will ask you to show a limit exists. It provides step-by-step instructions and this is the only time I will ask you to do this.

\subsubsection{Limit Laws}

The basic rules are all the same. Suppose that \(a, b, c \in \R\) and \(f, g \colon \R^{2} \to \R\) and that \(\lim_{(x, y) \to (a, b)} f(x, y) = L\) and \(\lim_{(x, y) \to (a, b)} g(x, y) = M\).

\begin{multicols}{2}
    \begin{itemize}
        \item \(\lim_{(x, y) \to (a, b)} c = c\)
        \item \(\lim_{(x, y) \to (a, b)} x = a\)
        \item \(\lim_{(x, y) \to (a, b)} y = b\)
        \item \(\lim_{(x, y) \to (a, b)} [f(x, y) + g(x, y)] = L + M\)
        \item \(\lim_{(x, y) \to (a, b)} [cf(x, y)] = cL\)
    \end{itemize}
\end{multicols}
\vspace*{-0.5cm}
\begin{itemize}
    \item if \(f\) has a single, simple algebraic definition and \((a, b)\) is in the domain, then\\ 
    \(\lim_{(x, y) \to (a, b)} f(x, y) = f(a, b)\).
\end{itemize}

\subsection{Continuity}

A function \(f \colon \R^{2} \to \R\) is continuous at \((a, b) \in \R^{2}\) provided that \(\lim_{(x, y) \to (a, b)} f(x, y)\) exists and is equal in value to \(f(a, b)\). This is exactly our usual definition of continuity from Calculus I again – and there is nothing special here about being in two dimensions. \\

All of the usual algebraic functions (polynomials, root functions, trig functions, exponentials, and logarithms) and their combinations by addition, subtraction, multiplication, division, exponentiation, and composition are continuous everywhere in their domains.

\subsection{Partial Derivatives}

Suppose \(f \colon \R^{n} \to \R\) is a scalar function. Then, the \cyanit{partial derivative} of \(f\) with respect to \(x_{i}\) is
\[
\dfrac{\partial f}{\partial x_{i}} = \lim_{h \to 0} \dfrac{f(x_{1}, x_{2}, \ldots, x_{i} + h, \ldots, x_{n}) - f(x_{1}, x_{2}, \ldots, x_{i}, \ldots, x_{n})}{h},
\]
if this limit exists. \\

That is, we hold all other variables constant and take the derivative as though \(x_{i}\) is the only variable. Just like in Calculus I, the partial derivative is a rate of change – it measures the ratio of how much the output of \(f\) changes relative to a small change in \(x_{i}\).

\subsubsection{Notation}

Suppose that \(f\) is a function and \(x\) is one of its variables. Then, we denote the partial derivative of \(f\) with respect to \(x\) by any of:
\begin{itemize}
    \item Leibnitz/Condorcet Notation: \(\dfrac{\partial f}{\partial x}\)
    \item Jacobi Notation: \(f_{x}\)
    \item Euler Operation Notation: \(D_{x}f\)
\end{itemize}
Higher order derivatives are notated as:
\begin{itemize}
    \item Leibnitz/Condorcet Notation: \(\dfrac{\partial^{n} f}{\partial x^{n}}\)
    \item Jacobi Notation: \(f_{xxx\ldots x}\)
    \item Euler Operation Notation: \(D_{n}f\)
\end{itemize}

\subsection{Mixed Partials}

Given a function \(f \colon \R^{2} \to \R\) we can determine the \cyanit{mixed partial derivative} of \(f\) with respect to \(x\) and then \(y\). In Leibnitz notation, this would be
\[
\dfrac{\partial}{\partial y} \left( \dfrac{\partial f}{\partial x} \right) \quad \text{or in Jacobi notation} \quad f_{xy}.
\]

\subsubsection{Clairaut’s Theorem}

Suppose that \(f \colon \R^{2} \to \R\) is defined on an open disk \(D\) containing the point \((a, b)\). If each of \(f_{xy}\) and \(f_{yx}\) are continuous on \(D\), then \(f_{xy}(a, b) = f_{yx}(a, b)\). \\

That is, for pretty much any function you’ll even encounter, order does not matter for the mixed partial derivatives! This is especially nice, since notationally, \(\dfrac{\partial^{2} f}{\partial x \partial y} = f_{yx}\) and \(\dfrac{\partial^{2} f}{\partial y \partial x} = f_{xy}\), and who wants to worry about keeping up with that!

\newpage

\setcounter{section}{3}

\section{Tangent Planes}

\subsection{Tangent Lines}

For a single-variable differentiable function \(f \colon \R \to \R\), in Calculus I we defined the \cyanit{tangent line} at \(x = x_{0}\) as the unique line
\[
y = f'(x_{0})(x - x_{0}) + f(x_{0}).
\]
This is the line which most closely matches the graph of \(y = f(x)\) at the desired point. Functions which have tangent lines are said to be “smooth” or “locally linear.”

\subsection{Tangent Planes}

For functions \(f \colon \R^{2} \to \R\) we have a similar idea. If the surface generated by such a function has no sharp corners or edges, you might see that as you zoom in, the surface becomes flatter and flatter – and will eventually resemble a plane. In fact, we define the \cyanit{tangent plane} as the unique plane at \((x, y) = (x_{0}, y_{0})\) which satisfies
\[
z = f(x_{0}, y_{0}) + f_{x}(x_{0}, y_{0})(x - x_{0}) + f_{y}(x_{0}, y_{0})(y - y_{0}).
\]

But notice that \(f(x_{0},y_{0})\) is just \(z_{0}\), the height of the surface at \((x_{0}, y_{0})\). So, the tangent plane is the unique plane which most closely matches the graph of \(z = f(x, y)\) at the desired point.

\subsection{Linearizations}

Recall from Calculus I that if \(f\) is differentiable at \(x_{0}\) and \(x\) is close to \(x_{0}\) then
\[
f(x) \approx f(x_{0}) + f'(x_{0})(x - x_{0}).
\]

This is the \cyanit{linear approximation} of \(f\) at \(x_{0}\). In the same way, if \(f \colon \R^{2} \to \R\) is differentiable at \((x_{0}, y_{0})\) and \((x, y)\) is near \((x_{0}, y_{0})\), then
\[
f(x, y) \approx f(x_{0}, y_{0}) + f_{x}(x_{0}, y_{0})(x - x_{0}) + f_{y}(x_{0}, y_{0})(y - y_{0}).
\]
This is the \cyanit{linearization} of \(f\) at \((x_{0}, y_{0})\).

\subsection{Examples}

\(f(x,y) = x\cos(\pi x) \sin(\pi y)\), \((x_{0},y_{0}) = \bigl(\frac{1}{3}, \frac{1}{2}\bigr)\). Our point of interest is the pair \(\bigl(\frac{1}{3}, \frac{1}{2}\bigr)\) and \(\frac{1}{6}\), because we can just plug in the values of \(x_{0}\) and \(y_{0}\) into the function to get the height. \\

To find the equation of the tangent plane, we need to find the partial derivatives to fill out the following equation:
\[
z = \underline{\hspace{1cm}} + \underline{\hspace{1cm}}\left(x - \frac{1}{3}\right) + \underline{\hspace{1cm}}\left(y - \frac{1}{2}\right).
\]
We found \(z_{0}\) to be \(\frac{1}{6}\), and the partial derivatives are
\[
    f_{x}(x,y) = \cos(\pi x)\sin(\pi y) - \pi x\sin(\pi x)\sin(\pi y),
\]
and 
\[
    f_{x}\left(\frac{1}{3},\frac{1}{2}\right) = \frac{1}{2} - \frac{\pi}{3}\left(\frac{\mysqrt{3}}{2}\right)(1) = \frac{1}{2} - \frac{\pi\mysqrt{3}}{3}.
\]
Similarly, we have
\[
    f_{y}\left(x,y\right) = \pi x\cos(\pi x)\cos(\pi y),
\]
and
\[
    f_{y}\left(\frac{1}{3},\frac{1}{2}\right) = 0.
\]
Now, we can fill out the rest of our equation:
\[
z = \frac{1}{6} + \left(\frac{1}{2} - \frac{\pi\mysqrt{3}}{3}\right)\left(x - \frac{1}{3}\right) + 0\left(y - \frac{1}{2}\right).
\]

\newpage

\section{The Chain Rule}

\subsection{One Dimensional Chain Rule}

Suppose that \(f(x)\) and \(g(x)\) are both single variable, differentiable functions. Then, if \(h(x) = f(g(x))\), \(h\) is also differentiable and
\[
h'(x) = f'\bigl(g(x)\bigr)g'(x).
\]
Alternatively, if we write \(u = g(x)\), then
\[
h'(x) = f'(u)g'(x) \quad \text{or} \quad \dfrac{dh}{dx} = \dfrac{df}{du}\dfrac{du}{dx}.
\]
\subsection{Higher Dimensional Chain Rules}

\subsubsection{The Chain Rule – One Independent Variable}

Suppose that \(x = f(t)\), \(y = h(t)\), and \(z = f(x, y)\) are all differentiable functions. Notice that \(z\) is “really” a function of the single independent variable \(t\). Then,
\[
\dfrac{dz}{dt} = \dfrac{\partial z}{\partial x} \cdot \dfrac{dx}{dt} + \dfrac{\partial z}{\partial y} \cdot \dfrac{dy}{dt}.
\]
\subsubsection{The Chain Rule – Two Independent Variables}

Suppose that \(x = g(u, v)\), \(y = h(u, v)\), and \(z = f(x, y)\). Then,
\[
\dfrac{\partial z}{\partial u} = \dfrac{\partial z}{\partial x} \cdot \dfrac{\partial x}{\partial u} + \dfrac{\partial z}{\partial y} \cdot \dfrac{\partial y}{\partial u},
\]
and
\[
\dfrac{\partial z}{\partial v} = \dfrac{\partial z}{\partial x} \cdot \dfrac{\partial x}{\partial v} + \dfrac{\partial z}{\partial y} \cdot \dfrac{\partial y}{\partial u}.
\]
\subsubsection{The Generalized Chain Rule}

Let \(w = f(x_{1}, x_{2}, \ldots, x_{m})\) be a function of \(m\) independent variables and each \(x_{i} = x_{i}(t_{1}, t_{2}, \ldots, t_{n})\) has \(n\) independent variables. Then,
\[
\dfrac{\partial w}{\partial t_{j}} = \dfrac{\partial w}{\partial x_{1}} \cdot \dfrac{\partial x_{1}}{\partial t_{j}} + \dfrac{\partial w}{\partial x_{2}} \cdot \dfrac{\partial x_{2}}{\partial t_{j}} + \cdots + \dfrac{\partial w}{\partial x_{m}} \cdot \dfrac{\partial x_{m}}{\partial t_{j}} = \sum_{i=1}^{m} \dfrac{\partial w}{\partial x_{i}} \cdot \dfrac{\partial x_{i}}{\partial t_{j}}.
\]

\subsection{Implicit Differentiation}

Suppose we have an equation involving \(x\) and \(y\). We can rewrite this as an equation of the form \(z = f(x, y)\), where we let \(z = 0\). Then,
\[
z = 0, \text{ defines a curve for } x, y
\]
\[
\dfrac{dy}{dx} = -\dfrac{\partial f}{\partial x} \cdot \dfrac{\partial f}{\partial y}.
\]
This gives the same answer as the Calculus I \cyanit{implicit differentiation} procedure.    

\newpage

\section{Directional Derivatives and the Gradient}

The function \(z = f(x, y)\) defines a 2-dimensional surface in \(\R^{3}\). The values of \(\dfrac{\partial z}{\partial x}\) and \(\dfrac{\partial z}{\partial y}\) tell us about the rate of change of \(z\) relative to the two independent (and orthogonal) directions \(x\) and \(y\). What if we want to know about the rate of change in some other direction? \\

Let \(v\) be a vector in the direction we are interested in. Then, the \cyanit{directional derivative} at the point \((x_{1}, x_{2}, \ldots, x_{n})\) of the function \(f\) in the direction of a vector \(v\) is given by
\[
D_{v} = \dfrac{\partial f_{x_{1}}(x_{1}, x_{2}, \ldots, x_{n})}{\|v\|}v_{1} + \dfrac{\partial f_{x_{2}}(x_{1}, x_{2}, \ldots, x_{n})}{\|v\|}v_{2} + \cdots + \dfrac{\partial f_{x_{n}}(x_{1}, x_{2}, \ldots, x_{n})}{\|v\|}v_{n}.
\]
\subsection{The Gradient, \(\nabla f\)}

Suppose that \(f \colon \R^{2} \to \R\). We define the \cyanit{gradient} of \(f\), denoted by \(\nabla f(x, y)\), as
\[
\nabla f(x, y) = \dfrac{\partial f}{\partial x}(x, y)\mathbf{i} + \dfrac{\partial f}{\partial y}(x, y)\mathbf{j}.
\]
We can extend this: If \(g \colon \R^{3} \to \R\), then
\[
\nabla g(x, y, z) = \dfrac{\partial g}{\partial x}(x, y, z)\mathbf{i} + \dfrac{\partial g}{\partial y}(x, y, z)\mathbf{j} + \dfrac{\partial g}{\partial z}(x, y, z)\mathbf{k},
\]
and so on to higher dimensions. \\

The symbol \(\nabla\) is called “nabla” or “del.” Sometimes the gradient is written as \(\text{grad } f\). The gradient is a vector which points in the direction of largest increase in the value of the function – if interpreted as height, it points in the direction that is the steepest uphill. \\

Note that the \(\nabla\) lives in the domain of the function. That is, if \(f \colon \R^{2} \to \R\), then \(\nabla f \colon \R^{2} \to \R^{2}\). The gradient is a vector field – it assigns a vector to each point in the domain of the function. \\

The gradient on a contour plot is the vector which is orthogonal to the level curve at that point. This is because the gradient points in the direction of the greatest increase, and the level curve is the set of points where the function does not change.

\subsubsection{Properties of the Gradient}

\begin{itemize}
    \item If \(\nabla f(x_{0}, y_{0}) = 0\), then \(D_{u}f(x_{0}, y_{0}) = 0\) for all \(u\).
    \item If \(\nabla f(x_{0}, y_{0}) \neq 0\), then \(D_{u}f(x_{0}, y_{0})\) is maximized when \(u\) and \(\nabla f(x_{0}, y_{0})\) point in the same direction.
    \item This maximum value is \(\| \nabla f(x_{0}, y_{0}) \|\).
    \item The gradient vector is always orthogonal to any level curve.
\end{itemize}

\subsubsection{Del – Vector Definition}

We can think of \(\nabla\) as the vector
\[
\nabla = \left\langle \dfrac{\partial}{\partial x}, \dfrac{\partial}{\partial y}, \dfrac{\partial}{\partial z} \right\rangle.
\]
From chapter 2, we know that when we multiply a vector (like \(\nabla\)) by a scalar (like \(f\)), we simply multiply each coordinate of the vector by the scalar. Thus, we can think of \(\nabla f\) as a “simple” vector operation. Later, when we have functions which have vector-values themselves, we will talk about the meaning of \(\nabla \cdot \mathbf{F}\) and \(\nabla \times \mathbf{F}\). \\ 

As a final note, we will start to blur a bit the distinction between \(\R^{2}\) and \(\R^{3}\) here. We can think of always working within \(\R^{3}\), and that \(\R^{2}\) is the special case when the \(k\) component is 0.

\subsection{Example}

Let \(f(x,y) = x^{2} + y^{2} - 2x - 6y + 14\). Find the direction derivative of \(f\) at \((4,1)\), in the \(\brackett{1,3}\) direction. \\

\sol{
    We first find the partial derivative of \(f\) with respect to \(x\) and \(y\):
    \[
    f_{x}(x,y) = 2x - 2 \quad \text{and} \quad f_{y}(x,y) = 2y - 6.
    \]
    Then, at the point \((4,1)\):
    \[
    f_{x}(4,1) = 6 \quad \text{and} \quad f_{y}(4,1) = -4.
    \]
    Then, we find the unit vector in the direction of \(\brackett{1,3}\):
    \[
        \frac{\brackett{6,-4} \cdot \brackett{1,3}}{\mysqrt{10}} = \frac{-6}{\mysqrt{10}}.
    \]
}

\newpage

\section{Maxima and Minima}

For this section, we will be considering the specific case \(f \colon \R^{2} \to \R\).

\subsection{Critical Points}

The statement that \((x_{0}, y_{0})\) is a \cyanit{critical point} of \(f\) means that either:
\begin{itemize}
    \item both \(f_{x}(x_{0}, y_{0}) = 0\) and \(f_{y}(x_{0}, y_{0}) = 0\), or
    \item one or both partials does not exist.
\end{itemize}

\subsection{Maxima and Minima}

\subsubsection{Local Extrema}

The function \(f\) has a \cyanit{local maximum} at \((x_{0}, y_{0})\) provided that \(f(x_{0}, y_{0}) \geq f(x, y)\) for all choices of \((x, y)\) in some disk centered at \((x_{0}, y_{0})\) – that is, in some neighborhood of \((x_{0}, y_{0})\). \\

In a similar manner, we can define a local minimum. \\


We will use the term \cyanit{extremum} to refer to something that is either a maximum or minimum; the plural is \cyanit{extrema}. \\

Note that if there is a local extrema, \(\nabla f = \mb{0}\). This is because the gradient points in the direction of greatest increase, and if we are at a maximum or minimum, the function does not change. \\

\textbf{Fermat's Theorem for Extrema:} Suppose that \(z = f(x, y)\) and that \((x_{0}, y_{0})\) is a local extremum. Then, \((x_{0}, y_{0})\) is also a critical point.

\subsubsection{Second Derivative Test}

\textbf{Calculus I Version:} In Calculus I, the sign of the second derivative tells you whether a critical point is a local max/min, or inconclusive: Suppose that \(g \colon \R \to \R\) is a function of one variable, and \(x_{0}\) is a critical point.

\begin{itemize}
    \item if \(g''(x_{0}) > 0\), then \(x_{0}\) is a local minimum
    \item if \(g''(x_{0}) < 0\), then \(x_{0}\) is a local maximum
    \item if \(g''(x_{0}) = 0\), this test is inconclusive – it could be a max, min, or neither.
\end{itemize}

\newpage

\textbf{Multivariable Calculus Version:} We have a similar test for \(f \colon \R^{2} \to \R\), where \((x_{0}, y_{0})\) is a critical point. Define
\[
D = f_{xx}(x_{0}, y_{0})f_{yy}(x_{0}, y_{0}) - (f_{xy}\bigl(x_{0}, y_{0})\bigr)^{2}.
\]

\begin{itemize}
    \item if \(D > 0\) and \(f_{xx}(x_{0}, y_{0}) > 0\), then \(f\) has a local minimum
    \item if \(D > 0\) and \(f_{xx}(x_{0}, y_{0}) < 0\), then \(f\) has a local maximum
    \item if \(D < 0\), then \(f\) has a saddle point
    \item if \(D = 0\), then the test is inconclusive
\end{itemize}

\subsubsection{Absolute Extrema}

The function \(f\) has an \cyanit{absolute maximum} at \((x_{0}, y_{0})\) provided that \(f(x_{0}, y_{0}) \geq f(x, y)\) for all choices of \((x, y)\) in the domain of \(f\). Likewise, we can define an \cyanit{absolute minimum}. \\

\textbf{Extreme Value Theorem:} Suppose that \(f\) is continuous on a closed, bounded domain \(D\). Then \(f\) has both an absolute maximum and absolute minimum. \\

In Calculus I, we saw that a continuous function, defined on a closed interval, has its absolute extrema either at interior critical points or at the endpoints of the domain interval. This idea transfers over here – but is somewhat more complicated. \\

If \(f \colon \R^{2} \to \R\) is continuous on a closed and bounded domain, then by the Extreme Value Theorem, it has absolute extrema. These occur either at interior critical points of the domain or somewhere along the edge of the domain. The interior critical points are fairly straightforward. However, the edge of the domain is more complicated.

\subsubsection{Extrema Finding Algorithm}

\begin{itemize}
    \item Find all critical points inside the region of interest by setting the two first partials equal to zero
    \item For each boundary, use the relationship between \(x\) and \(y\) to convert \(f(x, y)\) to a new function, say \(g(x)\), of a single variable
    \item This function will have an interval \([a, b]\) as its domain
    \item Work the Calculus I problem of finding absolute extrema for \(g\) on \([a, b]\). This will generate additional points of interests.
    \begin{itemize}
        \item Absolute extrema occur at:
        \begin{enumerate}[label=\arabic*.]
            \item Interior critical points.
            \item Along the boundary.
        \end{enumerate}
    \end{itemize}
    \item Find the value of \(f(x, y)\) for each point of interest – the largest value is the absolute maximum, the smallest is the absolute minimum.
\end{itemize}

\subsection{Examples}

Let \(f(x,y) = x^{2} + y^{2} - 2x - 6y + 14\).  \\

\sol{
    We know that the critical points are \((1,3)\). We can use the second derivative test to determine that this is a local minimum.
\[
    f_{xx} = 2 \quad f_{yy} = 2 \quad f_{xy} = 0 \quad D = 4 > 0.
\]
}

Let \(f(x,y) = x^{4} + y^{4} -4xy + 1\). \\

\sol{
    Find critical points:
    \[
        f_{x} = 4x^{3} - 4y \quad f_{y} = 4y^{3} - 4x,
    \]
    and set to 0:
    \[
        x^{3} = y \quad y^{3} = x.
    \]
    Substituting, we see:
    \begin{align*}
        x^{9} - x &= 0 \\
        x(x^{8} - 1) &= 0 \\
        x(x^{4} + 1)(x^{2} + 1)(x + 1)(x - 1) &= 0. \\
        x &= 0, \pm 1 \\ 
        \therefore y &= 0, \pm 1.
    \end{align*}
    Now we have our critical points \((0,0),(1,1),(-1,-1)\). We can use the second derivative test to determine the maximum or minimums. But, to use this formula, we need the second partials for the formula:
    \[
        f_{xx} = 12x^{2} \quad f_{yy} = 12y^{2} \quad f_{xy} = -4.
    \]
    For \((0,0)\):
    \[
        0 \cdot 0 - (4)^{2} = - 16 < 0.
    \]
    Therefore, \((0,0)\) is a saddle point. For \((1,1)\):
    \[
        12 \cdot 12 - 16 = 128 > 0.
    \]
    Therefore, \((1,1)\) is a local minimum. For the same reasons, \((-1,-1)\) is also a local minimum.
}
\newpage
From Calculus I, remember we found the absolute extrema. Thus, let \(g(x) = x^{2} - 6x + 1\), on \([0,4]\). 
\sol{
    To find the absolute extrema of this function, get the critical points from the function, and also evaluate the function at the endpoints of the interval. 
    \[
        g'(x) = 2x - 6 = 0 \implies x = 3.
    \]
    Then, \(g(0) = 1\), \(g(3) = -8\), and \(g(4) = -7\). Thus, the absolute minimum is \((-8)\) and the absolute maximum is 1.
}

With that refresher, lets return to our \(f(x,y)\) function and find the absolute extrema. Specifically, find absolute extreme on the triangle with vertices \((0,0),(0,1),(\frac{1}{2},1)\). 

\sol{

}

\chapter{Multiple Integration}\label{multiple integration}
\vspace*{-0.25in}
\section{Double Integrals}

\subsection{Integration in \(\R\)}

Recall that for a continuous function \(f : \R \to \R\), the \cyanit{signed area} between the graph of \(y = f(x)\) and the
\(x\)-axis on the interval \([a, b]\) is given by 
\[
\int_a^b f(x) \, dx.
\]
Recall also that
\[
\int_a^b f(x) \, dx \approx \sum_{i=1}^n f(x^*_i) \Delta x,
\]
where \(\Delta x = \frac{b-a}{n}\) and \(x^*_i\) is any convenient point in the \(i\)th subinterval. The sum above is called a \cyanit{Riemann Sum}.

\subsection{Integration in \(\R^2\)}

Let \(R\) be a rectangle defined by

\[
R = [a, b] \times [c, d] = \{(x, y) \in \R^2 : a \leq x \leq b, \, c \leq y \leq d\}.
\]
Suppose that \(f : R \to \R\) be continuous. If we interpret \(f(x, y)\) as a height above or below the \(xy\)-plane, we can find the \cyanit{signed volume} of the solid
\[
S = \{(x, y, z) \in \R^3 : (x, y) \in R, z \text{ lives between } 0 \text{ and } f(x, y)\}.
\]
We can divide \(R\) into \(m \cdot n\) rectangles by
\[
\Delta x = \frac{b-a}{m} \quad \text{and} \quad \Delta y = \frac{d-c}{n}
\]
and then write 
\[
\Delta A = \Delta x \Delta y.
\]
Then,
\[
V \approx \sum_{i=1}^m \sum_{j=1}^n f(x^*_{ij}, y^*_{ij}) \Delta A,
\]
where \((x^*_{ij}, y^*_{ij})\) is a point selected from the \(ij\)th rectangle.
\newpage
If \(f\) is continuous, then
\[
\lim_{m,n \to \infty} \sum_{i=1}^m \sum_{j=1}^n f(x^*_{ij}, y^*_{ij}) \Delta A
\]
exists thus we define the \cyanit{double integral} of \(f\) over the rectangle \(R\) as
\[
\iint_R f(x, y) \, dA = \lim_{m,n \to \infty} \sum_{i=1}^m \sum_{j=1}^n f(x^*_{ij}, y^*_{ij}) \Delta A.
\]
\subsubsection{Double Integral – Properties}
Basically, the properties are exactly what you’d expect – we’ll assume \(R\) is a rectangle, \(c\) is constant, and \(f, g\) are continuous:
\begin{itemize}
    \item \(\iint_R (f + g) \, dA = \iint_R f \, dA + \iint_R g \, dA\)
    \item \(\iint_R cf \, dA = c \iint_R f \, dA\)
    \item \(\iint_R f \, dA = \iint_S f \, dA + \iint_T f \, dA\), if \(S\) and \(T\) are rectangles, \(R = S \cup T\) and \(S \cap T\) is one-dimensional
    \item if \(f \geq g\) on \(R\), then \(\iint_R f \, dA \geq \iint_R g \, dA\)
    \item if \(m \leq f(x, y) \leq M\) on \(R\) then \(m \cdot \text{area}(R) \leq \iint_R f \, dA \leq M \cdot \text{area}(R)\)
\end{itemize}
\subsubsection{Factorization Property}  
Suppose that \(f(x, y) = g(x) \cdot h(y)\) – that is, suppose we can factor \(f\) into a product of one function involving only \(x\) and another involving only \(y\). Then,
\[
\iint_R f(x, y) \, dA = \left( \int_a^b g(x) \, dx \right) \left( \int_c^d h(y) \, dy \right).
\]
\subsubsection{Iterated Integrals}
For a fixed \(x\), define
\[
A(x) = \int_c^d f(x, y) \, dy.
\]
This can be called a \cyanit{partial integral}, and is a function of \(x\). It is the area of one slice of the solid. Then,
\[
\int_a^b A(x) \, dx = \int_a^b \left( \int_c^d f(x, y) \, dy \right) dx = \int_a^b \int_c^d f(x, y) \, dy dx,
\]

where we call this an \cyanit{iterated integral}. (There is, of course, nothing special about doing \(y\) first and then \(x\) here\ldots.)

\subsubsection{Fubini’s Theorem}

If \(f\) is continuous on \(R = [a, b] \times [c, d]\) then
\[
\iint_R f(x, y) \, dA = \int_b^a \int_d^c f(x, y) \, dy dx = \int_d^c \int_b^a f(x, y) \, dx dy.
\]
This is just like Clariaut’s Theorem, but for integrals, rather than partial derivatives. You can work the integral in whatever order is most convenient.
\subsubsection{Area}
Suppose that \(R\) is a rectangle. Then, the area of \(R\) is given by
\[
\text{area}(R) = \iint_R 1 \, dA.
\]

\subsubsection{Average Value}

The \cyanit{average value} of \(f\) over \(R\) is given by
\[
\text{ave}(f) = \frac{1}{\text{area}(R)} \iint_R f(x, y) \, dA.
\]

\begin{tikzcd}
    Y \arrow[r] \arrow[d] 
      & X \arrow[d] \\
    T \arrow[r]
      & S
\end{tikzcd}

\(\frac{d}{dx}(x^{3} + x^{2} + 1)\)

\chapter{Vector Calculus}\label{vector calculus}
\vspace*{-0.25in}
\section{Vector Fields}

\subsection{Introduction}
So far, we have talked about:

\begin{itemize}
    \item Parametric Functions: \( \mb{F} \colon \mathbb{R} \to \mathbb{R}^n \) (also called vector-valued functions)
    \item Scalar Functions: \( \mb{F} \colon \mathbb{R}^n \to \mathbb{R} \)
\end{itemize}

We conclude the course by considering:

\begin{itemize}
    \item \( \mb{F} \colon \mathbb{R}^n \to \mathbb{R}^n \), specifically:
    \begin{itemize}
        \item \( \mb{F} \colon \mathbb{R}^2 \to \mathbb{R}^2 \)
        \item \( \mb{F} \colon \mathbb{R}^3 \to \mathbb{R}^3 \)
    \end{itemize}
\end{itemize}

\subsection{Vector Field}
A vector field is an assignment to each point in \( \mathbb{R}^n \) a vector in \( \mathbb{R}^n \). For example:

\begin{itemize}
    \item At each point on the Earth's surface, assign the current wind speed and direction.
    \item A force field – gravity, electromagnetism, etc.
    \item Fluid flow – in a pipe, the rate and direction of flow of water.
    \item A gradient field – given a surface, its gradient at each point.
\end{itemize}

Formally, a vector field is a function \( \mathbf{F} : \mathbb{R}^n \to \mathbb{R}^n \) with some domain \( D \subseteq \mathbb{R}^n \). We will only be concerned with \( n = 2 \) and \( n = 3 \). \\

\textbf{Common notation:}
\[ \mathbf{F}(x, y) = \langle P(x, y), Q(x, y) \rangle = P(x, y) \mathbf{i} + Q(x, y) \mathbf{j} \]
and
\[ \mathbf{F}(x, y, z) = \langle P(x, y, z), Q(x, y, z), \mb{r}(x, y, z) \rangle = P(x, y, z) \mathbf{i} + Q(x, y, z) \mathbf{j} + \mb{r}(x, y, z) \mathbf{k} \]

\subsection{Graph}
To truly graph a vector field, you need 4 or 6 dimensions, which is somewhat inconvenient. Instead, we plot representative vectors in either \( \mathbb{R}^2 \) or \( \mathbb{R}^3 \).

The applet at \url{https://www.geogebra.org/m/QPE4PaDZ} is one way to graph a 2-D vector field.

\subsection{Special Vector Fields}
There are a few commonly seen special types of vector fields:

\begin{itemize}
    \item \textbf{Radial} – each vector points either directly toward or away from the origin. For example, \( \langle x, y \rangle \).
    \item \textbf{Rotational} – each vector points tangent to a circle centered at the origin. For example, \( \langle -y, x \rangle \).
    \item \textbf{Unit} – each output vector has magnitude 1. For example, \( \langle \sin(x + y), \cos(x + y) \rangle \).
    \item \textbf{Velocity} – the vectors represent the velocity of some particle.
    \item \textbf{Gradient} – the vectors are the gradient of some function \( \mb{F} \colon \mathbb{R}^n \to \mathbb{R} \).
\end{itemize}

\subsection{Gradient Field}
A vector field \( \mathbf{F} \) is called a gradient vector field if there exists some scalar function \( \mb{F} \colon \mathbb{R}^n \to \mathbb{R} \) such that \( \nabla f = \mathbf{F} \). That is, if \( \mathbf{F} \) is the gradient of some scalar function \( f \).

\subsection{Conservative Vector Fields}
A vector field \( \mathbf{F} \) is a conservative vector field when it is the gradient field for some scalar function \( f \). We will call \( f \) the \textbf{potential function} of \( \mathbf{F} \).

\textbf{Theorem.} Potential functions are unique, except for a constant. That is, if \( f \) and \( g \) are each potential functions for \( \mathbf{F} \), then there exists some constant \( c \) such that \( f - g = c \).

\subsection{Cross Partial Property}
\textbf{Theorem.} Suppose that \( \mathbf{F} \) is a conservative vector field with continuous mixed second partials. Then:

\begin{itemize}
    \item In two dimensions, if \( \mathbf{F} = \langle P,Q \rangle \), then \( \frac{\partial P}{\partial y} = \frac{\partial Q}{\partial x} \).
    \item In three dimensions, if \( \mathbf{F} = \langle P, Q, R \rangle \), then:
    \begin{itemize}
        \item \( \frac{\partial P}{\partial y} = \frac{\partial Q}{\partial x} \)
        \item \( \frac{\partial Q}{\partial z} = \frac{\partial R}{\partial y} \)
        \item \( \frac{\partial R}{\partial x} = \frac{\partial P}{\partial z} \)
    \end{itemize}
\end{itemize}

\newpage

\section{Line Integrals}

\subsection{Scalar Line Integral}
Suppose that \( \mb{F} \colon \mathbb{R}^n \to \mathbb{R} \) is a continuous scalar function and that \( C \) is a continuous curve in \( \mathbb{R}^n \) included in the domain of \( f \). We want to find the area of the sheet which extends down from \( f \) to the domain along the curve \( C \). We parameterize \( C \) as \( \mb{r}(t) \), \( a \leq t \leq b \) and break \( C \) into small subintervals of length \( \Delta t = \frac{b-a}{n} \). Then, we have:
\begin{align*}
    \Delta s_1 &= [\mb{r}(a), \mb{r}(a + \Delta t)] \\
    \Delta s_2 &= [\mb{r}(a + \Delta t), \mb{r}(a + 2\Delta t)] \\
    &\vdots \\
    \Delta s_n &= [\mb{r}(a + (n - 1)\Delta t), \mb{r}(a + n\Delta t)]
\end{align*}
In each \( \Delta s_i \), choose \( t^*_i \) and find \( f(\mb{r}(t^*_i )) \). Then, we have a Riemann sum:
\[ \sum_{i=1}^{n} f(\mb{r}(t^*_i ))\Delta s_i, \]
which, if the limit as \( n \to \infty \) exists, we will write as one of:
\[ \int_C f(x, y) \, ds \quad \text{or} \quad \int_C f(x, y, z) \, ds. \]

The term \( ds \) represents the infinitesimal step taken along the path \( C \), analogous to \( dx \) in single-variable integration. Since working with \( ds \) directly is cumbersome, we use:
\[ \int_C f(x, y) \, ds = \int_C f(x(t), y(t)) \sqrt{(x'(t))^2 + (y'(t))^2} \, dt = \int_a^b f(\mb{r}(t)) \| \mb{r}'(t) \| \, dt, \]
which simplifies computations and ensures that parametrization does not affect the integral's value, though orientation does. If \( C \) and \( D \) are the same curve with opposite orientations:
\[ \int_C f \, ds = - \int_D f \, ds, \]
analogous to:
\[ \int_a^b f(x) \, dx = - \int_b^a f(x) \, dx. \]

\subsection{Line Integrals over Vector Fields}
Suppose that \( \mb{F} \colon \mathbb{R}^n \to \mathbb{R}^n \) is a vector field, and \( C \) is a continuous curve in \( \mathbb{R}^n \) within the domain of \( \mb{F} \). We wish to evaluate:
\[ \int_C \mb{F} \cdot d\mb{s}, \]
which essentially ``multiplies'' \(\mb{F}\) by a small infinitesimal along the curve \(C\) (denoted by \(d\mb{s}\)). \\

We choose the points \(P_{0}, P_{1}, P_{2}, \ldots, P_{i},\ldots, P_{n}\) to divide up our curve \(C\) and then approximate our line integral as a Riemann sum. We can write the line integral as:
\[ \int_C \mb{F} \cdot d\mb{s} = \lim_{n\to\infty} \sum_{i=1}^{n} F(P^*_i) \cdot \Delta \mb{s}_i. \]
Since \( d\mb{s} = \mb{r}'(t) dt \), the integral simplifies to:
\[ \int_C \mb{F} \cdot d\mb{s} = \int_a^b F(\mb{r}(t)) \cdot \mb{r}'(t) \, dt, \]

Where \( \mb{r}(t) \), \(a \leq t \leq b\) is any parametrization of \(C\). \\ 

Equivalent notations include:
\[ \int_C \mb{F} \cdot d\mb{r}, \quad \int P \, dx + Q \, dy, \quad \text{or} \quad \int P \, dx + Q \, dy + R \, dz. \]

\subsection{Properties of Line Integrals over Vector Fields}
\begin{itemize}
    \item \( \int_C (\mb{F} + G) \cdot d\mb{r} = \int_C \mb{F} \cdot d\mb{r} + \int_C G \cdot d\mb{r} \)
    \item \( \int_C k\mb{F} \cdot d\mb{r} = k \int_C \mb{F} \cdot d\mb{r} \), for constant \( k \)
    \item \( \int_C \mb{F} \cdot d\mb{r} = - \int_{-C} \mb{F} \cdot d\mb{r} \), where \( -C \) denotes \( C \) with reversed orientation
    \item If \( C = C_1 + C_2 \) with intersection only at endpoints, then:
    \[ \int_C \mb{F} \cdot d\mb{r} = \int_{C_1} \mb{F} \cdot d\mb{r} + \int_{C_2} \mb{F} \cdot d\mb{r}. \]
\end{itemize}

\subsection{Circulation Along a Curve}
The circulation of \( \mb{F} \) along \( C \) is defined as:
\[ \int_C \mb{F} \cdot dr. \]
If \( \mb{F} \) represents a force field, this integral gives the total work done by \( \mb{F} \) in moving an object along \( C \).

\subsection{Flux Across a Curve}
Given a vector field \(\mb{F}\) and curve \(C\) in \(R^{2}\), the \cyanit{flux} of \(\mb{F}\) across \(C\) is the total amount of \(\mb{F}\) that crosses orthogonally across \(C\). We note that there is an orientation issue here. We define positive flux as moving from left to right across the curve where forward is the direction of the curve’s orientation. In two dimensions, this is actually opposite of \(N\) from chapter 2, but we’ll fix that easily by defining \(n\) to be the unit vector which points left-to-right across the curve. We can see that \(n(t) = \langle y'(t), -x'(t) \rangle\).
Thus, the flux integral is given by:
\[ \int_C \mb{F} \cdot N \, ds = \int_a^b \mb{F}(\mb{r}(t)) \cdot \langle y'(t), -x'(t) \rangle \, dt, \]
or equivalently:
\[ \int_C -Q \, dx + P \, dy. \]

\section{Examples}
\subsection{Example 1: Scalar Line Integral}
Evaluate \( \int_C (x^2 + y^2) ds \) where \( C \) is the quarter-circle \( \mb{r}(t) = (\cos t, \sin t) \), \( 0 \leq t \leq \frac{\pi}{2} \).

\textbf{Solution:}
\begin{align*}
    ds &= \|\mb{r}'(t)\| dt = \sqrt{(-\sin t)^2 + (\cos t)^2} dt = dt. \\
    f(\mb{r}(t)) &= \cos^2 t + \sin^2 t = 1. \\
    \int_C (x^2 + y^2) ds &= \int_0^{\pi/2} 1 \, dt = \frac{\pi}{2}.
\end{align*}

\subsection{Example 2: Vector Line Integral}
Evaluate \( \int_C \mb{F} \cdot dr \) where \( \mb{F} = \langle y, x \rangle \) and \( C \) is the line from \( (0,0) \) to \( (1,1) \).

\textbf{Solution:}
\begin{align*}
    \mb{r}(t) &= (t, t), \quad 0 \leq t \leq 1. \\
    \mb{r}'(t) &= (1,1). \\
    F(\mb{r}(t)) &= \langle t, t \rangle. \\
    \int_C \mb{F} \cdot dr &= \int_0^1 \langle t, t \rangle \cdot \langle 1, 1 \rangle \, dt = \int_0^1 (t + t) \, dt = \int_0^1 2t \, dt = 1.
\end{align*}




%-%-%-%-%-%-%-%-%-%-%-%-%-%-%-%-%-%-%-%-%-%-%-%-%-%-%-%-%-%-%-%-%-%-%-%-%-%-%
\end{document}
%-%-%-%-%-%-%-%-%-%-%-%-%-%-%-%-%-%-%-%-%-%-%-%-%-%-%-%-%-%-%-%-%-%-%-%-%-%-%

%-%-%-%-%-%-%-%-%-%-%-%-%-%-%-%-%-%-%-%-%-%-%-%-%-%-%-%-%-%-%-%-%-%-%-%-%-%-%