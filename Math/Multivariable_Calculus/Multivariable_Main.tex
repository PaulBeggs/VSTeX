%-%-%-%-%-%-%-%-%-%-%-%-%-%-%-%-%-%-%-%-%-%-%-%-%-%-%-%-%-%-%-%-%-%-%-%-%-%-%
%%% MAIN DOCUMENT %%%

%-%-%-%-%-%-%-%-%-%-%-%-%-%-%-%-%-%-%-%-%-%-%-%-%-%-%-%-%-%-%-%-%-%-%-%-%-%-%

\documentclass[12pt, oneside]{book}
\usepackage{C:/Users/paulb/VSTeX/local/draculatheme}
% \usepackage[T1]{fontenc}


%%% AESTHETICS %%%
%-%-%-%-%-%-%-%-%-%-%-%-%-%-%-%-%-%-%-%-%-%-%-%-%-%-%-%-%-%-%-%-%-%-%-%-%-%-%


%%% Dimensions and Spacing %%%
\usepackage[margin=1in]{geometry}
\usepackage{setspace}
\linespread{1}
\usepackage{listings}
\usepackage{tikz}
\usetikzlibrary{shapes,backgrounds,calc,patterns,positioning}
\usepgflibrary{shadings}

%%% Define new colors %%%
\usepackage{xcolor}
\definecolor{orangehdx}{rgb}{0.96, 0.51, 0.16}

% Normal colors
\definecolor{xred}{HTML}{BD4242}
\definecolor{xblue}{HTML}{4268BD}
\definecolor{xgreen}{HTML}{52B256}
\definecolor{xpurple}{HTML}{7F52B2}
\definecolor{xorange}{HTML}{FD9337}
\definecolor{xdotted}{HTML}{999999}
\definecolor{xgray}{HTML}{777777}
\definecolor{xcyan}{HTML}{80F5DC}
\definecolor{xpink}{HTML}{F690EA}
\definecolor{xgrayblue}{HTML}{49B095}
\definecolor{xgraycyan}{HTML}{5AA1B9}

% Dark colors
\colorlet{xdarkred}{red!85!black}
\colorlet{xdarkblue}{xblue!85!black}
\colorlet{xdarkgreen}{xgreen!85!black}
\colorlet{xdarkpurple}{xpurple!85!black}
\colorlet{xdarkorange}{xorange!85!black}
\definecolor{xdarkcyan}{HTML}{008B8B}
\colorlet{xdarkgray}{xgray!85!black}

% Very dark colors
\colorlet{xverydarkblue}{xblue!50!black}

% Document-specific colors
\colorlet{normaltextcolor}{black}
\colorlet{figtextcolor}{xblue}

% Enumerated colors
\colorlet{xcol0}{black}
\colorlet{xcol1}{xred}
\colorlet{xcol2}{xblue}
\colorlet{xcol3}{xgreen}
\colorlet{xcol4}{xpurple}
\colorlet{xcol5}{xorange}
\colorlet{xcol6}{xcyan}
\colorlet{xcol7}{xpink!75!black}

% Blue-Purple (should just used colorbrewer...)
\definecolor{xrainbow0}{HTML}{e41a1c}
\definecolor{xrainbow1}{HTML}{a24057}
\definecolor{xrainbow2}{HTML}{606692}
\definecolor{xrainbow3}{HTML}{3a85a8}
\definecolor{xrainbow4}{HTML}{42977e}
\definecolor{xrainbow5}{HTML}{4aaa54}
\definecolor{xrainbow6}{HTML}{629363}
\definecolor{xrainbow7}{HTML}{7e6e85}
\definecolor{xrainbow8}{HTML}{9c509b}
\definecolor{xrainbow9}{HTML}{c4625d}
\definecolor{xrainbow10}{HTML}{eb751f}
\definecolor{xrainbow11}{HTML}{ff9709}

%------- %
% XHFILL %
%------- %



%%% Chapter Headings %%%

\newcommand{\gradientrule}{
    \begin{tikzpicture}
        \shade[left color=orangehdx, right color=black, middle color=gray] (0,0) rectangle (\linewidth,0.4pt);
    \end{tikzpicture}
}

\usepackage[Glenn]{fncychap}
\ChTitleVar{\bfseries\scshape\color{draculafg}} % Needed for Dracula theme
\ChNumVar{\large\selectfont\color{draculafg}} % Needed for Dracula theme
\ChNameVar{\large\color{draculafg}} % Needed for Dracula theme
% \ChTitleVar{\bfseries\scshape\color{black}}
% \ChNumVar{\large\selectfont\color{black}}
% \ChNameVar{\large\color{black}}
\usepackage{xpatch}

% \xpatchcmd{\DOCH}
%   {\mghrulefill}{\gradientrule\mghrulefill}
%   {}{\PatchFailed}
% \xpatchcmd{\DOTI}
%   {\mghrulefill}{\gradientrule\mghrulefill}
%   {}{\PatchFailed}
% \xpatchcmd{\DOTIS}
%   {\mghrulefill}{\gradientrule\mghrulefill}
%   {}{\PatchFailed}

\xpatchcmd\DOCH
{\mghrulefill}{\color{orangehdx}\mghrulefill}
{}{\PatchFailed}
\xpatchcmd\DOTI
{\mghrulefill}{\color{orangehdx}\mghrulefill}
{}{\PatchFailed}
\xpatchcmd\DOTIS
{\mghrulefill}{\color{orangehdx}\mghrulefill}
{}{\PatchFailed}


% \usepackage[Bjornstrup]{fncychap}

% \newcommand{\gradient}[1]{
% \begin{tikzpicture}
%     \node (rect) at (0,0) [fill=blue,,path fading=East,minimum width=\linewidth,minimum height=2.5cm] {};
%     \node(title)[above left = 10pt and 10pt of rect.south east, anchor=south east, font=\CTV] {\textcolor{horange}{#1}};
%     \ifnum \thechapter>0\node[left = 10pt of rect.north east,  anchor=center, font=\CNoV] {\textcolor{horange}{\thechapter}};\fi%
% \end{tikzpicture}%
% \vskip 40pt
% }


% \renewcommand{\DOCH}{}
% \renewcommand{\DOTI}[1]{\gradient{#1}}
% \renewcommand{\DOTIS}[1]{\gradient{#1}}

%% Change Chapter Heading Placement %%
\usepackage{etoolbox}
\makeatletter
\patchcmd{\@makechapterhead}{\vspace*{50\p@}}{\vspace*{-20\p@}}{}{}
\patchcmd{\@makeschapterhead}{\vspace*{50\p@}}{\vspace*{-20\p@}}{}{}
\patchcmd{\DOTI}{\vskip 80\p@}{\vskip 40\p@}{}{}
\patchcmd{\DOTIS}{\vskip 40\p@}{\vskip 0\p@}{}{}
\makeatother

% \usepackage[explicit]{titlesec}
% \newcommand*\chapterlabel{}\pmod
% \titleformat{\chapter}
%   {\gdef\chapterlabel{}
%    \normalfont\sffamily\Huge\bfseries\scshape}
%   {\gdef\chapterlabel{\thechapter\ }}{0pt}
%   {\begin{tikzpicture}[remember picture,overlay]
%     \node[yshift=-3cm] at (current page.north west)
%       {\begin{tikzpicture}[remember picture, overlay]
%         \draw[fill=LightSkyBlue] (0,0) rectangle
%           (\paperwidth,3cm);
%         \node[anchor=east,xshift=.9\paperwidth,rectangle,
%               sharp corners=downhill=20pt,inner sep=11pt,
%               fill=MidnightBlue]
%               {\color{white}\chapterlabel#1};
%        \end{tikzpicture}
%       };
%    \end{tikzpicture}
%   }
% \titlespacing*{\chapter}{0pt}{50pt}{-60pt}

% \usepackage[Conny]{fncychap}
% \usepackage[Rejne]{fncychap}
% \ChNameVar{\bfseries}  % Makes the chapter "Chapter #" bold
% \ChNumVar{\bfseries}   % Makes the chapter number bold
% \ChTitleVar{\bfseries} % Makes the chapter title bold
% % Define a custom color, for example:
% \definecolor{mycolor}{RGB}{0,128,255}

% % Redefine the chapter style in Rejne to change the line colors
% \makeatletter
% \ChRuleWidth{2pt}   % Change the thickness of the lines
% \renewcommand{\DOCH}{%
%   \vspace*{-50\p@}% Moves the chapter title up/down if needed
%   {\color{mycolor} \hrule \@chapapp{} \space \thechapter \hrule}% Customizes the chapter header with color
% }
% \renewcommand{\DOTI}[1]{%
%   \vskip 20\p@ % Adjusts the space above the title
%   \bfseries #1\par % Embolden the title
%   \vskip 20\p@ % Adjusts the space below the title
% }
% \makeatother

% %% Change Chapter Heading Placement %%
% \usepackage{etoolbox}
% \makeatletter
% \patchcmd{\@makechapterhead}{\vspace*{50\p@}}{\vspace*{-20\p@}}{}{}
% \patchcmd{\@makeschapterhead}{\vspace*{50\p@}}{\vspace*{-20\p@}}{}{}
% \patchcmd{\DOTI}{\vskip 80\p@}{\vskip 40\p@}{}{}
% \patchcmd{\DOTIS}{\vskip 40\p@}{\vskip 0\p@}{}{}
% \makeatother

\renewcommand{\thesection}{\thechapter.\arabic{section}} %% Chapter.Section Numbering


%%% FIGURES %%%
\usepackage{graphicx}  
\graphicspath{ {images/} }  
% \numberwithin{figure}{section}
\usepackage{float}
\usepackage{caption}

%%% Hyperlinks %%%
\usepackage{hyperref}
\definecolor{horange}{HTML}{f58026}
\hypersetup{
	colorlinks=true,
	linkcolor=horange,
	filecolor=horange,      
	urlcolor=horange,
}


%% Headers and Footers %%
\usepackage{fancyhdr} % This should be set AFTER setting up the page geometry
\pagestyle{fancy} % options: empty , plain , fancy
\renewcommand{\chaptermark}[1]{\markboth{\thechapter.\ #1}{}}
\fancyhead[R]{\leftmark}
\fancyhead[L]{Hendrix College}
% \fancyhead[C]{\includegraphics[height=.50cm]{images/small logo.png}}
\fancyhead[C]{\includegraphics[height=.50cm]{images/small logo_white.png}} % Use for Dracula
\usepackage{xpatch}
\xpretocmd\headrule{\color{orangehdx}}{}{\PatchFailed}
\setlength{\footskip}{0.5in}
\setlength{\headheight}{18.5764pt}


%%%%% Colored Boxes %%%%%
\usepackage{tcolorbox}
\tcbuselibrary{skins}
\tcbuselibrary{theorems}
\tcbuselibrary{minted}
\newcounter{BoxCounter}


\newtcolorbox{note}{colback=black!15!white, colframe=black, boxrule=0.5mm, arc=4mm, left=4mm, right=4mm, top=2mm, bottom=2mm}

%% Numbered Theorems %%
\newtcolorbox[use counter=BoxCounter, number within=section]{theorem}[1][]{%
enhanced,
sharp corners=uphill,
    colframe=xblue, %color of frame
    colback=xblue!15, %color of background
    coltitle=white, %color of title text
    label={thm:\theBoxCounter},
    title={\large \textbf{Theorem \theBoxCounter}},
    attach boxed title to top left={xshift=0.5cm},
    boxed title style={colback=xblue},
    #1
}


%% Named Theorems %%
\newtcolorbox[use counter=BoxCounter, number within=section]{ntheorem}[2][]{%
enhanced,
sharp corners=south,
    colframe=xblue, %color of frame
    colback=xblue!15, %color of background
    coltitle=white, %color of title text
    label={thm:#2},
    title={\large \textbf{Theorem \theBoxCounter: #2}},
    attach boxed title to top center,
    boxed title style={colback=xblue},
    #1
}


%% Definitions %%
\newtcolorbox[use counter=BoxCounter, number within=section]{definition}[1][]{%
enhanced,
sharp corners=downhill,
    colframe=xred,
    colback=xred!15,
    coltitle=white,
    label={def:\theBoxCounter},
    title={\large \textbf{Definition \theBoxCounter}},
    attach boxed title to top right={xshift=-0.5cm},
    boxed title style={colback=xred},
    #1
}


%% Axioms %%
\newtcolorbox[]{axiom}[2][]{%
enhanced,
sharp corners=downhill,
    colframe=xdarkgreen,
    colback=xdarkgreen!15,
    coltitle=white,
    label={ax:#2},
    title={\large \textbf{Axiom of #2}},
    attach boxed title to top left={xshift=0.5cm},
    boxed title style={colback=xdarkgreen},
    #1
}


%% Exercises %%
\newtcolorbox[]{exercise}[2][]{%
    enhanced,
    sharp corners=uphill,
    colframe=orangehdx!75,
    colback=orangehdx!10,
    coltitle=black,
    label={exerc:#2},
    title={\large \textbf{Exercise: #2}},
    attach boxed title to top left={xshift=0.5cm},
    boxed title style={colback=orangehdx!75},
    #1,
}


%% Examples %%
\newcounter{ExampleCounter}
\newtcolorbox[use counter=ExampleCounter, number within=chapter]{example}[2][]{%
enhanced,
sharp corners=south,
    colframe=teal,
    colback=teal!15,
    coltitle=white,
    label={ex:\theExampleCounter},
    title={\large \textbf{Example \theExampleCounter: #2}},
    attach boxed title to top center,
    boxed title style={colback=teal},
    #1,
}


%% Corollaries %%
\newtcolorbox[]{corollary}[3][]{%
enhanced,
sharp corners=downhill,
    colframe=xgrayblue,  
    colback=xgrayblue!15,        
    coltitle=white,
    title={\large \textbf{Corollary to #3: #2}},
    attach boxed title to top left={xshift=0.5cm},
    boxed title style={colback=xgrayblue},
    #1
}


%% Lemmas %%
\newtcolorbox[use counter=BoxCounter, number within=chapter]{lemma}[1][]{%
enhanced,
sharp corners=uphill,
    colframe=xpurple,
    colback=xpurple!15,
    coltitle=white,
    label={lm:\theBoxCounter},
    title={\large \textbf{Lemma \theBoxCounter}},
    attach boxed title to top left={xshift=0.5cm},
    boxed title style={colback=xpurple},
    #1,
}

%% Referencing Macros %%

% Exercises
\newcommand{\defex}[1]{\hyperref[ex:#1]{Exercise #1}}

% Definitions
\newcommand{\defword}[2]{\hyperref[def:#2]{#1}}
\newcommand{\defref}[1]{\hyperref[def:#1]{Definition #1}}

% Theorems
\newcommand{\numrefthm}[1]{\hyperref[thm:#1]{Theorem #1}}
\newcommand{\namrefthm}[1]{\hyperref[thm:#1]{#1}}

\newcommand{\dydx}{\dfrac{dy}{dx}}
\newcommand{\dxdy}{\dfrac{dx}{dy}}
\newcommand{\dydt}{\dfrac{dy}{dt}}
\newcommand{\dxdt}{\dfrac{dx}{dt}}
\newcommand{\dzdt}{\dfrac{dz}{dt}}

% 
\newcommand{\barNotationT}[1]{\bigg|_{t = #1}}

\newcommand{\cynit}[1]{\textit{\textcolor{cyan}{#1}}}



%-%-%-%-%-%-%-%-%-%-%-%-%-%-%-%-%-%-%-%-%-%-%-%-%-%-%-%-%-%-%-%-%-%-%-%-%-%-%

%% MATH PACKAGES, ENVIRONMENTS, COMMANDS %%
%-%-%-%-%-%-%-%-%-%-%-%-%-%-%-%-%-%-%-%-%-%-%-%-%-%-%-%-%-%-%-%-%-%-%-%-%-%-%
%You'll need your own packages, theorem types, and commands.

\usepackage{fix-cm}
\usepackage{amsmath,amsthm} 
\usepackage{mathtools}
\usepackage{amssymb}
\usepackage[framemethod=tikz]{mdframed}
\usepackage{mathrsfs}
\usepackage{changepage}
\usepackage{multicol}
\usepackage{slashed}
\usepackage{enumerate}
\usepackage{booktabs}
\usepackage{enumitem}
\usepackage{kantlipsum}  %This package lets us generate random text for example purposes.
\usepackage{pgfplots}


%%% Custom Commands %%%
% Natural Numbers 
\newcommand{\N}{\mathbb{N}}

% Whole Numbers
\newcommand{\W}{\mathbb{W}}

% Integers
\newcommand{\Z}{\mathbb{Z}}

% Rational Numbers
\newcommand{\Q}{\mathbb{Q}}

% Real Numbers
\newcommand{\R}{\mathbb{R}}

% Complex Numbers
\newcommand{\C}{\mathbb{C}}

\newcommand{\I}{\mathbb{I}}

\newcommand{\pfs}{\noindent\makebox[\linewidth]{\rule{\textwidth}{0.4pt}}\vspace{0.5cm}}

% \newcommand{}{\marginnote{\includegraphics[width=2em]{caution.png}}}

\newmdenv[
  topline=false,
  bottomline=true,
  rightline=false,
  leftline=true,
  linewidth=1.5pt,
  linecolor=black, % default color, will be overridden in custom commands
  backgroundcolor=draculabg, % Needed for Dracula theme
  fontcolor=draculafg, % Needed for Dracula theme
  innertopmargin=0pt,
  innerbottommargin=5pt,
  innerrightmargin=10pt,
  innerleftmargin=10pt,
  leftmargin=0pt,
  rightmargin=0pt,
  skipabove=\topsep,
  skipbelow=\topsep,
]{customframedproof}

\newenvironment{proofpart}[2][black]{
    \begin{mdframed}[
        topline=false,
        bottomline=false,
        rightline=false,
        leftline=true,
        linewidth=1pt,
        linecolor=#1!40, % Custom color
        % innertopmargin=10pt,
        % innerbottommargin=10pt,
        innerleftmargin=10pt,
        innerrightmargin=10pt,
        leftmargin=0pt,
        rightmargin=0pt,
        % skipabove=\topsep,
        % skipbelow=\topsep%
    ]
    \noindent
    \begin{minipage}[t]{0.08\textwidth}%
        \textbf{#2}%
    \end{minipage}%
    \begin{minipage}[t]{0.90\textwidth}%
        \begin{adjustwidth}{0pt}{0pt}%
}{
    \end{adjustwidth}
    \end{minipage}
    \end{mdframed}
}

% Define a new environment 'proofscratch' for Proof and Scratch Paper
\newenvironment{proofscratch}[2]{%
    \begin{mdframed}[
        topline=false,
        bottomline=false,
        rightline=false,
        leftline=false,
        backgroundcolor=draculabg, % Background color for Dracula theme
        fontcolor=draculafg,       % Font color for Dracula theme
        innerleftmargin=-6pt,
        innertopmargin=0pt,
        leftmargin=0pt,
        rightmargin=0pt,
        skipabove=\topsep,
        skipbelow=\topsep,
    ]
    % % Begin TikZ picture for the vertical dotted line
    % \begin{tikzpicture}[overlay, remember picture]
    %     % Draw a vertical dotted line at approximately the center of the mdframed width
    %     \draw[dotted, thick] ([xshift=0.005\linewidth]current bounding box.north west) -- ([xshift=0.005\linewidth]current bounding box.south west);
    % \end{tikzpicture}%
    % % Create the left minipage for Proof
    \begin{minipage}[t]{0.47\textwidth}%
        \noindent\textit{Proof.} #1 \hfill \(\qed\)
    \end{minipage}%
    \hfill
    % Create the right minipage for Scratch Paper
    \begin{minipage}[t]{0.47\textwidth}%
        \textit{Scratch Paper.} #2
    \end{minipage}%
}{
    \end{mdframed}
}

\newcommand{\pfscratch}[3]{%
    \begin{customframedproof}[linecolor=#3]
        \begin{proofscratch}
            {#1}{#2}
        \end{proofscratch}
    \end{customframedproof}
}

\newcommand{\lepfscratch}[2]{\pfscratch{#1}{#2}{teal}}

\newcommand{\sbseteqpfbase}[5][We need to show these expressions are subsets of each other in order to prove they are equivalent.]{%
    \begin{customframedproof}[linecolor=#5]
        \begin{proof}
                #1
            \begin{proofpart}[#5]{(\(\subseteq\))}
                #2
            \end{proofpart}
    \vspace{2mm}
            \begin{proofpart}[#5]{(\(\supseteq\))}
                #3
            \end{proofpart}
    \vspace{2mm}
            #4
        \end{proof}
    \end{customframedproof}
}

\newcommand{\sbseteqpf}[5][We need to show these expressions are subsets of each other in order to prove they are equivalent.]{
    \raisebox{0.85em}{\begin{minipage}[t]{0.90\textwidth}
        \sbseteqpfbase[#1]{#2}{#3}{#4}{#5}
    \end{minipage}}
}

% \newcommand{\iffpf}[4]{
%     \begin{customframedproof}[linecolor=#4]
%         \begin{proof}
%             We show this by proving both implications:
%             \begin{proofpart}[#4]{(\(\Rightarrow\))}
%                 #1
%             \end{proofpart}
%     \vspace{2mm}
%             \begin{proofpart}[#4]{(\(\Leftarrow\))}
%                 #2
%             \end{proofpart}
%     \vspace{2mm}
%             #3
%         \end{proof}
%     \end{customframedproof}
% }

\newcommand{\iffpfbase}[5][We show this by proving both implications:]{
    \begin{customframedproof}[linecolor=#5]
        \begin{proof}
            #1 % This is the optional text, with a default value
            \begin{proofpart}[#5]{(\(\Rightarrow\))}
                #2
            \end{proofpart}
    \vspace{2mm}
            \begin{proofpart}[#5]{(\(\Leftarrow\))}
                #3
            \end{proofpart}
    \vspace{2mm}
            #4
        \end{proof}
    \end{customframedproof}
}

\newcommand{\iffpf}[5][We will show this by proving both implications:]{
    \raisebox{0.85em}{\begin{minipage}[t]{0.90\textwidth}
        \iffpfbase[#1]{#2}{#3}{#4}{#5}
    \end{minipage}}
}



% Command for exercise proofs with a predefined color
\newcommand{\expf}[1]{
    \begin{customframedproof}[linecolor=orangehdx!75,]
        \begin{proof}
        #1
        \end{proof}
    \end{customframedproof}
}

\newcommand{\lemmapf}[1]{
    \begin{customframedproof}[linecolor=xpurple!75,]
        \begin{proof}
        #1
        \end{proof}
    \end{customframedproof}
}

% Command for proofs that are contained within solution environments. 
\newcommand{\innerpfbase}[1]{%
    \begin{customframedproof}[linecolor=xgray]
        \begin{proof}
            #1
        \end{proof}
    \end{customframedproof}
}

\newcommand{\innerpf}[1]{
    \raisebox{0.85em}{
    \begin{minipage}[t]{0.90\textwidth}
        \innerpfbase{#1}
    \end{minipage}}
}


% Command for theorems proofs with a predefined color
\newcommand{\tpf}[1]{
    \begin{customframedproof}[linecolor=xblue]
        \begin{proof}
        #1
        \end{proof}
    \end{customframedproof}
}
% Command for corollaries proofs with a predefined color
\newcommand{\cpf}[1]{
    \begin{customframedproof}[linecolor=purplehdx]
        \begin{proof}
        #1
        \end{proof}
    \end{customframedproof}
}

\newcommand{\sol}[1]{
    \begin{customframedproof}[linecolor=orangehdx!75,]
        \begin{solution}
        #1
        \end{solution}
    \end{customframedproof}
}

\newcommand{\lesol}[1]{
    \begin{customframedproof}[linecolor=teal]
        \begin{solution}
        #1
        \end{solution}
    \end{customframedproof}
}

\newcommand{\lepf}[1]{
    \begin{customframedproof}[linecolor=teal]
        \begin{proof}
        #1
        \end{proof}
    \end{customframedproof}
}

\newcommand{\pb}[2]{
    \begin{exercise}{#1}[label=ex:#1]
    #2
    \end{exercise}
}

\def \proofDistance {10pt}
\newenvironment{solution}
  {\textit{Solution.}}

% Command for theorems proofs with a predefined color
\newcommand{\epf}[1]{
    \begin{customframedproof}[linecolor=teal]
        \begin{proof}
        #1
        \end{proof}
    \end{customframedproof}
}

\newcommand{\disabs}[1]{\ensuremath{\left|#1\right|}}
\newcommand{\limn}{\lim_{n \rightarrow \infty}}
\newcommand{\limx}[2]{\lim_{x \rightarrow #1}#2}

\newcommand{\limc}[1]{\lim_{x \rightarrow c}#1}

\newcommand{\ninn}{n \in \N}


\newcommand{\limsupn}[1]{\limsup_{n\rightarrow \infty}#1}
\newcommand{\liminfn}[1]{\liminf_{n\rightarrow \infty}#1}

\renewcommand{\theenumi}{\alph{enumi}} 
\renewcommand{\labelenumi}{(\theenumi)}


% end of preamble
%-%-%-%-%-%-%-%-%-%-%-%-%-%-%-%-%-%-%-%-%-%-%-%-%-%-%-%-%-%-%-%-%-%-%-%-%-%-%

\begin{document}

\numberwithin{BoxCounter}{section}


%-%-%-%-%-%-%-%-%-%-%-%-%-%-%-%-%-%-%-%-%-%-%-%-%-%-%-%-%-%-%-%-%-%-%-%-%-%-%
%%% COVER PAGE %%%
%-%-%-%-%-%-%-%-%-%-%-%-%-%-%-%-%-%-%-%-%-%-%-%-%-%-%-%-%-%-%-%-%-%-%-%-%-%-%

% Do not use all caps.
\newcommand{\titlestandin}[0]{Multivariable Calculus Notes}
\newcommand{\cussubtitle}[0]{MATH 230}
% Date format should be like: "January 1, 2001"
\newcommand{\startdate}[0]{January 22, 2025}
\newcommand{\customenddate}[0]{May 13, 2025}
% Be sure to include degree recognition (e.g., B.S., M.S., Ph.D)
\newcommand{\professor}[0]{Prof. Lars Seme, M.S.}

%-%-%-%-%-%-%-%-%-%-%-%-%-%-%-%-%-%-%-%-%-%-%-%-%-%-%-%-%



\begin{titlepage}
    \begin{center}

        \vspace*{-2cm}
        % \includegraphics[width=0.8\textwidth]{images/Hendrix Logo.png}\\
        \includegraphics[width=0.8\textwidth]{images/logo_white_text.png}\\ % Use for Dracula. Its the all white version of the logo.
        \vfill

        % Horizontal line above the title in 'horange' color
        \textcolor{horange}{\rule{\textwidth}{1.0pt}}

        \vspace{2em}

        {\huge \textbf{\titlestandin}}

        \vspace{1em} % Space between the title and the bottom line

        \textcolor{horange}{\rule{\textwidth}{1.0pt}}

        \vspace*{1\baselineskip}

        {\LARGE \textbf{\cussubtitle}}

        \begin{large}
            \vspace*{2\baselineskip}

            \textit{Start}  \\[1ex]
            {\scshape \startdate} \\[0.3\baselineskip] % Year published

            \vspace*{1\baselineskip}

            \emph{Author} \\[1ex]
            %Submitted by \\[\baselineskip]
            {\Large Paul Beggs \\ \par} % Editor list
            {\href{mailto:BeggsPA@Hendrix.edu}{{BeggsPA@Hendrix.edu}}}\\ % Editor affiliation

            \vspace*{1\baselineskip}

            \textit{Instructor} \\[1ex] % Tagline(s) or further description
            \professor

            \vspace*{1\baselineskip}

            \textit{End}\\[1ex]
            {\scshape  \customenddate} \\[0.3\baselineskip] % Year published

            \thispagestyle{empty}

        \end{large}
    \end{center}
\end{titlepage}
\pagebreak


%-%-%-%-%-%-%-%-%-%-%-%-%-%-%-%-%-%-%-%-%-%-%-%-%-%-%-%-%-%-%-%-%-%-%-%-%-%-%

%%% Table of Contents %%%


% -%-%-%-%-%-%-%-%-%-%-%-%-%-%-%-%-%-%-%-%-%-%-%-%-%-%-%-%-%-%-%-%-%-%-%-%-%-%

% Helper command to fix header for double paged ToC.
% Temporarily adjust section marks for Table of Contents
\begin{spacing}{1} % Can change spacing between entries in TOC
    \renewcommand{\contentsname}{\Large\textbf{Table of Contents}} % Can rename here
    \markboth{}{} % Clear the header marks for TOC
    \pagestyle{fancy} % Ensure fancy style is active
    \fancyhead[R]{Table of Contents} % Set TOC-specific header manually
    \tableofcontents
    \addtocontents{toc}{\protect\enlargethispage{\baselineskip}}
\end{spacing}

% Reset the headers to default for the rest of the document
\fancyhead[R]{\leftmark}




% -%-%-%-%-%-%-%-%-%-%-%-%-%-%-%-%-%-%-%-%-%-%-%-%-%-%-%-%-%-%-%-%-%-%-%-%-%-%

\chapter{Parametric Eqs and Polar Coords}\label{para eqs and polar coords}
\vspace*{-0.25in}
\section{Parametric Equations}

\subsection{Introduction}
Most of your calculus experience has been single variable, so that the functions under consideration were typically \(f : \mathbb{R} \to \mathbb{R}\). Our course is divided into roughly 3 sections:
\begin{itemize}
    \item Parametric Equations/Functions: Functions of the form \(f : \mathbb{R} \to \mathbb{R}^n\) (Chapters 1 - 3)
    \item Scalar Functions: Functions of the form \(f : \mathbb{R}^n \to \mathbb{R}\) (Chapters 4 - 5)
    \item Vector Fields: Functions of the form \(f : \mathbb{R}^n \to \mathbb{R}^n\) (Chapter 6)
\end{itemize}

\subsection{Parametric Equations}

A \cynit{parametric equation} (or, \cynit{sometimes parametric function} or \cynit{ vector-valued function}) is a function of the form \(f \colon \mathbb{R} \to \mathbb{R}^{n}\). We will typically consider \(n = 2\) or \(n = 3\) and call the input variable the parameter, usually denoted by \(t\). We write them as

\[
    f(t) =
    \begin{cases}
        x(t) \\
        y(t)
    \end{cases}
    \quad \text{or} \quad
    f(t) =
    \begin{cases}
        x(t) \\
        y(t) \\
        z(t)
    \end{cases}.
\]

A \cynit{parametric curve} is the set of points \((x(t), y(t))\) in \(\R^{2}\) or \((x(t), y(t), z(t))\) in \(\R^{3}\) traced out. Note that in general, the curve may not be a function for \(y\) in terms of \(x\), but is a function of the parameter \(t\).

\subsection{Graphing Parametric Curves in the Second Dimension}

\subsubsection{Elimination of the Parameter}

In some cases, we can explicitly solve for \(t\) in terms of one of \(x\) or \(y\). When this is possible, you can write \(y(x)\) or \(x(y)\) and use your “regular” algebraic knowledge. We call this process \cynit{eliminating the parameter}.

\subsubsection{Using Technology}

\begin{itemize}
    \item Your TI-84 can graph this if you switch to \texttt{par} mode.
    \item Likewise, GeoGebra can do this, using the \texttt{curve} function.
          \begin{itemize}
              \item In general, the syntax is: \texttt{curve(x(t), y(t), t, min, max)}
          \end{itemize}
\end{itemize}

\subsection{The Cycloid}
A wheel of radius \(a\) is rolling along a flat road at a constant velocity. The curve generated by a point along the edge of the wheel traces out a shape called a \cynit{cycloid}. Let \(t\) represent the angle - in radians!!!! - rotated through, and that the point of interest starts at the origin. Before we find the equations for the point, let's find the location of the center of the circle:
\[
    f_{\text{center}}(t) = \begin{cases}
        x(t) = a t \\
        y(t) = a
    \end{cases}
\]

Then, relative to the center, our point along the edge has equations
\[
    f(t) = \begin{cases}
        x(t) = -a \sin(t) \\
        y(t) = -a \cos(t)
    \end{cases}
\]

Thus, our point has parametric equations
\[
    f(t) = \begin{cases}
        x(t) = a(t - \sin(t)) \\
        y(t) = a(1 - \cos(t))
    \end{cases}
\]

\subsection{Final Notes}

Next time, we'll start asking Calculus-y questions: What are the velocities in the \(x\), \(y\), and total directions? What total distance does it travel? What is the area of the region under one period of the cycloid?
\begin{itemize}
    \item The syllabus has a number of practice problems to work on. These are not required, and not to be
          turned in, but are for you to work before class next time.
    \item We will talk about them at the start of the next class. You should try them beforehand.
    \item The most common reason for a lack of success in this class is not spending time working problems on
          your own.
\end{itemize}

\section{Calculus of Parametric Curves}

For this section, we will have a parametric curve in R2, defined by \(f(t) = \begin{cases}
    x(t) \\
    y(t)
\end{cases}.\) 
In many cases, the curve does not describe \(y\) as a function of \(x\). However, we can still carry over many ideas from single variable calculus.

\subsection{Slope for a Parametric Curve}

Given a point \(t_{0}\), the \cynit{slope of the curve} in the \(xy\)-plane is given by
\[
    \dydx\barNotationT{t_{0}} = \frac{dy/dt}{dx/dt}\barNotationT{t_{0}}. \\
\]
Note that this is undefined when \(x'(t_{0}) = 0\). \\

\noindent The \cynit{tangent line} at \(t_{0}\) is given by
\[
    y = \left(\dydx\barNotationT{t_{0}}\right)(x - x(t_{0})) + y(t_{0}).
\]

\subsection{Second Derivative}

The value of the second derivative for the curve at \(t_{0}\) is given by
\[
    \dfrac{d^{2}y}{dx^{2}}\barNotationT{t_{0}} = \frac{d}{dt}\left(\dydx\right)\barNotationT{t_{0}} = \frac{d}{dt}\left(\frac{dy/dt}{dx/dt}\right)\barNotationT{t_{0}}. \\
\]
Note the benefit of Leibnitz notation for each of these two derivatives!

\subsection{Area Under a Curve}

Suppose that a parametric curve is non-self intersecting. Then, the signed area of the region between the curve and the \(x\)-axis on the \(t\) interval \([t_{a},t_{b}]\) is given by 

\[
    A = \int_{t_{a}}^{t_{b}} y(t) \frac{dx}{dt} \, dt = \int_{t_{a}}^{t_{b}} y(t) \frac{dx}{dt} \, dt.
\]

\subsection{Arc Length}

The \cynit{arc length} of a parametric curve over the \(t\) interval \([t_{a},t_{b}]\) is given by

\[
    s = \int_{t_{a}}^{t_{b}} \sqrt{\left(\frac{dx}{dt}\right)^{2} + \left(\frac{dy}{dt}\right)^{2}} \, dt.
\]

\subsection{Surface Area}

The \cynit{surface area} of the region obtained by rotating a non-self intersecting parametric curve is given by

\[
    S = \int_{t_{a}}^{t_{b}} 2\pi y(t) \sqrt{\left(\frac{dx}{dt}\right)^{2} + \left(\frac{dy}{dt}\right)^{2}} \, dt.
\]

\subsection{The Cycloid}

We can apply each of the above to the cycloid:
\begin{itemize}
    \item \cynit{Derivative}: \(\dydx = \frac{dy}{dx} = \frac{\sin(t)}{1 - \cos(t)}\). Note that the slope is then independent of the radius of the wheel and
    that the slope is undefined at each of \(t = \dots, -4\pi, -2\pi, 0, 2\pi, 4\pi, \dots\). 
    \item \cynit{Concavity}: \(\dfrac{d^{2}y}{dx^{2}} = \frac{d}{dt}\left(\frac{dy}{dx}\right) = \frac{d}{dt}\left(\frac{\sin(t)}{1 - \cos(t)}\right)\). After some work, we find that \(\frac{d^{2}y}{dx^{2}} = -\frac{a}{y^{2}}\), which shows that the cycloid is always concave down.
    \item \cynit{Area}: The area of one period of the cycloid \(A = 3\pi a2\), after some work.

    \begin{align*}
        A &= \int_{0}^{2\pi} (a - a \cos t)(a - a\cos t) dt \\
        &= a^{2} \int_{0}^{2\pi} (1 - 2\cos t + \cos^{2} t) dt \\
        &= a^{2} \left(2\pi + \int_{0}^{2\pi} 1 - 2\cos t + \cos^{2} t \right) dt \\
        &= a^{2}\left(2\pi + \left(\frac{t}{2} + \frac{1}{4}\sin(2t)\right)\right)\bigg|_{0}^{2\pi} \\
        &= 3\pi a^{2}.
    \end{align*}

    \item \cynit{Arc Length}: The arc length of one period of the cycloid is \(s = 8a\), again after some work.

    \begin{align*}
        S &= \int_{0}^{2\pi} \sqrt{(a- a \cos t)^{2} + (a \sin t)^{2}} \ dt \\
        &= a \int_{0}^{2\pi} \sqrt{1 - 2\cos t + \cos^{2} t + \sin^{2} t} \ dt \\
        &= a \int_{0}^{2\pi} \sqrt{2 - 2\cos t} \ dt \\
        &= \sqrt{2}a \int_{0}^{2\pi} \sqrt{1 - cost} \ dt \\
        &= \sqrt{2}a \int_{0}^{2\pi} \sqrt{2\sin^{2}\left(\frac{t}{2}\right)} \ dt \\
        &= \sqrt{2}a \cdot \sqrt{2} \int_{0}^{2\pi} \sin\left(\frac{t}{2}\right) \ dt \\
        &= 2a\left(-2 \cos\left(\frac{t}{2}\right)\right)\bigg|_{0}^{2\pi} \\ 
        &= 8a.
    \end{align*}

    \item \cynit{Surface Area}: The surface area of the solid obtained by rotating one period of the cycloid around the \(x\)-axis is \(S = \frac{64\pi a^{2}}{3}\), after a lot of tedious work.
\end{itemize}

\setcounter{chapter}{1}


%-%-%-%-%-%-%-%-%-%-%-%-%-%-%-%-%-%-%-%-%-%-%-%-%-%-%-%-%-%-%-%-%-%-%-%-%-%-%
\end{document}
%-%-%-%-%-%-%-%-%-%-%-%-%-%-%-%-%-%-%-%-%-%-%-%-%-%-%-%-%-%-%-%-%-%-%-%-%-%-%

%-%-%-%-%-%-%-%-%-%-%-%-%-%-%-%-%-%-%-%-%-%-%-%-%-%-%-%-%-%-%-%-%-%-%-%-%-%-%