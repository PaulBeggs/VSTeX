\documentclass[11pt]{article}

\usepackage{fix-cm}
\usepackage{tikz}
\usepackage{forest}
\usepackage{amsmath,amsthm} 
\usepackage{amssymb}
\usepackage[framemethod=tikz]{mdframed}
% \usepackage{draculatheme}
\usepackage{mathrsfs}
\usepackage{changepage}
\usepackage{multicol}
\usepackage{mathtools}
\usepackage{hyperref}
\usepackage{slashed}
\usepackage{enumerate}
\usepackage{booktabs}
\usepackage{enumitem}
\usepackage{kantlipsum} 
\usepackage{tabularx}
\usepackage{array}
\usepackage{makecell}
\usepackage{pgfplots}
\pgfplotsset{compat=1.18}
\usetikzlibrary{
	decorations.markings,
	trees,
	arrows,
	automata,
	shapes,
	backgrounds,
	calc,
	patterns
}

\setlength{\parindent}{0pt}

\newmdenv[
	topline=false,
	bottomline=true,
	rightline=false,
	leftline=true,
	linewidth=1.5pt,
	linecolor=black, % default color, will be overridden in custom commands
	% backgroundcolor=draculabg, % Needed for Dracula theme
	% fontcolor=draculafg, % Needed for Dracula theme
	innertopmargin=0pt,
	innerbottommargin=5pt,
	innerrightmargin=10pt,
	innerleftmargin=10pt,
	leftmargin=0pt,
	rightmargin=0pt,
	skipabove=\topsep,
	skipbelow=\topsep,
]{customframedproof}

\newenvironment{proofpart}[2][black]{
    \begin{mdframed}[
        topline=false,
        bottomline=false,
        rightline=false,
        leftline=true,
        linewidth=1pt,
        linecolor=#1!40, % Custom color
        innertopmargin=10pt,
        innerbottommargin=10pt,
        innerleftmargin=10pt,
        innerrightmargin=10pt,
        leftmargin=0pt,
        rightmargin=0pt,
        % skipabove=\topsep,
        % skipbelow=\topsep%
    ]
    \noindent
    \begin{minipage}[t]{0.08\textwidth}%
        \textbf{#2}%
    \end{minipage}%
    \begin{minipage}[t]{0.90\textwidth}%
        \begin{adjustwidth}{0pt}{0pt}%
}{
    \end{adjustwidth}
    \end{minipage}
    \end{mdframed}
}

\newenvironment{solution}
  {\textit{Solution.}}



%%% AESTHETICS %%%
%-%-%-%-%-%-%-%-%-%-%-%-%-%-%-%-%-%-%-%-%-%-%-%-%-%-%-%-%-%-%-%-%-%-%-%-%-%-%


%%% Dimensions and Spacing %%%
\usepackage[left=0.5in,right=0.5in,top=1in,bottom=1in]{geometry}
% \usepackage{setspace}
% \linespread{1}
\usepackage{listings}

%%% Define new colors %%%
\usepackage{xcolor}
\definecolor{orangehdx}{rgb}{0.96, 0.51, 0.16}

% Normal colors
\definecolor{xred}{HTML}{BD4242}
\definecolor{xblue}{HTML}{4268BD}
\definecolor{xgreen}{HTML}{52B256}
\definecolor{xpurple}{HTML}{7F52B2}
\definecolor{xorange}{HTML}{FD9337}
\definecolor{xdotted}{HTML}{999999}
\definecolor{xgray}{HTML}{777777}
\definecolor{xcyan}{HTML}{80F5DC}
\definecolor{xpink}{HTML}{F690EA}
\definecolor{xgrayblue}{HTML}{49B095}
\definecolor{xgraycyan}{HTML}{5AA1B9}

% Dark colors
\colorlet{xdarkred}{red!85!black}
\colorlet{xdarkblue}{xblue!85!black}
\colorlet{xdarkgreen}{xgreen!85!black}
\colorlet{xdarkpurple}{xpurple!85!black}
\colorlet{xdarkorange}{xorange!85!black}
\definecolor{xdarkcyan}{HTML}{008B8B}
\colorlet{xdarkgray}{xgray!85!black}

% Very dark colors
\colorlet{xverydarkblue}{xblue!50!black}

% Document-specific colors
\colorlet{normaltextcolor}{black}
\colorlet{figtextcolor}{xblue}

% Enumerated colors
\colorlet{xcol0}{black}
\colorlet{xcol1}{xred}
\colorlet{xcol2}{xblue}
\colorlet{xcol3}{xgreen}
\colorlet{xcol4}{xpurple}
\colorlet{xcol5}{xorange}
\colorlet{xcol6}{xcyan}
\colorlet{xcol7}{xpink!75!black}

% Blue-Purple (should just used colorbrewer...)
\definecolor{xrainbow0}{HTML}{e41a1c}
\definecolor{xrainbow1}{HTML}{a24057}
\definecolor{xrainbow2}{HTML}{606692}
\definecolor{xrainbow3}{HTML}{3a85a8}
\definecolor{xrainbow4}{HTML}{42977e}
\definecolor{xrainbow5}{HTML}{4aaa54}
\definecolor{xrainbow6}{HTML}{629363}
\definecolor{xrainbow7}{HTML}{7e6e85}
\definecolor{xrainbow8}{HTML}{9c509b}
\definecolor{xrainbow9}{HTML}{c4625d}
\definecolor{xrainbow10}{HTML}{eb751f}
\definecolor{xrainbow11}{HTML}{ff9709}

%%% FIGURES %%%
\usepackage{graphicx}  
% \graphicspath{ {images/} }  
% \numberwithin{figure}{section}
\usepackage{float}
\usepackage{caption}

%%% Hyperlinks %%%
\usepackage{hyperref}
\definecolor{horange}{HTML}{f58026}
\hypersetup{
	colorlinks=true,
	linkcolor=horange,
	filecolor=horange,      
	urlcolor=horange,
}

\newcommand{\mysqrt}[1]{%
  \mathpalette\foo{#1}%
}
\newcommand{\dmysqrt}[1]{%
  \mathpalette\foodisplay{#1}%
}

\newcommand{\sol}[1]{
    \begin{customframedproof}[linecolor=orangehdx!75,]
        \begin{solution}
        #1
        \end{solution}
    \end{customframedproof}
}

% !TeX spellcheck = off
\newcommand{\foo}[2]{%
  % #1: math style, #2: content
  \sbox0{\(#1\sqrt{#2}\)}% Measure the size of the standard sqrt in the current style
  \begin{tikzpicture}[baseline=(sqrt.base)]
    \node[inner sep=0, outer sep=0] (sqrt) {\(#1\sqrt{#2}\)}; % Use the current math style
    \draw([yshift=-0.045em]sqrt.north east) -- ++(0,-0.5ex); % Draw the tick
  \end{tikzpicture}%
}
% !TeX spellcheck = off
\newcommand{\foodisplay}[2]{%
    % #1: math style, #2: content
    \sbox0{\(#1\sqrt{#2}\)}% Measure the size of the standard sqrt in the current style
    \begin{tikzpicture}[baseline=(sqrt.base)]
		\node[inner sep=0, outer sep=0] (sqrt) {\(\displaystyle\sqrt{#2}\)}; % Force displaystyle
    	\draw[line width=0.4pt] ([yshift=-0.044em]sqrt.north east) -- ++(0,-0.5ex); % Draw the tick
	\end{tikzpicture}%
}

\newcommand{\barNotationT}[1]{\bigg|_{t = #1}}

\newcommand{\cyanit}[1]{\textit{\textcolor{cyan}{#1}}}

\newcommand{\brackett}[1]{\left\langle #1 \right\rangle}

\newcommand{\norm}[1]{\left\lVert \mathbf{#1}\right\rVert}

\newcommand{\imb}{\mb{i}}
\newcommand{\jmb}{\mb{j}}
\newcommand{\kmb}{\mb{k}}
\newcommand{\rmb}{\mb{r}}
\newcommand{\umb}{\mb{u}}

\newcommand{\vecfuc}[2]{\mb{#1}(#2)}
\newcommand{\dvecfuc}[2]{\mb{#1}'(#2)}
\newcommand{\normdvecfuc}[2]{||\mb{#1}'(#2)||}

\newcommand{\proj}{\text{proj}}

\newcommand{\mb}[1]{\mathbf{#1}}

% \renewcommand{\theenumi}{\arabic{enumi}} 
% \renewcommand{\labelenumi}{\theenumi.}

\title{Probability and Statistics: Practice Set 2}
\author{Paul Beggs}
\date{\today}

%%% Custom Comands %%%
% Natural Numbers 
\newcommand{\N}{\ensuremath{\mathbb{N}}}

% Whole Numbers
\newcommand{\W}{\ensuremath{\mathbb{W}}}

% Integers
\newcommand{\Z}{\ensuremath{\mathbb{Z}}}

% Rational Numbers
\newcommand{\Q}{\ensuremath{\mathbb{Q}}}

% Real Numbers
\newcommand{\R}{\ensuremath{\mathbb{R}}}

% Complex Numbers
\newcommand{\C}{\ensuremath{\mathbb{C}}}

\newcommand{\I}{\ensuremath{\mathbb{I}}}

\newcommand{\p}{\partial}

\newcommand{\mbi}{\mathbf{i}}
\newcommand{\mbj}{\mathbf{j}}
\newcommand{\mbk}{\mathbf{k}}
\newcommand{\mbr}{\mathbf{r}}

\begin{document}

\maketitle

\begin{enumerate}
    \item (2 points each) A discrete random variable \(X\) has its pmf as given in the table below: \\[0.5em]
    \begin{tabular}{c|l}
        \(x\) & \(P(X = x)\) \\
        \hline
        1 & 0.2 \\
        3 & 0.1 \\
        4 & 0.4 \\
        7 & 0.3 \\
    \end{tabular}
    \begin{enumerate}
        \item Find (as a piecewise-defined function) the CDF for \(X\).
        \sol{
            The CDF for \(X\) can be defined as 
			\[
				F(x) = \begin{cases}
					0, & \text{if } x < 1, \\
					0.2, & \text{if } 1 \leq x < 3, \\
					0.3, & \text{if } 3 \leq x < 4, \\
					0.7, & \text{if } 4 \leq x < 7, \\
					1, & \text{if } x \geq 7.
				\end{cases}
			\]
        }
        \item Find the mean \(\mu = E(X)\).
        \sol{
			The mean can be calculated with the following:
			\[
				\mu = \sum_{x \in \{1,3,4,7\}} x f(x) = 1(0.2) + 3(0.1) + 4(0.4) + 7(0.3) = 4.2.
			\]
        }

        \item Find the variance, \(\sigma^2 = \text{Var}(X)\).
        \sol{
			Similarly to the previous problem:
			\[
				\sigma^{2} = \sum_{x \in \{1,3,4,7\}} (x - 4.2)^{2} f(x) = 4.56.
			\]
        }
        \item Find the Moment Generating Function, \(M(t) = E(e^{tX})\) for \(X\).
        \sol{
			The mgf can be found by computing the following:
			\[
				M(t) = \sum_{x \in \{1,3,4,7\}} e^{tx}f(x) = 0.2e^{t1} + 0.1e^{t3} + 0.4e^{t4} + 0.3e^{t7}.
			\]
        }

\newpage

    \end{enumerate}
    \item (2 points each) Suppose that \(Y = aX + b\) and let \(M_{Y}(t)\) and \(M_{X}(t)\) be the moment generating functions for \(Y\) and \(X\) respectively. 
    \begin{enumerate}
        \item Show that \(M_{Y}(t) = e^{tb}M_{X}(at)\).
        \sol{
            We will expand the moment generating function for \(Y\):
			\[
				M_{Y}(t) = E\left( e^{tY} \right) = E\left( e^{t(aX + b)} \right) = E\left( e^{taX}e^{tb} \right) = E\left( e^{tb} \cdot e^{taX} \right) = e^{tb} \cdot E\left( e^{taX} \right) = e^{tb} M_{X}(at).
			\]
            Therefore, \(M_{Y}(t) = e^{tb}M_{X}(at)\).
        }
        \item Use the previous result to show that \(E(Y) = aE(X) + b\).
        \sol{
			To use the previous result, we can find the first derivative of \(M_{Y}(t)\). Then, we can evaluate it at 0 since \(M_{Y}'(0) = E(Y)\). Hence, we will find the derivative first:
			\begin{align*}
				M_{Y}'(t) &= \frac{d}{dt}\left[ e^{tb}M_{X}(at) \right] \\
				&= \frac{d}{dt} \left[ e^{tb} \right] M_{X}(at) + e^{tb} \cdot \frac{d}{dt} \left[ M_{X}(at) \right] \\
				&= be^{tb} M_{X}(at) + ae^{tb}M_{X}'(at) . \tag{1}\label{eq:1}
				\intertext{Evaluating at \(t = 0\):}
				E(Y) = M_{Y}'(0) &= be^{0b} M_{X}(a0) + ae^{0b} M_{X}'(a0) \\
				&= bM_{X}(0) + a M_{X}'(0) \\
				&= b \cdot (1) + aE(X) \\
				&= aE(X) + b.
			\end{align*}
			Therefore, \(E(Y) = aE(X) + b\).
        }

        \item Use the previous two results to show that \(\text{Var}(Y) = a^2\text{Var}(X)\).
        % \sol{
		% 	Similarly to the previous problem, we know that \(M_{Y}''(0) = \text{Var}(Y)\), so we can take the derivative of (\ref{eq:1}), and evaluate it at 0. Again, we will find the derivative first:
		% 	\begin{align*}
		% 		M_{Y}''(t) &= \frac{d}{dt} \left[ {be^{tb} M_{X}(at)} \right] + \frac{d}{dt} \left[ {ae^{tb}M_{X}'(at)} \right] \\
		% 		&= b^{2}e^{tb}M_{X}(at) + abe^{tb} M_{X}'(at)  + abe^{tb} M_{X}'(at) + a^{2}e^{tb} M_{X}''(at) \\
		% 		\intertext{Evaluating at \(t = 0\):}
		% 		\text{Var} (Y) = \frac{d^{2}}{d^{2}t} \left[ {M_{Y}(0)} \right] &= \left( b^{2}e^{0b}M_{X}(a0) \right) + \left( abe^{0b}M_{X}'(a0) \right) + \left( abe^{0b}M'_{X}(a0) \right) + \left( a^{2}e^{0b}M''_{X}(a0) \right) \\
		% 		&= \left( b^{2} \cdot 1 \cdot M_{X}(0) \right) + (ab \cdot 1 \cdot E(X)) + (ab \cdot 1 \cdot E(X)) + (a^{2} \cdot 1 \cdot \text{Var}(X)) 
		% 	\end{align*}
        % }
		\sol{
            We know that \(\text{Var}(Y) = E(Y^2) - [E(Y)]^2\). From part (b), we have \(E(Y) = aE(X) + b\). We need to find \(E(Y^2)\) using \(M_Y''(0) = E(Y^2)\).
            
            Taking the second derivative of equation (\ref{eq:1}):
            \begin{align*}
                M_Y''(t) &= \frac{d}{dt}\left[be^{tb}M_X(at) + ae^{tb}M_X'(at)\right] \\
                &= b^2e^{tb}M_X(at) + abe^{tb}M_X'(at) + abe^{tb}M_X'(at) + a^2e^{tb}M_X''(at) \\
                &= b^2e^{tb}M_X(at) + 2abe^{tb}M_X'(at) + a^2e^{tb}M_X''(at).
            \end{align*}
            
            Evaluating at \(t = 0\):
            \begin{align*}
                E(Y^2) = M_Y''(0) &= b^2M_X(0) + 2abM_X'(0) + a^2M_X''(0) \\
                &= b^2 \cdot 1 + 2ab \cdot E(X) + a^2 \cdot E(X^2) \\
                &= b^2 + 2abE(X) + a^2E(X^2).
            \end{align*}

			\begin{center}
				\textsc{(Solution continued on the next page)}
			\end{center}
            Therefore:
            \begin{align*}
                \text{Var}(Y) &= E(Y^2) - [E(Y)]^2 \\
                &= [b^2 + 2abE(X) + a^2E(X^2)] - [aE(X) + b]^2 \\
                &= b^2 + 2abE(X) + a^2E(X^2) - [a^2E(X)^2 + 2abE(X) + b^2] \\
                &= a^2E(X^2) - a^2E(X)^2 \\
                &= a^2[E(X^2) - E(X)^2] \\
                &= a^2\text{Var}(X).
            \end{align*}
        }

    \end{enumerate}

    \item (2 points each) A basketball player makes \(82\%\) of their free-throws. Suppose they take 12 total shots and let \(X\) be the number of shots they make. Assume each shot is independent of the others.
    \begin{enumerate}
        \item Find \(P(X = 7)\).
        \sol{
			If \(X\) is the number of shots that they make, then \(X \sim \texttt{Binomialpdf}(12,.82,7)\). Thus, 
			\[
				P(X = 7) = f(7) = \binom{12}{7}(.82)^{7}(.18)^{5} \approx  3.73\%.
			\]
        }
        \item Find \(P(4 \leq X \leq 10)\).
        \sol{
			This question can be rearranged and solved like the following:
			\[
				P(4 \leq X \leq 10) = P(X \leq 10) - P(X \leq 3) \approx  66.41\%.
			\]	
        }
        \item Find \(P(X \geq 7 \mid 4 \leq X \leq 10)\).
        \sol{
			This problem relies upon a conditional probability:
			\[
				P(A \mid B) = \frac{P(A \cap B)}{P(B)},
			\]
			where event \(A\) is the probability of the basketball player making more than 7 shots, given event \(B\) where the total successful shots is between 4 and 10 inclusively. Thus, we need to find the following values with the \(\texttt{Binomialcdf}\) function:
			\begin{itemize}
				\item \(P(A \cap B) = P(7 \leq X \leq 10) = P(X \leq 10) - P(X \leq 6) \approx 0.6525\), and
				\item \(P(B) = P(4 \leq X \leq 10) \approx 0.6641\) (from (b)).
			\end{itemize}
			This allows us to solve the conditional probability:
			\[	
				P(A \mid B) = \frac{P(A \cap B)}{P(B)} \approx \frac{0.6525}{0.6641} \approx 98.25\%.
			\]
        }
    \end{enumerate}

	\newpage

    \item (2 points) A multiple choice exam has 4 choices for each question and 10 total questions. Find the probability that a student guessing at random scores at least a \(50\%\) on the exam.
    \sol{
		For this problem, we will be using a binomial distribution. We let \(X\) count the number of successes, and set \(p = .25\) to be the probability of getting a question correct. Thus, our goal is to find the following:
		\[
			P(X \geq 5) = \sum_{k=5}^{10}\binom{10}{k}(.25)^{k}(.75)^{10 - k} \approx 7.81\%.
		\]
		(Or, we could have equivalently calculated \(1 - \texttt{binomcdf(10,.25,4)}\). I just wanted to experiment with this summation route on my calculator.)
    }
    \item (2 points) Prof.\ Seme has a shelf with 25 books. Of these, 8 are Agatha Christie Hercule Poirot novels. If he selects 7, without replacement, what is the probability that exactly 3 are Poirot novels?
    \sol{
		This question requires the use of the hypergeometric distribution. We'll let \(N\) be the population of 25 books, \(N_{1} = 8\) being the Agatha Christie Hercule Poirot novels, and \(N_{2} = 17\) be the rest of them. We will let \(X\) be the number of Poirot novels picked. Hence, our goal is to calculate the following:
		\[
			P(X = 3) = f(3) = \left[ {\binom{8}{3}\binom{17}{4}} \right] \bigg/ \binom{25}{7} \approx 27.73\%.
		\]
    }

\newpage

    \item (3 points) An insurance company offers a product which will payout to a business that needs to close for snow. They will pay nothing for the first snow storm. For each additional storm which causes a closure, they will pay \$10,000, up to a maximum total payment of \$45,000 in a year. The number of snow storms in a given year follows a geometric distribution, with mean \(\mu = 2.5\) storms per year. What should their premium (i.e., what should they charge per year) so that they bring in \(110\%\) of their expected payout?
    \sol{
        Let $X$ count the number of snow storms that occur in one year. The insurance company's payout, $Y$, is defined by the piecewise function:
        \[
            Y(X) = \begin{cases}
                0, & X \leq 1, \\
                10,000 \cdot (X - 1), & 2 \leq X \leq 5, \\
                45,000, & X \geq 6.
            \end{cases}
        \]
        The formula for the mean of the geometric distribution depends on its definition. Since the random variable $X$ represents the number of storms in a year, it can take the value of zero. This requires the version of the geometric distribution that counts the number of \textbf{failures before the first success}, which has a support of $S = \{0, 1, 2, \ldots\}$. The mean for this definition is $\mu = \frac{1-p}{p}$.
        \[
            \mu = \frac{1 - p}{p} \quad \implies \quad 2.5 = \frac{1 - p}{p} \quad \implies \quad 2.5p = 1 - p \quad \implies \quad p = \frac{2}{7}.
        \]
        The expected payout is the sum of each possible payout multiplied by its probability: 
        \[
            E[Y] = \sum_{x = 2}^{5} Y(x) P(X = x) + Y(6) P(X \geq 6).
        \]
        We can compute this expected value using a table:
        \begin{center}
            \begin{tabular}[htbp]{llll}
                \toprule
                Storms ($x$) & Payout ($Y$) & Probability \(P(X = x) = \left( \frac{5}{7} \right)^{x}(\frac{2}{7})\) & Payout $\cdot$ Probability \\ \midrule
                0 or 1 & \$0 & - & \$0 \\ \midrule
                2 & \$10,000 & $(\frac{5}{7})^2(\frac{2}{7}) \approx 0.1458$ & \$1,457.73 \\ \midrule
                3 & \$20,000 & $(\frac{5}{7})^3(\frac{2}{7}) \approx 0.1041$ & \$2,082.47 \\ \midrule
                4 & \$30,000 & $(\frac{5}{7})^4(\frac{2}{7}) \approx 0.0744$ & \$2,231.21 \\ \midrule
                5 & \$40,000 & $(\frac{5}{7})^5(\frac{2}{7}) \approx 0.0531$ & \$2,124.99 \\ \midrule
                $\geq 6$ & \$45,000 & $P(X \geq 6) = (\frac{5}{7})^6 \approx 0.1328$ & \$5,976.48 \\ \midrule
                \textbf{Total} & & & \textbf{\$13,872.88} \\ \bottomrule
            \end{tabular}
        \end{center}
        Finally, the premium is 110\% of the expected payout:
        \[
            \boxed{\text{Premium} = E[Y] \cdot 1.10 = \$\text{13,872.88} \cdot 1.10 \approx \$\text{15,260.17}.}
        \]
    }

\newpage

    \item (3 points) Suppose that \(X \sim b(m,p)\) and \(Y \sim b(n,p)\) are two independent random variables. Let \(Z = X + Y\). Explain why \(Z \sim b(m+n,p)\).
    \sol{
        Since \(Z = X + Y\), for \(Z\) to equal some value \(z\), it could be that \(X\) is 0 and \(Y\) could be \(z\), or \(X\) is \(1\), and \(Y\) is \(z - 1\), and so on. Thus, we can write this as a sum:
        \[
            P(Z=z) = \sum_{k = 0}^{z} P((X = k) \cap (Y = z - k)). 
        \]
        Because \(X\) and \(Y\) are independent events, we can rewrite this as the following:
        \[
            P(Z=z) = \sum_{k = 0}^{z} P(X = k)P(Y = z - k).
        \]
        Thus, from here, we can show that \(Z \sim b(m + n, p)\):
        \begin{align*}
            P(Z=z) &= \sum_{k=0}^{z} P(X=k)P(Y=z-k) \\
            &= \sum_{k=0}^{z} \left[ \binom{m}{k}p^{k} (1 - p)^{m - k} \right] \left[ \binom{n}{z - k}p^{z - k} (1 - p)^{n - (z - k)} \right] \\
            &= \sum_{k=0}^{z} \binom{m}{k} \binom{n}{z - k} p^{k + (z-k)} (1-p)^{(m-k) + (n-(z-k))} \\
            &= \sum_{k=0}^{z} \binom{m}{k} \binom{n}{z - k} p^{z} (1-p)^{m+n-z}. \\
            \intertext{Now, factor the terms that don't depend on \(k\) outside the sum:}
            &= p^{z} (1-p)^{m+n-z} \sum_{k=0}^{z} \binom{m}{k} \binom{n}{z - k}. \\
            \intertext{By Vandermonde's Identity, the entire sum simplifies to \(\binom{m + n}{z}\):}
            &= p^{z} (1-p)^{m+n-z} \left[ \binom{m+n}{z} \right] \\
            &= \binom{m+n}{z} p^{z} (1-p)^{(m+n)-z}.
        \end{align*}
        which is corresponds exactly to the pmf formula for the binomial distribution with \(m + n\) trials and a success probability of \(p\). Therefore, we have shown that \(Z \sim b(m + n, p)\).
    }
\end{enumerate}

\end{document}