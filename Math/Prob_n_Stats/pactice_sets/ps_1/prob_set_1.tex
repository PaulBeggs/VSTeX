\documentclass[11pt]{article}

\usepackage{fix-cm}
\usepackage{tikz}
\usepackage{forest}
\usepackage{amsmath,amsthm} 
\usepackage{amssymb}
\usepackage[framemethod=tikz]{mdframed}
% \usepackage{draculatheme}
\usepackage{mathrsfs}
\usepackage{changepage}
\usepackage{multicol}
\usepackage{mathtools}
\usepackage{hyperref}
\usepackage{slashed}
\usepackage{enumerate}
\usepackage{booktabs}
\usepackage{enumitem}
\usepackage{kantlipsum} 
\usepackage{tabularx}
\usepackage{array}
\usepackage{makecell}
\usepackage{pgfplots}
\pgfplotsset{compat=1.18}
\usetikzlibrary{
	decorations.markings,
	trees,
	arrows,
	automata,
	shapes,
	backgrounds,
	calc,
	patterns
}

\setlength{\parindent}{0pt}

\newmdenv[
  topline=false,
  bottomline=true,
  rightline=false,
  leftline=true,
  linewidth=1.5pt,
  linecolor=black, % default color, will be overridden in custom commands
  % backgroundcolor=draculabg, % Needed for Dracula theme
  % fontcolor=draculafg, % Needed for Dracula theme
  innertopmargin=0pt,
  innerbottommargin=5pt,
  innerrightmargin=10pt,
  innerleftmargin=10pt,
  leftmargin=0pt,
  rightmargin=0pt,
  skipabove=\topsep,
  skipbelow=\topsep,
]{customframedproof}

\newenvironment{proofpart}[2][black]{
    \begin{mdframed}[
        topline=false,
        bottomline=false,
        rightline=false,
        leftline=true,
        linewidth=1pt,
        linecolor=#1!40, % Custom color
        innertopmargin=10pt,
        innerbottommargin=10pt,
        innerleftmargin=10pt,
        innerrightmargin=10pt,
        leftmargin=0pt,
        rightmargin=0pt,
        % skipabove=\topsep,
        % skipbelow=\topsep%
    ]
    \noindent
    \begin{minipage}[t]{0.08\textwidth}%
        \textbf{#2}%
    \end{minipage}%
    \begin{minipage}[t]{0.90\textwidth}%
        \begin{adjustwidth}{0pt}{0pt}%
}{
    \end{adjustwidth}
    \end{minipage}
    \end{mdframed}
}

\newenvironment{solution}
  {\textit{Solution.}}



%%% AESTHETICS %%%
%-%-%-%-%-%-%-%-%-%-%-%-%-%-%-%-%-%-%-%-%-%-%-%-%-%-%-%-%-%-%-%-%-%-%-%-%-%-%


%%% Dimensions and Spacing %%%
\usepackage[left=0.5in,right=0.5in,top=1in,bottom=1in]{geometry}
% \usepackage{setspace}
% \linespread{1}
\usepackage{listings}

%%% Define new colors %%%
\usepackage{xcolor}
\definecolor{orangehdx}{rgb}{0.96, 0.51, 0.16}

% Normal colors
\definecolor{xred}{HTML}{BD4242}
\definecolor{xblue}{HTML}{4268BD}
\definecolor{xgreen}{HTML}{52B256}
\definecolor{xpurple}{HTML}{7F52B2}
\definecolor{xorange}{HTML}{FD9337}
\definecolor{xdotted}{HTML}{999999}
\definecolor{xgray}{HTML}{777777}
\definecolor{xcyan}{HTML}{80F5DC}
\definecolor{xpink}{HTML}{F690EA}
\definecolor{xgrayblue}{HTML}{49B095}
\definecolor{xgraycyan}{HTML}{5AA1B9}

% Dark colors
\colorlet{xdarkred}{red!85!black}
\colorlet{xdarkblue}{xblue!85!black}
\colorlet{xdarkgreen}{xgreen!85!black}
\colorlet{xdarkpurple}{xpurple!85!black}
\colorlet{xdarkorange}{xorange!85!black}
\definecolor{xdarkcyan}{HTML}{008B8B}
\colorlet{xdarkgray}{xgray!85!black}

% Very dark colors
\colorlet{xverydarkblue}{xblue!50!black}

% Document-specific colors
\colorlet{normaltextcolor}{black}
\colorlet{figtextcolor}{xblue}

% Enumerated colors
\colorlet{xcol0}{black}
\colorlet{xcol1}{xred}
\colorlet{xcol2}{xblue}
\colorlet{xcol3}{xgreen}
\colorlet{xcol4}{xpurple}
\colorlet{xcol5}{xorange}
\colorlet{xcol6}{xcyan}
\colorlet{xcol7}{xpink!75!black}

% Blue-Purple (should just used colorbrewer...)
\definecolor{xrainbow0}{HTML}{e41a1c}
\definecolor{xrainbow1}{HTML}{a24057}
\definecolor{xrainbow2}{HTML}{606692}
\definecolor{xrainbow3}{HTML}{3a85a8}
\definecolor{xrainbow4}{HTML}{42977e}
\definecolor{xrainbow5}{HTML}{4aaa54}
\definecolor{xrainbow6}{HTML}{629363}
\definecolor{xrainbow7}{HTML}{7e6e85}
\definecolor{xrainbow8}{HTML}{9c509b}
\definecolor{xrainbow9}{HTML}{c4625d}
\definecolor{xrainbow10}{HTML}{eb751f}
\definecolor{xrainbow11}{HTML}{ff9709}

%%% FIGURES %%%
\usepackage{graphicx}  
% \graphicspath{ {images/} }  
% \numberwithin{figure}{section}
\usepackage{float}
\usepackage{caption}

%%% Hyperlinks %%%
\usepackage{hyperref}
\definecolor{horange}{HTML}{f58026}
\hypersetup{
	colorlinks=true,
	linkcolor=horange,
	filecolor=horange,      
	urlcolor=horange,
}

\newcommand{\mysqrt}[1]{%
  \mathpalette\foo{#1}%
}
\newcommand{\dmysqrt}[1]{%
  \mathpalette\foodisplay{#1}%
}

\newcommand{\sol}[1]{
    \begin{customframedproof}[linecolor=orangehdx!75,]
        \begin{solution}
        #1
        \end{solution}
    \end{customframedproof}
}

% !TeX spellcheck = off
\newcommand{\foo}[2]{%
  % #1: math style, #2: content
  \sbox0{\(#1\sqrt{#2}\)}% Measure the size of the standard sqrt in the current style
  \begin{tikzpicture}[baseline=(sqrt.base)]
    \node[inner sep=0, outer sep=0] (sqrt) {\(#1\sqrt{#2}\)}; % Use the current math style
    \draw([yshift=-0.045em]sqrt.north east) -- ++(0,-0.5ex); % Draw the tick
  \end{tikzpicture}%
}
% !TeX spellcheck = off
\newcommand{\foodisplay}[2]{%
    % #1: math style, #2: content
    \sbox0{\(#1\sqrt{#2}\)}% Measure the size of the standard sqrt in the current style
    \begin{tikzpicture}[baseline=(sqrt.base)]
		\node[inner sep=0, outer sep=0] (sqrt) {\(\displaystyle\sqrt{#2}\)}; % Force displaystyle
    	\draw[line width=0.4pt] ([yshift=-0.044em]sqrt.north east) -- ++(0,-0.5ex); % Draw the tick
	\end{tikzpicture}%
}

\newcommand{\barNotationT}[1]{\bigg|_{t = #1}}

\newcommand{\cyanit}[1]{\textit{\textcolor{cyan}{#1}}}

\newcommand{\brackett}[1]{\left\langle #1 \right\rangle}

\newcommand{\norm}[1]{\left\lVert \mathbf{#1}\right\rVert}

\newcommand{\imb}{\mb{i}}
\newcommand{\jmb}{\mb{j}}
\newcommand{\kmb}{\mb{k}}
\newcommand{\rmb}{\mb{r}}
\newcommand{\umb}{\mb{u}}

\newcommand{\vecfuc}[2]{\mb{#1}(#2)}
\newcommand{\dvecfuc}[2]{\mb{#1}'(#2)}
\newcommand{\normdvecfuc}[2]{||\mb{#1}'(#2)||}

\newcommand{\proj}{\text{proj}}

\newcommand{\mb}[1]{\mathbf{#1}}

% \renewcommand{\theenumi}{\arabic{enumi}} 
% \renewcommand{\labelenumi}{\theenumi.}

\title{Probability and Statistics: Practice Set 1}
\author{Paul Beggs}
\date{\today}

%%% Custom Comands %%%
% Natural Numbers 
\newcommand{\N}{\ensuremath{\mathbb{N}}}

% Whole Numbers
\newcommand{\W}{\ensuremath{\mathbb{W}}}

% Integers
\newcommand{\Z}{\ensuremath{\mathbb{Z}}}

% Rational Numbers
\newcommand{\Q}{\ensuremath{\mathbb{Q}}}

% Real Numbers
\newcommand{\R}{\ensuremath{\mathbb{R}}}

% Complex Numbers
\newcommand{\C}{\ensuremath{\mathbb{C}}}

\newcommand{\I}{\ensuremath{\mathbb{I}}}

\newcommand{\p}{\partial}

\newcommand{\mbi}{\mathbf{i}}
\newcommand{\mbj}{\mathbf{j}}
\newcommand{\mbk}{\mathbf{k}}
\newcommand{\mbr}{\mathbf{r}}

\tikzset{
  treenode/.style = {align=center, inner sep=0pt, text centered},
  arn/.style = {treenode, circle, draw, fill=white, text width=4.75ex},
  rectnode/.style = {treenode, rectangle, draw, fill=white, inner sep=4pt, text width=8.4ex}
}

\begin{document}

\maketitle
\begin{enumerate}
\vspace*{-0.25in}
  	\item (2 points) Four people are choosing an integer from \(\{1,2,\dots,10\}\) at random\textemdash assume that each choice is equally likely. What is the probability that all four choose different numbers?
  	\sol{
        For all 4 people, each integer has a \(\frac{1}{10}\) chance of being picked. That means there are \(10^{4} = 10000\) possible combinations. To find how many permutations, we calculate \(_{10}P_{4} = 5040\). Using this number, we find a \(\frac{5040}{10000} = 50.4\%\) chance that all four choose different numbers.
  	}

	\item (2 points) A pair of fair, 6-sided dice are thrown. Find the probability that the sum has a total above 9.
  	\sol{
		Each two pair outcome has a \(\frac{1}{36}\) chance of being picked. There are one, two, and three ways to get a 12, 11, and 10, respectively. Therefore, there is a \(\frac{1 + 2 + 3}{36} = \frac{1}{6}\) chance of getting a sum above 9.
  	}
	\item (2 points each) Equals-sign-itis is a disease where some people's face break out into little red equals signs if they take too many mathematics courses. Currently, 2.1\% of Hendrix students are infected. A test can be taken before the break to determine if you are infected\textemdash it is 94\% reliable.
	\begin{enumerate}[label=(\alph*)]
		\item What is the probability that you have the disease if your test comes back positive?
		\begin{customframedproof}[linecolor=orangehdx!75]
			\vspace*{-.8em}
			\begin{multicols}{2}
				\begin{minipage}[t]{\linewidth}
					\textit{Solution.}\ \ The probability of having the disease and testing positive can be calculated by finding the following:
					\begin{align*}
						P(\text{sick} \mid +) &= \frac{P(\text{sick} \cap +)}{P(+)} \\
						&= \frac{0.01974}{0.01974 + 0.05874} = 0.2515.
					\end{align*}
					Therefore, you have a \(\boxed{25.15\%}\) chance of having the disease if your test comes back positive.
				\end{minipage}

				\begin{forest}
					my x/.style n args=2{
						edge label={node [midway, #2, font=\sffamily, text=black] {$#1$}},
					},
					for tree={
						l sep+=3.5mm,
						% s sep+=9mm
					}
					[ , circle, for descendants={arn}
						[Sick, my x={2.1\%}{left}
							[+, my x={94\%}{left}
								[ , rectnode, content={1.974\%}
								]
							]
							[-, my x={6\%}{right}
								[ , rectnode, content={0.126\%}
								]
							]
							]
							[Not, my x={97.9\%}{right}
							[+, my x={6\%}{left}
								[ , rectnode, content={5.874\%}
								]
							]
							[-, my x={94\%}{right}
								[ , rectnode, content={92.026\%}
								]
							]
						]
					]
				\end{forest}
			\end{multicols}
		\end{customframedproof}

		\item Determine the probability that you have the disease if you test comes back negative?
		\sol{
			Similarly to the previous problem:
			\[
				P(\text{sick} \mid -) = \frac{P(\text{sick} \cap -)}{P(-)} = \frac{0.00126}{0.00126 + 0.92026} = 0.00137.
			\]
			Therefore, you have a \boxed{0.14\%} chance of having the disease if your test comes back negative.
		}


	\end{enumerate}
	\item An urn contains 5 Red, 7 Blue, and 1 White marbles. Two balls will be selected, without replacement.
	\begin{enumerate}
		\item (2 points) Assuming that order matters, write the sample space for this experiment, and each outcome's probability. [Hint: If you draw a Red, followed by a White, you might write \((R,W),0.12345\). A probability tree might be useful here.]
		\sol{
			For the top level, there are \(\frac{1}{13}\), \(\frac{5}{13}\), and \(\frac{7}{13}\) chances for White, Red, and Blue, respectively. After we remove a marble, then the next layer has 12 total marbles, and 1 less colored marble respective to the color in the above layer, as we can see below:

			\begin{center}
				\begin{forest}
					my x/.style n args=2{
						edge label={node [midway, #2, font=\sffamily, text=black] {\(#1\)}},
					},
					for tree={
						l sep+=4mm,
						% s sep+=9mm
					}
					[ , circle, for descendants={arn}
						[W, my x={\frac{1}{13}}{fill=white}
							[R, my x={\frac{5}{12}}{fill=white}
								[ , rectnode, content={5/156}
								]
							]
							[B, my x={\frac{7}{12}}{fill=white}
								[ , rectnode, content={7/156}
								]
							]
						]
						[R, my x={\frac{5}{13}}{right}
							[R, my x={\frac{4}{12}}{fill=white}
								[ , rectnode, content={5/39}
								]
							]
							[B, my x={\frac{7}{12}}{fill=white}
								[ , rectnode, content={35/156}
								]
							]
							[W, my x={\frac{1}{12}}{fill=white}
								[ , rectnode, content={5/156}
								]
							]
						]
						[B, my x={\frac{7}{13}}{fill=white}
							[R, my x={\frac{5}{12}}{fill=white}
								[ , rectnode, content={35/156}
								]
							]
							[B, my x={\frac{6}{12}}{fill=white}
								[ , rectnode, content={7/26}
								]
							]
							[W, my x={\frac{1}{12}}{fill=white}
								[ , rectnode, content={7/156}
								]
							]
						]
					]
				\end{forest}
			\end{center}

			Thus, the sample space is 
			\[
				S = \left\{RW = \frac{5}{156}, WB = \frac{7}{156}, RR = \frac{5}{39}, RB = \frac{35}{156}, RW = \frac{5}{156}, BR = \frac{35}{156}, BB = \frac{7}{26}, BW = \frac{7}{156} \right\}
			\]
		}

		\item (2 points) Let event \(A\) be getting at least one Red marble. Find \(P(A)\).
		\sol{
			The probability of getting at least one marble entails adding each combination of Red and another marble. Given the values from the tree, we can calculate the following equation:
			\[
				P(A) = \frac{5}{156} + \frac{5}{39} + \frac{35}{156} + \frac{5}{156} + \frac{35}{156} = \frac{25}{39}.
			\]
		}

		\item (2 points) Let event \(B\) be getting at a White marble on the second draw. Find \(P(B)\).
		\sol{
			Since it's only possible to get a White marble on the second draw after pulling a Red or Blue, we have to take that into account with our calculation:
			\[
				P(B) = P(R \cap W) + P(B \cap W) = \frac{5}{156} + \frac{7}{156} = \frac{1}{13}.
			\]
		}
		\item (2 points) Find \(P(A \mid B)\), using the events in the previous two parts.
		\sol{
			\(\boxed{P(A \mid B) = \frac{P(A \cap B)}{P(B)} = \frac{5/156}{1/13} = \frac{5}{12}}\)
		}
	\end{enumerate}

\newpage
	\item (2 points) An urn contains 10 marbles: 4 Red and 6 Blue. A second urn contains 16 Red marbles and an unknown number of Blue marbles. A single marble is drawn from each urn. The probability that both marbles are the same color is \(0.44\). How many blue marbles are there in the second urn?
	\sol{
		We have two urns: Urn\(_{1}\): 4 Red, 6 Blue: 10 total, and Urn\(_{2}\): 16 Red, \(b\) Blue: \(16 + b\) total. Let event \(A\) be getting both red. Thus, we find the probability to be:
		\[
			P(A) = \frac{4}{10} \cdot \frac{16}{16 + b}.
		\]
		Similarly, let event \(B\) be getting both blue. Hence:
		\[
			P(B) = \frac{6}{10} \cdot \frac{b}{16 + b}.
		\]
		Thus, the total probability is
		\[
			P(A \cup B) = \frac{4}{10} \cdot \frac{16}{16 + b} + \frac{6}{10} \cdot \frac{b}{16 + b} = \frac{(64 + 6b)}{10(16 + b)}.
		\]
		Setting this equal to \(0.44\), we can solve for \(b\):
		\begin{align*}
			\frac{(64 + 6b)}{10(16 + b)} &= 0.44 \\
			64 + 6b &= 4.4(16 + b) \\
			64 + 6b &= 70.4 + 4.4b \\
			1.6b &= 6.4 \\
			b &= 4.
		\end{align*}
		Therefore, the total amount of Blue marbles in Urn\(_{2}\) is \(\boxed{4}\).
	}

	\item (2 points) A fair coin is flipped three times. Given that you have at least one Head, find the probability that you have at least two Heads.
	\sol{
		Let event \(A\) be the probability of getting at least 2 Heads, and event \(B\) being at least 1 Head. The sample space consists of 8 outcomes, only one of which does not contain at least one head: \(\{TTT\}\). Hence, \(P(B) = \frac{7}{8}\). Then, there are 4 outcomes that contain at least one head, so \(P(A) = \frac{4}{8} = \frac{1}{2}\). It follows that \(P(A \cap B) = P(A)\) given that having at least two heads implies having at least one. Hence:
		\[
			P(\text{At least } 2 H\mid \text{At least }1 H) = \frac{P(A \cap B)}{P(B)} = \frac{1/2}{7/8} = \frac{4}{7}.
		\]
		Therefore, the probability of at least 2 Heads given 1 Head is \(\boxed{\tfrac{4}{7}}\).
	}

	\item (2 points) In a certain state, car license plates have the format of three letters followed by three numbers. How many possible license plates are there?
	\sol{
		Since we can use duplicate letters and numbers in the license plates, we find the total possible license plates with the following equation: \(\boxed{26^{3} \cdot 10^{3} = 17576000}\).
	}

	\item (2 points) Three cards are drawn from a standard 52-card deck. Find the probability that at least one card is an Ace.
	\sol{
		The probability of getting 3 cards that have no Aces in them is
		\[
			P(OOO) = \frac{48}{52} \cdot \frac{47}{51} \cdot \frac{46}{50} = \frac{4324}{5525}.
		\]
		We can find the probability of at least one Ace by finding the complement of this probability:
		\[
			P(\text{at least 1 Ace}) = P'(OOO) = 1 - \frac{4324}{5525} = \frac{1201}{5525}.
		\]
		Therefore, we have approximately a \(\boxed{21.73\%}\) chance of getting at least one Ace.
	}

	\item (3 points) Show that \(\displaystyle\binom{n}{r} = \binom{n - 1}{r - 1} + \binom{n - 1}{r}\).
	\sol{
		Expanding the right side we get the following:
		\begin{align*}
			\binom{n - 1}{r - 1} + \binom{n - 1}{r} &= \frac{(n - 1)!}{(r - 1)!(n - 1 - (r - 1))!} + \frac{(n - 1)!}{r!(n - 1 - r)!} \\
			&= \frac{(n - 1)!}{(r - 1)!(n - r)!} + \frac{(n - 1)!}{r!(n - r - 1)!}. \\
			\intertext{To continue, we need to find a common denominator, and to do so, we need to adjust two denominator terms \((r - 1)!\) and \((n - r - 1)!\). Multiplying these terms by a common term \(r(n - r)\), we get the following:}
			&= \frac{r \cdot (n - 1)!}{r \cdot (r - 1)!(n - r)!} + \frac{(n - r)(n - 1)!}{r!(n - r) \cdot (n - r - 1)!}. \\
			\intertext{Now, when we multiply those terms \(r \cdot (r - 1)! = r!\) and \((n - r) \cdot (n - r - 1) = (n - r)!\), we are ``going back'' one term to get matching denominators.}
			&= \frac{r(n - 1)! + (n - r)(n - 1)!}{r!(n- r)!} \\
			&= \frac{(n - 1)!(r + (n - r))}{r!(n- r)!} \\
			&= \frac{(n - 1)!n}{r!(n- r)!} \\
			&= \frac{n!}{r!(n - r)!}.
		\end{align*}
		Therefore, by definition of the binomial coefficient, the two sides of the equation are equal.
	}

\newpage
	\item (3 points) Show that \(\displaystyle\sum_{r=0}^{n}\binom{n}{r} = 2^{n}\).
	\sol{
		Starting with the right side, we can factor \(2^{n}\), use the Binomial Theorem, and then simplify to show the specified property:
		\begin{align*}
			2^{n} &= (1 + 1)^{n} \\
			(1 + 1)^{n} &= \sum_{r = 0}^{n} \binom{n}{r}(1)^{n - r}(1)^{r} \\
			2^{n} &= \sum_{r = 0}^{n} \binom{n}{r}.
		\end{align*}
	}
\end{enumerate}

\end{document}