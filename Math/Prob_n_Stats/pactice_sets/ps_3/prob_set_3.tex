\documentclass[11pt]{article}

\usepackage{fix-cm}
\usepackage{tikz}
\usepackage{forest} 
\usepackage{amsmath,amsthm} 
\usepackage{amssymb}
\usepackage[framemethod=tikz]{mdframed}
% \usepackage{draculatheme}
\usepackage{mathrsfs}
\usepackage{changepage}
\usepackage{multicol}
\usepackage{mathtools}
\usepackage{hyperref}
\usepackage{slashed}
\usepackage{enumerate}
\usepackage{booktabs}
\usepackage{enumitem}
\usepackage{kantlipsum} 
\usepackage{tabularx}
\usepackage{array}
\usepackage{makecell}
\usepackage{pgfplots}
\pgfplotsset{compat=1.18}
\usetikzlibrary{
	decorations.markings,
	trees,
	arrows,
	automata,
	shapes,
	backgrounds,
	calc,
	patterns
}

\setlength{\parindent}{0pt}

\newmdenv[
	topline=false,
	bottomline=true,
	rightline=false,
	leftline=true,
	linewidth=1.5pt,
	linecolor=black, % default color, will be overridden in custom commands
	% backgroundcolor=draculabg, % Needed for Dracula theme
	% fontcolor=draculafg, % Needed for Dracula theme
	innertopmargin=0pt,
	innerbottommargin=5pt,
	innerrightmargin=10pt,
	innerleftmargin=10pt,
	leftmargin=0pt,
	rightmargin=0pt,
	skipabove=\topsep,
	skipbelow=\topsep,
]{customframedproof}

\newenvironment{proofpart}[2][black]{
    \begin{mdframed}[
        topline=false,
        bottomline=false,
        rightline=false,
        leftline=true,
        linewidth=1pt,
        linecolor=#1!40, % Custom color
        innertopmargin=10pt,
        innerbottommargin=10pt,
        innerleftmargin=10pt,
        innerrightmargin=10pt,
        leftmargin=0pt,
        rightmargin=0pt,
        % skipabove=\topsep,
        % skipbelow=\topsep%
    ]
    \noindent
    \begin{minipage}[t]{0.08\textwidth}%
        \textbf{#2}%
    \end{minipage}%
    \begin{minipage}[t]{0.90\textwidth}%
        \begin{adjustwidth}{0pt}{0pt}%
}{
    \end{adjustwidth}
    \end{minipage}
    \end{mdframed}
}

\newenvironment{solution}
  {\textit{Solution.}}



%%% AESTHETICS %%%
%-%-%-%-%-%-%-%-%-%-%-%-%-%-%-%-%-%-%-%-%-%-%-%-%-%-%-%-%-%-%-%-%-%-%-%-%-%-%


%%% Dimensions and Spacing %%%
\usepackage[left=0.5in,right=0.5in,top=1in,bottom=1in]{geometry}
% \usepackage{setspace}
% \linespread{1}
\usepackage{listings}

%%% Define new colors %%%
\usepackage{xcolor}
\definecolor{orangehdx}{rgb}{0.96, 0.51, 0.16}

% Normal colors
\definecolor{xred}{HTML}{BD4242}
\definecolor{xblue}{HTML}{4268BD}
\definecolor{xgreen}{HTML}{52B256}
\definecolor{xpurple}{HTML}{7F52B2}
\definecolor{xorange}{HTML}{FD9337}
\definecolor{xdotted}{HTML}{999999}
\definecolor{xgray}{HTML}{777777}
\definecolor{xcyan}{HTML}{80F5DC}
\definecolor{xpink}{HTML}{F690EA}
\definecolor{xgrayblue}{HTML}{49B095}
\definecolor{xgraycyan}{HTML}{5AA1B9}

% Dark colors
\colorlet{xdarkred}{red!85!black}
\colorlet{xdarkblue}{xblue!85!black}
\colorlet{xdarkgreen}{xgreen!85!black}
\colorlet{xdarkpurple}{xpurple!85!black}
\colorlet{xdarkorange}{xorange!85!black}
\definecolor{xdarkcyan}{HTML}{008B8B}
\colorlet{xdarkgray}{xgray!85!black}

% Very dark colors
\colorlet{xverydarkblue}{xblue!50!black}

% Document-specific colors
\colorlet{normaltextcolor}{black}
\colorlet{figtextcolor}{xblue}

% Enumerated colors
\colorlet{xcol0}{black}
\colorlet{xcol1}{xred}
\colorlet{xcol2}{xblue}
\colorlet{xcol3}{xgreen}
\colorlet{xcol4}{xpurple}
\colorlet{xcol5}{xorange}
\colorlet{xcol6}{xcyan}
\colorlet{xcol7}{xpink!75!black}

% Blue-Purple (should just used colorbrewer...)
\definecolor{xrainbow0}{HTML}{e41a1c}
\definecolor{xrainbow1}{HTML}{a24057}
\definecolor{xrainbow2}{HTML}{606692}
\definecolor{xrainbow3}{HTML}{3a85a8}
\definecolor{xrainbow4}{HTML}{42977e}
\definecolor{xrainbow5}{HTML}{4aaa54}
\definecolor{xrainbow6}{HTML}{629363}
\definecolor{xrainbow7}{HTML}{7e6e85}
\definecolor{xrainbow8}{HTML}{9c509b}
\definecolor{xrainbow9}{HTML}{c4625d}
\definecolor{xrainbow10}{HTML}{eb751f}
\definecolor{xrainbow11}{HTML}{ff9709}

%%% FIGURES %%%
\usepackage{graphicx}  
% \graphicspath{ {images/} }  
% \numberwithin{figure}{section}
\usepackage{float}
\usepackage{caption}

%%% Hyperlinks %%%
\usepackage{hyperref}
\definecolor{horange}{HTML}{f58026}
\hypersetup{
	colorlinks=true,
	linkcolor=horange,
	filecolor=horange,      
	urlcolor=horange,
}

\newcommand{\mysqrt}[1]{%
  \mathpalette\foo{#1}%
}
\newcommand{\dmysqrt}[1]{%
  \mathpalette\foodisplay{#1}%
}

\newcommand{\sol}[1]{
    \begin{customframedproof}[linecolor=orangehdx!75,]
        \begin{solution}
        #1
        \end{solution}
    \end{customframedproof}
}

% !TeX spellcheck = off
\newcommand{\foo}[2]{%
	% #1: math style, #2: content
	\sbox0{\(#1\sqrt{#2}\)}% Measure the size of the standard sqrt in the current style
	\begin{tikzpicture}[baseline=(sqrt.base)]
    	\node[inner sep=0, outer sep=0] (sqrt) {\(#1\sqrt{#2}\)}; % Use the current math style
    	\draw([yshift=-0.045em]sqrt.north east) -- ++(0,-0.5ex); % Draw the tick
  	\end{tikzpicture}%
}
% !TeX spellcheck = off
\newcommand{\foodisplay}[2]{%
    % #1: math style, #2: content
    \sbox0{\(#1\sqrt{#2}\)}% Measure the size of the standard sqrt in the current style
    \begin{tikzpicture}[baseline=(sqrt.base)]
		\node[inner sep=0, outer sep=0] (sqrt) {\(\displaystyle\sqrt{#2}\)}; % Force displaystyle
    	\draw[line width=0.4pt] ([yshift=-0.044em]sqrt.north east) -- ++(0,-0.5ex); % Draw the tick
	\end{tikzpicture}%
}

\newcommand{\barNotationT}[1]{\bigg|_{t = #1}}

\newcommand{\cyanit}[1]{\textit{\textcolor{cyan}{#1}}}

\newcommand{\brackett}[1]{\left\langle #1 \right\rangle}

\newcommand{\norm}[1]{\left\lVert \mathbf{#1}\right\rVert}

\newcommand{\imb}{\mb{i}}
\newcommand{\jmb}{\mb{j}}
\newcommand{\kmb}{\mb{k}}
\newcommand{\rmb}{\mb{r}}
\newcommand{\umb}{\mb{u}}

\newcommand{\vecfuc}[2]{\mb{#1}(#2)}
\newcommand{\dvecfuc}[2]{\mb{#1}'(#2)}
\newcommand{\normdvecfuc}[2]{||\mb{#1}'(#2)||}

\newcommand{\proj}{\text{proj}}

\newcommand{\mb}[1]{\mathbf{#1}}

% \renewcommand{\theenumi}{\arabic{enumi}} 
% \renewcommand{\labelenumi}{\theenumi.}

\title{Probability and Statistics: Practice Set 3}
\author{Paul Beggs}
\date{\today}

%%% Custom Comands %%%
% Natural Numbers 
\newcommand{\N}{\ensuremath{\mathbb{N}}}

% Whole Numbers
\newcommand{\W}{\ensuremath{\mathbb{W}}}

% Integers
\newcommand{\Z}{\ensuremath{\mathbb{Z}}}

% Rational Numbers
\newcommand{\Q}{\ensuremath{\mathbb{Q}}}

% Real Numbers
\newcommand{\R}{\ensuremath{\mathbb{R}}}

% Complex Numbers
\newcommand{\C}{\ensuremath{\mathbb{C}}}

\newcommand{\I}{\ensuremath{\mathbb{I}}}

\newcommand{\p}{\partial}

\newcommand{\mbi}{\mathbf{i}}
\newcommand{\mbj}{\mathbf{j}}
\newcommand{\mbk}{\mathbf{k}}
\newcommand{\mbr}{\mathbf{r}}

\newcommand{\bigpfrac}[2]{\left( \frac{#1}{#2} \right)}
\newcommand{\negbigpfrac}[2]{\left(- \frac{#1}{#2} \right)}


\begin{document}

\maketitle

\begin{enumerate}
    \item (2 points each) Consider the function \(f(x) = cx^{5}\) on \([0,1]\).
    \begin{enumerate}
        \item Find the value of \(c\) for which this is a probability density function.
        \sol{
            \[
                \int_{0}^{1} cx^{5} \, dx = 1 \implies c \int_{0}^{1} x^{5} \, dx = 1 \implies c \left[ \frac{x^{6}}{6} \right]_{0}^{1} = 1 \implies c \cdot \frac{1}{6} = 1 \implies c = 6.
            \]
        }
        \item Using this value of \(c\), determine \(E(X)\).
        \sol{
            \[
                E(X) = \int_{0}^{1} x \cdot 6x^{5} \, dx = 6 \left[ \frac{x^{7}}{7} \right]_{0}^{1} = \frac{6}{7}.
            \]  
        }
        \item Again, using this \(c\), determine \(\text{Var}(X)\).
        \sol{
            \begin{align*}
                \text{Var}(X) &= \int_{0}^{1} \left(x - \frac{6}{7}\right)^{2}6x^{5} \, dx \\
                &= \int_{0}^{1} \left(x^{2} - \frac{12}{7}x + \frac{36}{49}\right) 6x^{5} \, dx \\
                &= \int_{0}^{1} \left( 6x^{7} - \frac{72}{7}x^{6} + \frac{216}{49}x^{5} \right) \, dx \\
                &= \frac{6}{8} - \frac{72}{49} + \frac{36}{49} \\
                &= \frac{3}{196}.
            \end{align*}
        }
        \item Find \(\pi_{0.25}\).
        \sol{
            \[
                \int_{0}^{b} 6x^{5} \, dx = 0.25 \implies 6 \left[ \frac{x^{6}}{6} \right]_{0}^{b} = 0.25 \implies b^{6} = 0.25 \implies b = 0.25^{\frac{1}{6}} \implies b \approx 0.7937.
            \]
            Therefore, \(\pi_{0.25} \approx 0.7937\).
        }
    \end{enumerate}
\newpage
    \item (2 points) Consider the function \(\displaystyle f(x) = \frac{\ln(x)}{x^{2}}\) on the interval \([1, \infty)\).
    \begin{enumerate}
        \item Show that this is a probability distribution. [Hint: You will likely need integration by parts here.]
        \sol{
            We choose \(u = \ln(x)\) so that \(du = \frac{1}{x}\), and \(dv = \frac{1}{x^{2}}\) so that \(v = -\frac{1}{x}\). This leaves us with the following:
            \begin{align*}
                \left[ {uv} \right]_{1}^{\infty} - \int_{1}^{\infty} v \, du &= \left[ {-\frac{\ln(x)}{x}} \right]_{1}^{\infty} + \int_{1}^{\infty} \frac{1}{x^{2}} \, dx \\
                &= \left( \frac{\ln(1)}{1} - \lim_{x \to \infty} \frac{\ln(x)}{x} \right) - \left( \lim_{x \to \infty} \frac{1}{x} - \frac{1}{1} \right)\\
                &= (0 - 0) - (0 - 1) \\
                &= 1
            \end{align*}
            Since \(f(x) \geq 0\) for all \(x \in [1, \infty)\) and the integral over this interval is 1, this is a valid probability density function.
        }
        \item Show that \(E(X)\) is undefined.
        \sol{
            \[
                E(X) = \int_{1}^{\infty} x \cdot \frac{\ln(x)}{x^{2}} \, dx = \int_{1}^{\infty} \frac{\ln(x)}{x} \, dx.
            \]
            For this problem, we can use \(u\)-substitution. Let \(u = \ln(x)\), then \(du = \frac{1}{x} \, dx\), which transforms the limits from \(1\) to \(\infty\) into \(0\) to \(\infty\):
            \[
                E(X) = \int_{0}^{\infty} u \, du = \left[ \frac{u^{2}}{2} \right]_{0}^{\infty} = \infty.
            \]
            Therefore, \(E(X)\) is undefined.
        }
    \end{enumerate}
\newpage
    \item (2 points) A pdf is given by \(\displaystyle f(x) = \frac{1}{2}\) for \(x \in [0,1] \cup [2,3]\), and 0 otherwise. [i.e., this is like a uniform distribution, but with a ``gap'' from 1 to 2.] Determine \(E(X)\) and \(\text{Var}(X)\).
    \sol{
        \[
            E(X) = \int_{0}^{1} x \frac{1}{2} \, dx + \int_{2}^{3} x \frac{1}{2} \, dx = \left[ \frac{x^{2}}{4} \right]_{0}^{1} + \left[ \frac{x^{2}}{4} \right]_{2}^{3} = \left[ \frac{1}{4} - 0 \right] + \left[ \frac{9}{4} - 1 \right] = \frac{3}{2}
        \]
        \[
            \text{Var}(x) =  \int_{0}^{1} x^{2} \frac{1}{2} \, dx + \int_{2}^{3} x^{2} \frac{1}{2} \, dx - \frac{9}{4} =  \left[ \frac{x^{3}}{6} \right]_{0}^{1} + \left[ \frac{x^{3}}{6} \right]_{2}^{3} - \frac{9}{4} = \left[ \frac{1}{6} - 0 \right] + \left[ \frac{27}{6} - \frac{8}{6} \right] - \frac{9}{4} = \frac{13}{12}
        \]
    }
    \item (2 points each) We have shown in class that the gamma function \(\Gamma\) has the two properties:
    \begin{itemize}
        \item \(\displaystyle \Gamma \left( \frac{1}{2} \right) = \mysqrt{\pi}\).
        \item \(\Gamma(x) = (x - 1)\Gamma(x - 1)\).
    \end{itemize}
    Use these to find the exact values of:
    \begin{enumerate}
        \item \(\Gamma \left( \dfrac{5}{2} \right)\).
        \sol{
            \[
                \Gamma\left( \frac{5}{2} \right) = \bigpfrac{3}{2} \Gamma\bigpfrac{3}{2} = \bigpfrac{3}{2}\bigpfrac{1}{2}\mysqrt{\pi} = \frac{3\mysqrt{\pi}}{4}
            \]
        }
        \item \(\Gamma \left( -\dfrac{7}{2} \right)\).
        \sol{
            \begin{align*}
                \Gamma(x) &= (x - 1)\Gamma(x - 1) \\
                \frac{\Gamma(x)}{(x - 1)} &= \Gamma(x - 1) \\
                \intertext{Let \(z = x - 1\):}
                \frac{\Gamma(z + 1)}{z} &= \Gamma(z)
            \end{align*}
            With this formula:
            \[
                \Gamma\negbigpfrac{2}{7} = \negbigpfrac{2}{7}\negbigpfrac{2}{5}\negbigpfrac{2}{3}\negbigpfrac{2}{1} \mysqrt{\pi} = \frac{16\mysqrt{\pi}}{105}
            \]
        }
    \end{enumerate}
\newpage
	\item (2 points each) Customers arrive at a certain bank according to an approximate Poisson process at a mean rate of 15 per hour.
	\begin{enumerate}
		\item What is the probability that between 10 and 13 customers come in a particular hour?
		\sol{
            Let \(X\) count the number of customers in 1 particular hour. Then,
            \[
                P(10 \le X \le 13) = P(X \le 13) - P(X \le 9) = 0.2934.
            \]
            Therefore, there is a 29.3\% chance that between 10 and 13 customers come in a particular hour.
		}
		\item What is the probability that the first customer comes between 5 and 10 minutes into the bank's opening?
		\sol{
            Let \(X\) be the waiting time for the first customer to arrive. Using the following theta:
            \[
                \theta = \frac{1 \text{ hour}}{15 \text{ customers}} \cdot \frac{60 \text{ minutes}}{1 \text{ hour}} = 4 \text{ minutes per customer},
            \]
            we can plug this into the equation:
            \[
                P(5 \le X \le 10) = P(X \le 10) - P(X \le 5) = 1 - e^{-10/4} - 1 + e^{-5/4} = 0.2044
            \]
            Therefore, there is a 20.4\% chance that the first customer comes between 5 and 10 minutes into the bank's opening.
		}
		\item What is the probability that the \textit{third} customer arrives between 5 and 10 minutes into the bank's opening?
		\sol{
            Let \(\alpha = 3\) with same theta from the previous part. Then:
            \[
                \int_{5}^{10} \frac{1}{\Gamma(\alpha) \theta^{\alpha}} \cdot x^{\alpha - 1} e^{-x/\theta} \, dx = \int_{5}^{10} \frac{x^{2} e^{-x/4}}{32} \, dx = 0.3247.
            \]
            Therefore, there is a 32.5\% chance that the third customer comes between 5 and 10 minutes into the bank's opening.
		}
	\end{enumerate}
\newpage
	\item (1 point each) The mean airspeed of an unladen swallow is 30 ft/s, with a standard deviation of \(4.3\) ft/s. Assuming the distribution of speed is normal:
	\begin{enumerate}
		\item What is the probability that a randomly selected swallow will have a speed between 28 and 38 ft/s?
		\sol{
            \[
                P(28 \le X \le 38) = \frac{1}{4.3\mysqrt{2\pi}} \int_{28}^{38} \exp \left( -\frac{(t-30)^{2}}{2(4.3)^{2}} \right) \, dt = 0.6477.
            \]
		}
		\item Suppose that 12 swallows are selected at random. What is the probability that exactly 8 have their speed between 28 and 38 ft/s?
		\sol{
            Let \(Y\) count the birds that have their speed between 28 and 38 ft/s. This implies \(Y \sim Binomial(12, 0.6477)\). Hence,
            \[
                P(Y = 8) = \binom{12}{8} 0.6477^{8} \left( 1 - 0.6477 \right)^{4} = 0.2361.
            \]
		}
		\item Determine \(\pi_{0.90}\) for the swallow speed.
		\sol{
            \[
                \pi_{0.90} = \text{InvNorm}(.9, 30, 4.3, \text{LEFT}) = 35.5107.
            \]
		}
	\end{enumerate}
\newpage

    \item (3 points) An auto insurance company insures an automobile worth \$15,000 for one year under a policy with a \$1,000 deductible. During the policy year there is a \(0.04\) chance of partial damage to the car and a \(0.02\) chance of a total loss of the car. If there is partial damage to the car, the amount \(X\) of damage (in thousands) follows a distribution with density function \(f(x) = 0.5003e^{-x/2}\), for \(0 \le x \le 15\). Find the expected value of the payment the insurance company makes in a year.
    \sol{
        Let \(E\) be the expected payment for the insurance company. We have three cases to consider:
        \begin{enumerate}[label=\arabic*.]
            \item \textbf{No Damage:} The probability of no damage is 0.94. In this case, the insurance company pays \$0.
            \item \textbf{Total Loss:} The probability of total loss is 0.02. The payment is the value of the car minus the deductible, which is \(\$\text{15,000} - \$\text{1,000} = \$\text{14,000}\).
            \item \textbf{Partial Damage:} The probability of partial damage is \(0.04\). The damage amount \(X\), in thousands of dollars, follows the probability density function \(f(x) = 0.5003e^{-x/2}\) for \(0 \le x \le 15\). The deductible is \$1,000. The insurance payment for partial damage is therefore \(1000(X-1)\) if \(X > 1\) and 0 otherwise.
        \end{enumerate}

        The expected payment is the sum of the probabilities of each event multiplied by the corresponding payment amount. Let \(Y\) correspond to the payment for partial damage:
        \[
            E = (0.94 \cdot \$0) + (0.02 \cdot \$\text{14,000}) + (0.04 \cdot E[Y]).
        \]

        To find the expected payment for partial damage, we calculate the following integral:
        \begin{align*}
            E[Y] &= \int_{1}^{15} 1000(x-1)(0.5003e^{-x/2}) \, dx \\
            &=  500.3 \int_{1}^{15} (x-1)e^{-x/2} \, dx \\
            &\approx \$1211.96.
        \end{align*}

        Now, we can find the total expected payment:
        \[
            E = \$0 + \$280 + 0.04 \cdot \$1211.96 = \$280 + \$48.48 = \$328.48
        \]
        Therefore, the expected value of the payment the insurance company makes in a year is approximately \textbf{\$328.48}.
    }
\end{enumerate}

\end{document}