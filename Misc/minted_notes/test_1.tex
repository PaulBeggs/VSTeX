\documentclass{article}

\usepackage[svgnames]{xcolor}
\usepackage{tikz}
\usetikzlibrary{shapes,backgrounds,calc,patterns,positioning}
\usepackage[most]{tcolorbox}
\usepgflibrary{shadings}
\tcbuselibrary{skins}
\tcbuselibrary{theorems}
\usepackage{dingbat}
\usepackage{etoolbox}
\usepackage{pgfkeys}
\usepackage{pdftexcmds}
\usepackage{shellesc}
\usepackage{geometry}

% Load necessary libraries for tcolorbox
\tcbuselibrary{minted, breakable}
\usepackage{minted}
\setminted{escapeinside=||}
\usepackage[framemethod=tikz]{mdframed}
\usepackage{fancyvrb}
% \usepackage{listings}
% \usemintedstyle{dracula}

\newtcblisting{coolminted}{
    listing engine=minted,
    colback=black!5,                  % background color
    colframe=blue!75!black,            % frame color
    listing only,                     % do not display the box content in two ways
    minted language=python,           % default language; change as needed
    enhanced,                         % enable enhanced features
    frame hidden,                     % hide default frame
    interior hidden,                  % hide default interior
    breakable,                        % allow breaking over pages
    overlay={
      % Draw a custom frame using the overlay feature
      \draw[blue!75!black, line width=2pt]
        ([xshift=5pt,yshift=-5pt]frame.south west)
        rectangle
        ([xshift=-5pt,yshift=5pt]frame.north east);
    },
    minted options={
        mathescape=true,           % enable math escape in minted
        linenos=true,              % show line numbers
        numbersep=5pt,             % space between line numbers and code
        gobble=2,                  % number of spaces to remove from the beginning of each line
        frame=lines,               % frame style (lines, none, leftline, rightline, topline, bottomline)
        framesep=2mm,              % space between frame and content
        fontsize=\Large,           % font size for the code
        breaklines=true,           % enable line breaking
        breakautoindent=true,      % auto indent after line breaks
    },
}


% Define Dracula colors
\definecolor{draculaBackground}{HTML}{282a36}
\definecolor{draculaForeground}{HTML}{F8F8F2}
\definecolor{draculaCurrentLine}{HTML}{44475A}
\definecolor{draculaComment}{HTML}{6272A4}
\definecolor{draculaRed}{HTML}{FF5555}
\definecolor{draculaOrange}{HTML}{FFB86C}
\definecolor{draculaYellow}{HTML}{F1FA8C}
\definecolor{draculaGreen}{HTML}{50FA7B}
\definecolor{draculaPurple}{HTML}{BD93F9}
\definecolor{draculaPink}{HTML}{FF79C6}
\definecolor{draculaCyan}{HTML}{8BE9FD}

\newtcblisting{dracpython}{
    listing engine=minted,
    minted language=python,
    listing only,
    minted options={
        linenos, 
        breaklines, 
        fontsize=\small
    },
    enhanced,
    breakable,
    colback=draculaBackground, % Box background matching Dracula theme
    coltext=draculaForeground, % Text color matching Dracula theme
    frame hidden,
    % interior hidden,
    overlay={
      % Draw a custom frame with a Dracula accent color
      \draw[draculaPurple, line width=2pt]
        ([xshift=5pt,yshift=-5pt]frame.south west)
        rectangle
        ([xshift=-5pt,yshift=5pt]frame.north east);
    },
}

\begin{document}

\begin{minted}{python}
def foo():
    return 0
\end{minted}

\begin{minted}[mathescape, linenos, numbersep=5pt, gobble=2, frame=lines, framesep=2mm]{csharp}
  string title = "This is a Unicode pi in the sky"
  /*
  Defined as $\pi=\lim_{n\to\infty}\frac{P_n}{d}$ where $P$ is the perimeter
  of an $n$-sided regular polygon circumscribing a
  circle of diameter $d$.
  */
  const double pi = 3.1415926535
\end{minted}
  
\begin{minted}[mathescape, breaklines]{py}
# $\sum_{i = 1}^{\infty}i + 1$
def foo():
    return 0
\end{minted}
  
\begin{minted}[breaklines, breakautoindent=false, breaksymbolleft=\raisebox{0.8ex}{\small\reflectbox{\carriagereturn}}, breaksymbolindentleft=0pt, breaksymbolsepleft=0pt, breaksymbolright=\small\carriagereturn, breaksymbolindentright=0pt, breaksymbolsepright=0pt]{python}
def f(x):
    return 'Some text ' + str(x) + 
    ' some more text ' + str(x) + 
        ' even more text that goes on for a while and a while longer, and so on forever and ever.'
\end{minted}
  
\begin{minted}{py}
def f(x):
    y = x|\colorbox{green}{**}|2
return y
\end{minted}

\begin{minted}[breaklines, breakbefore=A]{python}
    some_string = 'SomeTextThatGoesOnAndOnForSoLongThatItCouldNeverFitOnOneLine'
\end{minted}


\begin{minted}[showspaces]{py}
    def boring(args = None):
        pass
\end{minted}

\begin{minted}[gobble=4, showspaces]{py}
    def boring(args = None):
        pass
\end{minted}

\begin{minted}[autogobble, showspaces]{py}
            def boring(args = None):
                pass
\end{minted}

\mint[linenos]{perl}|x=~/foo/|


\begin{minted}[linenos]{python}
    def f(x):
        return x**2
\end{minted}
\begin{minted}[linenos]{ruby}
    def func
        puts "message"
    end
    sike = "sike"
\end{minted}
\begin{minted}[linenos, firstnumber=last]{python}
    def g(x):
        return 2*x
\end{minted}

% Other frames options are: none, leftline, rightline, topline, bottomline, lines
\begin{minted}[highlightlines={1,3}, highlightcolor=blue, linenos, frame=lines, framerule=1.5pt, framesep=5mm]{python} 
    def f(x):
        return x**2
    def g(x):
        return 2*x    
\end{minted}

\begin{coolminted}
    def hello():
        print("Hello, world!")
\end{coolminted}

\begin{coolminted}
    def foo():
    # This is a comment that contains math: $\sum_{i=1}^{n} i$.     
       return 0
\end{coolminted}


\renewcommand{\theFancyVerbLine}{
    \sffamily
    \textcolor[rgb]{0.5,0.5,1.0}{
        \scriptsize\oldstylenums{
            \arabic{FancyVerbLine}}}}
\begin{minted}[linenos, firstnumber=11]{python}
    def all(iterable):
        for i in iterable:
            if not i:
                return False
        return True
\end{minted}

% Other options for numbers are: left, right, none, both
\begin{minted}[numbers=both, breaklines=true, samepage=true]{java}
    public boolean isRowValid(TextField textField, int numRows) {
        try {
            int row = Integer.parseInt(textField.getText()) - 1;
            if (row >= 0 && row < numRows) {
                return false;
            } else {
                System.out.println("Invalid row selection. Please choose a valid row.");
                Platform.runLater(textField::clear);
                return true;
            }
        } catch (NumberFormatException e) {
            System.out.println("Invalid input. Please enter a valid integer.");
            Platform.runLater(textField::clear);
            return true;
        }
    }
\end{minted}

\begin{minted}[breaklines=true, samepage=true]{java}
    public static Matrix convertBackToOriginalForm(String[][] matrix) {
        String[][] originalFormMatrix = new String[matrix.length][];
        System.out.println("Matrix length: " + matrix.length);
//        System.out.println("OG Matrix: \n");
        printStringMatrix(matrix);
        for (int i = 0; i < matrix.length; i++) {
            originalFormMatrix[i] = new String[matrix[i].length];
            for (int j = 0; j < matrix[i].length; j++) {
                originalFormMatrix[i][j] = MatrixApp.isFractionMode() ? convertDecimalToFraction(matrix[i][j]) : convertFractionToDecimalString(matrix[i][j]);
            }
        }
        System.out.println("Original Form Matrix: \n");
        return new Matrix(originalFormMatrix);
    }
\end{minted}


\begin{minted}[breaklines=true, numbersep=32pt, numbers=left]{java}
    public static void printStringMatrix(String[][] matrix) {
        for (String[] row : matrix) {
            System.out.println(Arrays.toString(row));
        }
    }
\end{minted}

\begin{dracpython}
    def helloworld():
        print("Hello, Dracula!")
\end{dracpython}

\end{document}