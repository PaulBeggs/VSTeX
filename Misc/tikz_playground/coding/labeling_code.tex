% Author: Inspired by Valeria Borodin and edited with ChatGPT
\documentclass[border={100pt 100pt 150pt 60pt}, % left bottom right top
  svgnames]{standalone} 
\usepackage[svgnames]{xcolor}
\usepackage{tikz}
\usetikzlibrary{shapes,backgrounds,calc,patterns,positioning,fit}
\usepackage[most]{tcolorbox}
\usepgflibrary{shadings}
\tcbuselibrary{skins}
\tcbuselibrary{theorems}
\usepackage{dingbat}
\usepackage{etoolbox}
\usepackage{pgfkeys}
\usepackage{pdftexcmds}
\usepackage{shellesc}
\usepackage{C:/Users/paulb/VSTeX/local/draculatheme}
\usepackage{geometry}


% Load necessary libraries for tcolorbox
\tcbuselibrary{minted, breakable}
\usepackage{minted}
\setminted{escapeinside=||}
\usepackage[framemethod=tikz]{mdframed}
\usemintedstyle{dracula}
\usepackage{fancyvrb}
\renewcommand{\theFancyVerbLine}{\textcolor{draculafg}{\scriptsize \arabic{FancyVerbLine}}}
% Configure the minted appearance
\setminted[cpp]{
    frame=lines,           % draw frame at top and bottom of code block
    tabsize=1,             % tab space width
    linenos,               % display line numbers on the left
    numbers=left,          % position of line numbers
    numbersep=10pt,        % space between line numbers and code
    framesep=3pt,          % space between frame and content
    rulecolor=\color{draculafg}, % color of frame lines
    bgcolor=\color{draculabg}, % background color
    fontfamily=tt,         % use teletype font
    fontsize=\normalsize,  % font size
    breaklines=true,       % allow line breaking
    baselinestretch=1.6,   % line spacing (increased for more space)
    xleftmargin=\parindent, % indentation
}

% Add hooks for better node placement with vertical position control
\newcommand{\nodeanchor}[2]{%
  \tikz[remember picture,overlay,baseline] \coordinate (#1) at (0,0.5ex);%bounding box.center)] \node[inner sep=0pt, outer sep=0pt] (#1) {#2};%
}
\newcommand{\nodeanchortop}[2]{%
  \tikz[remember picture,overlay,baseline] \coordinate (#1) at (0,1.5ex);%bounding box.center)] \node[inner sep=0pt, outer sep=0pt] (#1) {#2};%
}

\newcommand{\boxanchor}[2]{%
  \tikz[remember picture] \node (#1) {#2};%
}

\begin{document}

% The minted environment displays the C++ code
% We use a custom escape command sequence to insert TikZ nodes
% Each \tikz command creates an invisible anchor node that will be used later
\vspace*{-1cm} % Adjust vertical space to fit the page
\begin{tcblisting}{listing engine=minted,
                  listing only,
                  minted language=c++,
                  minted options={escapeinside=||, numbers=left, numbersep=5pt},
                  colback=draculabg,
                  width=.6\linewidth,
                  colframe=draculafg,}
 i|\nodeanchortop{a}{}|nt puis|\nodeanchortop{b}{}|sance (int x,|\nodeanchortop{c}{}| int n) { 

    int i, p = 1; |\nodeanchor{d}{}|           

    for (i = 1; i <= n; i++) {|\boxanchor{top}{}|
        p = p * x; |\nodeanchor{e}{}| 
  |\boxanchor{bottom}{}|} 

    return p; |\nodeanchor{f}{}|  
 }
\end{tcblisting}

% This tikzpicture creates the arrows and labels pointing to parts of the code
% 'remember picture, overlay' allows referencing nodes from the previous listing
% The arrows connect from the label nodes to the anchor nodes created in the code listing

\begin{tikzpicture}[remember picture, overlay,
    every edge/.append style = { ->, thick, >=stealth, draculafg, dashed, line width = 1pt },
    every node/.append style = { align = center, minimum height = 10pt, font = \bfseries, fill= draculared},
    text width = 2.5cm, ]
  \node [above left = .75cm and -.75 cm of a,text width = 2.2cm] (A) {return value type};
  \node [right = 0.25cm of A, text width = 1.9cm] (B) {function name};
  \node [right = 0.5cm of B] (C) {list of formal parameters};
  \node [right = 4.cm of d]  (D) {local variables declaration};
  \node [below = .57cm of D]  (E) {instructions};
  \node [below = .54cm of E]  (F) {instruction \texttt{\bfseries return}};
  
  % Box covering the for loop between nodes (d) and (e)bottom)
  \node[draw, thick, dashed, inner sep=5pt, fit={(top) (bottom)}, fill=none] {};
  
  % Create a coordinate at the edge of the code box
  % Uses calc library to calculate a point on the edge
  \coordinate (edge_point) at ($(e)+(2.6,0)$);
  
  % The following nodes are used to create the arrows with corrected anchors for vertical alignment with corrected anchors for vertical alignment
  \draw (A) edge[out=270, in=90] (a);
  \draw (B) edge[out=270, in=90] (b);
  \draw (C) edge[out=270, in=90] (c);
  \draw (D) edge[out=180, in=0] (d);
  \draw (E) edge[out=180, in=0] (edge_point); % Point to edge coordinate instead of top
  \draw (F) edge[out=180, in=0] (f);
\end{tikzpicture}
 
\end{document}