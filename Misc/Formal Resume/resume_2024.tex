\documentclass[10pt, letterpaper]{article}

% Packages:
\usepackage[
    ignoreheadfoot, % set margins without considering header and footer
    top=.6 in, % seperation between body and page edge from the top
    bottom=.6 in, % seperation between body and page edge from the bottom
    left=.75 in, % seperation between body and page edge from the left
    right=.75 in, % seperation between body and page edge from the right
    footskip=1.0 cm, % seperation between body and footer
    % showframe % for debugging 
]{geometry} % for adjusting page geometry
\usepackage{titlesec} % for customizing section titles
\usepackage{tabularx} % for making tables with fixed width columns
\usepackage{array} % tabularx requires this
\usepackage[dvipsnames]{xcolor} % for coloring text
\definecolor{primaryColor}{RGB}{0, 79, 144} % define primary color
\usepackage{enumitem} % for customizing lists
\usepackage{fontawesome5} % for using icons
\usepackage{amsmath} % for math
\usepackage[
    pdftitle={Paul Beggs's CV},
    pdfauthor={Paul Beggs},
    pdfcreator={LaTeX with RenderCV},
    colorlinks=true,
    urlcolor=primaryColor
]{hyperref} % for links, metadata and bookmarks
\usepackage[pscoord]{eso-pic} % for floating text on the page
\usepackage{calc} % for calculating lengths
\usepackage{bookmark} % for bookmarks
\usepackage{lastpage} % for getting the total number of pages
\usepackage{changepage} % for one column entries (adjustwidth environment)
\usepackage{paracol} % for two and three column entries
\usepackage{ifthen} % for conditional statements
\usepackage{needspace} % for avoiding page brake right after the section title
\usepackage{iftex} % check if engine is pdflatex, xetex or luatex
\usepackage{datetime} % for formatting dates

\newcommand{\MonthYear}{\monthname[\the\month] \the\year} % format date as Month Year

% Ensure that generate pdf is machine readable/ATS parsable:
\ifPDFTeX
    \input{glyphtounicode}
    \pdfgentounicode=1
    % \usepackage[T1]{fontenc} % this breaks sb2nov
    \usepackage[utf8]{inputenc}
    \usepackage{lmodern}
\fi



% Some settings:
\AtBeginEnvironment{adjustwidth}{\partopsep0pt} % remove space before adjustwidth environment
\pagestyle{empty} % no header or footer
\setcounter{secnumdepth}{0} % no section numbering
\setlength{\parindent}{0pt} % no indentation
\setlength{\topskip}{0pt} % no top skip
\setlength{\columnsep}{0cm} % set column seperation
\makeatletter
\let\ps@customFooterStyle\ps@plain % Copy the plain style to customFooterStyle
\patchcmd{\ps@customFooterStyle}{\thepage}{
    \color{gray}\textit{\small Paul Beggs - Page \thepage{} of \pageref*{LastPage}}
}{}{} % replace number by desired string
\makeatother
\pagestyle{customFooterStyle}

\titleformat{\section}{\needspace{4\baselineskip}\bfseries\large}{}{0pt}{}[\vspace{1pt}\titlerule]

\titlespacing{\section}{
    % left space:
    -1pt
}{
    % top space:
    0.3 cm
}{
    % bottom space:
    0.2 cm
} % section title spacing

\renewcommand\labelitemi{\(\circ\)} % custom bullet points
\newenvironment{highlights}{
    \begin{itemize}[
        topsep=0.10 cm,
        parsep=0.10 cm,
        partopsep=0pt,
        itemsep=0pt,
        leftmargin=0.4 cm + 10pt
    ]
}{
    \end{itemize}
} % new environment for highlights

\newenvironment{highlightsforbulletentries}{
    \begin{itemize}[
        topsep=0.10 cm,
        parsep=0.10 cm,
        partopsep=0pt,
        itemsep=0pt,
        leftmargin=10pt
    ]
}{
    \end{itemize}
} % new environment for highlights for bullet entries


\newenvironment{onecolentry}{
    \begin{adjustwidth}{
        0.2 cm + 0.00001 cm
    }{
        0.2 cm + 0.00001 cm
    }
}{
    \end{adjustwidth}
} % new environment for one column entries

\newenvironment{twocolentry}[2][]{
    \onecolentry
    \def\secondColumn{#2}
    \setcolumnwidth{\fill, 4.5 cm}
    \begin{paracol}{2}
}{
    \switchcolumn \raggedleft \secondColumn
    \end{paracol}
    \endonecolentry
} % new environment for two column entries

\newenvironment{header}{
    \setlength{\topsep}{0pt}\par\kern\topsep\centering\linespread{1.5}
}{
    \par\kern\topsep
} % new environment for the header

% \newcommand{\lastupdated}{Last updated in \MonthYear}

% \newcommand{\placelastupdatedtext}{% \placetextbox{<horizontal pos>}{<vertical pos>}{<stuff>}
%   \AddToShipoutPictureFG*{% Add <stuff> to current page foreground
%     \put(
%         \LenToUnit{\paperwidth-2 cm-0.2 cm+0.05cm},
%         \LenToUnit{\paperheight-1.0 cm}
%     ){\vtop{{\null}\makebox[0pt][c]{
%         \small\color{gray}\textit{\lastupdated}\hspace{\widthof{\lastupdated}}
%     }}}%
%   }%
% }%

% save the original href command in a new command:
\let\hrefWithoutArrow\href

% new command for external links:
\renewcommand{\href}[2]{\hrefWithoutArrow{#1}{\ifthenelse{\equal{#2}{}}{ }{#2 }\raisebox{.15ex}{\footnotesize \faExternalLink*}}}


\begin{document}
\newcommand{\AND}{\unskip
    \cleaders\copy\ANDbox\hskip\wd\ANDbox
    \ignorespaces
}
\newsavebox\ANDbox
\sbox\ANDbox{}

% \placelastupdatedtext
\begin{header}
    \textbf{\fontsize{24 pt}{24 pt}\selectfont Paul Beggs}

    \vspace{0.3 cm}

    \normalsize
    \mbox{{\color{black}\footnotesize\faMapMarker*}\hspace*{0.13cm}1600 Washington Ave., Conway, AR}%
    \kern 0.25 cm%
    \AND%
    \kern 0.25 cm%
    \mbox{\hrefWithoutArrow{mailto:paulbeggs03@gmail.com}{\color{black}{\footnotesize\faEnvelope[regular]}\hspace*{0.13cm}paulbeggs03@gmail.com}}%
    \kern 0.25 cm%
    \AND%
    \kern 0.25 cm%
    \mbox{\hrefWithoutArrow{tel:(573)587-9258}{\color{black}{\footnotesize\faPhone*}\hspace*{0.13cm}(573)587-9258}}%
    \kern 0.25 cm%
    \AND%
    \kern 0.25 cm%
    % \mbox{\hrefWithoutArrow{https://yourwebsite.com/}{\color{black}{\footnotesize\faLink}\hspace*{0.13cm}yourwebsite.com}}%
    % \kern 0.25 cm%
    % \AND%
    \kern 0.25 cm%
    \mbox{\hrefWithoutArrow{https://www.linkedin.com/in/paul-beggs/}{\color{black}{\footnotesize\faLinkedinIn}\hspace*{0.13cm}paul-beggs}}%
    \kern 0.25 cm%
    % \AND%
    % \kern 0.25 cm%
    % \mbox{\hrefWithoutArrow{https://github.com/PaulBeggs}{\color{black}{\footnotesize\faGithub}\hspace*{0.13cm}PaulBeggs}}%
\end{header}

\vspace{0.3 cm - 0.3 cm}


% \section{Welcome to RenderCV!}




%     \begin{onecolentry}
%         \href{https://rendercv.com}{RenderCV} is a LaTeX-based CV/resume version-control and maintenance app. It allows you to create a high-quality CV or resume as a PDF file from a YAML file, with \textbf{Markdown syntax support} and \textbf{complete control over the LaTeX code}.
%     \end{onecolentry}

%     \vspace{0.2 cm}

%     \begin{onecolentry}
%         The boilerplate content was inspired by \href{https://github.com/dnl-blkv/mcdowell-cv}{Gayle McDowell}.
%     \end{onecolentry}



% \section{Quick Guide}

% \begin{onecolentry}
%     \begin{highlightsforbulletentries}


%     \item Each section title is arbitrary and each section contains a list of entries.

%     \item There are 7 unique entry types: \textit{BulletEntry}, \textit{TextEntry}, \textit{EducationEntry}, \textit{ExperienceEntry}, \textit{NormalEntry}, \textit{PublicationEntry}, and \textit{OneLineEntry}.

%     \item Select a section title, pick an entry type, and start writing your section!

%     \item \href{https://docs.rendercv.com/user_guide/}{Here}, you can find a comprehensive user guide for RenderCV.


%     \end{highlightsforbulletentries}
% \end{onecolentry}

\section{Education}

% \vspace{0.2cm}

\begin{twocolentry}{
        \textbf{Conway, AR}

        Expected May 2026

    }
    \textbf{Hendrix College}

    \textit{Bachelor of Arts in Mathematics and Psychology}
\end{twocolentry}

\vspace{0.10 cm}

\begin{onecolentry}
    \begin{highlights}
        \item GPA: 3.4/4.0
        \item \textit{Relevant Coursework:} Brain \& Behavior, Cognitive Psychology, and Research Methods.
    \end{highlights}
\end{onecolentry}


% \begin{twocolentry}{
%         \textbf{Cape Girardeau, MO}

%         August 2017 \textendash\ May 2021}
%     \textbf{Notre Dame Regional High School}

%     \textit{Cum Laude}
% \end{twocolentry}

% \vspace{0.10 cm}
% \begin{onecolentry}
%     \begin{highlights}
%         \item GPA: 3.5/4.0
%         \item \textit{Competencies:} College level courses in Calculus, Physics, Biology, Chemistry, and Composition.
%     \end{highlights}
% \end{onecolentry}

\section{Work Experience}

\begin{samepage}

    \begin{twocolentry}{
            \textbf{Cape Girardeau, MO}

            February 2022 \textendash\ Present}
        \textbf{Licensed Pharmaceutical Technician}

        \textit{CVS Health}
    \end{twocolentry}

    \vspace{0.10 cm}
    \begin{onecolentry}
        \begin{highlights}
            \item Work closely with patients that have chronic illnesses to ensure they receive the correct medications.
            \item Learn new medical-specific terminology to effectively communicate with patients and other healthcare professionals.
            \item Advocate on behalf of patients for insurance claims, and work with insurance companies to ensure patients receive the correct medications.
        \end{highlights}
    \end{onecolentry}
\end{samepage}

\section{Certifications \& Training}

\begin{twocolentry}{
        \textbf{Conway, AR}

        August 2024}
    \textbf{Applied Suicide Intervention Skills Training (ASIST)}

    \textit{\href{https://livingworks.net/training/livingworks-asist/}{LivingWorks Education}}
\end{twocolentry}

\vspace{0.10 cm}
\begin{onecolentry}
    \begin{highlights}
        \item Participated in an intensive program specifically focused on intervening for people who are suicidal.
        % \item Practiced the \textit{Pathway for Assisting Life} (PAL) Model in depth:
        % \begin{highlightsforbulletentries}
        %     \item \textbf{Connecting with Suicide:} Notice, evaluate, and act on suicidal behavior. This involved explicitly asking if the person was considering suicide. (With emphasis on using the word ``suicide.'')
        %     \item \textbf{Understanding Choices:} Let them lead the conversation. Engage in empathetic listening, and ascertain the reason(s) for why they are suicidal. Do not assume or interject, always ask clarifying questions, and allow silence. Silence is crucial for self-evaluation of the current conversation.
        %     \item \textbf{Assisting Life:} After listening, analyze the motivation behind their suicidal thoughts. Ask questions, and guide them to identify the factors that are contributing to an overwhelming feeling of hopelessness. Address these feelings. Make a safety plan by establishing attainable goals to work toward a resolve. The goal is not to \textit{solve} their problem, but to keep them safe until you can get them to a long-term professional.
        % \end{highlightsforbulletentries}
    \end{highlights}
\end{onecolentry}



\vspace{0.2 cm}

\begin{twocolentry}{
        \textbf{Conway, AR}

        August 2024}
    \textbf{CPR \& AED Certified}

    \textit{American Red Cross}
\end{twocolentry}

% \vspace{0.10 cm}
% \begin{onecolentry}
%     \begin{highlights}
%         \item Completed training in CPR and AED for adults, children, and infants.
%         \item Proficient in recognizing and responding to life-threatening emergencies.
%         \item Learned how to provide immediate care until advanced medical personnel arrive.
%     \end{highlights}
% \end{onecolentry}


\vspace{0.2 cm}



\begin{twocolentry}{
        \textbf{Conway, AR}

        August 2024}
    \textbf{NARCAN\textsuperscript{\tiny\textregistered} Certified}

    \textit{Addiction Studies at University of Central Arkansas}
\end{twocolentry}

% \vspace{0.10 cm}
% \begin{onecolentry}
%     \begin{highlights}
%         \item Completed training in the administration of NARCAN\textsuperscript{\tiny\textregistered} nasal spray for opioid overdose reversal.
%         \item Learned how to recognize the signs of an opioid overdose and quickly move the person into the recovery position.
%     \end{highlights}
% \end{onecolentry}

\vspace{0.2cm}


\begin{twocolentry}{
        \textbf{Conway, AR}


        August 2024}
    \textbf{Mental Health First Aid}

    \textit{\href{https://www.thenationalcouncil.org/our-work/mental-health-first-aid/}{National Council for Mental Wellbeing}}
\end{twocolentry}

% \vspace{0.10cm}
% \begin{onecolentry}
%     \begin{highlights}
%         \item Trained in recognizing the signs and symptoms of mental health and substance use challenges.
%         \item Learned how to offer initial help and guide a person toward appropriate care.
%         \item Practiced the \textit{ALGEE} action plan.
%     \end{highlights}
% \end{onecolentry}



\section{Volunteer Experience}

\begin{twocolentry}{
        \textbf{Virtual}

        February 2024 \textendash\ July 2024
    }
    \textbf{Sexual Assault Crisis Intervention Training}

    \textit{\href{https://rainn.org/}{RAINN}}


    \vspace{0.10 cm}

\end{twocolentry}

\begin{onecolentry}
    \begin{highlights}
        \item  Completed 80 hours of RAINN training on support techniques for survivors of abuse.
        % \item Practiced active listening, crisis intervention, and de-escalation strategies.
        % \item Learned how to provide resources and referrals to survivors of sexual assault.
    \end{highlights}
\end{onecolentry}

\section{Leadership \& Involvement}

\begin{twocolentry}{
    \textbf{Conway, AR}

    August 2024 \textendash\ Present
}
\textbf{Student Mentor}

\textit{\href{https://www.hendrix.edu/SOS/page.aspx?id=63461}{Student Outreach Additional Recourses (SOAR)}}


\vspace{0.10 cm}


\end{twocolentry}


\vspace{0.10 cm}

\begin{onecolentry}
\begin{highlights}
    \item Mentor students by providing resources, addressing emotional or social concerns, and ensuring their safety.
    \item Facilitate discussions on topics such as loneliness, anxiety, mental health, and suicidal thoughts.
\end{highlights}
\end{onecolentry}



\vspace{0.2 cm}

\begin{twocolentry}{
        \textbf{Conway, AR}

        August 2024 \textendash\ Present
    }

    \textbf{Member}

    \textit{Hendrix College Psychology Club}

    \vspace{0.10 cm}
\end{twocolentry}

% \vspace{0.10 cm}
% \begin{onecolentry}
    % \begin{highlights}
    %     \item Participate in discussions on current research in psychology and mental health.
    %     \item Network with peers and faculty to learn about research opportunities and internships.
    % \end{highlights}
% \end{onecolentry}

\vspace{0.2 cm}

\begin{twocolentry}{
    \textbf{Conway, AR}

    May 2024 \textendash\ Present
}
\textbf{Member}

\textit{\href{https://www.kdp.org/home}{Kappa Delta Pi International Honor Society}}

\end{twocolentry}

\vspace{0.10 cm}

\begin{onecolentry}
\begin{highlights}
    \item Recognition for academic excellence and commitment to education.
\end{highlights}
\end{onecolentry}



\section{Technical Skills}

\begin{onecolentry}
    \begin{highlights}
        \item \textbf{Software:} Microsoft Office Suite, Google Workspace, Adobe Creative Suite, and R.
        % \item \textbf{Languages:} Proficient in Python, R, and Java. Familiar with C, C++, JavaScript, HTML, and CSS.
        \item \textbf{Data Analysis:} Experience with data visualization, statistical analysis, and machine learning.
        \item \textbf{Research:} Proficient in conducting literature reviews, data collection, and data analysis.
    \end{highlights}
\end{onecolentry}
% \section{Academic Projects}

% \begin{twocolentry}
%     {
%         \textbf{Conway, AR}

%         March 2024 \textendash\ July 2024 
%     }
%     \textbf{Department of Mathematics and Computer Science}

%     \textit{Linear Algebra Calculator}

% \end{twocolentry}

% \vspace{0.1cm}
% \begin{highlights}
%     \item Designed and implemented a Java-based application using Scene Builder to aid 2nd-year math students in understanding abstract linear algebra concepts.
%     \item Developed an intuitive graphical user interface (GUI) for performing matrix operations, eigenvalue computations, and vector transformations.
%     \item Focused on enhancing accessibility and engagement for students struggling with complex theoretical material.
%     \item Looking to expand the project to include additional features and functionality.
% \end{highlights}



% \section{Projects}




%     \begin{twocolentry}{


%     \textit{\href{https://github.com/sinaatalay/rendercv}{github.com/name/repo}}}
%         \textbf{Multi-User Drawing Tool}
%     \end{twocolentry}

%     \vspace{0.10 cm}
%     \begin{onecolentry}
%         \begin{highlights}
%             \item Developed an electronic classroom where multiple users can simultaneously view and draw on a "chalkboard" with each person's edits synchronized
%             \item Tools Used: C++, MFC
%         \end{highlights}
%     \end{onecolentry}


%     \vspace{0.2 cm}

%     \begin{twocolentry}{


%     \textit{\href{https://github.com/sinaatalay/rendercv}{github.com/name/repo}}}
%         \textbf{Synchronized Desktop Calendar}
%     \end{twocolentry}

%     \vspace{0.10 cm}
%     \begin{onecolentry}
%         \begin{highlights}
%             \item Developed a desktop calendar with globally shared and synchronized calendars, allowing users to schedule meetings with other users
%             \item Tools Used: C\#, .NET, SQL, XML
%         \end{highlights}
%     \end{onecolentry}


%     \vspace{0.2 cm}

%     \begin{twocolentry}{


%     \textit{2002}}
%         \textbf{Custom Operating System}
%     \end{twocolentry}

%     \vspace{0.10 cm}
%     \begin{onecolentry}
%         \begin{highlights}
%             \item Built a UNIX-style OS with a scheduler, file system, text editor, and calculator
%             \item Tools Used: C
%         \end{highlights}
%     \end{onecolentry}




% \section{Technologies}




%     \begin{onecolentry}
%         \textbf{Languages:} C++, C, Java, Objective-C, C\#, SQL, JavaScript
%     \end{onecolentry}

%     \vspace{0.2 cm}

%     \begin{onecolentry}
%         \textbf{Technologies:} .NET, Microsoft SQL Server, XCode, Interface Builder
%     \end{onecolentry}




\end{document}