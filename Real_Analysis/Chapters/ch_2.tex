\textbf{What is a \textit{sequence}?} \\

A countable, ordered list of elements. An example could be $1,2,3,4,5,\dots$. Note that this is \textit{ordered}, therefore distinguishing it from a sequence like $3,1,2,4,5,6,\dots$. Hence, order matters.

% This needs to be a part of the definition environment
A \textit{sequence} is a function whose domain is $\N$. \textbf{Note:} The domain $\N$ refers to each element's position in the list. For example, $(a_n) = a_1,a_2,a_3,\dots$.

We will focus on the \textit{limit} of a sequence. We use sequences to approximate other things.

\begin{example}
    {Sequence}$3,3.1,3.14,3.141,3.1415,\dots \approx \pi$. \\
    $x, x - \frac{x^3}{6}, x - \frac{x^3}{6} + \frac{x^5}{120}, x - \frac{x^3}{6} + \frac{x^5}{120} - \frac{x^7}{7!}, \dots \approx \sin(x)$
\end{example}

\noindent\textbf{What is a \textit{series}?} \\

An infinite sum. We look at the sequence of partial sums. We ask, do the partial sums approach a limit?

\begin{example}
    {Alternating Harmonic Series}$1 - 1/2 + 1/3 - 1/4 + 1/5 - 1/6 + \dots$
\end{example}

We can rearrange these terms such that we can `force' the series to converge to a specific number. Therefore, we will need to be careful with our definitions.

\section{Discussion: Rearrangement of Infinite Series}

\noindent \textbf{Questions:}

\begin{itemize}
    \item What happens if I add these in a different order?
    \item Are infinite sums commutative? \\
          In general, no. But sometimes, yes.
\end{itemize}

\section{The Limit of a Sequence}

\begin{definition}
    A \textit{sequence} is a function whose domain is $\N$. We write $(a_n) = a_1,a_2,a_3,\dots$.
\end{definition}
\setcounter{BoxCounter}{2}
\begin{definition}
    The sequence $(a_n)$ converges to $L$ if for all $\epsilon > 0$, there exists $N \in \N$ such that for all $n \geq \N$, $|a_n - L| < \epsilon$. In other words, there exists $N \in \N$ such that
    \begin{itemize}
        \item \textbf{(In the interval)} $a_N \in (L - \epsilon, L + \epsilon)$.
        \item \textbf{(Stays in the interval)} $\forall n \geq N$, $a_n \in (L - \epsilon, L + \epsilon)$.
    \end{itemize}
\end{definition}

\begin{example}
    {In-class}Let $a_n = \frac{1}{n}$. $(a_n) = (1,\frac{1}{2},\frac{1}{3},\dots$)
\end{example}

\epf{
    Our claim is $\displaystyle \lim_{1 \rightarrow \infty} \frac{1}{n} = 0$. Thus, let $\epsilon = .01$. Does the sequence eventually get inside $(-.01,.01)$? We will set $N = 101$. So, for any $n \geq |0|$, $$\left|\frac{1}{n} - 0\right| = \frac{1}{n} \leq \frac{1}{101} < .01.$$ From $A_n$ and on, the sequence stayed within $\epsilon$ of $0$. But what about $\epsilon = .001$, $\epsilon = .00001$ and so on? \\

    Actual proof let $\epsilon > 0$. By the Archimedean Principle, there exists $N \in \N$ such that $N > \frac{1}{\epsilon}$. Now, for any $n \geq N$, $$\left|\frac{1}{n}\right| = \frac{1}{n} \leq \frac{1}{N} < \frac{1}{1/\epsilon}.$$ (Where $\frac{1}{1/\epsilon} = \epsilon$, but is in that form for demonstration purposes.) Therefore $\displaystyle \lim_{n \rightarrow \infty} \frac{1}{n} = 0$
}

``To get close'' means is that we are finding a bigger and bigger $N$ as $\epsilon$ gets smaller. Note that the choice of $N$ certainly depends on $\epsilon$.

\subsection{Basic Structure of a Limit Proof}

Claim: $\displaystyle \lim_{n \rightarrow \infty} a_n = L$. \\
Proof: Let $\epsilon > 0$. There exists $N \in \N$ such that \{something involving $\epsilon$\}. Assume $n \geq N$. Then, $$|a-n - L| \ \boxed{\dots} < \epsilon$$ (Where $\boxed{\dots}$ is going to be where the majority of the work is going to lie.

\newpage

\begin{example}
    {In-class}Claim: $\displaystyle \lim_{n \rightarrow \infty} \frac{2n-3}{2n} = 1$
\end{example}

\epf{
    Let $\epsilon > 0$. \textit{Scratch paper:} Solve for: $$\left|\frac{2n - 3}{2n} - 1\right| = \left|\frac{-3}{2n}\right| = \frac{3}{2n} < \epsilon \Rightarrow \frac{3}{2\epsilon} < n.$$ By the Archimedean Principle, there exists $N \in \N$ such that $N > \frac{3}{2\epsilon}$. Assume $n \geq N$, (want to know what happens past this point) $$\left|\frac{2n-3}{2n}-1\right| \leq \frac{3}{2N} < \frac{3}{3/2\epsilon} = \epsilon.$$ Therefore, $\displaystyle \lim_{n \rightarrow \infty} \frac{2n-3}{2n} = 1$
}

\begin{example}
    Claim: $\limn{\frac{2n^2 + 1}{n^2}} = 2$
\end{example}

\epf{
    Let $\epsilon > 0$. By the Archimedean Principle, there exists $N \in \N$ such that [leave off] \textit{Scratch paper:} Solve for $$\disabs{\frac{2n^2 + 1}{n^2} - 2} =  \frac{2n^2}{n^2} < \epsilon \Rightarrow \frac{3}{2\epsilon} < n$$ [pick up] there exists $N \in \N$ such that $$N > \frac{1}{\sqrt{\epsilon}}.$$ Assume $n \geq N$, then
    \begin{align*}
        \disabs{\frac{2n^2 + 1}{n^2} - 2} & = \frac{1}{n^2}                    \\
                                          & \leq \frac{1}{N^2}                 \\
                                          & < \frac{1}{(1/(\sqrt{\epsilon})^2} \\
                                          & = \frac{1}{1/\epsilon}             \\
                                          & = \epsilon
    \end{align*}
    Therefore, $\limn{\frac{2n^2 + 1}{n^2}} = 2$
}

\begin{example}
    {In-class}Claim: $\limn{\frac{7n + 8}{3n + 6}} = \frac{7}{3}$
\end{example}

\epf{
    \begin{align*}
        \disabs{\frac{7n+8}{3n+6} - \frac{7}{3}} & = \disabs{\frac{21n + 24}{3(3n + 6)} - \frac{21n + 42}{3(3n + 6}} \\
                                                 & = \disabs{\frac{-18}{9n + 18}}                                    \\
                                                 & = \frac{18}{9n + 18} < \epsilon**                                 \\
                                                 & = \frac{18}{3} < 9n + 18                                          \\
                                                 & = \frac{18}{3} - 18 < 9n                                          \\
                                                 & = \frac{18/\epsilon - 18}{9} < n
    \end{align*}
    $** \frac{18}{9n + 8} < \frac{18}{9n} < \epsilon \Rightarrow \frac{2}{\epsilon} < N$. $\exists N \in \N$ such that $N > \frac{2}{\epsilon}$. Assume $n \geq N$, \begin{align*}
        \disabs{\frac{7n + 8}{3n + 6} - \frac{7}{3}} & = \frac{18}{9n + 18}   \\
                                                     & = \frac{2}{n+2}        \\
                                                     & < \frac{2}{n}          \\
                                                     & \leq \frac{2}{N}       \\
                                                     & < \frac{2}{\epsilon/2} \\
                                                     & = \epsilon
    \end{align*}
}

\textbf{Does every sequence have a limit?}

\begin{ntheorem}
    {Uniqueness of Limits}The limit when it exists, is unique.
\end{ntheorem}


\tpf{
    Let $(x_n)$ be a convergent sequence. Suppose $L$ and $M$ are limits of this sequence. Without the loss of generality, we are going to assume $M > L$ Let $$\epsilon = \frac{M - L}{3}.$$ Since $n_x$ converges to $L$, there exists $N_1 \in \N$ such that for all $n \geq N$, $\disabs{(x_n) - L} < \epsilon$. Since $(x_n)$ converges to $M$, there exists an $N_2 \in \N$ such that for all $n \geq N_2$, $\disabs{(x_n) - M} < \epsilon$. Consider $n = \max\{N_1, N_2\}$. Since $n \geq N_1$, $\disabs{(x_n) - L} < \epsilon$. Since $n \geq N_2$, $\disabs{(x_n) - M} < \epsilon$. Then $L - \epsilon < x_n < L + \epsilon$ and $M - \epsilon < x_n < M + \epsilon$. By our choice of $\epsilon$, we now have $$(x_n) < L + \epsilon < M - \epsilon < (x_n).$$ This is a contradiction. Thus, $(x_n)$ cannot have two different limits.
}

\begin{example}
    {}Let $(x_n) = \displaystyle \frac{\cos(n)}{3n}$. Claim: $\limn{(x_n)} = 0$
\end{example}

\epf{
    Let $\epsilon > 0$. By the \namrefthm{Archimedean Principle}, there exists $N \in \N$ such that $N > \frac{1}{3\epsilon}$ for all $n \geq N$,
    \begin{align*}
        \disabs{\frac{\cos(n)}{3n} - 0} & = \disabs{\frac{\cos(n)}{3n}} \\
                                        & \leq \frac{1}{3n}             \\
                                        & \leq \frac{1}{3N}             \\
                                        & < \frac{1}{3(1/ 3\epsilon)}   \\
                                        & = \epsilon
    \end{align*}
}

\begin{example}
    {}Let $(y_n) = \frac{4n-1}{n^2}$. Claim: $\limn{y_n} = 0$.
\end{example}

\epf{
    Let $\epsilon > 0$. By the Archimedean Principle, there exists $N \in \N$ such that $N > \frac{1}{\epsilon}$. For all $n \geq N$,
    \begin{align*}
        \disabs{\frac{4n-1}{n^2} - 0} & = \disabs{\frac{4n-1}{n^2}} \\
                                      & = \frac{4n-1}{n}            \\
                                      & < \frac{4n}{n^2}            \\
                                      & = \frac{4}{n}               \\
                                      & \leq \frac{4}{N}            \\
                                      & < \frac{4}{4 / \epsilon}    \\
                                      & = \epsilon
    \end{align*}
}

\section*{Exercises}

\begin{exercise}
    {2.2.2(b)}Verify, using \defref{2.2.3}, that
    the following sequences converge to the proposed limit.
    \begin{enumerate}
        \setcounter{enumi}{1}
        \item \(\limn{\frac{2n^2}{n^3+3}} = 0\)
    \end{enumerate}
\end{exercise}

\expf{\hfill
    \begin{enumerate}
        \setcounter{enumi}{1}
        \item Let \(\epsilon > 0\). Choose \(N = \frac{2}{\epsilon}\). Then, for \(n \geq N\),
              \begin{align*}
                  \disabs{\frac{2n^2}{n^3 + 3} - 0} & = \disabs{\frac{2n^2}{n^3 + 3}} \\
                                                    & = \frac{2n^2}{n^3 + 3}          \\
                                                    & < \frac{2n^2}{n^3}              \\
                                                    & = \frac{2}{n}                   \\
                                                    & \leq \frac{2}{N}                   \\
                                                    & = \frac{2}{2/\epsilon}          \\
                                                    & = \epsilon.
              \end{align*}
        Therefore, \(\limn{\frac{2n^2}{n^3 + 3}} = 0\). \qedhere
    \end{enumerate}
}

\begin{exercise}
    {2.2.3} Describe what we would have to demonstrate in order to disprove each of the following statements.
    \begin{enumerate}
        \item At every college in the United States, there is a student who is at least
              seven feet tall.
        \item For all colleges in the United States, there exists a professor who gives every student a grade of either A or B.
        \item There exists a college in the United States where every student is at least six feet tall.
    \end{enumerate}
\end{exercise}

\sol{
    \begin{enumerate}
        \item There is at least one college in the United States where all students are less than seven feet tall.
        \item There is at least one college in the United States where all professors give at least one student a grade of C or lower.
        \item For all colleges in the United States, there exists a student who is less than six feet tall.
    \end{enumerate}
}

\begin{exercise}
    {2.2.4} Give an example of each or state that the request is impossible.
    For any that are impossible, give a compelling argument for why that is the case.
    \begin{enumerate}
        \item A sequence with an infinite number of ones that converges to one.
        \item A sequence with an infinite number of ones that converges to a limit not equal to one.
        \item A divergent sequence such such that for every \(n \in \N\) it is possible to find \(n\) consecutive ones somewhere in the sequence.
    \end{enumerate}
\end{exercise}

\sol{
    \begin{enumerate}
        \item Possible. Example: \(a_n = 1 \text{ for all } n \in \N\).
        \item Impossible. We can use the \namrefthm{Uniqueness of Limits} to show that this is impossible. Essentially, because we found in (a) that the limit of an infinite number of ones converges to 1, it must be the case that if the limit is not equal to one, then the sequence cannot converge to one.
        \item Possible. Example: \(\{2,2,2,2,4,4,4,4,1,1,1,1,1,3,3,\dots\}\)
    \end{enumerate}
}

\section{The Algebraic and Order Limit Theorems}

\begin{definition}
    A sequence $(x_n)$ is \textit{bounded} if there exists some $M > 0$ such that every term in the sequence belongs to $[-M,M]$.
\end{definition}

\begin{theorem}
    Every convergent sequence is bounded.
\end{theorem}

\tpf{
Let $(x_n)$ be a convergent sequence with limit $L$. There exists $N \in \N$ such that for all $n \geq N$, $\disabs{(x_n) - L} < 1$. Equivalently, $(x_n) \in (L - 1, L + 1)$. Let $$M = \max\{|x_1|,|x_2|,\dots,|x_{N-1}|,|L + 1|, |L - 1|\}.$$ We claim that for all $n \in \N$, $|x_n| \leq M$.
\begin{enumerate}
    \item This is true for $n < N$.
    \item For $n \geq N$, we know $L - 1 < x_n < L + 1$, so $(x_n) \leq \max\{| L - 1 |, |L + 1|\}$
\end{enumerate}
Thus, every term is in $[-M,M]$.
}

\begin{ntheorem}
    {Algebraic Limit Theorem}Let \(\limn{a_n} = a\) and \(\limn{b_n} = b\). Then,
    \begin{enumerate}[label=(\roman*)]
        \item \(\limn{ca_n} = ca\) for all $c \in \R$;
        \item \(\limn{(a_n + b_n)} = a + b\);
        \item \(\limn{(a_nb_n)} = ab\);
        \item \(\limn{\frac{a_n}{b_n}} = \frac{a}{b}\) provided $b \neq 0$.
    \end{enumerate}
\end{ntheorem}

\textit{Scratch Paper:}
\begin{align*}
    \disabs{ca_n - ca} = \disabs{C}\disabs{a_n - a} & < \epsilon                    \\
    \disabs{a_n - a}                                & < \frac{\epsilon}{\disabs{c}}
\end{align*}
\text{Leave off and go back to proof$^1$}
\tpf{(i) \\
Let $\epsilon > 0$.$^1$ Since $(a_n)$ converges to $a$, there exists $N \in \N$ such that for all $n \geq N$, $\disabs{a_n - a} < \frac{\epsilon}{\disabs{c}}$. Now, for any $n \geq N$ we have two case because we want to avoid dividing by 0:
\begin{itemize}
    \item If $c = 0$: \\
          then each $ca_n = 0$. So $(ca_n)$ converges to 0, which can equal $ca$.
    \item If $c > 0$: \\
          $\disabs{ca_n - ca} = \disabs{c}\disabs{a_n - a} < \disabs{c}\frac{\epsilon}{\disabs{c}} = \epsilon$. \qedhere
\end{itemize}

(ii) \\
\textit{Scratch paper:}
\begin{align}
    \disabs{(a_n + b_n)} & = \disabs{(a_n - a) + (b_n - b)}          \\
                         & \leq \disabs{a_n - a} + \disabs{b_n - b}  \\
                         & < \frac{\epsilon}{2} + \frac{\epsilon}{2}
\end{align}
Note that (2.2) is from the triangle inequality. Now, we will pick up to back at $\epsilon > 0$. \\

Let \(\epsilon > 0\). Since $(a_n)$ converges to $a$, there exists $N_1 \in \N$ such that for all $n \geq N_1$, $\disabs{a_n - a} < \frac{\epsilon}{2}$. Since $(b_n)$ converges to $b$, there exists $N_2 \in \N$ such that for all $n \geq N_2$, $\disabs{b_n - b} < \frac{\epsilon}{2}$. Now, let $N = \max\{N_1, N_2\}$. Thus, for any $n \geq N$, (refer back to scratch paper). \\

(iii) \\
\textit{Scratch paper:}
\begin{align}
    \disabs{a_nb_n - ab} & = \disabs{a_nb_n - ab_n + ab_n - ab}                                    \\
                         & = \disabs{a_n(b_n - b) + b(b_n - b)}                                    \\
                         & \leq \disabs{a_n}\disabs{b_n - b} + \disabs{b}\disabs{b_n - b}          \\
                         & \leq M\disabs{b_n - b} + M\disabs{a_n - a}.                             \\
                         & < M\left(\frac{\epsilon}{2M}\right) + M\left(\frac{\epsilon}{2M}\right) \\
                         & = \epsilon
\end{align}
Note that: (2.4) is where we added 0, (2.5) is from the triangle inequality, and (2.6) is just factored. Additionally, we choose \(N\) to get the fractions in (2.8) Now, we will pick up to back at $\epsilon > 0$. \\

Let \(\epsilon > 0\). Since convergent sequences are bounded, then there exists $M > 0$ such that for all $n \in \N$, $|a_n| \leq M$. We can choose \(M\) so that \(|b_n| \leq M\) as well. Since $(a_n)$ converges to $a$, there exists $N_1 \in \N$ such that for all $n \geq N_1$, $\disabs{a_n - a} < \frac{\epsilon}{2M}$. Since $(b_n)$ converges to $b$, there exists $N_2 \in \N$ such that for all $n \geq N_2$, $\disabs{b_n - b} < \frac{\epsilon}{2M}$. Now, let $N = \max\{N_1, N_2\}$. Thus, for any $n \geq N$, (refer back to scratch paper, and change (2.4)'s sign from an `=' to `$\leq$'). \\

(iv) \\
\textit{Scratch paper:}
\begin{align*}
    \disabs{\frac{a_n}{b_n} - \frac{a}{b}} & = \disabs{\frac{a_nb - ab_n}{b_nb}}                                                     \\
                                           & = \disabs{\frac{a_nb - ab_n + ab_n - ab}{b_nb}}                                         \\
                                           & = \disabs{\frac{a_n(b - b_n) + b(b_n - b)}{b_nb}}                                       \\
                                           & = \disabs{\frac{a_n(b - b_n)}{b_nb} + \frac{b(b_n - b)}{b_nb}}                          \\
                                           & \leq \disabs{\frac{a_n}{b_n}}\disabs{b - b_n} + \disabs{b}\disabs{\frac{b_n - b}{b_nb}} \\
                                           & < \epsilon
\end{align*}
Let \(\epsilon > 0\). Since \((b_n)\) converges to \(b\), there exists \(N_1 \in \N\) such that for all \(n \geq N_1\), \(\disabs{b_n } > \disabs{\frac{b}{2}}\). There also exists \(N_2 \in \N\) such that for all \(n \geq N_2\), \(\disabs{b_n - b} < \frac{\epsilon|b|^{2}}{2}\). Now, let \(N = \max\{N_1, N_2\}\). Let \(n \geq N\), (refer back to scratch paper). \qedhere
}

\begin{lemma}
    Let \((a_n)\) and \(c < a\). there exists \(N \in \N\) such that for all \(n \geq N\), \(a_n > c\). Similarly, if \(a < d\), there exists \(N \in \N\) such that for all \(n \geq N\), \(a_n < d\).
\end{lemma}

\subsection{Limits and Order}

\begin{ntheorem}
    {Order Limit Theorem}Let \((a_n)\) and \((b_n)\) be sequences. If \(\limn{a_n} = a\) and \(\limn{b_n} = b\), then
    \begin{enumerate}[label=(\roman*)]
        \item If \(a_n \geq c\) for all \(n \in \N\), then \(a \geq c\).
        \item If \(a_n \leq b_n\) for all \(n \in \N\), then \(a \leq b\).
        \item If \(a_n \leq b_n\) for all \(n \in \N\), then \(a \leq b\).
    \end{enumerate}
\end{ntheorem}



\section*{Exercises}

\begin{exercise}
    {2.3.1}
    \begin{enumerate}
        \item Assume \(\limn{x_n = 0}\) with \(x_n \geq 0\). Show that \(\limn{\sqrt{x_n}} = 0\).
        \item Assume \(\limn{x_n = 49}\) with \(x_n \geq 0\). Show that \(\limn{\sqrt{x_n} = 7}\)
    \end{enumerate}
\end{exercise}

\expf{
    \begin{enumerate}
        \item Let \(\epsilon > 0\). Since \((x_n)\) converges to 0, there exists \(N \in \N\) such that for all \(n \geq N\), \(\disabs{x_n - 0} < \epsilon^{2}\). Now, for any \(n \geq N\), \(\disabs{\sqrt{x_n} - 0} = \sqrt{x_n} < \sqrt{\epsilon^{2}} = \epsilon\). Therefore, \(\limn \sqrt{x_n} = 0\).
        \item Let \(\epsilon > 0\). Since \((x_n)\) converges to 49, there exists \(N \in \N\) such that for all \(n \geq N\),
              \begin{align*}
                  \disabs{x_n - 7} & = \disabs{\frac{(x_n - 7)(x_n + 7)}{\sqrt{x_n} + 7}} \\
                                   & = \disabs{\frac{x_n - 49}{\sqrt{x_n} + 7}}           \\
                                   & \leq \frac{\disabs{x_n - 49}}{7}
              \end{align*}
    \end{enumerate}
}

\begin{exercise}
    {2.3.2}Using only \defref{2.2.3}, prove that if $(x_n) \rightarrow 2$, then
    \begin{enumerate}
        \item $\displaystyle \left(\frac{2x_n - 1}{3}\right) \rightarrow 1$;
        \item $\displaystyle \left(1 / x_n\right) \rightarrow 1 / 2$.
    \end{enumerate}
    (For this exercise the Algebraic Limit Theorem is off-limits, so to speak.)
\end{exercise}

\sol{ \hfill
    \begin{enumerate}
        \item \textit{Proof.} Let $\epsilon > 0$. Since $(x_n)$ converges to 2, there exists $N \in \N$ such that for all $n \geq N$, $\disabs{x_n - 2} < \epsilon$. Now, for any $n \geq N$,
              \begin{align*}
                  \disabs{\frac{2x_n - 1}{3} - 1} & = \disabs{\frac{2x_n - 1 - 3}{3}} \\
                                                 & = \disabs{\frac{2x_n - 4}{3}}      \\
                                                 & = \frac{2}{3}\disabs{x_n - 2}      \\
                                                 & < \disabs{x_n - 2} \\
                                                    & < \epsilon
              \end{align*} 
              Therefore, \(\frac{2x_n - 1}{3} \rightarrow 1\) \qed
        \item \textit{Proof.} Let $\epsilon > 0$. We will choose \(N_1\) so that \(x_n - < 1\) for all \(n \geq N_1\). Then, \(|x_n| > 1\) for all \(n \geq N_1\), so \(\frac{1}{x_n} < 1\). We will choose \(N_2\) so that \(|x_n - 2 < 2\epsilon\) for all \(n \geq N\). Then take \(N = \max\{N_1, N_2\}\). And note that for \(n \geq N\), 

              \begin{align*}
                  \frac{\disabs{x_n - 2}}{2|x_n|} & < \frac{2\epsilon}{2} \\
                  & = \epsilon
              \end{align*} \qed
    \end{enumerate}
}

\section{The Monotone Convergence Theorem and a First Look at Infinite Series}

\begin{definition}
    A sequence \(a_n\) is \textit{increasing} if \(a_n \leq a_{n+1}\) for all \(n \in \N\) and \textit{decreasing} if \(a_n \geq a_{n+1}\) for all \(n \in \N\). A sequence is \textit{monotone} if it is either increasing or decreasing.
\end{definition}

\begin{ntheorem}
    {Monotone Convergence Theorem}If a sequence is monotone and bounded, then it converges.
\end{ntheorem}

\tpf{
    Let \((a_n)\) be an increasing and bounded sequence. Since \((a_n)\) is bounded, the set \(A = \{a_n \ |\ n \in \N\}\) is clearly also bounded. Since \(A\) is bounded, \(\sup{A}\) exists. We claim that \(\limn{a_n} = \sup{A}\). Thus, for all \(\epsilon > 0\) and by our definition of supremum, there exists \(N \in \N\) such that \(\sup{A} - \epsilon < a_N \leq \sup{A}\). Since \((a_n)\) is increasing, for all \(n \geq N\), \(\sup{A} - \epsilon < a_N \leq a_n \leq \sup{A}\). It follows that \(\disabs{a_n - \sup{A}} < \epsilon\). Therefore, \(\limn{a_n} = \sup{A}\). 
}

\begin{example}
    {MCT}Consider the recursively defined sequence \(x_n\) where \(x_1 = 3\) and for all \(n \in \N\), \(x_{n + 1} = \frac{1}{4 - x_n}\). Show that \(x_n\) converges.
\end{example}

\epf{
    We will show that \(x_n\) is monotone and bounded. 
    \begin{itemize}
        \item \textbf{Part 1: Monotone Decreasing}
        \begin{itemize}
        \item \underline{Base case}: \(x_1 = 3\), \(x_2 = 1\). 
        \item \underline{Induction step}: Assume for some \(n \in \N\), \(x_n \geq x_{n+1}\). It follows that 
        \begin{align*}
            x_n & \geq x_{n + 1} \\
            4 - x_n & \leq 4 - x_{n + 1} \\
            \frac{1}{4 - x_n} & \geq \frac{1}{4 - x_{n + 1}} \\
            x_{n + 1} & \geq x_{n + 2}
        \end{align*}
        \end{itemize}
    \item \textbf{Part 2: Bounded Below}
    Claim: Sequence is bounded below by 0.
    \begin{itemize}
        \item \underline{Base case}: \(x_1 = 3 > 0\).
        \item \underline{Induction step}: Assume for some \(n \in \N\), \(x_n \geq 0\). It follows that \(4 - x_n \leq 4\), and when we take the reciprocal, we get \begin{align*}
               \frac{1}{4 - x_n} &\leq \frac{1}{4} \\
               x_{n+1} &\geq 1/4 \\
               &> 0
        \end{align*}
        By math induction, \(x_n\) is bounded below by 0. 
    \end{itemize}
    \end{itemize}
    By the Monotone Convergence Theorem, \(x_n\) converges.

    So, what is the limit? We know \((x_n)\) converges so let \(L = \limn{x_n}\). Then, \(\limn{x_{n+1}} = L\). We also know \(x_{n+1} = \frac{1}{4 - x_n}\). So \(L = \limn{x_{n+1}} = \limn{\frac{1}{4- x_n} = \frac{1}{4 - L}}\). It must be true that \(L = \frac{1}{4 - L}\). Solving for \(L\), we get \begin{align*}
    L(4 - L) &= 1 \\
    4L - L^2 &= 1 \\
    L^2 - 4L + 1 &= 0
\end{align*} Hence, \(L = 2 - \sqrt{3}\) or \(L = 2 + \sqrt{3}\). Notice that it cannot be the latter because it is bigger than 3.
}

