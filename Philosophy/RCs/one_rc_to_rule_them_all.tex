\documentclass[11pt]{article}
\usepackage{draculatheme}
\usepackage{amsmath}
\usepackage{amssymb}
\usepackage{enumitem}
\usepackage{geometry}
\usepackage{hyperref}
\linespread{1}

\begin{document}

\setcounter{section}{1}
\section{RC: Wittgenstein Private Language Argument}

To explain the private language argument, we first must review what Wittgenstein calls ``language-games.'' These language-games are more of a conceptual tool for each person. That is, the ``game'' aspect is regarding the different intentions of one's grammar. (Put simply, we should not assign meaning to language used by other people. Their language is part of a different whole—of which, we are unaware.) It could be the case that when one yells out in pain, they could be deceitful, and we would be none-the-wiser on if we knew that person was in pain, or just lying. Moreover, if we examine the case where that person is in their own type of pain (epistemological approach), we should acknowledge the fact that we are unaware of the specific kind—or feeling—of pain that person may be experiencing.

Thus, we can broach the topic of a skeptic's view on these language-games. That being, what if other bodies simply are. What if a body moves without sensations? Essentially, this view is questioning the foundation of what makes a language-game into a game. What if the other players are not playing? This is where Wittgenstein brings up the private language argument. By asking a series of follow up hypothetical questions based on language itself. If these other bodies are not able to feel sensations, then they should be unable to understand the private language of someone who can feel, understand, label, and recognize those sensations when they appear—that being the private linguist. Through the examination of varying examples Wittgenstein eliminates the practicality of each instance of the use of a private language, as for every example, the criterion needed to support the existence / need of this language are all rooted in language that we the common language speakers understand. He mentions the ostensive definition—that which can be defined by pointing out examples; this approach forces the private linguist to use grammar that everyone can understand, thus mitigating the existence of a private language.

It is difficult to provide objections to Wittgenstein's methodology because throughout the article, he flags counterexamples and rejects them in their tracks. It is almost as if he is systematically ridding the existence of the problem before it has even broken through the dirt (e.g., a murderer cannot murder if s/he is killed as an infant). If I had to conjure up a possible objection, it would be that one could argue in lieu of practicality of whether one is able to label an unidentifiable feeling for themselves, that does not necessarily mean that those around them are capable of understanding what that person may be feeling. 

But this objection is disingenuous because I do not think this is a proper response to Wittgenstein's argument. Fundamentally, the original objection is flawed, and that approach has been systematically gutted; is revival of such a creature even possible? How do you pose an objection to an objection of an objection wherein the second objection to the first objection mitigates any viability for that first objection? 

Therefore, I am sorry, but I will not be able to complete this RC in full. I am out of time and energy trying to conjure up a fledged proof to counter the counter of the counter of the main argument. 

\section{RC: Person's Approach}

Strawson presents a nuanced exploration of how we understand the concept of a person. He introduces a distinction between M-predicates, which describe physical characteristics or locations, and P-predicates, which pertain to mental states or actions suggesting consciousness. Strawson asserts that a person is a single, unified entity to which both types of predicates apply, thereby challenging the Cartesian dualism that separates mind and body. By emphasizing the primitive nature of the concept of a person, he argues that this concept is foundational and cannot be reduced to simpler components such as a mere body or a mind. 

However, I find myself questioning the completeness of Strawson's argument. While Strawson effectively critiques the Cartesian split between mind and body, his criteria for ascribing P-predicates seems to oversimplify the intricacies of consciousness. Mental states like ``is thinking'' or ``believes in God'' often lack direct physical manifestations. Strawson appears to assume that the observable behavior of individuals can sufficiently ground the ascription of such predicates, but this overlooks the subjective, internal experiences that are not always externally visible. In this sense, Strawson's argument may simplify the complexities of mental life to what can be outwardly observed, which seems to overlook the depth and complexity of consciousness. 

Further, Strawson's approach might inadvertently diminish the role of the first-person perspective, which is central to our understanding of consciousness. While he emphasizes the need for logical criteria in the ascription of P-predicates, he may be missing the significance of subjective experience, which cannot always be captured through observable criteria alone. Avramides specifically commented on this with her notion of ``the lived position.'' In that, by acknowledging the philosophical starting ground is too rooted in conceptual solipsism, she pivots to a more generic viewpoint; thereby, circumventing the roadblock and instead, exploring the conception of the lived position. 

If Strawson were to respond to these concerns, he might claim that his focus on observable criteria is not intended to diminish the importance of subjective experience but rather to ensure that discussions of personhood remain grounded. He could posit that while subjective experiences are vital, our ability to meaningfully communicate and understand these experiences relies on the presence of observable behavior that can be recognized and discussed by the public. Strawson might also assert that the integration of M-predicates and P-predicates within a unified concept of a person does not negate the subjective nature of consciousness but rather provides a framework where both the physical and mental aspects of personhood can coexist meaningfully. 

\section{RC: Phenomenological Approach}

According to Merleau-Ponty, when we observe someone—particularly in emotional situations—our thoughts and actions are more strongly tied to perception rather than inference. That is, if we see someone crying in distress, we know they are experiencing that emotion—distress—because we have experienced it ourselves in our bodily experience. Thus, we are capable of directly recognizing that emotion by relying mostly on perception. For a more thorough thesis, Merleau-Ponty believes that through an embodied interaction with the world (e.g., our bodily experiences heavily influence our perception and understanding of the world), we are capable of seeing/recognizing someone's anger in their posture, facial expressions, and so on, because we are embodied beings who interact with others in a shared, lived world. 

From the previous paragraph, one may ask, ``What about those of whom fake their intent as a means of deceit? According to Merleau-Ponty, if we rely mainly on our perception, then by that logic, we should be easily fooled by other's performance of such behaviors.'' In my opinion, this is a great question—at first. While Merleau-Ponty emphasized the importance of bodily gestures—even going on to say that gestures are intrinsically linked to emotion—he clears this up by stating that the process behind those gestures are different. Essentially, for real emotion, the body is genuinely responding to an engagement with the world; whereas, with fake emotion, the body is performing a gesture that is disconnect from that engagement, altogether. Moreover, this dichotomy not only rebuttals the critic's question, but simultaneously pushes against behaviorism (emotions can be reduced to bodily movements) and dualism (emotions are separate from the body).

Where exactly does the distinction lie between Merleau-Ponty and behaviorists' arguments? To answer this, we need to investigate behaviorism and its core philosophy. Essentially, behaviorists argue that emotions can be explained solely through the observation of behaviors (e.g., crying, clenching fists); in that, these behaviors can be reduced to simple labels like `anger,' `happiness,' or `joy,' without needing to consider any possible underlying mental or emotional processes. 
	
This leads us to the divergence in question: While behaviorists argue that emotional states can be fully explained through observation of behavior, Merleau-Ponty posits that emotions cannot be reduced to mere physical behavior. Thus, while both may acknowledge the notion that expressions and gestures are manifestations of emotions, Merleau-Ponty is explicitly concentrated on the intention behind these behaviors. That is, gestures are not confined solely to bodily reactions; they are meaningful actions that are deeply embedded in the individual's broader engagement with the world. In contrast to the behaviorist reduction of emotions to observable behaviors, Merleau-Ponty sees these gestures as inseparable from the person's lived experience and interaction with their environment.\footnote{While this distinction alone may not fully dispel the claim that Merleau-Ponty could be mistaken for a behaviorist, it provides a critical foundation for understanding the deeper philosophical difference between them.}

For dualism, it is said that the body and mind are divided and serve different purposes. The mind—thoughts, feelings, emotions—is believed to be separate from the physical body—gestures, movements, actions. Essentially, mental states exist independently of bodily expressions and are instead contained within the mind. Merleau-Ponty refutes this view because it separates mind from body. According to him, emotions are not internal experiences that are simply reflected in the body, but rather, they are lived, embodied experiences that manifest through the body in interaction with the world.

\section{RC: The Intentional Stance, Dennett}

Dennett's intentional strategy is a method for predicting and explaining behavior by attributing intentionality—such as beliefs, desires, and rationality—to what he calls, rational agents. In his explanation, he contrasts two different strategies that people take up for explaining the nature of belief. These are the physical and design stances. In essence, the physical stance looks at mechanisms that determine behavior (e.g., physiological psychology or body movements), and the design stance assumes behaviors are a consequence of intentional design (e.g., computer hardware); whereas the intentional strategy implements both aspects of the previous stances. In that, we observe rational agents through both an objective lens (realism) and a ``predictive strategy'' (intentionalism). For instance, when watching someone run towards an elevator while the doors are closing, we can predict their behavior by assuming they want on the elevator and that the elevator is about to close. Thus, Dennett's theory provides a tool for interpreting (complex) systems, such as humans, animals, machines, and even plain objects. 

However, one may question the extent to which the intentional strategy attributes beliefs and desires to systems that may not possess consciousness. For example, when Dennett applies the intentional stance to clams and thermostats, it may oversimply the difference between human consciousness and other systems. One could argue that even if these attributions do help in prediction, there would still be a risk of anthropomorphizing these entities that do not possess the same cognitive functionalities that humans do. If we extend this rationale to something such as a specialized machine (like the chess robot that was mentioned), it may nullify the complexity of human consciousness to the likes of simple preprogrammed rules. For a more modern example, one could argue that a generative pre-trained transformer such as ChatGPT has a consciousness, but this would be facetious because ChatGPT is merely an advanced text correction algorithm that is constrained by weights that have been established by humans. In such a case, the intentional stance might lead us to make assumptions about behavior that are not grounded in mental processes.

As a response to this point, Dennett might point out that the intentional strategy is not an attempt to prove that animals or machines literally possess human-like consciousness but rather to provide a practical tool for understanding their behavior. Thus, the claim can be made that the intentional strategy should be used as a tool, not philosophical purity. Moreover, the aim is not to attribute consciousness where it does not exist, but rather, to deconstruct complicated (sometimes incomprehensible) topic into smaller, digestible pieces. He could argue that treating a machine or animal as if it had beliefs and desires would give practical insight to how they may behave, and that is its purpose. While we cannot observe consciousness directly, we can make use of the intentional strategy to predict human behavior. Hence, for this reason, the strategy is justified in its use because of this predictive power, even if it does not offer a full account of the inner workings of consciousness in non-human systems.

\section{RC: Theory Theory -- Ravenscroft and Hutto}
    
In Daniel Hutto and Ian Ravenscroft's article, ``Folk Psychology as a Theory,'' the authors investigate the Theory-theory approach to folk psychology, which posits that humans use a theoretical approach to infer the mental states of others, such as beliefs, desires, and intentions to predict behavior. Throughout the article, the overarching theme focuses on whether folk psychology is best understood as a learned, theoretical skill (TT), or more so if it operates more naturally and intuitively though everyday social interaction. Because of the enormity of variance among diverse cultures, this is a challenging endeavor. Unsurprisingly, it appears that WEIRD (Western, Educated, Industrialized, Rich, and Democratic) societies and within Asian populations such as Chinese and Indian children come to learn folk psychology in similar, but varied ways. Because of the posed challenge, the authors take a nuanced look at empirical findings and philosophical challenges.

Furthermore, Hutto and Ravenscroft contrast Theory-theory with a more practical, embodied approach to understand others' minds, suggesting the former may place too much emphasis on abstract theorizing. Through their vast investigations into how diverse cultures approach folk psychology reveals that, despite variations, there is a shared ability to infer mental states in others. Moreover, the authors argue that folk psychology may not require the kind of detached scientific theorizing that Theory-theory implies, but instead offer a proposition that includes a model embedded in everyday social practices. 
In analyzing their argument, a potential objection may arise regarding their critique of Theory-theory. While Hutto and Ravenscroft suggest that theory over-intellectualizes our social cognition, an opposing view may argue that even if folk psychology is intuitive and socially grounded, it may still involve an underlying theoretical framework. For example, research in developmental psychology indicates that children develop increasingly complex understandings of others' mental states as they grow, suggesting they may be engaging in a kind of theory-building, albeit unconsciously. From this perspective, even if we are not from a cognitive that operates similarly to the scientific method—forming hypotheses about others' beliefs and assessing them against observed behaviors. 

The authors may respond by asserting that while humans do develop the ability to infer mental states (eventually), this process is fundamentally different from formal theory-building. Rather than consciously constructing a theoretical framework, they may argue that individuals learn to navigate complex social environments through practical engagement with others. Moreover, they may emphasize that experience strengthens folk psychology more-so than a structured cognitive process. That is, children's interactions with parents and peers, for example, help them refine their understanding of others' thoughts, but this happens in the context of lived social interactions, not through abstract theorizing. 

Thus, Hutto and Ravenscroft's critique of Theory-theory raises critical questions about how humans come to understand the minds of others. While they argue that folk psychology is more intuitive and socially grounded than the theory suggests, valid objections in defense of the cognitive processes may underlie this ability. However, their focus on the practical, embodied aspects of social cognition offers a compelling alternative to the more abstract reasoning models that are typically associated with folk psychology. This nuanced perspective highlights the complexity of understanding how we engage with the mental states of others across varying cultural contexts.

\section{RC: Simulation Theory -- Goldman}

Let us compare both simulation theory (ST) and theory-theory (TT) with one another. I will present an example and walk through how each theorist would diagnose/explain what is happening in the scene. From a narrative perspective, say we are privy to the thoughts of everyone in this example. Thus, consider the following scenario: A boy named Joe is being scolded for being mean to his sister. His mom is standing above him and jabbing a finger in the air while emphasizing a point unbeknownst to us. Furthermore, we see Joe's sister standing behind her mom's leg, crying because Joe did something to put her in distress. After being scolded, Joe tells his mom that he is sorry, and tries to give his side of what happened, but it has no effect. We, the narrator, note an observer, Eve, who is eavesdropping on the happenings.  

For the first example, suppose Eve is a TTist. They see an interaction occurring but are unaware of the reason it is happening. They scan the environment, and state an initial hypothesis that Joe is angry. After declaring their hypothesis, they look for clues to either prove or disprove their opinion on Joe's emotional state. This process would involve Eve looking at Joe's red face, his trembling body, and his direct eye contact with his mom. Then, she would examine Joe's mom and notice her body language: flared nostrils, wide eyes, loud voice, and her back is arched so she is hovering over Joe. This contextualizes why Joe is mad in the first place. Finally, Eve can see Joe yell back at his mom, and she does not understand the whole story. While Eve does not presume knowing that Joe's emotion is anger rather than frustration or sadness, she instead uses this theory as a baseline for understanding. If Eve were to continue to observe, her theory may evolve to incorporate more details that become apparent over time. Note that throughout the process, Eve was adhering to these guidelines: 

\begin{enumerate}
    \item Recognize a mental state (anger) in Joe,
    \item understand that mental state's characteristics, and
    \item notice how varying emotional states are interrelated.
\end{enumerate}

Now, for the second example, let us revisit Joe's process of apologizing and appealing to his mom's grace. Internally, Joe desires to be heard and believes that his perspective is pertinent to the situation. When he executes these mental states through his decision-making mechanism, he outputs a decision to talk back to his mom. Thus, Joe used his decision-making mechanism for its primary function. Let us check back in with Eve. Suppose this time that Eve is a STist. The process of her analyzing the scene unfolds differently. Upon initially observing the interaction between Joe and his mom, she engages in a process of emotional simulation. She is aware that she is not being scolded, nor does she believe that she must appeal to an authority figure for hurting her sister. However, Eve can still imagine what it would feel like to have the desire to share a belief like Joe's. Importantly, Eve is not using her decision-making mechanism for its primary function. Instead, she is re-using her decision-making mechanism to output an imagined decision to appeal to the mom in this case.

In both cases of TT and ST, we see that Eve does not ever posit that she knows exactly what Joe is thinking. Rather, in either case, she is utilizing the theory for a better understanding of what is occurring. These theories are operationalized in starkly different capacities, but still strive to achieve the same goal.

\section{RC: Cognitive and Affective Empathy -- Spaulding and Maibom}

First, I want to say that I really enjoyed reading both articles. In both, the language is clear, direct, non-esoteric or jargon heavy. They were able to tell me about their theories while maintaining a natural illustrative tone throughout the paper…something that is extremely rare among writers (especially academics, no offense). With that out of the way, Shannon Spaulding explains what cognitive empathy is. In brief, cognitive empathy is a hybrid system of theory-theory and simulation theory. In that, we use both aspects of TT and ST when we are trying to ``mind-read.'' Additionally, cognitive empathy can be swayed toward either pole of the TT and ST axis. If we relate to the person (e.g., share the same beliefs, religion, desire, gender, etc.), then we are more likely to use more of the ST-esque methods rather than the TT methods. When I say, ``[you] relate to the person,'' I mean that you recognize the person as being a part of your in-group. Thus, you tend to trust their beliefs more than out-group members. 

For Heidi Maibom's paper, she talks about affective empathy. In essence, affective empathy points a magnifying glass at the difference between empathetic emotions and personal emotions (or distress, for simplicity). She mentions many examples about what it means to feel an emotion that you ``catch'' from someone else, but the most notable example was of the two friends in their dull town. Building upon this story in relation to the story about the sad coworker, she posits that because of the duality of focus on personal distress vs empathetic distress, there must be more to cognitive empathy than what Spaulding is thinking. 

To be honest, I had a tough time following the contrasting of empathy, despondence, personal distress, self-focusing, and some of the other terms. However, I think I can understand the overall effect that Maibom was trying to capture. 

(This paragraph is going to be very personal for me, and I am sharing my true thoughts. If I offend you in any way by my retelling, I am truly sorry. I have no intention of offending you.) For example, in class on Monday, when a majority of people in class started packing up their things, you responded by asking us if we thought the conversation was boring. You had a look on your face that showed you were hurt. You looked taken-aback, and I felt guilty for causing that look. Then, I thought about what you had posted the Wednesday before break, that you were anxious and depressed. Reflecting upon that, it made me feel even worse during the situation because I have MDD. It is awful, and it is hard to find any pleasure in anything. Clearly, you are passionate about this class, so not only did I professionally offend you, but I also personally offended you. 

So, that brings me full circle: The initial empathetic emotion quickly turned into personal distress without me having any control over the situation. Therefore, I agree with Maibom; there is simply too much unaccounted for with the theory of cognitive empathy to explain the course of my emotions that day. And not that it matters anymore, but the reason I personally started packing up my things was because someone kept hitting my chair, and it felt like my personal space was being impeded upon. I know it does not sound like much, but it gave me a feeling of bugs crawling under my skin. I felt like I had to get out of the class as quickly as I could to relieve me of that situation I was put in. 

\section{RC: Defining Empathy -- Coplan}

In Coplan's article for defining empathy, she argues that there is a critical distinction between self-oriented and other-oriented perspective-taking that is crucial for understanding empathy. To summarize what self-oriented and other-oriented perspective-taking are, the former is when a person imagines themselves in another's situation, bringing their own beliefs, feelings, and dispositions to the simulation (closely resembling simulation theory in my opinion). Whereas with other-oriented perspective-taking, one attempts to imagine being the other person, taking on their characteristics and viewpoint. She illustrates self-oriented perspective-taking with an example of two sisters—one is introverted, and the other is extroverted—wherein the extrovert imagines herself in the introvert's situation of being alone, she might feel anxious and sad, while the introvert actually feels calm and relaxed. This is an important example because it shows that self-oriented perspective-taking can lead to a false representation of one's reality, but it could also lead to personal distress—even if there is no cause for it. Thus, other-oriented perspective-taking would allow us to have a more accurate understanding of others' true experiences. 

While this distinction is theoretically interesting, it may be practically impossible to achieve genuine other-oriented perspective-taking. In that, our own experiences, beliefs, and emotional dispositions are fundamental to how we process and understand the world. Even when attempting to imagine being another person, we must draw (only) from our own experiential framework to construct the imagination. The very act of simulating another's experience requires us to use our own cognitive and emotional ``machinery.'' Moreover, how can we truly verify that we have successfully adopted another's perspective rather than just constructing our best guess based on our own understanding? The extrovert/introvert example might be relatively straightforward, but most real-world situations involve complex combinations of personality traits, life experiences, and cultural factors that would be nearly impossible to fully simulate from another's perspective.

For a response, Coplan may say that while perfect other-oriented perspective-taking may be impossible, this does not invalidate the distinction or its utility. She could point to the empirical evidence she cites showing distinct neural activation patterns between self-oriented and other-oriented perspective-taking, suggesting these are indeed different psychological processes. The activation of inhibitory and regulatory mechanisms in the frontopolar cortex during other-oriented perspective-taking indicates that we can, to some degree, suppress our self-perspective and ``quarantine our own preferences, values, and beliefs'' (15). While we may never achieve perfect simulation of another's experience, the conscious effort to adopt their perspective rather than merely projecting ourselves into their situation leads to demonstrably different outcomes in terms of understanding and emotional regulation. The goal isn't perfect simulation, but rather a more accurate approximation of another's experience that what self-oriented perspective-taking provides. 

\section{RC: Empathy and Altruism -- Schramme}

In the section “Objections to the link between empathy and altruism,” Schramme built upon on the discussion about empathy's role in moral decision-making. That is, he noted how empathy might sometime lead us astray in moral judgements.  He specifically highlighted how empathetic responses can create biases in moral reasoning, such as when the immediate urgency of one person's need may overshadow the more abstract but equally valid (and possibly more urgent) needs of others. Additionally, he pointed out that certain moral problems, particularly those involving distributive justice affecting groups rather than individuals, may be difficult to address through empathy alone since there is not a specific individual with whom to empathize (see \href{https://en.wikipedia.org/wiki/Medical\%E2\%80\%93industrial_complex}{Medical-Industrial Complex} for more).

This self-stated objection raises a significant challenge to empathy-based moral theories. If empathy can lead to biased or incomplete moral judgements, then how can we rely on it as a foundation for moral reasoning? For example, consider a scenario where we oversee some allocation of limited medical resources. An empathetic response may lead us to prioritize a single, visibly suffering individual over implementing a systematic approach that could say more lives but lacks immediate emotional impact. This suggests that empathy might impede rational moral decision-making by introducing emotional biases that conflict with one's impartiality for taking care of those in need.  

Moreover, the problem can be exacerbated further when we consider chronic large-scale moral issues such as climate change or global poverty. Clearly, these challenges affect millions of people in long-term, innumerable ways that do not easily trigger empathetic responses. For example, consider the lack of educating the youth of a country. This leads to generational disparity. If empathy is crucial to moral motivation and judgement, how can we adequately address these systemic issues that resist empathetic engagement? 

The author provides a sophisticated response to these objections. He argues that critics often misunderstand the proper role of empathy in moral development and decision-making. The key point is that advocates for empathy's moral importance are not claiming that raw, unrefined empathetic responses should directly determine moral decisions. Instead, they argue for what he calls “regulated or “morally contoured” empathy—a developed skill rather than an immediate emotional reaction. 

This refined version of empathy is meant to contribute to the development of a moral perspective rather than serve as the sole basis for individual moral judgements. Just as other skills require development and refinement, empathy needs to be cultivated and guided by rational principles. He suggests that the real question is not whether raw empathy alone can guide moral behavior (he acknowledges it cannot), but rather how properly developed empathetic capacities contribute to the broader development of moral understanding and motivation. 

This perspective shifts the debate from whether empathy can resolve specific moral dilemmas to how it contributes of the development of moral agents capable of incorporating both emotional understanding and rational principles in their ethical deliberations. Thus, empathy is not a replacement for rational moral reasoning, but is a crucial component in the development of full moral agency. 

\end{document}