% This is the `` preamble''  of the document. This is where the format options get set.
% Pro-tip: things following the % mark will not be compiled by LaTeX. I'll be using them extensively to explain things as we go.
% Note: not to scare you off of LaTeX, but it's normal to have problems. And ya girl has been having some. I've included the copyright info at the bottom of the document from the guy who wrote this package, because his documentation doesn't entirely match how it's actually used. So this is a combination of his working preamble along with my added commentary or explanation. 
%% 


\documentclass[stu,12pt,floatsintext]{apa7}
% Document class input explanation ________________
% LaTeX files need to start with the document class, so it knows what it's using
% - This file is using the apa7 document class, as it has a lot of the formatting built in
% There are two sets of brackets in LaTeX, for each command (the things that start with the slash \ )
% - The squiggle brackets {} are mandatory for executing the command
% - The square brackets [] are options for that command. There can be more than one set of square brackets for some commands
% Options used in this document (general note - for each of these, if you want to use the other options, swap it out in that spot in the square brackets):
% - stu: this sets the `document mode' as the `` student paper''  version. Other options are jou (journal), man (manuscript, for journal submission), and doc (a plain document)
% --- The student setting includes things like 'duedate', 'course', and 'professor' on the title page. If these aren't wanted/needed, use the 'man' setting. It also defaults to including the tables and figures at the end of the document. This can be changed by including the 'floatsintext' option, as I have for you. If the instructor wants those at the end, remove that from the square brackets.
% --- The manuscript setting is roughly what you would use to submit to a journal, so uses 'date' instead of 'duedate', and doesn't include the 'course' or 'professor' info. As with 'stu', it defaults to putting the tables and figures at the end rather than in text. The same option will bump those images in text.
% --- Journal ('jou') outputs something similar to a common journal format - double columned text and figurs in place. This can be fun, especially if you are sumbitting this as a writing sample in applications.
% --- Document ('doc') outputs single columned, single spaced text with figures in place. Another option for producing a more polished looking document as a writing sample.
% - 12pt: sets the font size to 12pt. Other options are 10pt or 11pt
% - floatsintext: makes it so tables and figures will appear in text rather than at the end. Unforunately, not having this option set breaks the whole document, and I haven't been able to figure out why. IT's GREAT WHEN THINGS WORK LIKE THEY'RE SUPPOSED TO.

\usepackage[american]{babel}

\usepackage{csquotes} % One of the things you learn about LaTeX is at some level, it's like magic. The references weren't printing as they should without this line, and the guy who wrote the package included it, so here it is. Because LaTeX reasons.
\usepackage[style=apa,sortcites=true,sorting=nyt,backend=biber]{biblatex}
% biblatex: loads the package that will handle the bibliographic info. Other option is natbib, which allows for more customization
% - style=apa: sets the reference format to use apa (albeit the 6th edition)
\DeclareLanguageMapping{american}{american-apa} % Gotta make sure we're patriotic up in here. Seriously, though, there can be local variants to how citations are handled, this sets it to the American idiosyncrasies 
\addbibresource{bibliography.bib} % This is the companion file to the main one you're writing. It contains all of the bibliographic info for your references. It's a little bit of a pain to get used to, but once you do, it's the best. Especially if you recycle references between papers. You only have to get the pieces in the holes once.`

\usepackage[T1]{fontenc} 
\usepackage{mathptmx} % This is the Times New Roman font, which was the norm back in my day. If you'd like to use a different font, the options are laid out here: https://www.overleaf.com/learn/latex/Font_typefaces
% Alternately, you can comment out or delete these two commands and just use the Overleaf default font. So many choices!


% Title page stuff _____________________
\title{Methods} % The big, long version of the title for the title page
\author{Paul Beggs}
\duedate{\today}
% \date{January 17, 2024} The student version doesn't use the \date command, for whatever reason
\affiliation{Department of Psychology, Hendrix College}
\course{PSYC 295: Research Methods} % LaTeX gets annoyed (i.e., throws a grumble-error) if this is blank, so I put something here. However, if your instructor will mark you off for this being on the title page, you can leave this entry blank (delete the PSY 4321, but leave the command), and just make peace with the error that will happen. It won't break the document.
\professor{Dr. Jericka Battle}  % Same situation as for the course info. Some instructors want this, some absolutely don't and will take off points. So do what you gotta.

%\keywords{APA style, demonstration} % If you need to have keywords for your paper, delete the % at the start of this line

\begin{document}

\maketitle % This tells LaTeX to make the title page

\subsection{Participants}

In the present study, we aim to explore the perspectives of Palestinian and Israeli Americans regarding the war in Israel-Palestine. To achieve a comprehensive understanding, we have decided to recruit a balanced sample consisting of 150 Palestinian Americans and 150 Israeli Americans. The participants are being sourced through the online recruitment platform, Prolific, known for its diverse and reliable participant pool.

The other demographic information that we would like to record would be the participant's age, their nationality (with follow up questions for verification), their gender, their sex, their education level, and if they have lived in Palestine or Israel in the past.

To be eligible to participate in this experiment, participants must be at least 18 years old. This age restriction is guided by ethical considerations, particularly given the sensitive nature of the content involved in the study. The decision to exclude individuals under 18 years old stems from the potential psychological impact of the experimental materials, which include video content that may evoke strong emotional responses. Engaging minors in such studies involves complex ethical approvals and heightened protective measures, which we decided to avoid to safeguard younger individuals from potentially distressing experiences.

Each participant will receive a compensation of \$5 upon completion of the survey. This token is intended to appreciate their time and effort, and to encourage thoughtful and committed participation. It also helps to ensure a higher completion rate, which is critical for the reliability and validity of the study's findings.

Furthermore, all participants will be provided with detailed information about the study’s aims, the nature of the content, and their rights, including confidentiality and the voluntary nature of their involvement. They will also be informed about the availability of support resources should they experience discomfort at any point during or after their participation.

\subsection{Materials}

The primary material used in the present study was a \textcite{tedvideo} Talk video, specifically selected for its content and relevance to the Israeli-Palestinian conflict. The video features a structured dialogue between an Israeli and a Palestinian, both of whom share personal narratives about their lives affected by the ongoing conflict. This conversation occurs in a safe, neutral setting, designed to foster open communication and mutual understanding.

The participants in the video discuss significant life events that illustrate the human impact of the conflict, such as personal losses, daily challenges, and moments of cross-cultural interaction that have shaped their perspectives. The setting of the dialogue ensures that both individuals can express their views without fear of retribution or violence, which is critical for encouraging honest and empathetic exchanges.

The choice of this video as a material for our study was based on research done by \textcite{sunstein2015wiser}. \textcite{sunstein2015wiser} have shown in their study that members of the same group--otherwise known as in-group--can have a more persuasive argument among other members of the same group, when compared to people who are not included (so called, out-group). Thus, we believe we can foster a sense of relatability using this avenue of study.

\subsection{Measures}

For the present study, we utilized the Inclusion of Other in Self (IOS) scale, developed by \textcite{aron1992inclusion}. This measure is particularly effective in quantifying the extent to which participants perceive themselves as overlapping with others, which in this context refers to members of the opposing group. We specifically recorded whether exposure to the \textcite{tedvideo} Talk elicits feelings of increased similarity between Israeli and Palestinian participants.

In terms of assessing willingness to engage in dialogue, we employed a Likert scale ranging from 1 to 7, where 1 indicates `` I am not at all interested in having a conversation''  and 7 signifies `` I am very interested in having a conversation.''  This scale provides a comprehensive range of responses, enabling participants to express varying degrees of willingness to talk, from complete disinterest to strong interest, with intermediate options reflecting more moderate stances.

This approach was chosen to ensure that any changes in attitudes towards engagement and perception of similarity could be accurately captured, providing clear insights into the effects of the video exposure.

\subsection{Design}

Our study employs a straightforward between-subjects experimental design, focusing on two distinct groups: a control group and an exposure group. This design allows for clear comparison between baseline attitudes and the influence of the intervention.

Participants in the control group serve as the baseline for our study. They do not watch the \textcite{tedvideo} Talk video and therefore do not receive any intervention. Their responses provide a benchmark against which the effects of the exposure can be measured, reflecting their uninfluenced attitudes towards the other group.

Participants in the exposure group view the \textcite{tedvideo} Talk video featuring a dialogue between an Israeli and a Palestinian. This group's responses post-exposure are critical for assessing the impact of narrative-sharing on their perceptions and willingness to engage in dialogue.

By comparing these two groups, we aim to identify any significant differences in the IOS scores and the willingness to communicate as measured by the Likert scale. This comparison will determine whether exposure to empathetic cross-narrative dialogue can positively shift attitudes compared to no intervention.

The simplicity of this design minimizes variables and focuses directly on the efficacy of the intervention, enhancing the clarity and reliability of our findings. This approach also allows us to efficiently attribute any observed changes in attitude specifically to the intervention.

\subsection{Procedure}

The procedure for the present study was designed to ensure a controlled and ethical approach to assessing the impact of cross-narrative exposure on Israeli and Palestinian participants. The process was structured to maintain the integrity of the experimental conditions while ensuring participant safety and ethical compliance.

As stated in the Participants section, our participants were recruited through an online platform, Prolific, ensuring a diverse demographic spread appropriate for the study's needs. Upon joining the study, all participants were provided with an informed consent form. This document outlined the study's purpose, what their participation would involve, potential risks, and their rights, including confidentiality and the right to withdraw at any time without penalty.

After consenting, participants completed an initial survey. This included the IOS scale and the Likert scale for willingness to communicate, which served as a baseline measure of their attitudes towards the other group. No indication was given that the present study involved exposure to a specific type of content, to prevent biasing their responses.

For those who were a part of the control group, they proceeded to a waiting period, during which they engaged in neutral activities such as reading informational content unrelated to the conflict. Then, participants in the exposure group watched the \textcite{tedvideo} Talk.

Following the waiting period or video exposure, participants completed the same IOS and Likert scales again. This second set of measures aimed to capture any changes in their attitudes resulting from the intervention.

Upon completion of the post-exposure survey, all participants were debriefed. This stage was crucial, especially for the exposure group. Participants were informed about the study's specific aims and the reason behind using the \textcite{tedvideo} Talk video. They were also told why complete transparency was not provided initially: revealing the full nature of the experiment upfront could have influenced their natural responses, thereby threatening internal validity.

Ethical considerations were paramount, particularly in ensuring that any deception used was justified by the present study's goals and was fully explained during the debriefing to maintain trust and integrity in the research process.


\printbibliography

\end{document}

%% 
%% Copyright (C) 2019 by Daniel A. Weiss <daniel.weiss.led at gmail.com>
%% 
%% This work may be distributed and/or modified under the
%% conditions of the LaTeX Project Public License (LPPL), either
%% version 1.3c of this license or (at your option) any later
%% version.  The latest version of this license is in the file:
%% 
%% http://www.latex-project.org/lppl.txt
%% 
%% Users may freely modify these files without permission, as long as the
%% copyright line and this statement are maintained intact.
%% 
%% This work is not endorsed by, affiliated with, or probably even known
%% by, the American Psychological Association.
%% 
%% This work is `` maintained''  (as per LPPL maintenance status) by
%% Daniel A. Weiss.
%% 
%% This work consists of the file  apa7.dtx
%% and the derived files           apa7.ins,
%%                                 apa7.cls,
%%                                 apa7.pdf,
%%                                 README,
%%                                 APA7american.txt,
%%                                 APA7british.txt,
%%                                 APA7dutch.txt,
%%                                 APA7english.txt,
%%                                 APA7german.txt,
%%                                 APA7ngerman.txt,
%%                                 APA7greek.txt,
%%                                 APA7czech.txt,
%%                                 APA7turkish.txt,
%%                                 APA7endfloat.cfg,
%%                                 Figure1.pdf,
%%                                 shortsample.tex,
%%                                 longsample.tex, and
%%                                 bibliography.bib.
%% 
%%
%%
%% This is file `./samples/shortsample.tex',
%% generated with the docstrip utility.
%%
%% The original source files were:
%%
%% apa7.dtx  (with options: `shortsample')
%% ----------------------------------------------------------------------
%% 
%% apa7 - A LaTeX class for formatting documents in compliance with the
%% American Psychological Association's Publication Manual, 7th edition
%% 
%% Copyright (C) 2019 by Daniel A. Weiss <daniel.weiss.led at gmail.com>
%% 
%% This work may be distributed and/or modified under the
%% conditions of the LaTeX Project Public License (LPPL), either
%% version 1.3c of this license or (at your option) any later
%% version.  The latest version of this license is in the file:
%% 
%% http://www.latex-project.org/lppl.txt
%% 
%% Users may freely modify these files without permission, as long as the
%% copyright line and this statement are maintained intact.
%% 
%% This work is not endorsed by, affiliated with, or probably even known
%% by, the American Psychological Association.
%% 
%% ----------------------------------------------------------------------