% \begin{center}
%     \textbf{NEW NOTES FOR 03/19/25} \\
%     \hrulefill
% \end{center}

\section{Overview — Unit 3}

\begin{coloredlist}
    \item Neurotransmitters and Neuromodulators
    \begin{coloredlist}
        \item Psychopharmacology
        \item Disorders
        \item Pain
    \end{coloredlist}
\end{coloredlist}

\subsection{Psychopharmacology in Detail}

\begin{coloredlist}
    \item The scientific study of the effects of drugs on the nervous system and behavior.
    \item Psychopharmacology is the study of how drugs affect the mind and behavior.
    \begin{coloredlist}
        \item Psychotherapeutic drugs
        \item Better understanding of how things normally work.
    \end{coloredlist}
\end{coloredlist}

\subsubsection{Principles of Drug Action}

\begin{coloredlist}
    \item \cyanit{Selective Action} — Drugs are selective in their action on the nervous system.
    \begin{coloredlist}
        \item \cyanit{Sites of Action} — The location at which a drug interacts with the body to produce its effects.
        \item Side effects are often due to the drug acting on sites other than the intended target.
        \begin{coloredlist}
            \item Thus, side effects are relative to what our preferred site of action is.
        \end{coloredlist}
        \item \textit{Example}: Opioids primarily affect the opioid receptor system.
    \end{coloredlist}
    \item Drugs don't CREATE effects, they \textit{modulate} ongoing cellular activity.
    \begin{coloredlist}
        \item That is, they affect behavior by affecting neural transmission in some way.
        \item \cyanit{Agonist} — A drug that mimics or enhances the \textbf{effects of a neurotransmitter}.
        \begin{coloredlist}
            \item Facilitates post synaptic effects.
            \item \textit{Example}: Morphine mimics endorphins, which are natural painkillers.
        \end{coloredlist}
        \item \cyanit{Antagonist} — A drug that blocks or inhibits the effects of a neurotransmitter.
        \begin{coloredlist}
            \item Inhibits post synaptic effects.
            \item \textit{Example}: Naloxone blocks the effects of opioids, reversing their effects.
        \end{coloredlist}
        \item Agonistic effects can become antagonistic if the drug is taken in excess.
        \begin{coloredlist}
            \item \textit{Example}: I make a neuron fire a neurotransmitter, but I also block the reuptake of that neurotransmitter.
        \end{coloredlist}
    \end{coloredlist}
\end{coloredlist}

\subsubsection{Basic Process}

\begin{coloredlist}
    \item \cyanit{Precursor} — A substance from which another substance is formed. AKA, the ingredients used to make a neurotransmitter.
    \item \[
        \text{Precursor} \xrightarrow{\text{Synthetic Enzyme}} \text{NT} \xrightarrow{\text{Metabolic Enzyme}} \text{Inactive Metabolite}
    \]
    \begin{coloredlist}
        \item \cyanit{Synthesis} — The process of creating a neurotransmitter from its precursors.
        \item Sometimes, we break down the precursor to build the neurotransmitter.
        % \item \cyanit{Storage} — The neurotransmitter is stored in vesicles until it is needed.
        % \item \cyanit{Release} — The neurotransmitter is released into the synaptic cleft when an action potential arrives at the axon terminal.
        % \item \cyanit{Binding} — The neurotransmitter binds to receptors on the post-synaptic neuron, causing a change in the neuron's activity.
        % \item \cyanit{Inactivation} — The neurotransmitter is removed from the synaptic cleft by reuptake or enzymatic degradation.
    \end{coloredlist}
\end{coloredlist}

% \begin{center}
%     \textbf{NEW NOTES FOR 03/21/25} \\
%     \hrulefill
% \end{center}

\subsection{Other Ways of Agonist and Antagonist}

\begin{coloredlist}
    \item Block Ca\(^{2+}\) channels from opening (antagonist)
    \item \cyanit{Mimetic} — Mimics the action of a neurotransmitter.
    \begin{coloredlist}
        \item \cyanit{Direct Agonist} — Binds to the same receptor as the neurotransmitter and mimics its effects.
        \item \cyanit{Indirect Agonist} — Binds to a different site on the receptor and enhances the effects of the neurotransmitter.
    \end{coloredlist}
    \item Blocking agent
    \begin{coloredlist}
        \item Competitive
        \begin{coloredlist}
            \item \cyanit{Direct Antagonist} — Binds to the same receptor as the neurotransmitter and blocks its effects.
        \end{coloredlist}
        \item Non-competitive
        \begin{coloredlist}
            \item \cyanit{Indirect Antagonist} — Binds to a different site on the receptor and blocks the effects of the neurotransmitter.
        \end{coloredlist}
        \item \cyanit{Inverse Agonist} — Binds to the same receptor as the neurotransmitter and produces the opposite effect.
    \end{coloredlist}
    \item \cyanit{Depolarizing} or \cyanit{Desensitizing Agent} — A drug that causes the AP to stay in a depolarized state; refusing to let the neuron go through another AP, and it stays in the absolute refractory period. (Antagonist)
    \item Interfere with vesicles (leaky or transporter proteins). (Antagonist)
    \item Interfere with docking proteins. (Antagonist)
    \item Selectively deactivate autoreceptors. (Agonist)
    \item Selectively activate autoreceptors. (Antagonist)
\end{coloredlist}

\section{What Do These Chemicals Between Neurons Do?}

\begin{coloredlist}
    \item Transmit information.
    \begin{coloredlist}
        \item Glutamate
        \item GABA
        \item Glycine
    \end{coloredlist}
    \item Modulate information.
    \begin{coloredlist}
        \item Every other neurotransmitter.
    \end{coloredlist}
\end{coloredlist}

\section{Classes of Neurotransmitters (Revisited from Lab)}

\begin{coloredlist}
    \item Amino Acids
    \begin{coloredlist}
        \item Glutamate, GABA, Glycine
        \begin{coloredlist}
            \item \cyanit{Glutamate} — Synthesized from precursor glutamine by an enzyme called \cyanit{glutaminase}. It is the most common excitatory neurotransmitter in the brain.
            \begin{coloredlist}
                \item Related closely with the \cyanit{NMDA} receptor, which is a type of glutamate receptor that is important for synaptic plasticity and memory formation.
                \item One drug that binds to this site, \cyanit{Phencyclidine (PCP)} (direct antagonist), is a drug that blocks the NMDA receptor and causes hallucinations and dissociation. Another drug that is thought to bind here, \cyanit{Ketamine} (direct antagonist), is a dissociative anesthetic that is used in surgery and is also being studied as a treatment for depression.
                \item \textbf{Reuptake and Deactivation}
                \begin{coloredlist}
                    \item Reuptake is done by the \cyanit{excitatory amino acid transporters (EAATs)}. These are important because it reduces the change of excitotoxicity, which is believed to be involved in damage to the brain in stroke and amyotrophic later sclerosis (ALS, Lou Gehrig's disease).
                \end{coloredlist}
            \end{coloredlist}
            \item \cyanit{GABA} — Synthesized from precursor glutamate by an enzyme called \cyanit{glutamic acid decarboxylase (GAD)}. It is the most common inhibitory neurotransmitter in the brain.
        \end{coloredlist}
    \end{coloredlist}
    \item \cyanit{Amines} (monoamines) — Derived from amino acids
    \begin{coloredlist}
        \item Catecholamines
        \begin{coloredlist}
            \item Contain catechol and derived from the amino acid tyrosine.
            \item \cyanit{Tyrosine} — Precursor for the catecholamines.
            \item Dopamine (DA), Norepinephrine (NE), Epinephrine (Adrenaline)
            \item Dopaminergic, Adrenergic, and Noradrenergic systems.
        \end{coloredlist}

        % \begin{center}
        %     \textbf{New Notes for 03/31/25} \\
        %     \hrulefill
        % \end{center}

        \item \cyanit{Indolamines} %— Derived from the amino acid tryptophan.
        \begin{coloredlist}
            \item \cyanit{Serotonin} (5-HT) %— Derived from the amino acid tryptophan.
            \item \cyanit{Melatonin} %— Derived from serotonin.
        \end{coloredlist}
        \item Peptides (AKA: Neuropeptides)
        \begin{coloredlist}
            \item Endogenous Opioids
        \end{coloredlist}
        \item Acetylcholine (ACh)
        \item Lipids
        \begin{coloredlist}
            \item \cyanit{Anadamide} (Sanskrit for ``bliss'') — Endogenous cannabinoid.
            \item These appear to be synthesized on demand; produced and released as needed and not stored in synaptic vesicles.
            \item Anadamide is deactivated by the enzyme \cyanit{fatty acid amide hydrolase (FAAH)}.
        \end{coloredlist}
        \item \textbf{Two Other Classes}
        \begin{coloredlist}
            \item Nucleosides
            \begin{coloredlist}
                \item Adenosine
            \end{coloredlist}
            \item Soluble Gases
            \begin{coloredlist}
                \item \cyanit{Nitric Oxide (NO)} — Required for an erection.
            \end{coloredlist}
        \end{coloredlist}
    \end{coloredlist}
\end{coloredlist}

\section{Acetylcholine (ACh)}

\begin{coloredlist}
    \item First neurotransmitter discovered.
    \begin{coloredlist}
        \item Otto von Loewy — Discovered ACh in 1921.
        \begin{coloredlist}
            \item This guy took a frog heart and put it in saline. Then, he took simulated the parasympathetic part of the vagus nerve, and saw that the heart slowed down.
            \item He then took the saline and put it in a different frog heart, and saw that the heart slowed down again.
            \item ``Vagusstoff'' (ACh) — The chemical that was released from the vagus nerve that slowed the heart down.
            \item Cholinergic — Referring to ACh.
        \end{coloredlist}
    \end{coloredlist}
    \item \textbf{\textit{Some} Functions}
    \begin{coloredlist}
        \item \textbf{Function in the ANS:}
        \begin{coloredlist}
            \item Sympathetic
            \begin{coloredlist}
                \item Spinal nerve leaves the cord and synapses in the paravertebral ganglion (ACh)
                \item Then makes neuromuscular junction with smooth muscles and glands (NE)
                \begin{coloredlist}
                    \item \cyanit{Neuromusclar Junction} — The synapse between a motor neuron and a muscle fiber.
                    \item \cyanit{Paravertebral Ganglion} — A ganglion located next to the spinal cord.
                \end{coloredlist}
                \begin{coloredlist}
                    \item Except sweat glands (ACh)
                \end{coloredlist}
                \item \cyanit{Sympathetic Chain} — A chain of ganglia that runs parallel to the spinal cord. This is the reason for when you get anxious, ALL of your body gets anxious.
            \end{coloredlist}
            \item Parasympathetic
            \begin{coloredlist}
                \item Spinal nerve leaves the cord and synapse in the parasympathetic ganglion (ACh)
                \item Then makes neuromuscular junction with smooth muscles and glands (ACh)
            \end{coloredlist}
            \item The only NT in the parasympathetic branch.
            \item NT of the preganglionic sympathetic branch.
        \end{coloredlist}
        \item \textbf{Function in the Somatic NS}
        \begin{coloredlist}
            \item Excites the neuromusclar junction (ACh)
            \item So, ACh is important for getting motor messages out to all kinds of muscles and glands.
        \end{coloredlist}
        % \begin{center}
        %     \textbf{NEW NOTES FOR 04/02/25} \\
        %     \hrulefill
        % \end{center}
        \item \textbf{Function in the CNS}
        \begin{coloredlist}
            \item ACh is important in:
            \begin{coloredlist}
                \item Learning and alertness (\cyanit{Basal Forebrain})—-activates the cortex and facilitates learning.
                \begin{coloredlist}
                    \item \cyanit{Nucleus Basalis} — Projects to the cortex
                    \item \cyanit{Medial Septal Nucleus} and \cyanit{Nucleus of Diagonal Band} — Projects to the hippocampus through the fornix.
                \end{coloredlist}
                \item Memory (\cyanit{medial septal nucleus})—-modulate the hippocampus
                \item REM sleep generation (\cyanit{Pedunculopontine nucleus (PPT)} and \cyanit{Laterodorsal Tegmental Nucleus (LDT)})—-projects to the pons and thalamus.
                \item Reward system.
            \end{coloredlist}
        \end{coloredlist}
    \end{coloredlist}
    \item \textbf{Synthesis and Metabolism}
    \item \textbf{Drugs and Disorders}
\end{coloredlist}

\subsection{ACh Synthesis and Metabolism}

\begin{coloredlist}
    \item \textbf{Synthesis}
    \begin{coloredlist}
        \item In a nutshell: A breakdown of lipids leads to Choline, which is the precursor for ACh. Acetate is the anion in vinegar (Acetic acid). Then, this is combined with Acetate to make ACh.
        \item In more detail:
        \begin{coloredlist}
            \item CoA attaches to an acetate ion (\cyanit{Acetylcoenzyme A (acetyl-CoA)}).
            \item Then, \cyanit{choline acetyltransferase (ChAT)} transfers the acetate from the acetyl-CoA to the choline molecule.
            \item Mnemonic: \textbf{ChAT}: From right to left: Transfers acetate to choline.
        \end{coloredlist}
    \end{coloredlist}
    \item \textbf{Metabolism}
    \begin{coloredlist}
        \item ACh is broken down by the enzyme \cyanit{acetylcholinesterase (AChE)} into acetate and choline. Nice and simple!
        \item The choline is taken back up by active transport and reused, and the acetate is broken down and eliminated.
    \end{coloredlist}
\end{coloredlist}

\subsection{Two Types of Cholinergic Receptors}

\begin{coloredlist}
    \item Nicotinic Receptors
    \begin{coloredlist}
        \item Agonist at low doses, but antagonist at high doses.
        \item Iontropic.
        \item Found at the Neuromusclar Function in the PNS.
        \item \cyanit{Curare} (direct antagonist) — A drug that blocks nicotinic receptors, causing paralysis.
        \begin{coloredlist}
            \item Competitive blocking agent
            \item Paralysis, surgery
        \end{coloredlist}
    \end{coloredlist}
    \item Muscarinic Receptors
    \begin{coloredlist}
        \item Comes from a hallucinogenic mushroom (Amanita muscaria).
        \begin{coloredlist}
            \item \textbf{Don't confuse with Serotonin's Mescaline: Cactus; nor Psilocybin: Mushroom.}
        \end{coloredlist}
        \item Vikings (probably took this drug before raiding) and Koryaks (Nordic people who used this mushroom in religious practices).
        \item Metabotropic receptors.
        \item Predominates in the CNS (although, both types are found in the CNS).
        \item \cyanit{Atropine} (direct antagonist) — A drug that blocks muscarinic receptors, causing pupil dilation and increased heart rate.
        \begin{coloredlist}
            \item Competitive blocking agent
            \item Belladonna alkaloids (deadly nightshade)
        \end{coloredlist}
    \end{coloredlist}
\end{coloredlist}

% \begin{center}
%     \textbf{NEW NOTES FOR 04/04/25} \\
%     \hrulefill
% \end{center}

\section{MORE Drugs and Toxins Affecting ACh}

\begin{coloredlist}
    \item \cyanit{Botulinum Toxin} — A waste product of \textit{Clostridium botulinum}, which are bacteria who grows without oxygen.
    \begin{coloredlist}
        \item Interferes with Ca\(^{2+}\) influx channels, preventing the release of ACh.
        \item Because Botox causes paralysis, it can interfere with emotional \textit{expression} because it paralyzes muscles like the orbicularis oculi.
        \item Additionally, since we know that expression influences experience, when we paralyze these muscles, then the emotional \textit{experience} is also negatively affected.
        \item \textbf{Does Botox Decrease Emotional Experiences?}
        \begin{coloredlist}
            \item Population: Women who want wrinkles gone.
            \item One IV: two levels: Botox or restylane (dermal filler).
            \item Method: Everyone had wrinkle reduction. AND, Everyone watches some emotion evoking movies.
            \item Results: Botox group had less emotional experience than the restylane group.
            \item \textbf{Is this a good thing?}
            \begin{coloredlist}
                \item Another study takes a sample of depressed people and gives them either Botox or a placebo.
                \item Results: 15\% of placebo had a decrease in depression, while 52\% of the Botox group had a decrease in depression.
            \end{coloredlist}
        \end{coloredlist}
        \item Botox can also be used to treat migraines, cerebral palsy, and hyperhidrosis (excessive sweating).
    \end{coloredlist}
    \item \cyanit{Black Widow Spider Venom} — A neurotoxin that causes the release of ACh at the neuromuscular junction, causing continual release of ACh and paralysis.
    \item \cyanit{Cobra and Krait Venom} — A neurotoxin that blocks the binding of ACh to nicotinic receptors, causing paralysis.
    \item \cyanit{AchE Blockers} — Comes into contact with the enzyme that breaks down ACh, causing an increase in ACh in the synaptic cleft.
    \begin{coloredlist}
        \item Irreversible
        % \begin{center}
        %     \textbf{NEW NOTES FOR 04/07/25} \\
        %     \hrulefill
        % \end{center}
        \begin{coloredlist}
            \item Insecticides (Parathion)
            \item Nerve gas: DFP (Diisopropylfluorophosphate (don't need to know the whole name)) and Sarin.
            \begin{coloredlist}
                \item Readily crosses the blood-brain barrier so PNS and CNS are affected.
                \item Antidote?
                \begin{coloredlist}
                    \item \cyanit{Atropine} — A drug that blocks muscarinic receptors, preventing the effects of excess ACh.
                    \item \cyanit{Pralidoxime} — A drug that reactivates AChE, allowing it to break down ACh again.
                \end{coloredlist}
            \end{coloredlist}
        \end{coloredlist}
        \item Reversible
        \begin{coloredlist}
            \item \cyanit{Neostgmine} (\textbf{Prostigmin}) and \cyanit{Physostigmine} (\textbf{Antilirum}) — Drugs that inhibit AChE, increasing the amount of ACh in the synaptic cleft.
            \begin{coloredlist}
                \item Doesn't cross the blood-brain barrier, so it only affects the PNS.
                \item Used to treat \cyanit{myasthenia gravis} (a disease that causes muscle weakness and fatigue).
                \begin{coloredlist}
                    \item Autoimmune disease that attacks nicotinic receptors at the neuromuscular junction.
                \end{coloredlist}
            \end{coloredlist}
            \item \cyanit{Donepezil} (\textbf{Aricept}) and \cyanit{rivastigmine} (\textbf{Exelon}) — These drugs do the same thing as the above drugs, but they cross the blood-brain barrier and are used to treat Alzheimer's disease and Parkinson's disease (only the cognitive part).
        \end{coloredlist}
    \end{coloredlist}
\end{coloredlist}

\section{New Drug for Schizophrenia}

\begin{coloredlist}
    \item We'll talk about dopamine drugs later in this unit.
    \item This new drug now:
    \begin{coloredlist}
        \item \cyanit{Xanomelne and trospium chloride} (\textbf{Cobenfy}) — A drug that blocks the muscarinic receptors in the CNS, but not in the PNS.
        \item Dopamine but also Ach!
    \end{coloredlist}
\end{coloredlist}

\section{Catecholamines}

\begin{coloredlist}
    \item Dopamine (DA)
    \item Norepinephrine (NE)
    \item Epinephrine (Adrenaline)
\end{coloredlist}

\subsection{Dopamine (DA)}

\begin{coloredlist}
    \item Synthesis and Metabolism
    \item Function
    \item Drugs and Disorders
\end{coloredlist}

\subsubsection{Dopamine Synthesis}

\begin{coloredlist}
    \item Tyrosine was first discovered from cheese (tyrosine = cheese).
    \begin{coloredlist}
        \item Tyrosine is the precursor for DA, NE, and Epi.
        \item \cyanit{Tyrosine Hydroxylase} — The rate-limiting enzyme in the synthesis of catecholamines.
        \begin{coloredlist}
            \item Converts tyrosine to L-DOPA.
            \item L-DOPA is the precursor for DA, NE, and Epi.
        \end{coloredlist}
        \item L-DOPA is converted to DA by the enzyme \cyanit{DOPA decarboxylase}.
        % \item DA is converted to NE by the enzyme \cyanit{Dopamine \(\beta\)-hydroxylase (DBH)}.
        % \item NE is converted to Epi by the enzyme \cyanit{Phenylethanolamine N-methyltransferase (PNMT)}.
    \end{coloredlist}
\end{coloredlist}

\subsubsection{Dopamine Metabolism}

\begin{coloredlist}
    \item DA is broken down by the enzyme \cyanit{Monoamine Oxidase (MAO)} into \cyanit{Dihydroxyphenylacetric acid (DOPAC)}.
    \item Then, \cyanit{Catechol-O-methyltransferase (COMT)} converts DOPAC into \cyanit{Homovanillic acid (HVA)}.
    \item Also, starting from DA, we can use COMT to convert it to 3-methoxytyramine (3-MT), then with MAO, we can convert it to HVA.
\end{coloredlist}

\subsubsection{DA Function}

\begin{coloredlist}
    \item \textbf{Movement/Motor systems}
    \begin{coloredlist}
        \item \cyanit{Nigrostriatal System} — Starts in the substantia nigra and ends in the striatum (caudate nucleus and putamen).
        \item Here's the route: We start at the striatum, which then sends an inhibitory GABA signal to the substantia nigra, who sends a reciprocal inhibitory DOPA signal back to the striatum nerve that sends an inhibitory GABA signal to the globus pallidus. Then, the globus pallidus excites the thalamus, who then excites the primary motor cortex, who then excites movement.
        \begin{coloredlist}
            \item Note that if the inhibitory signal to the substantia nigra is limited, then the signal that the striatum sends to the globus pallidus is much stronger, which leads to a weaker signal to the thalamus, and thus to movements.
        \end{coloredlist}
        \item Parkinson's Disease symptoms:
        \begin{coloredlist}
            \item Weakness,
            \item Tremor at rest,
            \item Muscle rigidity,
            \item Problems with balance,
            \item Abnormal gait,
            \item Trouble learning
        \end{coloredlist}
        \item Treatment
        % \begin{center}
        %     \textbf{NEW NOTES FOR 04/09/25} \\
        %     \hrulefill
        % \end{center}
        \begin{coloredlist}
            \item \cyanit{Reserpine} (\textbf{Raudixin}) for \(\downarrow\) \textbf{BP} (Not in use anymore because it caused Parkinson's-like symptoms)
            \begin{coloredlist}
                \item 1960's
                \begin{coloredlist}
                    \item Blocks monoamine transporters
                    \begin{coloredlist}
                        \item Developed Parkinson's symptoms
                        \item Can't fill vesicles and DA is lowered
                    \end{coloredlist}
                \end{coloredlist}
                \item Then, discovered Substantia Nigra was pale.
            \end{coloredlist}
            \item L-DOPA can be a direct treatment for Parkinson's as well.
            \item \cyanit{MPTP} — Neurotoxin for DA cells in the Nigrostriatal System (which is not endogenous).
            \item \textbf{History of MPTP — or why you shouldn't use illicit drugs}
            \begin{coloredlist}
                \item 1982 — young California heroin users
                \item Had used what they THOUGHT was synthetic heroin
                \begin{coloredlist}
                    \item \cyanit{MPPP} — Opioid analgesic drug
                    \begin{coloredlist}
                        \item Not used clinically
                        \item Illegally manufactured for recreational drug use
                    \end{coloredlist}
                    \item INSTEAD it was MPTP (oh no!)
                    \begin{coloredlist}
                        \item They instantly developed Parkinson's-like symptoms
                        \item Bad for them, but good for us because we can study it.
                        \begin{coloredlist}
                            \item Led to animal model development and possible treatment ideas.
                        \end{coloredlist}
                    \end{coloredlist}
                \end{coloredlist}
                \item We don't know why Parkinson's patient's cells are dying, but maybe something similar.
                \item MPTP is converted to the chemical \cyanit{MPP+} by the enzyme MAO (which is what breaks down DA), which is what damaged the cells.
                \begin{coloredlist}
                    \item Question: Could MAO-I improve Parkinson's?
                    \item Yes!
                    \item \cyanit{Deprenyl}, also called \cyanit{selegiline} (\textbf{Eldepryl}, \textbf{Jumex}) — A drug that inhibits MAO, can slow down progression of the disease.
                \end{coloredlist}
            \end{coloredlist}        
        \end{coloredlist}
        \item New treatment
        \begin{coloredlist}
            \item Molecule keeps proteins from misfolding
            \item \cyanit{Lewy Bodies} — Misfolded proteins that are found in the brains of people with Parkinson's.
            \begin{coloredlist}
                \item These are toxic to DA cells
            \end{coloredlist}
        \end{coloredlist}
        \item \cyanit{Huntington's Chorea} — A genetic disorder that leads to uncontrolled movements and cognitive decline.
        \begin{coloredlist}
            \item Too little GABA from the Striatum to the Substantia Nigra causes an increase in dopamine back to the Striatum which, in turn, lessens the signal to the Globus Palidus, which increases overall movements.
            \item \cyanit{Tetrabenazine} (\textbf{Xenazine}) — Drug that inhibits the DA vesicle transporters.
            \item \cyanit{Pallidotomy} — A surgery that affects the Globus Pallidus to inhibit movement.
        \end{coloredlist}
        \item \cyanit{Choreoathetotic Movements} — too much movement
        \begin{coloredlist}
            \item \cyanit{Athetosis} — Slow continually writing movements
            \item \cyanit{Choreic} (to dance) — Rapid, purposeless, involvuntary movements
        \end{coloredlist}
        % \begin{center}
        %     \textbf{NEW NOTES FOR 04/11/25} \\
        %     \hrulefill
        % \end{center}
    \end{coloredlist}
    \item \textbf{Behavioral Arousal and Attention}
    \begin{coloredlist}
        \item Narcolepsy
        \begin{coloredlist}
            \item \cyanit{Methylphenidate} (\textbf{Ritalin}) — A drug that increases DA and NE in the brain, used to treat ADHD, but can also be used for narcolepsy.
            \item \cyanit{Hypocretine} — A neuropeptide that is involved in the regulation of sleep and wakefulness.
            \begin{coloredlist}
                \item Created by the lateral hypothalamus.
                \item Hypocretine: \textit{Hypo} for \textit{hypo}thalamus, \textit{cretine} for \textit{secretin} (a hormone).
                \item \cyanit{Orexin} — Another name for hypocretine; makes you want to eat.
            \end{coloredlist}
            \item From hypocretine, researchers developed an antagonist for the orexin receptor, which is used to treat insomnia. This drug is called \cyanit{Suvorexant} (\textbf{Belsomra}).
            \item \textbf{Treatment}
            \begin{coloredlist}
                \item \cyanit{TAK-994} — OX2R (Orexin-2 receptor) Agonist
                \item \cyanit{Hcrt-1} — Intranasal hypocretine-1 (orexin-1) agonist
                \item Hypocretine Cell Transplant
                \item Gene Therapy: \textit{introduce} preprohypocretin gene into the brain to make more hypocretine.
                \item Opiates (exogenous) can increase the number of hypocretin-producing cells in the brain.
                \item Indirect role for opiate agonists in treating narcolepsy.
            \end{coloredlist}            
        \end{coloredlist}
        \item ADHD
        \begin{coloredlist}
            \item Uses Methylphenidate for selective attention.
        \end{coloredlist}
        \item \cyanit{Mesocortical System}
        \begin{coloredlist}
            \item From ventral tegmental nucleus to prefrontal cortex, limbic CORTEX, hippocampus, all frontal lobes, and association areas of parietal and temporal lobes in primates.
            \item Short-term memories, planning, and problem-solving are all associated with this system.
        \end{coloredlist}
    \end{coloredlist}
    \item \textbf{Reinforcement and Reward}
    \begin{coloredlist}
        \item \cyanit{Mesolimbic System (MLS)} — Responsible for reward and reinforcement.
        \begin{coloredlist}
            \item From vental tegmental nucleus to limbic system
            \begin{coloredlist}
                \item Amygdala, hippocampus, and nucleus accumbens.
                \item Opioids cause the release of dopamine at the nucleus accumbens, which is the pleasure center of the brain.
            \end{coloredlist}
        \end{coloredlist}
        \item James Olds \& Peter Milner (1954)
        \begin{coloredlist}
            \item They asked: ``Does electrical stimulation of the reticular formation facilitate learning?''
            \item James Olds visits a conference and listens to Neal Miller, who says electrical stimulation is aversive, so it should be avoided.
            \item One lone rat was put in a box with a lever, and when the rat pressed the lever, it would get a shock to the reticular formation. He ended up pressing the level 700 times per hour.
        \end{coloredlist}
        \item More studies of this
        \begin{coloredlist}
            \item Skinner box
            \begin{coloredlist}
                \item Rats press 2000 times per hour for a shock to the MLS.
                \item Monkeys press 8000 times per hour for a shock to the MLS.
                \item Starving animals will choose the MLS over food 80\% of the time. 
                \item They also press the button for these conditions too:
                \begin{coloredlist}
                    \item Thirsty,
                    \item Getting shocked (at their feet),
                    \item Mother instincts.
                \end{coloredlist}
            \end{coloredlist}
        \end{coloredlist}
        \item Delgado (1969) — For people who were getting their brain stimulated for seizures, this researcher also asked them about what they thought of the stimulation. They all thought that it was pleasurable.   
    \end{coloredlist}
\end{coloredlist}

\begin{center}
    \textbf{NEW NOTES FOR 04/16/25} \\
    \hrulefill
\end{center}

\section{Schizophrenia in Focus}

\subsection{General Description}

\begin{coloredlist}
    \item \cyanit{Dementia Praecox} — A term used to describe a group of disorders characterized by a decline in cognitive function and emotional regulation. Found by Emil Kraepelin, a German psychiatrist (1887).
    \begin{coloredlist}
        \item Premature deterioration of the mind.
    \end{coloredlist}
    \item Age of onset: Late teens to mid 30s.
    \item \cyanit{Schizophrenia} — Same definition as before, duh. Found by Eugen Bleuler, a Swiss psychiatrist (1911).
    \begin{coloredlist}
        \item Split of the mind from reality, not split personalities.
    \end{coloredlist}
\end{coloredlist}

\subsection{Theory behind Negative and Positive Symptoms**}
**(and Cognitive Symptoms)
\begin{coloredlist}
    \item ``Positive'' and ``Negative'' are not used in the traditional sense. Instead, they are used to describe the presence or absence of certain symptoms.
    \begin{coloredlist}
        \item Positive symptoms are the presence of abnormal behaviors, while negative symptoms are the diminution or absence of normal behaviors.
        \item Cognitive symptoms are the presence of cognitive deficits.
    \end{coloredlist}
\end{coloredlist}

\subsubsection{Positive Symptoms (Escalation over normal functioning)}

\begin{coloredlist}
    \item \cyanit{Hallucinations} — Perception of something that does not have a basis in reality.
    \begin{coloredlist}
        \item Auditory (most common), visual, tactile, olfactory, and gustatory.
    \end{coloredlist}
    \item \cyanit{Delusions} — A false belief that is resistant to reason or confrontation with actual fact. Actually built on reality, but misinterpreted.
    \begin{coloredlist}
        \item Patently unrealistic.
        \item For example:
        \begin{coloredlist}
            \item \cyanit{Referential} — Believing that something is meant for you (e.g., the TV is talking to you).
            \item \cyanit{Persecutory} — Believing that someone is out to get you (e.g., the government is watching you).
            \item \cyanit{Grandiose} — Believing that you are more important than you are (e.g., you are the king of the world).
            \item \cyanit{Control} — Believing that someone is controlling your thoughts or actions (e.g., the government is controlling your mind).
        \end{coloredlist}
        \item On the difference between an \textit{illusion} and a \textit{delusion} is that everyone can experience and illusion (not unique and based on manipulations of our nervous systems), whereas a delusion is unique to the person experiencing.
    \end{coloredlist}
    \item \cyanit{Disorganized Speech} — A pattern of incoherent or illogical speech that is difficult to follow.
    \begin{coloredlist}
        \item \cyanit{Derailment} — A pattern of speech in which the speaker jumps from one topic to another without any logical connection between them.
        \item \cyanit{Tangentiality} — A pattern of speech in which the speaker goes off on tangents and does not return to the main topic.
        \item \cyanit{Word Salad} — A pattern of speech in which the speaker uses words that are not related to each other in any meaningful way.
    \end{coloredlist}
    \item \cyanit{Grossly disorganized behavior} — A pattern of behavior that is inappropriate for the situation or that is not goal-directed. It is unpredictable and unprovoked.
    \item \cyanit{Catatonic behavior} — A pattern of behavior in which the person is unresponsive to the environment and does not move or speak.
    \begin{coloredlist}
        \item Maintaining a rigid or bizarre posture.
        \item Purposeless excessive motor activity.
        \begin{coloredlist}
            \item For example, continuously spinning your hair in circles or pacing back and forth.
        \end{coloredlist}
    \end{coloredlist}
\end{coloredlist}

\subsubsection{Negative Symptoms}

\begin{coloredlist}
    \item \cyanit{Affective Flattening} — Restricted range of emotional expression, including facial expressions, voice tone, and body language.
    \item \cyanit{Alogia} — A lack of speech or a decrease in the amount of thought, which is reflected in a decrease in speech produced.
    \item \cyanit{Avolition} — A lack of motivation or a decrease in the ability to initiate and persist in activities.
    \item \cyanit{Anhedonia} — A lack of pleasure or a decrease in the ability to experience pleasure from activities that are normally pleasurable.
    \item Notice that these negative symptoms are shared with a lot of other disorders, including depression and anxiety, or even brain damage.
\end{coloredlist}

\subsubsection{Positive versus Negative (Summary)}

\begin{coloredlist}
    \item Positive
    \begin{coloredlist}
        \item Relatively unique and historically easier to treat.
    \end{coloredlist}
    \item Negative
    \begin{coloredlist}
        \item Common in a number of disorders.
        \item Less responsible to treatment.
        \item Usually emerge first.
    \end{coloredlist}
\end{coloredlist}

\begin{center}
    \textbf{NEW NOTES FOR 04/18/25} \\
    \hrulefill
\end{center}

\subsection{Negative Symptoms and Brain Damage}

\begin{coloredlist}
    \item \cyanit{Discordant Twins} — Twins that are discordant for a disorder, meaning that one twin has the disorder and the other does not. 
    \item Brain atrophy is larger than normal ventricles and cortical sulci.
    \item Abnormal neurological systems (physiological tests that are used to assess the function of the nervous system).
    \begin{coloredlist}
        \item Jerky or non-existent visual pursuit, and cannot do it without moving their head.
        \item Absence of a blink reflex.
        \item Poor pupillary light reactions.
        \item Unusual facial expressions.
        \item Continuous elevation of the eyebrows.
    \end{coloredlist}
    \item \cyanit{Cytoarchitectual Abnormalities} — Abnormalities in the structure of the brain cells and their organization.
    \begin{coloredlist}
        \item \textbf{Hippocampus} — Cell bodies are aligned with healthy people, but not aligned for people with schizophrenia. It indicates that this is a developmental disorder.
        \item List of reasons for why these abnormalities \textbf{MIGHT} occur:
        \begin{coloredlist}
            \item Exposure to a virus during pregnancy (e.g., influenza).
            \begin{coloredlist}
                \item Either the literal virus or the immune response to the virus from your mom.
                \item \cyanit{Seasonality Effect} — The idea that people born in late winter/early spring months are more likely to develop schizophrenia than those born in the summer months.
                \begin{coloredlist}
                    \item 2nd trimester of pregnancy is important to brain development.
                    \begin{coloredlist}
                        \item In Finland 1957, there was a flu epidemic in the winter months
                        \item \textit{Note:} Flu is not the only virus that can cause this. Measles, polio, and chicken pox can also cause this.
                    \end{coloredlist}
                    \item The seasonality is more pronounced in cities.
                    \item Other disorders that are affected by this include autism, bipolar disorder, narcolepsy, and depression.
                \end{coloredlist}
            \end{coloredlist}
            \item Vitamin D deficits
            \begin{coloredlist}
                \item \cyanit{Latitude effect} — The idea that people who live in higher latitudes (further from the equator) are more likely to develop schizophrenia than those who live closer to the equator.
            \end{coloredlist}
            \item Stress?
            \begin{coloredlist}
                \item Huttunen and Niskanen (1978) — Studied people who were born in Finland during the war and found that they had a higher incidence of schizophrenia than those who were not born during the war.
                \begin{coloredlist}
                    \item When the mother's husbands were away, the mothers became much more stressed and their babies had a higher incidence of schizophrenia.
                \end{coloredlist}
            \end{coloredlist}
            \item Babies who have pre or perinatal complications (e.g., low birth weight, hypoxia, and obstetric complications) are more likely to develop schizophrenia than those who do not have these complications.
        \end{coloredlist}
    \end{coloredlist}
\end{coloredlist}

\subsection{Which Parts of the Brain are Affected?}

\begin{coloredlist}
    \item \cyanit{Dorsolateral Prefrontal Cortex (DLPFC)} — The part of the brain that is responsible for organization, motor planning, regulation, self-reflection, directed thought, and attention.
    \begin{coloredlist}
        \item Activity here
        \begin{coloredlist}
            \item Low
            \begin{coloredlist}
                \item Blood flow and cerebral metabolism
            \end{coloredlist}
        \end{coloredlist}
        \item Post mortem studies
        \begin{coloredlist}
            \item Deterioration of DA neurons here
            \item Lower \(\text{D}_{1}\) receptors
            \begin{coloredlist}
                \item Correlated with severe negative symptoms
                \item \cyanit{D1 Receptors} — A type of dopamine receptor that is involved in the regulation of movement and cognition.
            \end{coloredlist}
        \end{coloredlist}
        \item Destroy DA input for \(\text{D}_{1}\) receptors
        \begin{coloredlist}
            \item \cyanit{Hypofrontality} — A decrease in metabolic activity in the prefrontal cortex, which is associated with negative symptoms of schizophrenia.
            \item Amphetamine?
            \begin{coloredlist}
                \item Increased blood blow and increased frontal lobe tasks.
            \end{coloredlist}
        \end{coloredlist}
        \item Why does it take so long to develop?
        \begin{coloredlist}
            \item Don't notice until synaptic pruning occurs (around 18-25 years old).
            \begin{center}
                \textbf{NEW NOTES FOR 04/21/25} \\
                \hrulefill
            \end{center}
            \item Something about hormone changes (estrogen and testosterone) that may be involved in the development of schizophrenia.
            \item Maybe not so long to develop, but rather a long time to be diagnosed.
            \begin{coloredlist}
                \item This could be due to the fact that parents do not think that their children have abnormal behavior until they reach teenage years. With the advent of cameras, we are able to video children when they are younger, and are able to see that people with schizophrenia have abnormal behavior when they are younger.
                \item Such as: more negative affect, poorer social adjustment, abnormal movements
            \end{coloredlist}
        \end{coloredlist}
    \end{coloredlist}
    \item Other places: Hippocampus (memory), amygdala (emotion), lateral temporal cortex (auditory processing), and thalamus (sensory processing).
\end{coloredlist}

\section{Positive Symptoms and Dopamine}

\begin{coloredlist}
    \item 1950s neuroleptics (antipsychotic drugs) were used to treat schizophrenia. (Old name: major tranquilizers)
    \begin{coloredlist}
        \item \cyanit{Chlorpromazine} (\textbf{Thorazine})
        \item \cyanit{Haloperidol} (\textbf{Haldol})
        \item Both of these drugs block post synatpic DA receptors and block the release of DA from presynaptic membrane.
        \item \cyanit{Dopamine Hypothesis} — The idea that schizophrenia is caused by an overactivity of dopamine in the brain.
        \begin{coloredlist}
            \item \textbf{Strengths of DA Hypothesis}
            \begin{coloredlist}
                \item Better antagonists are more effective at treating symptoms.
                \item \cyanit{Amphetamine Psychosis} — DA Agonists (e.g., amphetamines) can cause positive symptoms in healthy people.
                \begin{coloredlist}
                    \item Agonists include:
                    \begin{coloredlist}
                        \item Cocaine
                        \item Amphetamines
                        \item L-Dopa
                    \end{coloredlist}
                \end{coloredlist}
                \item Neurological symptoms in patients with Schizophrenia like excessive blinking led researchers to further believe that DA was involved in the disorder.
            \end{coloredlist}
            \item \textbf{Post Mortem Studies in HVA}
            \begin{coloredlist}
                \item Given the dopamine hypothesis, we would expect to see an increase in HVA in the brains of people with schizophrenia, but this is not the case.
                \begin{coloredlist}
                    \item \cyanit{The Revised Dopamine Hypothesis} — Instead of an increase in dopamine, we see an increase in the number of dopamine receptors in the brains of people. However, this was also not true.
                \end{coloredlist}
                \item \cyanit{Modified Dopamine Hypothesis} — The idea that schizophrenia is caused by an imbalance of dopamine in the brain, rather than an increase or decrease in dopamine levels.
                \begin{coloredlist}
                    \item What you need to know first:
                    \begin{coloredlist}
                        \item Prefrontal neurons inhibit subcortial DA activity.
                        \begin{coloredlist}
                            \item Lesion studies of the DLPFC result in an increase in HVA for mesolimbic areas.
                        \end{coloredlist}
                    \end{coloredlist}
                    \item The hypothesis:
                    \begin{coloredlist}
                        \item Loss of neurons in the brain leads to a decrease of dopamine input in the DLPFC, which leads to hypofrontality.
                        \begin{coloredlist}
                            \item This leads to negative symptoms.
                            \item And an increase in dopamine in the mesolimbic system, which leads to positive symptoms.
                        \end{coloredlist}
                        \item Overview: Too little dopamine in the mesocortial system leads to an increase in dopamine in the mesolimbic system, which leads to positive symptoms.
                    \end{coloredlist}
                \end{coloredlist}
            \end{coloredlist}
        \end{coloredlist}
    \end{coloredlist}
\end{coloredlist}

\subsection{New Drug Treatment}

\begin{coloredlist}
    \item From the prior section, we know that Schizophrenia needs
    \begin{coloredlist}
        \item An agonist in the mesocortical system (where there is too little dopamine)
        \item An antagonist in the mesolimbic system (where there is too much dopamine)
        \item \textbf{A partial agonist could do this.}
    \end{coloredlist}
\end{coloredlist}

\subsubsection{Atypical Antipsychotics}

\begin{coloredlist}
\begin{multicols}{2}
    \item \cyanit{Clozapine} (\textbf{Clozaril})
    \item \cyanit{Olanzapine} (\textbf{Zyprexa})
    \item \cyanit{Cariprazine} (\textbf{Vraylar})
    \item \cyanit{Risperidone} (\textbf{Risperdal})
    \item \cyanit{Arirpiprazole} (\textbf{Abilify})
    \item \cyanit{Brexpiprazole} (\textbf{Rexulti})
\end{multicols}
    \item These drugs are all used to treat schizophrenia and are partial agonists at the D2 receptor.
    \item An increase in treatment for positive and negative symptoms.
    \item Partial agonists are responsible for a decrease in dopamine in the mesolimbic system, and an increase in dopamine in the mesocortical system.
    \item A decrease in Tardive Dyskinesia
    \item \cyanit{Tardive Dyskinesia} — Rapid involuntary movements of the tounge, jaw, trunk, or extremities developed in association with the use of neuroleptics (20\% - 30\% of long-term users).
    \item Drugs that impact serotonin receptors: Mescaline (from cactus) and Psilocybin (from mushrooms). 
    \item Also impact serotonin receptors (increase in 1A, decrease in 2A).
    \begin{coloredlist}
        \item \cyanit{Dopamine-Serotonin Interaction Hypothesis} — 
    \end{coloredlist}
\end{coloredlist}

\begin{center}
    \textbf{NEW NOTES FOR 04/23/25} \\
    \hrulefill
\end{center}

\subsection{Phencyclidine Theory of Schizophrenia}

\begin{coloredlist}
    \item Deficit of Glutamate?
    \begin{coloredlist}
        \item People with Schizophrenia have 1/2 as much glutamate in their brains as healthy people.
        \item Then how are DA antagonists helping?
        \begin{coloredlist}
            \item Dopamine is a direct antagonistic to glutamate. That is, DA inhibits the release of Glutamate.
            \item Think: Major tranquilizers (antipsychotics) are antagonists of DA, which leads to an increase in glutamate.
        \end{coloredlist}
    \end{coloredlist}
    \item Evidence?
    \begin{coloredlist}
        \item \cyanit{PCP (Phencyclidine)} — Glutmate antagonist that blocks NMDA receptors (a type of glutamate receptor).
        \begin{multicols}{2}
            \begin{coloredlist}
                \item Hallucinations
                \item Depersonalization
                \item Cognitive disorganization
                \item Negative and hostility
                \item Frontal lobe impairments
            \end{coloredlist}
        \end{multicols}
        \item Chronic PCP use leads to a decrease in DA and metabolic activity in the DLPFC.
        \item \cyanit{Ketamine (antagonist)} — Anesthetic for children and nonhuman animals NOT adults.
        \begin{coloredlist}
            \item Causes psychotic reactions in adults (but not children).
        \end{coloredlist}
        \item Treatment possibility?
        \begin{coloredlist}
            \item Direct agonists cause seizures and brain damage, so we want to avoid those. 
            \item However, indirect agonists, such as gylcine leads to a decrease in symptomns.
            \item \cyanit{Lumateperone} (\textbf{Caplyta}) — Though this drug is used to treat schizophrenia, the mechanism of action is not well understood. However, we do know that this drug DOES work on Serotonin, dopamine, and glutamate receptors.
        \end{coloredlist}
    \end{coloredlist}
\end{coloredlist}

\section{Norepinephrine}

\begin{coloredlist}
    \item Cell bodies in the pons, medulla, and some in the thalamus.
    \item \cyanit{Locus Coeruleus (LC)} — The main source of norepinephrine in the brain. It is located in the pons and is involved in arousal, attention, and the sleep-wake cycle.
    \item The plan for this section is to go over:
    \begin{coloredlist}
        \item Receptors
        \item Functions
        \item Synthesis and metabolism
        \item Depression
    \end{coloredlist}
\end{coloredlist}

\subsection{Adrenergic Receptors}

\begin{coloredlist}
    \item \cyanit{\(\alpha\) Receptors} — Strong affinity for norepinephrine, but epinephrine can also bind to them (so far only found in the brain).
    \item \cyanit{\(\beta\) Receptors} — Strong affinity for epinephrine, but norepinephrine can also bind to them.
    \item Agonists increase BP
    \begin{coloredlist}
        \item \(\alpha\) agonist = vasoconstriction
        \item \(\beta\) agonist = force and rate of cardiac contractions
    \end{coloredlist}
    \item Inversely, \(\beta\)-blockers and \(\alpha\)-blockers decrease BP.
    \begin{coloredlist}
        \item \cyanit{Doxazosin} (\textbf{Cardura}) — \(\alpha\)-blocker that is used to treat high blood pressure and benign prostatic hyperplasia (BPH).
        \item \cyanit{Metaprolol} (\textbf{Lopressor}) — \(\beta\)-blocker that is used to treat high blood pressure and heart failure.
    \end{coloredlist}
\end{coloredlist}


\begin{center}
    \textbf{NEW NOTES FOR 04/25/25} \\
    \hrulefill
\end{center}


\subsection{Function}

\begin{coloredlist}
    \item Increases vigilance, arousal, selective attention, and orienting.
    \begin{coloredlist}
        \item With amphetamines, they increase levels of dopamine, but also norepinephrine, which leads to increased orientation.
        \begin{coloredlist}
            \item Works by blocking reuptake and causes the transporters to work in reverse.
        \end{coloredlist}
        \item \cyanit{Atomoxetine} (\textbf{Strattera}) — (ADHD tx) Reuptake inhibitor especially in the PFC.
        \item \cyanit{Guanfacine} (\textbf{Intuniv}) — (ADHD tx) \(\alpha\text{-}2A\)-agonist reduces activity in the sympathetic nervous system.
        \begin{coloredlist}
            \item Binds to autoreceptors and improves PTSD along with anxiety.
        \end{coloredlist}
    \end{coloredlist}
    \item Feeding
    \begin{coloredlist}
        \item Antagonists and agonists can increase feeding.
    \end{coloredlist}
    \item Stress response
    \begin{coloredlist}
        \item Used in the neuromuscular junction of sympathetic nervous system.
    \end{coloredlist}
\end{coloredlist}

\subsubsection{Anxiety}

\begin{coloredlist}
    \item \(\beta\)-blockers used to be used to treat anxiety.
    \begin{coloredlist}
        \item \cyanit{Propranolol} (\textbf{Inderal}) — A non-selective \(\beta\)-blocker that is used to treat high blood pressure and anxiety.
    \end{coloredlist}
    \item Now, 
    \begin{coloredlist}
        \item We use antidepressants and benzodiazepines to treat anxiety.
    \end{coloredlist}
\end{coloredlist}

\subsection{Synthesis and Metabolism}

\subsubsection{Synthesis}

\begin{coloredlist}
    \item Most NT synthesized in terminal button cytoplasm, then stored in synaptic vesicles.
    \item \textbf{NE is Different!}
    \item In a nutshell, it follows from the synthesis of dopamine. The rest of the process is on the handout.
\end{coloredlist}

\subsubsection{Metabolism}

\begin{coloredlist}
    \item On the handout.
\end{coloredlist}


\begin{center}
    \textbf{NEW NOTES FOR 04/28/25} \\
    \hrulefill
\end{center}

\subsection{Depression}

\begin{coloredlist}
    \item Some symptoms
    \begin{coloredlist}
        \item Depressed or irritable mood
        \item Slowed walking and talking
        \item Disturbed sleep
        \item Disturbed eating
        \item Avolition, apathy, anhedonia
    \end{coloredlist}
    \item Two types
    \begin{coloredlist}
        \item Major depressive disorder
        \item Bipolar (Manic-depressive disorder)
    \end{coloredlist}
\end{coloredlist}

\subsubsection{Major Depressive Disorder}

\begin{coloredlist}
    \item Low levels of NE (and serotonin… but wait)
    \begin{coloredlist}
        \item Evidence?
        \begin{coloredlist}
            \item Low levels of VMA and MHPG
            \item Antagonist
            \begin{coloredlist}
                \item Reserpine (Raudixin) (Remember from DA?)
                \item Decreased blood pressure but produce depressive sxs in 15\%
            \end{coloredlist}
            \item Agonists
            \begin{coloredlist}
                \item Treatment
            \end{coloredlist}
        \end{coloredlist}
    \end{coloredlist}
    \item Agonists as Treatments (about 30 years ago)
    \begin{coloredlist}
        \item Trycyclic antidepressants
        \begin{multicols}{2}
            \item \cyanit{Amitriptyline} (\textbf{Elavil})
            \item \cyanit{Clomipramine} (\textbf{Anafranil})
            \item \cyanit{Desipramine} (\textbf{Norpramin})
            \item \cyanit{Imipramine} (\textbf{Tofranil})
        \end{multicols}
        \item Block reuptake (5-HT too, didn’t know about the 5-HT back then)
        \item Lots of side effects
        \begin{coloredlist}
            \item Dry mouth, blurred vision, constipation, difficulty with urination, hyperthermia, drowsiness, anxiety, restlessness, cognitive and memory difficulties, confusion, dizziness, hypersensitivity reactions, increased appetite with weight gain, sweating, decrease in sexual ability and muscle twitches, weakness, nausea and vomiting, hypotension, tachycardia, and rarely, irregular heart rhythms (she wants us to know four: sleep, sex, eating, easy to OD).
        \end{coloredlist}
        \item MAO inhibitor (MAO breaks down NE and 5-HT)
        \begin{coloredlist}
            \item People took MAO inhibitor (isoniazid) for Tuberculosis and had improved mood
            \item \cyanit{Isocarboxazid} (\textbf{Marplan})
        \end{coloredlist}
        \item NE, DA, and 5-HT
        \item Cheese effect
        \begin{coloredlist}
            \item Really: cheese, yogurt, wine, yeast breads, chocolate, some fruits and nuts
        \end{coloredlist}
    \end{coloredlist}
    \item Pressor amines
    \begin{coloredlist}
        \item Sympathetic reaction
        \begin{coloredlist}
            \item As bad as intracranial bleeding, cardiovascular collapse
        \end{coloredlist}
        \item Normally broken down by MAO
    \end{coloredlist}
\end{coloredlist}



\subsubsection{More Treatments}

\begin{coloredlist}
    \item COMT Inhibitor
    \item Sleep Deprivation
    \item ECT
    \begin{coloredlist}
        \item 1930's Cerletti and Bini
        \begin{coloredlist}
            \item Gave people seizures to treat depression.
        \end{coloredlist}
        \item NOW
        \begin{coloredlist}
            \item Memory loss is very short
            \item Tx effectiveness in 80-90\% vs. 67\% (works well in 80-90\% of people who get it, but only treatment resistant people are really getting it) (67\% effectiveness from drug treatment)
            \item Seizures, Schizoaffective PD, Parkinson's, Bipolar
            \item How does it work
            \begin{coloredlist}
                \item 
            \end{coloredlist}
        \end{coloredlist}
    \end{coloredlist}
    \item Deep brain stimulation (one area ventral striatum)
    \item Vagus nerve stimulation
    \item Ketamine
    \item Bright light
    \item Transcranial magnetic stimulation
\end{coloredlist}

\begin{center}
    \textbf{NEW NOTES FOR 04/30/25} \\
    \hrulefill
\end{center}

\section{Seratonin 5-HT}

\begin{coloredlist}
    \item Sero = blood; tonus = tension
    \begin{coloredlist}
        \item Vasoconstriction
    \end{coloredlist}
    \item Where it is
    \item Receptors
    \item Function
    \item Synthesis
    \item Metabolism
\end{coloredlist}

\subsection{Where It Is}

\begin{coloredlist}
    \item Most of it is produced in the raphe nuclei of the midbrain, pons, and medulla.
    \begin{coloredlist}
        \item ``Raphe'' = seam; the middle part of the reticular formation.
    \end{coloredlist}
\end{coloredlist}

\subsection{Receptors}

\begin{coloredlist}
    \item Metabotropic
    \item At least 12 different receptors. 
    \begin{coloredlist}
        \item They all follow the same pattern of \(5\text{-HT}_{\text{Number},\text{Letter}}\).
    \end{coloredlist}
\end{coloredlist}

\subsection{Function}

\begin{coloredlist}
    \item Mood (emotionally stable, happy, calm).
    \item Sleep and dreams (increase before bed, decrease in the morning)
    \item Can cause hallucinations (mescaline, psilocybin).
    \begin{coloredlist}
        \item Remember:
        \begin{coloredlist}
            \item Mescaline — From cactus (Psychoactive)
            \item Psilocybin — From mushrooms (Psychoactive)
        \end{coloredlist}
    \end{coloredlist}
\end{coloredlist}

\subsection{Synthesis}

\begin{coloredlist}
    \item \cyanit{Tryptophan} — An amino acid that is the precursor to serotonin (and melatonin).
    \begin{coloredlist}
        \item \cyanit{Tryptophan hydroxylase (TPH)} — The enzyme that converts tryptophan to 5-hydroxytryptophan (5-HTP).
        \item \cyanit{5-HTP decarboxylase} — The enzyme that converts 5-HTP to 5-HT (5-hydroxytryptamine, or serotonin).
    \end{coloredlist}
\end{coloredlist}

\subsection{Metabolism}

\begin{coloredlist}
    \item May be taken back into the cell with no metabolism
    \item Order of events:
    \begin{coloredlist}
        \item 5-HT \(\xrightarrow{\text{MAO}}\) 5-hydroxy-indolacet-aldehyde \(\xrightarrow[\text{Dehydrogenase}]{\text{Aldehyde}}\) 5-hydroxy-indole-acetic acid (5-HIAA)
        \item 
    \end{coloredlist}
\end{coloredlist}