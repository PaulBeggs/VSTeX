\begin{center}
    \textbf{NEW NOTES FOR 03/19/25} \\
    \hrulefill
\end{center}

\section{Overview -- Unit 3}

\begin{coloredlist}
    \item Neurotransmitters and Neuromodulators
    \begin{coloredlist}
        \item Psychopharmacology
        \item Disorders
        \item Pain
    \end{coloredlist}
\end{coloredlist}

\subsection{Psychopharmacology in Detail}

\begin{coloredlist}
    \item The scientific study of the effects of drugs on the nervous system and behavior.
    \item Psychopharmacology is the study of how drugs affect the mind and behavior.
    \begin{coloredlist}
        \item Psychotherapeutic drugs
        \item Better understanding of how things normally work.
    \end{coloredlist}
\end{coloredlist}

\subsubsection{Principles of Drug Action}

\begin{coloredlist}
    \item \cyanit{Selective Action} -- Drugs are selective in their action on the nervous system.
    \begin{coloredlist}
        \item \cyanit{Sites of Action} -- The location at which a drug interacts with the body to produce its effects.
        \item Side effects are often due to the drug acting on sites other than the intended target.
        \begin{coloredlist}
            \item Thus, side effects are relative to what our preferred site of action is.
        \end{coloredlist}
        \item \textit{Example}: Opioids primarily affect the opioid receptor system.
    \end{coloredlist}
    \item Drugs don't CREATE effects, they \textit{modulate} ongoing cellular activity.
    \begin{coloredlist}
        \item That is, they affect behavior by affecting neural transmission in some way.
        \item \cyanit{Agonist} -- A drug that mimics or enhances the \textbf{effects of a neurotransmitter}.
        \begin{coloredlist}
            \item Facilitates post synaptic effects.
            \item \textit{Example}: Morphine mimics endorphins, which are natural painkillers.
        \end{coloredlist}
        \item \cyanit{Antagonist} -- A drug that blocks or inhibits the effects of a neurotransmitter.
        \begin{coloredlist}
            \item Inhibits post synaptic effects.
            \item \textit{Example}: Naloxone blocks the effects of opioids, reversing their effects.
        \end{coloredlist}
        \item Agonistic effects can become antagonistic if the drug is taken in excess.
        \begin{coloredlist}
            \item \textit{Example}: I make a neuron fire a neurotransmitter, but I also block the reuptake of that neurotransmitter.
        \end{coloredlist}
    \end{coloredlist}
\end{coloredlist}

\subsubsection{Basic Process}

\begin{coloredlist}
    \item \cyanit{Precursor} -- A substance from which another substance is formed. AKA, the ingredients used to make a neurotransmitter.
    \item \[
        \text{Precursor} \xrightarrow{\text{Synthetic Enzyme}} \text{NT} \xrightarrow{\text{Metabolic Enzyme}} \text{Inactive Metabolite}
    \]
    \begin{coloredlist}
        \item \cyanit{Synthesis} -- The process of creating a neurotransmitter from its precursors.
        \item Sometimes, we break down the precursor to build the neurotransmitter.
        % \item \cyanit{Storage} -- The neurotransmitter is stored in vesicles until it is needed.
        % \item \cyanit{Release} -- The neurotransmitter is released into the synaptic cleft when an action potential arrives at the axon terminal.
        % \item \cyanit{Binding} -- The neurotransmitter binds to receptors on the post-synaptic neuron, causing a change in the neuron's activity.
        % \item \cyanit{Inactivation} -- The neurotransmitter is removed from the synaptic cleft by reuptake or enzymatic degradation.
    \end{coloredlist}
\end{coloredlist}

\begin{center}
    \textbf{NEW NOTES FOR 03/21/25} \\
    \hrulefill
\end{center}

\subsection{Other Ways of Agonist and Antagonist}

\begin{coloredlist}
    \item Block Ca\(^{2+}\) channels from opening (antagonist)
    \item \cyanit{Mimetic} -- Mimics the action of a neurotransmitter.
    \begin{coloredlist}
        \item \cyanit{Direct Agonist} -- Binds to the same receptor as the neurotransmitter and mimics its effects.
        \item \cyanit{Indirect Agonist} -- Binds to a different site on the receptor and enhances the effects of the neurotransmitter.
    \end{coloredlist}
    \item Blocking agent
    \begin{coloredlist}
        \item Competitive
        \begin{coloredlist}
            \item \cyanit{Direct Antagonist} -- Binds to the same receptor as the neurotransmitter and blocks its effects.
        \end{coloredlist}
        \item Non-competitive
        \begin{coloredlist}
            \item \cyanit{Indirect Antagonist} -- Binds to a different site on the receptor and blocks the effects of the neurotransmitter.
        \end{coloredlist}
        \item \cyanit{Inverse Agonist} -- Binds to the same receptor as the neurotransmitter and produces the opposite effect.
    \end{coloredlist}
    \item \cyanit{Depolarizing} or \cyanit{Desensitizing Agent} -- A drug that causes the AP to stay in a depolarized state; refusing to let the neuron go through another AP, and it stays in the absolute refractory period. (Antagonist)
    \item Interfere with vesicles (leaky or transporter proteins). (Antagonist)
    \item Interfere with docking proteins. (Antagonist)
    \item Selectively deactivate autoreceptors. (Agonist)
    \item Selectively activate autoreceptors. (Antagonist)
\end{coloredlist}

\section{What Do These Chemicals Between Neurons Do?}

\begin{coloredlist}
    \item Transmit information.
    \begin{coloredlist}
        \item Glutamate
        \item GABA
        \item Glycine
    \end{coloredlist}
    \item Modulate information.
    \begin{coloredlist}
        \item Every other neurotransmitter.
    \end{coloredlist}
\end{coloredlist}

\section{Classes of Neurotransmitters (Revisited from Lab)}

\begin{coloredlist}
    \item Amino Acids (book 4.3.1)
    \begin{coloredlist}
        \item Glutamate, GABA, Glycine
    \end{coloredlist}
    \item \cyanit{Amines} (monoamines) -- Derived from amino acids
    \begin{coloredlist}
        \item Catecholamines
        \begin{coloredlist}
            \item Contain catechol and derived from the amino acid tyrosine.
            \item \cyanit{Tyrosine} -- Precursor for the catecholamines.
            \item Dopamine (DA), Norepinephrine (NE), Epinephrine (Adrenaline)
            \item Dopaminergic, Adrenergic, and Noradrenergic systems.
        \end{coloredlist}

\begin{center}
    \textbf{New Notes for 03/31/25} \\
    \hrulefill
\end{center}

        \item \cyanit{Indolamines} %-- Derived from the amino acid tryptophan.
        \begin{coloredlist}
            \item \cyanit{Serotonin} (5-HT) %-- Derived from the amino acid tryptophan.
            \item \cyanit{Melatonin} %-- Derived from serotonin.
        \end{coloredlist}
        \item Peptides (AKA: Neuropeptides)
        \begin{coloredlist}
            \item Endogenous Opioids 
        \end{coloredlist}
        \item Acetylcholine (ACh)
        \item Lipids (book 4.3.5)
        \begin{coloredlist}
            \item Anadamide (Sanskrit for ``bliss'') -- Endogenous cannabinoid.
        \end{coloredlist}
        \item \textbf{Two Other Classes (from book)}
        \begin{coloredlist}
            \item Nucleosides
            \begin{coloredlist}
                \item Adenosine
            \end{coloredlist}
            \item Soluble Gases
            \begin{coloredlist}
                \item \cyanit{Nitric Oxide (NO)} -- Required for an erection.
            \end{coloredlist}
        \end{coloredlist}
    \end{coloredlist}
\end{coloredlist}

\section{Acetylcholine (ACh)}

\begin{coloredlist}
    \item First neurotransmitter discovered.
    \begin{coloredlist}
        \item Otto von Loewy -- Discovered ACh in 1921.
        \begin{coloredlist}
            \item This guy took a frog heart and put it in saline. Then, he took simulated the parasympathetic part of the vagus nerve, and saw that the heart slowed down. 
            \item He then took the saline and put it in a different frog heart, and saw that the heart slowed down again.
            \item ``Vagusstoff'' (ACh) -- The chemical that was released from the vagus nerve that slowed the heart down.
            \item Cholinergic -- Referring to ACh.
        \end{coloredlist}
    \end{coloredlist}
    \item \textbf{\textit{Some} Functions}
    \begin{coloredlist}
        \item \textbf{Function in the ANS:}
        \begin{coloredlist}
            \item Sympathetic
            \begin{coloredlist}
                \item Spinal nerve leaves the cord and synapses in the paravertebral ganglion (ACh)
                \item Then makes neuromuscular junction with smooth muscles and glands (NE)
                \begin{coloredlist}
                    \item \cyanit{Neuromusclar Junction} -- The synapse between a motor neuron and a muscle fiber.
                    \item \cyanit{Paravertebral Ganglion} -- A ganglion located next to the spinal cord.
                \end{coloredlist}
                \begin{coloredlist}
                    \item Except sweat glands (ACh)
                \end{coloredlist}
                \item \cyanit{Sympathetic Chain} -- A chain of ganglia that runs parallel to the spinal cord. This is the reason for when you get anxious, ALL of your body gets anxious.
            \end{coloredlist}
            \item Parasympathetic
            \begin{coloredlist}
                \item Spinal nerve leaves the cord and synapse in the parasympathetic ganglion (ACh)
                \item Then makes neuromuscular junction with smooth muscles and glands (ACh)
            \end{coloredlist}
            \item The only NT in the parasympathetic branch.
            \item NT of the preganglionic sympathetic branch.
        \end{coloredlist}
        \item \textbf{Function in the Somatic NS}
        \begin{coloredlist}
            \item Excites the neuromusclar junction (ACh)
            \item So, ACh is important for getting motor messages out to all kinds of muscles and glands.
        \end{coloredlist}
        \begin{center}
            \textbf{NEW NOTES FOR 04/02/25} \\
            \hrulefill
        \end{center}
        \item \textbf{Function in the CNS}
        \begin{coloredlist}
            \item ACh is important in:
            \begin{coloredlist}
                \item Learning and alertness (\cyanit{Basal Forebrain})---activates the cortex and facilitates learning.
                    \begin{coloredlist}
                        \item \cyanit{Nucleus Basalis} -- Projects to the cortex
                        \item \cyanit{Medial Septal Nucleus} and \cyanit{Nucleus of Diagonal Band} -- Projects to the hippocampus through the fornix.
                    \end{coloredlist}
                \item Memory (\cyanit{medial septal nucleus})---modulate the hippocampus
                \item REM sleep generation (\cyanit{Pedunculopontine nucleus (PPT)} and \cyanit{Laterodorsal Tegmental Nucleus (LDT)})---projects to the pons and thalamus.
                \item Reward system.
            \end{coloredlist}
        \end{coloredlist}
    \end{coloredlist}
    \item \textbf{Synthesis and Metabolism}
    \item \textbf{Drugs and Disorders}
\end{coloredlist}

\subsection{ACh Synthesis and Metabolism}

\begin{coloredlist}
    \item \textbf{Synthesis}
    \begin{coloredlist}
        \item In a nutshell: A breakdown of lipids leads to Choline, which is the precursor for ACh. Acetate is the anion in vinegar (Acetic acid). Then, this is combined with Acetate to make ACh.
        \item In more detail:
        \begin{coloredlist}
            \item CoA attaches to an acetate ion (\cyanit{Acetylcoenzyme A (acetyl-CoA)}).
            \item Then, \cyanit{choline acetyltransferase (ChAT)} transfers the acetate from the acetyl-CoA to the choline molecule.
            \item Mnemonic: \textbf{ChAT}: From right to left: Transfers acetate to choline.
        \end{coloredlist}
    \end{coloredlist}
    \item \textbf{Metabolism}
    \begin{coloredlist}
        \item ACh is broken down by the enzyme \cyanit{acetylcholinesterase (AChE)} into acetate and choline. Nice and simple!
        \item The choline is taken back up by active transport and reused, and the acetate is broken down and eliminated.
    \end{coloredlist}
\end{coloredlist}

\subsection{Two Types of Cholinergic Receptors}

\begin{coloredlist}
    \item Nicotinic Receptors
    \begin{coloredlist}
        \item Agonist at low doses, but antagonist at high doses.
        \item Iontropic.
        \item Found at the Neuromusclar Function in the PNS.
        \item \cyanit{Curare} (direct antagonist) -- A drug that blocks nicotinic receptors, causing paralysis.
        \begin{coloredlist}
            \item Competitive blocking agent
            \item Paralysis, surgery
        \end{coloredlist}
    \end{coloredlist}
    \item Muscarinic Receptors
    \begin{coloredlist}
        \item Comes from a hallucinogenic mushroom (Amanita muscaria).
        \begin{coloredlist}
            \item \textbf{Don't confuse with Serotonin's Mescaline: Cactus; nor Psilocybin: Mushroom.}
        \end{coloredlist}
        \item Vikings (probably took this drug before raiding) and Koryaks (Nordic people who used this mushroom in religious practices).
        \item Metabotropic receptors.
        \item Predominates in the CNS (although, both types are found in the CNS).
        \item \cyanit{Atropine} (direct antagonist) -- A drug that blocks muscarinic receptors, causing pupil dilation and increased heart rate.
        \begin{coloredlist}
            \item Competitive blocking agent
            \item Belladonna alkaloids (deadly nightshade)
        \end{coloredlist}
    \end{coloredlist}
\end{coloredlist}

\begin{center}
    \textbf{NEW NOTES FOR 04/04/25} \\
    \hrulefill
\end{center}

\section{MORE Drugs and Toxins Affecting ACh}

\begin{coloredlist}
    \item \cyanit{Botulinum Toxin} -- A waste product of \textit{Clostridium botulinum}, which are bacteria who grows without oxygen. 
    \begin{coloredlist}
        \item Interferes with Ca\(^{2+}\) influx channels, preventing the release of ACh.
        \item Because Botox causes paralysis, it can interfere with emotional \textit{expression} because it paralyzes muscles like the orbicularis oculi.
        \item Additionally, since we know that expression influences experience, when we paralyze these muscles, then the emotional \textit{experience} is also negatively affected.
        \item \textbf{Does Botox Decrease Emotional Experiences?}
        \begin{coloredlist}
            \item Population: Women who want wrinkles gone.
            \item One IV: two levels: Botox or restylane (dermal filler).
            \item Method: Everyone had wrinkle reduction. AND, Everyone watches some emotion evoking movies.
            \item Results: Botox group had less emotional experience than the restylane group.
            \item \textbf{Is this a good thing?}
            \begin{coloredlist}
                \item Another study takes a sample of depressed people and gives them either Botox or a placebo.
                \item Results: 15\% of placebo had a decrease in depression, while 52\% of the Botox group had a decrease in depression.
            \end{coloredlist}
        \end{coloredlist}
        \item Botox can also be used to treat migraines, cerebral palsy, and hyperhidrosis (excessive sweating).
    \end{coloredlist}
    \item \cyanit{Black Widow Spider Venom} -- A neurotoxin that causes the release of ACh at the neuromuscular junction, causing continual release of ACh and paralysis.
    \item \cyanit{Cobra and Krait Venom} -- A neurotoxin that blocks the binding of ACh to nicotinic receptors, causing paralysis.
    \item \cyanit{AchE Blockers} -- Comes into contact with the enzyme that breaks down ACh, causing an increase in ACh in the synaptic cleft.
    \begin{coloredlist}
        \item Irreversible
        \begin{center}
            \textbf{NEW NOTES FOR 04/07/25} \\
            \hrulefill
        \end{center}
        \begin{coloredlist}
            \item Insecticides (Parathion)
            \item Nerve gas: DFP (Diisopropylfluorophosphate (don't need to know the whole name)) and Sarin.
            \begin{coloredlist}
                \item Readily crosses the blood-brain barrier so PNS and CNS are affected.
                \item Antidote? 
                \begin{coloredlist}
                    \item \cyanit{Atropine} -- A drug that blocks muscarinic receptors, preventing the effects of excess ACh.
                    \item \cyanit{Pralidoxime} -- A drug that reactivates AChE, allowing it to break down ACh again.
                \end{coloredlist}
            \end{coloredlist}
        \end{coloredlist}
        \item Reversible
        \begin{coloredlist}
            \item \cyanit{Neostgmine} (\textbf{Prostigmin}) and \cyanit{Physostigmine} (\textbf{Antilirum}) -- Drugs that inhibit AChE, increasing the amount of ACh in the synaptic cleft.
            \begin{coloredlist}
                \item Doesn't cross the blood-brain barrier, so it only affects the PNS.
                \item Used to treat \cyanit{myasthenia gravis} (a disease that causes muscle weakness and fatigue).
                \begin{coloredlist}
                    \item Autoimmune disease that attacks nicotinic receptors at the neuromuscular junction.
                \end{coloredlist}
            \end{coloredlist}
            \item \cyanit{Donepezil} (\textbf{Aricept}) and \cyanit{rivastigmine} (\textbf{Exelon}) -- These drugs do the same thing as the above drugs, but they cross the blood-brain barrier and are used to treat Alzheimer's disease and Parkinson's disease (only the cognitive part).
        \end{coloredlist}
    \end{coloredlist}
\end{coloredlist}

\section{New Drug for Schizophrenia}

\begin{coloredlist}
    \item We'll talk about dopamine drugs later in this unit.
    \item This new drug now:
    \begin{coloredlist}
        \item \cyanit{Xanomelne and trospium chloride} (\textbf{Cobenfy}) -- A drug that blocks the muscarinic receptors in the CNS, but not in the PNS.
        \item Dopamine but also Ach!
    \end{coloredlist}
\end{coloredlist}

\section{Catecholamines}

\begin{coloredlist}
    \item Dopamine (DA)
    \item Norepinephrine (NE)
    \item Epinephrine (Adrenaline)
\end{coloredlist}

\subsection{Dopamine (DA)}

\begin{coloredlist}
    \item Synthesis and Metabolism
    \item Function
    \item Drugs and Disorders
\end{coloredlist}

\subsubsection{Dopamine Synthesis}

\begin{coloredlist}
    \item Tyrosine was first discovered from cheese (tyrosine = cheese).
    \begin{coloredlist}
        \item Tyrosine is the precursor for DA, NE, and Epi.
        \item \cyanit{Tyrosine Hydroxylase} -- The rate-limiting enzyme in the synthesis of catecholamines.
        \begin{coloredlist}
            \item Converts tyrosine to L-DOPA.
            \item L-DOPA is the precursor for DA, NE, and Epi.
        \end{coloredlist}
        \item L-DOPA is converted to DA by the enzyme \cyanit{DOPA decarboxylase}.
        % \item DA is converted to NE by the enzyme \cyanit{Dopamine \(\beta\)-hydroxylase (DBH)}.
        % \item NE is converted to Epi by the enzyme \cyanit{Phenylethanolamine N-methyltransferase (PNMT)}.
    \end{coloredlist}
\end{coloredlist}

\subsubsection{Dopamine Metabolism}

\begin{coloredlist}
    \item DA is broken down by the enzyme \cyanit{Monoamine Oxidase (MAO)} into \cyanit{Dihydroxyphenylacetric acid (DOPAC)}.
    \item Then, \cyanit{Catechol-O-methyltransferase (COMT)} converts DOPAC into \cyanit{Homovanillic acid (HVA)}.
    \item Also, starting from DA, we can use COMT to convert it to 3-methoxytyramine (3-MT), then with MAO, we can convert it to HVA. 
\end{coloredlist}

\subsubsection{DA Function}

\begin{coloredlist}
    \item Movement/Motor systems
    \begin{coloredlist}
        \item \cyanit{Nigrostriatal System} -- Starts in the substantia nigra and ends in the striatum (caudate nucleus and putamen).
        \item Here's the route: We start at the striatum, which then sends an inhibitory GABA signal to the substantia nigra, who sends a reciprocal inhibitory DOPA signal back to the striatum nerve that sends an inhibitory GABA signal to the globus pallidus. Then, the globus pallidus excites the thalamus, who then excites the primary motor cortex, who then excites movement.
        \begin{coloredlist}
            \item Note that if the inhibitory signal to the substantia nigra is limited, then the signal that the striatum sends to the globus pallidus is much stronger, which leads to a weaker signal to the thalamus, and thus to movements.
        \end{coloredlist}
        \item Parkinson's Disease symptoms:
        \begin{coloredlist}
            \item Weakness,
            \item Tremor at rest,
            \item Muscle rigidity,
            \item Problems with balance,
            \item Abnormal gait,
            \item Trouble learning
        \end{coloredlist}
        \item \textbf{How long have we known?}
        \begin{coloredlist}
            \item \cyanit{Reserpine} (\textbf{Raudixin}) for \(\downarrow\) \textbf{BP} (Not in use anymore because it caused Parkinson's-like symptoms)
            \begin{coloredlist}
                \item Blocks monoamine transporters
            \end{coloredlist}
        \end{coloredlist}
    \end{coloredlist}
    \item Behavioral Arousal and Attention
    \item Reinforcement and Reward
\end{coloredlist}


