\section{Prehistoric}

\begin{coloredlist}
    \item A million years or more, people have been interested in the brain. Archaeological evidence shows that skulls are bashed in (jagged, not precise). As a result, the person dies, and therefore the brain is vital to life.
\end{coloredlist}

\section{7000 Years Ago}

\begin{coloredlist}
    \item New holes in the brain, but these holes show signs of healing. Therefore, these new holes are intended to help the person who is suffering.
    The fancy name is trephination.
    \item The theory for these holes is that they were drilled to cure the person. In other words, to relieve a person of a wicked spirit.
\end{coloredlist}

\section{5000 Years Ago}

\begin{coloredlist}
    \item \cyanit{Egyptian} physicians show that they were aware of brain damage through their writings.
    \item Complications arise because they thought the heart contained the soul--you need it to live and emotions effect it.
\end{coloredlist}

\section{Ancient Greece--4th Century, BC}

\label{person:hippocrates}%
\subsection{\cyanit{Hippocrates}}

\begin{coloredlist}
    \item Ponder the correlation between structure and function. Now, extend this thought to the brain/head.
    \item The brain is the place where sensation and intelligence reside. Not the heart.
\end{coloredlist}

\label{person:aristotle}%
\subsection{\cyanit{Aristotle}}

\begin{coloredlist}
    \item Clung to the idea of the heart being the one in charge.
    \item Figured the brain was a radiator. That is, we would send heated blood to the brain for it to be cooled off. This ``heated blood'' arose from our emotions. Thus, humans are more rational because we have a lot of cooling when compared to other animals.
\end{coloredlist}
\label{person:galen}%
\section{Roman Empire--\cyanit{Galen} 2nd Century, AD}

\begin{coloredlist}
    \item Galen is a physician to gladiators.
    \item Thought the cerebellum was for motor control (because the cerebellum is hard, like muscles) and the cerebrum is for memory because it is soft, and you can ``write on it.''
    \item Noticed there were large spaces (called ``ventricles,'' or ``spaces'') that were filled with fluid.
    \item From here, we get the four humors (fluids).
    \item Galen thought that these fluids are what control the brain, NOT the brain structure itself. Think of the purpose of canned vegetables. The tin container does not actively contribute to the liquid / vegetables; rather, it is disposable.
    \item These ideas were jumpstarted by the invention of aqueducts. The movement of water was so important from aqueducts, so the idea this idea was extended to the brain.
\end{coloredlist}

\section{Analysis by Analogy--17th Century}

\begin{coloredlist}
    \item \cyanit{French} developed hydraulically controlled machines.
    \item Again, this is adding to the idea that liquids (which can flow through things and cause movements) are responsible for the brain's functionality.
\end{coloredlist}
\label{person:descartes}%
\section{\cyanit{René Descartes}--1596-1650}

\begin{coloredlist}
    \item Believed that non-humans---what he called animals---are controlled by fluid.
    \item From this, he posited that the human body is a material entity functioning as a machine (like animals)--these are known as reflexes.
    \item But, the mind is nonmaterial and free from the laws of the universe and was uniquely human.
    \item Question: How does the nonmaterial part of the body (the mind) communicate with the material part of the body? Through the pineal gland! This gland would move around like a joystick and would manipulate the fluid that came from the third ventricle.
\end{coloredlist}

\section{The Mind/Body Problem}

\begin{coloredlist}
    \item What is the basic relationship between mental events and physical events?
    \item \cyanit{Dualism}--The mind exists independently of the brain and exerts some control over it.
    \item Strengths: Commonsense view.
    \item Weaknesses: The universe is composed of matter or energy.
    \item Modern neuroscientific explanation: Everything the body does rests on the events taking place in specific, definable parts of the nervous system--the ``mind'' is the product of the nervous system activity.
\end{coloredlist}

\section{The Scientific Method--17th and 18th Century}

\begin{coloredlist}
    \item A new world view at the end of the Renaissance.
    \begin{coloredlist}
        \item Replace \cyanit{Rationalism} with \cyanit{Scientific Method}.
    \end{coloredlist}
    \item Closer look at the substance of the brain:
    \begin{coloredlist}
        \item Gray and white matter change the way we look at the brain. That is, why would these parts of the brain that are clearly different, be different if the brain is used just to move fluids around.
        \item Also, everyone has the same brain structure, so these bumps and groves must mean something.
    \end{coloredlist}
\end{coloredlist}

\section{Electricity}

\begin{coloredlist}
    \item \label{person:newton}\cyanit{Isaac Newton} showed it is possible to electrically stimulate nerves.
    \item Then, \label{{person:galvani}} \cyanit{Luigi Galvani} and \label{person:bois-reymond}\cyanit{Emil du Bois-Reymond} showed that electricity can make muscles contract.
    \item Later on, \cyanit{Hermann von Helmholtz} showed that the speed of nerve conduction is not instantaneous.
    \item This important distinction shows that these nerves are not like wires--such as \cyanit{Luigi Galvani} and \cyanit{Emil du Bois-Reymond} thought.
    \item \cyanit{Bell} and \cyanit{Magendie} showed that the dorsal nerve root and the ventral nerve root are different.
    \begin{coloredlist}
        \item Specifically, Bell showed that the ventral nerve root is for motor information, and Magendie showed that the dorsal nerve root is for sensory information.
    \end{coloredlist}
    \item The dorsal nerve root is for sensory information, and the ventral nerve root is for motor information.
    \item \cyanit{Dorsal} = \cyanit{Sensory}: Think of the dorsal fin of a shark sensing vibrations in the water.
    \item \cyanit{Ventral} = \cyanit{Motor}: Think of a vent (like a car exhaust) pushing out movement.
    \item \hypertarget{person:Muller}{}\cyanit{Johannes M\"uller} came up with the doctrine of \cyanit{Specific Nerve Energies}.
    \begin{coloredlist}
        \item This doctrine states that the nature of a sensation depends on which nerve is stimulated, not on how the nerve is stimulated.
        \item For example, if you stimulate the optic nerve, you will see something. If you stimulate the auditory nerve, you will hear something.
    \end{coloredlist}
    \item Spawned the \cyanit{Great Debate}: Is the brain a homogenous mass or is it made up of different parts?
\end{coloredlist}

\section{The Great Debate}

\begin{coloredlist}
    \item \cyanit{Franz Joseph Gall} and \cyanit{Johann Spurzheim} thought the bumps and groves on the head were due to the size of the brain parts.
    \item They concluded that the size of the brain parts was correlated to the use of that part.
    \item This is known as \cyanit{phrenology}.
    \item \cyanit{Localization of Functions}--brain function can be localized to regions, pathways, or neurons.
    \begin{coloredlist}
        \item Basically, if you cut out a piece of brain, and the animal (a pigeon) is no longer able to do a specific task, then that part of the brain is responsible for that task.
        \item However, it turns out that these pigeons were able to relearn the task, so the brain is not as localized as we thought (this research is from Flourens).
    \end{coloredlist}
    \item \cyanit{Aggregate Field Theory}--the brain is a homogenous mass.
    \begin{coloredlist}
        \item Complex brain functions emerge from the collective interactions of numerous simple neuronal activities.
        \item Unlike localizationist models, this theory emphasizes the distributed nature of cognitive processes across neural networks.
    \end{coloredlist}
    \item \cyanit{Pierre Flourens} (1794--1867)
    \begin{coloredlist}
        \item Studied the effect of brain damage with pigeons and supported the Aggregate Field Theory.
    \end{coloredlist}
    \item \cyanit{Paul Broca} (1824--1880)
    \begin{coloredlist}
        \item Found a patient who \cyanit{could speak} but could \cyanit{not understand language}.
        \item After the patient died, Broca found a lesion in the \cyanit{left frontal lobe}.
        \item This area is now known as \cyanit{Broca's area}.
        \item This area is responsible for \cyanit{speech production}.
        \item These results put us back into the realm of Localization of Function.
    \end{coloredlist}
    \item In comes \cyanit{Carl Wernicke} (1874)
    \begin{coloredlist}
        \item Found a patient who \cyanit{could understand language} but could \cyanit{not produce language}.
        \item After the patient died, Wernicke found a lesion in the \cyanit{left temporal lobe}.
        \item This area is now known as \cyanit{Wernicke's area}.
        \item This area is responsible for \cyanit{language comprehension}.
    \end{coloredlist}
    \item Then, we have \cyanit{Gustav Fritsch} and \cyanit{Eduard Hitzig} (1870)
    \begin{coloredlist}
        \item Similarly to \cyanit{Luigi Galvani} and \cyanit{Emil du Bois-Reymond}, they electrically stimulated the brain.
        \item They found that the \cyanit{motor cortex} is responsible for \cyanit{movement}.
    \end{coloredlist}
    \item \cyanit{Shepherd Ivory Franz} (in D.C. from 1907--1924)
    \begin{coloredlist}
        \item Found that people are able to relearn tasks after brain damage.
    \end{coloredlist}
\end{coloredlist}

\section{Same Resolution?}

\begin{coloredlist}
    \item \cyanit{Modified Aggregate Field Theory}
    \begin{coloredlist}
        \item \cyanit{Karl S. Lashley} (1890-1958)
        \begin{coloredlist}
            \item \cyanit{The Principles of Mass Action}
            \begin{coloredlist}
                \item Complex behavior--such as learning--is dependent on the total mass of the brain.
            \end{coloredlist}
            \item \cyanit{Equipotentiality}
            \begin{coloredlist}
                \item Specialization of function is not tied to specific brain regions.
                \item All parts of the cortex contribute equally to complex behavior.
            \end{coloredlist}
            \item \cyanit{Vicarious functioning}
            \begin{coloredlist}
                \item If one part of the brain is damaged, another part can take over.
            \end{coloredlist}
        \end{coloredlist}
    \end{coloredlist}
\end{coloredlist}

\section{Analysis}

\begin{enumerate}
    \item \textbf{Prehistoric}: Recognition of the brain's vital role in life through skull injuries. No scientific theories yet.

    \item \textbf{7000 Years Ago}: Trephination (skull drilling) practiced to release ``evil spirits,'' indicating early medical intervention.

    \item \textbf{5000 Years Ago}: Egyptians documented brain damage but prioritized the heart as the seat of the soul.

    \item \textbf{Ancient Greece—Hippocrates (4th Century BCE)}: Proposed the brain as the center of sensation/intelligence, countering heart-centric views.

    \item \textbf{Ancient Greece—Aristotle}: Defended the heart as the command center, viewing the brain as a blood-cooling ``radiator.''

    \item \textbf{Roman Empire—Galen (2nd Century CE)}: Linked cerebellum to motor control and cerebrum to memory; emphasized ventricular fluids (humors) over brain structure.

    \item \textbf{17th Century (Analysis by Analogy)}: Hydraulic systems inspired fluid-based brain theories.

    \item \textbf{René Descartes (1596–1650)}: Dualism (mind vs. body); proposed pineal gland as the mind-body interface.

    \item \textbf{17th–18th Century (Scientific Method)}: Shift to empirical study; recognition of gray/white matter differences.

    \item \textbf{Electricity Discoveries}: Newton (nerve stimulation), Galvani/du Bois-Reymond (muscle contraction via electricity), Helmholtz (nerve conduction speed), Bell/Magendie (sensory/motor nerve roots), Müller (specific nerve energies).

    \item \textbf{The Great Debate}:
          \begin{table}[h]
              \centering
              \caption{Key Figures in the Great Debate: Localization vs.\ Aggregate Theory}
              \label{tab:great_debate}
              \begin{tabular}{@{}ll@{}}
                  \toprule
                  \textbf{Localization} & \textbf{Aggregate Theory} \\
                  \midrule
                  Johannes Müller       & Pierre Flourens           \\
                  Franz Joseph Gall     & Shepherd Ivory Franz      \\
                  Paul Broca            &                           \\
                  Carl Wernicke         &                           \\
                  Gustav Fritsch        &                           \\
                  Eduard Hitzig         &                           \\
                  \bottomrule
              \end{tabular}
          \end{table}

    \item \textbf{Modified Aggregate Theory}: Karl Lashley emphasized mass action and equipotentiality.
\end{enumerate}

\begin{landscape}
    \pagestyle{plain}
    \vfill
    \begin{table}[h]
        \centering
        \caption{Key Scientists and Contributions}
        \label{tab:neuroscientists}
        \begin{tabular}{p{5.2cm}p{9.5cm}}
            \toprule
            \textbf{Scientist}    & \textbf{Contributions}                              \\
            \midrule
            Hippocrates           & Brains as seat of sensation/intelligence                                   \\
            Aristotle             & Heart as command center; brain as radiator                                 \\
            Galen                 & Cerebellum (motor), cerebrum (memory); humors                              \\
            René Descartes        & Mind-body dualism; pineal gland                                            \\
            Isaac Newton          & Early nerve stimulation via electricity                                    \\
            Luigi Galvani         & Electricity-induced muscle contraction                                     \\
            Emil du Bois-Reymond  & Same as Galvani                                                            \\
            Hermann von Helmholtz & Measured nerve conduction speed                                            \\
            Charles Bell          & Ventral nerve = motor                                                      \\
            François Magendie     & Dorsal nerve = sensory                                                     \\
            Johannes Müller       & Doctrine of specific nerve energies                                        \\
            Franz Joseph Gall     & Phrenology (brain localization)                                            \\
            Johann Spurzheim      & Promoted phrenology                                                        \\
            Pierre Flourens       & Aggregate theory                                                           \\
            Paul Broca            & Localized speech production (Broca's area)                                 \\
            Carl Wernicke         & Localized language comprehension (Wernicke's area)                         \\
            Gustav Fritsch        & Mapped motor cortex                                                        \\
            Eduard Hitzig         & Same as Fritsch                                                            \\
            Shepherd Ivory Franz  & Relearning post-brain damage                                               \\
            Karl S. Lashley       & Mass action, equipotentiality                                              \\
            \bottomrule
        \end{tabular}

    \end{table}
    \vfill

\end{landscape}