% \section{Acetylcholine (ACh)}
%     • First neurotransmitter discovered.
%         • Otto von Loewy -- Discovered ACh in 1921.
%             • This guy took a frog heart and put it in saline. Then, he took simulated the parasympathetic part of the vagus nerve, and saw that the heart slowed down.
%             • He then took the saline and put it in a different frog heart, and saw that the heart slowed down again.
%             • ``Vagusstoff'' (ACh) -- The chemical that was released from the vagus nerve that slowed the heart down.
%             • Cholinergic -- Referring to ACh.
%     • \textbf{\textit{Some} Functions}
%         • \textbf{Function in the ANS:}
%             • Sympathetic
%                 • Spinal nerve leaves the cord and synapses in the paravertebral ganglion (ACh)
%                 • Then makes neuromuscular junction with smooth muscles and glands (NE)
%                     • \cyanit{Neuromusclar Junction} -- The synapse between a motor neuron and a muscle fiber.
%                     • \cyanit{Paravertebral Ganglion} -- A ganglion located next to the spinal cord.
%                     • Except sweat glands (ACh)
%                 • \cyanit{Sympathetic Chain} -- A chain of ganglia that runs parallel to the spinal cord. This is the reason for when you get anxious, ALL of your body gets anxious.
%             • Parasympathetic
%                 • Spinal nerve leaves the cord and synapse in the parasympathetic ganglion (ACh)
%                 • Then makes neuromuscular junction with smooth muscles and glands (ACh)
%             • The only NT in the parasympathetic branch.
%             • NT of the preganglionic sympathetic branch.
%         • \textbf{Function in the Somatic NS}
%             • Excites the neuromusclar junction (ACh)
%             • So, ACh is important for getting motor messages out to all kinds of muscles and glands.
%         • \textbf{Function in the CNS}        
%             • ACh is important in:            
%                 • Learning and alertness (\cyanit{Basal Forebrain})---activates the cortex and facilitates learning.                
%                     • \cyanit{Nucleus Basalis} -- Projects to the cortex
%                     • \cyanit{Medial Septal Nucleus} and \cyanit{Nucleus of Diagonal Band} -- Projects to the hippocampus through the fornix.                
%                 • Memory (\cyanit{medial septal nucleus})---modulate the hippocampus
%                 • REM sleep generation (\cyanit{Pedunculopontine nucleus (PPT)} and \cyanit{Laterodorsal Tegmental Nucleus (LDT)})---projects to the pons and thalamus.
%                 • Reward system.
%     • \textbf{Synthesis and Metabolism}
%     • \textbf{Drugs and Disorders}
% \subsection{ACh Synthesis and Metabolism}
%     • \textbf{Synthesis}
%         • In a nutshell: A breakdown of lipids leads to Choline, which is the precursor for ACh. Acetate is the anion in vinegar (Acetic acid). Then, this is combined with Acetate to make ACh.
%         • In more detail:
%             • CoA attaches to an acetate ion (\cyanit{Acetylcoenzyme A (acetyl-CoA)}).
%             • Then, \cyanit{choline acetyltransferase (ChAT)} transfers the acetate from the acetyl-CoA to the choline molecule.
%             • Mnemonic: \textbf{ChAT}: From right to left: Transfers acetate to choline.
%     • \textbf{Metabolism}
%         • ACh is broken down by the enzyme \cyanit{acetylcholinesterase (AChE)} into acetate and choline. Nice and simple!
%         • The choline is taken back up by active transport and reused, and the acetate is broken down and eliminated.
% \subsection{Two Types of Cholinergic Receptors}
%     • Nicotinic Receptors
%         • Agonist at low doses, but antagonist at high doses.
%         • Iontropic.
%         • Found at the Neuromusclar Function in the PNS.
%         • \cyanit{Curare} (direct antagonist) -- A drug that blocks nicotinic receptors, causing paralysis.
%             • Competitive blocking agent
%             • Paralysis, surgery
%     • Muscarinic Receptors
%         • Comes from a hallucinogenic mushroom (Amanita muscaria).
%             • \textbf{Don't confuse with Serotonin's Mescaline: Cactus; nor Psilocybin: Mushroom.}
%         • Vikings (probably took this drug before raiding) and Koryaks (Nordic people who used this mushroom in religious practices).
%         • Metabotropic receptors.
%         • Predominates in the CNS (although, both types are found in the CNS).
%         • \cyanit{Atropine} (direct antagonist) -- A drug that blocks muscarinic receptors, causing pupil dilation and increased heart rate.
%             • Competitive blocking agent
%             • Belladonna alkaloids (deadly nightshade)
% \section{MORE Drugs and Toxins Affecting ACh}
%     • \cyanit{Botulinum Toxin} -- A waste product of \textit{Clostridium botulinum}, which are bacteria who grows without oxygen.
%         • Interferes with Ca\(^{2+}\) influx channels, preventing the release of ACh.
%         • Because Botox causes paralysis, it can interfere with emotional \textit{expression} because it paralyzes muscles like the orbicularis oculi.
%         • Additionally, since we know that expression influences experience, when we paralyze these muscles, then the emotional \textit{experience} is also negatively affected.
%         • \textbf{Does Botox Decrease Emotional Experiences?}
%             • Population: Women who want wrinkles gone.
%             • One IV: two levels: Botox or restylane (dermal filler).
%             • Method: Everyone had wrinkle reduction. AND, Everyone watches some emotion evoking movies.
%             • Results: Botox group had less emotional experience than the restylane group.
%             • \textbf{Is this a good thing?}
%                 • Another study takes a sample of depressed people and gives them either Botox or a placebo.
%                 • Results: 15
%         • Botox can also be used to treat migraines, cerebral palsy, and hyperhidrosis (excessive sweating).
%     • \cyanit{Black Widow Spider Venom} -- A neurotoxin that causes the release of ACh at the neuromuscular junction, causing continual release of ACh and paralysis.
%     • \cyanit{Cobra and Krait Venom} -- A neurotoxin that blocks the binding of ACh to nicotinic receptors, causing paralysis.
%     • \cyanit{AchE Blockers} -- Comes into contact with the enzyme that breaks down ACh, causing an increase in ACh in the synaptic cleft.
%         • Irreversible
%             • Insecticides (Parathion)
%             • Nerve gas: DFP (Diisopropylfluorophosphate (don't need to know the whole name)) and Sarin.
%                 • Readily crosses the blood-brain barrier so PNS and CNS are affected.
%                 • Antidote?
%                     • \cyanit{Atropine} -- A drug that blocks muscarinic receptors, preventing the effects of excess ACh.
%                     • \cyanit{Pralidoxime} -- A drug that reactivates AChE, allowing it to break down ACh again.
%         • Reversible
%             • \cyanit{Neostgmine} (\textbf{Prostigmin}) and \cyanit{Physostigmine} (\textbf{Antilirum}) -- Drugs that inhibit AChE, increasing the amount of ACh in the synaptic cleft.
%                 • Doesn't cross the blood-brain barrier, so it only affects the PNS.
%                 • Used to treat \cyanit{myasthenia gravis} (a disease that causes muscle weakness and fatigue).
%                     • Autoimmune disease that attacks nicotinic receptors at the neuromuscular junction.
%             • \cyanit{Donepezil} (\textbf{Aricept}) and \cyanit{rivastigmine} (\textbf{Exelon}) -- These drugs do the same thing as the above drugs, but they cross the blood-brain barrier and are used to treat Alzheimer's disease and Parkinson's disease (only the cognitive part).
% \section{New Drug for Schizophrenia}
%     • We'll talk about dopamine drugs later in this unit.
%     • This new drug now:
%         • \cyanit{Xanomelne and trospium chloride} (\textbf{Cobenfy}) -- A drug that blocks the muscarinic receptors in the CNS, but not in the PNS.
%         • Dopamine but also Ach!
% \section{Catecholamines}
%     • Dopamine (DA)
%     • Norepinephrine (NE)
%     • Epinephrine (Adrenaline)
% \subsection{Dopamine (DA)}
%     • Synthesis and Metabolism
%     • Function
%     • Drugs and Disorders
% \subsubsection{Dopamine Synthesis}
%     • Tyrosine was first discovered from cheese (tyrosine = cheese).
%         • Tyrosine is the precursor for DA, NE, and Epi.
%         • \cyanit{Tyrosine Hydroxylase} -- The rate-limiting enzyme in the synthesis of catecholamines.
%             • Converts tyrosine to L-DOPA.
%             • L-DOPA is the precursor for DA, NE, and Epi.
%         • L-DOPA is converted to DA by the enzyme \cyanit{DOPA decarboxylase}.
% \subsubsection{Dopamine Metabolism}
%     • DA is broken down by the enzyme \cyanit{Monoamine Oxidase (MAO)} into \cyanit{Dihydroxyphenylacetric acid (DOPAC)}.
%     • Then, \cyanit{Catechol-O-methyltransferase (COMT)} converts DOPAC into \cyanit{Homovanillic acid (HVA)}.
%     • Also, starting from DA, we can use COMT to convert it to 3-methoxytyramine (3-MT), then with MAO, we can convert it to HVA.
% \subsubsection{DA Function}
%     • \textbf{Movement/Motor systems}
%         • \cyanit{Nigrostriatal System} -- Starts in the substantia nigra and ends in the striatum (caudate nucleus and putamen).
%         • Here's the route: We start at the striatum, which then sends an inhibitory GABA signal to the substantia nigra, who sends a reciprocal inhibitory DOPA signal back to the striatum nerve that sends an inhibitory GABA signal to the globus pallidus. Then, the globus pallidus excites the thalamus, who then excites the primary motor cortex, who then excites movement.
%             • Note that if the inhibitory signal to the substantia nigra is limited, then the signal that the striatum sends to the globus pallidus is much stronger, which leads to a weaker signal to the thalamus, and thus to movements.
%         • Parkinson's Disease symptoms:
%             • Weakness,
%             • Tremor at rest,
%             • Muscle rigidity,
%             • Problems with balance,
%             • Abnormal gait,
%             • Trouble learning
%         • Treatment
%             • \cyanit{Reserpine} (\textbf{Raudixin}) for \(\downarrow\) \textbf{BP} (Not in use anymore because it caused Parkinson's-like symptoms)
%                 • 1960's
%                     • Blocks monoamine transporters
%                         • Developed Parkinson's symptoms
%                         • Can't fill vesicles and DA is lowered
%                 • Then, discovered Substantia Nigra was pale.
%             • L-DOPA can be a direct treatment for Parkinson's as well.
%             • \cyanit{MPTP} -- Neurotoxin for DA cells in the Nigrostriatal System (which is not endogenous).
%             • \textbf{History of MPTP -- or why you shouldn't use illicit drugs}
%                 • 1982 -- young California heroin users
%                 • Had used what they THOUGHT was synthetic heroin
%                     • \cyanit{MPPP} -- Opioid analgesic drug
%                         • Not used clinically
%                         • Illegally manufactured for recreational drug use
%                     • INSTEAD it was MPTP (oh no!)
%                         • They instantly developed Parkinson's-like symptoms
%                         • Bad for them, but good for us because we can study it.
%                             • Led to animal model development and possible treatment ideas.
%                 • We don't know why Parkinson's patient's cells are dying, but maybe something similar.
%                 • MPTP is converted to the chemical \cyanit{MPP+} by the enzyme MAO (which is what breaks down DA), which is what damaged the cells.
%                     • Question: Could MAO-I improve Parkinson's?
%                     • Yes!
%                     • \cyanit{Deprenyl}, also called \cyanit{selegiline} (\textbf{Eldepryl}, \textbf{Jumex}) -- A drug that inhibits MAO, can slow down progression of the disease.
%         • New treatment
%             • Molecule keeps proteins from misfolding
%             • \cyanit{Lewy Bodies} -- Misfolded proteins that are found in the brains of people with Parkinson's.
%                 • These are toxic to DA cells
%         • \cyanit{Huntington's Chorea} -- A genetic disorder that leads to uncontrolled movements and cognitive decline.
%             • Too little GABA from the Striatum to the Substantia Nigra causes an increase in dopamine back to the Striatum which, in turn, lessens the signal to the Globus Palidus, which increases overall movements.
%             • \cyanit{Tetrabenazine} (\textbf{Xenazine}) -- Drug that inhibits the DA vesicle transporters.
%             • \cyanit{Pallidotomy} -- A surgery that affects the Globus Pallidus to inhibit movement.
%         • \cyanit{Choreoathetotic Movements} -- too much movement
%             • \cyanit{Athetosis} -- Slow continually writing movements
%             • \cyanit{Choreic} (to dance) -- Rapid, purposeless, involvuntary movements
%     • \textbf{Behavioral Arousal and Attention}
%         • Narcolepsy
%             • \cyanit{Methylphenidate} (\textbf{Ritalin}) -- A drug that increases DA and NE in the brain, used to treat ADHD, but can also be used for narcolepsy.
%             • \cyanit{Hypocretine} -- A neuropeptide that is involved in the regulation of sleep and wakefulness.
%                 • Created by the lateral hypothalamus.
%                 • Hypocretine: \textit{Hypo} for \textit{hypo}thalamus, \textit{cretine} for \textit{secretin} (a hormone).
%                 • \cyanit{Orexin} -- Another name for hypocretine; makes you want to eat.
%             • From hypocretine, researchers developed an antagonist for the orexin receptor, which is used to treat insomnia. This drug is called \cyanit{Suvorexant} (\textbf{Belsomra}).
%             • \textbf{Treatment}
%                 • \cyanit{TAK-994} -- OX2R (Orexin-2 receptor) Agonist
%                 • \cyanit{Hcrt-1} -- Intranasal hypocretine-1 (orexin-1) agonist
%                 • Hypocretine Cell Transplant
%                 • Gene Therapy: \textit{introduce} preprohypocretin gene into the brain to make more hypocretine.
%                 • Opiates (exogenous) can increase the number of hypocretin-producing cells in the brain.
%                 • Indirect role for opiate agonists in treating narcolepsy.
%         • ADHD
%             • Uses Methylphenidate for selective attention.
%         • \cyanit{Mesocortical System}
%             • From ventral tegmental nucleus to prefrontal cortex, limbic CORTEX, hippocampus, all frontal lobes, and association areas of parietal and temporal lobes in primates.
%             • Short-term memories, planning, and problem-solving are all associated with this system.
%     • \textbf{Reinforcement and Reward}
%         • \cyanit{Mesolimbic System (MLS)} -- Responsible for reward and reinforcement.
%             • From vental tegmental nucleus to limbic system
%                 • Amygdala, hippocampus, and nucleus accumbens.
%                 • Opioids cause the release of dopamine at the nucleus accumbens, which is the pleasure center of the brain.
%         • James Olds \& Peter Milner (1954)
%             • They asked: ``Does electrical stimulation of the reticular formation facilitate learning?''
%             • James Olds visits a conference and listens to Neal Miller, who says electrical stimulation is aversive, so it should be avoided.
%             • One lone rat was put in a box with a lever, and when the rat pressed the lever, it would get a shock to the reticular formation. He ended up pressing the level 700 times per hour.
%         • More studies of this
%             • Skinner box
%                 • Rats press 2000 times per hour for a shock to the MLS.
%                 • Monkeys press 8000 times per hour for a shock to the MLS.
%                 • Starving animals will choose the MLS over food 80\
%                 • They also press the button for these conditions too:
%                     • Thirsty,
%                     • Getting shocked (at their feet),
%                     • Mother instincts.
%         • Delgado (1969) -- For people who were getting their brain stimulated for seizures, this researcher also asked them about what they thought of the stimulation. They all thought that it was pleasurable.   


    

% Our goal is to take the above information and turn it into a quiz. The quiz can have various formats, including multiple choice, true/false, fill-in-the-blank, and short answer questions. The quiz should be designed to test the reader's understanding of the material presented in the overview and the sections on psychopharmacology, neurotransmitters, and specific drugs and their effects on the nervous system. The quiz should be comprehensive and cover all key concepts discussed in the text.

% We need to cover the following points using the information given:


% ≈15 on general information from the beginning of the psychopharmacology unit

%                 Psychopharmacology and why we study it

%                 Principles of drug action

% Basic processes in the life of a neurotransmitter 😊 (Precursor, synthetic enzyme, NT (can be taken back up or metabolized) metabolic enzymes, by-products (aka: metabolites)).

% How drugs can be agonist and antagonist and names for such drugs (like direct agonist

% and inverse agonist)

% Neurotransmitter vs Neuromodulator

% Classes of neurotransmitters and information presented on that big list about the classes

 

% ≈20 on Ach

%                 Discovery          

% Functions in ANS, Somatic NS, and CNS (and the parts of the brain/pathways associated

% with each of these CNS functions).

%                 Synthesis and Metabolism

%                 Types of receptors

%                 Drugs and Toxins and disorders associated with Ach

 

% ≈20 on DA

%                 Synthesis and Metabolism

%                 4 functions

%                 Nigrostriatal system

%                                 Function

% Drugs and disorders and treatments for those disorders related to the NSS

% (including Huntington’s even though that is a GABA pathway and how it has to

% do with DA).

% Why you shouldn’t use illicit drugs

%                 Mesocortical system

%                                 Functions                          

% Drugs and disorders related to it

%                 An aside about Hypocretin, narcolepsy, treatments for narcolepsy and

% insomnia

%                 Mesolimbic system

%                                 How it was discovered to be related to reinforcement

%                                 Drugs related to it

                               

% ≈10 points on the BOOK information

% ONLY those parts I listed above about amino acids, lipids, nucleosides and soluble gases

 

% To be clear about synthesis and metabolism of the neurotransmitters…YES you need to know the whole names (where I said) and not just their abbreviations of all the precursors, synthetic and metabolic enzymes, and the metabolites).

 

% To be clear about drugs and drug names…there are questions about drugs…sometimes those questions are about drug categories (e.g. benzodiazepines) sometimes more specific…like how does the drug peszkaenzine (Studilify) work?  If the question is a multiple choice question then I will give both generic and trade names in the questions or answer choices, when necessary (it wouldn’t be necessary for things like opium and atropine which don’t have trade names…but of course things like that are still drugs and you should definitely know them because things like this will be on the test).  So knowing either the trade or generic would be sufficient on those MC questions.  BUT I would lean towards learning both the trade and generic names for things in case someone asks you to answer a fill-in-the-blank question…then you would DEFINTELY get partial credit, but not full credit, if you only knew one of the names.  Remember that sometimes I gave you more than one drug of the same category and told you that you only had to know one of them…probably you only wrote down one…that is true…knowing one will allow you to answer any question I ask.   

 

% I think you should try making yourself a Drug Review Sheet…put all the drugs on it from this unit, group them by neurotransmitter or by function, know their name(s), how each works and what each works to do.   Drugs get mixed up in your head if you don’t have them organized in some way.

% Make a LITTLE of the guide, then study it…then take a break.

% Make a little more of your guide, then study it…then take a break.

% Do that maybe 5 times.  This spaced studying will help you to keep them from getting jumbled. 


\section*{3.1}
\begin{enumerate}[label=\textbf{Q3.1.\arabic*}]
    \item \textbf{Short Answer:} What is psychopharmacology, and why do we study it? \\
        \textit{Answer:} % The scientific study of the effects of drugs on the nervous system and behavior; to learn about psychotherapeutic drugs and gain a better understanding of how these things work. (This answer needs work. IF you have a better one, please let me know and I'll update this document.)

    \item \textbf{Multiple Choice:} Which of the following is NOT a function of psychopharmacology?
        \begin{tasks}[label=\textcolor{\documentTheme}{(\Alph*)}, item-format=\color{\documentTheme}, label-width=1.5em, item-indent=1.7em](1)
            \task Study the effects of drugs on the nervous system
            \task Study the effects of drugs on behavior
            \task Study the effects of drugs on the immune system
            \task Study the effects of drugs on neurotransmitter systems
        \end{tasks}
        %F \textit{Answer:} C.

    \item \textbf{Fill in the Blank:} The location at which a drug interacts with the body to produce its effects is called the \underline{\hspace{3cm}}.
        %F \textit{Answer:} Site of action.

    \item \textbf{Short Answer:} What is the difference between an agonist and an antagonist? \\
        \textit{Answer:} % An agonist mimics or enhances the effects of a neurotransmitter, while an antagonist blocks or inhibits the effects of a neurotransmitter.

    \item \textbf{True or False:} Drugs directly create effects in the body. \\
        \textit{Answer:} % False. They modulate ongoing cellular activity.

    \item \textbf{Multiple Choice:} Which of the following is an example of a drug that acts as an agonist? (Some of these drugs we haven't talked about, so pick the one you \textit{know} is an agonist.)
        \begin{tasks}[label=\textcolor{\documentTheme}{(\Alph*)}, item-format=\color{\documentTheme}, label-width=1.5em, item-indent=1.7em](4)
            \task Naloxone
            \task Morphine
            \task Curare
            \task Atropine
        \end{tasks}
        %F \textit{Answer:} B.

    \item \textbf{Short Answer:} What is selective action? \\
        \textit{Answer:} % The ability of a drug to affect only certain types of receptors or neurotransmitter systems, minimizing side effects.

    \item \textbf{Short Answer:} What is an example of how an agonistic effect can become antagonistic? \\
        \textit{Answer:} % If a drug increases the release of a neurotransmitter and also blocks its reuptake, it can lead to an excess of the neurotransmitter, which may inhibit further release.

        \newpage
    \item \textbf{Multiple Choice:} What is a precursor?
        \begin{tasks}[label=\textcolor{\documentTheme}{(\Alph*)}, item-format=\color{\documentTheme}, label-width=1.5em, item-indent=1.7em](1)
            \task A substance that inhibits neurotransmitter release
            \task A substance that enhances neurotransmitter release
            \task A substance from which another substance is formed
            \task A substance that blocks neurotransmitter receptors
        \end{tasks}
        \textit{Answer:} % C.

    \item \textbf{Fill in the Blank:} The process of creating a neurotransmitter from its precursors is called \underline{\hspace{3cm}}.
        %F \textit{Answer:} Synthesis.

    \item \textbf{Fill in the Blanks:} A(n) \underline{\hspace{3cm}} agonist binds to the same receptor as the neurotransmitter and \underline{\hspace{3cm}} its effects, while a(n) \underline{\hspace{3cm}} agonist binds to a different site on the receptor and \underline{\hspace{3cm}} the effects of the neurotransmitter.
        %F \textit{Answer:} Direct, mimics, indirect, enhances.

    \item \textbf{Fill in the Blanks:} A(n) \underline{\hspace{3cm}} antagonist binds to a different site on the receptor and \underline{\hspace{3cm}} the effects of the neurotransmitter, while a(n) \underline{\hspace{3cm}} antagonist binds to the same receptor as the neurotransmitter and \underline{\hspace{3cm}} its effects.
        %F \textit{Answer:} Indirect, blocks, direct, inhibits.

    \item \textbf{Multiple Choice:} Drugs that cause the action potential to stay in a depolarized state are called:
        \begin{tasks}[label=\textcolor{\documentTheme}{(\Alph*)}, item-format=\color{\documentTheme}, label-width=1.5em, item-indent=1.7em](2)
            \task Agonists
            \task Depolarizing agents
            \task Antagonists
            \task Inverse agonists
        \end{tasks}
        %F \textit{Answer:} B 
\newpage
    \item \textbf{Matching:} Match the following examples with them either being an antagonist or an agonist. (Some of these may be direct or indirect. Specify each one.)
    \begin{wordbox}
        \begin{enumerate}
            \item \cyanit{Curare} % Direct antagonist %a
            \item \cyanit{Atropine} % Direct antagonist %b
            \item \cyanit{Morphine} % Direct agonist %c
            \item \cyanit{Naloxone} % Direct antagonist %d
            \item \cyanit{Botulinum Toxin} % Indirect antagonist %e
            \item Interfering with docking proteins % Antagonist %f
            \item Blocking the reuptake of a neurotransmitter % Agonist %g
            \item \cyanit{Sarin} % Indirect agonist %h
            \item Interfering with vesicles % Antagonist %i
            \item Blocking receptors % Antagonist %j
            \item Black widow spider venom % Direct agonist %k
            \item Cobra and Krait Venom % Direct antagonist %l
            \item Parathion % Indirect agonist %m
            \item DFP % Indirect agonist %n
            \item \cyanit{Physostigmine} % Indirect agonist %o
        \end{enumerate}
    \end{wordbox}
    \begin{enumerate}[label=(\arabic*)]
        \item Direct antagonist \quad \dotfill \quad \underline{\hspace{3cm}} 
        \item Indirect antagonist \quad \dotfill \quad \underline{\hspace{3cm}} 
        \item Direct agonist \quad \dotfill \quad \underline{\hspace{3cm}} 
        \item Indirect agonist \quad \dotfill \quad \underline{\hspace{3cm}} 
        \item Antagonist \quad \dotfill \quad \underline{\hspace{3cm}} 
        \item Agonist \quad \dotfill \quad \underline{\hspace{3cm}} 
    \end{enumerate}
    % Answers: (1): a, b, d, l (2): e, (3): c, k, (4): h, m, n, o (5): f, i, j (6): g

    \item \textbf{Fill in the Blanks:} In the following diagram, label the specific enzyme for each arrow, and identify the outcome:
    \[
        \text{Precursor} \xrightarrow{\text{\underline{\phantom{Synthetic}} Enzyme}} \text{NT} \xrightarrow{\text{\underline{\phantom{Metabolic}} Enzyme}} \text{\underline{\phantom{Inactive Metabolite}}}
    \]
    %F \textit{Answer:} Synthetic; Metabolic; Inactive Metabolite

    \item \textbf{Long(-ish) Answer:} Describe the difference between a neurotransmitter and a neuromodulator. \\
        \textit{Answer:} \\[1cm] % A neurotransmitter is a chemical messenger that transmits signals across a synapse from one neuron to another, while a neuromodulator is a substance that modulates the activity of neurotransmitters, often affecting a larger area of the brain and influencing the overall tone of neural activity.

\end{enumerate}

\squigglyline
\section*{3.2}

\begin{enumerate}[label=\textbf{Q3.2.\arabic*}]
    \item \textbf{Multiple Choice:} Which of the following neurochemicals does NOT transmit information (according to our notes)?
        \begin{tasks}[label=\textcolor{\documentTheme}{(\Alph*)}, item-format=\color{\documentTheme}, label-width=1.5em, item-indent=1.7em](2)
            \task Dopamine
            \task Glutamate
            \task GABA
            \task Glycine
        \end{tasks}
        %F \textit{Answer:} A.

    \item \textbf{Fill in the Blank:} Peptides are short chains of \underline{\hspace{3cm}}.
        %F \textit{Answer:} Amino acids.

    \item \textbf{Fill in the Blanks:} The difference between opioids and opiates are that opioids are \underline{\hspace{3cm}} and opiates are \underline{\hspace{3cm}}.
        %F \textit{Answer:} endogenous; exogenous.

    \item \textbf{Short Answer:} What is the pain pathway for the face? What about from the neck down? (Generally speaking.) \\
        \textit{Answer:} % From the face (specifically the trigeminal nerve); A-delta fibers (mylinated) fast conducting, C-fibers (unmyelinated) slow conducting fibers

    \item \textbf{Fill in the Blanks:} The three types of opioid receptors are \underline{\hspace{3cm}}, \underline{\hspace{3cm}}, and \underline{\hspace{3cm}}.
        %F \textit{Answer:} Mu, Kappa, Delta.

    \item \textbf{Long Answer:} What are each of the three opioid receptors responsible for, and what neurochemicals bind to each the most? \\
        \textit{Answer:} \\[1cm] % Mu: Analgesia and euphoria, Endorphins; Delta: Analgesia, Enkephalins; Kappa: Colocalized with certain catecholamines, learning and memory, emotional control, stress response, and analgesia, Dynorphins.

    \item \textbf{Multiple Choice:} Prostaglandins become active during
        \begin{tasks}[label=\textcolor{\documentTheme}{(\Alph*)}, item-format=\color{\documentTheme}, label-width=1.5em, item-indent=1.7em](2)
            \task Resting-and-Digesting
            \task Crying
            \task Daydreaming 
            \task Bleeding
        \end{tasks}
        %F \textit{Answer:} D.

    \item \textbf{Short Answer:} What are the functions of opioids? (List the main effects and the side effects.) \\
        \textit{Answer:} % Prevents diarrhea, gives euphoria, analgesia, changes the stress response, body temperature, emotion, feeding motivation, sexual behavior, learning, drowsiness, and promotes pro-social behavior in some cases. 

    \item \textbf{Fill in the Blank:} \textit{Cylooxygenase (COX)} is an enzyme that converts inactive \underline{\hspace{3cm}} to its active state.
        %F \textit{Answer:} Prostaglandins

    \item \textbf{Short Answer:} List the characteristics for the direct pain pathway and the indirect pain pathway. \\
        \textit{Answer:} % Direct: Sharp and well localized pain, immediate and brief, mechanical (strong) and thermal (extreme temperature); Indirect: Slow pain, throbbing, aching and dull pain, takes longer, but lingers, chemical (inflammatory) pain,  

    \item \textbf{Fill in the Blanks:} Pain arrives at the \underline{\hspace{4cm}}, then travels to the \underline{\hspace{3cm}}. Once there, it is processed by several brain regions. First, the \underline{\hspace{3cm}} contributes to arousal. Then, the \underline{\hspace{3cm}}, particularly the anterior cingulate cortex (ACC), processes the emotional aspects of pain. When the pain is overwhelming, the \underline{\hspace{3cm}} activates and releases endogenous opioids to reduce the sensation—this allows a person, for example, to escape danger despite a severe injury. Finally, the \underline{\hspace{3cm}} and other areas help interpret and associate the pain with context. 
        %F \textit{Answer:} Brain stem reticular formation (BSRF); Thalamus; Thalamus; Limbic system (ACC); Periaqueductal gray region; Frontal lobes

    \item \textbf{Matching:} Match the following drugs with their respective NSAID class.
    \begin{wordbox}
        \begin{enumerate}
            \item Ibuprofen
            \item Aspirin
            \item Diflunisal
            \item Naproxen
            \item Salsalate
            \item Ketoprofen
        \end{enumerate}
    \end{wordbox}
    \begin{enumerate}[label=(\arabic*)]
        \item Proprionic Acid Derivatives \quad \dotfill \quad \underline{\hspace{4cm}}
        \item Salicylates \quad \dotfill \quad \underline{\hspace{4cm}}
    \end{enumerate}
    %F \textit{Answer:} (1): a, d, f; (2): b, c, e

    \item \textbf{True or False:} \textit{Celecoxib} (\textbf{Celebrex}) -- A COX-2 Inhibitor was removed from the market because it causes heart attacks and stroke. \\
        \textit{Answer:} % False; Rofecoxib (Vioxx) is the actual COX-2 inhibitor.

    \item \textbf{Long Answer:} Some studies show that both the placebo effect and acupuncture can be blocked by Naloxone, an opioid antagonist. What does this suggest about the mechanism of acupuncture’s pain-relieving effects? Does this prove that acupuncture is not entirely a placebo? \\
        \textit{Answer:} \\[2cm] %This suggests that acupuncture’s pain-relieving effects may be mediated by the body’s endogenous opioid system, similar to the placebo effect. However, this does not prove that acupuncture is entirely a placebo; it only indicates that placebo-like mechanisms (such as expectation-induced opioid release) may contribute to its effects. Other mechanisms may also be involved.
    
    \item \textbf{True or False:} The term \textit{colocalized} means two or more neurotransmitters are released from two separate neurons at the same time. \\ 
        \textit{Answer:} % False; from the same neuron. 

    \item \textbf{Short Answer:} What is the definition of pain? (DO NOT say this exam!!!!!!!!) \\
        \textit{Answer:} % Unpleasant sensory and emotional experience associated with actual or potential tissue damage. 
\end{enumerate}

\squigglyline
\section*{3.3}

\begin{enumerate}[label=\textbf{Q3.3.\arabic*}]
    \item \textbf{True or False:} The amines (monoamines) are derived from amino acids. \\
        \textit{Answer:} % True.

    \item \textbf{Short Answer:} Name three neurotransmitters that fall under the amino acid category. \\
        \textit{Answer:} % Glutamate, GABA, Glycine.

    \item \textbf{Fill in the Blank:} The two indolamines are \underline{\hspace{3cm}} and \underline{\hspace{3cm}}.
        %F \textit{Answer:} Serotonin (5-HT) and Melatonin.

    \item \textbf{Multiple Choice:} What amino acid are indolamines derived from?
        \begin{tasks}[label=\textcolor{\documentTheme}{(\Alph*)}, item-format=\color{\documentTheme}, label-width=1.5em, item-indent=1.7em](4)
            \task Tryptophan
            \task Tyrosine
            \task Glutamate
            \task Glycine
        \end{tasks}
        %F \textit{Answer:} A.
    
    \item \textbf{Fill in the Blank:} The precursor to glutamate is \underline{\hspace{3cm}}, and the enzyme that synthesizes glutamate from it is \underline{\hspace{3cm}}.
        %F \textit{Answer:} The precursor is glutamine; the enzyme is glutaminase.

    \item \textbf{Short Answer:} What receptor does ketamine bind to, and what is its effect? \\
        \textit{Answer:} % Ketamine binds to the NMDA receptor and acts as a dissociative anesthetic, which is being studied as a treatment for depression.

    \item \textbf{Fill in the Blank:} The enzyme \underline{\hspace{3cm}} deactivates anandamide.
        %F \textit{Answer:} Fatty acid amide hydrolase (FAAH)

    \item \textbf{True or False:} The most common excitatory neurotransmitter in the brain is GABA. \\
        \textit{Answer:} % False; it is glutamate.

    \item \textbf{Fill in the Blank:} The drug \underline{\hspace{3cm}} is a direct antagonist of the NMDA receptor and can cause hallucinations and dissociation.
        %F \textit{Answer:} Phencyclidine (PCP).

    \item \textbf{Short Answer:} What transporters are responsible for glutamate reuptake, and why is this process important? \\
        \textit{Answer:} % Excitatory amino acid transporters (EAATs); it prevents excitotoxicity, which can lead to brain damage (e.g., in stroke or ALS).

    \item \textbf{Multiple Choice:} Which receptor is closely associated with glutamate and is important for synaptic plasticity and memory formation?
        \begin{tasks}[label=\textcolor{\documentTheme}{(\Alph*)}, item-format=\color{\documentTheme}, label-width=1.5em, item-indent=1.7em](2)
            \task GABA receptor
            \task NMDA receptor
            \task Serotonin receptor
            \task Dopamine receptor
        \end{tasks}
        %F \textit{Answer:} B.

    \item \textbf{Short Answer:} What enzyme converts glutamate into GABA, and what type of neurotransmitter is GABA? \\
        \textit{Answer:} % Glutamic acid decarboxylase (GAD); GABA is the most common inhibitory neurotransmitter in the brain.
    
    \item \textbf{Fill in the Blanks:} The three catecholamine neurotransmitters are \underline{\hspace{3cm}}, \underline{\hspace{3cm}}, and \underline{\hspace{3cm}}.
        %F \textit{Answer:} Dopamine (DA), Norepinephrine (NE), Epinephrine (Adrenaline).

    \item \textbf{Multiple Choice:} What do all catecholamines contain, and what amino acid are they derived from?
        \begin{tasks}[label=\textcolor{\documentTheme}{(\Alph*)}, item-format=\color{\documentTheme}, label-width=1.5em, item-indent=1.7em](1)
            \task Catechol and are derived from tryptophan
            \task Catechol and are derived from tyrosine
            \task Indole and are derived from tryptophan
            \task Indole and are derived from tyrosine
        \end{tasks}
        %F \textit{Answer:} B.

    \item \textbf{Fill in the Blank:} The enzyme \underline{\hspace{3cm}} converts tyrosine into L-DOPA.
        %F \textit{Answer:} Tyrosine hydroxylase.

    \item \textbf{True or False:} Tyrosine is the precursor for serotonin. \\
        \textit{Answer:} % False; it is the precursor for catecholamines.

    \item \textbf{Short Answer:} What are the names of the systems that use dopamine, norepinephrine, and epinephrine?  
        \textit{Answer:} % Dopaminergic, Noradrenergic, and Adrenergic systems

    \item \textbf{Fill in the Blanks:} Melatonin is synthesized from \underline{\hspace{3cm}} and is involved in regulating \underline{\hspace{3cm}}.
        %F \textit{Answer:} Serotonin; sleep-wake cycles (circadian rhythms).

    \item \textbf{Short Answer:} What is another name for peptides in the context of neurotransmitters, and give an example.  
        \textit{Answer:} % Neuropeptides; example: Endogenous opioids

    \item \textbf{Multiple Choice:} What is the name of the endogenous cannabinoid neurotransmitter whose name means ``bliss'' in Sanskrit?
        \begin{tasks}[label=\textcolor{\documentTheme}{(\Alph*)}, item-format=\color{\documentTheme}, label-width=1.5em, item-indent=1.7em](2)
            \task Anandamide
            \task Cannabidiol
            \task Tetrahydrocannabinol (THC)
            \task 2-Arachidonoylglycerol (2-AG)
        \end{tasks}
        %F \textit{Answer:} A.

    \item \textbf{Short Answer:} How are lipid-based neurotransmitters synthesized and stored? \\
        \textit{Answer:} % They are synthesized on demand and not stored in synaptic vesicles.

    \item \textbf{Fill in the Blank:} The gaseous neurotransmitter that is required for an erection is \underline{\hspace{3cm}}.
        %F \textit{Answer:} Nitric Oxide (NO).

    \item \textbf{Short Answer:} Name one neurotransmitter that is a nucleoside. What is its function? \\
        \textit{Answer:} % Adenosine; it is involved in sleep regulation and has inhibitory effects on neurotransmission.

    \item \textbf{Fill in the Blanks:} Fill in the following spaces that describe the process of dopamine metabolism:
    DA is broken down by \underline{\hspace{3cm}} into \underline{\hspace{3cm}}. Then, \underline{\hspace{3cm}} converts it into \underline{\hspace{3cm}}. \\
        %F \textit{Answer:} Monoamine oxidase (MAO); Dihydroxyphenylacetric acid (DOPAC); Catechol-O-methyltransferase (COMT); Homovanillic acid (HVA).
\end{enumerate}

\squigglyline
\section*{3.4: Acetylcholine (ACh)}

\begin{enumerate}[label=\textbf{Q3.4.\arabic*}]
    \item \textbf{Multiple Choice:} Who first discovered acetylcholine in 1921?
        \begin{tasks}[label=\textcolor{\documentTheme}{(\Alph*)}, item-format=\color{\documentTheme}, label-width=1.5em, item-indent=1.7em](2)
            \task Otto von Loewy
            \task James Olds
            \task Neal Miller
            \task Peter Milner
        \end{tasks}
        %F \textit{Answer:} A.

    \item \textbf{Short Answer:} What experiment led to the discovery of acetylcholine? \\
        \textit{Answer:} % Otto von Loewy took a frog heart, put it in saline, stimulated the parasympathetic part of the vagus nerve which slowed the heart. When he put the saline in another frog heart, it also slowed down, showing a chemical was released.

    \item \textbf{Fill in the Blank:} The original name given to acetylcholine by its discoverer was \underline{\hspace{3cm}}.
        %F \textit{Answer:} Vagusstoff.

    \item \textbf{True or False:} Acetylcholine is the only neurotransmitter used in the parasympathetic branch of the autonomic nervous system. \\
        \textit{Answer:} % True.

    \item \textbf{Fill in the Blanks:} In the sympathetic nervous system, ACh is used at the \underline{\hspace{3cm}}, while NE is used at the \underline{\hspace{3cm}}.
        %F \textit{Answer:} Preganglionic synapse (or paravertebral ganglion); neuromuscular junction with smooth muscles and glands.

    \item \textbf{Multiple Choice:} Which of the following is NOT a function of ACh in the CNS?
        \begin{tasks}[label=\textcolor{\documentTheme}{(\Alph*)}, item-format=\color{\documentTheme}, label-width=1.5em, item-indent=1.7em](2)
            \task Learning and alertness
            \task Memory
            \task REM sleep generation
            \task Pain modulation
        \end{tasks}
        %F \textit{Answer:} D.

    \item \textbf{Fill in the Blanks:} The precursor to acetylcholine is \underline{\hspace{3cm}} and the enzyme that synthesizes acetylcholine is \underline{\hspace{3cm}}.
        %F \textit{Answer:} Choline; choline acetyltransferase (ChAT).

    \item \textbf{Short Answer:} Explain how acetylcholine is metabolized. \\
        \textit{Answer:} % ACh is broken down by the enzyme acetylcholinesterase (AChE) into acetate and choline. The choline is taken back up by active transport and reused, while acetate is broken down and eliminated.

    \item \textbf{Multiple Choice:} Which type of ACh receptor is ionotropic?
        \begin{tasks}[label=\textcolor{\documentTheme}{(\Alph*)}, item-format=\color{\documentTheme}, label-width=1.5em, item-indent=1.7em](2)
            \task Nicotinic receptors
            \task Muscarinic receptors
            \task Both nicotinic and muscarinic receptors
            \task Neither nicotinic nor muscarinic receptors
        \end{tasks}
        %F \textit{Answer:} A.

    \item \textbf{Fill in the Blank:} The drug \underline{\hspace{3cm}} is a direct antagonist of nicotinic receptors, causing paralysis.
        %F \textit{Answer:} Curare.

    \item \textbf{True or False:} Atropine blocks muscarinic receptors and is derived from the plant known as deadly nightshade. \\
        \textit{Answer:} % True.

    \item \textbf{Short Answer:} How does Botulinum Toxin interfere with acetylcholine function? \\
        \textit{Answer:} % It interferes with Ca²⁺ influx channels, preventing the release of ACh.

    \item \textbf{Fill in the Blanks:} Black Widow Spider venom causes \underline{\hspace{3cm}} of ACh, while Cobra venom \underline{\hspace{3cm}} ACh receptors.
        %F \textit{Answer:} Continual release; blocks.

    \item \textbf{Multiple Choice:} Which of the following is a reversible AChE blocker used to treat myasthenia gravis?
        \begin{tasks}[label=\textcolor{\documentTheme}{(\Alph*)}, item-format=\color{\documentTheme}, label-width=1.5em, item-indent=1.7em](2)
            \task Sarin
            \task Parathion
            \task Neostigmine (Prostigmin)
            \task DFP (Diisopropylfluorophosphate)
        \end{tasks}
        %F \textit{Answer:} C.

    \item \textbf{True or False:} Donepezil (Aricept) crosses the blood-brain barrier and is used to treat the cognitive symptoms of Alzheimer's disease. \\
        \textit{Answer:} % True.
\end{enumerate}

\squigglyline
\section*{3.5: Dopamine and Related Topics}

\begin{enumerate}[label=\textbf{Q3.5.\arabic*}]
    \item \textbf{Fill in the Blanks:} The three catecholamines are \underline{\hspace{3cm}}, \underline{\hspace{3cm}}, and \underline{\hspace{3cm}}.
        %F \textit{Answer:} Dopamine (DA); Norepinephrine (NE); Epinephrine (Adrenaline).

    \item \textbf{Multiple Choice:} What is the precursor for dopamine?
        \begin{tasks}[label=\textcolor{\documentTheme}{(\Alph*)}, item-format=\color{\documentTheme}, label-width=1.5em, item-indent=1.7em](2)
            \task Tyrosine
            \task L-DOPA
            \task Tryptophan
            \task Choline
        \end{tasks}
        %F \textit{Answer:} A.

    \item \textbf{Fill in the Blank:} The rate-limiting enzyme in the synthesis of catecholamines is \underline{\hspace{3cm}}.
        %F \textit{Answer:} Tyrosine Hydroxylase.

    \item \textbf{Short Answer:} Describe the pathway of dopamine synthesis from its amino acid precursor. \\
        \textit{Answer:} % Tyrosine is converted to L-DOPA by tyrosine hydroxylase, then L-DOPA is converted to dopamine by DOPA decarboxylase.

    \item \textbf{True or False:} The word ``tyrosine'' is derived from a word meaning ``cheese.'' \\
        \textit{Answer:} % True.

    \item \textbf{Multiple Choice:} Which pathway is involved in movement and motor control?
        \begin{tasks}[label=\textcolor{\documentTheme}{(\Alph*)}, item-format=\color{\documentTheme}, label-width=1.5em, item-indent=1.7em](2)
            \task Nigrostriatal system
            \task Mesocortical system
            \task Mesolimbic system
            \task Tuberoinfundibular system
        \end{tasks}
        %F \textit{Answer:} A.

    \item \textbf{Fill in the Blanks:} The nigrostriatal pathway starts in the \underline{\hspace{3cm}} and ends in the \underline{\hspace{3cm}}.
        %F \textit{Answer:} Substantia nigra; striatum (caudate nucleus and putamen).

    \item \textbf{Short Answer:} List four symptoms of Parkinson's disease. \\
        \textit{Answer:} % Any four of: weakness, tremor at rest, muscle rigidity, problems with balance, abnormal gait, trouble learning.

    \item \textbf{Multiple Choice:} What neurotoxin led to the development of an animal model for Parkinson's disease?
        \begin{tasks}[label=\textcolor{\documentTheme}{(\Alph*)}, item-format=\color{\documentTheme}, label-width=1.5em, item-indent=1.7em](2)
            \task MPTP
            \task MPPP
            \task MPP+
            \task MAO
        \end{tasks}
        %F \textit{Answer:} A.

    \item \textbf{Fill in the Blank:} The misfolded proteins found in the brains of people with Parkinson's disease are called \underline{\hspace{3cm}}.
        %F \textit{Answer:} Lewy Bodies.

    \item \textbf{True or False:} In Huntington's Chorea, there is too much GABA from the Striatum to the Substantia Nigra. \\
        \textit{Answer:} % False. In Huntington's Chorea, there is too little GABA from the Striatum to the Substantia Nigra.

    \item \textbf{Long Answer:} Explain how the MPTP incident in 1982 contributed to our understanding of Parkinson's disease. \\
        \textit{Answer:} \\[2cm] % In 1982, young California heroin users thought they were using synthetic heroin (MPPP), but were actually exposed to MPTP. They instantly developed Parkinson's-like symptoms. MPTP is converted to MPP+ by the enzyme MAO, which damaged dopaminergic cells in the substantia nigra. This led to the development of animal models for Parkinson's research and potential treatment approaches, including MAO inhibitors like deprenyl (selegiline).

    \item \textbf{Fill in the Blank:} \textit{Methylphenidate} (\textbf{Ritalin}) increases levels of \underline{\hspace{3cm}} and \underline{\hspace{3cm}} in the brain.
        %F \textit{Answer:} Dopamine (DA); Norepinephrine (NE).

    \item \textbf{Multiple Choice:} Which system is primarily responsible for reward and reinforcement?
        \begin{tasks}[label=\textcolor{\documentTheme}{(\Alph*)}, item-format=\color{\documentTheme}, label-width=1.5em, item-indent=1.7em](2)
            \task Nigrostriatal system
            \task Mesocortical system
            \task Mesolimbic system
            \task Tuberoinfundibular system
        \end{tasks}
        %F \textit{Answer:} C.

    \item \textbf{Short Answer:} What neuropeptide, also called orexin, is involved in the regulation of sleep and wakefulness? \\
        \textit{Answer:} % Hypocretin.

    \item \textbf{Fill in the Blank:} The drug \underline{\hspace{3cm}} is an orexin receptor antagonist used to treat insomnia.
        %F \textit{Answer:} Suvorexant (Belsomra).

    \item \textbf{True or False:} The mesocortical system is involved in short-term memory, planning, and problem-solving. \\
        \textit{Answer:} % True.

    \item \textbf{Multiple Choice:} Which researchers discovered that electrical stimulation of certain brain areas could be rewarding rather than aversive?
        \begin{tasks}[label=\textcolor{\documentTheme}{(\Alph*)}, item-format=\color{\documentTheme}, label-width=1.5em, item-indent=1.7em](2)
            \task Otto von Loewy and Vagusstoff
            \task James Olds and Peter Milner
            \task Neal Miller and Delgado
            \task Lateral hypothalamus researchers
        \end{tasks}
        %F \textit{Answer:} B.

    \item \textbf{Short Answer:} What structure within the limbic system is considered the ``pleasure center'' of the brain? \\
        \textit{Answer:} % Nucleus accumbens.
\end{enumerate}