\begin{center}
    \textbf{We're starting with three studies:}
\end{center}

\begin{enumerate}
    \item \textbf{Study 1:} Blinking:
    \begin{coloredlist}
        \item Three levels of blinking:
        \begin{coloredlist}
            \item Reflexive blinking. Ex: When a puff of air is directed at the eye.
            \item Voluntary blinking. Ex: When you're asked to blink.
            \item Endogenous blinking. Meaning: ``originating from or due to internal causes.''
        \end{coloredlist}
        \item \cyanit{Endogenous blinking} is the focus of this study.
        \begin{coloredlist}
            \item Endogenous blinks occur during reading or speaking and reflect changes of attention and changes in thought processes. The more attention required by a visual task; the fewer endogenous blinks occur.
            \item More attention required is associated with fewer endogenous blinks. Especially for visual tasks.
            \item \textbf{The harder the tasks \(\rightarrow\) the fewer the blinks.}
            \item Even when a task is not visual, there is a decrease in endogenous blink rate (EBR) during a difficult task followed by flurry of blinks when task is over.
            \item \textbf{But wait!}
            \begin{coloredlist}
                \item EBR has been shown to increase when a cognitive secondary task is performed concurrently, and the cognitive task does not involve visual attention.
            \end{coloredlist}
            \item \textbf{WHY?}
            \begin{coloredlist}
                \item EB is a dopaminergic activity.
                \item Dopamine plays a big role in selective attention.
            \end{coloredlist}
        \end{coloredlist} 
        \item Through this study, we learned that endogenous blinking (DV) is affected by cognitive load (IV)
    \end{coloredlist}
    \item \textbf{Study 2:} Cartoon Judgement:
    \begin{coloredlist}
        \item Group 1 and 2 membership.
        \item Follow group instructions then rate the 3 cartoons that follow on scale from 1-10.
        \begin{coloredlist}
            \item 1 is NOT funny
            \item 10 is VERY funny
            \item Answers (Lips = Pen in lips; Teeth = Pen in teeth):

            \begin{tabular}[htbp]{ccccc}
                \toprule
                \textbf{Groups} & \textbf{Pic 1} & \textbf{Pic 2} & \textbf{Pic 3} & \textbf{Average} \\ \midrule
                \textbf{Lips} & 3 & 3 & 4 & \(3\frac{1}{3}\) \\ \midrule
                \textbf{Teeth} & 4 & 4 & 3 & \(3\frac{2}{3}\) \\ \midrule
                \textbf{Stretch} & 4 & 5 & 6 & 5 \\ \midrule
                \textbf{J. Jacks} & 4 & 2 & 3 & 3 \\ \bottomrule
            \end{tabular}

        \end{coloredlist}
        \item \textbf{Facial Feedback Hypothesis}
        \begin{coloredlist}
            \item Selective activation or inhibition of facial muscles has a strong impact on emotional responses to stimuli.
            \item Zygomatic major muscle.
            \begin{coloredlist}
                \item When we had the pen in our teeth, we were activating the zygomatic major muscle.
                \item This muscle is responsible for smiling.
            \end{coloredlist}
            \item Our data supported this hypothesis with a probability of \(p < 0.02\).
        \end{coloredlist}
        \item \textbf{Arousal}
        \begin{coloredlist}
            \item Increased heart rate in many emotions.
            \item Heart rate and attraction
            \begin{coloredlist}
                \item 1973 Dutton and Aron
                \begin{coloredlist}
                    \item Shaky high bridge vs. low stable bridge.
                    \item Woman on the other side who is asking questionnaire questions (faux DV).
                    \item She gave her phone number to the guys once they got done answering the questions.
                    \item The actual DV was the amount of phone calls she received and the sexual content in questionnaire answers.
                    \item The high bridge group had more sexual content in their messages.
                \end{coloredlist}
                \item 15 minutes of physical activity, then rate attractiveness of potential mates.
            \end{coloredlist}
        \end{coloredlist}
    \end{coloredlist}
\end{enumerate}

\cyanit{Psychophysiology:} Behavioral, cognitive, emotional, and social events are all mirrored in physiological processes. \\

The idea is that we can get a peep into your psychology by looking at what your biology is doing. \\

\textbf{Sleep:} EEG (Electroencephalogram; measuring brain activity), EOG (Electrooculogram; measuring eye movement), EMG (Electromyography; measuring muscle movement), ERP (Event-Related Potential; measuring event-related potential). \\ 

When your brain is hooked up to the ERP and you are asked either task relevant-stimulus, important stimulus, or surprising stimulus, it will produce a higher \(p\)\textemdash300 amplitude compared to otherwise. Think about how this would be used for interrogating a suspect, for example. \\ 

Then, there are \(n\)\textemdash350 amplitudes which are activated when we are asleep, and we hear our name, for example. \\

Notice the implications of this with sleeping: the more tired you are (or closer to falling asleep), the harder it will be to find the \(p\)\textemdash300 amplitude. \\

\cyanit{Omitted Stimulus Paradigm} -- Given a constant stimulus, this is the phenomena wherein there is a gap in the pattern. This will result in a higher \(p\)\textemdash300 amplitude. \\

\textbf{Polygraph:} Respiration, GSR (EDA), Blood flow, Blood pressure, and heart rate. \\

With a polygraph, we're looking at all the components of the peripheral nervous system. That is, when we're ``looking at  '' a polygraph, we're not measuring lying, but all the physiological responses that are associated with lying. \\



\section{EDA}

\begin{coloredlist}
    \item \cyanit{Electrodermal Activity}
    \begin{coloredlist}
        \item Old name: Galvanic Skin Response
        \item Measuring sympathetic nervous system activity by detecting sweat gland activity by measuring the conductance of an electrical signal from one electrode to another.
        \item More rapid conductance with more activity.
        \item Particularly good for emotion.
        \item Maybe attention.
        \item This also is a good measure for the sympathetic nervous system because it is the only one that can enervate the sweat glands.
    \end{coloredlist}
\end{coloredlist}

\subsection{What if I Want to Know}

\begin{coloredlist}
    \item If you're lying
    \item Cognitive or emotional states when you're sleeping
    \item If you're attending to stimuli I'm presenting
\end{coloredlist}

\textbf{Psychophysiology is different from Physiological Psychology}. Note that Psychophysiology is where mind/behavior is the IV, and physiology is the DV. Similarly, Physiological Psychology uses the same IVs and DVs, but notice that we are manipulating the physiological psychology to measure psychology. Remember the independent variable is first (mnemonic). \\

\subsubsection{Examples}

\begin{coloredlist}
    \item Present snake photo or not. Then measure effects on physiology that evidence fear.
    \item Presenting in-group and out-group photographs. Then measure effects on physiology that evidence prejudicial cognition.
    \item Drugs: Measure effects on psychology (behavior/aggression)
    \item Lesion: Measure effects on psychology (cognition/memory)
    \item Manipulate heart rate: (behavior/attractiveness ratings)
    \item Electrical Stim of Brain [tDCS (transcranial direct current stimulation)]: Measure effects on psychology (mood/emotion/depression)
\end{coloredlist}

\section{EEG}

\begin{coloredlist}
    \item Activity in large groups of neurons.
    \item Difference in electrical activity at a reference point and site of interest.
    \item You get: Wavelike patterns.
    \item \cyanit{International 10-20 system}.
    \begin{coloredlist}
        \item Fz, Fp1, Fp2, F7, F8, F3, F4, T3, T4, Cz, C3, C4, T5, T6, Pz, P3, P4, O1, O2.
        \begin{multicols}{2}
            \begin{coloredlist}
                \item F = Frontal
                \item T = Temporal
                \item C = Central
                \item P = Parietal
                \item O = Occipital
                \item Odd numbers = left hemisphere.
                \item Even numbers = right hemisphere.
                \item Z = Midline
            \end{coloredlist}
        \end{multicols}
    \end{coloredlist}
    \item Researchers use \cyanit{visual inspection} to look at the EEG data.
    \item Measurements
    \begin{coloredlist}
        \item \cyanit{Frequency} -- \(\frac{1}{\text{time}}\) (in Hz)
        \item \cyanit{Amplitude} -- Height of wave (in \(\mu V\))
    \end{coloredlist}
\end{coloredlist}

\section{Neurofeedback}

\begin{coloredlist}
    \item Learn to control your brain activity.
    \begin{coloredlist}
        \item See the activity
        \item Get reinforcement or punishment
        \item Make changes
        \item Even if you don't ``know'' what you're doing, you can still learn to control your brain activity.
    \end{coloredlist}
\end{coloredlist}

