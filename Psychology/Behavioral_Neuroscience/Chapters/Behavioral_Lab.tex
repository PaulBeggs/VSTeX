\begin{center}
    \textbf{We're starting with three studies:}
\end{center}

\begin{enumerate}
    \item \textbf{Study 1:} Blinking:
    \begin{coloredlist}
        \item Three levels of blinking:
        \begin{coloredlist}
            \item Reflexive blinking. Ex: When a puff of air is directed at the eye.
            \item Voluntary blinking. Ex: When you're asked to blink.
            \item Endogenous blinking. Meaning: ``originating from or due to internal causes.''
        \end{coloredlist}
        \item \cyanit{Endogenous blinking} is the focus of this study.
        \begin{coloredlist}
            \item Endogenous blinks occur during reading or speaking and reflect changes of attention and changes in thought processes. The more attention required by a visual task; the fewer endogenous blinks occur.
            \item More attention required is associated with fewer endogenous blinks. Especially for visual tasks.
            \item \textbf{The harder the tasks \(\rightarrow\) the fewer the blinks.}
            \item Even when a task is not visual, there is a decrease in endogenous blink rate (EBR) during a difficult task followed by flurry of blinks when task is over.
            \item \textbf{But wait!}
            \begin{coloredlist}
                \item EBR has been shown to increase when a cognitive secondary task is performed concurrently, and the cognitive task does not involve visual attention.
            \end{coloredlist}
            \item \textbf{WHY?}
            \begin{coloredlist}
                \item EB is a dopaminergic activity.
                \item Dopamine plays a big role in selective attention.
            \end{coloredlist}
        \end{coloredlist} 
        \item Through this study, we learned that endogenous blinking (DV) is affected by cognitive load (IV)
    \end{coloredlist}
    \item \textbf{Study 2:} Cartoon Judgement:
    \begin{coloredlist}
        \item Group 1 and 2 membership.
        \item Follow group instructions then rate the 3 cartoons that follow on scale from 1-10.
        \begin{coloredlist}
            \item 1 is NOT funny
            \item 10 is VERY funny
            \item Answers (Lips = Pen in lips; Teeth = Pen in teeth):

            \begin{tabular}[htbp]{ccccc}
                \toprule
                \textbf{Groups} & \textbf{Pics 1} & \textbf{Pics 2} & \textbf{Pics 3} & \textbf{Average} \\ \midrule
                \textbf{Lips} & 3 & 3 & 4 & \(3\frac{1}{3}\) \\ \midrule
                \textbf{Teeth} & 4 & 4 & 3 & \(3\frac{2}{3}\) \\ \midrule
                \textbf{Stretch} & 4 & 5 & 6 & 5 \\ \midrule
                \textbf{J. Jacks} & 4 & 2 & 3 & 3 \\ \bottomrule
            \end{tabular}

        \end{coloredlist}
        \item \textbf{Facial Feedback Hypothesis}
        \begin{coloredlist}
            \item Selective activation or inhibition of facial muscles has a strong impact on emotional responses to stimuli.
            \item Zygomatic major muscle.
            \begin{coloredlist}
                \item When we had the pen in our teeth, we were activating the zygomatic major muscle.
                \item This muscle is responsible for smiling.
            \end{coloredlist}
            \item Our data supported this hypothesis with a probability of \(p < 0.02\).
        \end{coloredlist}
        \item \textbf{Arousal}
        \begin{coloredlist}
            \item Increased heart rate in many emotions.
            \item Heart rate and attraction
            \begin{coloredlist}
                \item 1973 Dutton and Aron
                \begin{coloredlist}
                    \item Shaky high bridge vs. low stable bridge.
                    \item Woman on the other side who is asking questionnaire questions (faux DV).
                    \item She gave her phone number to the guys once they got done answering the questions.
                    \item The actual DV was the amount of phone calls she received and the sexual content in questionnaire answers.
                    \item The high bridge group had more sexual content in their messages.
                \end{coloredlist}
                \item 15 minutes of physical activity, then rate attractiveness of potential mates.
            \end{coloredlist}
        \end{coloredlist}
    \end{coloredlist}
\end{enumerate}

\cyanit{Psychophysiology:} Behavioral, cognitive, emotional, and social events are all mirrored in physiological processes. \\

The idea is that we can get a peep into your psychology by looking at what your biology is doing. \\

\textbf{Sleep:} EEG (Electroencephalogram; measuring brain activity), EOG (Electrooculogram; measuring eye movement), EMG (Electromyography; measuring muscle movement), ERP (Event-Related Potential; measuring electrical activity in the brain in response to a stimulus). \\ 

When your brain is hooked up to the ERP and you are asked either task relevant-stimulus, important stimulus, or surprising stimulus, it will produce a higher \(p\)\textemdash300 amplitude compared to otherwise. Think about how this would be used for interrogating a suspect, for example. \\ 

Then, there are \(n\)\textemdash350 amplitudes which are activated when we are asleep, and we hear our name, for example. \\

Notice the implications of this with sleeping: the more tired you are (or closer to falling asleep), the harder it will be to find the \(p\)\textemdash300 amplitude. \\

\cyanit{Omitted Stimulus Paradigm} -- Given a constant stimulus, this is the phenomena wherein there is a gap in the pattern. This will result in a higher \(p\)\textemdash300 amplitude. \\

\textbf{Polygraph:} Respiration, GSR (EDA), Blood flow, Blood pressure, and heart rate. \\

With a polygraph, we're looking at all the components of the peripheral nervous system. That is, when we're ``looking at  '' a polygraph, we're not measuring lying, but all the physiological responses that are associated with lying. \\



\section*{EDA}

\begin{coloredlist}
    \item \cyanit{Electrodermal Activity}
    \begin{coloredlist}
        \item Old name: Galvanic Skin Response
        \item Measuring sympathetic nervous system activity by detecting sweat gland activity by measuring the conductance of an electrical signal from one electrode to another.
        \item More rapid conductance with more activity.
        \item Particularly good for emotion.
        \item Maybe attention.
        \item This also is a good measure for the sympathetic nervous system because it is the only one that can enervate the sweat glands.
    \end{coloredlist}
\end{coloredlist}

\subsection*{What if I Want to Know}

\begin{coloredlist}
    \item If you're lying
    \item Cognitive or emotional states when you're sleeping
    \item If you're attending to stimuli I'm presenting
\end{coloredlist}

\textbf{Psychophysiology is different from Physiological Psychology}. Note that Psychophysiology is where mind/behavior is the IV, and physiology is the DV. Similarly, Physiological Psychology uses the same IVs and DVs, but notice that we are manipulating the physiological psychology to measure psychology. Remember the independent variable is first (mnemonic). \\

\subsubsection*{Examples}

\begin{coloredlist}
    \item Present snake photo or not. Then measure effects on physiology that evidence fear.
    \item Presenting in-group and out-group photographs. Then measure effects on physiology that evidence prejudicial cognition.
    \item Drugs: Measure effects on psychology (behavior/aggression)
    \item Lesion: Measure effects on psychology (cognition/memory)
    \item Manipulate heart rate: (behavior/attractiveness ratings)
    \item Electrical Stim of Brain [tDCS (transcranial direct current stimulation)]: Measure effects on psychology (mood/emotion/depression)
\end{coloredlist}

\section*{EEG}

\begin{coloredlist}
    \item Activity in large groups of neurons.
    \item Difference in electrical activity at a reference point and site of interest.
    \item You get: Wavelike patterns.
    \item \cyanit{International 10-20 system}.
    \begin{coloredlist}
        \item Fz, Fp1, Fp2, F7, F8, F3, F4, T3, T4, Cz, C3, C4, T5, T6, Pz, P3, P4, O1, O2.
        \begin{multicols}{2}
            \begin{coloredlist}
                \item F = Frontal
                \item T = Temporal
                \item C = Central
                \item P = Parietal
                \item O = Occipital
                \item Odd numbers = left hemisphere.
                \item Even numbers = right hemisphere.
                \item Z = Midline
            \end{coloredlist}
        \end{multicols}
    \end{coloredlist}
    \item Researchers use \cyanit{visual inspection} to look at the EEG data.
    \item Measurements
    \begin{coloredlist}
        \item \cyanit{Frequency} -- \(\frac{1}{\text{time}}\) (in Hz)
        \item \cyanit{Amplitude} -- Height of wave (in \(\mu \text{V}\))
    \end{coloredlist}
\end{coloredlist}

\section*{Neurofeedback}

\begin{coloredlist}
    \item Learn to control your brain activity.
    \begin{coloredlist}
        \item See the activity
        \item Get reinforcement or punishment
        \item Make changes
        \item Even if you don't ``know'' what you're doing, you can still learn to control your brain activity.
    \end{coloredlist}
\end{coloredlist}

\newpage

\begin{center}
    \textbf{NEW NOTES FOR 03/20/25} \\
    \hrulefill
\end{center}

\section{Physiological Psychology (Video Lab) Experiment}

Research Question: \cyanit{Do opioids play a role in social behavior?}

\begin{coloredlist}
    \item Pros of videoing:
    \begin{coloredlist}
        \item Minimizes the effect of the experimenter.
        \begin{coloredlist}
            \item Called ``reactivity''
        \end{coloredlist}
        \item Permanent data record
        \item Double blind procedures
        \begin{coloredlist}
            \item Observer bias/experimenter effect
        \end{coloredlist}
    \end{coloredlist}
\end{coloredlist}

\subsection{Good Introductions}

\begin{coloredlist}
    \item \cyanit{Rationale} -- Why is this study important?
    \begin{coloredlist}
        \item 
    \end{coloredlist}
    \item \cyanit{Questions} -- What are you trying to answer?
    \item \cyanit{Hypothesis} -- What do you think will happen?
    \item \cyanit{Support for Hypotheses} -- Why do you think this will happen?
    \item 
\end{coloredlist}

\section{Classes of Neurotransmitters}

\begin{coloredlist}
    \item Peptides (AKA: Neuropeptides)
    \begin{coloredlist}
        \item Endogenous opioids
    \end{coloredlist}
\end{coloredlist}

\subsection{Peptides---Opioids}

\begin{coloredlist}
    \item \cyanit{Peptides} -- Short chain of amino acids.
    \item \textbf{Opioid vs. Opiates:}
    \begin{coloredlist}
        \item \cyanit{Opioid}: Endogenous (from inside the body).
        \item \cyanit{Opiate}: Exogenous (from outside the body) morphine, codeine, heroin, percodan (\textit{levorphanol}), opium, OxyCotin, fentanyl, etc.
        \begin{coloredlist}
            \item All agonists.
        \end{coloredlist}
        \item Richard Deyo, the guy whose data we're going to be using says opiates are the ones derived by plants.
    \end{coloredlist}
\end{coloredlist}

\subsection{Antagonists}

\begin{coloredlist}
    \item NARCAN (Naloxone) -- Opioid antagonist.
    \begin{coloredlist}
        \item Reverses opiate intoxication.
    \end{coloredlist}
    \item Vivitrol (Naltrexone) -- Opioid antagonist.
    \begin{coloredlist}
        \item Blocks euphoric effects but may not work that well on addiction because doesn't get rid of cravings.
    \end{coloredlist}
\end{coloredlist}

\subsection{New NonOpioid Pain Relievers}

\begin{coloredlist}
    \item VX-548 (January 2024---Late stage trials)
    \begin{coloredlist}
        \item Acute post-surgical pain.
        \item No addiction risk.
    \end{coloredlist}
    \item Blocks Na\(^{+}\) channels in peripheral (not CNS) pain pathways.
    \item Better than placebo but not as good as hydrocodone + acetaminophen.
\end{coloredlist}

\section{Opioid Receptors}

\begin{coloredlist}
    \item 3 Kinds:
    \begin{coloredlist}
        \item \cyanit{Mu} (\(\mu\)) -- Analgesia and euphoria.
        \item \cyanit{Delta} (\(\delta\)) -- Analgesia
        \item \cyanit{Kappa} (\(\kappa\)) -- \textit{Colocalized} with certain \textit{catecholamines}, learning and memory, emotional control, stress response and analgesia.
        \begin{coloredlist}
            \item \cyanit{Colocalized} -- When two or more neurotransmitters are released from the same neuron.
            \item \cyanit{Catecholamines} -- Dopamine, norepinephrine, and epinephrine.
        \end{coloredlist}        
    \end{coloredlist}
    \item\ [The process was] Discovered in early 1970s.
    \item Mid 1970s endogenous opioids were discovered.
    \begin{coloredlist}
        \item \cyanit{Endorphins} -- Binds to mu receptors the most.
        \item \cyanit{Enkephalins} -- Binds to delta receptors the most.
        \item \cyanit{Dynorphins} -- Binds to kappa receptors the most.
    \end{coloredlist}
\end{coloredlist}

\section{Functions of Opioids}

\begin{coloredlist}
    \item Prevents diarrhea.
    \item Euphoria.
    \item Analgesia.
    \item Stress response, body temperature, emotions, feeding  motivation, sexual behavior, learning (reinforcement), drowsiness.
    \item Social behavior?
\end{coloredlist}

\subsection{Pain is Good}

\begin{coloredlist}
    \item Unpleasant sensory and emotional experience associated with actual or potential tissue damage.
\end{coloredlist}

\subsubsection{Pain Pathway}

\begin{coloredlist}
    \item From face
    \begin{coloredlist}
        \item Comes in via cranial nerve V (trigeminal nerve).
    \end{coloredlist}
    \item From neck down
    \begin{coloredlist}
        \item Cones in via a spinal nerve in one of 2 pathways:
        \begin{coloredlist}
            \item Direct (fast pain pathway)
            \begin{coloredlist}
                \item Sharp and well localized pain.
                \item Immediate and brief.
                \item \cyanit{A-delta fibers} (myelinated) -- Fast conducting fibers.
                \item Mechanical (strong) and thermal (extreme temperature).
            \end{coloredlist}
            \item Indirect (slow pain pathway)
            \begin{coloredlist}
                \item Slow pain.
                \item Throbbing, aching and dull pain.
                \item Takes longer, but lingers.
                \item \cyanit{C-fibers} (unmyelinated) -- Slow conducting fibers.
                \item Chemical (inflammatory) pain.
                \item Receptors are sensitive to chemical simulation.
                \item \cyanit{Prostaglandins} -- Inactive until tissue damage occurs.
                \begin{coloredlist}
                    \item \cyanit{Cyclooxygenase} (COX) -- Enzyme that converts inactive prostaglandins to active prostaglandins.
                \end{coloredlist}         
                \item Pain arrives at the brain stem reticular formation (BSRF)
                \item After the BSRF:
                \begin{coloredlist}
                    \item Thalamus (Arousal) 
                    \item Limbic system: Anterior Cingulate Cortex (ACC) (emotional)
                    \item Periaqueductal gray region (Opioids)
                    \begin{coloredlist}
                        \item This is activated when there is too much pain.
                        \item For example, a person gets their finger cut off; instead of incapacitating the person, the periaqueductal gray region will activate and release opioids to help the person get away from whatever is causing the pain.
                    \end{coloredlist}
                    \item Frontal lobes (Association)
                    \item and others.
                \end{coloredlist}       
            \end{coloredlist}
        \end{coloredlist}
    \end{coloredlist}    
\end{coloredlist}

\section{NSAIDs}

\begin{coloredlist}
    \item 2 Classes:
    \begin{coloredlist}
        \item \cyanit{Proprionic Acid Derivatives} -- Ibuprofen, naproxen, ketoprofen.
        \item \cyanit{Salicylates} -- Aspirin, diflunisal, salsalate.
    \end{coloredlist}
    \item Block cyclooxygenase
    \item \cyanit{COX-2 Inhibitors} -- \textit{celecoxib} (Celebrex), \textit{rofecoxib} (Vioxx) (removed from market because heart attack and stroke).
\end{coloredlist}

\section{Opioid Systems at Work?}

\begin{coloredlist}
    \item Acupuncture
    \item The placebo effect activates the opioid system.
    \item For example, if I were to give you a pill, and said it was going to relieve your pain, but it was actually just a sugar pill, you would still feel relief from the pain. However, if I gave you a pill, such as Nalaxone (which is an opioid antagonist), you would not feel relief from the pain. 
\end{coloredlist}

\begin{center}
    \textbf{New Notes for 04/03/25} \\
    \hrulefill
\end{center}

\section{Carefully Plan Some Hypotheses}

\begin{coloredlist}
    \item \cyanit{Conceptual Hypothesis} -- A statement of the expected relationship between the IV and DV.
    \item \cyanit{Statistical Hypothesis} -- A statement of the expected relationship between the IV and DV in statistical terms.
    \item Consider the example ``Does technology use disrupt sleep quality.''
    \begin{coloredlist}
        \item Conceptual: 
        \begin{coloredlist}
            \item Null: TU (tech users) do NOT have lower sleep quality than NTU (non-tech users).
            \item Alternative: TU have lower sleep quality than NTU.
        \end{coloredlist} 
        \item Statistical:
        \begin{coloredlist}
            \item Null: \(\mu_{TU} \geq \mu_{NTU}\)
            \item Alternative: \(\mu_{TU} < \mu_{NTU}\)
        \end{coloredlist}
        \item For the test that we would use to test this hypothesis, we would use a paired t-test.
    \end{coloredlist}
    \item Consider another example ``Does too little vitamin D lead to depressed mood?''
    \begin{coloredlist}
        \item Conceptual:
        \begin{coloredlist}
            \item Null: Adequate VD (vitamin D) participants do NOT have lower mood than inadequate IVD (inadequate vitamin D) participants.
            \item Alternative: Adequate VD participants have lower mood than IVD participants.  
        \end{coloredlist}
        \item Alternative:
        \begin{coloredlist}
            \item Null: \(\mu_{VD} \leq \mu_{IVD}\)
            \item Alternative: \(\mu_{VD} > \mu_{IVD}\)
        \end{coloredlist}
    \end{coloredlist}
    \item For our question, ``Does nalaxone increase play?'' we need to be careful how we operationalize play.
    \item For rats, they play by:
    \begin{coloredlist}
        \item Boxing, chasing, dorsal contact, PINNING, and social proximity. But these are not sufficient enough.
    \end{coloredlist}
\end{coloredlist}

\subsection{Nonspecific Effects}

\begin{coloredlist}
    \item The IV produces changes in the DV by acting on a system not directly related to the system in question.
    \item For example, I sleep deprive people and they have less sex. Thus, does sleep deprivation interfere with sex OR maybe, sleep deprivation reduces motivation and that leads to sex declines along with lots of other things indirectly.
    \item \textbf{Could the effect we see on play be due to something else?}
    \begin{coloredlist}
        \item Level of behavior
        \begin{coloredlist}
            \item Spontaneous activity
            \item Crossing
        \end{coloredlist}
        \item Motivation
        \begin{coloredlist}
            \item Exploratory behavior
            \item Rearing
        \end{coloredlist}
        \item Fear of emotionality
        \begin{coloredlist}
            \item If no food is available, grooming
        \end{coloredlist}
    \end{coloredlist}
\end{coloredlist}

\subsection{Results}

\begin{coloredlist}
    \item IRR
    \begin{coloredlist}
        \item For play and nonspecific effect (crossing)
        \item Descriptive (includes graph)
    \end{coloredlist}
    \item Play
    \begin{coloredlist}
        \item Descriptive (includes graph)
        \item Inferential
    \end{coloredlist}
    \item Nonspecific effect
    \begin{coloredlist}
        \item Descriptive (includes graph)
        \item Inferential
    \end{coloredlist}
    \item Play and nonspecific effect separately
    \item How can I show the relationship between rater 1 and rater 2's ratings?
    \begin{coloredlist}
        \item Graph?
        \begin{coloredlist}
            \item Scatterplot
        \end{coloredlist}
        \item Statistic?
        \begin{coloredlist}
            \item Perason's Correlation coefficient (r)
        \end{coloredlist}
        \item Jamovi Variables?
        \begin{coloredlist}
            \item Rater1\_pins, Rater2\_pins, Rater1\_ns, Rater2\_ns
        \end{coloredlist}
    \end{coloredlist}
\end{coloredlist}
