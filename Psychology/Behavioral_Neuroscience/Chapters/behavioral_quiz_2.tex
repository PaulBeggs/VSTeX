% ===========================
% Section 2.1 -- The Neuron
% ===========================
\section*{2.1}
\begin{enumerate}[label=\textbf{Q2.1.\arabic*}]
      \item \textbf{Short Answer:} Define a neuron. \\
            % \textit{Answer:} Basic information processing unit of the nervous system.
      \item \textbf{Multiple Choice:} Which cell type forms the myelin sheath in the CNS?
            \begin{tasks}[label=\textcolor{draculafg}{(\Alph*)}, item-format=\color{draculafg}, label-width=1.5em, item-indent=1.7em](2)
                  \task Oligodendrocytes
                  \task Schwann Cells
                  \task Astrocytes
                  \task Satellite Cells
            \end{tasks}
            % \textit{Answer:} A
      \item \textbf{Fill in the Blank:} The process by which glial cells remove debris is called \underline{\hspace{3cm}}. \\
            % \textit{Answer:} Phagocytosis.
      \item \textbf{Short Answer:} Name two functions of astrocytes. \\
            % \textit{Answer:} Provide structural support and deliver nutrients to neurons.
      \item \textbf{Fill in the Blank:} \underline{\hspace{3cm}} are responsible for myelination in the PNS. \\
            % \textit{Answer:} Schwann Cells.
      \item \textbf{Multiple Choice:} What is one function of myelin?
            \begin{tasks}[label=\textcolor{draculafg}{(\Alph*)}, item-format=\color{draculafg}, label-width=1.5em, item-indent=1.7em](2)
                  \task Insulate axons
                  \task Synthesize proteins
                  \task Produce neurotransmitters
                  \task Break down debris
            \end{tasks}
            % \textit{Answer:} A.
      \item \textbf{Fill in the Blank:} Glial cells compose about half of nervous tissue volume, but are approximately \underline{\hspace{3cm}} times more numerous than neurons. \\
            % \textit{Answer:} 10-50.

      \item \textbf{Short Answer:} What are two similarities between neurons and other animal cells? \\
            % \textit{Answer:} They have a cell membrane, a nucleus, and a variety of organelles.
      \item \textbf{Short Answer:} In neurons, what is the primary role of mitochondria? \\
            % \textit{Answer:} Producing energy for the cell.
      \item \textbf{Fill in the Blank:} The organelle responsible for protein synthesis in neurons is the \underline{\hspace{3cm}}. \\
            % \textit{Answer:} Endoplasmic Reticulum.
      \item \textbf{Fill in the Blank:} The \underline{\hspace{3cm}} packages neurotransmitters for transport. \\
            % \textit{Answer:} Golgi Apparatus.
      \item \textbf{Fill in the Blank:} Organelles that break down waste products in neurons are called \underline{\hspace{3cm}}. \\
            % \textit{Answer:} Lysosomes.
      \item \textbf{Short Answer:} What distinguishes the morphology of neurons from typical animal cells? \\
            % \textit{Answer:} Neurons have long processes (axons and dendrites) specialized for signal transmission.
      \item \textbf{Fill in the Blank:} Neurons communicate via an \underline{\hspace{3cm}} process. \\
            % \textit{Answer:} electrochemical.
      \item \textbf{Multiple Choice:} Which cell type is primarily responsible for debris removal in the CNS?
            \begin{tasks}[label=\textcolor{draculafg}{(\Alph*)}, item-format=\color{draculafg}, label-width=1.5em, item-indent=1.7em](2)
                  \task Microglia
                  \task Astrocytes
                  \task Oligodendrocytes
                  \task Satellite Cells
            \end{tasks}
            % \textit{Answer:} A.
      \item \textbf{Multiple Choice:} Which cells line the ventricles and help form CSF?
            \begin{tasks}[label=\textcolor{draculafg}{(\Alph*)}, item-format=\color{draculafg}, label-width=1.5em, item-indent=1.7em](2)
                  \task Astrocytes
                  \task Ependymal Glia
                  \task Schwann Cells
                  \task Microglia
            \end{tasks}
            % \textit{Answer:} B.
      \item \textbf{Short Answer:} What is the function of satellite cells in the PNS? \\
            % \textit{Answer:} They provide nutrients and physical support to neurons.

      \item \textbf{Multiple Choice:} Which glial cell in the PNS is notably associated with neuronal regeneration?
            \begin{tasks}[label=\textcolor{draculafg}{(\Alph*)}, item-format=\color{draculafg}, label-width=1.5em, item-indent=1.7em](2)
                  \task Oligodendrocytes
                  \task Microglia
                  \task Satellite Cells
                  \task Schwann Cells
            \end{tasks}
            % \textit{Answer:} D.

      \item \textbf{Multiple Choice:} What happens during maintenance of internal consistency?
            \begin{tasks}[label=\textcolor{draculafg}{(\Alph*)}, item-format=\color{draculafg}, label-width=1.5em, item-indent=1.7em](1)
                  \task Microglia remove cellular debris.
                  \task Oligodendrocytes myelinate axons.
                  \task Astrocytes absorb excess potassium ions.
                  \task Schwann cells provide nutrients to neurons.
            \end{tasks}
            % \textit{Answer:} C.
      \item \textbf{Long Answer:} What evidence supports the notion that glial cells' malfunctioning may be contributing to Alzheimer's Disease? (Your answer must include beta amyloid, Tau, and the possible cause of Alzheimer's Disease.) \\
            % \textit{Answer:} Glial cells normally remove beta amyloid plaques, but if it builds up too much outside the cell, Tau builds up inside the cell. This leads to inflammation, which may be the problem of Alzheimer's Disease.

\end{enumerate}
\newpage
\squigglyline
% ===========================
% Section 2.2 Different Kinds of Neurons
% ===========================

\section*{2.2}
\begin{enumerate}[label=\textbf{Q2.2.\arabic*}]

      \item \textbf{Fill in the Blank:} The three structural classifications of neurons are \underline{\hspace{3cm}}, \underline{\hspace{3cm}}, and \underline{\hspace{3cm}}. \\
            % \textit{Answer:} Unipolar; Bipolar; Multipolar.



      \item \textbf{Fill in the Blank:} Neurons are based on \underline{\hspace{3cm}} and \underline{\hspace{3cm}}. \\
            % \textit{Answer:} Structure; function.


      \item \textbf{Fill in the Blank:} Sensory neurons carry information from the \underline{\hspace{3cm}} to the \underline{\hspace{3cm}}. \\
            % \textit{Answer:} PNS; CNS.

      \item \textbf{Short Answer:} What is the primary function of motor neurons? \\
            % \textit{Answer:} To carry information from the CNS to muscles and glands.

      \item \textbf{Multiple Choice:} What does ``Efferent'' mean?
            \begin{tasks}[label=\textcolor{draculafg}{(\Alph*)}, item-format=\color{draculafg}, label-width=1.5em, item-indent=1.7em](4)
                  \task Incoming
                  \task Sensory
                  \task Motor
                  \task Outgoing
            \end{tasks}
            % \textit{Answer:} D.

\end{enumerate}
\squigglyline

% ===========================
% Section 2.3 Neural Communication
% ===========================

\section*{2.3}
\begin{enumerate}[label=\textbf{Q2.3.\arabic*}]


      \item \textbf{Short Answer:} What are the 2 systems of neuronal communication? \\
            % \textit{Answer:} Binary and analogue

      \item \textbf{Fill in the Blank:} A stronger-than-normal stimulus is required to cause another action potential during the \underline{\hspace{3cm}} because of hyperpolarization. \\
            % \textit{Answer:} Relative refractory period.

      \item \textbf{Fill in the Blank:} The \textit{phospholipid bilayer} is made up of \underline{\hspace{3cm}} and \underline{\hspace{3cm}}. \\
            % \textit{Answer:} Hydrophilic heads; hydrophobic tails.

      \item \textbf{Fill in the Blank:} The period during which no amount of stimulation can cause another action potential is called the \underline{\hspace{3cm}}. \\
            % \textit{Answer:} Absolute refractory period.

      \item \textbf{Short Answer:} What is \textit{diffusion}? What else is it also known as? \\
            % \textit{Answer:} The movement of ions from high to low concentration; also known as the concentration gradient.

      \item \textbf{Short Answer:} What is the threshold of excitation, and what happens when it is reached? \\
            % \textit{Answer:} The threshold of excitation is +15 mV. When it is reached, an action potential is generated.

\newpage

      \item \textbf{Matching:} Match the following terms with their definitions.
            \begin{wordbox}
                  \begin{enumerate}
                        \item Ions move toward the opposing charge.
                        \item The charge the ion ``prefers'' to be at.
                        \item Cell is resistant to reexciation for some time after AP.
                        \item The cell membrane is more open to some ions than others.
                        \item \(-70\) mV (relative to outside).
                        \item The overshoot of negative voltage after an AP.
                        \item Costs 1 ATP. 
                        \item The phase where there is a jump in voltage after \(+15\) mV is reached.
                        \item The minimum amount needed to generate an action potential. (\(+15\) mV)
                        \item \underline{\hspace{10cm}} \\
                        \textit{Answer:} (for repolarization) the cell's charge declines. 
                  \end{enumerate}
            \end{wordbox}
            \begin{enumerate}[label=(\arabic*)]
                  \item Differential Permeability \quad \dotfill \quad \underline{\hspace{1cm}} \\
                  \item Resting Membrane Potential \quad \dotfill \quad \underline{\hspace{1cm}}\\
                  \item Threshold of Excitation \quad \dotfill \quad \underline{\hspace{1cm}}\\
                  \item Depolarization \quad \dotfill \quad \underline{\hspace{1cm}}\\
                  \item Repolarization \quad \dotfill \quad \underline{\hspace{1cm}}\\
                  \item Hyperpolarization \quad \dotfill \quad \underline{\hspace{1cm}}\\
                  \item Electrostatic Pressure \quad \dotfill \quad \underline{\hspace{1cm}}\\
                  \item Equilibrium Potential \quad \dotfill \quad \underline{\hspace{1cm}}\\
                  \item Sodium Potassium Pump \quad \dotfill \quad \underline{\hspace{1cm}}\\
                  \item Refractory Period \quad \dotfill \quad \underline{\hspace{1cm}}
            \end{enumerate}
            % \textit{Answers:} 1. d; 2. e; 3. i; 4. h; 5. j; 6. f; 7. a; 8. b; 9. g; 10. c. 
      \newpage

      

      \item \textbf{Fill in the Blank:} Fill out the following diagram of the resting membrane potential of a neuron. Draw the size of each and permeability ion to ``scale'', and draw the diffusion and electric arrows with the appropriate directions. \\

      % \begin{tikzpicture}
      %       % Labels for compartments
      %       \node at (-3,4.3) {\large \textbf{\underline{In}}};
      %       \node at (3,4.3) {\large \textbf{\underline{Out}}};
      %       \node at (-5.9,4.3) {\large \textbf{\underline{Permeability}}};
          
      %     %   % Draw compartments as boxes
      %     %   \draw (-5,-4) rectangle (-1,4); % Inside compartment
      %     %   \draw (1,-4) rectangle (5,4);   % Outside compartment
          
      %       % Draw the membrane as a double line:
      %       % Left (solid) side denotes the negative (inside) and right (dotted) side the positive (outside)
      %       \draw[line width=1pt] (0.1,-4) -- (0.1,4);  
      %       \draw[dotted, line width=1pt] (-0.1,-4) -- (-0.1,4);
          
      %       % Place ion symbols on the "in" side (left) with relative sizes.
      %       % Here, potassium is large since it is more prevalent on the inside.
      %       \node (K_in) at (-3,2.5) {\huge \(\text{K}^{+}\)};
      %       \node (Na_in) at (-3,0) {\normalsize \(\text{Na}^{+}\)};
      %       \node (Cl_in) at (-3,-2.5) {\normalsize \(\text{Cl}^{-}\)};
          
      %       \node (P_K) at (-5.9,2.5) {\huge \(\text{P}\phantom{^{+}}\)};
      %       \node (P_K) at (-5.9,0) {\tiny \(\text{P}\phantom{^{+}}\)};
      %       \node (P_K) at (-5.9,-2.5) {\Large \(\text{P}\phantom{^{-}}\)};
          
      %       % Place ion symbols on the "out" side (right)
      %       % Here, sodium is made larger, etc.
      %       \node (K_out) at (3,2.5) {\normalsize \(\text{K}^{+}\)};
      %       \node (Na_out) at (3,0) {\huge \(\text{Na}^{+}\)};
      %       \node (Cl_out) at (3,-2.5) {\Large \(\text{Cl}^{-}\)};
          
      %       % Create nodes on the membrane for arrow connections
      %       \node (mem2_left) at (-0.5,0) {};
      %       \node (mem3_left) at (1.5,-2.5) {};
      %       \node (mem3_right) at (-1.5,-2.5) {};
          
      %       % Define arrow styles:
      %       \tikzset{
      %         diffusion/.style={->, >=Stealth, line width = 2pt, red},
      %         electric/.style={->, >=Stealth, line width = 2pt, blue}
      %       };
          
      %       % For each ion on the inside, draw two arrows (curved differently) going from the ion to the membrane.
      %       % The differences in the bending (and thus length) illustrate different magnitudes.
      %       % --- For K_in:
      %       \draw[electric, bend right=15] (K_in) to (K_out);
      %       \draw[diffusion, bend right=15, line width = 6pt] (K_out) to (K_in);
      %       % --- For Na_in:
      %       \draw[electric, bend left=15] (Na_out) to (mem2_left);
      %       \draw[diffusion, bend right=15] (Na_out) to (mem2_left);
      %       % --- For Cl_in:
      %       \draw[electric, bend right=15] (Cl_in) to (mem3_left);
      %       \draw[diffusion, bend right=15] (Cl_out) to (mem3_right);
          
      %       % Add a legend (key) to denote arrow styles.
      %       \node[draw, rectangle, right=1cm of K_out] (legend) {
      %         \begin{tabular}{ll}
      %           \textbf{Key:} & \\
      %           \tikz{\draw[diffusion, ->, >=Stealth, thick, red] (0,0) -- (1,0);} & Diffusion \\
      %           \tikz{\draw[electric, ->, >=Stealth, thick, blue] (0,0) -- (1,0);} & Electric \\
      %         \end{tabular}
      %       };
      % \end{tikzpicture}
            
            \begin{tikzpicture}
                  % Labels for compartments
                  \node at (-2,4.3) {\large \textbf{\underline{In}}};
                  \node at (2,4.3) {\large \textbf{\underline{Out}}};
                  \node at (-5.9,4.3) {\large \textbf{\underline{Permeability}}};

                  %   % Draw compartments as boxes
                  %   \draw (-5,-4) rectangle (-1,4); % Inside compartment
                  %   \draw (1,-4) rectangle (5,4);   % Outside compartment

                  % Draw the membrane as a double line:
                  % Left (solid) side denotes the negative (inside) and right (dotted) side the positive (outside)
                  \draw[line width=1pt] (0.1,-4) -- (0.1,4);
                  \draw[dotted, line width=1pt] (-0.1,-4) -- (-0.1,4);

                  % Place ion symbols on the "in" side (left) with relative sizes.
                  % Here, potassium is large since it is more prevalent on the inside.
                  \node (K_in) at (-2,2.5) {\huge \underline{\phantom{K}}};
                  \node (Na_in) at (-2,0) {\huge \underline{\phantom{K}}};
                  \node (Cl_in) at (-2,-2.5) {\huge \underline{\phantom{K}}};

                  \node (P_K) at (-5.9,2.5) {\huge \underline{\phantom{K}}};
                  \node (P_K) at (-5.9,0) {\huge \underline{\phantom{K}}};
                  \node (P_K) at (-5.9,-2.5) {\huge \underline{\phantom{K}}};

                  % Place ion symbols on the "out" side (right)
                  % Here, sodium is made larger, etc.
                  \node (K_out) at (2,2.5) {\huge \underline{\phantom{K}}};
                  \node (Na_out) at (2,0) {\huge \underline{\phantom{K}}};
                  \node (Cl_out) at (2,-2.5) {\huge \underline{\phantom{K}}};

                  % Create nodes on the membrane for arrow connections
                  \node (mem2_left) at (-0.5,0) {};
                  \node (mem3_left) at (1.5,-2.5) {};
                  \node (mem3_right) at (-1.5,-2.5) {};

                  % Define arrow styles:
                  \tikzset{
                        diffusion/.style={->, >=Stealth, line width = 2pt, red},
                        electric/.style={->, >=Stealth, line width = 2pt, blue}
                  };

                  % Add a legend (key) to denote arrow styles.
                  \node[draw, rectangle, right=1cm of K_out] (legend) {
                        \begin{tabular}{ll}
                              \textbf{Key:} &           \\
                                            & Diffusion \\
                                            & Electric  \\
                        \end{tabular}
                  };
            \end{tikzpicture}

      \item \textbf{Short Answer:} Why don't the \(\text{Na}^{+}\) channels reopen during repolarization? (Question directly from notes.) \\
            % \textit{Answer:} Because of the refractory period.

      \item \textbf{Multiple Choice:} During which phase of the action potential is a neuron unable to fire another action potential, no matter how strong the stimulus?
            \begin{tasks}[label=\textcolor{draculafg}{(\Alph*)}, item-format=\color{draculafg}, label-width=1.5em, item-indent=1.7em](2)
                  \task Depolarization
                  \task Repolarization
                  \task Absolute refractory period
                  \task Relative refractory period
            \end{tasks}
            % \textit{Answer:} C.

      \item \textbf{Multiple Choice:} How can an all-or-none action potential signal convey information about stimulus intensity?
            \begin{tasks}[label=\textcolor{draculafg}{(\Alph*)}, item-format=\color{draculafg}, label-width=1.5em, item-indent=1.7em](1)
                  \task By increasing the amplitude of the action potentials
                  \task By decreasing the duration of action potentials
                  \task By changing the direction of action potentials
                  \task By increasing the frequency of action potentials
            \end{tasks}
            % \textit{Answer:} D.

      \item \textbf{Short Answer:} What are equilibrium values? \\
            % \textit{Answer:} It is the voltage at which the ion’s driving force in equals the driving force out.

      \item \textbf{Fill in the Blank:} The ratio of Na\(^{+}\) to K\(^{+}\) for the sodium-potassium pump is \underline{\hspace{0.5cm}} out for \underline{\hspace{0.5cm}} in. \\
            % \textit{Answer:} 3:2.

      \item \textbf{Short Answer:} What happens in a failed attempt at an action potential? \\
            % \textit{Answer:} The threshold is not reached; the membrane potential may change slightly but returns to -70 mV.

      \item \textbf{Short Answer:} What role does electrostatic pressure play in neuronal membrane potential? \\
            % \textit{Answer:} It drives ions toward the opposing charge, aiding in establishing the RMP.
% Make this multiple choice
      \item \textbf{Fill in the Blank:} The equilibrium values for K\(^{+}\) and Na\(^{+}\) are \underline{\hspace{0.5cm}} mV and \underline{\hspace{0.5cm}} mV, respectively. \\
            % \textit{Answer:} -80 mV; +55 mV. (I found a \href{https://www.ncbi.nlm.nih.gov/books/NBK538338/#:~:text=Since%20the%20plasma%20membrane%20at%20rest%20has,than%20it%20is%20for%20Na+%20(+65%20mV).}{paper online} that said -90 mV and +65 mV, respectively, so I don't know if these are correct.)

      \item \textbf{Short Answer:} \textit{Saltatory conduction} is the process of action potentials ``jumping'' from node to node. Why do people describe this process as ``jumping''? \\
            % \textit{Answer:} Because the action potential is regenerated only at each node, so it appears to jump along the axon. 

      \item \textbf{Short Answer:} What is the resting membrane potential (RMP) of a neuron and what does it represent? \\ 
            % \textit{Answer:} \(-70\) mV; it represents the voltage difference between the inside and outside of the cell at rest.

      \item \textbf{Short Answer:} What is meant by semipermeability in the context of the cell membrane? \\
            % \textit{Answer:} Only molecules that are lipid, lipid soluble, small, and neutral can cross the membrane.

      \item \textbf{Short Answer:} List the four main functions of embedded proteins in the cell membrane. \\
            % \textit{Answer:} They function as receptors, channels (including gated channels), pumps, and enzymes.

      \item \textbf{Multiple Choice:} Differential permeability says that this ion is more permeable than others. 
      \begin{tasks}[label=\textcolor{draculafg}{(\Alph*)}, item-format=\color{draculafg}, label-width=1.5em, item-indent=1.7em](4)
            \task Na\(^{+}\)
            \task K\(^{+}\)
            \task Cl\(^{-}\)
            \task Ca\(^{2+}\)
      \end{tasks}
      % \textit{Answer: B.}

      \item \textbf{Short Answer:} How does diffusion contribute to the RMP? \\
      % \textit{Answer:} Ions move from high to low concentration across the membrane.

      \item \textbf{Multiple Choice:} What is the rate law? 
            \begin{tasks}[label=\textcolor{draculafg}{(\Alph*)}, item-format=\color{draculafg}, label-width=1.5em, item-indent=1.7em](1)
                  \task The speed at which a neuron conducts an action potential along its axon.
                  \task The time delay between the onset of a stimulus and the initiation of an action potential.
                  \task The relationship between the amplitude of an action potential and the strength of the stimulus.
                  \task Variations in the intensity of a stimulus are represented by the rate of action potentials.
                  \task None of the above.
            \end{tasks}
            % \textit{Answer: D.}


      \newpage 

      \item \textbf{Matching:} For the diagram below, identify the processes of an action potential at each labeled phase.


            \begin{tikzpicture}[scale=1.2]

                  % Draw axes
                  \draw[->, thick] (0,-4) -- (7,-4) node[right] {Time (ms)};
                  \draw[->, thick] (0,-4) -- (0,3) node[above] {Voltage (mV)};

                  % y-axis ticks and labels (scaling: 1 unit = 20 mV)
                  \foreach \y/\label in {-3.5/{-70},-2.5/{-55}, -2/{-40}, -1/{-20}, 0/{0}, 1/{20}, 2/{40}} {
                  \draw (0,\y) -- (-0.2,\y) node[left] {\label};
                  }

                  % x-axis ticks
                  \foreach \x in {0,1,2,3,4,5,6} {
                              \draw (\x,-4) -- (\x,-4.15) node[below] {\x};
                        }

                  % Draw action potential waveform using coordinates:
                  % Points (x, y) where y = mV/20:
                  % (0,-3.5):   Rest (-70 mV)
                  % (1,-3.5):   Rest
                  % (2,-2.75):  Threshold (-55 mV, since -55/20 = -2.75)
                  % (3,2):      Peak depolarization (+40 mV, 40/20 = 2)
                  % (4,-0.5):   Repolarization (-10 mV, -10/20 = -0.5)
                  % (5,-4):     Hyperpolarization (-80 mV, -80/20 = -4)
                  % (6,-3.5):   Return to rest (-70 mV)
                  \draw[thick, horange] plot coordinates {
                              (0,-3.5)
                              (1,-3.5)
                              (1.5,-3.5)
                        };
                  \draw[thick, horange, smooth] plot coordinates {
                              (1.5,-3.5)
                              (1.90,-2.55)
                        };
                  \draw[thick, horange] plot coordinates {
                              (1.90,-2.55)
                              (2,-2.5)
                        };
                  \draw[thick, horange, smooth] plot coordinates {
                              (2,-2.5)
                              (3,2)
                              (4,-0.5)
                              (5,-3.55)
                              (6,-3.5)
                        };
                  \draw[thick, horange] plot coordinates {
                              (6,-3.5)
                              (7,-3.5)
                        };

                  % Add phase labels near the waveform
                  \tikzset{
                        circ/.style = {draw, circle, scale=0.7}
                  };
                  % Receives stimulus.
                  \node[circ] at (1.3,-3.2) {1};
                  % Opens sodium channels.
                  \node[circ] at (1.75,-2.2) {2};
                  % Potassium channels open.
                  \node[circ] at (2.5,1.5) {3};
                  % Sodium channels close.
                  \node[circ] at (3.2,2.3) {4};
                  % Potassium channels close.
                  \node[circ] at (4.7,-3.75) {5};

                  % \draw[decorate, decoration={brace, amplitude=5.5pt}]
                  % (5,-3.5) -- (6,-3.5);
                  % Add text labels for phases
                  % \node at (2.2, 0) [rotate=78] {\textbf{Depolarization}};
                  % % The potassium is leaving the cell.
                  % \node at (4.2, 0) [rotate=287] {\textbf{Repolarization}};
                  % \node at (5.4, -3.1) {\textbf{Hyperpolarization}};

                  % Arrow pointing hyperpolarization to -80 mv
                  % \draw[->, thick] (7, -2) -- (4.93,-3.9) node[below] {};

            \end{tikzpicture}


            \begin{wordbox}
                  \begin{enumerate}[label=(\alph*)]
                        \item Potassium channels close.
                        \item Resting potential.
                        \item Potassium channels open.
                        \item Opens sodium channels.
                        \item Sodium channels close.
                  \end{enumerate}
            \end{wordbox}
            \begin{enumerate}[label=(\arabic*)]
                  \item \(-70\) mV \quad \dotfill \quad \underline{\hspace{1cm}}\\
                  \item \(-55\) mV \quad \dotfill \quad \underline{\hspace{1cm}}\\
                  \item \(+35\) mV \quad \dotfill \quad \underline{\hspace{1cm}}\\
                  \item \(+40\) mV \quad \dotfill \quad \underline{\hspace{1cm}}\\
                  \item \(-75\) mV \quad \dotfill \quad \underline{\hspace{1cm}}
            \end{enumerate}

\newpage
      



            \item \textbf{Multiple Choice:} What is the main advantage of saltatory conduction in myelinated axons?
            \begin{tasks}[label=\textcolor{draculafg}{(\Alph*)}, item-format=\color{draculafg}, label-width=1.5em, item-indent=1.7em](1)
                  \task It eliminates the need for Na\(^+\) channels.
                  \task It increases conduction speed and reduces energy expenditure.
                  \task It allows action potentials to travel in both directions.
                  \task It prevents action potentials from occurring at all.
            \end{tasks}
            % \textit{Answer:} B. 
            
      \item \textbf{Fill in the Blank:} The \underline{\hspace{3cm}} is the site of action potential initiation in a neuron. \\
            % \textit{Answer:} Axon hillock.

      \item \textbf{Short Answer:} What is decremental conduction, and why does it occur? \\
            % \textit{Answer:} Decremental conduction is the gradual weakening of a signal as it travels due to resistance and leakage.

      \item \textbf{Short Answer:} Why does the action potential require active regeneration in unmyelinated axons? \\
            % \textit{Answer:} Because it is regenerated at each point along the axon to prevent signal loss.

      \item \textbf{Short Answer:} Why can't myelin sheaths extend indefinitely without nodes? \\
            % \textit{Answer:} Because the decremental conduction would cause the action potential to weaken too much before reaching the end.

      \item \textbf{Short Answer:} Explain multiple sclerosis in terms of saltatory conduction. \\
            % \textit{Answer:} Because saltatory conduction is what neurons that have myelin sheaths use to conduct action potentials, multiple sclerosis, which destroys myelin, disrupts this process.

\end{enumerate}

\squigglyline
% ===========================
% Section 2.4 Conversion from Electrical to Chemical Signals
% ===========================

\section*{2.4}
\begin{enumerate}[label=\textbf{Q2.4.\arabic*}]
      
      \item \textbf{Multiple Choice:} An \textit{angstrom} (\AA) is a unit of measurement equivalent to
      \begin{tasks}[label=\textcolor{draculafg}{(\Alph*)}, item-format=\color{draculafg}, label-width=1.5em, item-indent=1.7em](4)
            \task \(10^{-6}\) mm
            \task \(10^{-5}\) mm
            \task \(10^{-7}\) mm
            \task \(10^{-9}\) mm
      \end{tasks}
      % \textit{Answer:} C.
      
      \item \textbf{Fill in the Blank:} The synaptic cleft is about \underline{\hspace{3cm}} angstroms wide. \\
            % \textit{Answer:} 20 \AA. (I have 200 \AA \ in my notes, but I think it's a typo. I Googled it, and it said 20 \AA.)

      \item \textbf{Multiple Choice:} What happens FIRST when an action potential reaches the presynaptic terminal?
            \begin{tasks}[label=\textcolor{draculafg}{(\Alph*)}, item-format=\color{draculafg}, label-width=1.5em, item-indent=1.7em](1)
                  \task The Na\(^+\)/K\(^+\) pump is activated.
                  \task The postsynaptic neuron immediately fires an action potential.
                  \task Voltage-gated Ca\(^{2+}\) channels open.
                  \task The synaptic vesicles dissolve.
            \end{tasks}
            % \textit{Answer:} C.

      \item \textbf{Fill in the Blank:} Neurotransmitters are released into the \underline{\hspace{3cm}} when synaptic vesicles fuse with the presynaptic membrane. \\
            % \textit{Answer:} Synaptic cleft.

      \item \textbf{Short Answer:} What is the function of docking proteins at the presynaptic membrane? \\
            % \textit{Answer:} They hold synaptic vesicles in place until an action potential triggers neurotransmitter release.

            
            

\end{enumerate}

\squigglyline
% ===========================
% Section 2.5 Conversion from Chemical Back to Electrical Signals
% ===========================

\section*{2.5}
\begin{enumerate}[label=\textbf{Q2.5.\arabic*}]
      
      \item \textbf{Multiple Choice:} Which of the following is an excitatory postsynaptic potential (EPSP)?
            \begin{tasks}[label=\textcolor{draculafg}{(\Alph*)}, item-format=\color{draculafg}, label-width=1.5em, item-indent=1.7em](1)
                  \task Opening of K\(^+\) channels
                  \task Binding of neurotransmitters to autoreceptors
                  \task Metabolism of neurotransmitters
                  \task Opening of Na\(^+\) channels
            \end{tasks}
            % \textit{Answer:} D.

      \item \textbf{Fill in the Blank:} The three possible fates of neurotransmitters after release are \underline{\hspace{3cm}}, \underline{\hspace{3cm}}, and \underline{\hspace{3cm}}. \\
            % \textit{Answer:} Active reuptake, metabolism, binding to autoreceptors.

      \item \textbf{Fill in the Blank:} When a neurotransmitter causes the opening of Na\(^+\) channels, this creates a(n) \underline{\hspace{3cm}} post-synaptic potential. \\
            % \textit{Answer:} Excitatory.
      
      \item \textbf{Multiple Choice:} What occurs during an inhibitory post-synaptic potential (IPSP)? 
            \begin{tasks}[label=\textcolor{draculafg}{(\Alph*)}, item-format=\color{draculafg}, label-width=1.5em, item-indent=1.7em](1)
                  \task Opening of K\(^+\) channels, allowing potassium to enter the cell
                  \task Opening of Na\(^+\) channels, bringing the cell closer to threshold
                  \task Opening of K\(^+\) channels, allowing potassium to leave the cell
                  \task Closing of all ion channels
            \end{tasks}
            % \textit{Answer:} C.
      
      \item \textbf{Short Answer:} Where does the summation of EPSPs and IPSPs occur in a neuron? \\
            % \textit{Answer:} At the axon hillock.

      \item \textbf{Multiple Choice:} What is the primary difference between ionotropic and metabotropic synapses?
            \begin{tasks}[label=\textcolor{draculafg}{(\Alph*)}, item-format=\color{draculafg}, label-width=1.5em, item-indent=1.7em](1)
                  \task Ionotropic causes direct ion exchange, metabotropic uses secondary messengers
                  \task Ionotropic uses ATP, metabotropic doesn't
                  \task Ionotropic is slower, metabotropic is faster
                  \task Ionotropic uses multiple neurotransmitters, metabotropic uses only one
            \end{tasks}
            % \textit{Answer:} A.
            
      \item \textbf{Short Answer:} How does binding to autoreceptors affect neurotransmitter release? \\
            % \textit{Answer:} It inhibits the release of more neurotransmitters.
            
      \item \textbf{Fill in the Blank:} The enzyme that breaks down neurotransmitters as part of their post-release fate is involved in \underline{\hspace{3cm}}. \\
            % \textit{Answer:} Metabolism.


      \item \textbf{Short Answer:} How do ionotropic and metabotropic synapses differ in terms of speed and duration? \\
            % \textit{Answer:} Ionotropic synapses produce rapid and short effects by directly opening ion channels, while metabotropic synapses have slower, longer-lasting effects through secondary messenger pathways.

      \item \textbf{Multiple Choice:} What role does cAMP play in metabotropic synapses?
            \begin{tasks}[label=\textcolor{draculafg}{(\Alph*)}, item-format=\color{draculafg}, label-width=1.5em, item-indent=1.7em](1)
                  \task It binds directly to ion channels to open them.
                  \task It activates Protein Kinase A, which leads to phosphorylation.
                  \task It metabolizes excess neurotransmitters.
                  \task It breaks down ATP into ADP.
            \end{tasks}
            % \textit{Answer:} B.

      \item \textbf{Fill in the Blank:} The enzyme \underline{\hspace{3cm}} converts ATP into cAMP in metabotropic signaling. \\
            % \textit{Answer:} Adenylate cyclase.

      \item \textbf{Fill in the Blank:} The enzyme that metabolizes residual cAMP is called \underline{\hspace{3cm}}. \\
            % \textit{Answer:} Phosphodiesterase.
            
      \item \textbf{Fill in the Blank:} \underline{\hspace{3cm}} removes the phosphate and resets the channel in the metabotropic pathway. \\
            % \textit{Answer:} Phosphoprotein phosphatase.

      \item \textbf{Fill in the Blank:} The fine tuning of electrical signals is accomplished through presynaptic \underline{\hspace{3cm}} and \underline{\hspace{3cm}}. \\
            % \textit{Answer:} inhibition; facilitation.

      \item \textbf{Fill in the Blank:} In the metabotropic signaling pathway, the \underline{\hspace{3cm}} subunit of the G-protein binds to adenylate cyclase. \\
            % \textit{Answer:} Alpha.
            
      \item \textbf{Multiple Choice:} What is the function of Protein Kinase A in metabotropic signaling?
            \begin{tasks}[label=\textcolor{draculafg}{(\Alph*)}, item-format=\color{draculafg}, label-width=1.5em, item-indent=1.7em](1)
                  \task It causes subunits to dissociate and converts ATP to ADP
                  \task It converts ATP to cAMP
                  \task It opens ion channels directly
                  \task It breaks down excess neurotransmitters
            \end{tasks}
            % \textit{Answer:} A.
            
\newpage
      
      \item \textbf{Matching:} Match each characteristic with either ionotropic or metabotropic synapses.
            \begin{wordbox}
                  \begin{enumerate}[label=(\alph*)]
                        \item No change in metabolism
                        \item Variable duration (can be very long)
                        \item At least 2 neuromodulator molecules bind to receptor
                        \item Fixed duration (rapid and short)
                        \item Direct change of ions
                        \item Indirect exchange of ions
                        \item 1 neurotransmitter binds to 1 receptor
                        \item Actual change in cellular metabolism
                  \end{enumerate}
            \end{wordbox}
            \begin{enumerate}[label=(\arabic*)]
                  \item Ionotropic \quad \dotfill \quad \underline{\hspace{3cm}} \\ 
                  \item Metabotropic \quad \dotfill \quad \underline{\hspace{3cm}} \\
            \end{enumerate}
                        
      \item \textbf{Short Answer:} How does presynaptic facilitation affect neurotransmitter release? \\
            % \textit{Answer:} It increases the amount of neurotransmitters released.
            
      \item \textbf{Multiple Choice:} In gap junctions, what crosses between neurons?
            \begin{tasks}[label=\textcolor{draculafg}{(\Alph*)}, item-format=\color{draculafg}, label-width=1.5em, item-indent=1.7em](1)
                  \task Neurotransmitters
                  \task Channels that allow direct electrical transmission
                  \task Neuromodulators
                  \task The \(G_{\beta\gamma}\) complex from the G-protein
            \end{tasks}
            % \textit{Answer:} B.
            
      \item \textbf{Multiple Choice:} Which of the following is NOT a characteristic of electrical synapses?
            \begin{tasks}[label=\textcolor{draculafg}{(\Alph*)}, item-format=\color{draculafg}, label-width=1.5em, item-indent=1.7em](1)
                  \task They are very fast
                  \task They lack neurotransmitters
                  \task They cannot be facilitated or inhibited
                  \task They are common in the mammalian brain
            \end{tasks}
            % \textit{Answer:} D.
            
      \item \textbf{Short Answer:} Give an example of where electrical synapses are more common than in mammals. \\
            % \textit{Answer:} Fish with simpler CNS.
\end{enumerate}
\newpage
\squigglyline

% ===========================
% Section 2.6 Somatic vs. Autonomic Nervous Systems
% ===========================

\section*{2.6}

\begin{enumerate}[label=\textbf{Q2.6.\arabic*}]
      \item \textbf{Matching:} Match each characteristic with either the somatic or autonomic nervous system.
            \begin{wordbox}
                  \begin{enumerate}[label=(\alph*)]
                        \item Innervates striated muscles
                        \item Functions as a whole
                        \item Contains both sensory and motor neurons
                        \item More differentiated
                        \item Purely motor
                        \item Relatively involuntary
                        \item Innervates smooth muscle, cardiac muscle, glands
                        \item Voluntary control
                  \end{enumerate}
            \end{wordbox}
            \begin{enumerate}[label=(\arabic*)]
                  \item Somatic \quad \dotfill \quad \underline{\hspace{3cm}} \\
                  \item Autonomic \quad \dotfill \quad \underline{\hspace{3cm}}
            \end{enumerate}
            % \textit{Answers:} 1. a, c, d, h; 2. b, e, f, g. 
\end{enumerate}
\squigglyline

% ===========================
% Section 2.7 Sensory Systems
% ===========================

\section*{2.7}

\begin{enumerate}[label=\textbf{Q2.7.\arabic*}]

      \item \textbf{Short Answer:} Where are photoreceptors located in the visual system? \\
            % \textit{Answer:} In the retina.
            
      \item \textbf{Short Answer:} Where are hair cells found in the auditory system? \\
            % \textit{Answer:} In the cochlea.

      \item \textbf{Multiple Choice:} Which of the following is NOT a component of the somatosensory system?
            \begin{tasks}[label=\textcolor{draculafg}{(\Alph*)}, item-format=\color{draculafg}, label-width=1.5em, item-indent=1.7em](2)
                  \task Proprioception
                  \task Cutaneous senses
                  \task Gustation
                  \task Vestibular sensation
            \end{tasks}
            % \textit{Answer:} C.

      \item \textbf{Matching:} Match each sensory system with its receptor or mechanism.
            \begin{wordbox}
                  \begin{enumerate}[label=(\alph*)]
                        \item Light waves hit photoreceptors
                        \item Chemicals bind to receptors (in the nose)
                        \item Sound waves vibrate hair cells
                        \item Sense of body position
                        \item Chemicals bind to receptors (in the mouth)
                        \item Sense of movement
                        \item Balance
                        \item Touch, temperature, pain
                  \end{enumerate}
            \end{wordbox}
            \begin{enumerate}[label=(\arabic*)]
                  \item Vision \quad \dotfill \quad \underline{\hspace{1cm}} \\
                  \item Vestibular Sensation \quad \dotfill \quad \underline{\hspace{1cm}} \\
                  \item Audition \quad \dotfill \quad \underline{\hspace{1cm}} \\
                  \item Kinesthesis \quad \dotfill \quad \underline{\hspace{1cm}} \\
                  \item Olfaction \quad \dotfill \quad \underline{\hspace{1cm}} \\
                  \item Gustation \quad \dotfill \quad \underline{\hspace{1cm}} \\
                  \item Proprioception \quad \dotfill \quad \underline{\hspace{1cm}} \\
                  \item Cutaneous \quad \dotfill \quad \underline{\hspace{1cm}} 
            \end{enumerate}
            
      
            

            
      \item \textbf{Fill in the Blank:} The difference between a physicist and psychologist studying sensory systems is that the physicist measures \underline{\hspace{3cm}} while the psychologist measures \underline{\hspace{3cm}}. \\
            % \textit{Answer:} stimulus, perceptions.
            
      \item \textbf{Short Answer:} According to the notes, why is it important to understand that perceptions don't exactly match stimuli? \\
            % \textit{Answer:} Because perceptions (not actual stimuli) form the basis of behavior, such as when someone drives through a red light believing it was green.
\end{enumerate}

\newpage

\squigglyline

% ===========================
% Section 2.8 Comprehensive Review
% ===========================

\section*{2.8}

\begin{enumerate}[label=\textbf{Q2.8.\arabic*}]
      \item \textbf{Multiple Choice:} What process might affect calcium channels during presynaptic inhibition?
            \begin{tasks}[label=\textcolor{draculafg}{(\Alph*)}, item-format=\color{draculafg}, label-width=1.5em, item-indent=1.7em](1)
                  \task Some Ca\(^{2+}\) channels that would normally open remain closed
                  \task More Ca\(^{2+}\) channels open than usual
                  \task Ca\(^{2+}\) is pumped out of the cell more quickly
                  \task Ca\(^{2+}\) channels are replaced with Na\(^{+}\) channels
            \end{tasks}
            % \textit{Answer:} A.
            
      \item \textbf{Fill in the Blank:} In an analogy explaining presynaptic facilitation, instead of 3 APs, the effect would seem like \underline{\hspace{3cm}} APs. \\
            % \textit{Answer:} 4 or 5.

      \item \textbf{Multiple Choice:} Which of these statements best explains why sensory systems are important in psychology?
            \begin{tasks}[label=\textcolor{draculafg}{(\Alph*)}, item-format=\color{draculafg}, label-width=1.5em, item-indent=1.7em](1)
                  \task Sensory systems are the only way to measure brain activity
                  \task All psychological disorders involve sensory system dysfunction
                  \task Perceptions from sensory systems form the basis for behavior
                  \task Sensory systems are the most complex part of the nervous system
            \end{tasks}
            % \textit{Answer:} C.
            
      \item \textbf{Long Answer:} Describe the complete sequence of events in a metabotropic synapse, from neuromodulator binding to the resetting of the channel. \\[1in]
            % \textit{Answer:} The neuromodulator binds to a receptor initiating the process. This activates a G-protein, causing its alpha subunit to bind to adenylate cyclase. Adenylate cyclase converts ATP to cAMP, which activates Protein Kinase A. This causes 2 subunits to dissociate, removing inhibition from the catalytic portion and allowing it to convert ATP to ADP, producing a phosphate group. The process ends when the neuromodulator dissociates, stopping cAMP production. Then phosphodiesterase metabolizes residual cAMP and phosphoprotein phosphatase removes the phosphate and resets the channel.
\end{enumerate}

\squigglyline

% ===========================
% Section 2.9 Vision and Color
% ===========================

\section*{2.9}

\begin{enumerate}[label=\textbf{Q2.9.\arabic*}]
      \item \textbf{Multiple Choice:} What is the range of the Electromagnetic Spectrum?
            \begin{tasks}[label=\textcolor{draculafg}{(\Alph*)}, item-format=\color{draculafg}, label-width=1.5em, item-indent=1.7em](1)
                  \task 380-760 \(\mu\)m
                  \task 380-760 nm
                  \task 380-760 mm
                  \task 420-690 \AA
            \end{tasks}
            % \textit{Answer:} B.

      \item \textbf{Short Answer:} What are the 3 types of cones in the retina? What colors do they respond to? \\
            % \textit{Answer:} S-cones (Blue), M-cones (medium wavelengths), L-cones (long wavelengths).

      \item \textbf{Multiple Choice:} Rods are sensitive to \underline{\hspace{3cm}} light levels, whereas cones are sensitive to \underline{\hspace{3cm}}. 
            \begin{tasks}[label=\textcolor{draculafg}{(\Alph*)}, item-format=\color{draculafg}, label-width=1.5em, item-indent=1.7em](2)
                  \task high; low
                  \task low; color
                  \task bright; dim
                  \task color; brightness
            \end{tasks}
            % \textit{Answer:} B.

      \item \textbf{Short Answer:} What is the problem of univariance, and how do we overcome it? \\
            % \textit{Answer:} A single type of photoreceptor (cone or rod) cannot differentiate between changes in wavelength and intensity of light; we developed multiple cones to overcome it.

      \item \textbf{Multiple Choice:} What is the trichromatic theory of color vision?
            \begin{tasks}[label=\textcolor{draculafg}{(\Alph*)}, item-format=\color{draculafg}, label-width=1.5em, item-indent=1.7em](1)
                  \task The retina contains 3 types of rods, each sensitive to a different range of wavelengths.
                  \task The retina contains 3 types of cones, each sensitive to the same range of wavelengths.
                  \task The retina contains 3 types of cones, each sensitive to a different range of wavelengths.
                  \task The retina contains 3 types of rods, each sensitive to the same range of wavelengths.
            \end{tasks}
            % \textit{Answer:} C.

      \item \textbf{Fill in the Blank:} Russians use the word \underline{\hspace{3cm}} to describe the dark blue, and \underline{\hspace{3cm}} for light blue. \\
            % \textit{Answer:} Siniy, goluboy.
      

      \item \textbf{Short Answer:} What is top-down processing? \\
            % \textit{Answer:} When the brain uses knowledge and expectations to interpret sensory information.

      \item \textbf{Short Answer:} What is bottom-up processing? \\
            % \textit{Answer:} When the brain uses sensory information to form perceptions.

      \item \textbf{Fill in the Blank:} Background changes your \underline{\hspace{3cm}} perception. \\
            % \textit{Answer:} Color OR size. (Because a circle surrounded by larger circles will look smaller than the same circle surrounded by smaller circles.)

      \item \textbf{Short Answer:} What does the research from Radel and Clement-Guillotin tell us about how hungry people perceive wordsd? \\
            % \textit{Answer:} They ``see food words better.''

\end{enumerate}