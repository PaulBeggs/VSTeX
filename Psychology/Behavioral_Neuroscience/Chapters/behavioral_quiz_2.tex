% ===========================
% Section 2.1 -- The Neuron
% ===========================
\section*{2.1 Neurons and Glial Cells}
\begin{enumerate}[label=\textbf{Q2.1.\arabic*}]
      \item \textbf{Short Answer:} Define a neuron. \\
            % \textit{answer:} Basic information processing unit of the nervous system.
      \item \textbf{Multiple Choice:} Which cell type forms the myelin sheath in the PNS?
            \begin{tasks}[label=(\Alph*), label-width=1.5em, item-indent=1.7em](2)
                  \task Oligodendrocytes
                  \task Schwann Cells
                  \task Astrocytes
                  \task Satellite Cells
            \end{tasks}
            % \textit{answer:} B.
      \item \textbf{Fill in the Blank:} The process by which microglia remove debris is called \underline{\hspace{3cm}}. \\
            % \textit{answer:} Phagocytosis.
      \item \textbf{Short Answer:} Name two functions of astrocytes. \\
            % \textit{answer:} Provide structural support and deliver nutrients to neurons.
      \item \textbf{True or False:} Oligodendrocytes are responsible for myelination in the PNS. \\
            % \textit{answer:} False.
      \item \textbf{Multiple Choice:} What is one function of myelin?
            \begin{tasks}[label=(\Alph*), label-width=1.5em, item-indent=1.7em](2)
                  \task Insulate axons
                  \task Synthesize proteins
                  \task Produce neurotransmitters
                  \task Break down debris
            \end{tasks}
            % \textit{answer:} A.
      \item \textbf{Fill in the Blank:} Glial cells compose about half of nervous tissue volume, but are approximately \underline{\hspace{3cm}} times more numerous than neurons. \\
            % \textit{answer:} 10-50.

      \item \textbf{Short Answer:} What are two similarities between neurons and other animal cells? \\
            % \textit{answer:} They have a cell membrane, a nucleus, and a variety of organelles.
      \item \textbf{Short Answer:} In neurons, what is the primary role of mitochondria? \\
            % \textit{answer:} Producing energy for the cell.
      \item \textbf{Fill in the Blank:} The organelle responsible for protein synthesis in neurons is the \underline{\hspace{3cm}}. \\
            % \textit{answer:} Endoplasmic Reticulum.
      \item \textbf{Fill in the Blank:} The \underline{\hspace{3cm}} packages proteins for transport. \\
            % \textit{answer:} Golgi Apparatus.
      \item \textbf{Fill in the Blank:} Organelles that break down waste products in neurons are called \underline{\hspace{3cm}}. \\
            % \textit{answer:} Lysosomes.
      \item \textbf{Short Answer:} What distinguishes the morphology of neurons from typical animal cells? \\
            % \textit{answer:} Neurons have long processes (axons and dendrites) specialized for signal transmission.
      \item \textbf{Fill in the Blank:} Neurons communicate via an \underline{\hspace{3cm}} process. \\
            % \textit{answer:} electrochemical.
      \item \textbf{Multiple Choice:} Which cell type is primarily responsible for debris removal in the CNS?
            \begin{tasks}[label=(\Alph*), label-width=1.5em, item-indent=1.7em](1)
                  \task Astrocytes
                  \task Microglia
                  \task Oligodendrocytes
                  \task Satellite Cells
            \end{tasks}
            % \textit{answer:} B.
      \item \textbf{Multiple Choice:} Which cells line the ventricles and help form CSF?
            \begin{tasks}[label=(\Alph*), label-width=1.5em, item-indent=1.7em](1)
                  \task Astrocytes
                  \task Ependymal Glia
                  \task Schwann Cells
                  \task Microglia
            \end{tasks}
            % \textit{answer:} B.
      \item \textbf{Short Answer:} What is the function of satellite cells in the PNS? \\
            % \textit{answer:} They provide nutrients and physical support to neurons. \\
      \item \textbf{Multiple Choice:} Which glial cell in the PNS is notably associated with neuronal regeneration?
            \begin{tasks}[label=(\Alph*), label-width=1.5em, item-indent=1.7em](1)
                  \task Oligodendrocytes
                  \task Schwann Cells
                  \task Microglia
                  \task Satellite Cells
            \end{tasks}
            % \textit{answer:} B.

      \item \textbf{Short Answer:} What is the purpose of \textit{phagocytosis} in the nervous system? \\
            % \textit{answer:} To remove cellular debris and pathogens.

      \item \textbf{Multiple Choice:} What happens during maintenance of internal consistency?
            \begin{tasks}[label=(\Alph*), label-width=1.5em, item-indent=1.7em](1)
                  \task Microglia remove cellular debris.
                  \task Oligodendrocytes myelinate axons.
                  \task Astrocytes absorb excess potassium ions.
                  \task Schwann cells provide nutrients to neurons.
            \end{tasks}
            % \textit{answer:} C.

      \item \textbf{Long Answer:} What evidence supports the notion that glial cells' malfunctioning may be contributing to Alzheimer's Disease? (Your answer needs to include beta amyloid, Tau, and the possible cause of Alzheimer's Disease.) \\
            % \textit{answer:} Glial cells normally remove beta amyloid plaques, but if it builds up too much outside the cell, Tau builds up inside the cell. This leads to inflammation, which may be the problem of Alzheimer's Disease.

      \item \textbf{Fill in the Blank:} The three structural classifications of neurons are \underline{\hspace{3cm}}, \underline{\hspace{3cm}}, and \underline{\hspace{3cm}}. \\
            % \textit{answer:} Multipolar, Bipolar, Unipolar.
\end{enumerate}

% ===========================
% Section 2.2 
% ===========================

\section*{2.2 Neural Communication}

\begin{enumerate}[label=\textbf{Q2.2.\arabic*}]
      \item \textbf{Fill in the Blank:} Sensory neurons carry information from the \underline{\hspace{3cm}} to the \underline{\hspace{3cm}}. \\
            % \textit{answer:} PNS, CNS.

      \item \textbf{Short Answer:} What is the primary function of motor neurons? \\
            % \textit{answer:} To carry information from the CNS to muscles and glands.

      \item \textbf{Short Answer:} What are the 2 systems of neuronal communication?
            % \textit{answer:} Binary and analogue

      \item \textbf{Fill in the Blank:} The \textit{phospholipid bilayer} is made up of \underline{\hspace{3cm}} and \underline{\hspace{3cm}}. \\
            % \textit{answer:} Hydrophilic, hydrophobic.

      \item \textbf{Short Answer:} What are the four jobs that we care about for embedded proteins? \\
            % \textit{answer:} Receptors, ion channels, pumps, and enzymes.

            \newpage

      \item \textbf{Fill in the Blank:} Fill out the following diagram of the resting membrane potential of a neuron. Draw the size of each and permeability ion to ``scale'' and draw the diffusion and electric arrows with the appropriate directions. \\

            \begin{tikzpicture}
                  % Labels for compartments
                  \node at (-2,4.3) {\large \textbf{\underline{In}}};
                  \node at (2,4.3) {\large \textbf{\underline{Out}}};
                  \node at (-5.9,4.3) {\large \textbf{\underline{Permeability}}};

                  %   % Draw compartments as boxes
                  %   \draw (-5,-4) rectangle (-1,4); % Inside compartment
                  %   \draw (1,-4) rectangle (5,4);   % Outside compartment

                  % Draw the membrane as a double line:
                  % Left (solid) side denotes the negative (inside) and right (dotted) side the positive (outside)
                  \draw[line width=1pt] (0.1,-4) -- (0.1,4);
                  \draw[dotted, line width=1pt] (-0.1,-4) -- (-0.1,4);

                  % Place ion symbols on the "in" side (left) with relative sizes.
                  % Here, potassium is large since it is more prevalent on the inside.
                  \node (K_in) at (-3,2.5) {\huge \underline{\phantom{K}}};
                  \node (Na_in) at (-3,0) {\huge \underline{\phantom{K}}};
                  \node (Cl_in) at (-3,-2.5) {\huge \underline{\phantom{K}}};

                  \node (P_K) at (-4.8,2.5) {\huge \underline{\phantom{K}}};
                  \node (P_K) at (-4.8,0) {\huge \underline{\phantom{K}}};
                  \node (P_K) at (-4.8,-2.5) {\huge \underline{\phantom{K}}};

                  % Place ion symbols on the "out" side (right)
                  % Here, sodium is made larger, etc.
                  \node (K_out) at (3,2.5) {\huge \underline{\phantom{K}}};
                  \node (Na_out) at (3,0) {\huge \underline{\phantom{K}}};
                  \node (Cl_out) at (3,-2.5) {\huge \underline{\phantom{K}}};

                  % Create nodes on the membrane for arrow connections
                  \node (mem2_left) at (-0.5,0) {};
                  \node (mem3_left) at (1.5,-2.5) {};
                  \node (mem3_right) at (-1.5,-2.5) {};

                  % Define arrow styles:
                  \tikzset{
                        diffusion/.style={->, >=Stealth, line width = 2pt, red},
                        electric/.style={->, >=Stealth, line width = 2pt, blue}
                  };

                  % Add a legend (key) to denote arrow styles.
                  \node[draw, rectangle, right=1cm of K_out] (legend) {
                        \begin{tabular}{ll}
                              \textbf{Key:} &           \\
                                            & Diffusion \\
                                            & Electric  \\
                        \end{tabular}
                  };
            \end{tikzpicture}
      \item \textbf{Short Answer:} What is the threshold of excitation, and what happens when it is reached? \\
            % \textit{answer:} The threshold of excitation is +15 mV. When it is reached, an action potential is generated.

      \item \textbf{Short Answer:} Why don't the \(\text{Na}^{+}\) channels reopen during repolarization? \\
            % \textit{answer:} Because of the refractory period.

      \item \textbf{Multiple Choice:} During which phase of the action potential is a neuron unable to fire another action potential, no matter how strong the stimulus?
            \begin{tasks}
                  \task Depolarization
                  \task Repolarization
                  \task Absolute refractory period
                  \task Relative refractory period
            \end{tasks}
            % \textit{answer:} C.

      \item \textbf{Multiple Choice:} How can an all-or-none action potential signal convey information about stimulus intensity?
            \begin{tasks}
                  \task By increasing the amplitude of the action potentials
                  \task By decreasing the duration of action potentials
                  \task By changing the direction of action potentials
                  \task By increasing the frequency of action potentials
            \end{tasks}
            % \textit{answer:} D.

      \item \textbf{Fill in the Blank:} The period during which no amount of stimulation can cause another action potential is called the \underline{\hspace{3cm}}. \\
            % \textit{answer:} Absolute refractory period.

      \item \textbf{Fill in the Blank:} A stronger-than-normal stimulus is required to cause another action potential during the \underline{\hspace{3cm}} because of hyperpolarization. \\
            % \textit{answer:} Relative refractory period.
\end{enumerate}



\newpage
\section*{2.3 Additional Questions: Membrane and Action Potentials}
\begin{enumerate}[label=\textbf{Q2.3.\arabic*}]
      \item What is the resting membrane potential (RMP) of a neuron and what does it represent? \\
            % \textit{answer:} -70 mV; it represents the voltage difference between the inside and outside of the cell at rest.
      \item What does the provided diagram (via a hyperlink) illustrate regarding the RMP? \\
            % \textit{answer:} It shows the relative permeability of the cell membrane to different ions.
      \item Name two key components of the cell membrane that help maintain the RMP. \\
            % \textit{answer:} The phospholipid bilayer and embedded proteins.
      \item Describe the phospholipid bilayer. \\
            % \textit{answer:} It consists of hydrophobic tails and hydrophilic heads.
      \item What is meant by semipermeability in the context of the cell membrane? \\
            % \textit{answer:} Only molecules that are lipid, lipid soluble, small, and neutral can cross the membrane.
      \item List the four main functions of embedded proteins in the cell membrane. \\
            % \textit{answer:} They function as receptors, channels (including gated channels), pumps, and enzymes.
      \item What does “differential permeability” mean and how does it affect ion movement? \\
            % \textit{answer:} It means the membrane is more permeable to some ions (e.g., K\(^{+}\)) than others (e.g., Na\(^{+}\)), influencing the net ion flow.
      \item How does diffusion contribute to the RMP? \\
            % \textit{answer:} Ions move from high to low concentration across the membrane.
      \item What role does electrostatic pressure play in neuronal membrane potential? \\
            % \textit{answer:} It drives ions toward the opposing charge, aiding in establishing the RMP.
      \item Define equilibrium potential and state the equilibrium values for K\(^{+}\) and Na\(^{+}\). \\
            % \textit{answer:} It is the voltage at which the ion's driving force in equals the driving force out; typically, K\(^{+}\) = -80 mV and Na\(^{+}\) = +55 mV.
      \item What is the function of the sodium-potassium pump in neurons? \\
            % \textit{answer:} It expends 1 ATP to pump 3 Na\(^{+}\) out and 2 K\(^{+}\) in, maintaining ion gradients.
      \item What is an action potential (AP) and what is its peak voltage? \\
            % \textit{answer:} An AP is the “ON” electrical signal with a peak around +40 mV.
      \item What is the total voltage difference during an action potential and what does it mean? \\
            % \textit{answer:} There is a 110 mV difference between the inside and outside, representing the full reversal of charge.
      \item What happens in a failed attempt at an action potential? \\
            % \textit{answer:} The threshold is not reached; the membrane potential may change slightly but returns to -70 mV.

      \item \textbf{Short Answer:} What is decremental conduction, and why does it occur? \\
            % \textit{answer:} Decremental conduction is the gradual weakening of a signal as it travels due to resistance and leakage.

      \item \textbf{Short Answer:} Why does the action potential require active regeneration in unmyelinated axons? \\
            % \textit{answer:} Because it is regenerated at each point along the axon to prevent signal loss.

      \item \textbf{Multiple Choice:} What is the main advantage of saltatory conduction in myelinated axons?
            \begin{tasks}
                  \task It eliminates the need for Na\(^+\) channels.
                  \task It increases conduction speed and reduces energy expenditure.
                  \task It allows action potentials to travel in both directions.
                  \task It prevents action potentials from occurring at all.
            \end{tasks}
            % \textit{answer:} B.

      \item \textbf{Fill in the Blank:} In saltatory conduction, the action potential is actively regenerated at \underline{\hspace{3cm}}. \\
            % \textit{answer:} Nodes of Ranvier.

      \item \textbf{Short Answer:} Why can't myelin sheaths extend indefinitely without nodes? \\
            % \textit{answer:} Because the decremental conduction would cause the action potential to weaken too much before reaching the end.

      \item \textbf{Multiple Choice:} What happens when an action potential reaches the presynaptic terminal?
            \begin{tasks}
                  \task Voltage-gated Ca\(^{+2}\) channels open.
                  \task The Na\(^+\)/K\(^+\) pump is activated.
                  \task The postsynaptic neuron immediately fires an action potential.
                  \task The synaptic vesicles dissolve.
            \end{tasks}
            % \textit{answer:} A.

      \item \textbf{Fill in the Blank:} Neurotransmitters are released into the \underline{\hspace{3cm}} when synaptic vesicles fuse with the presynaptic membrane. \\
            % \textit{answer:} Synaptic cleft.

      \item \textbf{Short Answer:} What is the function of docking proteins at the presynaptic membrane? \\
            % \textit{answer:} They hold synaptic vesicles in place until an action potential triggers neurotransmitter release.

      \item \textbf{Multiple Choice:} Which of the following is an excitatory postsynaptic potential (EPSP)?
            \begin{tasks}
                  \task Opening of Na\(^+\) channels
                  \task Opening of K\(^+\) channels
                  \task Binding of neurotransmitters to autoreceptors
                  \task Metabolism of neurotransmitters
            \end{tasks}
            % \textit{answer:} A.

      \item \textbf{Fill in the Blank:} The three possible fates of neurotransmitters after release are \underline{\hspace{3cm}}, \underline{\hspace{3cm}}, and \underline{\hspace{3cm}}. \\
            % \textit{answer:} Active reuptake, metabolism, binding to autoreceptors.

      \item \textbf{Short Answer:} How do ionotropic and metabotropic synapses differ in terms of speed and duration? \\
            % \textit{answer:} Ionotropic synapses produce rapid and short effects by directly opening ion channels, while metabotropic synapses have slower, longer-lasting effects through secondary messenger pathways.

      \item \textbf{Multiple Choice:} What role does cAMP play in metabotropic synapses?
            \begin{tasks}
                  \task It binds directly to ion channels to open them.
                  \task It activates Protein Kinase A, which leads to phosphorylation.
                  \task It metabolizes excess neurotransmitters.
                  \task It breaks down ATP into ADP.
            \end{tasks}
            % \textit{answer:} B.

      \item \textbf{Fill in the Blank:} The enzyme \underline{\hspace{3cm}} converts ATP into cAMP in metabotropic signaling. \\
            % \textit{answer:} Adenylate cyclase.

\end{enumerate}
