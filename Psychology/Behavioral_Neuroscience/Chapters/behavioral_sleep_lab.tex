\begin{center}
    \textbf{NEW NOTES FOR 04/10/25} \\
    \hrulefill
\end{center}

\section{What Do We Know About Sleep?}

\begin{coloredlist}
    \item Sleep is an active process.
    \item Sleepiness and alertness are controlled in part by a biological clock.
    \item Things can go wrong
    \begin{coloredlist}
        \item There are whole books that discuss sleep disorders.
    \end{coloredlist}
    \item \cyanit{Suprachiasmatic nucleus (SCN)} is the master clock of the body.
    \begin{coloredlist}
        \item The SCN is located in the hypothalamus and is responsible for regulating circadian rhythms.
        \item Note the name of the SCN, it is \textit{above} the \textit{chiasm} of the optic nerve.
    \end{coloredlist}
    \item Three kinds of rhythms:
    \begin{coloredlist}
        \item \cyanit{Ultradian} rhythms: cycles shorter than 24 hours (e.g., heart rate, respiration).
        \item \cyanit{Circadian} rhythms: cycles of about 24 hours (e.g., sleep-wake cycle, body temperature).
        \begin{coloredlist}
            \item Our circadian rhythm is an endogenous clock that is influenced by exogenous factors.
            \item \cyanit{Free-running} is when the circadian rhythm is not influenced by external cues (e.g., light, temperature).
            \item It is about 24.2 hours in humans.
        \end{coloredlist}
        \item \cyanit{Infradian} rhythms: cycles longer than 24 hours (e.g., menstrual cycle, seasonal changes).
    \end{coloredlist}
    \item \cyanit{Zeitgeber} is a stimulus that helps to regulate the biological clock (e.g., light, temperature).
    \item Human clocks run long when left free running. Rats are short.
\end{coloredlist}

\section{What is Sleep?}

\begin{coloredlist}
    \item For regular people, sleep is behaviorally defined as a state of reduced movement, species specific posture, reduced response to stimuli, and reversibility.
    \item For sleep researchers, they take a more physiological definition of sleep.
    \begin{coloredlist}
        \item \cyanit{Polysomnography (PSG)} is a method of recording various physiological signals during sleep, including:
        \begin{coloredlist}
            \item \textbf{Electroencephalography (EEG):} measures electrical activity in the brain.
            \item \textbf{Electromyogram (EMG):} measures muscle activity.
            \item \textbf{Electrooculogram (EOG):} measures eye movements.
            \item Rechtschaffen and Kales (1968) defined sleep stages based on EEG patterns.
            \item In 2007, the American Academy of Sleep Medicine (AASM) updated the sleep stage criteria.
        \end{coloredlist}
    \end{coloredlist}
\end{coloredlist}

\begin{table}[htbp]
    \centering
    \begin{tabularx}{\linewidth}{l 
                                >{\raggedright\arraybackslash}X 
                                >{\raggedright\arraybackslash}X 
                                >{\raggedright\arraybackslash}X 
                                >{\raggedright\arraybackslash}X}
        \toprule
        \textbf{Name} & \textbf{Frequency} & \textbf{Amplitude} & \textbf{Description} & \textbf{State} \\
        \midrule
        \textbf{Beta \(\beta\)}  & 12 -- 50 Hz (variable) & Lower and Variable & Desynchronous  & Awake and Paying Attention\\[25pt]
        \textbf{Alpha \(\alpha\)} & 8 -- 12 Hz & 50 Microvolts & Synchronous & Relaxed Wakefulness (eye closed, not fully attending, and usually largest occipitally)\\[80pt]
        \textbf{Theta \(\theta\)} & 3.5 -- 7.5 Hz & Low in voltage Microvolts & Synchronous & Drowsy, Light Sleep (Stage 1)\\[25pt]
        \textbf{Delta \(\delta\)} & 1 -- 3.5 Hz  & 20 -- 200 Microvolts & Synchronous & Deep Sleep (Stages 3 and 4)\\
        \bottomrule
    \end{tabularx}
    \caption{Summary of EEG Wave Characteristics}
    \label{tab:EEGwaves}
\end{table}
