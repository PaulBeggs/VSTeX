\begin{center}
    \textbf{NEW NOTES FOR 04/10/25} \\
    \hrulefill
\end{center}

\section{What Do We Know About Sleep?}

\begin{coloredlist}
    \item Sleep is an active process.
    \item Sleepiness and alertness are controlled in part by a biological clock.
    \item Things can go wrong
    \begin{coloredlist}
        \item There are whole books that discuss sleep disorders.
    \end{coloredlist}
    \item \cyanit{Suprachiasmatic nucleus (SCN)} is the master clock of the body.
    \begin{coloredlist}
        \item The SCN is located in the hypothalamus and is responsible for regulating circadian rhythms.
        \item Note the name of the SCN, it is \textit{above} the \textit{chiasm} of the optic nerve.
    \end{coloredlist}
    \item \cyanit{International Classification of Sleep Disorders (ICSD)} -- A system for classifying sleep disorders.
    \item Three kinds of rhythms:
    \begin{coloredlist}
        \item \cyanit{Ultradian} rhythms: cycles shorter than 24 hours (e.g., heart rate, respiration).
        \item \cyanit{Circadian} rhythms: cycles of about 24 hours (e.g., sleep-wake cycle, body temperature).
        \begin{coloredlist}
            \item Our circadian rhythm is an endogenous clock that is influenced by exogenous factors.
            \item \cyanit{Free-running} is when the circadian rhythm is not influenced by external cues (e.g., light, temperature).
            \item It is about 24.2 hours in humans.
        \end{coloredlist}
        \item \cyanit{Infradian} rhythms: cycles longer than 24 hours (e.g., menstrual cycle, seasonal changes).
    \end{coloredlist}
    \item \cyanit{Zeitgeber} is a stimulus that helps to regulate the biological clock (e.g., light, temperature).
    \item Human clocks run long when left free running. Rats are short.
\end{coloredlist}

\section{What is Sleep?}

\begin{coloredlist}
    \item For regular people, sleep is behaviorally defined as a state of reduced movement, species specific posture, reduced response to stimuli, and reversibility.
    \item For sleep researchers, they take a more physiological definition of sleep.
    \begin{coloredlist}
        \item \cyanit{Polysomnography (PSG)} is a method of recording various physiological signals during sleep, including:
        \begin{coloredlist}
            \item \textbf{Electroencephalography (EEG):} measures electrical activity in the brain.
            \item \textbf{Electromyogram (EMG):} measures muscle activity.
            \item \textbf{Electrooculogram (EOG):} measures eye movements.
            \item Rechtschaffen and Kales (1968) defined sleep stages based on EEG patterns.
            \item In 2007, the American Academy of Sleep Medicine (AASM) updated the sleep stage criteria.
        \end{coloredlist}
    \end{coloredlist}
\end{coloredlist}

\begin{table}[htbp]
    \centering
    \begin{tabularx}{\linewidth}{l 
                                >{\raggedright\arraybackslash}X 
                                >{\raggedright\arraybackslash}X 
                                >{\raggedright\arraybackslash}X 
                                >{\raggedright\arraybackslash}X}
        \toprule
        \textbf{Name} & \textbf{Frequency} & \textbf{Amplitude} & \textbf{Description} & \textbf{State} \\
        \midrule
        \textbf{Beta \(\beta\)}  & 12 -- 50 Hz (variable) & Lower and Variable & Desynchronous  & Awake and Paying Attention\\[25pt]
        \textbf{Alpha \(\alpha\)} & 8 -- 12 Hz & 50 Microvolts & Synchronous & Relaxed Wakefulness (eye closed, not fully attending, and usually largest occipitally)\\[80pt]
        \textbf{Theta \(\theta\)} & 3.5 -- 7.5 Hz & Low in voltage Microvolts & Synchronous & Drowsy, Light Sleep (Stage 1)\\[25pt]
        \textbf{Delta \(\delta\)} & 1 -- 3.5 Hz  & 20 -- 200 Microvolts & Synchronous & Deep Sleep (Stages 3 and 4)\\
        \bottomrule
    \end{tabularx}
    \caption{Summary of EEG Wave Characteristics}
    \label{tab:EEGwaves}
\end{table}

\section{Two States of Consciousness}

\subsection{Being Awake}

\begin{coloredlist}
    \item Physiological definition of wakefulness:
    \begin{coloredlist}
        \item Supposed to be awake for \(\frac{2}{3}^{\text{rd}}\) of the day.
        \item \(<\) 5\% of the day is spent in REM sleep.
        \item Predominantly alpha and beta waves in the brain when drowsy.
        \item Muscle activity -- high muscle tone when awake.
        \begin{coloredlist}
            \item Lose muscle tone when you are sleeping.
            \begin{coloredlist}
                \item EMG is high when awake
            \end{coloredlist}
        \end{coloredlist}
        \item Variability in the eye movement
    \end{coloredlist}
\end{coloredlist}

\subsection{Being Asleep}

\begin{coloredlist}
    \item Delta waves and theta waves predominantly.
    \item EMG is low
    \item Slow rolling eye movements everytime you fall asleep.
    \item \cyanit{Hypnic Jerk} -- a sudden muscle contraction that occurs when falling asleep.
    \item \cyanit{NREM Sleep} -- non-rapid eye movement sleep, which is divided into three stages:
    \begin{coloredlist}
        \item \cyanit{N1} -- light sleep, theta waves, low EMG, and slow rolling eye movements (5-10\% of the night).
        \begin{coloredlist}
            \item Transition between wakefulness and sleep.
            \item Hypnic jerks can occur in this stage.
            \item Muscle tone is reduced, but not completely lost.
        \end{coloredlist}
        \item \cyanit{N2} -- light sleep, theta waves, sleep spindles, K-complexes, low EMG, and slow rolling eye movements (45-55\% of the night).
        \begin{coloredlist}
            \item Sleep spindles and K-complexes are characteristic of this stage.
            \item Muscle tone is further reduced compared to N1.
        \end{coloredlist}
        \item \cyanit{N3} -- deep sleep, delta waves (3-8\%), low EMG, and slow rolling eye movements (10-15\% of the night).
        \begin{coloredlist}
            \item Characterized by high-amplitude delta waves.
            \item Muscle tone is at its lowest in this stage.
        \end{coloredlist}
    \end{coloredlist}
    \item \cyanit{REM Sleep} -- rapid eye movement sleep.
    \begin{coloredlist}
        \item Aserinsky (1952): Discovered 70\% in infants, 20-25\% for healthy adults
        \item Characteristics:
        \begin{coloredlist}
            \item EEG -- Low voltage, random, fast with sawtooth waves.
            \item Fast activity, low amplitude, and desynchronous.
            \item EMG -- atonia (loss of muscle tone).
            \item Paradoxical sleep: brain is active, but body is paralyzed.
            \item Intercostal muscles are paralyzed, but not the diaphragm (obviously).
            \item EOG -- Bursts of rapid eye movements
        \end{coloredlist}
    \end{coloredlist}
\end{coloredlist}
