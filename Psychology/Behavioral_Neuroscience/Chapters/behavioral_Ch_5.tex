Behavioral neuroscience research involves the efforts of science in many disciples, including physiology, neuroanatomy, biochemistry, psychology, endocrinology, and histology. An enormous array of research methods is available to researchers in behavioral neuroscience. The goal of this chapter is to provide an overview of the most common methods used in the field. \\

This chapter will mainly focus on the following research methods:
\begin{coloredlist}
    \item Experimental Ablation
    \item Recording and Stimulating Neural Activity
    \item Neurochemical Methods
    \item Genetic Methods
\end{coloredlist}
Each research method has a multitude of techniques that we will explore in detail for each section.

\section{Experimental Ablation}

\subsection{Terms}
\textsc{EVALUATING THE BEHAVIORAL EFFECTS OF BRAIN DAMAGE}
\begin{coloredlist}
    \item \cyanit{Experimental Ablation} -- Destroying a part of the brain and evaluating an animal's subsequent behavior. (Synonymous with \cyanit{lesion study})
    \item \textit{Lesion} -- The damaged tissue.
\end{coloredlist}
\textsc{PRODUCING BRAIN LESIONS}
\begin{coloredlist}
    \item \textit{Kainic Acid} -- An excitatory amino acid that kills neurons by stimulating them to death.
    \item \textit{Cannula} -- A small metal tube.
    \item \cyanit{Excitotoxic Lesions} (\textit{ek sigh tow tok sik}) -- A brain lesion produced by intracerbral injection of an excitatory amino acid, such as kainic acid.
    \item \cyanit{Sham Lesions} -- A placebo procedure that duplicates all the steps of producing a brain lesion except the one that actually causes the brain damage.
\end{coloredlist}
\textsc{STEREOTAXIC SURGERY}
\begin{coloredlist}
    \item \cyanit{Sterotaxic Surgery} (\textit{stair ee oh tak sik}) -- Brain surgery using a stereotaxic apparatus to position an electrode or cannula in a specified position of the brain.
    \item \cyanit{Steotaxic Atlas} -- A collection of drawings of sections of the brain of a particular animal with measurements that provide coordinates for stereotaxic surgery.
    \item \cyanit{Bregma} -- The junction of the sagittal and coronal structures of the skull; often used as a reference point for stereotaxic brain surgery.
    \item \cyanit{Stereotaxic Apparatus} -- A device that permits a surgeon to position an electrode or cannula into a specific part of the brain.
    \item \cyanit{Deep Brain Stimulation} -- A technique using sterotaxic surgery to implant a permanent electrode in the brain; used to treat chronic pain, movement disorders, epilepsy, depression, and OCD.
\end{coloredlist}
\textsc{HISTOLOGICAL METHODS}
\begin{coloredlist}
    \item \textit{Histological Methods} -- Methods of preparing and examining brain tissue to determine the effects of behavior, injury, or disease.
    \item \cyanit{Formalin} (\textit{for mal lin}) -- The aqueous solution of formaldehyde gas; the most commonly used tissue fixative. 
    \item \cyanit{Fixative} -- A chemical such as formalin; used to prepare and preserve body tissue.
    \item \cyanit{Microtome} -- An instrument that produces very thin slices of body tissue.
    \item \cyanit{Cryostat} -- An instrument used to prepare very thin slices of body tissue inside a freezer chamber.
    \item \cyanit{Immunocytochemical Method} -- A histological method that uses radioactive antibodies or antibodies bound with a dye molecule to indicate the presence of particular proteins of peptides.
    \item \cyanit{Transmission Electron Microscope} -- A microscope that passes a focused beam of electrons through thin slices of tissue to reveal minuscule details.
    \item \cyanit{Scanning Electron Microscope} -- A microscope that provides three-dimensional information about the shape of the surface of a small object by scanning the object with a thin beam of electrons.
    \item \cyanit{Confocal Laser Scanning Microscope} -- A microscope that provides high-resolution images of various depths of thick tissue that contains fluorescent molecules by scanning the tissue with light from a laser beam.
\end{coloredlist}
\textsc{TRACING NEURAL CONNECTIONS} 
\begin{coloredlist}
    \item \cyanit{Anterograde Labeling Method} -- A histological method that labels the axons and terminal buttons of neurons whose cell bodies are located in a particular region.
    \item \cyanit{Retrograde Labeling Method} -- A histological method that labels cell bodies that give rise to the terminal buttons that form synapses with cells in a particular region.
\end{coloredlist}
\textsc{STUDYING THE STRUCTURE OF THE LIVING HUMAN BRAIN}
\begin{coloredlist}
    \item \cyanit{Computerized Tomography (CT)} -- The use of a device that employs a computer to analyze data obtained by a scanning beam of X-rays to produce a two-dimensional picture of a ``slice'' through the body. 
    \item \cyanit{Magnetic Resonance Imaging (MRI)} -- A technique whereby the interior of the body can be accurately imaged; involves the interaction between radio waves and a strong magnetic field.
    \item \cyanit{Diffusion Tensor Imaging (DTI)} -- An imaging method that uses a modified MRI scanner to reveal bundles of myelinated axons in the living human brain.
\end{coloredlist}

\subsection{Evaluating the Behavioral Effects of Brain Damage}

An example of experimental ablation (or lesion study) would be if, after part of the brain is destroyed, an animal can no longer perform tasks that require vision, we can conclude that the damaged area plays some role in vision. (See \hyperlink{person:Muller}{Johannes M\"uller} and the doctrine of specific nerve energies for relevant information.)

What makes lesion studies so important, is that, we can distinguish between brain function and behavior. For example, reading involves functioned required for controlling eye movements, focusing the lens of the eye, perceiving and recognizing words and letters, comprehending the meaning of words, and so on. Some of these functions also participate in other behaviors; for example, controlling eye movement and focusing are required for any task that involves looking, and brain mechanisms used for comprehending the meanings of words also participate in comprehending speech.