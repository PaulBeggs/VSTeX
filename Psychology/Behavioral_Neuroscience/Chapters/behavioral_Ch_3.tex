\section{Neuroanatomy}

\subsection{Nervous System Structure}

\subsubsection{Structural Nervous System}

\cyanit{How are neurons organized into systems? }

\begin{coloredlist}
    \item \textbf{Central Nervous System (CNS)}
    \begin{coloredlist}
        \item Brain
        \item Spinal Cord
    \end{coloredlist}
    \item \textbf{Peripheral Nervous System (PNS)}

\end{coloredlist}

\subsubsection{Functional Nervous System}

\cyanit{What are the `jobs' of the nervous system?}

\begin{coloredlist}
    \item Somatic Nervous System
    \begin{coloredlist}
        \item Skeletal Muscles (Striated)
        \item Sensory information in
        \item Voluntary motion out
    \end{coloredlist}
    \item Autonomic Nervous System
    \begin{coloredlist}
        \item Uses smooth muscles
        \item Glands
        \item Sympathetic Nervous System
        \begin{coloredlist}
            \item Fight or Flight
            \item Heart rate, blood pressure, respiration, and alertness.
        \end{coloredlist}
        \item Parasympathetic Nervous System
        \begin{coloredlist}
            \item Rest and Digest
        \end{coloredlist}
        \item Enteric Nervous System
        \begin{coloredlist}
            \item A mesh-like system of neurons that governs the function of the gastrointestinal system.
            \item AKA: `Second Brain'
            \item GI problems are correlated with psychological disorders.
            \item The GI track houses a lot of our microbiota.
            \item Fecal Microbiota Transplant
            \begin{coloredlist}
                \item Rat studies showed that when a skinny rat has a fecal transplant from a fat rat, the skinny rat becomes fat. This works in reverse too.
                \item Therefore, the microbiota change the \textit{behavior} of the rat.
            \end{coloredlist}
            \item Elevated Plus Maze
            \begin{coloredlist}
                \item A test to measure anxiety in rats.
                \item The rats with the fecal transplant from the anxious rats were more anxious. 
                \item \textbf{This is huge!} This shows that the microbiota can change if a rat is anxious or not!
            \end{coloredlist}
        \end{coloredlist}
    \end{coloredlist}
\end{coloredlist}

Starting a new list because I don't know where to put this. But we are starting with Meninges.

\section{Meninges}

\begin{coloredlist}
    \item Cover the outside of the nervous system.
    \begin{coloredlist}
        \item Three for the CNS and two for the PNS.
        \item The PNS does not use the arachnoid mater.
    \end{coloredlist}
    \item \textbf{Dura Mater}
    \begin{coloredlist}
        \item ``Hard Mother''
        \item The outermost layer.
        \item Tough and fibrous.
        \item Contains blood vessels.
        \item Early anatomists called it ``pachymeninges'' because similar to elephant skin.
    \end{coloredlist}
    \item \textbf{Arachnoid Mater}
    \begin{coloredlist}
        \item ``Spider Mother''
        \item Middle layer.
        \item Web-like structure.
        \item Contains blood vessels.
        \item Subarachnoid Space
        \begin{coloredlist}
            \item Between the arachnoid and Pia mater.
            \item Contains cerebrospinal fluid (CSF).
        \end{coloredlist}
        \item Arachnoid trabeculae
        \begin{coloredlist}
            \item Web-like structures that connect the arachnoid mater to the Pia mater.
        \end{coloredlist}
    \end{coloredlist}
    \item \textbf{Pia Mater}
    \begin{coloredlist}
        \item ``Soft Mother''
        \item Innermost layer.
        \item Thin and delicate.
        \item Fllows overy every sulci (grooves), fissure (deep indentations), and gyri (bumps).
        \item Follows the contours of the brain and spinal cord.
    \end{coloredlist}
    \item \textbf{Meningitis}
    \begin{coloredlist}
        \item Inflammation of the meninges.
        \item Can cause symptoms such as headache, fever, a stiff neck, or hallucinations.
    \end{coloredlist}
\end{coloredlist}

\section{Cerebrospinal Fluid (CSF)}

\begin{coloredlist}
    \item Similar to blood plasma.
    \item Functions of CSF
    \begin{coloredlist}
        \item Protection
        \begin{coloredlist}
            \item Failures:
            \begin{coloredlist}
                \item Brain is injured.
                \item AND even Contrecoup--when the brain is injured on the opposite side of the impact--injuries.
                \item Chronic Traumatic Encephalopathy (CTE)
                \begin{coloredlist}
                    \item Old name: Dementia Pugilistica (boxer's dementia).
                    \item Symptoms (not exhaustive): Memory loss, confusion, impaired judgment, impulse control problems, aggression, depression, Parkinson's-like symptoms, insomnia, and progressive dementia.
                    \item Causes ventricular enlargement. In other words, the larger your ventricles, the less brain matter you have.
                    \item Also causes atrophy of the fornix. The fornix is a C-shaped bundle of nerve fibers in the brain that acts as the major output tract of the hippocampus.
                    \item Tau are abnormally phosphorylated aggregate into tangles. They accumulate both inside neurons and even released into extracellular space.
                \end{coloredlist}
            \end{coloredlist}
        \end{coloredlist}
    \end{coloredlist}
\end{coloredlist}

