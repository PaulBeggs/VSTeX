\section{Neuroanatomy}

Neuroscience is the study of the nervous system. Behavioral neuroscience is understanding the nervous system's underlying behavior.

\subsection{Nervous System Structure}

\subsubsection{Structural Nervous System}

\cyanit{How are neurons organized into systems?}

\begin{coloredlist}
    \item \textbf{Central Nervous System (CNS)}
    \begin{coloredlist}
        \item Brain
        \item Spinal Cord
    \end{coloredlist}
    \item \textbf{Peripheral Nervous System (PNS)}

\end{coloredlist}

\subsubsection{Functional Nervous System}

\cyanit{What are the `jobs' of the nervous system?}

\begin{coloredlist}
    \item Somatic Nervous System
    \begin{coloredlist}
        \item Skeletal Muscles (Striated)
        \item Sensory information in
        \item Voluntary motion out
    \end{coloredlist}
    \item Autonomic Nervous System
    \begin{coloredlist}
        \item Uses smooth muscles
        \item Glands
        \item Sympathetic Nervous System
        \begin{coloredlist}
            \item Fight or Flight
            \item Heart rate, blood pressure, respiration, and alertness.
        \end{coloredlist}
        \item Parasympathetic Nervous System
        \begin{coloredlist}
            \item Rest and Digest
        \end{coloredlist}
        \item Enteric Nervous System
        \begin{coloredlist}
            \item A mesh-like system of neurons that governs the function of the gastrointestinal system.
            \item AKA: `Second Brain'
            \item GI problems are correlated with psychological disorders.
            \item The GI track houses a lot of our microbiota.
            \item Fecal Microbiota Transplant
            \begin{coloredlist}
                \item Rat studies showed that when a skinny rat has a fecal transplant from a fat rat, the skinny rat becomes fat. This works in reverse too.
                \item Therefore, the microbiota change the \textit{behavior} of the rat.
            \end{coloredlist}
            \item Elevated Plus Maze
            \begin{coloredlist}
                \item A test to measure anxiety in rats.
                \item The rats with the fecal transplant from the anxious rats were more anxious.
                \item \textbf{This is huge!} This shows that the microbiota can change if a rat is anxious or not!
            \end{coloredlist}
        \end{coloredlist}
    \end{coloredlist}
\end{coloredlist}

Starting a new list because I don't know where to put this. But we are starting with Meninges.

\section{Meninges}

\begin{coloredlist}
    \item Cover the outside of the nervous system.
    \begin{coloredlist}
        \item Three for the CNS and two for the PNS.
        \item The PNS does not use the arachnoid mater.
    \end{coloredlist}
    \item \textbf{Dura Mater}
    \begin{coloredlist}
        \item ``Hard Mother''
        \item The outermost layer.
        \item Tough and fibrous.
        \item Contains blood vessels.
        \item Early anatomists called it ``pachymeninges'' because similar to elephant skin.
    \end{coloredlist}
    \item \textbf{Arachnoid Mater}
    \begin{coloredlist}
        \item ``Spider Mother''
        \item Middle layer.
        \item Web-like structure.
        \item Contains blood vessels.
        \item Subarachnoid Space
        \begin{coloredlist}
            \item Between the arachnoid and Pia mater.
            \item Contains cerebrospinal fluid (CSF).
        \end{coloredlist}
        \item Arachnoid trabeculae
        \begin{coloredlist}
            \item Web-like structures that connect the arachnoid mater to the Pia mater.
            \item Allows for the subarachnoid space to be filled with CSF.
        \end{coloredlist}
    \end{coloredlist}
    \item \textbf{Pia Mater}
    \begin{coloredlist}
        \item ``Soft Mother''
        \item Innermost layer.
        \item Thin and delicate.
        \item Flows over every sulcus (grooves), fissure (deep indentations), and gyri (bumps).
        \item Follows the contours of the brain and spinal cord.
    \end{coloredlist}
    \item \textbf{Meningitis}
    \begin{coloredlist}
        \item Inflammation of the meninges.
        \item Can cause symptoms such as headache, fever, a stiff neck, or hallucinations.
    \end{coloredlist}
\end{coloredlist}

\section{Cerebrospinal Fluid (CSF)}

\begin{coloredlist}
    \item Similar to blood plasma.
    \item Functions of CSF
    \begin{coloredlist}
        \item Protection
        \begin{coloredlist}
            \item Failures:
            \begin{coloredlist}
                \item Brain is injured.
                \item AND even Contrecoup--when the brain is injured on the opposite side of the impact--injuries.
                \item Chronic Traumatic Encephalopathy (CTE)
                \begin{coloredlist}
                    \item Old name: Dementia Pugilistica (boxer's dementia).
                    \item Symptoms (not exhaustive): Memory loss, confusion, impaired judgment, impulse control problems, aggression, depression, Parkinson's-like symptoms, insomnia, and progressive dementia.
                    \item Causes ventricular enlargement. In other words, the larger your ventricles, the less brain matter you have.
                    \item Also causes atrophy of the fornix. The fornix is a C-shaped bundle of nerve fibers in the brain that acts as the major output tract of the hippocampus.
                    \item Tau are abnormally phosphorylated aggregate into tangles. They accumulate both inside neurons and even released into extracellular space.
                \end{coloredlist}
            \end{coloredlist}
        \end{coloredlist}
        \item The CSF also moves neurotransmitters, waste, hormones, nutrients, and other substances from one place to another.
        \begin{coloredlist}
            \item For example, the CSF moves \(\beta\)-amyloid (in-between cells) from the brain to the blood.
        \end{coloredlist}
    \end{coloredlist}
    \item \textbf{Choroid Plexus}
    \begin{coloredlist}
        \item \textbf{Ependymal cells}
        \begin{coloredlist}
            \item Lines the ventricles.
            \item These are the cells that produce the CSF.
        \end{coloredlist}
        \item If the choroid plexus is not working properly, it can cause hydrocephalus.
        \item Derives from the Pia mater.
        \item \textbf{Hydrocephalus}
        \begin{coloredlist}
            \item ``Water on the brain''
            \item Swelling of the brain due to the accumulation of CSF.
        \end{coloredlist}

    \end{coloredlist}
\end{coloredlist}

\subsection{Flow of CSF}

\begin{coloredlist}
    \item \textbf{Lateral ventricles}
    \begin{coloredlist}
        \item
    \end{coloredlist}
    \item \textbf{Third Ventricle}
    \begin{coloredlist}
        \item Looks like a duck's head.
        \item Is connected to the pituitary gland through the infundibulum.
    \end{coloredlist}
    \item The CSF routed through the medial longitudinal fissure and into the Superior Sagittal Sinus.
    \item \textbf{Interperducular Fossa}
    \begin{coloredlist}
        \item The space between the two cerebral peduncles.
    \end{coloredlist}
    \item \textbf{Interventricular Foramen}
    \begin{coloredlist}
        \item Connects the lateral ventricles to the third ventricle.
    \end{coloredlist}
    \item \textbf{Cerebral Aqueduct}
    \begin{coloredlist}
        \item Connects the third and fourth ventricles.
    \end{coloredlist}
    \item \textbf{Central Canal}
    \begin{coloredlist}
        \item Connects the fourth ventricle to the spinal cord.
        \item For remembering purposes, the \cyanit{cerebral} aqueduct is in the \cyanit{brain} and the central canal is in the spinal cord.
    \end{coloredlist}
    \item \textbf{Subarachnoid Space}
    \begin{coloredlist}
        \item Foramen of Magendie (Medial) and Luschka (Lateral)
        \begin{coloredlist}
            \item Two tiny little holes in the fourth ventricle.
        \end{coloredlist}
    \end{coloredlist}
\end{coloredlist}

\subsection{Dumping of CSF}

\begin{coloredlist}
    \item Arachnoid Villi/Granulations
    \begin{coloredlist}
        \item Absorbed into blood stream from the superior sagittal sinus.
    \end{coloredlist}
\end{coloredlist}

\subsection{Getting Some CSF Out -or- Putting Something Into It}

\begin{coloredlist}
    \item Where would you have them stick that needle?
    \begin{coloredlist}
        \item \textbf{Dural Sac}
        \begin{coloredlist}
            \item Enlarged space in the lumbar region.
            \item Testing and introduction of anesthetic agents.
            \item Epi = Something in
        \end{coloredlist}
        \item \textbf{Lumbar Puncture}
        \begin{coloredlist}
            \item AKA: Spinal Tap.
            \item Tap = Taking something out
        \end{coloredlist}
    \end{coloredlist}
\end{coloredlist}

\section{Cranial Nerves}

\begin{table}[htbp]
    \centering
    \begin{tabular}{p{0.7cm}p{3.3cm}cp{9cm}}
        \toprule
        \multicolumn{1}{c}{\textbf{\#}} & \multicolumn{1}{c}{\textbf{Name}}     & \multicolumn{1}{c}{\textbf{Type}} & \multicolumn{1}{c}{\textbf{Information Carried}}                                                                           \\
        \midrule
        I                               & Olfactory                             & S                                 & Smell                                                                                                                      \\
        II                              & Optic                                 & S                                 & Vision                                                                                                                     \\
        III                             & Oculomotor                            & M                                 & Eye movement, pupil constriction                                                                                           \\
        IV                              & Trochlear                             & M                                 & Eye movement                                                                                                               \\
        V                               & Trigeminal                            & B                                 & Touch to face, motor control of mandibles                                                                                  \\
        VI                              & Abducens                              & M                                 & Eye movement                                                                                                               \\
        VII                             & Facial                                & B                                 & Taste and facial expression                                                                                                \\
        VIII                            & Auditory \newline (Vestibulocochlear) & S                                 & Hearing                                                                                                                    \\
        IX                              & Glossopharyngeal                      & B                                 & Taste and swallowing                                                                                                       \\
        X                               & Vagus                                 & B                                 & Taste and sensation from neck, thorax, abdomen, swallowing, control of larynx, parasympathetic nerves to heart and viscera \\
        XI                              & Spinal Accessory                      & M                                 & Movement of shoulders                                                                                                      \\
        XII                             & Hypoglossal                           & M                                 & Movement of tongue                                                                                                         \\
        \bottomrule
    \end{tabular}
\end{table}

\subsection{Mnemonic for Cranial Nerves}

\textbf{Ol}d \textbf{Op}ie \textbf{oc}casionally \textbf{tr}ies \textbf{tr}igonometry \textbf{a}nd \textbf{f}eels \textbf{ve}ry \textbf{glo}omy, \textbf{vagu}e, \textbf{a}nd \textbf{hypo}active. 

\section{Terms}

\begin{coloredlist}
    \item Santiago Ramon y Cajal (1911)
    \begin{coloredlist}
        \item Used the Golgi stain to show that neurons are separate cells.
    \end{coloredlist}
    \item \cyanit{Soma} -- Cell Body
    \item \cyanit{Nucleus} -- Contains DNA
    \item \cyanit{Dendrites} -- ``Branches''
    \begin{coloredlist}
        \item Purpose is to increase the surface area of the neuron, so it can receive the most amount of information.
    \end{coloredlist}
    \item \cyanit{Axon terminal botton} -- The ends of the neuron that send information.
    \item \cyanit{Glial cells} -- Support cells by insulating the axon for better communication.
    \item \cyanit{Myelin sheath} -- Insulates the axon.
    \item \cyanit{Unmylinated axons} are called grey matter.
    \item \cyanit{Ganglion} -- A collection of cell bodies in the PNS.
    \item \cyanit{Nerve} -- A collection of axons in the PNS.
    \item \cyanit{Nucleus} -- A collection of cell bodies in the CNS.
    \item \cyanit{Tract} -- A collection of axons in the CNS.
\end{coloredlist}

\begin{table}[htbp]
    \centering
    \begin{tabular}{cc}
        \toprule
        \textbf{Grey Matter} & \textbf{White Matter} \\ \midrule
        Cell bodies          & Myelinated axons      \\
        Dendrites            &                       \\
        Unmyelinated axons   &                       \\
        \bottomrule
    \end{tabular}
    \caption{Gray vs White Matter}\label{tab:}
\end{table}

\begin{table}[htbp]
    \centering
    \begin{minipage}[t]{0.45\textwidth}
        \centering
        \begin{tabular}{cc}
            \toprule
            \textbf{CNS} & \textbf{PNS} \\ \midrule
            Nucleus      & Ganglion     \\
            \bottomrule
        \end{tabular}
        \caption{Gray Matter Division}
        \label{tab:gray}
    \end{minipage}%
    \hfill
    \begin{minipage}[t]{0.45\textwidth}
        \centering
        \begin{tabular}{cc}
            \toprule
            \textbf{CNS} & \textbf{PNS} \\ \midrule
            Tract        & Nerve        \\
            \bottomrule
        \end{tabular}
        \caption{White Matter}
        \label{tab:white}
    \end{minipage}
\end{table}

\subsection{Hindbrain}

\begin{coloredlist}
    \item \cyanit{Myenlencephalon}
    \begin{coloredlist}
        \item \cyanit{Medulla Oblongata}
        \begin{coloredlist}
            \item Enlargement of the cord.
            \item Lots of gray matter.
            \item \cyanit{Reticular Formation}
            \begin{coloredlist}
                \item A network of nuclei.
                \item Regulates sleep, wakefulness, and arousal.
                \item Also regulates heart rate, blood pressure, respiration, and skeletal muscle tone.
            \end{coloredlist}
            \item \cyanit{Pyramids}
            \begin{coloredlist}
                \item Two ridges on the ventral surface.
                \begin{coloredlist}
                    \item Voluntary motor system.
                \end{coloredlist}
            \end{coloredlist}
            \item \cyanit{Olives}
            \begin{coloredlist}
                \item Audition and motor learning.
                \item Located on the lateral surface.
            \end{coloredlist}
        \end{coloredlist}
    \end{coloredlist}
    \item \cyanit{Metencephalon}
    \begin{coloredlist}
        \item \cyanit{Pons} -- ``Bridge''
        \begin{coloredlist}
            \item White matter on the outside and gray on the inside.
            \item \cyanit{Locus Coeruleus}
            \begin{coloredlist}
                \item Produces norepinephrine.
                \item The norepinephrine is sent to the forebrain.
            \end{coloredlist}
        \end{coloredlist}
    \end{coloredlist}
\end{coloredlist}

\section{Midbrain}