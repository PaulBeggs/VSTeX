\section{Neuroanatomy}

Neuroscience is the study of the nervous system. Behavioral neuroscience is understanding the nervous system's underlying behavior.

\subsection{Nervous System Structure}

\subsubsection{Structural Nervous System}

\textit{How are neurons organized into systems?}

\begin{coloredlist}
    \item \textbf{Central Nervous System (CNS)}
    \begin{coloredlist}
        \item Brain
        \item Spinal Cord
    \end{coloredlist}
    \item \textbf{Peripheral Nervous System (PNS)}

\end{coloredlist}

\subsubsection{Functional Nervous System}

\textit{What are the `jobs' of the nervous system?}

\begin{coloredlist}
    \item \cyanit{Somatic Nervous System}
    \begin{coloredlist}
        \item Skeletal Muscles (Striated)
        \item Sensory information in
        \item Voluntary motion out
    \end{coloredlist}
    \item \cyanit{Autonomic Nervous System}
    \begin{coloredlist}
        \item Uses smooth muscles
        \item Glands
        \item Sympathetic Nervous System
        \begin{coloredlist}
            \item Fight or Flight
            \item Heart rate, blood pressure, respiration, and alertness.
        \end{coloredlist}
        \item Parasympathetic Nervous System
        \begin{coloredlist}
            \item Rest and Digest
        \end{coloredlist}
        \item \cyanit{Enteric Nervous System}
        \begin{coloredlist}
            \item A mesh-like system of neurons that governs the function of the gastrointestinal system.
            \item AKA: `Second Brain'
            \item GI problems are correlated with psychological disorders.
            \item The GI track houses a lot of our microbiota.
            \item Fecal Microbiota Transplant
            \begin{coloredlist}
                \item Rat studies showed that when a skinny rat has a fecal transplant from a fat rat, the skinny rat becomes fat. This works in reverse too.
                \item Therefore, the microbiota change the \textit{behavior} of the rat.
            \end{coloredlist}
            \item Elevated Plus Maze
            \begin{coloredlist}
                \item A test to measure anxiety in rats.
                \item The rats with the fecal transplant from the anxious rats were more anxious.
                \item \textbf{This is huge!} This shows that the microbiota can change if a rat is anxious or not!
            \end{coloredlist}
        \end{coloredlist}
    \end{coloredlist}
\end{coloredlist}

\section{Meninges}

\begin{coloredlist}
    \item Cover the outside of the nervous system.
    \begin{coloredlist}
        \item Three for the CNS and two for the PNS.
        \item The PNS does not use the arachnoid mater.
    \end{coloredlist}
    \item \cyanit{Dura Mater}
    \begin{coloredlist}
        \item ``Hard Mother''
        \item The outermost layer.
        \item Tough and fibrous.
        \item Contains blood vessels.
        \item Early anatomists called it ``pachymeninges'' because similar to elephant skin.
    \end{coloredlist}
    \item \cyanit{Arachnoid Mater} = ``Spider Mother''
    \begin{coloredlist}
        \item Middle layer.
        \item Web-like structure.
        \item Contains blood vessels.
        \item Subarachnoid Space
        \begin{coloredlist}
            \item Between the arachnoid and Pia mater.
            \item Contains cerebrospinal fluid (CSF).
        \end{coloredlist}
        \item Arachnoid trabeculae
        \begin{coloredlist}
            \item Web-like structures that connect the arachnoid mater to the Pia mater.
            \item Allows for the subarachnoid space to be filled with CSF.
        \end{coloredlist}
    \end{coloredlist}
    \item \cyanit{Pia Mater} = ``Soft Mother''
    \begin{coloredlist}
        \item Innermost layer.
        \item Thin and delicate.
        \item Flows over every sulcus (grooves), fissure (deep indentations), and gyri (bumps).
        \item Follows the contours of the brain and spinal cord.
    \end{coloredlist}
    \item \cyanit{Meningitis}
    \begin{coloredlist}
        \item Inflammation of the meninges.
        \item Can cause symptoms such as headache, fever, a stiff neck, or hallucinations.
    \end{coloredlist}
\end{coloredlist}

\section{Cerebrospinal Fluid (CSF)}

\begin{coloredlist}
    \item Similar to blood plasma.
    \item Functions of CSF
    \begin{coloredlist}
        \item Protection
        \begin{coloredlist}
            \item Failures:
            \begin{coloredlist}
                \item Brain is injured.
                \item AND even Contrecoup--when the brain is injured on the opposite side of the impact--injuries.
                \item \cyanit{Chronic Traumatic Encephalopathy (CTE)}
                \begin{coloredlist}
                    \item Old name: Dementia Pugilistica (boxer's dementia).
                    \item Symptoms (not exhaustive): Memory loss, confusion, impaired judgment, impulse control problems, aggression, depression, Parkinson's-like symptoms, insomnia, and progressive dementia.
                    \item Causes ventricular enlargement. In other words, the larger your ventricles, the less brain matter you have.
                    \item Also causes atrophy of the fornix. The fornix is a C-shaped bundle of nerve fibers in the brain that acts as the major output tract of the hippocampus.
                    \item Tau are abnormally phosphorylated aggregate into tangles. They accumulate both inside neurons and even released into extracellular space.
                \end{coloredlist}
            \end{coloredlist}
        \end{coloredlist}
        \item The CSF also moves neurotransmitters, waste, hormones, nutrients, and other substances from one place to another.
        \begin{coloredlist}
            \item For example, the CSF moves \(\beta\)-amyloid (in-between cells) from the brain to the blood.
        \end{coloredlist}
    \end{coloredlist}
    \item \cyanit{Choroid Plexus}
    \begin{coloredlist}
        \item \cyanit{Ependymal cells}
        \begin{coloredlist}
            \item Lines the lateral ventricles.
            \item These are the cells that produce the CSF.
        \end{coloredlist}
        \item If the choroid plexus is not working properly, it can cause hydrocephalus.
        \item \cyanit{Hydrocephalus}
        \begin{coloredlist}
            \item ``Water on the brain''
            \item Swelling of the brain due to the accumulation of CSF.
        \end{coloredlist}
        \item Derives from the Pia mater.


    \end{coloredlist}
\end{coloredlist}

\subsection{Flow of CSF}

\begin{coloredlist}
    \item \cyanit{Lateral ventricles}
    \begin{coloredlist}
        \item CSF is produced here and flows through the interventricular foramen.
    \end{coloredlist}
    \item \cyanit{Third Ventricle}
    \begin{coloredlist}
        \item Looks like a duck's head.
        \item Is connected to the \cyanit{pituitary gland} through the \cyanit{infundibulum}.
    \end{coloredlist}
    \item The CSF routed through the medial longitudinal fissure and into the \cyanit{Superior Sagittal Sinus}.
    \item \cyanit{Interperducular Fossa}
    \begin{coloredlist}
        \item The space between the two cerebral peduncles.
    \end{coloredlist}
    \item \cyanit{Interventricular Foramen}
    \begin{coloredlist}
        \item Connects the lateral ventricles to the third ventricle.
    \end{coloredlist}
    \item \cyanit{Cerebral Aqueduct}
    \begin{coloredlist}
        \item Connects the third and fourth ventricles.
    \end{coloredlist}
    \item \cyanit{Central Canal}
    \begin{coloredlist}
        \item Connects the fourth ventricle to the spinal cord.
        \item For remembering purposes, the \textit{cerebral} aqueduct is in the \textit{brain} and the central canal is in the spinal cord.
    \end{coloredlist}
    \item \cyanit{Subarachnoid Space}
    \begin{coloredlist}
        \item Foramen of Magendie (Medial) and Luschka (Lateral)
        \begin{coloredlist}
            \item Two tiny little holes in the fourth ventricle.
        \end{coloredlist}
    \end{coloredlist}
\end{coloredlist}

\subsection{Dumping of CSF}

\begin{coloredlist}
    \item Arachnoid Villi/Granulations
    \begin{coloredlist}
        \item Absorbed into blood stream from the superior sagittal sinus.
    \end{coloredlist}
\end{coloredlist}

\subsection{Getting Some CSF Out -or- Putting Something Into It}

\begin{coloredlist}
    \item Where would you have them stick that needle?
    \begin{coloredlist}
        \item \cyanit{Dural Sac}
        \begin{coloredlist}
            \item Enlarged space in the lumbar region.
            \item Testing and introduction of anesthetic agents.
            \item Epi = Something in
        \end{coloredlist}
        \item \cyanit{Lumbar Puncture}
        \begin{coloredlist}
            \item AKA: Spinal Tap.
            \item Tap = Taking something out
        \end{coloredlist}
    \end{coloredlist}
\end{coloredlist}

\section{Cranial Nerves}

\begin{table}[htbp]
    \centering
    \begin{tabular}{p{0.7cm}p{3.3cm}cp{9cm}}
        \toprule
        \multicolumn{1}{c}{\textbf{\#}} & \multicolumn{1}{c}{\textbf{Name}}     & \multicolumn{1}{c}{\textbf{Type}} & \multicolumn{1}{c}{\textbf{Information Carried}}                                                                           \\
        \midrule
        I                               & Olfactory                             & S                                 & Smell                                                                                                                      \\
        II                              & Optic                                 & S                                 & Vision                                                                                                                     \\
        III                             & Oculomotor                            & M                                 & Eye movement, pupil constriction                                                                                           \\
        IV                              & Trochlear                             & M                                 & Eye movement                                                                                                               \\
        V                               & Trigeminal                            & B                                 & Touch to face, motor control of mandibles                                                                                  \\
        VI                              & Abducens                              & M                                 & Eye movement                                                                                                               \\
        VII                             & Facial                                & B                                 & Taste and facial expression                                                                                                \\
        VIII                            & (Vestibulocochlear) & S                                 & Hearing                                                                                                                    \\
        IX                              & Glossopharyngeal                      & B                                 & Taste and swallowing                                                                                                       \\
        X                               & Vagus                                 & B                                 & Taste and sensation from neck, thorax, abdomen, swallowing, control of larynx, parasympathetic nerves to heart and viscera \\
        XI                              & Spinal Accessory                      & M                                 & Movement of shoulders                                                                                                      \\
        XII                             & Hypoglossal                           & M                                 & Movement of tongue                                                                                                         \\
        \bottomrule
    \end{tabular}
\end{table}

\subsection{Mnemonic for Cranial Nerves}

\textbf{Ol}d \textbf{Op}ie \textbf{oc}casionally \textbf{tr}ies \textbf{tr}igonometry \textbf{a}nd \textbf{f}eels \textbf{ve}ry \textbf{glo}omy, \textbf{vagu}e, \textbf{a}nd \textbf{hypo}active.

\section{Terms \label{Terms}}

\begin{coloredlist}
    \item Santiago Ramon y Cajal (1911)
    \begin{coloredlist}
        \item Used the Golgi stain to show that neurons are separate cells.
    \end{coloredlist}
    \item \cyanit{Soma} -- Cell Body
    \item \cyanit{Dendrites} -- ``Branches''
    \begin{coloredlist}
        \item Purpose is to increase the surface area of the neuron, so it can receive the most amount of information.
    \end{coloredlist}
    \item \cyanit{Axon terminal botton} -- The ends of the neuron that send information.
    \item \cyanit{Glial cells} -- Support cells by insulating the axon for better communication.
    \item \cyanit{Myelin sheath} -- Insulates the axon.
    \item \cyanit{Nodes of Ranvier} -- Gaps in the myelin sheath.
    \item \cyanit{Unmylinated axons} are called grey matter.
    \item \cyanit{Ganglion} -- A collection of cell bodies in the PNS.
    \item \cyanit{Nerve} -- A collection of axons in the PNS.
    \item \cyanit{Nucleus} -- A collection of cell bodies in the CNS.
    \item \cyanit{Tract} -- A collection of axons in the CNS.
\end{coloredlist}

\begin{table}[htbp]
    \centering
    \begin{tabular}{cc}
        \toprule
        \textbf{Grey Matter} & \textbf{White Matter} \\ \midrule
        Cell bodies          & Myelinated axons      \\
        Dendrites            &                       \\
        Unmyelinated axons   &                       \\
        \bottomrule
    \end{tabular}
    \caption{Gray vs. White Matter}\label{tab:}
\end{table}


\begin{table}[htbp]
    \centering
    \begin{tabular}{lcc}
        \toprule
        & \textbf{Gray Matter} & \textbf{White Matter} \\
        \cmidrule(lr){2-3}
        \textbf{Location} & \textbf{Cell Bodies} & \textbf{Axons} \\
        \midrule
        \textbf{CNS} & Nucleus & Tract \\
        \textbf{PNS} & Ganglion & Nerve \\
        \bottomrule
    \end{tabular}
    \caption{Differentiation of Gray and White Matter in the CNS and PNS}\label{tab:gray_white}
\end{table}


\section{Brainstem}

\subsection{Hindbrain}

\begin{coloredlist}
    \item \cyanit{Myenlencephalon}
    \begin{coloredlist}
        \item \cyanit{Medulla Oblongata}
        \begin{coloredlist}
            \item Enlargement of the cord.
            \item Lots of gray matter.
            \item \cyanit{Reticular Formation}
            \begin{coloredlist}
                \item A network of nuclei.
                \item Regulates sleep, wakefulness, and arousal.
                \item Also regulates heart rate, blood pressure, respiration, and skeletal muscle tone.
            \end{coloredlist}
            \item \cyanit{Pyramids}
            \begin{coloredlist}
                \item Two ridges on the ventral surface.
                \begin{coloredlist}
                    \item Voluntary motor system.
                \end{coloredlist}
            \end{coloredlist}
            \item \cyanit{Olives}
            \begin{coloredlist}
                \item Audition and motor learning.
                \item Located on the lateral surface.
            \end{coloredlist}
        \end{coloredlist}
    \end{coloredlist}
    \item \cyanit{Metencephalon}
    \begin{coloredlist}
        \item \cyanit{Pons} -- ``Bridge''
        \begin{coloredlist}
            \item White matter on the outside and gray on the inside.
            \item \cyanit{Locus Coeruleus}
            \begin{coloredlist}
                \item Produces norepinephrine.
                \item The norepinephrine is sent to the forebrain.
            \end{coloredlist}
        \end{coloredlist}
        \item \cyanit{Cerebellum}
        \begin{coloredlist}
            \item Caudal portion of the brain.
            \item Balance, hand/eye coordination, soothes movements.
            \item Shifting attention between vision and hearing, sensory timing (judging rhythms), language, emotional control, and reward valuation.
            \item Cerebellar agenesis -- the cerebellum is not developed.
        \end{coloredlist}
    \end{coloredlist}
\end{coloredlist}

\subsection{Midbrain}

\subsubsection{Mesencephalon}

\begin{coloredlist}
    \item \cyanit{Techtum} = ``Roof''
    \begin{coloredlist}
        \item \cyanit{Superior Colliculus} -- Visual Reflexes
        \begin{coloredlist}
            \item Pupils opening and closing in response to light.
        \end{coloredlist}
        \item \cyanit{Inferior Colliculus} -- Auditory Reflexes
        \item Colliculus = ``Little Hill''
        \item \cyanit{Pineal Gland} -- Melatonin
    \end{coloredlist}
    \item \cyanit{Tegmentum} = ``Floor''
    \begin{coloredlist}
        \item \cyanit{Substantia Nigra} = ``Black substance.''
        \begin{coloredlist}
            \item Get its black coloring from the creation of dopamine.
            \item Clearly, this brain structure makes a majority of dopamine (1 of 3).
        \end{coloredlist}
        \item \cyanit{Red Nucleus} -- Motor coordination.
        \begin{coloredlist}
            \item Get its red color from iron oxidation.
            \item Connects to the cerebellum for that motor coordination.
        \end{coloredlist}
        \item \cyanit{Periaqueductal Gray Area} -- Opioids.
        \begin{coloredlist}
            \item Peri = around, so peri-aqueductual = around-the cerebral aqueduct.
            \item Handles endogenous pain relief.
        \end{coloredlist}
    \end{coloredlist}
\end{coloredlist}


\subsection{Forebrain}

\subsubsection{Diencephalon}

\begin{coloredlist}
    \item \cyanit{Thalamus}
    \begin{coloredlist}
        \item Massa Intermedia = intermediate mass. This connects the two halves together.
        \item Made up of many specific relay nuclei.
        \begin{coloredlist}
            \item \cyanit{Lateral Geniculate Nucleus} -- Vision
            \item \cyanit{Dorsal Medial Nucleus} -- Pain
            \begin{coloredlist}
                \item Routes the pain from the thalamus to the prefrontal cortex.
            \end{coloredlist}
        \end{coloredlist}
        \item \dots and of non-specific relay nuclei.
        \begin{coloredlist}
            \item \cyanit{Nucleus Reticularis} -- Promotes wakefulness.
            \begin{coloredlist}
                \item Goes to different parts of the brain, not just one specific part like the specific relay nuclei.
            \end{coloredlist}
        \end{coloredlist}
    \end{coloredlist}
    \item \cyanit{Hypothalamus}
    \begin{coloredlist}
        \item Irregular shape, size of a thumbnail.
        \item Encases the ventral part of the third ventricle.
        \item \textbf{Survival of the individual}
        \begin{coloredlist}
            \item Eating
            \item Drinking (water)
            \begin{coloredlist}
                \item Salt regulation
            \end{coloredlist}
            \item \cyanit{Suprachiasmatic Nucleus}
            \begin{coloredlist}
                \item Circadian rhythms
                \item Daily fluctuations of temperature
            \end{coloredlist}
        \end{coloredlist}
        \item \textbf{Survival of the species}
        \begin{coloredlist}
            \item Territoriality
            \item Sexual activity
            \item Reproduction
        \end{coloredlist}
        \item \textbf{Integration of information}
        \begin{coloredlist}
            \item Endocrine system
            \item Autonomic nervous system
        \end{coloredlist}
    \end{coloredlist}


    \subsubsection{Telencephalon}
    \item \cyanit{Corpus callosum}
    \begin{coloredlist}
        \item Connects the two hemispheres.
        \item Remember that the neurons in this structure go from lateral to lateral, and not from dorsal to ventral.
        \item Creates the roof of the lateral ventricles.
        \item Agenesis of the cc
        \begin{coloredlist}
            \item AKA: Callosal Agenesis
            \item Vision impairments,
            \item hypotonia,
            \item poor motor coordination,
            \item delays in motor milestones,
            \begin{coloredlist}
                \item (Such as sitting and walking.)
            \end{coloredlist}
            \item cognitive disability,
            \begin{coloredlist}
                \item (Disability in complex problem solving.)
            \end{coloredlist}
            \item and social difficulties.
            \begin{coloredlist}
                \item (Missing subtle social cues maybe cause of impaired fair processing.)
            \end{coloredlist}
        \end{coloredlist}
        \item \cyanit{Corpus Callosotomy} -- Split brain surgeries.
        \begin{coloredlist}
            \item Used to treat epilepsy.
            \item Gives information about lateralization of hemispheres.
            \item \textbf{Left Hemisphere}
            \begin{coloredlist}
                \item language
                \item serial events
            \end{coloredlist}
            \item \textbf{Right Hemisphere}
            \begin{coloredlist}
                \item creativity
                \item synthesis
            \end{coloredlist}
        \end{coloredlist}
    \end{coloredlist}
    \item \cyanit{Basal Ganglia}
    \begin{coloredlist}
        \item \textbf{Function:}
        \begin{coloredlist}
            \item Initiation of Voluntary Movements.
            \item \hyperlink{parkinson}{Click here for Parkinson's continuation.}
        \end{coloredlist}
        \item Curls laterally around the thalamus.
        \item \cyanit{Striatum}
        \begin{coloredlist}
            \item \cyanit{Caudate Nucleus} = ``Nucleus with a Tail''
            \begin{coloredlist}
                \item Obsessive Compulsive Disorder (OCD)
                \begin{coloredlist}
                    \item MIXED RESULTS
                    \begin{coloredlist}
                        \item Too much activity, too large.
                    \end{coloredlist}
                \end{coloredlist}
                \item Romantic Love
                \begin{coloredlist}
                    \item Fisher, Aron, and Brown
                    \begin{coloredlist}
                        \item Anthropologist used fMRI with a picture of neutral and romantic partners.
                        \item The CN activity was increased for loved one.
                    \end{coloredlist}
                \end{coloredlist}
                \item Larger in folks with incredible episodic memories (superior autobiographical memory).
                \begin{coloredlist}
                    \item How large? 7-8 SDs larger.
                \end{coloredlist}
            \end{coloredlist}
            \item \cyanit{Putamen} = ``Shell''
            \item \cyanit{Nucleus Accumbens}
            \begin{coloredlist}
                \item Nucleus Accumbens Septi = ``Nucleus leaning against the septum.''
                \begin{coloredlist}
                    \item Where the head of the caudate and the most anterior portion of the putamen come together.
                    \item Plays an important role in reinforcement, pleasure, and addiction.
                \end{coloredlist}
            \end{coloredlist}
        \end{coloredlist}
        \item \cyanit{Globus Pallidus} = ``Pale Globe''
        \item \textit{Note:} When people mention the putamen and the globus pallidus, they call it the lentiform nucleus.
    \end{coloredlist}
    \item \cyanit{Limbic System}
    \begin{coloredlist}
        \item \cyanit{Hippocampus}
        \begin{coloredlist}
            \item In charge of moving memories from short-term to long-term.
            \item Emotion, selective attention, learning, and memory.
        \end{coloredlist}
        \item \cyanit{Amygdala}
        \begin{coloredlist}
            \item In charge of emotions.
            \item Fear and aggression, territoriality, odor processing, and sexual activity.
            \item \textbf{Amygdala and Fear}
            \begin{coloredlist}
                \item 1930's.
                \item Lesions to amygdala in monkeys.
                \item Many things happened\ldots
                \begin{coloredlist}
                    \item Exploratory behavior of objects (put in mouth--hyperorality).
                    \item Hypersexuality.
                    \item Loss of fear.
                    \begin{coloredlist}
                        \item Freezing, increased heart rate, hair standing on end, etc.
                        \item Lost their fear of the human experimenters.
                    \end{coloredlist}
                \end{coloredlist}
                \item \cyanit{Kluver-Bucy Syndrome}
                \begin{coloredlist}
                    \item Damaging the anterior temporal lobes.
                    \item Herpes encephalitis and trauma.
                \end{coloredlist}
                \begin{coloredlist}
                    \item Loss of normal fear and anger responses.
                \end{coloredlist}
                \item Facial mimicry
                \begin{coloredlist}
                    \item Seeing fear in others lead to fear expression.
                    \item AND has amygdala activity.
                \end{coloredlist}
                \item \textbf{Other things}
                \begin{coloredlist}
                    \item Social networks (Bickart et al., 2010)
                    \begin{coloredlist}
                        \item Size and complexity of social network + correlated with amygdala size.
                        \item MAYBE: More effectively identify, learn about, and recognize socioemotional cues.
                    \end{coloredlist}
                    \item Political views (Rees et al., 2011)
                    \begin{coloredlist}
                        \item Took extreme liberals and extreme conservatives and found that the more extreme conservatives had a larger amygdala than the extreme liberals.
                    \end{coloredlist}
                \end{coloredlist}
                \item \textbf{How burnout is related to your brain\ldots}
                \begin{coloredlist}
                    \item Worse at suppressing negative emotions.
                    \item Big amygdala \& weak connection to frontal lobe.
                \end{coloredlist}
            \end{coloredlist}
        \end{coloredlist}
        \item \cyanit{Cingulate Gyrus}
        \begin{coloredlist}
            \item Selective attention.
            \item Love (like the cingulate gyrus)
            \begin{coloredlist}
                \item Same studies show increased activity for loved ones.
            \end{coloredlist}
            \item Pain.
            \begin{coloredlist}
                \item Serves as alarm for distress
                \item Association of the emotional components and the sensory components of pain.
                \item Sympathetic pain (empathy).
                \item Social rejection.
                \begin{coloredlist}
                    \item Eisenberger (1990s)
                    \item Cyberball
                    \begin{coloredlist}
                        \item A computer game where you play catch.
                        \item The other players stop throwing the ball to you.
                        \item The cingulate gyrus lights up.
                    \end{coloredlist}
                \end{coloredlist}
            \end{coloredlist}
        \end{coloredlist}
        \item \cyanit{Fornix}
        \item \cyanit{Mammillary Bodies}
        \item \cyanit{Septal Nucleus}
    \end{coloredlist}
\end{coloredlist}
\subsubsection{Cerebral Cortex}
\begin{coloredlist}
    \item Cortex = ``bark''
    \item Many convolutions
    \begin{coloredlist}
        \item Sulci/fissures
        \item Gyri
    \end{coloredlist}
    \item Gray matter.
    \item 6 Layers
    \item Four lobes
    \begin{coloredlist}
        \item \cyanit{Frontal Lobe}
        \begin{coloredlist}
            \item Executive functions, motor control, and language production (Broca's area).
        \end{coloredlist}
        \item \cyanit{Parietal Lobe}
        \begin{coloredlist}
            \item Lips, toes, and spacial awareness.
        \end{coloredlist}
        \item \cyanit{Temporal Lobe}
        \begin{coloredlist}
            \item Memory, hearing, and language comprehension (Wernicke's area).
        \end{coloredlist}
        \item \cyanit{Occipital Lobe}
        \begin{coloredlist}
            \item Vision
        \end{coloredlist}
        \item \textbf{How they are separated:}
        \begin{coloredlist}
            \item Frontal \(\leftrightarrow\) Parietal: \cyanit{Central Sulcus}
            \item Parietal \(\leftrightarrow\) Occipital: \cyanit{Parieto-Occipital Sulcus}
            \item Temporal \(\leftrightarrow\) Frontal/Parietal: \cyanit{Lateral Sulcus (Sylvian Fissure)}
        \end{coloredlist}
    \end{coloredlist}
    \item \cyanit{Nucleus Accumbens}
\end{coloredlist}




\hypertarget{parkinson}{}
\section{Parkinson's Disease}
\begin{coloredlist}
    \item \cyanit{Bradykinesia}
    \begin{coloredlist}
        \item Slowness of movement.
    \end{coloredlist}
    \item \cyanit{Akinesia}
    \begin{coloredlist}
        \item Difficulty initiating voluntary movements.
    \end{coloredlist}
    \item \cyanit{Rigidity}
    \begin{coloredlist}
        \item Increased muscle tone.
    \end{coloredlist}
    \item \cyanit{Tremors}
    \begin{coloredlist}
        \item Involuntary shaking of hands and jaw most prominent at rest.
    \end{coloredlist}
\end{coloredlist}

\section{Alzheimer's Disease}

\begin{coloredlist}
    \item Progressive memory loss.
    \item Affects the cortex and hippocampus.
    \item Suffers from both retrograde and anterograde amnesia.
\end{coloredlist}


% \begin{coloredlist}
%     \item 
% \end{coloredlist}