% ===========================
% Section 1.1: Neuroanatomy \& Nervous System Structure
% ===========================
\section*{1.1 \squigglyline}

\textbf{Directions:} For short answer questions, write your response in the space provided. Most of the short answer questions should be only a sentence or two long. For multiple choice questions, circle the letter of the correct response. For fill-in-the-blank questions, write your response in the space provided. For matching questions, write the letter of the correct response in the space to the right. For true or false questions, write ``True'' if the statement is true, and if the statement is false, \textit{explain why} the statement is false. Lastly, the sections (1.1, 1.2, etc.) are provided to break the exam into smaller parts. Good luck!

\begin{enumerate}[label=\textbf{Q1.1.\arabic*}]
      % Short answer question
      \item \textbf{Short Answer:} Define neuroscience. \\
            \textit{Answer:} \\%The study of the nervous system.

            % T/F question
      \item \textbf{True or False:} Behavioral neuroscience involves understanding the nervous system’s underlying behavior. \\
            \textit{Answer:} %True.

% Change this question
            % Fill in the blank
      \item \textbf{Fill in the Blank:} The Central Nervous System (CNS) is composed of the \underline{\hspace{2cm}} and the \underline{\hspace{2cm}}.
            \\
            %\textit{Answer:} Brain; spinal cord.

            % Matching question
      \item \textbf{Matching:} Match the following systems with their primary functions.
            \begin{wordbox}
                  \begin{enumerate}[label=(\roman*)]
                        \item Voluntary motor control and sensory input.
                        \item Involuntary control of the gastrointestinal system.
                        \item Involuntary control of smooth muscle and glands.
                  \end{enumerate}
            \end{wordbox}
            \begin{enumerate}[label=(\alph*)]
                  \item Somatic Nervous System \quad \dotfill \quad \underline{\hspace{3cm}}\\[0.5em]
                  \item Autonomic Nervous System \quad \dotfill \quad \underline{\hspace{3cm}}\\[0.5em]
                  \item Enteric Nervous System \quad \dotfill \quad \underline{\hspace{3cm}}
            \end{enumerate}
            %\textit{Answer:} (a) Voluntary motor control and sensory input; (b) Involuntary control of smooth muscle and glands; (c) Involuntary control of the gastrointestinal system.

% Change this question
      \item \textbf{Fill in the Blank} The autonomic nervous system regulates the body's \underline{\hspace{3cm}} AND \underline{\hspace{3cm}} response. \\
            %\textit{Answer} Fight or flight; rest and digest. OR Sympathetic; parasympathetic.

% Change this question
      \item \textbf{Short Answer:} How does the flight-or-flight response affect your body? \\
            \textit{Answer:} \\%Increases Heart rate, blood pressure, respiration, and alertness.

% Change this question
      \item \textbf{Fill in the Blank:} In the fecal microbiota transplant study, researchers found that the microbiota change the \underline{\hspace{3cm}} of the mice. \\
            %\textit{Answer:} Behavior.

% Change this question
      \item \textbf{Fill in the Blank:} In the elevated plus maze study, researchers found that the anxious mice were more willing to \underline{\hspace{3cm}} after the fecal microbiota transplant. \\
            %\textit{Answer:} Walk on the open arms.
\end{enumerate}

% ===========================
% Section 1.2: Meninges
% ===========================
\section*{1.2 \squigglyline}
\begin{enumerate}[label=\textbf{Q1.2.\arabic*}]
      \item \textbf{Short Answer:} What is the primary function of the meninges? \\
            \textit{Answer:} \\%To cover and protect the nervous system.

      \item \textbf{Multiple Choice:} Which of the following is the middle meningeal layer?
            \begin{tasks}[label=(\Alph*), label-width=1.5em, item-indent=1.7em](2) % two columns 
                  \task Arachnoid Membrane
                  \task Dura Mater
                  \task Pia Mater
                  \task Subarachnoid Space
            \end{tasks}
            %\textit{Answer:} A.

% Change this question. This is too easy.
      \item \textbf{Short Answer:} Why did early anatomists call the outermost meningeal layer ``pachymenginges''? \\
            \textit{Answer:} \\%Because it was similar to elephant skin.

      \item \textbf{Fill in the Blank} The Peripheral Nervous System uses \underline{\hspace{3cm}} layer(s) of the meninges. \\
            %\textit{Answer:} Two.

      \item \textbf{Fill in the Blank:} Arachnoid \underline{\hspace{3cm}} are web-like structures that connect the Arachnoid Membrane to the \textit{pia mater}
            %\textit{Answer:} trabeculae.

% Change this question
      \item \textbf{Fill in the Blank:} The subarachnoid space is filled with \underline{\hspace{3cm}}. \\
            %\textit{Answer:} Cerebrospinal fluid.

      \item \textbf{Short Answer:} What is meningitis? \\
            \textit{Answer:} \\%Inflammation of the meninges.

% Change this question
      % Changed question: Making the question more challenging by evaluating symptom combinations.
      \item \textbf{Multiple Choice:} Which combination of symptoms is LEAST consistent with a diagnosis of acute bacterial meningitis?
            \begin{tasks}[label=(\Alph*), label-width=1.5em, item-indent=1.7em](2)
                  \task Fever, headache, and nuchal rigidity.
                  \task Photophobia, vomiting, and altered mental status.
                  \task None of the above
            \end{tasks}
            %\textit{Answer:} D.
\end{enumerate}

% ===========================
% Section 1.3: Cerebrospinal Fluid (CSF)
% ===========================
\section*{1.3 \squigglyline}
\begin{enumerate}[label=\textbf{Q1.3.\arabic*}]
      \item \textbf{Short Answer:} List two functions of CSF. \\
            \textit{Answer:} \\%Protection, transportation of neurotransmitters, waste, hormones, and nutrients.

% Change this question
      \item \textbf{True or False:} CSF is similar in composition to blood plasma. \\
            \textit{Answer:} %True.

      \item \textbf{Fill in the Blank:} CSF is produced by the \underline{\hspace{3cm}} cells lining the lateral ventricles.
            %\textit{Answer:} ependymal.

      \item \textbf{Fill in the Blank:} A contra-coup injury is an injury that occurs on the \underline{\hspace{3cm}} side of the brain from the impact site.
            %\textit{Answer:} opposite.
\end{enumerate}

% ===========================
% Section 1.4: Flow of CSF
% ===========================
\section*{1.4 \squigglyline}
\begin{enumerate}[label=\textbf{Q1.4.\arabic*}]
      \item \textbf{Matching:} Match the location with its description.
            \begin{wordbox}
                  \begin{enumerate}[label=(\roman*)]
                        \item Connected to the pituitary gland via the infundibulum.
                        \item Production of CSF and connection to the third ventricle via the interventricular foramen.
                        \item Connects the third and fourth ventricles.
                  \end{enumerate}
            \end{wordbox}
            \begin{enumerate}[label=(\alph*)]
                  \item Lateral Ventricles \quad \dotfill \quad \underline{\hspace{3cm}}\\[0.5em]
                  \item Third Ventricle \quad \dotfill \quad \underline{\hspace{3cm}}\\[0.5em]
                  \item Cerebral Aqueduct \quad \dotfill \quad \underline{\hspace{3cm}}
            \end{enumerate}
            %\textit{Answer:} (a) Production of CSF and connection to the third ventricle via the interventricular foramen; (b) Connected to the pituitary gland via the infundibulum; (c) Connects the third and fourth ventricles.

      \item \textbf{True or False:} The central canal connects the fourth ventricle to the spinal cord. \\
            \textit{Answer:} %True.
\end{enumerate}

% ===========================
% Section 1.5: Dumping of CSF
% ===========================
\section*{1.5 \squigglyline}
\begin{enumerate}[label=\textbf{Q1.5.\arabic*}]
      \item \textbf{Short Answer:} Where is CSF absorbed into the bloodstream? \\
            \textit{Answer:} \\%Through the arachnoid villi/granulations into the superior sagittal sinus.

      \item \textbf{Fill in the Blank:} \underline{\hspace{3cm}} is caused by swelling of the brain due to a blockage in the CSF flow. \\
            %\textit{Answer:} Hydrocephalus.

% Change this question to a fill in the blank or multiple choice.
      \item \textbf{True or False:} The \textit{Interventricular Foramen} connects the lateral ventricles to the third ventricle. \\
            \textit{Answer:} %True.
\end{enumerate}

% ===========================
% Section 1.6: Getting Some CSF Out -- or -- Putting Something Into It
% ===========================
\section*{1.6 \squigglyline}

\begin{enumerate}[label=\textbf{Q1.6.\arabic*}]
      
% Change this question
      \item \textbf{Fill in the Blank:} The term ``\textit{epi}'' means \underline{\hspace{3cm}} and the term ``\textit{tap}'' means \underline{\hspace{3cm}}. \\
            %\textit{Answer:} %Something in; Taking something out.

      \item \textbf{Multiple Choice:} A lumbar puncture is sometimes called a:
            \begin{tasks}[label=(\Alph*), label-width=1.5em, item-indent=1.7em](2) % two columns 
                  \task Brain Tap
                  \task Spinal Tap
                  \task CSF Drain
                  \task Ventricular Tap
            \end{tasks}
            %\textit{Answer:} %B.

      \item \textbf{Fill in the Blank:} The needle for a lumbar puncture is typically inserted into the \underline{\hspace{3cm}} sac in the lumbar region.
            \\
            %\textit{Answer:} dural.
\end{enumerate}

% ===========================
% Section 1.7: Cranial Nerves
% ===========================
\section*{1.7 \squigglyline}
\begin{enumerate}[label=\textbf{Q1.7.\arabic*}]

      \item \textbf{Matching:} Match the cranial nerve number with its function.
            \begin{wordbox}
                  \begin{enumerate}[label=(\roman*)]
                        \item Facial sensation or motor control of the mandible
                        \item Vision
                        \item Smell
                  \end{enumerate}
            \end{wordbox}
            \begin{enumerate}[label=(\alph*)]
                  \item I \quad \dotfill \quad \underline{\hspace{3cm}}\\[0.5em]
                  \item II \quad \dotfill \quad \underline{\hspace{3cm}}\\[0.5em]
                  \item V \quad \dotfill \quad \underline{\hspace{3cm}}
            \end{enumerate}
            %\textit{Answer:} (a) Olfaction (smell); (b) Vision; (c) Facial sensation or motor control of the mandible.

      \item \textbf{True or False:} Cranial nerve IV controls pupil constriction and eye movement. \\
            \textit{Answer:} %False. (It only controls eye movement. OR, the statement is true if you're talking about cranial nerve III.)

      \item \textbf{Short Answer:} What is the function of cranial nerve X? \\
            \textit{Answer:} \\ %Taste and sensation from neck, thorax, abdomen, swallowing, control of laryx, parasympathetic nerves to heart and viscera.

% Change this question
      \item \textbf{Fill in the Blank:} There are \underline{\hspace{3cm}} cranial nerve(s) that modulate eye movements, and \underline{\hspace{3cm}} cranial nerve(s) that control the sense of vision. \\
            %\textit{Answer:} %Three; one.

      \item \textbf{True or False:} The term ``\textit{glossal}'' means ``taste''. \\
            \textit{Answer:} %False. (It means tongue.)

      \item \textbf{Fill in the Blank:} For the following table, fill in the blanks with the correct name.

% Add in sensory or motor or both column.
            \begin{multicols}{2}
                  \begin{tabular}{cc}
                        \toprule
                        \textbf{Cranial Nerve} & \textbf{Name}            \\ \midrule
                        I                      & \underline{\hspace{3cm}} \\[0.5em]
                        II                     & \underline{\hspace{3cm}} \\[0.5em]
                        III                    & \underline{\hspace{3cm}} \\[0.5em]
                        IV                     & \underline{\hspace{3cm}} \\[0.5em]
                        V                      & \underline{\hspace{3cm}} \\[0.5em]
                        VI                     & \underline{\hspace{3cm}} \\[0.5em]
                        \bottomrule
                  \end{tabular}
                  \begin{tabular}{cc}
                        \toprule
                        \textbf{Cranial Nerve} & \textbf{Name}            \\ \midrule
                        VII                    & \underline{\hspace{3cm}} \\[0.5em]
                        VIII                   & \underline{\hspace{3cm}} \\[0.5em]
                        IX                     & \underline{\hspace{3cm}} \\[0.5em]
                        X                      & \underline{\hspace{3cm}} \\[0.5em]
                        XI                     & \underline{\hspace{3cm}} \\[0.5em]
                        XII                    & \underline{\hspace{3cm}} \\[0.5em]
                        \bottomrule
                  \end{tabular}
            \end{multicols}

\end{enumerate}

% ===========================
% Section 1.8: Terms
% ===========================
\section*{1.8 \squigglyline}
\begin{enumerate}[label=\textbf{Q1.8.\arabic*}]

      \item \textbf{Short Answer:} What does the term \textit{Soma} refer to? \\
            \textit{Answer:}% The cell body of a neuron.

      \item \textbf{True or False:} The term ``\textit{nucleus}'' refers to a collection of cell bodies in the PNS. \\
            \textit{Answer:} %False. (It refers to a collection of cell bodies in the CNS.)

      \item \textbf{Fill in the Blank:} Fill in the following table for the terms that match its definition. \\

            \begin{tabular}[htbp]{cc}
                  \toprule
                  \textbf{Term}            & \textbf{Definition}                           \\ \midrule
                  Tract                    & \underline{\hspace{10cm}}                     \\[0.5em]
                  \underline{\hspace{3cm}} & The ends of the neuron that send information. \\[0.5em]
                  \underline{\hspace{3cm}} & Extends surface area of the neuron.           \\[0.5em]
                  \underline{\hspace{3cm}} & A collection of axons in the PNS.             \\        \bottomrule
            \end{tabular}
            %\textit{Answer:} %A collection of axons in the CNS; axonal buttons; dendrites; nerve.

      \item \textbf{True or False:} \textit{Ganglion} are collections of axons. \\
            \textit{Answer:} %False. (They are collections of cell bodies.)

      \item \textbf{Short Answer:} What is the function of the myelin sheath? \\
            \textit{Answer:} %To insulate the axon and speed up the transmission of the action potential.
\end{enumerate}

% ===========================
% Section 1.9: Brainstem --- Hindbrain and Midbrain
% ===========================
\section*{1.9 \squigglyline}
\begin{enumerate}[label=\textbf{Q1.9.\arabic*}]
      \item \textbf{Short Answer:} Name two principal structures of the hindbrain. \\
            \textit{Answer:} \\%Medulla Oblongata and Pons (or Cerebellum).

      \item \textbf{Short Answer:} What vital functions are regulated by the Reticular Formation in the Medulla Oblongata? \\
            \textit{Answer:} \\ % Heart rate, blood pressure, respiration.

      \item \textbf{Short Answer:} Describe the role of the Pons in the brainstem. \\
            \textit{Answer:} % Acts as a bridge connecting various brain regions and houses the Locus Coeruleus.

      \item \textbf{Fill in the Blank:} Pons directly translates to \underline{\hspace{3cm}} in Latin. \\
            %\textit{Answer:} %Bridge.

      \item \textbf{Fill in the Blank:} The \underline{\hspace{3cm}} (specific) is a structure in the pons that produces norepinephrine. \\
            %\textit{Answer:} %Locus Coeruleus.

% Change this question. It directly answers the question before it.
      \item \textbf{True or False:} The norepinephrine produced by the Locus Coeruleus is sent primarily to the hind brain. \\
            \textit{Answer:} %False. (It is sent to the forebrain.)

      \item \textbf{Short Answer:} What is the primary function of the cerebellum? \\
            \textit{Answer:} %Coordination of movement and balance.

      \item \textbf{\textit{Bonus} Short Answer:} Can you name every function of the cerebellum? (There's 8!) \\
            \textit{Answer:} \\[0.5em]%Balance, hand-eye coordination, soothes movements, shifts attention between vision and hearing, sensory timing (judging rhythm), language, emotional control, and reward valuation.

      \item \textbf{Fill in the Blank:} The rare malformation wherein the cerebellum is not developed is called \underline{\hspace{3cm}}. \\
            %\textit{Answer:} %Cerebellar agenesis.

% Change this question. It is answered by a later question.
      \item \textbf{Multiple Choice:} Which of the following midbrain structures is primarily responsible for visual reflexes? \\
            (A) Olives \quad (B) Pyramids \quad (C) Superior Colliculus \quad (D) Substantia Nigra \\
            %\textit{Answer:} %C.

      \item \textbf{Fill in the Blank:} The \underline{\hspace{3cm}} in the midbrain is critical for motor coordination. \\
            %\textit{Answer:} %Red nucleus.

      \item \textbf{Short Answer:} What does the \textit{periaqueductal gray area} produce? \\
            \textit{Answer:} \\%Opioiods

% Change this question to fill in the blank.
      \item \textbf{True or False:} The medulla oblongata contains the reticular formation which regulates vital functions like heart rate and respiration. \\
            \textit{Answer:} %True.

% Change this question. It answers the question referenced before.
      \item \textbf{Fill in the Blank:} The midbrain’s \underline{\hspace{3cm}} Colliculus is important for visual reflexes, while the \underline{\hspace{3cm}} Colliculus is involved in auditory reflexes.
            % \textit{Answer:} Superior; Inferior.

      \item \textbf{Multiple Choice:} The \textit{substantia nigra} is known for its role in:
            \begin{enumerate}[label=(\Alph*)]
                  \item Serotonin production
                  \item Dopamine production
                  \item GABA production
                  \item Acetylcholine production
            \end{enumerate}
            %\textit{Answer:} %B.
\end{enumerate}

% ===========================
% Section 1.10: Forebrain --- Diencephalon and Telencephalon
% ===========================
\section*{1.10 \squigglyline}
\begin{enumerate}[label=\textbf{Q1.10.\arabic*}]
      \item \textbf{Short Answer:} What is the role of the thalamus in the brain? \\
            \textit{Answer:} %It relays sensory and motor signals

      \item \textbf{Fill in the Blank:} Fill in the following table with the correct terms.
            \begin{table}[htbp]
                  \centering
                  \begin{tabular}{cc}
                        \toprule
                        \textbf{Thalamic Relay Nuclei} & \textbf{Description} \\ \midrule
                        \underline{\hspace{3cm}}       & Vision               \\
                        \underline{\hspace{3cm}}       & Pain                 \\
                        \underline{\hspace{3cm}}       & Promotes wakefulness \\
                        \bottomrule
                  \end{tabular}
            \end{table}
            %\textit{Answer:} %Lateral Geniculate Nucleus; Dorsal Medial Nucleus; Nucleus Reticularis

% Change this question
      \item \textbf{Short Answer:} What does it mean to be a non-specific relay nuclei? \\
            \textit{Answer:} %It means that the thalamic nuclei do not have a specific function.

      \item \textbf{Multiple Choice:} The thalamic nucleus responsible for pain routing sends those signals to the:
            \begin{tasks}[label=(\Alph*), label-width=1.5em, item-indent=1.7em](2) % two columns 
                  \task Prefrontal Cortex
                  \task Primary Somatosensory Cortex
                  \task Primary Motor Cortex
                  \task Amygdala
            \end{tasks}
            %\textit{Answer:} %A.

% Either move this question to a different spot or change it.
      \item \textbf{Fill in the Blank:} The \textit{Massa Intermedia} connects the left \underline{\hspace{3cm}} half with the right half. \\
            %\textit{Answer:} %Thalamus.

      \item \textbf{Multiple Choice:} Which of the following is NOT a function of the hypothalamus?
            \begin{enumerate}[label=(\Alph*)]
                  \item Survival of the individual
                  \item Survival of the species
                  \item Regulation of endocrine system
                  \item Integration of information
            \end{enumerate}
            %\textit{Answer:} %C.



% Change this question
      \item \textbf{True or False:} The hypothalamus is involved in both individual survival functions (like eating and drinking) and species survival functions (such as reproduction). \\
            \textit{Answer:} %True.


% Change this question. Possibly multiple choice instead.
      \item \textbf{True or False:} The \textit{Suprachiasmatic Nucleus} is responsible for circadian rhythms. \\
            \textit{Answer:} %True.

% Change this question. Too easy with process of elimination.   
      \item \textbf{Matching:} Match the following telencephalic structures with their functions.
            \begin{wordbox}
                  \begin{enumerate}[label=(\roman*)]
                        \item Sensory integration and spatial awareness.
                        \item Executive functions, motor control, language production.
                        \item Memory, hearing, language comprehension.
                        \item Vision
                  \end{enumerate}
            \end{wordbox}
            \begin{enumerate}[label=(\alph*)]
                  \item Frontal Lobe \quad \dotfill \quad \underline{\hspace{3cm}}\\[0.5em]
                  \item Temporal Lobe \quad \dotfill \quad \underline{\hspace{3cm}}\\[0.5em]
                  \item Parietal Lobe \quad \dotfill \quad \underline{\hspace{3cm}}\\[0.5em]
                  \item Occipital Lobe \quad \dotfill \quad \underline{\hspace{3cm}}
            \end{enumerate}
            %\textit{Answer:} (a) Executive functions, motor control, language production; (b) Memory, hearing, language comprehension; (c) Sensory integration and spatial awareness. (d) Vision.

      \item \textbf{Fill in the Blank:} The \textit{Corpus Callosum}'s neurons go from \underline{\hspace{3cm}} to \underline{\hspace{3cm}} and not from \underline{\hspace{3cm}} to \underline{\hspace{3cm}}. (Your answers should directions.) \\
            %\textit{Answer:} %Lateral to lateral; dorsal to ventral.

      \item \textbf{Multiple Choice:} Which of the following is NOT a symptom of Callosal Agenesis?
            \begin{tasks}[label=(\Alph*), label-width=1.5em, item-indent=1.7em](2) % two columns 
                  \task Impaired motor coordination
                  \task Impaired language comprehension
                  \task Impaired spatial awareness
                  \task Impaired executive functions
            \end{tasks}
            %\textit{Answer:} %B.

      \item \textbf{Fill in the Blank:} The \underline{\hspace{3cm}} is involved in the regulation of voluntary motor control. \\
            %\textit{Answer:} %Basal Ganglia.

      \item \textbf{Fill in the Blank:} The \textit{Striatum} consists of the \underline{\hspace{3cm}}, \underline{\hspace{3cm}}, and the \underline{\hspace{3cm}}. (\textit{Note:} The \textit{Globus Pallidus} is not a viable answer here.) \\
            %\textit{Answer:} %Caudate Nucleus; Putamen, Nucleus accumbens.

      \item \textbf{True or False:} When people mention the Nucleus Accumbens and the Globus Pallidus, they call it the lentiform nucleus. \\
            \textit{Answer:} %False. (The putamen and the globus pallidus are called the lentiform nucleus.)

      \item \textbf{Fill in the Blank:} Participants in Rees et al.'s (2011) study showed extreme conservatives had a larger \underline{\hspace{3cm}} \\
            %\textit{Answer:} %Amygdala.

      \item \textbf{Short Answer:} How does the hippocampus have an impact upon memory? \\
            \textit{Answer:} %Moves memories from short-term to long-term storage.

      \item \textbf{Short Answer:} What is the left hemisphere of the brain responsible for? And the right hemisphere? (2 each.) \\
            \textit{Answer:} \\%Left: Language, serial events; Right: Creativity, synthesis of information. 

      \item \textbf{Matching:} Arrange the following structures in the brain in the right order.
      \begin{wordbox}
            \begin{enumerate}[label=(\roman*)]
                  \item Tectum
                  \item Medulla Oblogata
                  \item Limbic System
                  \item Hypothalamus
            \end{enumerate}
      \end{wordbox}
      \begin{enumerate}[label=(\alph*)]
            \item Myelencephalon \quad \dotfill \quad \underline{\hspace{3cm}}\\[0.5em]
            \item Diencephalon \quad \dotfill \quad \underline{\hspace{3cm}}\\[0.5em]
            \item Telencephalon \quad \dotfill \quad \underline{\hspace{3cm}}\\[0.5em]
            \item Mesencephalon \quad \dotfill \quad \underline{\hspace{3cm}}\\
      \end{enumerate}
      %\textit{Answer:} %B, D, C, A.

      \item \textbf{Short Answer:} What is \textit{Kluver-Bucy Syndrome}? \\
            \textit{Answer:} %A condition where the amygdala is damaged, leading to loss of normal fear and anger responses.

      \item \textbf{Fill in the Blank:} The \underline{\hspace{3cm}} is involved in the regulation of emotions, memory, and motivation. \\
            %\textit{Answer:} %Limbic System.

      \item \textbf{Fill in the Blank:} Selective attention, love, and pain are all functions of the \underline{\hspace{3cm}}. \\
            %\textit{Answer:} %Cingulate gyrus.

      \item \textbf{Short Answer:} What lobes make up the \textit{Cerebral Cortex}? \\
            \textit{Answer:} \\%Frontal, Parietal, Temporal, Occipital.

      \item \textbf{Short Answer:} We learned that the \textit{Temporal Lobe} is responsible for language comprehension. Who was the scientist that discovered this? Similarly, we learned that the \textit{Frontal Lobe} is responsible for language production. Who was the scientist that discovered this? \\
            \textit{Answer:} \\%Wernicke; Broca.

      \item \textbf{Multiple Choice:} In a study by Eisenberger (1990) called the Cyberball study, participants who were excluded from the game showed increased activity in the:
            \begin{tasks}[label=(\Alph*), label-width=1.5em, item-indent=1.7em](2) % two columns 
                  \task Amygdala
                  \task Hippocampus
                  \task Anterior Cingulate Cortex
                  \task Septal Nucleus
            \end{tasks}
            %\textit{Answer:} %C.

      \item \textbf{Fill in the Blank:} Fill out the following table with the correct terms.
\end{enumerate}

\begin{table}[ht]
      \centering
      \begin{tabular}{l l l l}
            \toprule
            \textbf{Major Divisions} & \textbf{Ventricle}                                       & \textbf{Subdivision}          & \textbf{Principal Structures} \\
            \midrule

            %--- Forebrain Block ----------------------------------
            \multirow{7}{*}{Forebrain}
                                     & \multirow{5}{*}{\underline{\hspace{3cm}}}
                                     & \multirow{4}{*}{\underline{\hspace{3cm}}}
                                     & \underline{\hspace{3cm}}                                                                                   \\
                                     &                                                          &
                                     & \underline{\hspace{3cm}}                                                                                   \\
                                     &                                                          &
                                     & \underline{\hspace{3cm}}                                                                                     \\
                                     &                                                          &
                                     & \underline{\hspace{3cm}}                                                                                     \\[0.5em]
            \cline{3-4}                                                                                                                                         \\[-0.5em]
                                     & \multirow{2}{*}{\underline{\hspace{3cm}}}             & \multirow{2}{*}{\underline{\hspace{3cm}}}
                                     & \underline{\hspace{3cm}}                                                                                          \\
                                     &                                                          &
                                     & \underline{\hspace{3cm}}                                                                                      \\
            \midrule

            %--- Midbrain Block ------------------------------------
            \multirow{4}{*}{Midbrain}
                                     & \multirow{4}{*}{\underline{\hspace{3cm}}}
                                     & \multirow{4}{*}{\underline{\hspace{3cm}}}
                                     & \underline{\hspace{3cm}}                                                                                             \\
                                     &                                                          &
                                     & \underline{\hspace{3cm}}                                                                                         \\
                                     &                                                          &
                                     & \underline{\hspace{3cm}}                                                                               \\
                                     &                                                          &
                                     & \underline{\hspace{3cm}}                                                                               \\
            \midrule

            %--- Hindbrain Block -----------------------------------
            \multirow{4}{*}{Hindbrain}
                                     & \multirow{4}{*}{\underline{\hspace{3cm}}}
                                     & \multirow{2}{*}{\underline{\hspace{3cm}}}
                                     & \underline{\hspace{3cm}}                                                                                        \\
                                     &                                                          &
                                     & \underline{\hspace{3cm}}                                                                                               \\[0.5em]
            \cline{3-4}                                                                                                                                         \\[-0.5em]
                                     &                                                          & \underline{\hspace{3cm}}
                                     & \underline{\hspace{3cm}}                                                                                 \\
            \bottomrule
      \end{tabular}
      \label{tab:brain-divisions}
\end{table}

% ===========================
% Section 1.11: Parkinson's and Alzheimer's Disease
% ===========================
\section*{1.11 \squigglyline}
\begin{enumerate}[label=\textbf{Q1.11.\arabic*}]
      \item \textbf{Multiple Choice:} Which of the following is NOT a typical symptom of Parkinson's Disease? \\
            (A) Bradykinesia \quad (B) Rigidity \quad (C) Tremors \quad (D) Hyperactivity
            %\textit{Answer:} %D

      \item \textbf{Short Answer:} Name one symptom associated with Alzheimer's Disease. \\
            \textit{Answer:} %Progressive memory loss.

      \item \textbf{True or False:} Only retrograde amnesia is observed in Alzheimer's Disease. \\
            \textit{Answer:} %False. (Both retrograde and anterograde amnesia are observed.)
\end{enumerate}
