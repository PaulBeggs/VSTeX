% Dementia Praecox
% A term used to describe a group of disorders characterized by a decline in cognitive function and emotional regulation, identified by Emil Kraepelin.

% Schizophrenia 
% A split of the mind from reality, not split personalities, identified by Eugen Bleuler.

% Positive Symptoms (Schizophrenia) 
% The presence of abnormal behaviors, (Escalation over normal functioning).

% Negative Symptoms (Schizophrenia) 
% The diminution or absence of normal behaviors.

% Cognitive Symptoms (Schizophrenia) 
% The presence of cognitive deficits.

% Hallucinations
% Perception of something that does not have a basis in reality; can be auditory, visual, tactile, olfactory, and gustatory.

% Delusions
% A false belief that is resistant to reason or confrontation with actual fact, often built on reality but misinterpreted.

% Referential Delusion 
% Believing that something is meant for you (e.g., the TV is talking to you).

% Persecutory Delusion
% Believing that someone is out to get you (e.g., the government is watching you).

% Grandiose Delusion 
% Believing that you are more important than you are (e.g., you are the king of the world).

% Control Delusion
% Believing that someone is controlling your thoughts or actions (e.g., the government is controlling your mind).

% Disorganized Speech 
% A pattern of incoherent or illogical speech that is difficult to follow.

% Derailment
% A pattern of speech in which the speaker jumps from one topic to another without any logical connection between them.

% Tangentiality
% A pattern of speech in which the speaker goes off on tangents and does not return to the main topic.

% Word Salad
% A pattern of speech in which the speaker uses words that are not related to each other in any meaningful way.

% Grossly disorganized behavior 
% A pattern of behavior that is inappropriate for the situation or that is not goal-directed.

% Catatonic behavior
% A pattern of behavior in which the person is unresponsive to the environment and does not move or speak.

% Affective Flattening
% Restricted range of emotional expression, including facial expressions, voice tone, and body language.

% Alogia
% A lack of speech or a decrease in the amount of thought, which is reflected in a decrease in speech produced.

% Avolition 
% A lack of motivation or a decrease in the ability to initiate and persist in activities.

% Anhedonia
% A lack of pleasure or a decrease in the ability to experience pleasure from activities that are normally pleasurable.

% Discordant Twins
% Twins that are discordant for a disorder, meaning that one twin has the disorder and the other does not.

% Seasonality Effect
% The idea that people born in late winter/early spring months are more likely to develop schizophrenia than those born in the summer months. 

% Latitude effect
% The idea that people who live in higher latitudes (further from the equator) are more likely to develop schizophrenia than those who live closer to the equator.

% Dorsolateral Prefrontal Cortex (DLPFC)
% The part of the brain that is responsible for organization, motor planning, regulation, self reflection, directed thought, and attention.

% Hypofrontality
% A decrease in metabolic activity in the prefrontal cortex, which is associated with negative symptoms of schizophrenia.

% D1 Receptors 
% A type of dopamine receptor that is involved in the regulation of movement and cognition.

% Dopamine Hypothesis 
% The idea that schizophrenia is caused by an overactivity of dopamine in the brain.

% Amphetamine Psychosis 
% DA Agonists (e.g., amphetamines) can cause positive symptoms in healthy people.

% Revised Dopamine Hypothesis
% Instead of an increase in dopamine, we see an increase in the number of dopamine receptors in the brains of people.

% Modified Dopamine Hypothesis
% The idea that schizophrenia is caused by an imbalance of dopamine in the brain, rather than an increase or decrease in dopamine levels.

% Tardive Dyskinesia
% Rapid involuntary movements of the tongue, jaw, trunk, or extremities developed in association with the use of neuroleptics (20\% - 30\% of long-term users). 

% PCP (Phencyclidine) 
% Glutmate antagonist that blocks NMDA receptors (a type of glutamate receptor).

% Locus Coeruleus (LC)
% The main source of norepinephrine in the brain. It is located in the pons and is involved in arousal, attention, and the sleep-wake cycle.

% \(\alpha\) Receptors
% Strong affinity for norepinephrine, but epinephrine can also bind to them (so far only found in the brain).

% \(\beta\) Receptors 
% Strong affinity for epinephrine, but norepinephrine can also bind to them.

% Doxazosin (Cardura) 
% \(\alpha\)-blocker that is used to treat high blood pressure and benign prostatic hyperplasia (BPH).

% Metaprolol (Lopressor) 
% \(\beta\)-blocker that is used to treat high blood pressure and heart failure.

% Atomoxetine (Strattera) 
% Reuptake inhibitor especially in the PFC used for ADHD treatment

% Guanfacine (Intuniv)
% \(\alpha\)-2A-agonist reduces activity in the sympathetic nervous system, ADHD treatment Binds to autoreceptors and improves PTSD along with anxiety.

% Propranolol (Inderal) 
% A non-selective \(\beta\)-blocker that is used to treat high blood pressure and anxiety.

% -----------

% Different ways to format questions:

% \begin{enumerate}
% % --- SHORT ANSWER ---
%     \item \textbf{Short Answer:} [Your short answer question here.] \\
%           \textit{Answer:} \\ [Be sure to leave this answer commented out, but do include it here.]

% % --- FILL IN THE BLANK ---
%     \item \textbf{Fill in the Blank:} [E.g.:] The central nervous system (CNS) is composed of \underline{\hspace{3cm}} and the \underline{\hspace{3cm}}. \\
%           % \textit{Answer:} [These answers will always be committed out, but be sure to include it in this space so the reviewer knows what the answer(s) to the question above. E.g.:] Brain; Spinal cord. 

% % --- T/F ---
%     \item \textbf{True or False:} [Answer goes here, try to give more False questions than True. This allows the reader to think about why the question is false, and correct it with the proper term. Avoid all true statements.] \\
%           \textit{Answer:} \\ % [Include the answer here in this commented out space. Be sure to leave the ``\textit{Answer:} \\'' there, too.]

% % --- MATCHING ---
%     \item \textbf{Matching:} [E.g.:] Match the following systems with their primary functions.
%     \begin{wordbox}
%         \begin{enumerate}[label=(\Alph*)]
%             \item Voluntary motor control and sensory input.
%             \item Involuntary control of the gastrointestinal system.
%             \item Involuntary control of smooth muscle and glands. 
%         \end{enumerate}
%     \end{wordbox}
%     \begin{enumerate}[label=(\alph*)]
%         \item Somatic Nervous System \quad \dotfill \quad \underline{\hspace{3cm}}\\[0.5em]
%         \item Autonomic Nervous System \quad \dotfill \quad \underline{\hspace{3cm}}\\[0.5em]
%         \item Enteric Nervous System \quad \dotfill \quad \underline{\hspace{3cm}}
%     \end{enumerate}

% % --- MULTIPLE CHOICE ---
%     \item \textbf{Multiple Choice:} [Phrase as a question, and try to use only similar and known vocabulary terms in for the choices. E.g.:] Which combination of symptoms is LEAST consistent with a diagnosis of acute bacterial meningitis?
%     \begin{tasks}[label=\textcolor{\documentTheme}{(\Alph*)}, item-format=\color{\documentTheme}, label-width=1.5em, item-indent=1.7em](2)
%         \task Fever, headache, and nuchal rigidity.
%         \task Photophobia, vomiting, and altered mental status.
%         \task None of the above
%     \end{tasks}
%     %\textit{Answer:} [Again, be sure to leave your answer commented out, but be sure to still include it. E.g.:] A.
% \end{enumerate}

\section*{4.1}
\begin{enumerate}[label=\textbf{Q4.1.\arabic*}]
% --- SHORT ANSWER ---
    % \item \textbf{Short Answer:} Who first identified the group of disorders known as Dementia Praecox? \\
    %       \textit{Answer:} \\ % Emil Kraepelin. 

    \item \textbf{Short Answer:} Describe the main difference between positive and negative symptoms of Schizophrenia. \\
          \textit{Answer:} \\ % Positive symptoms are the presence of abnormal behaviors (escalation over normal functioning), while negative symptoms are the diminution or absence of normal behaviors.

    \item \textbf{Short Answer:} What is the Dorsolateral Prefrontal Cortex (DLPFC) responsible for?
          \textit{Answer:} % It is responsible for organization, motor planning, regulation, self-reflection, directed thought, and attention.

    \item \textbf{Short Answer:} Other than the DLPFC, what are four other parts of the brain that are affected in people with Schizophrenia?
          \textit{Answer:} % Hippocampus, amygdala, lateral temporal cortex, and thalamus.
% --- FILL IN THE BLANK ---
    % \item \textbf{Fill in the Blank:} The term Schizophrenia, meaning a split of the mind from reality, was identified by \underline{\hspace{4cm}}.
    %       % \textit{Answer:} Eugen Bleuler.

    % \item \textbf{Fill in the Blank:} The \underline{\hspace{3cm}} is the idea that those who live in higher altitudes have a higher susceptibility to develop Schizophrenia than those who do not.
    %       % \textit{Answer:} Latitude effect

    % \item \textbf{Fill in the Blank:} A \underline{\hspace{3cm}} delusion is characterized by the belief that someone is controlling your thoughts or actions, while a \underline{\hspace{3cm}} delusion involves believing you are more important than you are.
    %       % \textit{Answer:} Control; Grandiose.

% --- T/F ---
    % \item \textbf{True or False:} Affective flattening is an example of a positive symptom of Schizophrenia. \\
    %       \textit{Answer:} \\ % False. Affective flattening is a negative symptom, characterized by a restricted range of emotional expression.

    % \item \textbf{True or False:} The Dopamine Hypothesis suggests that Schizophrenia is caused by a decrease in dopamine activity in the brain. \\
    %       \textit{Answer:} \\ % False. The Dopamine Hypothesis suggests Schizophrenia is caused by an overactivity of dopamine.

% --- MATCHING ---
    % \item \textbf{Matching:} Match the following terms related to disorganized speech with their descriptions.
    % \begin{wordbox}
    %     \begin{enumerate}[label=(\Alph*)]
    %         \item Jumping from one topic to another without logical connection.
    %         \item Using words unrelated to each other in any meaningful way.
    %         \item Going off on tangents and not returning to the main topic.
    %     \end{enumerate}
    % \end{wordbox}
    % \begin{enumerate}[label=(\alph*)]
    %     \item Word Salad \quad \dotfill \quad \underline{\hspace{3cm}}\\[0.5em]
    %     \item Tangentiality \quad \dotfill \quad \underline{\hspace{3cm}}\\[0.5em]
    %     \item Derailment \quad \dotfill \quad \underline{\hspace{3cm}}
    % \end{enumerate}
    % % Answers: a) B, b) C, c) A

    % \item \textbf{Matching:} Match the drug with its primary mechanism or use.
    % \begin{wordbox}
    %     \begin{enumerate}[label=(\Alph*)]
    %         \item \(\alpha\)-blocker for high blood pressure and BPH.
    %         \item Non-selective \(\beta\)-blocker for high blood pressure and anxiety.
    %         \item Reuptake inhibitor for ADHD, especially in the PFC.
    %         \item \(\alpha\)-2A-agonist for ADHD and PTSD.
    %     \end{enumerate}
    % \end{wordbox}
    % \begin{enumerate}[label=(\alph*)]
    %     \item Atomoxetine (Strattera) \quad \dotfill \quad \underline{\hspace{3cm}}\\[0.5em]
    %     \item Doxazosin (Cardura) \quad \dotfill \quad \underline{\hspace{3cm}}\\[0.5em]
    %     \item Propranolol (Inderal) \quad \dotfill \quad \underline{\hspace{3cm}}\\[0.5em]
    %     \item Guanfacine (Intuniv) \quad \dotfill \quad \underline{\hspace{3cm}}
    % \end{enumerate}
    % % Answers: a) C; b) A; c) B; d) D

% --- MULTIPLE CHOICE ---
    \item \textbf{Multiple Choice:} Which of the following best describes Hypofrontality?
    \begin{tasks}[label=\textcolor{\documentTheme}{(\Alph*)}, item-format=\color{\documentTheme}, label-width=1.5em, item-indent=1.7em](1)
        \task An increase in metabolic activity in the prefrontal cortex.
        \task A decrease in metabolic activity in the prefrontal cortex, associated with negative symptoms of schizophrenia.
        \task An overactivity of dopamine receptors in the frontal lobe.
        \task A blockage of NMDA receptors in the prefrontal cortex.
    \end{tasks}
    %\textit{Answer:} B.

    \item \textbf{Multiple Choice:} Tardive Dyskinesia is a potential side effect associated with long-term use of which type of medication?
    \begin{tasks}[label=\textcolor{\documentTheme}{(\Alph*)}, item-format=\color{\documentTheme}, label-width=1.5em, item-indent=1.7em](2)
        \task \(\beta\)-blockers
        \task Glutamate antagonists
        \task Neuroleptics
        \task Reuptake inhibitors
    \end{tasks}
    %\textit{Answer:} C.

    % \item \textbf{Short Answer:} What are Cytoarchitectual Abnormalities, and what are three main reasons for why they might occur (not including stress)? \\
    %       \textit{Answer:} \\ % They are abnormalities in the structure of the brain cells and their organization. Three reasons: Exposure to a virus, Vitamin D deficits, and pre or perinatal complications (e.g., low birth weight, hypoxia, and obsteric complications).

    % \item \textbf{Short Answer:} Describe the study that went into stress and its relationship with Schizophrenia. \\
    %       \textit{Answer:} \\ % Hutten and Niskanen studied people who were born in Finland during the war and found that they had a higher incidence of schizophrenia than those who were not born during the war. Possible Cause: When the mother's husbands were away, the mothers became much more stressed and their babies had a higher incidence of schizophrenia.

    % \item \textbf{True or False:} Chlorpromazine and Haloperidol are examples of major tranquilizers that increase the release of dopamine from the presynaptic membrane. \\
    %       \textit{Answer:} \\ % False. Chlorpromazine (Thorazine) and Haloperidol (Haldol) block postsynaptic DA receptors and block the release of DA from the presynaptic membrane.

    % \item \textbf{Multiple Choice:} The original Dopamine Hypothesis proposed that schizophrenia is primarily caused by:
    % \begin{tasks}[label=\textcolor{\documentTheme}{(\Alph*)}, item-format=\color{\documentTheme}, label-width=1.5em, item-indent=1.7em](1)
    %     \task A decrease in dopamine levels in the brain.
    %     \task An overactivity of dopamine in the brain.
    %     \task An increase in the number of dopamine receptors.
    %     \task An imbalance of dopamine between different brain regions.
    % \end{tasks}
    % %\textit{Answer:} B.

    % \item \textbf{Short Answer:} What is Amphetamine Psychosis, and how did it lend support to the Dopamine Hypothesis? \\
    %       \textit{Answer:} \\ % Amphetamine Psychosis is a state where DA agonists (like amphetamines) cause positive symptoms of schizophrenia in healthy people. This supported the Dopamine Hypothesis because it showed that increasing dopamine activity could induce schizophrenia-like symptoms, suggesting that an overactivity of dopamine might be a cause of schizophrenia.

    \item \textbf{Fill in the Blank:} The \underline{\hspace{3cm}} suggests an imbalance: \underline{\hspace{3cm}} dopamine in the meso\underline{\hspace{2cm}} system (leading to negative symptoms and \underline{\hspace{3cm}}) causes a(n) \underline{\hspace{3cm}} in dopamine in the meso\underline{\hspace{2cm}} system (leading to positive symptoms).
        % \textit{Answer:} Modified Dopamine Hypothesis; too little; mesocortical; hypofrontality; increase; mesolimbic.
\newpage

    \item \textbf{Multiple Choice:} Which of the following best describes the Revised Dopamine Hypothesis?
    \begin{tasks}[label=\textcolor{\documentTheme}{(\Alph*)}, item-format=\color{\documentTheme}, label-width=1.5em, item-indent=1.7em](1)
        \task Instead of an increase in dopamine, there was an increase in dopaminergic receptors.
        \task Instead of an increase of dopamine, there was a decrease of dopaminergic receptors.
        \task There is an imbalance of dopamine in the brain, rather than an increase or decrease in dopamine levels.
        \task Loss of neurons in the brain leads to a decrease of dopamine input in the DLPFC, which leads to hypofrontality.
    \end{tasks}
    %\textit{Answer:} A.

    \item \textbf{Short Answer:} According to the Modified Dopamine Hypothesis, how does a decrease of dopamine input in the DLPFC contribute to both negative and positive symptoms of schizophrenia? \\
        \textit{Answer:} \\ % A decrease of dopamine input in the DLPFC leads to hypofrontality, which is associated with negative symptoms. This hypofrontality also leads to a disinhibition of subcortical DA activity, resulting in an increase in dopamine in the mesolimbic system, which in turn leads to positive symptoms.
    % \item \textbf{Short Answer:} Name one antipsychotic and its function in the brain. \\
    %       \textit{Answer:} \\ % Aripiprazole (\textbf{Abilify}) These are partial agonists at the D2 receptor. They decrease dopamine in the mesolimbic system, which causes an increase in the mesocortical system. 
\end{enumerate}
\squigglyline
\newpage
\section*{4.2}
\begin{enumerate}[label=\textbf{Q4.2.\arabic*}]
    % \item \textbf{Short Answer:} What postmortem finding led to the inclusion of glutamate to the treatment conversation for Schizophrenia? \\
    %       \textit{Answer:} \\ % They have 1/2 as much glutamate in their brains as healthy people.
    
    % \item \textbf{Fill in the Blank:} DA antagonists directly affect glutamate by \underline{\hspace{3cm}} it.
    %       % \textit{Answer:}  increasing

    \item \textbf{Short Answer:} What is some evidence that support this finding in the previous problem? \\
          \textit{Answer:} \\ % People who take PCP (a glutamate antagonist) have Schizophrenic symptoms. Chronic use leads to a decrease in DA and metabolic activity in the DLPFC.

    % \item \textbf{Short Answer:} Name the two adrenergic receptors and their affinities. \\
    %       \textit{Answer:} \\ % \(\alpha\)-receptors: norepinephrine; \(\beta\)-receptors: epinephrine

    % \item \textbf{Fill in the Blank:} Both of these receptors increase \underline{\hspace{3cm}}.
    %       % \textit{Answer:} Blood pressure.

    \item \textbf{Short Answer:} As a direct result of the previous question, there are medications that block these receptors. What are they, and which receptor do they respectively target?
          \textit{Answer:} \\ % Doxazosin (\textbf{Cardura}) (\(\alpha\)-blocker ) and Metaprolol (\textbf{Lopressor}) (\(\beta\)-blocker).

        \item \textbf{Multiple Choice:} Amphetamines can lead to increased orientation and vigilance primarily by:
    \begin{tasks}[label=\textcolor{\documentTheme}{(\Alph*)}, item-format=\color{\documentTheme}, label-width=1.5em, item-indent=1.7em](1)
        \task Selectively increasing dopamine reuptake.
        \task Blocking norepinephrine reuptake and causing its transporters to work in reverse.
        \task Acting as an \(\alpha\)-2A-agonist in the PFC.
        \task Potentiating the effects of glutamate.
    \end{tasks}
    % \textit{Answer:} B.

    % \item \textbf{Fill in the Blank:} \textit{Atomoxetine} (\textbf{Strattera}) is a norepinephrine \underline{\hspace{3cm}} primarily used for the treatment of \underline{\hspace{4cm}}, with its effects being most notable in the \underline{\hspace{3cm}}. \\
    %     % \textit{Answer:} agonist; ADHD; PFC.

    % \item \textbf{True or False:} \textit{Guanfacine} (\textbf{Intuniv}) is an \(\alpha\)-2A-agonist that reduces activity in the sympathetic nervous system; it can help improve PTSD and anxiety. \\
    %     \textit{Answer:} \\ % True.

    % \item \textbf{Short Answer:} Besides its use in ADHD, Guanfacine (Intuniv) can also be beneficial for what other conditions due to its mechanism of binding to autoreceptors? \\
    %     \textit{Answer:} \\ % PTSD and anxiety.

    \item \textbf{Matching:} Match the older antidepressant class with its characteristic or primary drug example.
    \begin{wordbox}
        \begin{enumerate}[label=(\Alph*)]
            \item Associated with the ``cheese effect'' due to interaction with pressor amines.
            \item Blocks reuptake of NE and 5-HT, with side effects like dry mouth and drowsiness.
            \item Example: Isocarboxazid (Marplan).
            \item Example: Amitriptyline (Elavil).
        \end{enumerate}
    \end{wordbox}
    \begin{enumerate}[label=(\alph*)]
        \item Tricyclic Antidepressants \quad \dotfill \quad \underline{\hspace{3cm}}\\[0.5em]
        \item MAO Inhibitors \quad \dotfill \quad \underline{\hspace{3cm}}
    \end{enumerate}
    % \textit{Answer:} a) B, D; b) A, C

    \item \textbf{Short Answer:} What are four common side effects associated with Tricyclic Antidepressants that patients should be aware of, particularly concerning overdose risk? \\
        \textit{Answer:} \\ % Sleep disturbances, sexual dysfunction, changes in eating habits (appetite/weight gain), and they are easy to overdose on. (Others include: dry mouth, blurred vision, constipation, etc.)

    % \item \textbf{True or False:} Propranolol (Inderal) is a selective \(\alpha\)-blocker used primarily for its sleep effects. \\
    %     \textit{Answer:} \\ % False. Propranolol (Inderal) is a non-selective \(\beta\)-blocker used to treat high blood pressure and anxiety.

    % \item \textbf{Fill in the Blank:} The ``cheese effect'' is a serious adverse reaction associated with \underline{\hspace{3cm}} inhibitors, occurring when individuals consume foods rich in \underline{\hspace{3cm}} amines, which are normally broken down by MAO. \\
    %     % \textit{Answer:} MAO; pressor.

    \item \textbf{Fill in the Diagram:} Fill in each chemical for the synthesis of Norepinephrine: \\
\end{enumerate}
\vspace*{-0.5cm}
    \noindent Tyrosine \(\xrightarrow[\underline{\hspace{2cm}}]{\text{Tyrosine-}}\) \underline{\hspace{2cm}} \(\xrightarrow[\underline{\hspace{2cm}}]{}\) \underline{\hspace{2cm}} \(\xrightarrow[\underline{\hspace{3cm}}]{\text{Dopamine-}}\) Norepinephrine.
    % \textit{Answer:} Tyrosine hydroxylase; Dihydroxyphenylalanine (DOPA); Dopamine; Dopamine-\(\beta\)-hydroxylase.
\begin{enumerate}[label = \textbf{Q4.2.16}]
    \item \textbf{Fill in the Diagram:} Fill in each chemical for the metabolism of Norepinephrine: \\
\end{enumerate}
\vspace*{-0.5cm}
    Norepinephrine \(\xrightarrow[\underline{\hspace{4cm}}]{}\) Normetanephrine \(\xrightarrow[\underline{\hspace{4cm}}]{}\) \\
    Normetanephrine aldehyde \(\xrightarrow[\underline{\hspace{4cm}}]{\text{Aldehyde}}\) \underline{\hspace{4cm}} \\
    \phantom{Normetanephrine aldel}(OR \(\xrightarrow[\underline{\hspace{4cm}}]{\text{Aldehyde}}\) \underline{\hspace{4cm}})
    % \textit{Answers:} MAO; COMT; Aldehyde dehydrogenase; VMA. (OR aldehyde reductase; MHPG)
\newpage
\begin{enumerate}[label = \textbf{Q4.2.\arabic*}]
    \setcounter{enumi}{16}
    \item \textbf{Multiple Choice:} Which of the following symptoms is LEAST likely to be considered a primary symptom of Major Depressive Disorder?
    \begin{tasks}[label=\textcolor{\documentTheme}{(\Alph*)}, item-format=\color{\documentTheme}, label-width=1.5em, item-indent=1.7em](1)
        \task Avolition and apathy.
        \task Grandiose delusions.
        \task Disturbed sleep and eating patterns.
        \task Depressed or irritable mood.
    \end{tasks}
    % \textit{Answer:} B.

    \item \textbf{Short Answer:} How did the observation of patients taking isoniazid for tuberculosis contribute to the development of antidepressant medications? (Be sure to name the drug it helped make.) \\
        \textit{Answer:} \\ % Patients taking isoniazid (an MAO inhibitor) for tuberculosis showed improved mood, which led to the investigation and development of MAO inhibitors like Isocarboxazid (Marplan) as antidepressants.

    \item \textbf{Fill in the Diagram:} Fill in each chemical for the synthesis of 5-HT: \\
\end{enumerate}
\vspace*{-0.5cm}
    \noindent Trypotophan \(\xrightarrow[\underline{\hspace{3cm}}]{\text{Tryptophan-}}\) \underline{\hspace{3cm}} \(\xrightarrow[\underline{\hspace{3cm}}]{\text{5-HTP}}\) 5-HT.
    % \textit{Answer:} Trpyotphan Hydroxylase (OH); 5-HTP; 5-HTP Decarboxylase (COOH) 
\begin{enumerate}[label = \textbf{Q4.2.20}]
    \item \textbf{Fill in the Diagram:} Fill in each chemical for the metabolism of 5-HT: \\
\end{enumerate}
\vspace*{-0.5cm}
    5-HT \(\xrightarrow[\underline{\hspace{2cm}}]{}\) \underline{\hspace{4cm}} \(\xrightarrow[\text{\underline{\hspace{3cm}}}]{\text{Aldehyde}}\) \underline{\hspace{3cm}} (5-HIAA)
    % \textit{Answer:} MAO; 5-hydroxy-indolacet-aldehyde; Aldehyde dehydrogenase; 5-hydroxyindolacetic acid

\begin{enumerate}[label = \textbf{Q4.2.\arabic*}]
    \setcounter{enumi}{20}
    \item \textbf{Short Answer:} What are the three functions of 5-HT? \\
          \textit{Answer:} \\ % Mood, sleep and dreams, and hallucinations from Mescaline and psilocybin (from cactus and mushrooms, respectively).
\end{enumerate}

\squigglyline
\newpage 
\section*{4.3}
\begin{enumerate}[label = \textbf{Q4.3.\arabic*}]
    \item \textbf{Short Answer:} Why did researchers think that DA was not involved with depression? \\
          \textit{Answer:} \\ % They found that DA agonists only elevated the mood for people without depression, but not for those with. 

    \item \textbf{Multiple Choice:} This drug is acts as a selective D3 recptor agonist.
    \begin{tasks}[label=\textcolor{\documentTheme}{(\Alph*)}, item-format=\color{\documentTheme}, label-width=1.5em, item-indent=1.7em](2)
        \task \textit{Buproprion} (\textbf{Wellbutrin})
        \task \textit{Pramipexole} (\textbf{Mirapex})
        \task \textit{Geniprone} (\textbf{Exxua})
        \task \textit{Nefazodone} (\textbf{Serzone})
     \end{tasks}

     \item \textbf{Short Answer:} List the functions of the other 3 drugs that were listed.  \\
           \textit{Answer:} \\ % Wellbutrin is a weak NDRI (norepinephrine and dopamine reuptake inhibitor) used to treat smoking and depression; Geniprone is a glutamate antagonist that fine-tunes production; Nefazodone led to liver problems, so it was discontinued. 

    \item \textbf{Matching:} Assign the correct drug with its corresponding drug class.
    \begin{wordbox}
        \begin{enumerate}[label = (\alph*)]
            \item \cyanit{Paroxetine} (\textbf{Paxil}) % SSRI
            \item \cyanit{Duloxetine} (\textbf{Cymbalta}) % SNRI
            \item \cyanit{Sertraline} (\textbf{Zoloft}) % SSRI
            \item \cyanit{Venlafaxine} (\textbf{Effexor}) % SNDRI
            \item \cyanit{Escitalopram} (\textbf{Lexapro, Cipralex}) % SSRI
            \item \textit{Ruoxinlin} (\textbf{Ansofaxine}) % SNDRI
            \item \cyanit{Desvenlafaxine} (\textbf{Pristiq}) % SNRI
        \end{enumerate}
    \end{wordbox}
    \begin{enumerate}[label=(\alph*)]
        \item SSRIs \quad \dotfill \quad \underline{\hspace{3cm}}\\[0.5em]
        \item SNRIS \quad \dotfill \quad \underline{\hspace{3cm}}\\[0.5em]
        \item Triple uptake inhibitors (SNDRIs) \quad \dotfill \quad \underline{\hspace{3cm}}
    \end{enumerate}
    % \textit{Answers:} A) (a), (c), (e) B) (b), (g) C) (d), (f)
\end{enumerate}

\squigglyline
\newpage
\section*{4.4}
\begin{enumerate}[label = \textbf{Q4.4.\arabic*}]
    \item \textbf{Fill in the Blank:} In the following table, put a checkmark next to the chronotype that fits the given characteristic.
    \begin{table}[htbp]
    \centering
    \begin{tabular}{cccc}
        \toprule
        \textbf{\underline{Characteristics}} & \textbf{Larks} & \textbf{Neither} & \textbf{Owls} \\ \midrule
        \textbf{Better on IQ Tests:} & & &  \\ \midrule
        \textbf{More Likely to Procrastinate:} & & &  \\ \midrule
        \textbf{Less Conscientious:} & & &  \\ \midrule
        \textbf{More Open:} & & &  \\ \midrule
        \textbf{Better at Baseball:} &  & &  \\ \midrule
        \textbf{More Sexual Partners:} & & &  \\ \midrule
        \textbf{More Likely to be Unfaithful:} & & &  \\ \midrule
        \textbf{Poorer diets:} & & &  \\ \midrule
        \textbf{Smoke and Drink More:} & & &  \\ \midrule
        \textbf{More Positive Affect:} &  & & \\ \midrule
        \textbf{More Social Jet Lag:} & & &  \\
        \bottomrule
    \end{tabular}
\end{table}
% \textit{Answers:} All should have a checkmark in the ``Owls'' column, except for ``Better on IQ Tests'', ``More positive affect'', and ``Better at baseball'', which should be in the ``Larks'' column. 

    \item \textbf{Multiple Choice:} What is the suprachiasmatic nucleus (SCN) responsible for?
    \begin{tasks}[label=\textcolor{\documentTheme}{(\Alph*)}, item-format=\color{\documentTheme}, label-width=1.5em, item-indent=1.7em](1)
        \task Regulating the sleep-wake cycle.
        \task Producing melatonin.
        \task Controlling the release of norepinephrine.
        \task All of the above.
    \end{tasks}
    % \textit{Answer:} D.
    
    \item \textbf{Fill in the Blank:} A \underline{\hspace{3cm}} is a stimulus that helps to regulate the circadian rhythm.
          % \textit{Answer:} Zeitgeber.

    \item \textbf{Short Answer:} Where is the Suprachiasmatic Nucleus (SCN) located? (Hint: just read the name): \\
          \textit{Answer:} \\ % It is located in the hypothalamus, just above the optic chiasm.

    \item \textbf{Short Answer:} What are the three kinds of rhythm cycles, and what are their respective time frames? \\
          \textit{Answer:} \\ % Circadian (24 hours), ultradian (less than 24 hours), and infradian (more than 24 hours).

    \item \textbf{True or False:} Sleep is a state of unconsciousness. \\
          \textit{Answer:} \\ % False; Sleep is behaviorally defined as a state of reduced activity, species specific posture, and decreased response to stimuli.

    \item \textbf{Fill in the Blank:} A \underline{\hspace{3cm}} is a method of recording various physiological measures during sleep. It consists of an \underline{\hspace{3cm}}, an \underline{\hspace{3cm}}, and an \underline{\hspace{3cm}}.
          % \textit{Answer:} Polysomnograph; electroencephalography (EEG); electromyography (EMG); electrooculography (EOG).

    \item \textbf{Fill in the Blank:} For the following table, fill in each section that is left blank with the correct term. (The first one is filled out for you.) \\
    \begin{table}[htbp]
    \centering
    \begin{tabularx}{\linewidth}{l 
                                >{\raggedright\arraybackslash}X 
                                >{\raggedright\arraybackslash}X 
                                >{\raggedright\arraybackslash}X 
                                >{\raggedright\arraybackslash}X}
        \toprule
        \textbf{Name} & \textbf{Frequency} & \textbf{Amplitude} & \textbf{Description} & \textbf{State} \\
        \midrule
        \textbf{Beta \(\beta\)}  & 12 -- 50 Hz (variable) & Lower and Variable & Desynchronous  & Awake and Paying Attention\\[25pt]
        \underline{\hspace{3cm}} & \underline{\hspace{3cm}} & 50 Microvolts & \underline{\hspace{3cm}} & \underline{\hspace{3cm}}\\[25pt]
        \underline{\hspace{3cm}} & \underline{\hspace{3cm}} & \underline{\hspace{3cm}} & \underline{\hspace{3cm}} & \underline{\hspace{3cm}}\\[25pt]
        \underline{\hspace{3cm}} & \underline{\hspace{3cm}} & \underline{\hspace{3cm}} & \underline{\hspace{3cm}} & \underline{\hspace{3cm}}\\
        \bottomrule
    \end{tabularx}
    \caption{Summary of EEG Wave Characteristics}
    \label{tab:EEGwaves}
\end{table}
    % \textit{Answers:}Alpha: 8 -- 12 Hz, 50 microvolts, synchronous, relaxed wakefulness (eye closed, not fully attending, and usually largest occipitally). Theta: 3.5 -- 7.5 Hz, low in voltage microvolts, synchronous, drowsy, light sleep (Stage 1). Delta: 1 -- 3.5 Hz, 20 -- 200 microvolts, synchronous, deep sleep (Stages 3 and 4).

    \item \textbf{Fill in the Blank:} What is the physiological definition of wakefulness? Predominantly \underline{\hspace{3cm}} and \underline{\hspace{3cm}} waves, \underline{\hspace{3cm}} muscle tone, and \underline{\hspace{3cm}} eye movements.
          % \textit{Answer:} Predominantly alpha and beta waves when drowsy, high muscle tone, variability in eye movements.

    \item \textbf{Fill in the Blank:} What is the physiological definition of sleep? Predominantly \underline{\hspace{3cm}} and \underline{\hspace{3cm}} waves, \underline{\hspace{3cm}} muscle tone, and \underline{\hspace{3cm}} eye movements.
          % \textit{Answer:} Predominantly theta and delta waves, low muscle tone, and slow eye movements.

    \item \textbf{Fill in the Blank:} What is the physiological definition of REM sleep? EEG: \underline{\hspace{3cm}}; EMG: \underline{\hspace{3cm}}; EOG: \underline{\hspace{3cm}} and \underline{\hspace{3cm}}, which gets its name because brain activity is similar to that of being awake.
             %\textit{Answer:} \\ EEG: Low voltage, random, fast with sawtooth waves. EMG: atonia (loss of muscle tone). EOG: Rapid eye movements. Paradoxical sleep (physiologically awake, but behaviorally asleep). 
\newpage
    \item \textbf{Matching:} Match the following NREM stages with their characteristics (some answers may be used more than once).
    \begin{wordbox}
\begin{multicols}{2}
        \begin{enumerate}[label=(\Alph*)]
            \item Light sleep
            \item Low EMG activity
            \item slow rolling eye movements
            \item Sleep spindles
            \item 10\%-15\% of sleep
            \item 45\%-55\% of sleep
            \item Delta waves
            \item 5\%-10\% of sleep 
            \item K-complexes
            \item Medium muscle tone
            \item Lowest muscle tone
            \item Highest muscle tone
        \end{enumerate}
\end{multicols}
    \end{wordbox}
    \begin{enumerate}[label=(\alph*)]
        \item N1 \quad \dotfill \quad \underline{\hspace{3cm}}\\[0.5em]
        \item N2 \quad \dotfill \quad \underline{\hspace{3cm}}\\[0.5em]
        \item N3 \quad \dotfill \quad \underline{\hspace{3cm}}
    \end{enumerate}
    % \textit{Answers:} N1: a, b, c, h, l, N2: a, b, c, d, f, i, j, N3: a, b, c, e, g, k
\end{enumerate}
