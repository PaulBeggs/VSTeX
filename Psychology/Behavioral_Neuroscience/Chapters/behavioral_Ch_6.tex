\begin{center}
    \textbf{NEW NOTES FOR 02/26/25} \\
    \hrulefill
\end{center}

\section{The Neuron}

\begin{coloredlist}
    \item Definition: Basic information processing unit of the NS.
    \item Similarities to an animal cell:
    \begin{coloredlist}
        \item \cyanit{Cell membrane}: Separates the inside of the cell from the outside environment.
        \item \cyanit{Nucleus}: Contains the genetic material of the cell.
        \item \cyanit{Organells}: Carry out the basic functions of the cell.
        \begin{coloredlist}
            \item \cyanit{Mitochondria}: Produce energy for the cell.
            \item \cyanit{Endoplasmic Reticulum}: Synthesizes proteins.
            \item \cyanit{Golgi Apparatus}: Packages proteins for transport.
            \item \cyanit{Lysosomes}: Break down waste products.
        \end{coloredlist}
        \item Basic cellular processes.
    \end{coloredlist}
    \item Differences:
    \begin{coloredlist}
        \item Special ``morphology'' (shape).
        \item Communicate through an electrochemical process.
    \end{coloredlist}
\end{coloredlist}

\subsection{Structure of the Neuron}
(Mostly a recap of \hyperref[Terms]{Terms})
\begin{coloredlist}
    \item \cyanit{Soma}
    \item \cyanit{Dendrite}
    \item \cyanit{Axon}
    \item \cyanit{Terminal Arboriza} -- Branches at the end of the axon.
    \item \cyanit{Terminal Buttons} -- End of the terminal arboriza.
    \item \cyanit{Axon Hillock}
    \item \cyanit{Myelin}
    \begin{coloredlist}
        \item Not all axons have it.
        \item Glial cells / 70\% Lipid / Nodes of Ranvier.
        \item Mutliple Sclerosis (MS) -- Demyelination.
    \end{coloredlist}
\end{coloredlist}

\subsection{Support cells in the Nervous System}

\begin{coloredlist}
    \item Glia/Glial Cells/ Neuroglia -- Support cells.
    \begin{coloredlist}
        \item Capable of cell division after birth/communication.
        \item Make up half of the volume, but are 10-50 times more numerous. (The other half is made up of neurons.)
        \item CNS:
        \begin{coloredlist}
            \item \cyanit{Macroglia} -- Large glial cells.
            \begin{coloredlist}
                \item \cyanit{Astrocytes} -- Star-shaped cells that provide physical support to neurons, clean up debris, and provide nutrients to neurons.
                \begin{coloredlist}
                    \item \textit{Note} that these cells do not help neurons grow when they are damaged. In fact, they inhibit growth by proliferating and forming a scar.
                \end{coloredlist}
                \item \cyanit{Oligodendrocytes} -- ``few branches (in contrast to Astrocytes)'' -- \textit{Form} myelin sheath around multiple axons in the CNS. 
            \end{coloredlist}
            \item \cyanit{Microglia} -- Small cells that remove debris from injured or dead cells.
            \item \cyanit{Ependymal Glia} -- Line the ventricles of the brain and spinal cord. (Remember the CSF?)
        \end{coloredlist}
        \item PNS:
        \begin{coloredlist}
            \item \cyanit{Satellite Cells} -- Provide nutrients and physical support to neurons.
            \item \cyanit{Schwann Cells} -- Form myelin sheath around axons in the PNS. These cells are monogamists; they wrap their arms around one axon.
            \begin{coloredlist}
                \item Neuronal Regeneration. 
            \end{coloredlist}
        \end{coloredlist}
        \item The Myelin Sheath is composed of Oligodendrocytes in the CNS and Schwann Cells in the PNS.
        \item \cyanit{Phagocytosis} -- When an injury occurs, the glial cells divide and eat the dead cells. (Done by Microglia and Schwann Cells.)
        \item Maintenance of Internal Consistency.
        \begin{coloredlist}
            \item When neurons undergo rapid firing, they release potassium ions. Astrocytes absorb these ions to maintain the internal consistency of the neuron, and dump them into the blood stream.
        \end{coloredlist}
    \end{coloredlist}
\end{coloredlist}

\begin{center}
    \textbf{NEW NOTES FOR 02/28/25} \\
    \hrulefill
\end{center}

\subsection{Are Glial Cells Contributing to Alzheimer's Disease?}

\begin{coloredlist}
    \item Normally,
    \begin{coloredlist}
        \item Beta amyloid cleared away through microglia.
    \end{coloredlist}
    \item IF beta amyloid builds up too much, Tau INSIDE cells builds up.
    \item This leads to inflammation, which maybe leads to the problems of Alzheimer's.
\end{coloredlist}

\section{Different Kinds of Neurons}

\begin{coloredlist}
    \item Based on Structure
    \item Based on Function
\end{coloredlist}

\subsection{Structural Classification of Neurons}

\begin{coloredlist}
    \item \cyanit{Unipolar/Pseudounipolar}
    \begin{coloredlist}
        \item \textit{The difference:} The axon and dendrite are fused together.
    \end{coloredlist}
    \item \cyanit{Bipolar}
    \item \cyanit{Multipolar}
\end{coloredlist}

\subsection{Functional Classification of Neurons}

\begin{coloredlist}
    \item Sensory Neurons (Afferent)
    \begin{coloredlist}
        \item Carry information from the sensory receptors to the CNS.
        \item Unipolar.
        \item ``Afferent'' -- ``bearing or conducting inward''
    \end{coloredlist}
    \item Interneurons
    \item Motor Neurons (Efferent)
    \begin{coloredlist}
        \item Carry information from the CNS to the muscles and glands.
        \item Multipolar.
        \item ``Efferent'' -- ``conducting outward''
    \end{coloredlist}
    \item Remember: \(\text{Ad = towards} \qquad \text{Ex = from} \qquad \text{Ferro = I carry}.\)
\end{coloredlist}

\section{Neural Communication}

\begin{coloredlist}
    \item \textbf{2 Systems of Neuronal Communication:}
    \begin{coloredlist}
        \item \cyanit{Binary} -- All or none (literally only 2 options).
        \item \cyanit{Analogue} --Graded, matter or degree.
    \end{coloredlist}
\end{coloredlist}

\begin{center}
    \textbf{NEW NOTES FOR 03/03/25 (kinda)} \\
    We added a lot to the notes that we already had, so there is new material springled throughout this section. \\
    \hrulefill
\end{center}

\subsection{Binary System (``Off'' and ``On'')}

\begin{coloredlist}
    \item \cyanit{The Resting Membrane Potential (RMP)}: ``OFF'' 
    \begin{coloredlist}
        \item \hyperref[resting potential]{Click here for a diagram.}
        \item \textit{Note:} Where the arrows land on either side of the cell membrane is supposed to represent the relative permeability of the cell membrane to different ions.
        \item -70 mV (relative to the outside).
        \item Understand the cell membrane.
        \begin{coloredlist}
            \item \cyanit{Phospholipid Bilayer} -- Hydrophobic tails and hydrophilic heads.
            \item \cyanit{Semipermeability} -- Some things can cross, others cannot.
            \begin{coloredlist}
                \item Lipid, lipid soluble, small, and neutral.
            \end{coloredlist}
            \item \cyanit{Embedded Proteins} -- Channels and pumps. 
            \begin{coloredlist}
                \item \textbf{4 Jobs We Care About:}
                \begin{coloredlist}
                    \item \cyanit{Receptors}
                    \begin{coloredlist}
                        \item High specificity and affinity.
                        \item ``Places where things can bind to the cell and cause a change.''
                    \end{coloredlist}
                    \item \cyanit{Channels}.
                    \begin{coloredlist}
                        \item \cyanit{Gated Channels}: 
                        \begin{coloredlist}
                            \item Passive movement. The cell itself does not expel any energy to move the ions.
                            \item Chemical (ligand) gated channels.
                            \item Voltage gated channels.
                        \end{coloredlist}
                    \end{coloredlist}
                    \item \cyanit{Pumps} -- Active transport.
                    \item \cyanit{Enzymes} -- Facilitates chemical reactions.
                    \begin{coloredlist}
                        \item Breaking neurochemicals down or putting them back together.
                    \end{coloredlist}
                \end{coloredlist}
            \end{coloredlist}
        \end{coloredlist}
        \item \textbf{3 Determinants of the RMP}
        \begin{coloredlist}
            \item \cyanit{Differential Permeability} -- The cell membrane is more permeable to some ions than others.
            \begin{coloredlist}
                \item For sodium, the membrane only allows a trickle of \(\text{Na}^{+}\) into the cell.
                \item Conversely, the membrane is more permeable to \(\text{K}^{+}\) and \(\text{CL}^{-}\). (\(\text{K}^{+}\) is the most permeable.)
            \end{coloredlist}
            \item \textbf{Driving Forces}:
            \begin{coloredlist}
                \item \cyanit{Diffusion} -- Ions move from high to low concentration (Concentration Gradient).
                \begin{coloredlist}
                    \item \textit{Note:} The cell membrane is more permeable to potassium ions than sodium ions.
                \end{coloredlist}
                \item \cyanit{Electrostatic Pressure} -- Ions move towards the opposite charge (Electrical Gradient).
                \item \cyanit{Equilibrium Potential} -- The charge the ion ``perfers'' if it were the only one and could pass freely through the membrane.
                \begin{coloredlist}
                    \item This answers the question: ``Why doesn't the cell get more and more negative?''
                    \item Driving force in (influx) = Driving force out (efflux).
                    \item \(\text{K}^+ = -80 \text{ mV}\) and \(\text{Na}^+ = +55 \text{ mV}\).
                \end{coloredlist}
            \end{coloredlist}
            \item \cyanit{Sodium-Potassium Pump} (\(\text{Na}^{+}\)/\(\text{K}^{+}\) Pump)
            \begin{coloredlist}
                \item 3 \(\text{Na}^{+}\) out for every 2 \(\text{K}^{+}\) in.
                \item Costs 1 ATP.
            \end{coloredlist}
        \end{coloredlist}
    \end{coloredlist}
    \item \cyanit{The Action Potential (AP)}: ``ON'' 
    \begin{coloredlist}
        \item \hyperref[action potential]{Click here for a diagram.}
        \begin{coloredlist}
            \item \textit{Note:} The above diagram neglects to show that there can be \textit{failed} attempts at an action potential wherein the threshold is not reached. In these cases, the charge of the cell can increase or decrease, but if it does not reach -55 mV, it will quickly settle back to its resting potential of -70 mV.
        \end{coloredlist}
        \item +40 mV (relative to the outside).
        \item Thus, during an action potential, there is a total of 110 mV difference between the inside and outside the cell.

        \begin{center}
            \textbf{NEW NOTES FOR 03/05/25} \\
            \hrulefill
        \end{center}
        \item \textbf{Electrical Current}
        \begin{coloredlist}
            \item \cyanit{Depolarization}
            \begin{coloredlist}
                \item A little 
                \item A lot
                \begin{coloredlist}
                    \item \cyanit{Threshold of Excitation} = +15 mV.
                    \item Action potential
                \end{coloredlist}
                \item Even more.
            \end{coloredlist}
            \item \cyanit{Repolarization}
            \begin{coloredlist}
                \item A little
                \item A lot
                \item Even more.
            \end{coloredlist}
        \end{coloredlist}
        \item \cyanit{Refractory Period}
        \begin{coloredlist}
            \item Cell is resistant to reexciation for a period after the AP peak.
            \begin{coloredlist}
                \item \cyanit{Absolute Refractory Period} -- No amount of stimulation will cause another AP.
                \item \cyanit{Relative Refractory Period} -- A stronger than normal stimulus is required to cause another AP. (This is because of hyperpolarization.)
            \end{coloredlist}
        \end{coloredlist}
        \item \textbf{Three Questions:}
        \begin{coloredlist}
            \item Why don't the \(\text{Na}^{+}\) channels reopen during repolarization?
            \begin{coloredlist}
                \item \textit{Answer:} Because of the refractory period.
            \end{coloredlist}
            \item How can an all or none signal convey analog information?
            \begin{coloredlist}
                \item \textit{Answer:} The frequency of the APs can convey the intensity of the stimulus.
                \item \cyanit{Rate Law} -- Variations in the intensity of a stimulus are represented by variations in the rate of firing.
            \end{coloredlist}
            \item Where does that signal come from that meets the threshold of excitation?
            \begin{coloredlist}
                \item Refer to \hyperlink{sec:electrical activity}{Conduction of Electrical Activity}.
            \end{coloredlist}
        \end{coloredlist}
    \end{coloredlist}
\end{coloredlist}

\hypertarget{sec:electrical activity}{}
\section{Conduction of Electrical Activity}

\subsection{Conduction of Hyper and (subthreshold) Depolarizations}

\begin{coloredlist}
    \item \cyanit{Decremental Conduction} -- The further the signal travels, the weaker it gets. (cable properties)
    \begin{coloredlist}
        \item Analogue communication.
        \begin{coloredlist}
            \item Degrading because for resistance and leakage.
        \end{coloredlist}
        \item \cyanit{Passive Conduction} -- No energy is expended.
    \end{coloredlist}
\end{coloredlist}

\subsection{Conduction of Action Potential in Unmyelinated Axons}

\begin{center}
    \textbf{NEW NOTES FOR 03/07/25} \\
    \hrulefill
\end{center}

\begin{coloredlist}
    \item \cyanit{All or Nothing Law} -- AP occurs or not once triggered, always the same size.
    \item \cyanit{Active Regeneration} -- The AP is regenerated at each point along the axon.
    \begin{coloredlist}
        \item Takes a lot of time (\(<\) 1 - 10 meters/second).
        \item Takes a lot of energy.
    \end{coloredlist}
\end{coloredlist}

\subsection{Conduction of Action Potential in Myelinated Axons}

\begin{coloredlist}
    \item \cyanit{Saltatory Conduction} = ``To jump'' -- AP jumps from node to node.
    \begin{coloredlist}
        % \item \textit{NOTE:} The action potential is not actually ``jumping.'' In the mylinated portions, there is decremental conduction, and at the nodes, there is active regeneration with the Na\(^+\) channels.
        \item Decremental conduction in the myelinated portions.
        \begin{coloredlist}
            \item No extracellular fluid.
            \item Almost absent Na\(^+\) channels.
        \end{coloredlist}
        \item Active regeneration at Nodes of Ranvier.
        \begin{coloredlist}
            \item High density of Na\(^+\) channels.
            \item AP is regenerated.
        \end{coloredlist}
        \item \textbf{Advantages:}
        \begin{coloredlist}
            \item Economic
            \begin{coloredlist}
                \item Much less work for the Na\(^+\)/K\(^+\) pump.
            \end{coloredlist}
            \item Speed
            \begin{coloredlist}
                \item in excess of 100 m/s (225 mph) (not as fast as electricty's 300 million m/s).
            \end{coloredlist}
            \item So why not evolve 1 long myelin sheath?
            \begin{coloredlist}
                \item \textit{Answer:} The AP would be too weak from the decremental conduction by the time it reached the end.
            \end{coloredlist}
        \end{coloredlist}
        \item Think of how this applies to multiple sclerosis: The myelin sheath is destroyed, and the AP can no longer jump from node to node. This results in a loss of sensation and motor control.
    \end{coloredlist}
\end{coloredlist}
\section{Conversion from Electrical to Chemical Signals}

\begin{coloredlist}
    \item Occurs at the synapse (Greek: ``syn'' = together, ``haptein'' = to clasp).
    \begin{coloredlist}
        \item Synaptic cleft (200 angstroms (\(\mathring{\text{A}}\)) across). \textit{Note:} 10\(^7\) \(\mathring{\text{A}}\) = 1 mm.
        \item Pre-synaptic membrane
        \begin{coloredlist}
            \item \cyanit{Golgi bodies} -- Synthesize neurotransmitters and package them into vesicles.
            \item \cyanit{Synaptic vesicles} -- Contain neurotransmitters.
            \begin{coloredlist}
                \item In the pre-synaptic cell, the golgi bodies 
            \end{coloredlist}
            \item \cyanit{Docking proteins} -- Hold the vesicles in place.
            \begin{coloredlist}
                \item Full synaptic vesicles migrate to membrane and attach.
            \end{coloredlist}
            \item Voltage gated \cyanit{Ca\(^{+2}\) channels} open.
            \item Once the AP arrives at the synapse, the docking proteins release the vesicles and the neurotransmitters are released into the synaptic cleft (this is due to the Ca\(^{+2}\) channels opening).
        \end{coloredlist}
        \item Post-synaptic membrane
        \begin{coloredlist}
            \item \cyanit{Receptors} -- Bind to the neurotransmitters.
        \end{coloredlist}
    \end{coloredlist}
\end{coloredlist}
\newpage
\begin{center}
    \textbf{NEW NOTES FOR 03/10/25} \\
    \hrulefill
\end{center}

\section{Conversion from Chemical Back to Electrical Signal}

\subsection{Two Kinds of Post Synaptic Potentials}

\begin{coloredlist}
    \item \cyanit{Excitatory Post Synaptic Potentials} (ESPSs)
    \begin{coloredlist}
        \item Bring the cell closer to firing.
        \item i.e., opening of Na\(^+\) channels.
    \end{coloredlist}
    \item \cyanit{Inhibitory Post Synaptic Potentials} (ISPSs)
    \begin{coloredlist}
        \item Take the cell further from firing.
        \item i.e., opening of K\(^+\) channels (and potassium leaves).
    \end{coloredlist}
    \item Thus, post synatpic into action potential by summing up of the EPSPs and IPSPs at the axon hillock.
\end{coloredlist}

\subsection{What Happens to Excess or Used Neurotransmitters?}

\begin{coloredlist}
    \item \textbf{Three things can occur:}
    \begin{coloredlist}
        \item \cyanit{Active Reuptake} -- The neurotransmitter is taken back up into the pre-synaptic cell.
        \item \cyanit{Metabolism} -- The neurotransmitter is broken down by enzymes.
        \item \cyanit{Bound to Autoreceptors} -- The neurotransmitter binds to autoreceptors on the pre-synaptic cell.
        \begin{coloredlist}
            \item This inhibits the release of more neurotransmitters.
        \end{coloredlist}
    \end{coloredlist}
\end{coloredlist}

\subsection{Two Types of Chemically Gated Channels}

2 Kinds of Synapse: \cyanit{ionotropic} and \cyanit{metabotropic}. 

\subsection{\cyanit{Ionotropic Synapse}}

\begin{coloredlist}
    % \item One neuron is transmitting a signal to another neuron.
    \item No change in metabolism. (No ATM expended.)
    \item Direct change of ions.
    \item Fixed duration (rapid and short).
    \item 1 neurotransmitter binds to 1 receptor.
    \item \textbf{Example:} Acetylcholine (ACh) binds to a receptor and opens a Na\(^+\) channel.
\end{coloredlist}

\subsection{\cyanit{Metabotropic Synapse}}

\begin{coloredlist}
    % \item One neuron is transmitting a signal to a muscle or gland.
    \item Actual change in cellular metabolism.
    \item Indirect exchange of ions.
    \item Variable duration (can be very long).
    \item At least 2 neuromodulator molecules bind to a receptor.
    \item \textbf{Example:} Dopamine binds to a receptor, which activates a G-protein, which activates an enzyme, which produces a second messenger, which opens a K\(^+\) channel.
    \item \textbf{At the Metabotropic Synapse:}
    \begin{coloredlist}
        \item Neuromodulator binds and initates process.
        \item Alpha subunit of G-protien binds to \cyanit{Adenylate Cyclase}.
        \begin{coloredlist}
            \item Activating adenylate cyclase to convert ATP to \cyanit{cAMP} (cyclic adenosine monophosphate).
        \end{coloredlist}
        \item cAMP activates \cyanit{Protein Kinase A}.
        \begin{coloredlist}
            \item Causing 2 subunits to dissociate.
            \item Catalytic portion is no longer inhibited.
            \item Allowing it to convert ATP to ADP. 
            \begin{coloredlist}
                \item This produces a phosphate group.
            \end{coloredlist}
        \item To end this process:
        \begin{coloredlist}
            \item Neuromodulator dissociates to end cAMP production.
            \item Enzymes
            \begin{coloredlist}
                \item \cyanit{Phosphodieterase} -- Metabolizes residual cAMP
                \item \cyanit{Phosphoprotein phophatase} -- Removes the phosphate and resets the channel.
            \end{coloredlist}
        \end{coloredlist}
        \end{coloredlist}
    \end{coloredlist}
\end{coloredlist}


\newpage
\begin{figure}[htbp]
    \centering
    \label{resting potential}
\begin{tikzpicture}
  % Labels for compartments
  \node at (-2,4.3) {\large \textbf{\underline{In}}};
  \node at (2,4.3) {\large \textbf{\underline{Out}}};
  \node at (-5.9,4.3) {\large \textbf{\underline{Permeability}}};

%   % Draw compartments as boxes
%   \draw (-5,-4) rectangle (-1,4); % Inside compartment
%   \draw (1,-4) rectangle (5,4);   % Outside compartment

  % Draw the membrane as a double line:
  % Left (solid) side denotes the negative (inside) and right (dotted) side the positive (outside)
  \draw[line width=1pt] (0.1,-4) -- (0.1,4);  
  \draw[dotted, line width=1pt] (-0.1,-4) -- (-0.1,4);

  % Place ion symbols on the "in" side (left) with relative sizes.
  % Here, potassium is large since it is more prevalent on the inside.
  \node (K_in) at (-3,2.5) {\huge \(\text{K}^{+}\)};
  \node (Na_in) at (-3,0) {\normalsize \(\text{Na}^{+}\)};
  \node (Cl_in) at (-3,-2.5) {\normalsize \(\text{Cl}^{-}\)};

  \node (P_K) at (-4.8,2.5) {\huge \(\text{P}\phantom{^{+}}\)};
  \node (P_K) at (-4.8,0) {\tiny \(\text{P}\phantom{^{+}}\)};
  \node (P_K) at (-4.8,-2.5) {\Large \(\text{P}\phantom{^{-}}\)};

  % Place ion symbols on the "out" side (right)
  % Here, sodium is made larger, etc.
  \node (K_out) at (3,2.5) {\normalsize \(\text{K}^{+}\)};
  \node (Na_out) at (3,0) {\huge \(\text{Na}^{+}\)};
  \node (Cl_out) at (3,-2.5) {\Large \(\text{Cl}^{-}\)};

  % Create nodes on the membrane for arrow connections
  \node (mem2_left) at (-0.5,0) {};
  \node (mem3_left) at (1.5,-2.5) {};
  \node (mem3_right) at (-1.5,-2.5) {};

  % Define arrow styles:
  \tikzset{
    diffusion/.style={->, >=Stealth, line width = 2pt, red},
    electric/.style={->, >=Stealth, line width = 2pt, blue}
  };

  % For each ion on the inside, draw two arrows (curved differently) going from the ion to the membrane.
  % The differences in the bending (and thus length) illustrate different magnitudes.
  % --- For K_in:
  \draw[electric, bend right=15] (K_in) to (K_out);
  \draw[diffusion, bend right=15, line width = 6pt] (K_out) to (K_in);
  % --- For Na_in:
  \draw[electric, bend left=15] (Na_out) to (mem2_left);
  \draw[diffusion, bend right=15] (Na_out) to (mem2_left);
  % --- For Cl_in:
  \draw[electric, bend right=15] (Cl_in) to (mem3_left);
  \draw[diffusion, bend right=15] (Cl_out) to (mem3_right);

  % Add a legend (key) to denote arrow styles.
  \node[draw, rectangle, right=1cm of K_out] (legend) {
    \begin{tabular}{ll}
      \textbf{Key:} & \\
      \tikz{\draw[diffusion, ->, >=Stealth, thick, red] (0,0) -- (1,0);} & Diffusion \\
      \tikz{\draw[electric, ->, >=Stealth, thick, blue] (0,0) -- (1,0);} & Electric \\
    \end{tabular}
  };
\end{tikzpicture}
\caption{The Resting Membrane Potential (RMP) of a Neuron}
\end{figure}

\begin{figure}[htbp]
    \centering
    \label{action potential}
\begin{tikzpicture}[scale=1.2]

    % Draw axes
    \draw[->, thick] (0,-4) -- (7,-4) node[right] {Time (ms)};
    \draw[->, thick] (0,-4) -- (0,3) node[above] {Voltage (mV)};

    % Draw threshold line at -55 mV and resting potential at -70 mV
    \draw[dashed] (0,-2.5) -- (7,-2.5) node[right] {Threshold (-55 mV)};
    \draw[dashed] (0,-3.5) -- (7,-3.5) node[right] {RMP (-70 mV)};
  
    % y-axis ticks and labels (scaling: 1 unit = 20 mV)
    \foreach \y/\label in {-3.5/{-70},-2.5/{-55}, -2/{-40}, -1/{-20}, 0/{0}, 1/{20}, 2/{40}} {
        \draw (0,\y) -- (-0.2,\y) node[left] {\label};
    }
    
    % x-axis ticks
    \foreach \x in {0,1,2,3,4,5,6} {
        \draw (\x,-4) -- (\x,-4.15) node[below] {\x};
    }
    
    % Draw action potential waveform using coordinates:
    % Points (x, y) where y = mV/20:
    % (0,-3.5):   Rest (-70 mV)
    % (1,-3.5):   Rest
    % (2,-2.75):  Threshold (-55 mV, since -55/20 = -2.75)
    % (3,2):      Peak depolarization (+40 mV, 40/20 = 2)
    % (4,-0.5):   Repolarization (-10 mV, -10/20 = -0.5)
    % (5,-4):     Hyperpolarization (-80 mV, -80/20 = -4)
    % (6,-3.5):   Return to rest (-70 mV)
    \draw[thick, horange] plot coordinates {
        (0,-3.5)
        (1,-3.5)
        (1.5,-3.5)
    };
    \draw[thick, horange, smooth] plot coordinates {
        (1.5,-3.5)
        (1.90,-2.55)
    };
    \draw[thick, horange] plot coordinates {
        (1.90,-2.55)
        (2,-2.5)
    };
    \draw[thick, horange, smooth] plot coordinates {
      (2,-2.5)
      (3,2)
      (4,-0.5)
      (5,-3.55)
      (6,-3.5)
    };

    % Add phase labels near the waveform
    \tikzset{
        circ/.style = {draw, circle, scale=0.7}
    };
    % Resting potential.
    \node[circ] at (1.3,-3.2) {1};
    % Opens sodium channels.
    \node[circ] at (1.75,-2.2) {2};
    % Potassium channels open.
    \node[circ] at (2.5,1.5) {3};
    % Sodium channels close.
    \node[circ] at (3.2,2.3) {4};
    % Potassium channels close.
    % \node[circ] at (6.1,-3.75) {5};
    
    \draw[decorate, decoration={brace, amplitude=5.5pt}]
    (5,-3.5) -- (6,-3.5);
    % Add text labels for phases
    \node at (2.2, 0) [rotate=78] {\textbf{Depolarization}};
    % The potassium is leaving the cell.
    \node at (4.2, 0) [rotate=287] {\textbf{Repolarization}};
    \node at (5.4, -3.1) {\textbf{Hyperpolarization}};

    % Arrow pointing hyperpolarization to -80 mv
    % \draw[->, thick] (7, -2) -- (4.93,-3.9) node[below] {};
  
  \end{tikzpicture}
    \caption{Action Potential Waveform}
\end{figure}

