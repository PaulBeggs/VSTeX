\begin{center}
    \textbf{NEW NOTES FOR 02/26/25} \\
    \hrulefill
\end{center}

\section{The Neuron}

\begin{coloredlist}
    \item Definition: Basic information processing unit of the NS.
    \item Similarities to an animal cell:
    \begin{coloredlist}
        \item \cyanit{Cell membrane}: Separates the inside of the cell from the outside environment.
        \item \cyanit{Nucleus}: Contains the genetic material of the cell.
        \item \cyanit{Organells}: Carry out the basic functions of the cell.
        \begin{coloredlist}
            \item \cyanit{Mitochondria}: Produce energy for the cell.
            \item \cyanit{Endoplasmic Reticulum}: Synthesizes proteins.
            \item \cyanit{Golgi Apparatus}: Packages proteins for transport.
            \item \cyanit{Lysosomes}: Break down waste products.
        \end{coloredlist}
        \item Basic cellular processes.
    \end{coloredlist}
    \item Differences:
    \begin{coloredlist}
        \item Special ``morphology'' (shape).
        \item Communicate through an electrochemical process.
    \end{coloredlist}
\end{coloredlist}

\subsection{Structure of the Neuron}
(Mostly a recap of \hyperref[Terms]{Terms})
\begin{coloredlist}
    \item \cyanit{Soma}
    \item \cyanit{Dendrite}
    \item \cyanit{Axon}
    \item \cyanit{Terminal Arboriza} -- Branches at the end of the axon.
    \item \cyanit{Terminal Buttons} -- End of the terminal arboriza.
    \item \cyanit{Axon Hillock}
    \item \cyanit{Myelin}
    \begin{coloredlist}
        \item Not all axons have it.
        \item Glial cells / 70\% Lipid / Nodes of Ranvier.
        \item Mutliple Sclerosis (MS) -- Demyelination.
    \end{coloredlist}
\end{coloredlist}

\subsection{Support cells in the Nervous System}

\begin{coloredlist}
    \item Glia/Glial Cells/ Neuroglia -- Support cells.
    \begin{coloredlist}
        \item Capable of cell division after birth/communication.
        \item Make up half of the volume, but are 10-50 times more numerous. (The other half is made up of neurons.)
        \item CNS:
        \begin{coloredlist}
            \item \cyanit{Macroglia} -- Large glial cells.
            \begin{coloredlist}
                \item \cyanit{Astrocytes} -- Star-shaped cells that provide physical support to neurons, clean up debris, and provide nutrients to neurons.
                \begin{coloredlist}
                    \item \textit{Note} that these cells do not help neurons grow when they are damaged. In fact, they inhibit growth by proliferating and forming a scar.
                \end{coloredlist}
                \item \cyanit{Oligodendrocytes} -- ``few branches (in contrast to Astrocytes)'' -- \textit{Form} myelin sheath around multiple axons in the CNS. 
            \end{coloredlist}
            \item \cyanit{Microglia} -- Small cells that remove debris from injured or dead cells.
            \item \cyanit{Ependymal Glia} -- Line the ventricles of the brain and spinal cord. (Remember the CSF?)
        \end{coloredlist}
        \item PNS:
        \begin{coloredlist}
            \item \cyanit{Satellite Cells} -- Provide nutrients and physical support to neurons.
            \item \cyanit{Schwann Cells} -- Form myelin sheath around axons in the PNS. These cells are monogamists; they wrap their arms around one axon.
            \begin{coloredlist}
                \item Neuronal Regeneration. 
            \end{coloredlist}
        \end{coloredlist}
        \item The Myelin Sheath is composed of Oligodendrocytes in the CNS and Schwann Cells in the PNS.
        \item \cyanit{Phagocytosis} -- When an injury occurs, the glial cells divide and eat the dead cells. (Done by Microglia and Schwann Cells.)
        \item Maintenance of Internal Consistency.
        \begin{coloredlist}
            \item When neurons undergo rapid firing, they release potassium ions. Astrocytes absorb these ions to maintain the internal consistency of the neuron.
        \end{coloredlist}
    \end{coloredlist}
\end{coloredlist}

\begin{center}
    \textbf{NEW NOTES FOR 02/28/25} \\
    \hrulefill
\end{center}

\subsection{Are Glial Cells Contributing to Alzheimer's Disease?}

\begin{coloredlist}
    \item Normally,
    \begin{coloredlist}
        \item Beta amyloid cleared away through microglia.
    \end{coloredlist}
    \item IF beta amyloid builds up too much, Tau INSIDE cells builds up.
    \item This leads to inflammation, which maybe leads to the problems of Alzheimer's.
\end{coloredlist}

\section{Different Kinds of Neurons}

\begin{coloredlist}
    \item Based on Structure
    \item Based on Function
\end{coloredlist}

\subsection{Structural Classification of Neurons}

\begin{coloredlist}
    \item \cyanit{Unipolar/Pseudounipolar}
    \begin{coloredlist}
        \item \textit{The difference:} The axon and dendrite are fused together.
    \end{coloredlist}
    \item \cyanit{Bipolar}
    \item \cyanit{Multipolar}
\end{coloredlist}

\subsection{Functional Classification of Neurons}

\begin{coloredlist}
    \item Sensory Neurons (Afferent)
    \begin{coloredlist}
        \item Carry information from the sensory receptors to the CNS.
        \item Unipolar.
        \item ``Afferent'' -- ``bearing or conducting inward''
    \end{coloredlist}
    \item Interneurons
    \item Motor Neurons (Efferent)
    \begin{coloredlist}
        \item Carry information from the CNS to the muscles and glands.
        \item Multipolar.
        \item ``Efferent'' -- ``conducting outward''
    \end{coloredlist}
    \item Remember: \(\text{Ad = towards} \qquad \text{Ex = from} \qquad \text{Ferro = I carry}.\)
\end{coloredlist}

\section{Neural Communication}

\begin{coloredlist}
    \item \textbf{2 Systems of Neuronal Communication:}
    \begin{coloredlist}
        \item \cyanit{Binary}
        \begin{coloredlist}
            \item All or none.
        \end{coloredlist}
        \item \cyanit{Analogue}
        \begin{coloredlist}
            \item Graded, matter or degree.
        \end{coloredlist}
    \end{coloredlist}
\end{coloredlist}

\subsection{Binary System (``Off'' and ``On'')}

\begin{coloredlist}
    \item \cyanit{The Resting Membrane Potential (RMP)}: ``OFF'' 
    \begin{coloredlist}
        \item -70 mV (relative to the outside).
        \item Understand the cell membrane.
        \begin{coloredlist}
            \item \cyanit{Phospholipid Bilayer} -- Hydrophobic tails and hydrophilic heads.
            \item Semipermeability.
            \begin{coloredlist}
                \item Lipid, lipid soluble, small, and neutral.
            \end{coloredlist}
            \item \cyanit{Embedded Proteins} -- Channels and pumps. 
            \begin{coloredlist}
                \item \textbf{4 Jobs We Care About:}
                \begin{coloredlist}
                    \item \cyanit{Receptors}
                    \begin{coloredlist}
                        \item High specificity and affinity.
                        \item ``Places where things can bind to the cell and cause a change.''
                    \end{coloredlist}
                    \item \cyanit{Channels}.
                    \begin{coloredlist}
                        \item \cyanit{Gated Channels}: Ligand gated channels. Voltage gated channels. 
                    \end{coloredlist}
                    \item \cyanit{Pumps} -- Active transport.
                    \item \cyanit{Enzymes} -- Breaks down neurotransmitters.
                \end{coloredlist}
            \end{coloredlist}
        \end{coloredlist}
    \end{coloredlist}
    \item \cyanit{The Action Potential (AP)}: ``ON''
\end{coloredlist}