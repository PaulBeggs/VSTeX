\section{Prehistoric}

\begin{coloredlist}
    \item A million years or more, people have been interested in the brain. Archaeological evidence shows that skulls are bashed in (jagged, not precise). As a result, the person dies, and therefore the brain is vital to life.
\end{coloredlist}

\section{7000 Years Ago}

\begin{coloredlist}
    \item New holes in the brain, but these holes show signs of healing. Therefore, these new holes are intended to help the person who is suffering.
    The fancy name is trephination.
    \item The theory for these holes is that they were drilled to cure the person. In other words, to relieve a person of a wicked spirit.
\end{coloredlist}

\section{5000 Years Ago}

\begin{coloredlist}
    \item Egyptian physicians show that they were aware of brain damage through their writings.
    \item Complications arise because they thought the heart contained the soul--you need it to live and emotions effect it.
\end{coloredlist}

\section{Ancient Greece--Hippocrates 4th Century, BC}

\begin{coloredlist}
    \item Ponder the correlation between structure and function. Now, extend this thought to the brain/head.
    \item The brain is the place where sensation and intelligence reside. Not the heart.
\end{coloredlist}

\section{Ancient Greece--Aristotle}

\begin{coloredlist}
    \item Clung to the idea of the heart being the one in charge.
    \item Figured the brain was a radiator. That is, we would send heated blood to the brain for it to be cooled off. This ``heated blood'' arose from our emotions. Thus, humans are more rational because we have a lot of cooling when compared to other animals.
\end{coloredlist}

\section{Roman Empire--Galen 2nd Century, AD}

\begin{coloredlist}
    \item Galen is a physician to gladiators.
    \item Thought the cerebellum was for motor control (because the cerebellum is hard, like muscles) and the cerebrum is for memory because it is soft, and you can "write on it."
    \item Noticed there were large spaces (called "ventricles" literally translated to spaces) that were filled with fluid.
    \item From here, we get the four humors (literally translated to fluids).
    \item Galen thought that these fluids are what control the brain, NOT the brain structure itself. Think of the purpose of canned vegetables. The tin container does not actively contribute to the liquid / vegetables; rather, it is disposable.
    \item These ideas were jumpstarted by the invention of aqueducts. The movement of water was so important from aqueducts, so the idea this idea was extended to the brain.
\end{coloredlist}

\section{Analysis by Analogy--17th Century}

\begin{coloredlist}
    \item French developed hydraulically controlled machines.
    \item Again, this is adding to the idea that water (which can flow through things and cause movements).
\end{coloredlist}

\section{Rene Descartes--1596-1650}

\begin{coloredlist}
    \item Believed that non-humans--what he called animals--are controlled by fluid.
    \item From this, he posited that the human body is a material entity functioning as a machine (like animals)--these are known as reflexes.
    \item But, the mind is nonmaterial and free from the laws of the universe and was uniquely human.
    \item Question: How does the nonmaterial part of the body (the mind) communicate with the material part of the body? Through the pineal gland! This gland would move around like a joystick and would manipulate the fluid that came from the third ventricle.
\end{coloredlist}

\section{The Mind/Body Problem}

\begin{coloredlist}
    \item What is the basic relationship between mental events and physical events?
    \item Dualism--The mind exists independently of the brain and exerts some control over it.
    Strengths: Commonsense view.
    \item Weaknesses: The universe is composed of matter or energy.
    \item Modern neuroscientific explanation: Everything the body does rests on the events taking place in specific, definable parts of the nervous system--the "mind" is the product of the nervous system activity.
\end{coloredlist}

\section{Recall and Overview}

\begin{coloredlist}
    \item Prehistorically: People knew you needed the brain to live.
    \item 7000 years ago: trephination
    \item Egyptians: mixed but mostly supported the heart
    \item Greeks: mixed between Hippocrates and Aristotle
    \item Roman Empire: Galen--Shift from brain structure to fluid
    \item Descartes: backed the fluid model
    \item What's next? The scientific method
\end{coloredlist}

\section{The Scientific Method--17th and 18th Century}

\begin{coloredlist}
    \item A new world view at the end of the Renaissance.
    \begin{coloredlist}
        \item Replace Rationalism with Scientific Method.
    \end{coloredlist}
    \item Closer look at the substance of the brain:
    \begin{coloredlist}
        \item Gray and white matter change the way we look at the brain. That is, why would these parts of the brain that are clearly different, be different if the brain is used just to move fluids around.
        \item Also, everyone has the same brain structure, so these bumps and groves must mean something.
    \end{coloredlist}
\end{coloredlist}

\section{Electricity}

\begin{coloredlist}
    \item Isaac Newton showed it is possible to electrically stimulate nerves.
    \item Then, Luigi Galvani and Emil du Bois-Reymond showed that electricity can make muscles contract.
    \item Later on, Hermann von Helmholtz showed that the speed of nerve conduction is not instantaneous.
    \item This important distinction shows that these nerves are not like wires--such as Luigi Galvani and Emil du Bois-Reymond thought.
    \item Bell and Magendie showed that the dorsal nerve root and the ventral nerve root are different.
    \item The dorsal nerve root is for sensory information, and the ventral nerve root is for motor information.
    \item Thus, dorsal = sensory and ventral = muscle.
    \item Johannes M\"uller came up with the doctrine of Specific Nerve Energies.
    \item This doctrine states that the nature of a sensation depends on which nerve is stimulated, not on how the nerve is stimulated.
    \item For example, if you stimulate the optic nerve, you will see something. If you stimulate the auditory nerve, you will hear something.
    \item Spawned the Great Debate: Is the brain a homogenous mass or is it made up of different parts?
\end{coloredlist}

\section{The Great Debate}

\begin{coloredlist}
    \item Franz Joseph Gall and Johann Spurzheim thought the brain was made up of different parts.
    \item They thought that the bumps and groves in the brain were due to the size of the brain parts.
    \item They also thought that the size of the brain parts was due to the amount of use of that part.
    \item This is known as phrenology.
    \begin{coloredlist}
        \item Phrenology was a pseudoscience because it was not based on empirical evidence.
        However, it did lead to the idea that the brain is made up of different parts.
    \end{coloredlist}
    \item Localization of Functions--brain function can be localized to regions, pathways, or neurons.
    \begin{coloredlist}
        \item Basically, if you cut out a piece of brain, and the animal (a pigon) is no longer able to do a specific task, then that part of the brain is responsible for that task.
        \item However, it turns out that these pigons were able to relearn the task, so the brain is not as localized as we thought.
    \end{coloredlist}
    \item Aggregate Field Theory
    \begin{coloredlist}
        \item Complex brain functions emerge from the collective interactions of numerous simple neuronal activities
        \item Unlike localizationist models, this theory emphasizes the distributed nature of cognitive processes across neural networks
    \end{coloredlist}
    \item Pierre Flourens thought the brain was a homogenous mass.
    \begin{coloredlist}
        \item He believed in Localization of Function, until his results nullified his beliefs.
    \end{coloredlist}
    \item Paul Broca
    \begin{coloredlist}
        \item Found a patient who could understand language but could not speak.
        \item After the patient died, Broca found a lesion in the left frontal lobe.
        \item This area is now known as Broca's area.
        \item This area is responsible for speech production.
        \item These results put us back into the realm of Localization of Function.
    \end{coloredlist}
    \item In comes Carl Wernicke (1874)
    \begin{coloredlist}
        \item Found a patient who could speak but could not understand language.
        \item After the patient died, Wernicke found a lesion in the left temporal lobe.
        \item This area is now known as Wernicke's area.
        \item This area is responsible for language comprehension.
    \end{coloredlist}
    \item Then, we have Gustav Fritsch and Eduard Hitzig (1870)
    \begin{coloredlist}
        \item Similarly to Luigi Galvani and Emil du Bois-Reymond, they electrically stimulated the brain.
        \item They found that the motor cortex is responsible for movement.
    \end{coloredlist}
    \item Shepherd Ivory Franz (in D.C. from 1907-1924)
    \begin{coloredlist}
        \item Found that people are able to relearn tasks after brain damage.
    \end{coloredlist}
\end{coloredlist}
\section{Same Resolution?}

\begin{coloredlist}
    \item Modified Aggregate Field Theory
    \begin{coloredlist}
        \item Karl S. Lashley (1890-1958)
        \begin{coloredlist}
            \item The Principles of Mass Action
            \begin{coloredlist}
                \item Complex behavior--such as learning--is dependent on the total mass of the brain.
            \end{coloredlist}
            \item Equipotentiality
            \begin{coloredlist}
                \item Specialization of function is not tied to specific brain regions.
                \item All parts of the cortex contribute equally to complex behavior.
            \end{coloredlist}
            \item Vicarious functioning
            \begin{coloredlist}
                \item If one part of the brain is damaged, another part can take over.
            \end{coloredlist}
        \end{coloredlist}
    \end{coloredlist}
\end{coloredlist}


\section{Analysis}

\begin{enumerate}
    \item \textbf{Prehistoric}: Recognition of the brain's vital role in life through skull injuries. No scientific theories yet.
    
    \item \textbf{7000 Years Ago}: Trephination (skull drilling) practiced to release ``evil spirits,'' indicating early medical intervention.
    
    \item \textbf{5000 Years Ago}: Egyptians documented brain damage but prioritized the heart as the seat of the soul.
    
    \item \textbf{Ancient Greece—Hippocrates (4th Century BCE)}: Proposed the brain as the center of sensation/intelligence, countering heart-centric views.
    
    \item \textbf{Ancient Greece—Aristotle}: Defended the heart as the command center, viewing the brain as a blood-cooling ``radiator.''
    
    \item \textbf{Roman Empire—Galen (2nd Century CE)}: Linked cerebellum to motor control and cerebrum to memory; emphasized ventricular fluids (humors) over brain structure.
    
    \item \textbf{17th Century (Analysis by Analogy)}: Hydraulic systems inspired fluid-based brain theories.
    
    \item \textbf{Rene Descartes (1596–1650)}: Dualism (mind vs. body); proposed pineal gland as the mind-body interface.
    
    \item \textbf{17th–18th Century (Scientific Method)}: Shift to empirical study; recognition of gray/white matter differences.
    
    \item \textbf{Electricity Discoveries}: Newton (nerve stimulation), Galvani/du Bois-Reymond (muscle contraction via electricity), Helmholtz (nerve conduction speed), Bell/Magendie (sensory/motor nerve roots), Müller (specific nerve energies).
    
    \item \textbf{The Great Debate}:
        \begin{itemize}
            \item Gall/Spurzheim: Phrenology (localization via brain bumps).
            \item Flourens: Shifted to aggregate theory after experiments.
            \item Broca/Wernicke: Localized language areas.
            \item Fritsch/Hitzig: Mapped motor cortex.
            \item Franz: Relearning post-injury supported aggregate theory.
        \end{itemize}
    
    \item \textbf{Modified Aggregate Theory}: Karl Lashley emphasized mass action and equipotentiality.
\end{enumerate}

\begin{landscape}
    \pagestyle{plain}
    \vfill
\begin{table}[h]
\centering
\caption{Key Scientists and Contributions}
\label{tab:neuroscientists}
\begin{tabular}{p{6.5cm}p{9.5cm}p{6.5cm}}
\toprule
\textbf{Scientist} & \textbf{Contributions} & \textbf{Relation to Others} \\
\midrule
Hippocrates & Brain as seat of sensation/intelligence & Opposed Aristotle \\
Aristotle & Heart as command center; brain as radiator & Opposed Hippocrates \\
Galen & Cerebellum (motor), cerebrum (memory); humors & Influenced by Roman aqueducts \\
Rene Descartes & Mind-body dualism; pineal gland & — \\
Isaac Newton & Early nerve stimulation via electricity & — \\
Luigi Galvani & Electricity-induced muscle contraction & Collaborated with du Bois-Reymond \\
Emil du Bois-Reymond & Same as Galvani & Collaborated with Galvani \\
Hermann von Helmholtz & Measured nerve conduction speed & — \\
Charles Bell \& François Magendie & Sensory/motor nerve roots (Bell-Magendie Law) & Collaborators \\
Johannes Müller & Doctrine of specific nerve energies & — \\
Franz Joseph Gall & Phrenology (brain localization) & Collaborated with Spurzheim \\
Johann Spurzheim & Promoted phrenology & Collaborated with Gall \\
Pierre Flourens & Shifted to aggregate theory & Opposed Gall/Spurzheim \\
Paul Broca & Localized speech production (Broca's area) & — \\
Carl Wernicke & Localized language comprehension (Wernicke's area) & Complementary to Broca \\
Gustav Fritsch \& Eduard Hitzig & Mapped motor cortex & Collaborators \\
Shepherd Ivory Franz & Relearning post-brain damage & Influenced Lashley \\
Karl S. Lashley & Mass action, equipotentiality & Built on Franz's work \\
\bottomrule
\end{tabular}
\end{table}
\vfill

\end{landscape}