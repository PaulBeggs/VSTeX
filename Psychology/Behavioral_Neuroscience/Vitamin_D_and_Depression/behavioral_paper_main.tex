\documentclass[stu]{apa7}
\usepackage{amsmath}
\usepackage{amssymb}
\usepackage{enumitem}
\usepackage{geometry}
\usepackage[style=apa,uniquename=false,backend=biber]{biblatex}
\usepackage{hyperref}
\usepackage{xcolor}
\definecolor{horange}{HTML}{f58026}
\hypersetup{
    colorlinks=true,
    linkcolor=horange,
    filecolor=horange,      
    urlcolor=horange,
    citecolor=horange
}
\usepackage{draculatheme}
\addbibresource{Behavioral_Bibliography.bib}

\title{Feeding and Nutrients: Vitamin D and Depression}
\shorttitle{Vitamin D and Depression}
\author{Paul Beggs}
\authorsaffiliations{Hendrix College}
\course{PSYC 360: Behavioral Neuroscience}
\professor{Dr.\ Jennifer Peszka}
\duedate{\today}


\begin{document}

\maketitle

\parencite{noauthor_depression_nodate}

The DSM-5 describes depression as a mood disorder characterized by sadness and/or a loss of interest or pleasure. People who have depression may experience feelings of sadness, irritability, sleep disturbances, loss of energy, anxiety, and suicidal ideation (Mayo Clinic, 2022). Additionally, people who have depression often find that it affects their daily life, including work and school  (Mayo Clinic, 2022). Data shows that in 2021, 14.5 million U.S. adults had a major depressive episode with severe impairment within the past year (NIH, 2023), showing the high prevalence of the disorder. Medication is commonly used as a treatment for depression, including Selective serotonin reuptake inhibitors (SSRIs) and Serotonin-norepinephrine reuptake inhibitors (SNRIs) (Mayo Clinic, 2022). These drugs have side effects, especially when first beginning treatment. These side effects include agitation, feeling sick, loss of appetite, insomnia, headaches, and loss of libido (NHS, 2021). Psychotherapy is also used to treat depression, which includes regular visits to a mental health professional (Mayo Clinic, 2022). While psychotherapy and medication are often effective for treating depression, they can be difficult to access within certain populations due to time and financial constraints. Because of this, it is important to examine more accessible treatments for depression. Some studies have found a relation between low levels of vitamin D and depression (Kjaegaard et al, 2012). Because of this, psychologists have hypothesized that low levels of vitamin D can cause depression or increase symptom severity. Natural sources of vitamin D are tied to certain foods and sun exposure, both of which can be altered by the presence of depression, potentially leading to decreases in vitamin D. Whatever the relationship, there is a clear relation between depression and vitamin D that needs to be examined, as vitamin D has potential to serve as a treatment for depression.

\printbibliography

\end{document}


