\section{Prehistoric}

A million years or more, people have been interested in the brain. Archaeological evidence shows that skulls are bashed in (jagged, not precise). As a result, the person dies, and therefore the brain is vital to life. 

\section{7000 Years Ago}

New holes in the brain, but these holes show signs of healing. Therefore, these new holes are intended to help the person who is suffering.
The fancy name is trephination. 
The theory for these holes is that they were drilled to cure the person. In other words, to relieve a person of a wicked spirit. 

\section{5000 Years Ago}

Egyptian physicians show that they were aware of brain damage through their writings.
Complications arise because they thought the heart contained the soul--you need it to live and emotions effect it.

\section{Ancient Greece--Hippocrates 4th Century, BC}

Ponder the correlation between structure and function. Now, extend this thought to the brain/head.
The brain is the place where sensation and intelligence reside. Not the heart.

\section{Ancient Greece--Aristotle}

Clung to the idea of the heart being the one in charge. 
Figured the brain was a radiator. That is, we would send heated blood to the brain for it to be cooled off. This "heated blood" arose from our emotions. Thus, humans are more rational because we have a lot of cooling when compared to other animals.

\section{Roman Empire--Galen 2nd Century, AD}

Galen is a physician to gladiators. 
Thought the cerebellum was for motor control (because the cerebellum is hard, like muscles) and the cerebrum is for memory because it is soft, and you can "write on it."
Noticed there were large spaces (called "ventricles" literally translated to spaces) that were filled with fluid.
From here, we get the four humors (literally translated to fluids).
Galen thought that these fluids are what control the brain, NOT the brain structure itself. Think of the purpose of canned vegetables. The tin container does not actively contribute to the liquid / vegetables; rather, it is disposable.
These ideas were jumpstarted by the invention of aqueducts. The movement of water was so important from aqueducts, so the idea this idea was extended to the brain.

\section{Analysis by Analogy--17th Century}

French developed hydraulically controlled machines.
Again, this is adding to the idea that water (which can flow through things and cause movements).

\section{Rene Descartes--1596-1650}

Believed that non-humans--what he called animals--are controlled by fluid.
From this, he posited that the human body is a material entity functioning as a machine (like animals)--these are known as reflexes.
But, the mind is nonmaterial and free from the laws of the universe and was uniquely human.
Question: How does the nonmaterial part of the body (the mind) communicate with the material part of the body? Through the pineal gland! This gland would move around like a joystick and would manipulate the fluid that came from the third ventricle.

\section{The Mind/Body Problem}

What is the basic relationship between mental events and physical events?
Dualism--The mind exists independently of the brain and exerts some control over it.
	Strengths: Commonsense view.
	Weaknesses: The universe is composed of matter or energy.
Modern neuroscientific explanation: Everything the body does rests on the events taking place in specific, definable parts of the nervous system--the "mind" is the product of the nervous system activity.

\section{Recall and Overview}

Prehistorically: People knew you needed the brain to live.
7000 years ago: trephination
Egyptians: mixed but mostly supported the heart
Greeks: mixed between Hippocrates and Aristotle
Roman Empire: Galen--Shift from brain structure to fluid
Descartes: backed the fluid model
What's next? The scientific method

\section{The Scientific Method--17th and 18th Century}

A new world view at the end of the Renaissance.
	Replace Rationalism with Scientific Method.
Closer look at the substance of the brain:
    Gray and white matter change the way we look at the brain. That is, why would these parts of the brain that are clearly different, be different if the brain is used just to move fluids around.
    Also, everyone has the same brain structure, so these bumps and groves must mean something.

\section{Electricity}

Isaac Newton showed it is possible to electrically stimulate nerves.
Then, Luigi Galvani and Emil du Bois-Reymond showed that electricity can make muscles contract.
Later on, Hermann von Helmholtz showed that the speed of nerve conduction is not instantaneous.
    This important distinction shows that these nerves are not like wires--such as Luigi Galvani and Emil du Bois-Reymond thought.
Bell and Magendie showed that the dorsal nerve root and the ventral nerve root are different.
    The dorsal nerve root is for sensory information, and the ventral nerve root is for motor information.
    Thus, dorsal = sensory and ventral = muscle.
Johannes Mu\"ller came up with the doctrine of Specific Nerve Energies.
    This doctrine states that the nature of a sensation depends on which nerve is stimulated, not on how the nerve is stimulated.
    For example, if you stimulate the optic nerve, you will see something. If you stimulate the auditory nerve, you will hear something.
    Spawned the Great Debate: Is the brain a homogenous mass or is it made up of different parts?

\section{The Great Debate}

Franz Joseph Gall and Johann Spurzheim thought the brain was made up of different parts.
    They thought that the bumps and groves in the brain were due to the size of the brain parts.
    They also thought that the size of the brain parts was due to the amount of use of that part.
    This is known as phrenology.
    Phrenology was a pseudoscience because it was not based on empirical evidence.
    However, it did lead to the idea that the brain is made up of different parts.
Localization of Functions--brain function can be localized to regions, pathways, or neurons.
    Basically, if you cut out a piece of brain, and the animal (a pigon) is no longer able to do a specific task, then that part of the brain is responsible for that task. 
    However, it turns out that these pigons were able to relearn the task, so the brain is not as localized as we thought.
Aggregate Field Theory
    Complex brain functions emerge from the collective interactions of numerous simple neuronal activities
    Unlike localizationist models, this theory emphasizes the distributed nature of cognitive processes across neural networks
Pierre Flourens thought the brain was a homogenous mass.
    He believed in Localization of Function, until his results nullified his beliefs.
Paul Broca
    Found a patient who could understand language but could not speak.
    After the patient died, Broca found a lesion in the left frontal lobe.
    This area is now known as Broca's area.
    This area is responsible for speech production.
    These results put us back into the realm of Localization of Function.
In comes Carl Wernicke (1874)
    Found a patient who could speak but could not understand language.
    After the patient died, Wernicke found a lesion in the left temporal lobe.
    This area is now known as Wernicke's area.
    This area is responsible for language comprehension.
Then, we have Gustav Fritsch and Eduard Hitzig (1870)
    Similarly to Luigi Galvani and Emil du Bois-Reymond, they electrically stimulated the brain.
    They found that the motor cortex is responsible for movement.
Shepherd Ivory Franz (in D.C. from 1907-1924)
    Found that people are able to relearn tasks after brain damage.

\section{Same Resolution?}

\begin{itemize}
\item Modified Aggregate Field Theory 
\begin{itemize}
    \item Karl S. Lashley (1890-1958)
    \begin{itemize}
        \item The Principles of Mass Action
        \begin{itemize}
            \item Complex behavior--such as learning--is dependent on the total mass of the brain.
        \end{itemize}
        \item Equipotentiality
        \begin{itemize}
            \item Specialization of function is not tied to specific brain regions.
            \item All parts of the cortex contribute equally to complex behavior.
        \end{itemize}
        \item Vicarious functioning
        \begin{itemize}
            \item If one part of the brain is damaged, another part can take over.
        \end{itemize}
    \end{itemize}
\end{itemize}
    Karl S. Lashley (1890-1958)
        The Principles of Mass Action
            Complex behavior--such as learning--is dependent on the total mass of the brain.
        Equipotentiality
            Specialization of function is not tied to specific brain regions.
            All parts of the cortex contribute equally to complex behavior.
        Vicarious functioning
            If one part of the brain is damaged, another part can take over.
\end{itemize}


\section{Analysis}
\begin{itemize}
    \item Localization of Function:
    \begin{itemize}
        \item Proponents: Gall, Spurzheim, Broca, Wernicke, Fritsch, Hitzig
        \item Contributions: Identified specific brain regions responsible for functions like speech and movement.
    \end{itemize}

    
    \item Aggregate Field Theory:
\begin{itemize}
    \item Proponents: Flourens, Franz
    \item Contributions: Emphasized the distributed nature of cognitive processes across neural networks.
    \item Impact: Advanced understanding of complex brain functions.
\end{itemize}
\end{itemize}

