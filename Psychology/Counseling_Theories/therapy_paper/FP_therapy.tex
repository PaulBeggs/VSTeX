\documentclass[stu]{apa7}
\usepackage{amsmath}
\usepackage{amssymb}
\usepackage{enumitem}
\usepackage{geometry}
\usepackage[style=apa,uniquename=false,backend=biber]{biblatex}
\usepackage{hyperref}
\usepackage{xcolor}
\definecolor{horange}{HTML}{f58026}
\hypersetup{
    colorlinks=true,
    linkcolor=horange,
    filecolor=horange,      
    urlcolor=horange,
    citecolor=horange
}
% \usepackage{draculatheme}
\addbibresource{references.bib}

\title{Theoretical Orientation: A Cognitive Behavioral Perspective on Mental Health and Therapy}
% \shorttitle{Personality Development}
\author{Paul Beggs}
\authorsaffiliations{Hendrix College}
\course{PSYC 243: Counseling Theory \& Practice}
\professor{Dr.\ Sarah Root}
\duedate{\today}

\begin{document}
\maketitle


The aim of this paper is to present a theoretical orientation grounded in cognitive behavioral therapy (CBT). This orientation reflects my views on the nature of humanity, the causes of mental illness, and the goals and processes of therapy. CBT offers an evidence-based approach that aligns with my belief that people are capable of meaningful change through the modification of thoughts and behaviors. The following sections will outline how CBT explains psychological distress, what therapy aims to achieve, and the role of the therapist in supporting client growth, with support from current research in the field.

\section{The Nature of Humanity and the Cause of Mental Illness}

Taking the standpoint of CBT, we know that human nature is shaped by both biology and learning, but our psychological experiences are largely determined by how we interpret and respond to events. CBT views individuals as capable of learning and change, with the ability to modify maladaptive thoughts and behaviors over time. This orientation assumes that distress arises not from external events themselves but from the meanings we attach to those events. For example, two people might experience the same life challenge, but only one develops depression due to underlying beliefs such as "I'm a failure if I make mistakes."

Mental illness, according to CBT, stems from distorted thinking patterns and learned behaviors that reinforce negative emotions and maladaptive actions. These patterns are often rooted in early experiences that shape our core beliefs about ourselves, others, and the world \parencite{clark2011cognitive}. Over time, these beliefs become automatic and rigid, contributing to anxiety, depression, and other mental health conditions when they go unchallenged \parencite{clark2011cognitive}. Unlike psychoanalytic theories that focus on unconscious drives or humanistic theories that emphasize blocked self-actualization, CBT emphasizes observable, modifiable processes—specifically cognition and behavior.

\section{Goals of Therapy and the Role of the Therapist}

The main goal of CBT is to help individuals recognize and change unhelpful thought patterns and behaviors that contribute to emotional distress. Therapy helps people learn to identify negative automatic thoughts, test their accuracy, and develop more balanced ways of thinking. Simultaneously, it teaches behavioral strategies such as exposure, activity scheduling, and problem-solving to help individuals respond more adaptively to life challenges \parencite{dobson2018evidence}. The ultimate aim is not just symptom relief but also the development of long-term coping skills that allow clients to manage future difficulties independently.

In CBT, the therapist serves as a collaborative guide rather than an authority figure. The therapist and client work together to identify problematic thoughts and behaviors, set goals, and evaluate progress. Sessions are structured and goal-directed, often including homework assignments that give clients the opportunity to practice skills between sessions. This collaborative, evidence-based approach empowers clients and aligns with the CBT belief that people can learn to change their own patterns of thinking and behavior (Beck, 2011).

\section{Evidence Supporting Cognitive Behavioral Therapy}

CBT is one of the most extensively researched therapeutic approaches and has strong empirical support for its effectiveness across a range of disorders. Numerous meta-analyses have found CBT to be highly effective in treating depression, anxiety disorders, post-traumatic stress disorder, and other conditions. For example, \textcite{hofmann2012efficacy} found that CBT had large effect sizes across various anxiety disorders, indicating its broad efficacy. Similarly, \textcite{cuijpers2008psychotherapy} concluded that CBT is one of the most effective interventions for depression, showing results comparable to or better than pharmacotherapy, especially in the long term.

In addition to outcome research, studies have also explored the mechanisms by which CBT works. Evidence suggests that changes in cognition often precede and predict symptom reduction, supporting the idea that altering thought patterns leads to emotional and behavioral change \parencite{stober2001worry}. Furthermore, CBT’s emphasis on relapse prevention and self-management contributes to its long-term effectiveness, with clients often maintaining gains well after therapy ends \parencite{hollon2006enduring}.

\section{Conclusion}

From a CBT perspective, human beings are resilient and capable of change. Mental illness is seen as a product of learned cognitive and behavioral patterns that can be modified through structured, collaborative, and evidence-based interventions. Therapy provides not only symptom relief but also tools for lifelong self-management. With strong empirical support and a focus on empowering the individual, CBT remains a leading model in modern psychotherapy.

\printbibliography
\end{document}