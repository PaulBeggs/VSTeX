\section{Sigmund Freud}

\subsection{Terms}

    \begin{coloredlist}
        \item \cyanit{Ego}
        \begin{coloredlist}
            \item Executive mediating between id impulses and superego demands
            \item Testing reality
            \item Rational
            \item Operates at the conscious level but also at the preconscious level.
        \end{coloredlist}
        \item \cyanit{Superego}
        \begin{coloredlist}
            \item Ideals and morals
            \item Striving for perfection
            \item Incorporated from parents
            \item becoming a person's conscience
            \item Operates at the conscious level
        \end{coloredlist}
        \item \cyanit{Id}
        \begin{coloredlist}
            \item Basic impulses (sex and aggression)
            \item Seeking immediate gratification
            \item Irrational and impulsive
            \item Operates at the unconscious level
        \end{coloredlist}
        \item \textbf{Unconscious Ego Defense Mechanisms}
        \begin{coloredlist}
            \item \cyanit{Repression}
            \begin{coloredlist}
                \item Banishes anxiety-arousing wishes and feelings from consciousness.
                \item Example: A child who is abused by a parent forms no memory of the abuse.
            \end{coloredlist}
            \item \cyanit{Denial}
            \begin{coloredlist}
                \item Refusing to believe or even perceive painful realities.
                \item Example: A partner denies evidence of his spouse's affair.
            \end{coloredlist}
            \item \cyanit{Reaction Formation}
            \begin{coloredlist}
                \item Switching unacceptable impulses into their opposites.
                \item Example: A person who is angry with a friend acts overly kind.
            \end{coloredlist}
            \item \cyanit{Projection}
            \begin{coloredlist}
                \item Disguising one's own threatening impulses by attributing them to others.
                \item Example: A person who is unfaithful accuses his partner of cheating.
            \end{coloredlist}
            \item \cyanit{Displacement}
            \begin{coloredlist}
                \item Shifting sexual or aggressive impulses toward a more acceptable or less threatening object or person.
                \item Example: A person who is angry with his boss kicks his dog.
            \end{coloredlist}
            \item \cyanit{Rationalization}
            \begin{coloredlist}
                \item Offering self-justifying explanations in place of the real, more threatening unconscious reasons for one's actions.
                \item Example: A person who drinks every day says he does so to be sociable.
            \end{coloredlist}
            \item \cyanit{Sublimation}
            \begin{coloredlist}
                \item Transferring of unacceptable impulses into socially valued motives.
                \item Example: A person who has aggressive impulses becomes a soldier.
            \end{coloredlist}
            \item \cyanit{Regression}
            \begin{coloredlist}
                \item Retreating to a more infantile psychosexual stage, where some psychic energy remains fixated.
                \item Example: A person who is under stress begins to suck his thumb.
            \end{coloredlist}
            \item \cyanit{Introjection}
            \begin{coloredlist}
                \item Taking in and swallowing the values and standards of others.
                \item Example: A person who is abused by a parent becomes an abusive parent.
            \end{coloredlist}
            \item \cyanit{Identification}
            \begin{coloredlist}
                \item Bolstering self-esteem by forming an imaginary or real alliance with some person or group.
                \item Example: A person who is insecure identifies with a famous person.
            \end{coloredlist}
            \item \cyanit{Compensation}
            \begin{coloredlist}
                \item Masking perceived weaknesses or developing certain positive traits to make up for limitations.
                \item Example: A person who is not athletic becomes a scholar.
            \end{coloredlist}
        \end{coloredlist}
        \item \textbf{Psychosexual Stages}
        \begin{coloredlist}
            \item \cyanit{Oral}
            \begin{coloredlist}
                \item \cyanit{Oral Passive}: trusting, gullible, passive, and needy.
                \begin{coloredlist}
                    \item Forceful feeding.
                    \item Underfed
                \end{coloredlist}
                \item \cyanit{Oral Aggressive}: sarcastic, aggressive, envious, and exploitative.
                \begin{coloredlist}
                    \item Overfed
                \end{coloredlist}
            \end{coloredlist}
            \item \cyanit{Anal}
            \begin{coloredlist}
                \item \cyanit{Anal Retentive}: stingy, orderly, rigid, and obsessive.
                \begin{coloredlist}
                    \item Harsh toilet training
                \end{coloredlist}
                \item \cyanit{Anal Expulsive}: messy, wasteful, destructive, and hostile.
                \begin{coloredlist}
                    \item Lenient toilet training
                \end{coloredlist}
            \end{coloredlist}
            \item \cyanit{Phallic}
            \begin{coloredlist}
                \item Abdnoral family set-up leading to unusual relationship with mother/father.
                \begin{coloredlist}
                    \item Vanity, self-obsession, sexual anxiety, inadequacy, inferiority, and envy.
                \end{coloredlist}
            \end{coloredlist}
            \item \cyanit{Latency}
            \begin{coloredlist}
                \item After the torment of sexual impulses of the preceding years, this period is relatively quiescent. Sexual interests are replaced by interests in school, playmates, sports, and a range of new activities. 
            \end{coloredlist}
            \item \cyanit{Genital}
            \begin{coloredlist}
                \item Old themes of the phallic stage are revived. The person seeks to establish a mature sexual relationship with a partner.
            \end{coloredlist}
        \end{coloredlist}
    \end{coloredlist}

    \section{Erik Erikson}

    \begin{coloredlist}
        \item Also focused on unconscious influences.
        \item Increased emphasis on Ego instead of Id.
        \item More focused on present.
    \end{coloredlist}