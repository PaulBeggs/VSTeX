\section{Carl Rogers}

\begin{coloredlist}
    \item One of the founders of the Humanistic approach
    \item First person to conduct major studies on therapy using quantitative methods.
    \item Applied the Person-Centered Approach to topics areas outside of counseling including world peace, education, and politics.
\end{coloredlist}

\section{Assumptions}

\begin{coloredlist}
    \item People are essentially good.
    \item People have a tendency to self-actualize under the right conditions.
    \item Clients are the expert of their own inner experience.
\end{coloredlist}

\section{Role of the Therapist}

\begin{coloredlist}
    \item Three therapist attributes create a growth-promoting environment of clients:
    \begin{coloredlist}
        \item Genuineness.
        \item Unconditional positive regard.
        \item Accurate Empathetic Understanding.
        \item These are often called WEG (Warmth, Empathy, Genuineness).
    \end{coloredlist}
\end{coloredlist}

\section{Maslow's Humanistic Theory}

\begin{coloredlist}
    \item Hierarchy of Needs
    \begin{coloredlist}
        \item Physiological needs.
        \begin{coloredlist}
            \item Air, water, food, shelter, etc.
        \end{coloredlist}
        \item Safety needs.
        \begin{coloredlist}
            \item Security, stability, freedom from fear.
        \end{coloredlist}
        \item Love and belongingness needs.
        \begin{coloredlist}
            \item Affection, acceptance, friendship.
        \end{coloredlist}
        \item Esteem needs.
        \begin{coloredlist}
            \item Self-respect, respect from others, competence, confidence.
        \end{coloredlist}
        \item Self-actualization.
        \begin{coloredlist}
            \item Realizing one's potential, self-fulfillment, seeking personal growth and peak experiences.
        \end{coloredlist}
    \end{coloredlist}
\end{coloredlist}
