\section{Fritz Perls}

\begin{coloredlist}
    \item Born in 1893 in Berlin, Germany
    \item After serving in WWI, Fritz studied medicine and provided treatment to soldiers with brain injuries.
    \item He fled Germany in 1933 and moved to South Africa.
    \item In 1946, he moved to New York City and began working with Karen Horney.
    \item Coauthored a book on Gestalt Therapy and founded the New York Institute for Gestalt Therapy.
\end{coloredlist}

\section{Nature of Humanity}

\begin{coloredlist}
    \item Basic assumption about people is that they have the capacity to self-regulate when they are aware of what is happening around them.
    \item The view is rooted in existentialism and phenomenological theories.
    \item People need to ``re-own'' and unify all parts of themselves, including the parts they have disowned or denied.
\end{coloredlist}

\section{Goals for Gestalt Therapy}

\begin{coloredlist}
    \item Focuses on the here and now.
    \item Clients need to gain awareness.
    \begin{coloredlist}
        \item Defined as knowing the environment, themselves, accepting themselves, and being able to make contact (interact with others and nature without losing themselves).
        \item Increased awareness itself is curative.
    \end{coloredlist}
\end{coloredlist}

\section{Role of the Therapist}

\begin{coloredlist}
    \item The work of therapy is done by the client.
    \item The therapists' job is to create a climate where clients can try new ways of being and behaving.
    \item Working in the here and now with an ``I/Thou'' relationship.
    \item Fritz Perls methodology: The therapist is directive and confrontational.
\end{coloredlist}