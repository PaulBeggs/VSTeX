\begin{center}
    \textbf{We're starting with three studies:}
\end{center}

\begin{enumerate}
    \item \textbf{Study 1:} Blinking:
    \begin{coloredlist}
        \item Three levels of blinking:
        \begin{coloredlist}
            \item Reflexive blinking. Ex: When a puff of air is directed at the eye.
            \item Voluntary blinking. Ex: When you're asked to blink.
            \item Endogenous blinking. Meaning: ``originating from or due to internal causes.''
        \end{coloredlist}
        \item \cyanit{Endogenous blinking} is the focus of this study.
        \begin{coloredlist}
            \item Endogenous blinks occur during reading or speaking and reflect changes of attention and changes in thought processes. The more attention required by a visual task; the fewer endogenous blinks occur.
            \item More attention required is associated with fewer endogenous blinks. Especially for visual tasks.
            \item \textbf{The harder the tasks \(\rightarrow\) the fewer the blinks.}
            \item Even when a task is not visual, there is a decrease in endogenous blink rate (EBR) during a difficult task followed by flurry of blinks when task is over.
            \item \textbf{But wait!}
            \begin{coloredlist}
                \item EBR has been shown to increase when a cognitive secondary task is performed concurrently, and the cognitive task does not involve visual attention.
            \end{coloredlist}
            \item \textbf{WHY?}
            \begin{coloredlist}
                \item EB is a dopaminergic activity.
                \item Dopamine plays a big role in selective attention.
            \end{coloredlist}
        \end{coloredlist} 
        \item Through this study, we learned that endogenous blinking (DV) is affected by cognitive load (IV)
    \end{coloredlist}
    \item \textbf{Study 2:} Cartoon Judgement:
    \begin{coloredlist}
        \item Group 1 and 2 membership.
        \item Follow group instructions then rate the 3 cartoons that follow on scale from 1-10.
        \begin{coloredlist}
            \item 1 is NOT funny
            \item 10 is VERY funny
            \item Answers (Lips = Pen in lips; Teeth = Pen in teeth):

            \begin{tabular}[htbp]{ccccc}
                \toprule
                \textbf{Groups} & \textbf{Pic 1} & \textbf{Pic 2} & \textbf{Pic 3} & \textbf{Average} \\ \midrule
                \textbf{Lips} & 3 & 3 & 4 & \(3\frac{1}{3}\) \\ \midrule
                \textbf{Teeth} & 4 & 4 & 3 & \(3\frac{2}{3}\) \\ \midrule
                \textbf{Stretch} & 4 & 5 & 6 & 5 \\ \midrule
                \textbf{J. Jacks} & 4 & 2 & 3 & 3 \\ \bottomrule
            \end{tabular}

        \end{coloredlist}
        \item \textbf{Facial Feedback Hypothesis}
        \begin{coloredlist}
            \item Selective activation or inhibition of facial muscles has a strong impact on emotional responses to stimuli.
            \item Zygomatic major muscle.
            \begin{coloredlist}
                \item When we had the pen in our teeth, we were activating the zygomatic major muscle.
                \item This muscle is responsible for smiling.
            \end{coloredlist}
            \item Our data supported this hypothesis with a probability of \(p < 0.02\).
        \end{coloredlist}
        \item \textbf{Arousal}
        \begin{coloredlist}
            \item Increased heart rate in many emotions.
            \item Heart rate and attraction
            \begin{coloredlist}
                \item 1973 Dutton and Aron
                \begin{coloredlist}
                    \item Shaky high bridge vs. low stable bridge.
                    \item Woman on the other side who is asking questionnaire questions (faux DV).
                    \item She gave her phone number to the guys once they got done answering the questions.
                    \item The actual DV was the amount of phone calls she received and the sexual content in questionnaire answers.
                    \item The high bridge group had more sexual content in their messages.
                \end{coloredlist}
                \item 15 minutes of physical activity, then rate attractiveness of potential mates.
            \end{coloredlist}
        \end{coloredlist}
        \item 
    \end{coloredlist}
\end{enumerate}

\textbf{Psychophysiology:} Behavioral, cognitive, emotional, and social events are all mirrored in physiological processes. \\

The idea is that we can get a peep into your psychology by looking at what your biology is doing. \\

\textbf{Sleep:} EEG (Electroencephalogram; measuring brain activity), EOG (Electrooculogram; measuring eye movement), EMG (Electromyography; measuring muscle movement), ERP (measuring event-related potential). \\

Respiration, GSR (EDA), Blood flow, Blood pressure, heart rate 