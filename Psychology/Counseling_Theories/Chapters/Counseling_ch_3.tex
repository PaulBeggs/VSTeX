\section{Codes Versus Laws}

\subsection{Mandatory Codes}

\begin{coloredlist}
    \item Ethical functioning at the minimum level.
\end{coloredlist}

\subsection{Aspirational Ethics}

\begin{coloredlist}
    \item Doing what is in the best interest of the clients.
    \item Going above and beyond the minimum requirements.
    \item Understanding the spirit of the code and the principles on which the code is based.
\end{coloredlist}

\subsection{Steps for Ethical Decision-Making}

\begin{coloredlist}
    \item Identify the problem.
    \item Look at the relevant ethics codes and laws.
    \item Seek consultation.
    \item Brainstorm various possible courses of action.
    \item List consequences.
    \item Decide and document the reasons for your actions.
    \item To a degree, include the client in all phases of your ethical decision-making process.
    \item When clinicians are unsure of what action to take, they may contact their liability insurers to discuss the best plan of action to ensure ethical decision-making.
\end{coloredlist}

\section{Informed Consent}

\begin{coloredlist}
    \item Clients must consent to treatment.
    \begin{coloredlist}
        \item Uninformed consent \textendash what problems could arise from this?
        \begin{coloredlist}
            \item Patients could be under the impression that their treatment would be one way, but is actually another.
        \end{coloredlist}
        \item Informed consent are conditions that you disclose to the patient about the treatment they are about to receive.
        \item These conditions include:
        \begin{coloredlist}
            \item The purpose of the treatment.
            \item Potential risks and benefits.
            \item Alternatives to the treatment.
            \item Disclosure about the right to refuse treatment.
            \item Any other information that the client may need to make an informed decision.
        \end{coloredlist}
    \end{coloredlist}
\end{coloredlist}

\section{Confidentiality}

Situations in which your therapist can legally break confidentiality:
\begin{coloredlist}
    \item If the client is at risk of:
    \begin{coloredlist}
        \item Harming themselves.
        \item Harming others.
        \item Being harmed by others.
    \end{coloredlist}
    \item If there is child or elder abuse or neglect.
    \item If the client has a lawsuit against them (and the therapist is subpoenaed).
    \item The client can also sign a release of information form.
    \item Involuntary hospitalization.
\end{coloredlist}

\section{Evidence-Based Practice}

\begin{coloredlist}
    \item Ethical treatment is effective treatment.
    \item How do you know that what you're doing is helping your clients?
    \begin{coloredlist}
        \item You can use evidence-based practice.
        \begin{coloredlist}
            \item Balancing treatment and relationship.
        \end{coloredlist}
        \item Outcome tracking.
        \item Problems with relying on intuition.
    \end{coloredlist}
\end{coloredlist}

\section{Multiple Relationships}

\begin{coloredlist}
    \item These occur when the therapist has a relationship with the client outside the therapeutic relationship.
    \item Can be sexual or non-sexual.
    \item How could these cause ethical concerns for your clients?
    \item Prevention is key. Set healthy boundaries from the start.
\end{coloredlist}