\begin{center}
    \textbf{1. What does `personality' mean to you? What causes adult individuals to have the same personality that they have?}
\end{center}

\begin{itemize}
    \item Personality is the \textit{unique} set of characteristics that make up an individual. This includes the way they think, feel, and behave. It is the way that they interact with the world around them.
    \item The causes of personality are varied and complex. Some of the factors that contribute to personality include:
    \begin{itemize}
        \item Genetics: Some personality traits are inherited from our parents. For example, if a parent is shy, their child may also be shy.
        \item Environment: The environment in which a person grows up can have a significant impact on their personality. For example, a person who grows up in a loving and supportive environment may be more likely to develop a positive and outgoing personality.
        \item Life experiences: The experiences that a person has throughout their life can also shape their personality. For example, a person who has experienced trauma may develop a more anxious or withdrawn personality.
        \item Culture: The culture in which a person grows up can also influence their personality. For example, a person who grows up in a collectivist culture may be more likely to develop a personality that values cooperation and harmony.
    \end{itemize}
    \item In general, personality is thought to be a combination of genetic, environmental, and experiential factors. It is a complex and dynamic trait that is shaped by a variety of influences.
\end{itemize}

\begin{center}
    \textbf{2. Read the case below and explain in your own words why you think these two individuals responded so differently to what happened.}
\end{center}

Because the two individuals were raised in similar environments, were the same age, and had the same upbringing from their shared parents, the only factor that has varience would be the the life experiences. Perhaps Paul had learned something earlier in his life about counseling and its benefits. This would predispose him to seek out counseling after the traumatic event. On the other hand, perhaps Gordon had an early experience with acohol and learned that it was a way to cope with stress. 

However, there could be an entirely different reason for a difference in their responses: personality. It's possible that Paul is more of an antisocial type of person, and was never exposed to alcohol because he prefers not to party. Gordon, on the other hand, may be more of an extrovert and enjoys socializing and drinking. Because Gordon knows about alcohol and its effects, he may have turned to it as a way to cope with the stress of the traumatic event.