\begin{coloredlist}
    \item When the ego is struggling to meet the demands of the Id and Superego, a person often resorts to using defense mechanisms to cope.
    \item Examples: Repression, denial, sublimation (pg 40 in text).
\end{coloredlist}

\section{Research on Dreams}

\begin{coloredlist}
    \item Frued stated unconscious impluses can't be suppressed forever, and one way they can come out is in our dreams.
    \item Research shows that participants deprived of sleep experience show anxiety when distressing events occur.
    \item Trauma survivors who suppress thoughts about trauma in the day have more trauma related dreams.
    \item In a study about dreams, participants were recruited to a study and were asked to either 
    \begin{enumerate}
        \item Journal about a person they were fond of or
        \item repress it and write something else.
    \end{enumerate}
    \item Results show that participants who repressed their thoughts had more dreams about their `target person' than participants who journaled about them.
\end{coloredlist}

\section{Research on Defense Mechanisms}

\begin{coloredlist}
    \item Research shows that the defense mechanisms that we use change as we age.
    \item Young chickens are more likely to use denial.
    \item Projection increases in older childhood while denial decreases. 
    \item More sophisticated defense mechanisms develop as we age, such as sublimation.
    \item Adults who continue to rely on denial and projection tend to have problems such as increased hostility and alcohol abuse.
\end{coloredlist}

\section{Research on Humor}

\begin{coloredlist}
    \item Why do we laugh at acts of aggression, injuries, or another person's embarrassment?
    \item Freud thought:
    \begin{coloredlist}
        \item Aggressive humor allows an outlet for expressing or experiencing subconscious aggressive urges.
        \item Believed aggressive humor allowed for catharsis (reduction in tension).
    \end{coloredlist}
    \item In a research study, participants reported that sexual and aggressive jokes were funnier than others jokes.
    \item People find jokes funnier if they poke at a group or person they don't like.
    \item A study in the 80s showed that women found jokes insulting men funnier and vice versa.
\end{coloredlist}

\section{Research on Hypnosis}

\begin{coloredlist}
    \item Freud believed that hypnosis could be used to access the unconscious mind.
    \item People who view hypnosis with distrust are harder to hypnotize.
    \item Those who become more absorbed in activities and who are prone to daydream are more responsive to hypnosis.
    \item When participants were told that they will forget being hypnotized, participants often cannot remember their experience.
    \item Why don't people remember being hypnotized after it happens?
\end{coloredlist}