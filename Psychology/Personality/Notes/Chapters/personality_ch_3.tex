\section{Sigmund Freud}

\subsection{Ego, Superego, and Id}

\begin{coloredlist}
    \item \cyanit{Ego} is the rational part of the personality that mediates between the demands of the id and the superego.
    \item \cyanit{Superego} is the preconscious level our personality that represents the conscience, the moral part of the personality.
    \item \cyanit{Id} is the unconscious level of our personality that contains our primitive drives and operates on the pleasure principle.
    \item When the Ego is struggling to meet the demands of the Id and Superego, a person often resorts to using defense mechanisms to cope.
\end{coloredlist}

\subsection{Psychosexual Stages of Development}

\begin{coloredlist}
    \item \cyanit{Oral Stage} (0-18 months) is the stage where the infant's pleasure centers on the mouth.
    \item \cyanit{Anal Stage} (18 months to 3 years) is the stage where the
    child's pleasure focuses on the anus and from elimination.
    \item \cyanit{Phallic Stage} (3-6 years) is the stage where the child's pleasure focuses on the genitals.
    \item \cyanit{Latency Stage} (7-11 years) is the stage where the child represses sexual interest and develops social and intellectual skills.
    \item \cyanit{Genital Stage} (12 years and older) is the stage where the adolescent's sexual interest matures into romantic attraction toward others.
\end{coloredlist}