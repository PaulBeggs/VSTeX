\textbf{Chapter Outline:}
\begin{coloredlist}
    \item \hyperlink{hypothesis testing}{The Hypothesis-Testing Approach}
    \item The Case Study Method
    \item Statistical Analysis of Data
    \item Personality assessment
    \item Summary
\end{coloredlist}

\section{The Hypothesis-Testing Approach}\hypertarget{hypothesis testing}{}   
\begin{coloredlist}
    \item \textbf{Theory}: General statement about the relationship between constructs or events.
    \begin{coloredlist}
        \item Differ in the range of events or phenomena they cover.
    \end{coloredlist}
    \item Characteristics of a good theory:
    \begin{coloredlist}
        \item Parsimonious \textendash Explains the phenomenon in simple terms.
        \item Useful \textendash Generates testable hypotheses.
    \end{coloredlist}
    \item \textbf{Hypothesis}: Formal prediction about the relationship between or more variables that is logically derived from a theory.
    \item A theory is not accepted if empirical investigations consistently fail to confirm predictions.
\end{coloredlist}

\subsection{Types of Experimental Variables}

\begin{coloredlist}
    \item \textbf{Independent variable}: Manipulated by the experimenter.
    \item \textbf{Dependent variable}: Measured by the experimenter.
    \item \textbf{Non-Manipulated independent variable}: 
    \begin{coloredlist}
        \item Exists without the researcher's intervention.
        \item Investigator does not randomly assign participants to a conditions
        \item Research cannot assume the participants in the two groups are identical.
        \item Difficult to find cause-and-effect relationships.
    \end{coloredlist}
    \item \textbf{Manipulated independent variable}:
    \begin{coloredlist}
        \item Begins with numerous participants.
        \item Randomly assigns participants to experimental groups.
        \item Researcher can assume that all the differences will be evened out.
        \item Random assignment increases confidence in causation relationships.
    \end{coloredlist}
\end{coloredlist}

\subsubsection{Interaction of Experimental Variables}

\begin{coloredlist}
    \item Research often has more than one independent variable.
    \item Interaction:
    \begin{coloredlist}
        \item How one independent variable affects the dependent variable depends on the other independent variable.
    \end{coloredlist}
\end{coloredlist}

\subsection{Predictions}

\begin{coloredlist}
    \item Accurate predictions can be made if a scientist has a legitimate theory.
    \item Purpose of research is to provide support for a hypothesis.
    \item Researchers:
    \begin{coloredlist}
        \item Generate a theory
        \item Make a hypothesis
        \item Collect data that supports or opposes the hypothesis
    \end{coloredlist}
    \underline{Unpredicted findings} by the researchers are the basis for future hypotheses and further research. 
\end{coloredlist}

\subsection{Replication}

\begin{coloredlist}
    \item Repetition of the research.
    \begin{coloredlist}
        \item Example: Pharmaceutical company finds their new medication treats depression in adult men.
    \end{coloredlist}
    \item Examines participant populations different from those used in the original research,
    \item Helps to determine whether the effect applies to larger number of people or is limited to the kind of individuals used in the original sample.
    \item Determining the strength of an effect by how often it is replicated is difficult because of the \textbf{\textit{File Drawer Problem}}.
    \begin{coloredlist}
        \item Harder to get published when you didn't find significant results.
        \item Researchers publish and report research only when they find significant effects.
    \end{coloredlist}
\end{coloredlist}

\section{The Case Study Method}

\begin{coloredlist}
    \item \textbf{Case study}: In-depth analysis of an individual, group, or event.
    \item \textbf{Case study method}:
    \begin{coloredlist}
        \item Involves the collection of data from a single individual.
        \item Can be used to study a single individual or a group.
    \end{coloredlist}
\end{coloredlist}

\subsection{Limitations of Case Study Method}

\begin{coloredlist}
    \item Determining cause-and-effect relationships.
    \item \textbf{Generalizability}:
    \begin{coloredlist}
        \item Difficulty in generalizing from a single case to a larger population.
    \end{coloredlist}
\end{coloredlist}

\subsection{Strengths of Case Study Method}

\begin{coloredlist}
    \item Offers insight into the richness of a person's life.
    \item Valuable for generating hypotheses about the nature of human personality.
    \item Acts as a useful research tool. Appropriate in examining a rare case.
\end{coloredlist}

\section{Statistical Analysis of Data}

\begin{coloredlist}
    \item Types of statistical tests appropriate for different types of data and research designs.
    \begin{coloredlist}
        \item Analysis of variance (ANOVA).
        \item \(\chi^{2}\) test.
        \item Correlational coefficients.
    \end{coloredlist}
\end{coloredlist}

\section{Reliability}

\begin{coloredlist}
    \item Extent to which a test measures consistently.
    \begin{coloredlist}
        \item Determined by calculating test-retest reliability coefficient.
    \end{coloredlist}
    \item Internal consistency
    \begin{coloredlist}
        \item All items on the test measure the same thing.
        \item \textbf{Internal consistency reliability coefficient}:
        \begin{coloredlist}
            \item High coefficient indicates that all items on the test measure the same thing.
            \item Low coefficient suggests items are measuring more than one concept.
        \end{coloredlist}
    \end{coloredlist}
\end{coloredlist}

\section{Validity}

\begin{coloredlist}
    \item Extent to which a test measures what it is supposed to measure.
    \item Easy to determine for some kinds of tests.
    \item Face validity:
    \begin{coloredlist}
        \item Way to decide whether a test measures what it is says it measures is to look at the test items.
    \end{coloredlist}
    \item Congruent validity:
    \begin{coloredlist}
        \item Extent to which scores from the test correlation with other measures of the same construct.
        \item Otherwise known as convergent validity.
    \end{coloredlist}
    \item Discriminant validity:
    \begin{coloredlist}
        \item Extent to which a test score does not correlate with the scores of theoretically unrelated measures.
    \end{coloredlist}
    \item Behavioral validation:
    \begin{coloredlist}
        \item Step in determining the construct validity of a test.
        \item Test scores predicting relevant behavior is important.
        \item Usefulness of the test must be questioned if the test scores cannot predict behavior.
    \end{coloredlist}
\end{coloredlist}