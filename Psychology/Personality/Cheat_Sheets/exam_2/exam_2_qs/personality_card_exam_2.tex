\documentclass[10pt]{book} 
% \usepackage[width=5in, height=3in, centering]{geometry} 
\usepackage{geometry}
\usepackage{amsmath, amsfonts, amssymb}

\usepackage{lipsum}
\usepackage{multicol}
\pagestyle{empty}
\usepackage{graphicx}

\usepackage{tcolorbox}
\tcbuselibrary{poster}
\usepackage{xcolor}
\colorlet{darkgreen}{green!85!black}
\newcommand{\topic}[1]{\textcolor{cyan}{\textbf{#1}}}
\newcommand{\subtopic}[1]{\textcolor{red}{#1}}
\newcommand{\shortanswer}[1]{\textcolor{darkgreen}{\textbf{#1}}}

\begin{document}

\begin{center}
\begin{tcbposter}[
  poster = {
    showframe,
    columns=1,
    rows=1,
    spacing=0mm,
  },
]

\posterbox[
top=1pt,
bottom=1pt,
left=1pt,
right=1pt,
tile,
colback=white,
]{
  column=1, row=1,
}{
  % \tiny
  % \scriptsize
 
  (NOTE: Introversion = IV; extraversion = EV)
  \quad \topic{Eyseck’s Theory}: Main Traits: IV-EV, Stability-Neuroticism, Psychoticism, $\uparrow$Psycho $\Rightarrow$ Egocentric, aggressive, and impersonal. 
  Found these traits through factor analysis. \quad \topic{Temperament:} Comes from different personalities in children. \subtopic{Future:} Depends on the genetic disposition and the environment. Three main ones are Emotionality, Activity, and Socialibility. \quad \topic{Biology Base?:} \subtopic{1:} Consistency of EV/IV over time \subtopic{2:} Cross-cultural records found some super factors in other countries. This makes the case that its biology, not environment. \subtopic{3:} Genetic makeup helps determine placement in EV/IV. \shortanswer{Challenges in testing genetics:} There is no way to assign people to genes or environments, so it is difficult to say whether personality is caused by genes or environment. Researchers can do family studies. Twin studies specifically. They can see whether the same genetics produce different results even when in the same home environment.  \quad \topic{Biological trait testing:} DNA sequencing, neuroimaging, heart rate/respiration/blood pressure, and skin conductance, hormones, and neurotransmitters. \quad \topic{Evolutionary theory of personality traits:} Uses natural selection to explain why people react to things in certain ways. Traits that have aided in survival up to nw are theorized to be adaptive/aids in survival and are passed down in genes. Psychological mechanisms that have been favorable over time stay in humans. Compassion, stranger danger, anger, desire to obey. \quad \topic{Research on heritability:} Primary way of studying is through monozygotic and dizygotic twins. Stronger correlation on Big 5 traits between MZ than compared to DZ twins. \quad \topic{EV/IV research:} MZ twins carry stronger correlations of personality pairs compared to DZ. MZ twins are more likely to both be the same when it comes to EV/IV than DZ twins. Genes are a strong indicator but not whole dictator. \quad \topic{EV/IV arousal:} EV require more stimulation to be engaged and focused. They need continuous stimulation, otherwise they become bored, unfocused and unmotivated. IV require less stimulation to be engaged and prefer quite spaces. \subtopic{EV generally experience more happiness because} they social more than IV. EV react stronger to + feedback; homophones come back positive; EVs are more impulsive; more likely to seek out tasks they think will make them happy; sensitive to reward can also have a bad influence. \topic{Evolutionary theory \& mate selection:} We select mates who will likely be successful at reproducing and child raising. \subtopic{Females:} Need continued support from partner; and concerned w bearing and raising child.
}
\end{tcbposter}
\end{center}

\newpage

\begin{center}
\begin{tcbposter}[
  poster = {
    % showframe,
    columns=1,
    rows=1,
    spacing=0mm,
  },
]

\posterbox[
top=1pt,
bottom=1pt,
left=1pt,
right=1pt,
tile,
colback=white,
]{
  column=1, row=1,
}{
  % \scriptsize
  
   \subtopic{Males:} Concerned with passing genes to next generation. Need offspring to be their own. \quad \topic{Central elements for humanistic viewpoint:} An emphasis on personal responsibility. 
   Freedom to choose what they will do. An emphasis on the here and now: live life as it happens. A focus on the individual experiences. Clients are experts and help themselves. An emphasis on personal growth. Motivated to grow toward an ultimately satisfying state, not just short-lived happiness. \subtopic{Carl Rogers:} Person-centered therapist; he was a humanist pioneer. Believed people are generally good and strive to grow if given the right environment. Believed people need unconditional love to grow well. \quad \shortanswer{Conditional Positive Regard:} Love that is dependent on the way someone behaves. Good behavior = Love; Bad behavior = love withdrawn. Only accept the parts deemed acceptable. It negatively affects us because we deny that we possess these unflattering characteristics and become less and less fully functioning. \quad \topic{Maslow's hierarchy of needs:} (Ordered from lowest \(\rightarrow\) highest) \textit{Physiological} (food, shelter, warmth) \(\rightarrow\) \textit{Safety} (security, health) \(\rightarrow\) \textit{Love \& belonging} (connection) \(\rightarrow\) \textit{Esteem} (respect) \(\rightarrow\) \textit{self-actualization} (attaining your full potential). \quad \topic{Optimal experience/flow:} Something that demands all your attention. Real pleasurew comes from the process, not the achievement. Demanding \& enjoyable. \quad \topic{Person-centered therapy:} The client understands themself better than the therapist. Client is responsible for change. Therapist provides the atmosphere wherein clients can grow. \quad \topic{Q-Sort technique:} A tool to show client is more fully functioning or closer to self-actualization. Categorize cards in how they describe you from extremely uncharacteristic to extremely characteristic. Then take it again and make it your ideal self. Then compare. Good is to accept self. \quad \topic{Humanism:} \subtopic{Strengths:} Focuses on the healthy side of personality. Applies to other fields. \subtopic{Cons:} Free will is difficult to study scientifically. Hard to define self-actualization. May not benefit all cultures.
}
\end{tcbposter}
\end{center}

\end{document}