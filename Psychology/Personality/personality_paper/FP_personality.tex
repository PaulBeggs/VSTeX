\documentclass[stu]{apa7}
\usepackage{amsmath}
\usepackage{amssymb}
\usepackage{enumitem}
\usepackage{geometry}
\usepackage[style=apa,uniquename=false,backend=biber]{biblatex}
\usepackage{hyperref}
\usepackage{xcolor}
\definecolor{horange}{HTML}{f58026}
\hypersetup{
    colorlinks=true,
    linkcolor=horange,
    filecolor=horange,      
    urlcolor=horange,
    citecolor=horange
}
\usepackage{draculatheme}
\addbibresource{references.bib}

\title{Personality Development: A Dynamic Framework}
\shorttitle{Personality Development}
\author{Paul Beggs}
\authorsaffiliations{Hendrix College}
\course{PSYC 370: Personality}
\professor{Dr.\ Sarah Root}
\duedate{\today}

\begin{document}
\maketitle

At the start of the semester, I defined personality as the unique constellation of characteristics that shape an individual's thinking, feeling, and behaving. I believed that personality arises from a combination of genetic inheritance, environmental context, life experiences, and cultural influences. Specifically, I posited that (a) genetic predispositions contribute to trait-like consistencies (e.g., shyness passed from parent to child), (b) one's childhood environment (supportive vs. neglectful) molds affective style, (c) significant life events (e.g., trauma) imprint patterns of reactivity, and (d) cultural norms (collectivist vs. individualist) steer one's values and interpersonal tendencies. Thus, I viewed personality as a relatively stable, multifactorial product of nature and nurture.

\section{Evolved Perspective on Personality Development}

Over the course of the class, my understanding of personality deepened from a static ``trait + context'' model to a dynamic, lifespan-oriented, interactionist framework (Burger, 2018). I now conceptualize personality as a complex system comprising multiple interrelated components:

\subsection{Trait Systems Embedded in Biology}

Contemporary research supports the Five-Factor Model as rooted in biological processes (McCrae \& Costa, 2008). % Get this information from the slides (Biological perspective research, slide twin study of heritability of the big 5)
% McCrae, R. R., & Costa, P. T. (2008).
% The five-factor theory of personality.
% In O. P. John, R. W.
% Robins, & L. A. Pervin (Eds.),
% Handbook of personality (3rd ed.,
% pp. 159–181). New York:
% Guilford.
 Neuroimaging and twin studies reveal that approximately 40–60\% of trait variance (e.g., extraversion, neuroticism) is heritable, suggesting robust genetic underpinnings (McCrae \& Costa, 2008). These findings indicate that personality traits have substantial biological foundations, with genetic factors providing baseline propensities for behavior and emotional reactivity.

Twin and adoption studies estimate substantial heritability for core traits (McCrae \& Costa, 2008). Neurobiological correlates (e.g., amygdala reactivity linked to neuroticism) further substantiate the biological basis of personality. These findings suggest that traits have a biological foundation that contributes to their relative stability across situations and time.

\subsection{Social-Cognitive Mechanisms}

Bandura's concept of reciprocal determinism emphasizes that personal factors (self-efficacy beliefs), behavior, and environmental contexts continually influence one another (Bandura, 1977). %
% Bandura, A. (1977a). Self-efficacy:
% Toward a unifying theory of
% behavioral change. Psychological
% Review, 84, 191–215.
 Similarly, Mischel and Shoda's Cognitive-Affective Processing System (CAPS) posits that situational cues activate stable cognitive–affective units (e.g., goals, beliefs), producing if-then behavioral signatures rather than uniform trait expression (Mischel \& Shoda, 1995).  %
%  Mischel, W., & Shoda, Y. (1995). A
%  cognitive-affective system theory
%  of personality: Reconceptualizing
%  situations, dispositions,
%  dynamics, and invariance in personality
%  structure. Psychological
%  Review, 102, 246–268.
  This perspective highlights how personality operates through dynamic cognitive and affective processes that mediate between traits and behavior.

Bandura's work demonstrates that individuals actively shape—and are shaped by—their environments, with self-efficacy emerging as a key driver of behavioral change (Bandura, 1977). Mischel and Shoda (1995) provide empirical support for person-specific ``if-then'' patterns across contexts. This research highlights how personality operates through dynamic cognitive and affective processes that mediate between traits and behavior.

\subsection{Lifespan Plasticity and Narrative Identity}

Rather than traits being fixed by early adulthood, longitudinal studies show modest but meaningful trait change across the lifespan, influenced by role transitions (e.g., marriage, career), intentional self-regulation, and cultural scripts (Roberts et al., 2006). Moreover, McAdams's narrative approach highlights that individuals construct evolving life stories to create coherent selves, integrating past, present, and anticipated futures into their personality (McAdams, 2001). This narrative dimension adds an important layer of meaning-making to personality development.

Meta-analytic findings reveal mean-level increases in conscientiousness and agreeableness from young adulthood to middle age, reflecting role-related maturation (Roberts et al., 2006). Self-reported life-story coherence predicts psychological well-being, linking narrative identity to adaptive functioning (McAdams, 2001). These findings demonstrate that personality continues to develop throughout adulthood in systematic ways.

\subsection{Cultural and Contextual Embeddedness}

Culture not only shapes trait endorsements but also the very conceptualization of personality. Indigenous personality research reveals culturally specific trait dimensions beyond the Big Five (Cheung et al., 2001), reminding us that personality science must account for cultural scripts and meanings. This cultural embeddedness suggests that personality is not merely influenced by culture but fundamentally constituted within particular cultural contexts.

Cheung et al. (2001) identified an ``Interpersonal Relatedness'' factor in Chinese samples, underscoring that cultural contexts yield both universal and culture-specific personality dimensions. This research emphasizes the importance of cultural context in shaping the expression and conceptualization of personality traits.
Conclusion

\section{Conclusion}

My current view positions personality as a dynamic, multi-layered system shaped by biological predispositions, social-cognitive processes, narrative construction, and cultural embedding. While genetic factors provide baseline propensities, individuals continuously interpret, enact, and re-narrate their traits through life transitions and cultural contexts. This enriched framework moves beyond a static trait model to embrace developmental, situational, and cultural complexities. This integrated perspective has important implications for both personality research and clinical applications, suggesting that personality should be understood as both stable and malleable, individually unique and culturally embedded.
\end{document}