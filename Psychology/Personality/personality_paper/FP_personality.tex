\documentclass[stu]{apa7}
\usepackage{amsmath}
\usepackage{amssymb}
\usepackage{enumitem}
\usepackage{geometry}
\usepackage[style=apa,uniquename=false,backend=biber]{biblatex}
\usepackage{hyperref}
\usepackage{xcolor}
\definecolor{horange}{HTML}{f58026}
\hypersetup{
    colorlinks=true,
    linkcolor=horange,
    filecolor=horange,      
    urlcolor=horange,
    citecolor=horange
}
% \usepackage{draculatheme}
\addbibresource{references.bib}

\title{Personality Development: A Dynamic Framework}
% \shorttitle{Personality Development}
\author{Paul Beggs}
\authorsaffiliations{Hendrix College}
\course{PSYC 370: Personality}
\professor{Dr.\ Sarah Root}
\duedate{\today}

\begin{document}
\maketitle

At the start of the semester, I defined personality as the unique constellation of characteristics that shape an individual's thinking, feeling, and behaving. I believed that personality arises from a combination of genetic inheritance, environmental context, life experiences, and cultural influences. Specifically, I posited that (a) genetic predispositions contribute to trait-like consistencies (e.g., shyness passed from parent to child), (b) one's childhood environment (supportive vs. neglectful) molds affective style, (c) significant life events (e.g., trauma) imprint patterns of reactivity, and (d) cultural norms (collectivist vs. individualist) steer one's values and interpersonal tendencies. Thus, I viewed personality as a relatively stable, multifactorial product of nature and nurture.

\section{Evolved Perspective on Personality Development}

Over the course of the class, my understanding of personality deepened from a static ``trait + context'' model to a dynamic, lifespan-oriented, interactionist framework \parencite{burger2018personality}. I now conceptualize personality as a complex system comprising multiple interrelated components:

\subsection{Trait Systems Embedded in Biology}

Contemporary research supports the Five-Factor Model as rooted in biological processes \parencite{mccrae2008empirical}. Twin studies reveal that a significant portion of trait variability is attributable to genetic influences. For instance, in one particularly robust study, researchers assessed 123 monozygotic (MZ) and 127 dizygotic (DZ) twin pairs using the NEO-PI-R, a measure of the Big Five traits \parencite{jang1996heritability}. Results showed significantly stronger correlations in personality traits among MZ twins compared to DZ twins, indicating that genetic similarity accounts for much of the personality overlap. More specifically, heritability estimates suggest that neuroticism is approximately 41\% heritable, extraversion 53\%, openness to experience 61\%, agreeableness 41\%, and conscientiousness 43\% \parencite{jang1996heritability}. These data support the idea that while environmental influences are important, genetics play a substantial role in shaping our enduring personality traits. These findings indicate that personality traits have substantial biological foundations, with genetic factors providing baseline propensities for behavior and emotional reactivity.

\subsection{Social-Cognitive Mechanisms}

Bandura's concept of reciprocal determinism emphasizes that personal factors (self-efficacy beliefs), behavior, and environmental contexts continually influence one another \parencite{bandura1977self}. Similarly, Mischel and Shoda's Cognitive-Affective Processing System (CAPS) posits that situational cues activate stable cognitive–affective units (e.g., goals, beliefs), producing if-then behavioral signatures rather than uniform trait expression \parencite{mischel1995cognitive}. This perspective highlights how personality operates through dynamic cognitive and affective processes that mediate between traits and behavior.

Experimental research supports these social-cognitive frameworks. Bandura's work demonstrates that individuals actively shape—and are shaped by—their environments, with self-efficacy emerging as a key driver of behavioral change \parencite{bandura1977self}. Meanwhile, \textcite{mischel1995cognitive} provide empirical evidence for person-specific ``if-then'' patterns across contexts, showing how the same individual displays consistent but contextualized behavioral signatures in response to different situational cues.

\subsection{Lifespan Plasticity and Narrative Identity}

Rather than traits being fixed by early adulthood, longitudinal studies show modest but meaningful trait change across the lifespan, influenced by role transitions (e.g., marriage, career), intentional self-regulation, and cultural scripts \parencite{roberts2006patterns}. Moreover, McAdams's narrative approach highlights that individuals construct evolving life stories to create coherent selves, integrating past, present, and anticipated futures into their personality \parencite{mcadams2001bad}. This narrative dimension adds an important layer of meaning to personality development.

% Meta-analytic findings reveal mean-level increases in conscientiousness and agreeableness from young adulthood to middle age, reflecting role-related maturation \parencite{roberts2006patterns}. Self-reported life-story coherence predicts psychological well-being, linking narrative identity to adaptive functioning \parencite{mcadams2001bad}. These findings demonstrate that personality continues to develop throughout adulthood in systematic ways.

\subsection{Cultural and Contextual Embeddedness}

My earlier view of personality underestimated the profound influence of culture—not only in shaping how traits are expressed, but in defining what traits even matter. Trait models like the Five-Factor Model originated in Western contexts and emphasize characteristics valued in individualistic societies, such as assertiveness and independence. However, cross-cultural research reveals important deviations in how people conceptualize the self and what they consider ideal personality traits.

For example, Americans commonly exhibit self-enhancement bias, rating themselves above average on skills and traits \parencite{taylor1989positive}. In contrast, individuals from collectivist cultures tend to avoid such self-inflation and see humility as a marker of psychological health \parencite{heine2007self}. What might be considered “low self-esteem” in Western frameworks is often a culturally appropriate form of modesty in collectivist societies. This divergence is so pronounced that American athletes, who are encouraged to express confidence and individuality, often struggle in team-centered environments like Japanese baseball clubs, where overt self-assertion is frowned upon \parencite{whiting1989you}.

Furthermore, self-esteem itself appears to be culturally malleable. \textcite{heine1999is} found that Asian individuals who had greater exposure to North American culture gradually developed higher self-esteem scores that aligned with Western norms of personal achievement. In essence, self-concept is not just individually defined—it’s socially and culturally constructed.

Cultural differences also shape what brings people satisfaction in life. In individualistic societies like the United States, happiness and emotional well-being are closely tied to life satisfaction \parencite{diener1995cross}. In contrast, individuals in collectivist cultures derive satisfaction from fulfilling social roles and adhering to group norms
\parencite{oishi2007dynamics,steger2008meaningful}. With such settings, a person’s sense of purpose and belonging—not necessarily their emotional highs—predicts how content they feel with life. These insights challenged my initial assumption that personality—and its outcomes—are universally defined. They are not; they are intricately shaped by the social script of the culture in which a person is embedded.

\section{Conclusion}

My current view positions personality as a dynamic, multi-layered system shaped by biological predispositions, social-cognitive processes, narrative construction, and cultural embedding. While genetic factors provide baseline propensities, individuals continuously interpret, enact, and re-narrate their traits through life transitions and cultural contexts. This enriched framework moves beyond a static trait model to embrace developmental, situational, and cultural complexities. This integrated perspective has important implications for both personality research and clinical applications, suggesting that personality should be understood as both stable and malleable, individually unique and culturally embedded.


\printbibliography
\end{document}