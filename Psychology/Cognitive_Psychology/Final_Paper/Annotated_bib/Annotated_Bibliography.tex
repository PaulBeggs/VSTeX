\documentclass[stu]{apa7}
\usepackage{amsmath}
\usepackage{amssymb}
\usepackage{enumitem}
\usepackage{geometry}
\usepackage[style=apa,backend=biber]{biblatex}
\usepackage{hyperref}
\usepackage{xcolor}
\definecolor{horange}{HTML}{f58026}
\hypersetup{
	colorlinks=true,
	linkcolor=horange,
	filecolor=horange,      
	urlcolor=horange,
    citecolor=horange
}
\usepackage{draculatheme}
\addbibresource{references.bib}

\title{Annotated Bibliography: The Effect of ADHD on Working Memory}
\shorttitle{The Effect of ADHD on Working Memory}
\author{Paul Beggs}
\authorsaffiliations{Hendrix College}
\course{PSYC 319: Cognitive Psychology}
\professor{Dr.\ Carmen Merrick}
\duedate{November 14, 2024}

% \abstract{
% 	TODO: Write abstract.
% }

\begin{document}

\maketitle

\section{Introduction}

For the introduction to the paper, I am planning on using \textcite{faraone_attention-deficithyperactivity_2015} \textbf{(non-empirical)} for an overview of ADHD. Then, I plan on giving an overview of how ADHD affects the different parts of cognition that we have talked about in class. 

\section{Body}

For this section, I intend on separating the various studies into their own respective subsections. Each will deal with a relevant part of working memory, and I am hoping to have a conclusion that will link all the research together. This would be the take-home message.

\begin{enumerate}
	\item The first is \textcite{fabio_working_2020} \textbf{(empirical)}. This paper is an empirical study that investigates three hypotheses:
	\begin{enumerate}
		\item children with ADHD show higher levels of DD (Delayed discounting---paradigms used to study the process leading to a choice) than control subjects,
		\item once the memory load increases, deferring a reward becomes harder for both children with ADHD and with TD (typical development), and
		\item the performances of children with ADHD are significantly worsened by the addition of a memory load compared to the control group.
	\end{enumerate}
	The researchers operationalize studying these independent variables by involving goal-directed actions during distractions and dual task paradigms. In other words, the participants were to listen and repeat an assigned sequence of numbers, while also evaluating which amount of money to take---e.g., some now, or more later. The distraction was the digit recall. Though, the results of this study should be taken with a grain of salt: It has a small sample size \((N = 32)\), and the study was conducted in Italy, where they used different tests like the ADHD Rating Scale for Teachers (SDAI). 

	\item The second is \textcite{raiker_phonological_2019} \textbf{(empirical)}. Through the lens of levels of processing, the researchers discuss the importance of remedying the deficit that the orthographic to phonological encoding places upon the phonological loop. 

	\item The third is from \textcite{friedman_reading_2017} \textbf{(empirical)}. They have also found that the orthographic to phonological encoding is critical for reading. They found that children with ADHD have a deficit in this area.

	\item The fourth is \textcite{kofler_working_2020} \textbf{(empirical)}. Due to highly varied results in the literature, the authors employed both visuospatial and phonological working memory tasks (bifactor model) to assess the working memory of children with ADHD. They found that a significant number of children with ADHD \((d = 1.62 \text{--} 2.03)\), whether exhibiting inattentive or hyperactive-impulsive symptoms, displayed high percentages of cognitive impairment (75\%--81\%). 
\end{enumerate}

An important point to see between both \textcite{raiker_phonological_2019} and \textcite{kofler_working_2020} is that while \textcite{kofler_working_2020} did not think ADHD has an effect upon the phonological loop, \textcite{raiker_phonological_2019} did. That is, \textcite{raiker_phonological_2019} found that (visuospatial-based) orthographic to phonological recording was to blame for a deficit in phonological performance.

% Basic processes as foundations of cognitive impairment in adult ADHD.
\textcite{butzbach_basic_2019} \textbf{(empirical)} bring up a point of contention that conflicts with \textcite{kofler_working_2020}: They cite that about 11\% of patients show no cognitive impairment \textemdash\ starkly different from the 81\% that \textcite{kofler_working_2020} found. Anyway, they show that basic functions like distractibility and processing speed affect higher order cognitive functions. (Perhaps tie this in with \cite{raiker_phonological_2019} about LoP?)
% Clinical correlates of working memory deficits in you with and without ADHD: A controlled study

\textcite{fried_clinical_2016} \textbf{(empirical)} questions the link between ADHD and WM.\@ Specifically, they question whether WM deficits are common with children that have ADHD, or if ADHD causes the WM deficits. They conclude that WM deficits have a much greater impact upon cognition for ADHD-afflicted students. Thus, WM-deficits could be a precursor to ADHD, and looking for these deficits could be a way to diagnose ADHD.\@

% Differences in executive functioning in children with ADHD
\textcite{elosua_differences_2017} \textbf{(empirical)} is similar to \textcite{fried_clinical_2016}. They are both interested in ADHD and WM, but \textcite{elosua_differences_2017} ``measures divided attention, updating, attentional shifting and inhibition, measuring through four tasks, the dual-task paradigm (digits and box-crossing), the N-back task, the Trail Making Test and the Stroop task, respectively.'' They found corroborating results with \textcite{fried_clinical_2016} that children with ADHD have deficits in WM.\@

% Executive functioning and problem solving: a bidirectional relation (non-empirical)
\textcite{drigas_executive_2019} \textbf{(non-empirical)} speak on how a bidirectional relationship exists between executive functioning and problem-solving. They argue that an enhancement in self-regulation would help in problem-solving. These results extend to both adults and children. These results could be extended with research conducted by \textcite{skalski_impact_2021} regarding motivation.

\textcite{mohamed_basic_2021} \textbf{(empirical)} found that foundational cognitive processes are critical for higher level processing. They mention that the lack of measuring motivation may have been a confounding variable in their study. This gives credence to \textcite{skalski_impact_2021}. In general, this study is more of a critique of how future research should be conducted, but there are some interesting points that could be used in the paper.


% Articles I still need to talk about:
% Impact of motivation on selected aspects of attention in children with ADHD
\subsection{Motivation}
The authors \textcite{skalski_impact_2021} \textbf{(empirical)} state that motivation may have a role in cognitive capacity. They---like many other reports---show that there is not an intellectual deficit in children with ADHD, but rather, they note a motivational deficit. This may imply that children with ADHD will attain stronger cognition if they are motivated.




\printbibliography

\end{document}