\documentclass[stu]{apa7}
\usepackage{amsmath}
\usepackage{amssymb}
\usepackage{enumitem}
\usepackage{geometry}
\usepackage[style=apa,backend=biber]{biblatex}
\usepackage{hyperref}
\usepackage{xcolor}
\definecolor{horange}{HTML}{f58026}
\hypersetup{
	colorlinks=true,
	linkcolor=horange,
	filecolor=horange,      
	urlcolor=horange,
    citecolor=horange
}
\usepackage{draculatheme}
\addbibresource{references.bib}

\title{Annotated Bibliography: The Effect of ADHD on Cognition}
\shorttitle{The Effect of ADHD on Cognition}
\author{Paul Beggs}
\authorsaffiliations{Hendrix College}
\course{PSYC 319: Cognitive Psychology}
\professor{Dr.\ Carmen Merrick}
\duedate{November 14, 2024}

\abstract{
	TODO: Write abstract.
}

\begin{document}

\maketitle


For the introduction to the paper, I am planning on using \textcite{faraone_attention-deficithyperactivity_2015} for an overview of ADHD. Then, I plan on giving an overview of how ADHD affects the different parts of cognition that we have talked about in class. 

\section{Working Memory \& Short-Term Memory}

In this section, I have a few papers that I am going to use to expand upon working memory. 

\begin{enumerate}
	\item The first is \textcite{fabio_working_2020}. This paper is an empirical study that investigates three hypotheses:
	\begin{enumerate}
		\item children with ADHD show higher levels of DD (Delayed discounting---paradigms used to study the process leading to a choice) than control subjects,
		\item once the memory load increases, deferring a reward becomes harder for both children with ADHD and with TD (typical development), and
		\item the performances of children with ADHD are significantly worsened by the addition of a memory load compared to the control group.
	\end{enumerate}
	The researchers operationalize studying these independent variables by involving goal-directed actions during distractions and dual task paradigms. In other words, the participants were to listen and repeat an assigned sequence of numbers, while also evaluating which amount of money to take---e.g., some now, or more later. The distraction was the digit recall. Though, the results of this study should be taken with a grain of salt: It has a small sample size \((N = 32)\), and the study was conducted in Italy, where they used different tests like the ADHD Rating Scale for Teachers (SDAI). 

	\item The second is \textcite{raiker_phonological_2019}. Through the lens of levels of processing, the researchers discuss the importance of remedying the deficit that the orthographic to phonological encoding places upon the phonological loop. 

	\item The third is from \textcite{friedman_reading_2017}. They have also found that the orthographic to phonological encoding is critical for reading. They found that children with ADHD have a deficit in this area.

	\item The fourth is \textcite{kofler_working_2020}. Due to highly varied results in the literature, the authors employed both visuospatial and phonological working memory tasks (bifactor model) to assess the working memory of children with ADHD. They found that a significant number of children with ADHD \((d = 1.62 \text{--} 2.03)\), whether exhibiting inattentive or hyperactive-impulsive symptoms, displayed high percentages of cognitive impairment (75\%--81\%). 

	An important point to see between both \textcite{raiker_phonological_2019} and \textcite{kofler_working_2020} is that while \textcite{kofler_working_2020} did not think ADHD has an effect upon the phonological loop, \textcite{raiker_phonological_2019} did. That is, \textcite{raiker_phonological_2019} found that (visuospatial-based) orthographic to phonological recording was to blame for a deficit in phonological performance.\footnote{I feel like throughout this section, I am focusing on things that are not relevant to my research question. Given that most of the available research is dedicated to clinical applications, trying to extrapolate the results to a general population is difficult.}
\end{enumerate}

\section{}




\printbibliography

\end{document}