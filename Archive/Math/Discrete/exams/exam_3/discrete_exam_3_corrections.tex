\documentclass[10pt]{article}
\usepackage{amsmath, amssymb, amsfonts}
\usepackage{amsthm}
\usepackage{geometry} % For 1 inch margins
\usepackage{graphicx} % For including images
\usepackage{times} % For Times New Roman font
\usepackage{lipsum} % For generating dummy text
\usepackage[utf8]{inputenc} % For UTF-8 encoding
\usepackage{titlesec} % For customizing titles
\usepackage{titling} % For customizing the title
\usepackage{fancyhdr} % For custom headers and footers
\usepackage{tikz}
\usepackage{slashed}
\usepackage{everypage}
\usepackage{titlesec}
\usepackage{ulem} % Needed for underlining
\usepackage{multicol}
\usepackage{listings}

\usepackage{tikz}

\usetikzlibrary{graphs}
\usetikzlibrary{shapes}
\usetikzlibrary{backgrounds}
\usetikzlibrary{calc}
\usetikzlibrary{patterns}
\usetikzlibrary{positioning, fit, arrows.meta}

\setlength{\leftskip}{2em}
\geometry{
    textheight=9in,
    textwidth=5.5in,
    top=1in,
    headheight=12pt,
    headsep=12.5pt,
    footskip=30pt
}

\newenvironment{solution}{\textit{Solution}.}

\newcommand{\proofseparator}{\noindent\makebox[\linewidth]{\rule{\textwidth}{0.4pt}}}

\newcommand{\solex}[1]{
    \vspace{5pt}
    \begin{solution}
    #1
    \end{solution}
}

% Natural Numbers 
\newcommand{\N}{\ensuremath{\mathbb{N}}}

% Whole Numbers
\newcommand{\W}{\ensuremath{\mathbb{W}}}

% Integers
\newcommand{\Z}{\ensuremath{\mathbb{Z}}}

% Rational Numbers
\newcommand{\Q}{\ensuremath{\mathbb{Q}}}

% Real Numbers
\newcommand{\R}{\ensuremath{\mathbb{R}}}

% Complex Numbers
\newcommand{\C}{\ensuremath{\mathbb{C}}}

\lstset{
    basicstyle=\small\ttfamily,
    keywordstyle=\color{blue},
    language=Python,
    breaklines=true,
    mathescape=true,
}


\pagestyle{fancy}
\fancyhf{} % Clear default header and footer settings
\fancyhead[L]{CORRECTIONS} % Custom left header
\fancyhead[R]{Paul Beggs} % Custom right header
\rfoot{Page \thepage} % Right footer with the page number

\title{
  Exam 3 Corrections \\[1ex]
  \large \today
}
\author{\textbf{By} \\ 
Paul Beggs}
\date{\today} % FILL IN DATE HERE 

\begin{document}
\maketitle

\section*{In-Class Exam}
\begin{enumerate}
    \setcounter{enumi}{1}
    \item Consider the recurrence relation,
          $B(n) = \begin{cases}
                  2,                & \text{if } n = 0 \\
                  3,                & \text{if } n = 1 \\
                  2B(n-1) - B(n-2), & \text{if } n > 1
              \end{cases}$
          \begin{enumerate}
              \setcounter{enumii}{1}
              \item What is the closed form solution to this recurrence relation? \\

                    \solex{$B(n) = 2 + n$ \\}

              \item Use induction to prove your answer in part (b). \\

                    \solex{
                        \begin{itemize}
                            \item \textbf{Base Cases}: \\
                                  For $n = 0$: $B(0) = 2 + 0 = 2$ \\
                                  For $n = 1$: $B(1) = 2 + 1 = 3$ \\

                            \item \textbf{Inductive Hypothesis}: \\
                                  Assume $B(n) = 2 + n$ is true for an $n \geq 1$. \\

                            \item \textbf{Inductive Step}: \\
                                  We must show that $B(n + 1) = 2 + (n + 1)$.\\
                                  Given the recurrence relationship that is defined, $B(n + 1) = 2B(n) - B(n - 1)$. We can substitute using the inductive hypothesis and solve: \begin{align*}
                                      B(n + 1) & = 2(2 + n) - (2 + n - 1) \\
                                               & = 4 + 2n - 2 - n + 1     \\
                                               & = 3 + n                  \\
                                               & = 2 + (n + 1)            \\
                                  \end{align*}
                        \end{itemize}
                    }
          \end{enumerate}
    \item Show that $n^n \geq n!$ for all integers $n \geq 1$ using induction. \\

          \solex{
              \begin{itemize}
                  \item \textbf{Base Case}: \\
                        For $n = 1$: $1^1 \geq 1! \Rightarrow 1 \geq 1$. \\

                  \item \textbf{Inductive Hypothesis}: \\
                        Assume the inequality $n^n \geq n!$ is true for some integer $n \geq 1$. \\

                  \item \textbf{Inductive Step}: \\
                        We need to show that $(n + 1)^{n+1} \geq (n + 1)!$. \\

                        To begin, we will thoroughly show that $(n + 1)^n \geq n^n$.\footnote{This explanation is very pedantic: Of course we know that $(n + 1)^n \geq n^n$, but for the sake of covering all bases, I must show for each term in both expressions that they are, indeed, greater than or equal.} Thus, for the left-hand side, we know that $(n + 1)^n = (n + 1) \times (n + 1) \times \cdots \times (n + 1) \text{ (Consisting of $n$ terms)}$. Then, for the right-hand side, we can show $n^n = n \times n \times \cdots \times n$ (Consisting of $n$ terms). We know that mathematically, it must be the case that $n + 1$ is greater than $n$, and we see this is true for each corresponding entry in both expanded expressions. \\

                        So, because we know it is the case that $n + 1$ is greater than or equal $n$ for $n$ terms, we can justify the statement $(n + 1)^n \geq n^n$. Then, we can apply that to our inductive hypothesis to get $(n + 1)^n \geq n^n \geq n!$, and by substitution, $(n + 1)^n \geq n!$. \\

                        In wrapping up, we can multiply both sides of the expression above by $n + 1$ to get
                        \begin{align*}
                            (n + 1) \times (n + 1)^n & \geq (n + 1) \times n! \\
                            (n + 1)^{n+1}            & \geq (n + 1)!
                        \end{align*}
              \end{itemize}
          }
    \item Prof. Seme has 9 books that he's bought recently. He is planning on taking a trip and wants to select 4 to read. How many different combinations are there? \\

          \solex{
              $\displaystyle \binom{n}{k} = \frac{n!}{k!(n-k)!} \Rightarrow \binom{9}{4} = \frac{9!}{4!5!} \Rightarrow 126$. \\
          }

    \item What is the coefficient of $x^8$ in the expansion of $(5x - 7)^{11}$? \\

          \solex{$\displaystyle \binom{11}{3}(5x)^8(-7)^3 = -22235625x^8$ \\}

          \begin{center}
              \textbf{** SKIP 6 \& 7 **}
          \end{center}


          \setcounter{enumi}{7}

    \item give a big-$\Theta$ estimate for the number of additions, as a function of $n$ in the code shown: \\

          \begin{lstlisting}
    
    y = 7
    for i in [1,2,...,9]:
        for j in [1,2,...,n]:
            for k in [i,i+1,...,n]:
                y = y + i + j + k
    \end{lstlisting}

          \solex{Outer loop: constant time. \\
              Middle loop: $n$ time. \\
              Inner loop: $n$ time. \\
              $\therefore \Theta(n^2)$}

\end{enumerate}

\newpage

\section*{Take-Home Exam}

\begin{enumerate}
    \item Suppose you have an unlimited number of 3 and 7 cent stamps. Show, using induction, that you can make any value of 12 cents or more.\footnote{See: Fundamental Theorem of Arithmetic. (And here I thought you were trying to teach us about what to do if we ever found ourselves in a situation with an infinite amount of 3-cent and 7-cent stamps. Definitely fooled me!)} \\

          \solex{
              \begin{itemize}
                  \item \textbf{Base Cases}: \\
                        $n = 12$: $3 \times 4 = 12$ \\
                        $n = 13$: $(3\times 2) + 7 = 13$ \\
                        $n = 14$: $7 \times 2 = 14$ \\
                        $n = 15$: $3 \times 5 = 15$ \\
                        $n = 16$: $(3 \times 3) + 7 = 16$ \\
                        $n = 17$: $(7 \times 2) + 3 = 17$ \\

                  \item \textbf{Inductive Hypothesis}: \\
                        Suppose it is possible to create any postage value of $k$ cents, for all values of $k$ such that $12 \leq k \leq n$, where $n \geq 17$. \\

                  \item \textbf{Inductive Step}: \\
                        \textit{This is more for me as a way of laying out what the proof form is supposed to be like}: \\

                        The goal is to show that for any stamp value $n$ that is 17 cents or more, we can subtract some amount of either 3 or 7 cents to fall back into the range set by the inductive hypothesis. This way, we can show that every larger amount can be `built up' from smaller amounts that we have shown to be possible.\\

                        Given that $n \geq 17$, we need to show that $n + 1$ can be created from smaller combinations laid out by the inductive hypothesis. Thus, from the inductive hypothesis, we can make $n - 2$ cents because $n - 2$ is at least 15. Thus, adding one $3$ cent stamp to $n - 2$ gives us $n + 1$. \\
              \end{itemize}
          }
          \begin{center}
              \textbf{** SKIP 2-4 ** \\}
          \end{center}

          \setcounter{enumi}{4}
    \item M\&M Candies come in 6 colors. How many do you need to select to be certain that you have at least 7 of the same color? \\

          \solex{There are 6 possible candies, with 6 possible colors. If we were to pick out 36 candies, it would be possible that for each candy, we got exactly 6 colors. Hence, we would need to pick one more candy to be certain that we have at least 7 of the same color. Thus, if you are \textit{extremely unlucky}, you need to pick 37 M\&M's to be sure that of the M\&M's you have, at least 7 are of the same color. \\}


\end{enumerate}
\end{document}