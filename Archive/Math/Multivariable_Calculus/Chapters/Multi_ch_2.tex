\section{Vectors in the Plane}

\subsection{Notation}

In print, we write vectors in bold like: $\mathbf{v}$, $\mathbf{w}$, $\mathbf{u}$, \dots.  In handwriting, we often write vectors with an arrow over the top: $\vec{v}$, $\vec{w}$, $\vec{u}$, \dots.

\subsection{Vectors}

A \cyanit{vector} is a quantity with both \cyanit{magnitude} (size, length, strength, \dots) and \cyanit{direction}.

Given two points in the plane \(P = (x_{1},y_{1})\) and \(Q = (x_{2},y_{2})\), the vector from \(P\) to \(Q\), denoted \(\overrightarrow{PQ} = \mathbf{PQ} = \langle x_{2}-x_{1},y_{2}-y_{1}\rangle\). \\

We can also simply state components (known as \cyanit{component form}): \(\mathbf{v} = \langle x, y\rangle\). \\

In \(\R^{2}\) the \cyanit{standard unit vectors} are \(\hat{\imath} = \mathbf{i} = \brackett{1,0}\) and \(\hat{\jmath} = \mathbf{j} = \brackett{0,1}\). This allows us to write \(\mathbf{v} = \brackett{2,3} = 2\mathbf{i} + 3\mathbf{j}\), for example. \\

In \(\R^{3}\), we have three stand unit vectors, \(\hat{\imath} = \mathbf{i} = \brackett{1,0,0}\), \(\hat{\jmath} = \mathbf{j} = \brackett{0,1,0}\), and \(\hat{k} = \mathbf{k} = \brackett{0,0,1}\).  \\

It is a picky detail, but \(\mathbf{i} \in \R^{2} \ne \mathbf{i} \in \R^{3}\). \\

The \cyanit{zero vector}, denoted \(\mathbf{0}\), is \(\mathbf{0} = \langle 0,0 \rangle\). Note that \(\mathbf{0} \ne 0\). \\

A \cyanit{scalar} is a real number (or a magnitude), without direction. \\

If \(c\) is a scalar and \(\mathbf{v} = \langle x,y\rangle\), then
\[
    c\mathbf{v} = c\langle x,y\rangle = \langle cx,cy\rangle.
\]
This operation is called \cyanit{scalar multiplication}. Scalar multiplication changes the magnitude of a vector, but not its direction. \\

Note that the individual components of a vector are themselves \cyanit{scalars}. You need to keep track of which is which. \\

If \(\mathbf{v} = \langle x_{1},y_{1} \rangle\) and \(\mathbf{w} = \brackett{x_{2},y_{2}}\), then the \cyanit{vector sum} 
\[
    \mathbf{v} + \mathbf{w} = \langle x_{1}+x_{2},y_{1}+y_{2}\rangle.
\] 
That is, we add component wise. \\

If \(\mathbf{v} = \langle x_{1},y_{1} \rangle\), then the \cyanit{magnitude} of \(\mathbf{v}\) is given by
\[
    \norm{v} = \sqrt{x_{1}^{2}+y_{1}^{2}}.
\]
This is really just the Pythagorean theorem.

\section{Vectors in Space}

In \(\R^{3}\), we have three axes, \(x\), \(y\), and \(z\), which follow the \cyanit{right-hand rule}: point the fingers of the right hand in the direction of the positive \(x\)-axis, curl them towards the positive \(y\)-axis, and the thumb points in the direction of the positive \(z\)-axis. \\

Since the distance formula in \(\R^{3}\) is \(d = \sqrt{(x_{2} - x_{1})^{2} + (y_{2} - y_{1})^{2} + (z_{2} - z_{1})^{2}}\), then \(\mathbf{u} = \brackett{x,y,z}\) we have \(\norm{u} = \sqrt{x^{2} + y^{2} + z^{2}}\). \\

To \cyanit{normalize} a vector, we divide by its magnitude: \(\mathbf{v} = \brackett{x,y,z}\), then \(\mathbf{u} = \frac{1}{\norm{v}}\mathbf{v} = \brackett{\frac{x}{\norm{v}},\frac{y}{\norm{v}},\frac{z}{\norm{v}}}\). This gives us a \cyanit{unit vector} in the direction of \(\mathbf{v}\). \\

Thus, a \cyanit{unit vector} is a vector \(\mathbf{u}\) such that \(\norm{u} = 1\). \\

\subsection{Vector Properties}

Suppose that each of \(\mathbf{u}\), \(\mathbf{v}\), and \(\mathbf{w}\) are vectors and \(r\) and \(s\) are scalars. Then the following properties hold:
\begin{itemize}
    \item \cyanit{Additive Commutativity}: \(\mathbf{v} + \mathbf{w} = \mathbf{w} + \mathbf{v}\).
    \item \cyanit{Additive Associativity}: \(\mathbf{u} + (\mathbf{v} + \mathbf{w}) = (\mathbf{u} + \mathbf{v}) + \mathbf{w}\).
    \item \cyanit{Additive Identity}: \(\mathbf{v} + \mathbf{0} = \mathbf{v}\).
    \item \cyanit{Additive Inverse}: \(-\mathbf{v} = (-1)\mathbf{v}\) and \(\mathbf{v} + (-\mathbf{v}) = \mathbf{0}\).
    \item \cyanit{Scalar Associativity}: \(r(s\mathbf{u})= (rs)\mathbf{u}\).
    \item \cyanit{Scalars Distributive over Vectors}: \(r(\mathbf{u} + \mathbf{v}) = r\mathbf{u} + r\mathbf{v}\).
    \item \cyanit{Vectors Distributive over Scalars}: \((r+s)\mathbf{u} = r\mathbf{u} + s\mathbf{u}\).
    \item \cyanit{Multiplicative Identity}: \(1\mathbf{u} = \mathbf{u}\).
    \item \cyanit{Zero Scalar}: \(0\mathbf{u} = \mathbf{0}\).
\end{itemize}

\newpage

\section{The Dot Product}

Suppose \(\mathbf{u} = \brackett{u_{1},u_{2},\dots u_{n}}\) and \(\mathbf{v} = \brackett{v_{1},v_{2},\dots v_{n}}\) are vectors in \(\R^{n}\). Then the \cyanit{dot product} of \(\mathbf{u}\) and \(\mathbf{v}\) is given by
\[
    \mathbf{u} \cdot \mathbf{v} = u_{1}v_{1} + u_{2}v_{2} + \cdots + u_{n}v_{n}.
\]
That is, we multiply the corresponding components and sum the results. \\

It should be clear that \(\mathbf{u} \cdot \mathbf{v}\) results in a scalar. The dot product is a special type of inner product. \\

Think of the dot product as a way to measure how much of one vector points in the same direction as another.

\subsection{Properties of the Dot Product}

Let \(\mathbf{u}, \mathbf{v}, \mathbf{w} \in \R^{n}\) and \(c\) be a scalar. Then the following properties hold:
\begin{itemize}
    \item \cyanit{Commutativity}: \(\mathbf{u} \cdot \mathbf{v} = \mathbf{v} \cdot \mathbf{u}\).
    \item \cyanit{Distributive Property}: \(\mathbf{u} \cdot (\mathbf{v} + \mathbf{w}) = \mathbf{u} \cdot \mathbf{v} + \mathbf{u} \cdot \mathbf{w}\).
    \item \cyanit{Scalar Associativity}: \((c\mathbf{u}) \cdot \mathbf{v} = c(\mathbf{u} \cdot \mathbf{v}) = \mathbf{u} \cdot (c\mathbf{v})\).
    \item \cyanit{Self-Product}: \(\mathbf{u} \cdot \mathbf{u} = \norm{u}^{2}\).
    \item \cyanit{Magnitude}: \(\norm{v} = \mysqrt{v} \cdot v\)
    \item \cyanit{Angle}: \(\mathbf{u} \cdot \mathbf{v} = \norm{u}\norm{v}\cos(\theta)\), where \(0 \leq \theta \leq \pi\) is the angle between \(\mathbf{u}\) and \(\mathbf{v}\). (Law of Cosines.)
    \item \cyanit{Orthogonality}: \(\mathbf{u} \cdot \mathbf{v} = 0\) if and only if \(\mathbf{u}\) and \(\mathbf{v}\) are orthogonal.
\end{itemize}

\subsection{Projections}

The \cyanit{projection} of \(\mathbf{u}\) onto \(\mathbf{v}\) is given by
\[
    \text{proj}_{\mathbf{v}}\mathbf{u} = \left(\frac{\mathbf{u} \cdot \mathbf{v}}{\norm{v}^{2}}\right)\mathbf{v}.
\]
This is a vector parallel to \(\mathbf{v}\), which has length equal to the amount of \(\mathbf{u}\) which points in the same direction as \(\mathbf{v}\). \\

Think of a projection as a measure of how much of one vector points in the same direction as another. 

\subsection{Work}

If a constant force \(\mathbf{F}\) moved an object from \(P\) to \(Q\), the \cyanit{work} done is given by
\[
    W = \mathbf{F} \cdot \overrightarrow{PQ}.
\]
Thus, if that force acts at an angle \(\theta\) to the line of motion, the work is:
\[
    W = (\norm{F})\norm{PQ}\cos(\theta).
\]
Later this semester, we will learn how to compensate for a non-constant force, and over a non-linear path.

% -------------------------------------------------------------------- %

\newpage

\section{The Cross Product}

Suppose that \(\mathbf{u},\mathbf{v} \in \R^{3}\). Then, the \cyanit{cross product} of \(\mathbf{u}\) and \(\mathbf{v}\), denoted by \(\mathbf{u} \times \mathbf{v}\), is the unique right-hand rule vector orthogonal to each of \(\mathbf{u}\) and \(v\) whose magnitude is equal to the area of the parallelogram spanned by \(\mathbf{u}\) and \(\mathbf{v}\). \\

Let \(\mathbf{u} = \brackett{u_{1},u_{2},u_{3}}\) and \(\mathbf{v} = \brackett{v_{1},v_{2},v_{3}}\). Then,
\[
    \mathbf{u} \times \mathbf{v} = \brackett{u_{2}v_{3} - u_{3}v_{2}, \ u_{3}v_{1} - u_{1}v_{3}, \ u_{1}v_{2} - u_{2}v_{1}}.
\]

\begin{center}
    \textsc{Note:} You will never multiply an \(v_{1}\)-coordinate by an \(u_{1}\)-coordinate. This is true for all \(v_{n}\) and \(u_{n}\) coordinates.
\end{center}

You can show by working the algebra that \(\mathbf{u} \cdot (\mb{u} \times \mb{v}) = 0 \) and \(\mb{v} \cdot (\mb{u} \times \mb{v}) = 0\). \\

With determinants, you can do this in one step:
\[
    \mb{u} \times \mb{v} = \begin{vmatrix}
        \mathbf{i} & \mathbf{j} & \mathbf{k} \\
        u_{1} & u_{2} & u_{3} \\
        v_{1} & v_{2} & v_{3}
    \end{vmatrix}
\]
Oddly, we can only define a cross-product in \(\R\), \(\R^{3}\), and \(\R^{7}\), while the dot product is \textit{always} defined.

\subsubsection{Example}

\begin{align*}
    \mathbf{u} \times \mathbf{v} &= \brackett{2,1,4} \cdot \brackett{1,-3,1} \\
    &= \brackett{(1)(1) - 4(-3), 4(1) - 2(1), 2(-3) - 1(1)} \\
    &= \brackett{13,2,-7}.
\end{align*}

\subsection{Properties of the Cross Product}

Let \(\mathbf{u}, \mathbf{v}, \mathbf{w} \in \R^{3}\) and \(c\) be a scalar. Then the following properties hold:
\begin{itemize}
    \item \cyanit{Anticommutativity}: \(\mathbf{u} \times \mathbf{v} = -\mathbf{v} \times \mathbf{u}\).
    \item \cyanit{Distributive Property}: \(\mathbf{u} \times (\mathbf{v} + \mathbf{w}) = \mathbf{u} \times \mathbf{v} + \mathbf{u} \times \mathbf{w}\).
    \item \cyanit{Scalar Associativity}: \((c\mathbf{u}) \times \mathbf{v} = c(\mathbf{u} \times \mathbf{v}) = \mathbf{u} \times (c\mathbf{v})\).
    \item \cyanit{Zero}: \(\mathbf{u} \times \mathbf{u} = \mathbf{0}\).
    \item \cyanit{Nilpotence}: \(\mathbf{u} \times \mathbf{v} = \mathbf{0}\) if and only if \(\mathbf{u}\) and \(\mathbf{v}\) are parallel.
    \item \cyanit{Scalar Triple Product}: \(\mb{u} \cdot (\mb{v} \times \mb{w}) = (\mb{u} \times \mb{v}) \cdot \mb{w}\).
    \item \cyanit{Angle}: \(\norm{u \times v} = \norm{u}\norm{v}\sin(\theta)\), where \(0 \leq \theta \leq \pi\) is the angle between \(\mathbf{u}\) and \(\mathbf{v}\).
\end{itemize}

\subsection{Standard Unit Vectors and the Cross Product}

\(\mb{i} \times \mb{i} = \mb{j} \times \mb{j} = \mb{k} \times \mb{k} = \mb{0}\).
\begin{multicols}{2}
    \begin{itemize}
        \item \(\mb{i} \times \mb{j} = \mb{k}\) 
        \item \(\mb{j} \times \mb{i} = -\mb{k}\)
        \item \(\mb{j} \times \mb{k} = \mb{i}\) 
        \item \(\mb{k} \times \mb{j} = -\mb{i}\)
        \item \(\mb{k} \times \mb{i} = \mb{j}\) 
        \item \(\mb{i} \times \mb{k} = -\mb{j}\)
    \end{itemize}
\end{multicols}

\subsection{Torque}

\cyanit{Torque}, denoted by \(\mb{\tau}\), measures the tendency to produce a rotation about an axis. \\

If \(\mb{r}\) is a radial vector from an axis to a force and \(\mb{F}\) is the force, then the torque induced on the axis by the force is given by:
\[
    \mb{\tau} = \mb{r} \times \mb{F} \qquad \text{or} \qquad \norm{\mb{\tau}} = \norm{r}\norm{F}\sin(\theta).
\]

% -------------------------------------------------------------------- %
% Section 4
% -------------------------------------------------------------------- %

\newpage

\section{Equations of Lines and Planes}

For the vector equation, parametric equation, and the symmetric equation, use these points for the examples: \((3,5,1) + (9,1,2)\).

\subsection{Lines}

\subsubsection{Lines in Two Dimensions}

A line in \(\R^{2}\) which contains the point \((x_{0},y_{0})\) and is parallel to the vector \(\mb{v} = \langle a,b \rangle\) has parametric form

\[
    f(t) = \begin{cases}
        x(t) = x_{0} + ta \\
        y(t) = y_{0} + tb
    \end{cases}.
\]

\subsubsection{Lines in Three Dimensions}

In \(\R^{3}\), we have more options for the form of a line. Suppose that our line contains the point \(\mb{r}_{0} = \langle x_{0},y_{0},z_{0} \rangle\) and is parallel to the vector \(\mb{v} = \langle a,b,c \rangle\). Then the following properties hold:

\subsubsection{Vector Equation}

The \cyanit{vector equation} of a line is given by \(\mathbf{r}(t) = \mb{r}_{0} + t \mb{v}\). \\

\textbf{Example}: Find all 3 equations of lines  \\

From our example, \(\mb{v} = \brackett{6,-4,1}\) and \(\mb{r}_{0} = \brackett{3,5,1}\). \\

Vector equation: \(\mb{r}(t) = \brackett{3,5,1} + t\brackett{6,-4,1}\).

\subsubsection{Parametric Equation}

The \cyanit{parametric equation} of a line is given by

\[
    f(t) = \begin{cases}
        x(t) = x_{0} + ta \\
        y(t) = y_{0} + tb \\
        z(t) = z_{0} + tc
    \end{cases}.
\]

From our example, we would get \(x(t) = 3 + 6t\), \(y(t) = 5 - 4t\), and \(z(t) = 1 + t\).

\subsubsection{Symmetric Equation}

For the following formula, we get \(a\), \(b\) and \(c\) from subtracting the \(x\), \(y\), and \(z\) components of the direction vector from the point vector. \\

As long as each of \(a\), \(b\), \(c \neq 0\), the symmetric equation is
\[
    \frac{x - x_{0}}{a} = \frac{y - y_{0}}{b} = \frac{z - z_{0}}{c}.
\]
(Notice that in two dimensions, this is just the equation of the line: \(\bigl(\frac{b}{a}\bigr)(x - x_{0}) + y_{0} = y\), when solved for \(y\).)

From our example, we would get
\[
\frac{x - 3}{9 - 3} = \frac{y - 5}{1 - 5} = \frac{z - 1}{1 - 2} \implies \frac{x - 3}{6} = \frac{y - 5}{-4} = -z + 1.
\]

\subsubsection{Line Segment}

Suppose that \(P = (x_{0},y_{0},z_{0})\) and \(Q = (x_{1},y_{1},z_{1})\). The line segment from \(P\) to \(Q\) is given by
\[
    \mathbf{r}(t) = (1 - t)\mathbf{p} + t\mathbf{q},
\]

where \(\mathbf{p} = \langle x_{0},y_{0},z_{0} \rangle\), \(\mathbf{q} = \langle x_{1},y_{1},z_{1} \rangle\), and \(0 \leq t \leq 1\). \\

The parametric equations for this segment are 
\[
    f(t) = \begin{cases}
        x(t) = x_{0} + t(x_{1} - x_{0}) \\
        y(t) = y_{0} + t(y_{1} - y_{0}) \\
        z(t) = z_{0} + t(z_{1} - z_{0})
    \end{cases}.
\]

\subsubsection{Distance Between Point and Line}

The distance from a point \(M\) to a line which contains the point \(P\) and has direction vector \(\mb{v}\) is given by
\[
    d = \frac{\|\mb{PM} \times \mb{v}\|}{\norm{v}}.
\]

Notice that you are free to choose any point on the line you'd like!

\subsubsection{Relationships Between Lines}

\begin{itemize}
    \item \cyanit{Equal}: Same direction vector, share a point.
    \item \cyanit{Parallel}: Same direction vector, do not share a point.
    \item \cyanit{Intersecting}: Different direction vectors, share a point.
    \item \cyanit{Skew}: Different direction vectors, do not share a point.
\end{itemize}

\subsection{Planes}
A plane can be defined by:
\begin{itemize}
    \item any three non-colinear points,
    \item any two intersection points,
    \item a line and a point not on the line, or
    \item given two orthogonal vectors with a common starting point: ``spin''  one vector in place; notice the other sweeps out a circle, which can be extended to a plane. * In notes *
\end{itemize}
Of particular importance for a plane is a \cyanit{normal vector}. A vector \(\mathbf{n}\) is a normal vector provided it is orthogonal to \(\overrightarrow{PQ}\) for any two points \(P\) and \(Q\) which are in the plane.

\subsection{Equations of a Plane}

Like lines, we have three equations of a plane. Let \(P\) and \(Q\) be points in the plane and \(n = \langle a,b,c \rangle\).

\subsubsection{Vector Equation}

The \cyanit{vector equation} of a plane is \(n \cdot \overrightarrow{PQ} = 0\). Note that this is an implicit definition (i.e. it is not useful for directly writing down an equation, but is the fundamental idea of why this all works)!

\subsubsection{Scalar Equation}

If \((x_{0},y_{0},z_{0})\) is any point in the plane, the \cyanit{scalar equation} of the plane is given by
\begin{align*}
    \brackett{x - x_{0}, y - y_{0}, z - z_{0}} \cdot \brackett{a,b,c} &= 0 \\
    a(x - x_{0}) + b(y - y_{0}) + c(z - z_{0}) &= 0
\end{align*}

\subsubsection{General Form}

The \cyanit{general form} of the equation of a plane is given by \(ax + by + cz + d = 0\), where \(d = -ax_{0} - by_{0} - cz_{0}\).

\subsubsection{Distance Between Point and Plane}

\begin{itemize}
    \item Equal: Share a common point, have parallel normal vectors
    \item Parallel: Do not share a common point, do have parallel normal vectors
    \item Intersecting: If their normal vectors are not parallel, the two planes intersect in a line.
    \begin{itemize}
        \item You can use algebra to find a point in common -- i.e. solve both equations for the planes
        \item Find the line's direction vector by taking the cross product of the planes' normal vectors.
    \end{itemize}
\end{itemize}

\subsection{Examples}

Use the points \(P(3,5,1)\), \(Q(9,1,2)\), and \(R(0,2,5)\).

\subsubsection{Example 1}

Find the scalar equation of the plane containing \(P\), \(Q\), and \(R\). \\

We know \(\mb{PQ} = \brackett{6,-4,1}\) and \(\mb{PR} = \brackett{-3,-3,4}\). Then, \(\mb{n} = \mb{PQ} \times \mb{PR} = \brackett{13,-27,-30}\). \\

Thus, the equation of the plane is \(13(x - 3) - 27(y - 5) - 30(z - 1) = 0\). \\

To check, plug in the points: \(13(3) - 27(5) - 30(1) = 0\), \(13(9) - 27(1) - 30(2) = 0\), and \(13(0) - 27(2) - 30(5) = 0\). \\

\textbf{Distance from point to line:} \\

Suppose \(M\) is a point and \(P\) is any point on some line \(l\). Refer to notes for graph. Let \(d = \norm{PM}\sin \theta\). Let \(\mb{v}\) be a directional vector of \(l\). \\ 

\(\norm{PM \times v} = \norm{PM}\norm{v}\sin \theta\). This gives us \(d = \frac{\norm{PM \times v}}{\norm{v}}\). \\

\textbf{Distance from point to plane:} \\

Let \(P\) be the point with \(\mb{n}\) as the norm vector. We start with \(d = \norm{\proj_{n}PQ}\). Then, \(d = \frac{\norm{PQ \times n}}{\norm{n}\cdot \norm{n}}\norm{n}\). Look in the book for the rest of this.


% -------------------------------------------------------------------- %
% Section 6
% -------------------------------------------------------------------- %


\newpage

\section{Quadratic Surfaces}

\subsection{Spheres}

A \cyanit{sphere}, centered at \((x_{0},y_{0},z_{0})\) with radius \(r\), is given by the equation
\[
    (x - x_{0})^{2} + (y - y_{0})^{2} + (z - z_{0})^{2} = r^{2}.
\]

\subsection{Cylinder}

A \cyanit{cylinder} is a surface in \(\R^{3}\) which consists of all lines that are parallel to a given line and pass through a given plane curve. The lines that make up a cylinder are called \cyanit{rulings}. The \cyanit{trace} of a cylinder is the cross section generated by intersecting the cylinder with a coordinate plane.

\subsection{Quadratic Surfaces}

A \cyanit{quadratic surface} is a surface in \(\R^{3}\) whose equation can be written as
\[
    Ax^{2} + By^{2} + Cz^{2} + Dxy + Exz + Fyz + Gx + Hy + Iz + J = 0.
\]
By a change of axes (rotations) and origin (translations), we can rewrite these always as one of 
\[
    Ax^{2} + By^{2} + Cz^{2} = 1, \qquad \text{or} \qquad Ax^{2} + By^{2} + Iz = 0
\]
While cylinders have rulings made of parallel lines, quadratic surfaces do not (at least, not in general).
However, their traces are always conic sections: lines, parabolas, circles, ellipses, or hyperbolas.
