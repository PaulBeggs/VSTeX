\documentclass[11pt]{article}

\usepackage{fix-cm}
\usepackage{tikz}
\usetikzlibrary{shapes,backgrounds,calc,patterns}
\usepackage{amsmath,amsthm} 
\usepackage{amssymb}
\usepackage[framemethod=tikz]{mdframed}
% \usepackage{draculatheme}
\usepackage{mathrsfs}
\usepackage{changepage}
\usepackage{multicol}
\usepackage{mathtools}
\usepackage{hyperref}
\usepackage{slashed}
\usepackage{enumerate}
\usepackage{booktabs}
\usepackage{enumitem}
\usepackage{kantlipsum} 
\usepackage{pgfplots}
\pgfplotsset{compat=1.18}
\usetikzlibrary{decorations.markings}

\setlength{\parindent}{0pt}

\newmdenv[
	topline=false,
	bottomline=true,
	rightline=false,
	leftline=true,
	linewidth=1.5pt,
	linecolor=black, % default color, will be overridden in custom commands
	% backgroundcolor=draculabg, % Needed for Dracula theme
	% fontcolor=draculafg, % Needed for Dracula theme
	innertopmargin=0pt,
	innerbottommargin=5pt,
	innerrightmargin=10pt,
	innerleftmargin=10pt,
	leftmargin=0pt,
	rightmargin=0pt,
	skipabove=\topsep,
	skipbelow=\topsep,
]{customframedproof}

\newenvironment{proofpart}[2][black]{
    \begin{mdframed}[
        topline=false,
        bottomline=false,
        rightline=false,
        leftline=true,
        linewidth=1pt,
        linecolor=#1!40, % Custom color
        % innertopmargin=10pt,
        % innerbottommargin=10pt,
        innerleftmargin=10pt,
        innerrightmargin=10pt,
        leftmargin=0pt,
        rightmargin=0pt,
        % skipabove=\topsep,
        % skipbelow=\topsep%
    ]
    \noindent
    \begin{minipage}[t]{0.08\textwidth}%
        \textbf{#2}%
    \end{minipage}%
    \begin{minipage}[t]{0.90\textwidth}%
        \begin{adjustwidth}{0pt}{0pt}%
}{
    \end{adjustwidth}
    \end{minipage}
    \end{mdframed}
}

\newenvironment{solution}
  {\textit{Solution.}}



%%% AESTHETICS %%%
%-%-%-%-%-%-%-%-%-%-%-%-%-%-%-%-%-%-%-%-%-%-%-%-%-%-%-%-%-%-%-%-%-%-%-%-%-%-%


%%% Dimensions and Spacing %%%
\usepackage[left=0.5in,right=0.5in,top=1in,bottom=1in]{geometry}
% \usepackage{setspace}
% \linespread{1}
\usepackage{listings}
\usepackage{minted}

%%% Define new colors %%%
\usepackage{xcolor}
\definecolor{orangehdx}{rgb}{0.96, 0.51, 0.16}

% Normal colors
\definecolor{xred}{HTML}{BD4242}
\definecolor{xblue}{HTML}{4268BD}
\definecolor{xgreen}{HTML}{52B256}
\definecolor{xpurple}{HTML}{7F52B2}
\definecolor{xorange}{HTML}{FD9337}
\definecolor{xdotted}{HTML}{999999}
\definecolor{xgray}{HTML}{777777}
\definecolor{xcyan}{HTML}{80F5DC}
\definecolor{xpink}{HTML}{F690EA}
\definecolor{xgrayblue}{HTML}{49B095}
\definecolor{xgraycyan}{HTML}{5AA1B9}

% Dark colors
\colorlet{xdarkred}{red!85!black}
\colorlet{xdarkblue}{xblue!85!black}
\colorlet{xdarkgreen}{xgreen!85!black}
\colorlet{xdarkpurple}{xpurple!85!black}
\colorlet{xdarkorange}{xorange!85!black}
\definecolor{xdarkcyan}{HTML}{008B8B}
\colorlet{xdarkgray}{xgray!85!black}

% Very dark colors
\colorlet{xverydarkblue}{xblue!50!black}

% Document-specific colors
\colorlet{normaltextcolor}{black}
\colorlet{figtextcolor}{xblue}

% Enumerated colors
\colorlet{xcol0}{black}
\colorlet{xcol1}{xred}
\colorlet{xcol2}{xblue}
\colorlet{xcol3}{xgreen}
\colorlet{xcol4}{xpurple}
\colorlet{xcol5}{xorange}
\colorlet{xcol6}{xcyan}
\colorlet{xcol7}{xpink!75!black}

% Blue-Purple (should just used colorbrewer...)
\definecolor{xrainbow0}{HTML}{e41a1c}
\definecolor{xrainbow1}{HTML}{a24057}
\definecolor{xrainbow2}{HTML}{606692}
\definecolor{xrainbow3}{HTML}{3a85a8}
\definecolor{xrainbow4}{HTML}{42977e}
\definecolor{xrainbow5}{HTML}{4aaa54}
\definecolor{xrainbow6}{HTML}{629363}
\definecolor{xrainbow7}{HTML}{7e6e85}
\definecolor{xrainbow8}{HTML}{9c509b}
\definecolor{xrainbow9}{HTML}{c4625d}
\definecolor{xrainbow10}{HTML}{eb751f}
\definecolor{xrainbow11}{HTML}{ff9709}

%%% FIGURES %%%
\usepackage{graphicx}  
\graphicspath{ {images/} }  
% \numberwithin{figure}{section}
\usepackage{float}
\usepackage{caption}

%%% Hyperlinks %%%
\usepackage{hyperref}
\definecolor{horange}{HTML}{f58026}
\hypersetup{
	colorlinks=true,
	linkcolor=horange,
	filecolor=horange,      
	urlcolor=horange,
}

\newcommand{\mysqrt}[1]{%
  \mathpalette\foo{#1}%
}
\newcommand{\dmysqrt}[1]{%
  \mathpalette\foodisplay{#1}%
}

\newcommand{\sol}[1]{
    \begin{customframedproof}[linecolor=orangehdx!75,]
        \begin{solution}
        #1
        \end{solution}
    \end{customframedproof}
}
\allowdisplaybreaks

% !TeX spellcheck = off
\newcommand{\foo}[2]{%
  % #1: math style, #2: content
  \sbox0{$#1\sqrt{#2}$}% Measure the size of the standard sqrt in the current style
  \begin{tikzpicture}[baseline=(sqrt.base)]
    \node[inner sep=0, outer sep=0] (sqrt) {$#1\sqrt{#2}$}; % Use the current math style
    \draw([yshift=-0.045em]sqrt.north east) -- ++(0,-0.5ex); % Draw the tick
  \end{tikzpicture}%
}
% !TeX spellcheck = off
\newcommand{\foodisplay}[2]{%
  % #1: math style, #2: content
  \sbox0{$#1\sqrt{#2}$}% Measure the size of the standard sqrt in the current style
  \begin{tikzpicture}[baseline=(sqrt.base)]
    \node[inner sep=0, outer sep=0] (sqrt) {$\displaystyle\sqrt{#2}$}; % Force displaystyle
    \draw[line width=0.4pt] ([yshift=-0.044em]sqrt.north east) -- ++(0,-0.5ex); % Draw the tick
  \end{tikzpicture}%
}

\newcommand{\barNotationT}[1]{\bigg|_{t = #1}}

\newcommand{\cyanit}[1]{\textit{\textcolor{cyan}{#1}}}

\newcommand{\brackett}[1]{\left\langle #1 \right\rangle}

\newcommand{\norm}[1]{\left\lVert \mathbf{#1}\right\rVert}

\newcommand{\imb}{\mb{i}}
\newcommand{\jmb}{\mb{j}}
\newcommand{\kmb}{\mb{k}}
\newcommand{\rmb}{\mb{r}}
\newcommand{\umb}{\mb{u}}

\newcommand{\vecfuc}[2]{\mb{#1}(#2)}
\newcommand{\dvecfuc}[2]{\mb{#1}'(#2)}
\newcommand{\normdvecfuc}[2]{||\mb{#1}'(#2)||}

\newcommand{\proj}{\text{proj}}

\newcommand{\mb}[1]{\mathbf{#1}}

% \renewcommand{\theenumi}{\arabic{enumi}} 
% \renewcommand{\labelenumi}{\theenumi.}

\title{Multivariable Calculus Take Home Exam I}
\author{Paul Beggs}
\date{\today}

%%% Custom Comands %%%
% Natural Numbers 
\newcommand{\N}{\ensuremath{\mathbb{N}}}

% Whole Numbers
\newcommand{\W}{\ensuremath{\mathbb{W}}}

% Integers
\newcommand{\Z}{\ensuremath{\mathbb{Z}}}

% Rational Numbers
\newcommand{\Q}{\ensuremath{\mathbb{Q}}}

% Real Numbers
\newcommand{\R}{\ensuremath{\mathbb{R}}}

% Complex Numbers
\newcommand{\C}{\ensuremath{\mathbb{C}}}

\newcommand{\I}{\ensuremath{\mathbb{I}}}


\begin{document}

\begin{enumerate}
  \item (6 points) Consider the parametric curve defined by \(x(t) = t^{2} + 3t\), \(y(t) = \sin(\pi t)\), for \(0 \leq t \leq 1\). Determine the area between this curve and the \(x\)-axis. Give your answer as exact, and it might be useful to use Integration by Parts for some of this!

        \sol{
        To find the area between the curve and the \(x\)-axis, we can use:
        \[
          A = \int_{0}^{1} y(t)\,x'(t)\,dt
          = \int_{0}^{1} \sin(\pi t)\,\bigl(2t + 3\bigr)\,dt.
        \]
        Split this into two integrals:
        \[
          \int_{0}^{1} \sin(\pi t)\,(2t)\,dt \quad+\quad \int_{0}^{1} 3\,\sin(\pi t)\,dt.
        \]
        For \(\displaystyle\int_{0}^{1} 3\,\sin(\pi t)\,dt\), we get:
        \begin{align*}
          3 \int_{0}^{1} \sin(\pi t)\,dt & = \frac{3}{\pi} \Bigl[ -\cos(\pi t) \Bigr]_{0}^{1}    \\
                                         & = \frac{3}{\pi} \bigl( -\cos(\pi) - (-\cos(0)) \bigr) \\
                                         & = \frac{3}{\pi} \bigl( -(-1) - (-1) \bigr)            \\
                                         & = \frac{3}{\pi} (1 + 1)                               \\
                                         & = \frac{6}{\pi}.
        \end{align*}

        For \(\displaystyle\int_{0}^{1} 2t\,\sin(\pi t)\,dt\), we use integration by parts, and find the required parts:
        \begin{align*}
          u                                 & = 2t,    & dv                           & = \sin(\pi t)\,dt,           \\
          \text{after differentiating, } du & = 2\,dt. & \text{after integrating, } v & = -\frac{1}{\pi}\cos(\pi t).
        \end{align*}
        Now, we plug our values into the following formula:
        \[
          \int_{0}^{1} u\,dv = uv \bigr|_{0}^{1} - \int_{0}^{1} v\,du.
        \]
        Evaluating the first half:
        \begin{align*}
          2t \cdot \Bigl(-\frac{1}{\pi}\cos(\pi t)\Bigr) \Bigr|_{0}^{1} & = -\frac{2}{\pi} \bigl( 1 \cdot \cos(\pi) - 0 \bigr) \\
                                                                        & = -\frac{2}{\pi} \bigl( -1 \bigr)                    \\
                                                                        & = \frac{2}{\pi}.
        \end{align*}
        Then the second:
        \begin{align*}
          -\int_{0}^{1} \Bigl(-\frac{1}{\pi}\cos(\pi t)\Bigr) \cdot 2\,dt & = \frac{2}{\pi} \int_{0}^{1} \cos(\pi t)\,dt                           \\
                                                                          & = \frac{2}{\pi} \cdot \dfrac{1}{\pi} \Bigl[ \sin(\pi t) \Bigr]_{0}^{1} \\
                                                                          & = \frac{2}{\pi^2} \bigl( \sin(\pi) - \sin(0) \bigr)                    \\
                                                                          & = 0.
        \end{align*}
        Summing up, \(\displaystyle\int_{0}^{1} 2t\,\sin(\pi t)\,dt = \frac{2}{\pi}\). Therefore, the total area is:
        \[
          \boxed{\frac{2}{\pi} + \frac{6}{\pi} = \frac{8}{\pi}.}
        \]

        }

  \item (6 points) Determine an equation, in symmetric form, for the line of intersection of the two planes with equations \(4x + 2y - z + 5 = 0\) and \(3x - y + z + 1 = 0\).

        \sol{
          To find an equation of the line of intersection, we need to add the plane equations to eliminate \(z\):
          \[
            \begin{array}{ccccccr}
              4x & + & 2y & - & z            & = & -5 \\
              3x & - & y  & + & z            & = & -1 \\[1pt]
              \noalign{\hrule height 1pt}
              7x & + & y  &   & \phantom{3y} & = & -6
            \end{array}
          \]
          Thus, \(y = -6 -7x\). Substitute this equation into the second equation to express \(z\) in terms of \(x\):
          \begin{align*}
            3x - (-6 - 7x) + z & = -1        \\
            10x + z            & = -7        \\
            z                  & = -10x - 7.
          \end{align*}
          With \(y\) and \(z\) in terms of \(x\), we need to define \(x\) in terms of \(t\). Choose parameter \(t\) as \(t = x\). When we substitute our value \(t\) back into the previous two equations, we see that the parametric equations for the line of intersection are \(x = t\), \(y = -6 - 7t \), and \(z = -7 - 10t\). Therefore, the symmetric equations for the line are:
          \[
            \boxed{x = \frac{y + 6}{-7} = \frac{z + 7}{-10}.}
          \]
        }
  \item (4 points each) Let \(\mb{r}(t) = \sin(t)\mb{i} + 7t\mb{j} - \cos(t)\mb{k}\) represent the position of a particle at time \(t\).
        \begin{enumerate}
          \item Find the velocity vector, \(\mb{v}(t) = \dvecfuc{r}{t}\).

                \sol{
                  \[
                    \mb{v}(t) = \mb{r}'(t) = \cos(t)\,\mb{i} + 7\,\mb{j} + \sin(t)\,\mb{k}.
                  \]
                }
                \newpage
          \item What total distance is travelled by the particle over the time interval \([0,\,4]\)? Find this as an exact value -- though, you are welcome to check the integral with your calculator.

                \sol{
                  The speed is \(\|\mb{v}(t)\| = \sqrt{\cos^2(t)+7^2+\sin^2(t)} = \sqrt{49 + (\cos^2(t)+\sin^2(t))} = \sqrt{50} = 5\sqrt{2}.\)

                  Hence, the distance travelled over \([0,4]\) is
                  \[
                    \int_{0}^{4} \|\mb{v}(t)\|\,dt \;=\; \int_{0}^{4} 5\sqrt{2}\,dt
                    = 5\sqrt{2} \cdot 4 = 20\sqrt{2}.
                  \]
                }
          \item Find the unit tangent vector, \(\mb{T}(t)\).

                \sol{
                  \[
                    \mb{T}(t) = \frac{\mb{v}(t)}{\|\mb{v}(t)\|}
                    = \frac{\langle \cos(t),\,7,\,\sin(t)\rangle}{5\sqrt{2}}
                    = \left\langle\frac{\cos(t)}{5\sqrt{2}},\,\frac{7}{5\sqrt{2}},\,\frac{\sin(t)}{5\sqrt{2}} \right\rangle.
                  \]
                }
          \item Find the unit normal vector, \(\mb{N}(t)\).

                \sol{
                  To find the unit normal vector, we first need to find the derivative of the unit tangent vector. Note that we don't have to deal with any quotient rule shenanigans because the denominator is a constant. Thus, we have:
                  \[
                    \mb{T}'(t) = \left\langle-\frac{\sin(t)}{5\sqrt{2}},\,0,\,\frac{\cos(t)}{5\sqrt{2}}\right\rangle,
                  \]
                  and
                  \(\|\mb{T}'(t)\| = \frac{1}{5\sqrt{2}}\sqrt{\sin^2(t)+\cos^2(t)} = \frac{1}{5\sqrt{2}}.\)

                  Therefore,
                  \[
                    \boxed{\mb{N}(t) = \frac{\mb{T}'(t)}{\|\mb{T}'(t)\|}
                      = \left\langle-\sin(t),\;0,\;\cos(t)\right\rangle.}
                  \]
                }
        \end{enumerate}
  \item (4 points each) Consider the vector valued function \(\vecfuc{r}{t} = \brackett{f(t),g(t)}\). Let \(\alpha\) be a fixed number and define
        \[
          \mb{r}_{\alpha}(t) = \brackett{f(t)\cos(\alpha) - g(t)\sin(\alpha),\, f(t)\sin(\alpha) + g(t)\cos(\alpha)}.
        \]
        \begin{enumerate}
          \item Show that for any \(t\), \(\| \mb{r}(t) \| = \| \mb{r}_{\alpha}(t) \|\).

                \sol{
                  We know that \(\| \mb{r}(t) \| = \mysqrt{f(t)^{2} + g(t)^{2}}\), so our goal is to show that \(\| \mb{r}_{\alpha}(t) \| = \mysqrt{f(t)^{2} + g(t)^{2}}\). Thus, we have:
                  \[
                    \| \mb{r}_{\alpha}(t) \| = \mysqrt{\bigl(f(t)\cos(\alpha) - g(t)\sin(\alpha)\bigr)^{2} + \bigl(f(t)\sin(\alpha) + g(t)\cos(\alpha)\bigr)^{2}}
                  \]
                  From here, we can expand and simplify:
                  \begin{equation*}
                    \| \mb{r}_{\alpha}(t) \| = \mysqrt{
                      \begin{aligned}
                          & f(t)^{2}\cos^{2}(\alpha) - 2f(t)g(t)\cos(\alpha)\sin(\alpha) + g(t)^{2}\sin^{2}(\alpha) \\
                        + & f(t)^{2}\sin^{2}(\alpha) + 2f(t)g(t)\sin(\alpha)\cos(\alpha) + g(t)^{2}\cos^{2}(\alpha) \\
                      \end{aligned}
                    }.
                  \end{equation*}
                  Simplifying, we get:
                  \[
                    \| \mb{r}_{\alpha}(t) \| = \mysqrt{f(t)^{2} + g(t)^{2}}.
                  \]
                  Therefore, we have shown that for any \(t\), \(\| \mb{r}(t) \| = \| \mb{r}_{\alpha}(t) \|\).
                }
                \newpage
          \item Show that for any \(t\), \(\| \mb{r}'(t) \| = \| \mb{r}_{\alpha}'(t) \|\). [Hint: Remember that \(\alpha\) is a constant!]

                \sol{
                  Because we know that \(\alpha\) is a constant, when we take the derivative of \(f(t)\cos(\alpha) - g(t)\sin(\alpha)\), for instance, we will simply get \(f'(t)\cos(\alpha) - g'(t)\sin(\alpha)\). With this revelation, we see that this problem is essentially the same as the last one, except with the derivative of the function. Thus, we have:
                  \[
                    \boxed{||\mb{r}'(t)\|  = \mysqrt{f'(t)^{2} + g'(t)^{2}} =  \|\mb{r}_{\alpha}'(t)||.}
                  \]
                }
          \item Determine \(\vecfuc{T}{t}\) and \(\mb{T}_{\alpha}(t)\), the unit tangent vectors for \(\mb{r}\) and \(\mb{r}_{\alpha}\), respectively.

                \sol{
                The tangent vector for \(\mb{r}(t)\) is \(\vecfuc{T}{t} = \frac{\mb{r}'(t)}{||\mb{r}'(t)\| }\). We have already found \(\mb{r}'(t)\) and \( \|\mb{r}'(t)||\), so we can plug these values in to find \(\vecfuc{T}{t}\):
                \[
                  \vecfuc{T}{t} = \left\langle\frac{\mb{r}'(t)}{||\mb{r}'(t)||} = \frac{f'(t)}{\mysqrt{f'(t)^{2} + g'(t)^{2}}},\,\frac{g'(t)}{\mysqrt{f'(t)^{2} + g'(t)^{2}}}\right\rangle.\footnotemark[1]
                \]
                Similarly, for \(\mb{r}_{\alpha}(t)\):
                \[
                  \mb{T}_{\alpha}(t) = \frac{\mb{r}_{\alpha}'(t)}{||\mb{r}_{\alpha}'(t)||} = \left\langle\frac{f'(t)\cos(\alpha) - g'(t)\sin(\alpha)}{\mysqrt{f'(t)^{2} + g'(t)^{2}}},\,\frac{f'(t)\sin(\alpha) + g'(t)\cos(\alpha)}{\mysqrt{f'(t)^{2} + g'(t)^{2}}}\right\rangle.\footnotemark[1]
                \]
                \footnotemark[1](\textit{Note:} This answer is only defined if \(f'(t)^{2} + g'(t)^{2} \neq 0\).)
                }
          \item Find the angle \(\theta\) between the unit tangent vectors for \(\vecfuc{r}{t}\) and \(\mb{r}_{\alpha}(t)\).

                \sol{
                To find the angle between any two vectors, we use the formula:
                \[
                  \mathbf{u} \cdot \mathbf{v} = \norm{u}\norm{v}\cos(\theta),
                \]
                where \(0 \leq \theta \leq \pi\) is the angle between \(\mathbf{u}\) and \(\mathbf{v}\). Since we are dealing with unit vectors, we can simplify this to:
                \[
                  \vecfuc{T}{t} \cdot \mb{T}_{\alpha}(t) = \cos(\theta).
                \]
                Thus, we need to find the dot product between the two tangent vectors:
                \[
                  \vecfuc{T}{t} \cdot \mb{T}_{\alpha}(t) = \left(\frac{f'(t)\bigl(f'(t)\cos(\alpha) - g'(t)\sin(\alpha)\bigr) + g'(t)\bigl(f'(t)\sin(\alpha) + g'(t)\cos(\alpha)\bigr)}{f'(t)^{2} + g'(t)^{2}}\right).
                \]
                Multiplying we see that:
                \[
                  \vecfuc{T}{t} \cdot \mb{T}_{\alpha}(t) = \left(\frac{f'(t)^{2}\cos(\alpha) - f'(t)g'(t)\alpha + f'(t)g'(t)\alpha + g'(t)^{2}\cos(\alpha)}{f'(t)^{2} + g'(t)^{2}}\right) = \frac{f'(t)^{2} + g'(t)^{2}}{f'(t)^{2} + g'(t)^{2}}\cos(\alpha),
                \]
                which simplifies to \(\cos(\alpha)\). Thus, returning to our original formula:
                \[
                  \boxed{cos(\theta) = \cos(\alpha) \quad \Rightarrow \quad \theta = \alpha \quad \text{ for } 0 \leq \theta \leq \pi.}
                \]
                [\textit{2 Notes:} (1) Same note as before about \(f'(t)^{2} + g'(t)^{2} \neq 0\), and (2) we get \(f'(t)^{2} + g'(t)^{2}\) in the main denominator because both vectors have the same denominator (\(\mysqrt{f'(x)^{2} + g'(x)^{2}}\).)]
                }
        \end{enumerate}
  \item (3 points each) Suppose that each of \(\mb{u}\) and \(\mb{v}\) are unit vectors and are orthogonal. Let \(r\) be a fixed positive number and define the vector-valued function \(\mb{r}(t) = r \cos(t)\mb{u} + r \sin(t)\mb{v}\).
        \begin{enumerate}
          \item Explain why there is a single plane that contains \(\vecfuc{r}{t}\) for each choice of \(t\).

                \sol{
                  Because \(\mb{u}\) and \(\mb{v}\) are orthogonal unit vectors, that means they span a plane. Thus, for any \(t\), the vector \(\mb{r}(t)\) is a linear combination of \(\mb{u}\) and \(\mb{v}\), and therefore lies in the plane spanned by \(\mb{u}\) and \(\mb{v}\).
                }
          \item Determine \(\| \vecfuc{r}{t} \|\).

                \sol{
                  \[
                    \| \vecfuc{r}{t} \| = \| r \cos(t)\mb{u} + r \sin(t)\mb{v} \| = \mysqrt{r^{2}\cos^{2}(t) + r^{2}\sin^{2}(t)} = \mysqrt{r^{2}} = r,
                  \]
                  where we used the fact that \(\mb{u}\) and \(\mb{v}\) are unit vectors and their magnitude is equal to 1.
                }
          \item Determine each of \(\proj_{\mb{u}}\mb{r}(t)\) and \(\proj_{\mb{v}}\mb{r}(t)\).

                \sol{
                  \vspace*{-1.25cm}
                  \begin{multicols}{2}
                    \phantom{\textit{text}}
                    \begin{align*}
                      \proj_{\mb{u}}\mb{r}(t) & = \frac{\mb{r}(t) \cdot \mb{u}}{||\mb{u}||^{2}}\mb{u}                            \\
                                              & = \left(\bigl(r\cos(t) \mb{u} + r \sin(t) \mb{v}\bigr) \cdot \mb{u}\right)\mb{u} \\
                                              & = \left(r\cos(t) \mb{u} \cdot \mb{u} + r\sin(t) \mb{v} \cdot \mb{u}\right)\mb{u} \\
                                              & = \left(r\cos(t) + 0\right)\mb{u}                                                \\
                                              & = r\cos(t)\mb{u}.
                    \end{align*}
                    \phantom{\textit{text}}
                    \begin{align*}
                      \proj_{\mb{v}}\mb{r}(t) & = \frac{\mb{r}(t) \cdot \mb{v}}{||\mb{v}||^{2}}\mb{v}                            \\
                                              & = \left(\bigl(r\cos(t) \mb{u} + r \sin(t) \mb{v}\bigr) \cdot \mb{v}\right)\mb{v} \\
                                              & = \left(r\cos(t) \mb{u} \cdot \mb{v} + r\sin(t) \mb{v} \cdot \mb{v}\right)\mb{v} \\
                                              & = \left(0 + r\sin(t)\right)\mb{v}                                                \\
                                              & = r\sin(t)\mb{v}.
                    \end{align*}
                  \end{multicols}

                }
          \item What does this tell you about the curve generated by \(\mb{r}(t)\)?

                \sol{
                  The curve generated by \(\mb{r}(t)\) is a circle of radius \(r\) in the plane spanned by \(\mb{u}\) and \(\mb{v}\).
                }
        \end{enumerate}
\end{enumerate}


\end{document}