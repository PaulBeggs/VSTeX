\documentclass[11pt]{article}

\usepackage{fix-cm}
\usepackage{tikz}
\usetikzlibrary{shapes,backgrounds,calc,patterns}
\usepackage{amsmath,amsthm} 
\usepackage{amssymb}
\usepackage[framemethod=tikz]{mdframed}
% \usepackage{draculatheme}
\usepackage{mathrsfs}
\usepackage{changepage}
\usepackage{multicol}
\usepackage{mathtools}
\usepackage{hyperref}
\usepackage{slashed}
\usepackage{enumerate}
\usepackage{booktabs}
\usepackage{enumitem}
\usepackage{kantlipsum} 
\usepackage{pgfplots}
\pgfplotsset{compat=1.18}
\usetikzlibrary{decorations.markings}

\setlength{\parindent}{0pt}

\newmdenv[
  topline=false,
  bottomline=true,
  rightline=false,
  leftline=true,
  linewidth=1.5pt,
  linecolor=black, % default color, will be overridden in custom commands
%   backgroundcolor=draculabg, % Needed for Dracula theme
%   fontcolor=draculafg, % Needed for Dracula theme
  innertopmargin=0pt,
  innerbottommargin=5pt,
  innerrightmargin=10pt,
  innerleftmargin=10pt,
  leftmargin=0pt,
  rightmargin=0pt,
  skipabove=\topsep,
  skipbelow=\topsep,
]{customframedproof}

\newenvironment{proofpart}[2][black]{
    \begin{mdframed}[
        topline=false,
        bottomline=false,
        rightline=false,
        leftline=true,
        linewidth=1pt,
        linecolor=#1!40, % Custom color
        % innertopmargin=10pt,
        % innerbottommargin=10pt,
        innerleftmargin=10pt,
        innerrightmargin=10pt,
        leftmargin=0pt,
        rightmargin=0pt,
        % skipabove=\topsep,
        % skipbelow=\topsep%
    ]
    \noindent
    \begin{minipage}[t]{0.08\textwidth}%
        \textbf{#2}%
    \end{minipage}%
    \begin{minipage}[t]{0.90\textwidth}%
        \begin{adjustwidth}{0pt}{0pt}%
}{
    \end{adjustwidth}
    \end{minipage}
    \end{mdframed}
}

\newenvironment{solution}
  {\textit{Solution.}}



%%% AESTHETICS %%%
%-%-%-%-%-%-%-%-%-%-%-%-%-%-%-%-%-%-%-%-%-%-%-%-%-%-%-%-%-%-%-%-%-%-%-%-%-%-%


%%% Dimensions and Spacing %%%
\usepackage[left=0.5in,right=0.5in,top=1in,bottom=1in]{geometry}
% \usepackage{setspace}
% \linespread{1}
\usepackage{listings}
% \usepackage{minted}

%%% Define new colors %%%
\usepackage{xcolor}
\definecolor{orangehdx}{rgb}{0.96, 0.51, 0.16}

% Normal colors
\definecolor{xred}{HTML}{BD4242}
\definecolor{xblue}{HTML}{4268BD}
\definecolor{xgreen}{HTML}{52B256}
\definecolor{xpurple}{HTML}{7F52B2}
\definecolor{xorange}{HTML}{FD9337}
\definecolor{xdotted}{HTML}{999999}
\definecolor{xgray}{HTML}{777777}
\definecolor{xcyan}{HTML}{80F5DC}
\definecolor{xpink}{HTML}{F690EA}
\definecolor{xgrayblue}{HTML}{49B095}
\definecolor{xgraycyan}{HTML}{5AA1B9}

% Dark colors
\colorlet{xdarkred}{red!85!black}
\colorlet{xdarkblue}{xblue!85!black}
\colorlet{xdarkgreen}{xgreen!85!black}
\colorlet{xdarkpurple}{xpurple!85!black}
\colorlet{xdarkorange}{xorange!85!black}
\definecolor{xdarkcyan}{HTML}{008B8B}
\colorlet{xdarkgray}{xgray!85!black}

% Very dark colors
\colorlet{xverydarkblue}{xblue!50!black}

% Document-specific colors
\colorlet{normaltextcolor}{black}
\colorlet{figtextcolor}{xblue}

% Enumerated colors
\colorlet{xcol0}{black}
\colorlet{xcol1}{xred}
\colorlet{xcol2}{xblue}
\colorlet{xcol3}{xgreen}
\colorlet{xcol4}{xpurple}
\colorlet{xcol5}{xorange}
\colorlet{xcol6}{xcyan}
\colorlet{xcol7}{xpink!75!black}

% Blue-Purple (should just used colorbrewer...)
\definecolor{xrainbow0}{HTML}{e41a1c}
\definecolor{xrainbow1}{HTML}{a24057}
\definecolor{xrainbow2}{HTML}{606692}
\definecolor{xrainbow3}{HTML}{3a85a8}
\definecolor{xrainbow4}{HTML}{42977e}
\definecolor{xrainbow5}{HTML}{4aaa54}
\definecolor{xrainbow6}{HTML}{629363}
\definecolor{xrainbow7}{HTML}{7e6e85}
\definecolor{xrainbow8}{HTML}{9c509b}
\definecolor{xrainbow9}{HTML}{c4625d}
\definecolor{xrainbow10}{HTML}{eb751f}
\definecolor{xrainbow11}{HTML}{ff9709}

%%% FIGURES %%%
\usepackage{graphicx}  
\graphicspath{ {images/} }  
% \numberwithin{figure}{section}
\usepackage{float}
\usepackage{caption}

%%% Hyperlinks %%%
\usepackage{hyperref}
\definecolor{horange}{HTML}{f58026}
\hypersetup{
	colorlinks=true,
	linkcolor=horange,
	filecolor=horange,      
	urlcolor=horange,
}

\newcommand{\mysqrt}[1]{%
  \mathpalette\foo{#1}%
}
\newcommand{\dmysqrt}[1]{%
  \mathpalette\foodisplay{#1}%
}

\newcommand{\sol}[1]{
    \begin{customframedproof}[linecolor=orangehdx!75,]
        \begin{solution}
        #1
        \end{solution}
    \end{customframedproof}
}
\allowdisplaybreaks

% !TeX spellcheck = off
\newcommand{\foo}[2]{%
	% #1: math style, #2: content
	\sbox0{$#1\sqrt{#2}$}% Measure the size of the standard sqrt in the current style
	\begin{tikzpicture}[baseline=(sqrt.base)]
		\node[inner sep=0, outer sep=0] (sqrt) {$#1\sqrt{#2}$}; % Use the current math style
		\draw([yshift=-0.045em]sqrt.north east) -- ++(0,-0.5ex); % Draw the tick
  \end{tikzpicture}%
}
% !TeX spellcheck = off
\newcommand{\foodisplay}[2]{%
	% #1: math style, #2: content
	\sbox0{$#1\sqrt{#2}$}% Measure the size of the standard sqrt in the current style
	\begin{tikzpicture}[baseline=(sqrt.base)]
		\node[inner sep=0, outer sep=0] (sqrt) {$\displaystyle\sqrt{#2}$}; % Force displaystyle
		\draw[line width=0.4pt] ([yshift=-0.044em]sqrt.north east) -- ++(0,-0.5ex); % Draw the tick
	\end{tikzpicture}%
}

\newcommand{\barNotationT}[1]{\bigg|_{t = #1}}

\newcommand{\cyanit}[1]{\textit{\textcolor{cyan}{#1}}}

\newcommand{\brackett}[1]{\left\langle #1 \right\rangle}

\newcommand{\norm}[1]{\left\lVert \mathbf{#1}\right\rVert}

\newcommand{\imb}{\mb{i}}
\newcommand{\jmb}{\mb{j}}
\newcommand{\kmb}{\mb{k}}
\newcommand{\rmb}{\mb{r}}
\newcommand{\umb}{\mb{u}}

\newcommand{\vecfuc}[2]{\mb{#1}(#2)}
\newcommand{\dvecfuc}[2]{\mb{#1}'(#2)}
\newcommand{\normdvecfuc}[2]{||\mb{#1}'(#2)||}

\newcommand{\proj}{\text{proj}}
\newcommand{\p}{\partial}
\newcommand{\mb}[1]{\mathbf{#1}}

% \renewcommand{\theenumi}{\arabic{enumi}} 
% \renewcommand{\labelenumi}{\theenumi.}

%%% Custom Comands %%%
% Natural Numbers 
\newcommand{\N}{\ensuremath{\mathbb{N}}}

% Whole Numbers
\newcommand{\W}{\ensuremath{\mathbb{W}}}

% Integers
\newcommand{\Z}{\ensuremath{\mathbb{Z}}}

% Rational Numbers
\newcommand{\Q}{\ensuremath{\mathbb{Q}}}

% Real Numbers
\newcommand{\R}{\ensuremath{\mathbb{R}}}

% Complex Numbers
\newcommand{\C}{\ensuremath{\mathbb{C}}}

\newcommand{\I}{\ensuremath{\mathbb{I}}}


\begin{document}

\begin{enumerate}
    \item Consider the vector field \(\mb{F} = \brackett{2x - 2y, \, 2x + 2y, \, 0}\)
    \begin{enumerate}
        \item (2 points) Show that \(\mb{F}\) is not conservative. \\

        \sol{
            For a vector field to be conservative, its curl must be zero. Computing the curl of \(\mb{F}\):
            \begin{align*}
              \nabla \times \mb{F} &= \left( \frac{\partial R}{\partial y} - \frac{\partial Q}{\partial z} \right) \mathbf{i} + \left( \frac{\partial P}{\partial z} - \frac{\partial R}{\partial x} \right) \mathbf{j} + \left( \frac{\partial Q}{\partial x} - \frac{\partial P}{\partial y} \right) \mathbf{k} \\
              &= (0 - 0)\mb{i} + (0 - 0)\mb{j} + (2 - (-2))\mb{k} \\
              &= 4\mb{k}.
            \end{align*}
            Since \(\nabla \times \mb{F} = \brackett{0,0,4} \neq \brackett{0,0,0}\), \(\mb{F}\) is not conservative.
        }
        \item (2 points) Show that \(\mb{F}\) is not solenoidal. \\

        \sol{
            A vector field is solenoidal if its divergence is zero. Thus:
            \begin{align*}
                \nabla \cdot \mb{F} &= \frac{\partial}{\partial x}(2x - 2y) + \frac{\partial}{\partial y}(2x + 2y) + \frac{\partial}{\partial z}(0) \\
                &= 2 + 2 + 0 \\
                &= 4.
            \end{align*}
            
            Since \(\nabla \cdot \mb{F} = 4 \neq 0\), \(\mb{F}\) is not solenoidal.
        }
        \item (2 points each) Let \(\mb{G} = \brackett{2x, \, 2y, \, 0}\) and \(\mb{H} = \brackett{-2y, \, 2x, \, 0}\). Notice that \(\mb{F} = \mb{G} + \mb{H}\).
        \begin{enumerate}
            \item Show that \(\mb{G}\) \textit{is} conservative, and find a potential function \(g\). \\

            \sol{
                \begin{align*}
					\nabla \times \mb{G} &= (0 - 0)\imb - (0 - 0)\jmb + \left(\frac{\partial}{\partial x}(2y) - \frac{\partial}{\partial y}(2x)\right)\kmb.
                \end{align*}
                
                Since \(\nabla \times \mb{G} = \brackett{0,0,0}\), \(\mb{G}\) is conservative. \\

				A potential function is one where \(\nabla g = \mb{G}\). Thus, since we have a (relatively) straightforward vector field, the following will work for a potential function:
				\[
					g(x,y,z) = x^{2} + y^{2} + K. \quad (\text{Setting } K = 0 \text{ for simplicity.})
				\]
				We can check this by finding \(\nabla g\):
				\[
					\nabla g(x,y,z) = \left\langle\frac{\p}{\p x}(x^{2} + y^{2}), \frac{\p}{\p y}(x^{2} + y^{2}), \frac{\p}{\p z}(x^{2} + y^{2})  \right\rangle = \langle 2x, 2y, 0\rangle = \mb{G}.
				\]
            }
			\newpage
            \item Show that \(\mb{H}\) \textit{is} solenoidal -- so that it is the curl of some other vector field \(\mb{C}\). Find such a \(\mb{C}\). [Hint: You might want to choose the \(z\)-component to be 0.] \\
            
            \sol{
                \begin{align*}
                    \nabla \cdot \mb{H} &= \frac{\partial}{\partial x}(-2y) + \frac{\partial}{\partial y}(2x) + \frac{\partial}{\partial z}(0) \\
                    &= 0 + 0 + 0 \\
                    &= 0.
                \end{align*}
                Since \(\nabla \cdot \mb{H} = 0\), the vector field \(\mb{H}\) is solenoidal. \\

				For finding the vector field \(\mb{C}\), we can look at the formula for the curl of a vector field from (a). We see that we need to have our \(x\)-component and \(y\)-component to be equal to \(\mb{H}\). Thus, for the \(x\)-component, we need to get \(-2y\). The only part of the formula that allows us that is the \(\frac{\p R}{\p y}\) part. The same argument can be said for the \(y\)-component. We need to get a positive \(2x\), and we can get that through \((-\frac{\p R}{\p x})\). Thus, this leads me to believe that we need to keep our function in \(R\) of \(\mb{C}\), and ensure \(P\) and \(Q\) are both 0. This leaves us with:
				\[
					\mb{C} = \langle 0, 0, -x^{2} - y^{2} \rangle.
				\]
				Taking the curl of this vector field, we can confirm our choice:
				\begin{align*}
					\nabla \times \mb{C} &= \left\langle \frac{\p}{\p y}(-x^{2} - y^{2}) - 0, 0 - \frac{\p}{\p x} (-x^{2} - y^{2}), 0 - 0 \right\rangle \\
					&= \langle -2y, -(-2x), 0 \rangle \\
					&= \langle -2y, 2x, 0 \rangle.
				\end{align*}
            }
        \end{enumerate}
        \item (2 points) Conclude that we have decomposed \(\mb{F}\) into a purely conservative (i.e., irrotational) part and a purely solenoidal (i.e., divergence-free) part, so that \(\mb{F} = \nabla g + \nabla \times \mb{C}\). [This can always be done, so long as everything is continuous enough.] \\

        \sol{
            We have shown that:
            \begin{itemize}
                \item \(\mb{G} = \brackett{2x, 2y, 0}\) is conservative with potential function \(g(x,y,z) = x^2 + y^2\), so \(\mb{G} = \nabla g\).
                \item \(\mb{H} = \brackett{-2y, 2x, 0}\) is solenoidal and can be expressed as \(\mb{H} = \nabla \times \mb{C}\) where \(\mb{C} = \brackett{0,0,-x^2-y^2}\).
                \item \(\mb{F} = \mb{G} + \mb{H}\).
            \end{itemize}
            Therefore, we have successfully decomposed \(\mb{F}\) as follows:
            \begin{align*}
                \mb{F} &= \mb{G} + \mb{H} \\
                &= \nabla g + \nabla \times \mb{C} \\
                &= \nabla(x^2 + y^2) + \nabla \times \brackett{0,0,-x^2-y^2} \\
				&= \brackett{2x, 2y, 0} + \brackett{-2y, 2x, 0} \\
				&= \brackett{2x - 2y, \, 2x + 2y, \, 0}.
            \end{align*}
        }
    \end{enumerate}
    \newpage
    \item (8 points) Let \(\mb{F} = \brackett{6x^{2}y, \, -6x - 4y}\). Let \(C\) be the rectangle with endpoints \((0, 0)\), \((4,0)\), \((4,1)\), and \((0,1)\), with positive orientation. Determine the exact value of the flux of \(\mb{F}\) over \(C\): \(\displaystyle \oint \mb{F} \cdot \mb{N} \, ds\).

    \sol{
		Since the rectangle is closed, we can use Green's Theorem for flux:
		\[
			\oint \mb{F} \cdot \mb{N} \, ds = \iint_{D} (\nabla \cdot \mb{F}) \, dA = \iint_{D} \left(\frac{\p P}{\p x} + \frac{\p Q}{\p y}\right) \, dA.
		\]
		Finding the integrand:
		\[
			\frac{\p P}{\p x} = \frac{\p}{\p x} (6x^{2}y) = 12xy, \quad \frac{\p Q}{\p y} = \frac{\p}{\p y} (-6x - 4y) = -4 \quad \Rightarrow \quad  \frac{\p P}{\p x} + \frac{\p Q}{\p y} = 12xy - 4.
		\]
		Leaving us with:
		\[
			\int_{0}^{4} \int_{0}^{1} (12xy - 4) \, dy \, dx.
		\]
		Integrating with respect to \(y\):
		\[
			\int_{0}^{1} (12xy - 4) \, dy = \left[6xy^{2} - 4y\right]_{0}^{1} = 6x - 4.
		\]
		Then, integrating with respect to \(x\):
		\[
			\int_{0}^{4} (6x - 4) \, dx = \left[3x^{2} - 4x\right]_{0}^{4} = (3 \cdot 16 - 4 \cdot 4) = \boxed{32}.
		\]
    }
    \newpage
    \item (8 points) Determine \(\displaystyle \iint_{S}z \, dS\) where \(S\) is the surface \(y = 3x + z^{2}\), where \(0 \leq x \leq 1\) and \(0 \leq z \leq 2\).

    \sol{
		Parameterize \(S\) with:
		\[
			r(x,z) = (x, \, 3x + z^{2}, \, z), \quad  0 \leq x \leq 1, \quad 0 \leq z \leq 2.
		\]
		Then, we need to find \(dS\):
		\[
			dS = \| r_{x} \times r_{z} \| \, dx\, dy,
		\]
		where
		\[
			r_{x} = (1,3,0) \quad \text{and} \quad r_{z} = (0,2z,1)
		\]
		gives
		\[
			r_{x} \times r_{z} = 
			\begin{vmatrix}
				\mb{i} & \mb{j} & \mb{k} \\
				1 & 3 & 0 \\
				0 & 2z & 1 
			\end{vmatrix} = \brackett{3, -1, 2z} \quad \Rightarrow \quad \|r_{x} \times r_{z} \| = \mysqrt{10 + 4z^{2}}.
		\]
		Setting up the integral:
		\[
			\iint_{S} z \, dS = \int_{0}^{1}\int_{0}^{2}z \mysqrt{10 + 4z^{2}} \, dz \, dx.
		\]
		Since the integrand does not depend on \(x\), we simply find:
		\[
			\int_{0}^{1} dx = 1.
		\]
		This leaves us with the \(z\)-integral:
		\begin{align*}
			\int_{0}^{2} z \mysqrt{10 + 4z^{2}} \, dz.
		\end{align*}
		Using \(u\)-substitution:
		\[
			u = 10 + 4z^{2} \quad \Rightarrow \quad du = 8z \, dz,
		\]
		with new bounds 10 and 26:
		\[
			\int_{10}^{26} \frac{1}{8}\mysqrt{u} \, du = \frac{1}{8}\left[ \frac{2}{3} u^{3/2} \right]_{10}^{26} = \boxed{\frac{1}{12}(26^{3/2} - 10^{3/2})}.
		\]
    }
    \newpage
    \item (8 points) Let \(\mb{F}(x,y,z) = \brackett{y, \, x, \, xz}\) and the surface \(S\) be the part of the paraboloid \(z = 4 - x^{2} - y^{2}\) that lies above \(0 \leq x \leq 1\) and \(0 \leq y \leq 1\), where positive orientation is directed upward. Determine
    \[
        \iint_{S} \mb{F} \cdot d\mb{S}.
    \]
    [Hint: The surface, projected into the \(xy\)-plane, is \textit{not} a quarter of the unit circle, and therefore, it is likely easiest to parameterize \(S\) in Cartesian coordinates.]

    \sol{
		Parameterize \(S\) with:
		\[
			\mb{r}(x,y) = (x,y, 4 - x^{2} - y^{2}), \quad 0 \leq x \leq 1, \quad 0 \leq y \leq 1.
		\]
		Then, finding \(d\mb{S}\):
		\[
			d\mb{S} = r_{x} \times r_{y} \, dx \, dy,
		\]
		where 
		\[
			r_{x} = (1,0,-2x) \quad \text{and} \quad r_{y} = (0,1,-2y)
		\]
		gives 
		\[
			r_{x} \times r_{y} = (2x, 2y, 1).
		\]
		Therefore:
		\[
			d\mb{S} = (2x, 2y, 1) \, dx \, dy.
		\]
		This leaves us with the following:
		\begin{align*}
			\iint_{S} \mb{F}(\mb{r}(x,y)) \cdot (r_{x} \times r_{y}) \, dx \, dy &= \int_{0}^{1} \int_{0}^{1} (y, x, x(4 - x^{2} - y^{2})) \cdot (2x, 2y, 1) \, dx \, dy \\
			&= \int_{0}^{1} \int_{0}^{1} (4xy + 4x - x^{3} - xy^{2}) \, dx \, dy \\
			&= \int_{0}^{1} \left[ 2xy^{2} + 4xy - x^{3}y - \frac{1}{3}xy^{3} \right]_{0}^{1} \, dx \\
			&= \int_{0}^{1} \left( 2x + 4x - \frac{1}{3}x - x^{3} \right) \, dx \\
			&= \left[ x^{2} + 2x^{2} - \frac{1}{6}x^{2} - \frac{1}{4}x^{4} \right]_{0}^{1} \\
			&= \boxed{\frac{31}{12}}.
		\end{align*}
    }
    \newpage
    \item (8 points) Calculate the flux of \(\mb{F}(x,y,z) = \brackett{x^{3} + y, \, y^{3} + z^{2}, \, z^{3} + x^{3}}\) across the surface of the sphere centered at the origin with radius 2, with positive orientation. [Since the surface is closed, there are two distinct ways to do this -- though both are things you can work out, one is \textit{significantly} easier than the other.]

    \sol{
		The equation of a sphere with radius 2 gives:
		\[
			x^{2} + y^{2} + z^{2} = 4.
		\]
		Then, by the divergence theorem:
		\[
			\iint_{S} \mb{F} \cdot d\mb{S} = \iiint_{V} (\nabla \cdot \mb{F}) \, dV, \quad \text{where} \quad V = \{x^{2} + y^{2} + z^{2} \leq 4\}. 
		\]
		Solving for \(\nabla \cdot \mb{F}\):
		\[
			\frac{\p}{\p x}(x^{3} + y) + \frac{\p}{\p y}(y^{3} + z^{2}) + \frac{\p}{\p z}(z^{3} + x^{3}) = 3x^{2} + 3y^{2} + 3z^{2} = 3(x^{2} + y^{2} + z^{2}).
		\]
		Switching to spherical coordinates:
		\[
			3(x^{2} + y^{2} + z^{2}) = 3r^{2}, \quad \text{with} \quad dV = r^{2}\sin(\phi)\, dr \, d\phi \, d\theta.
		\]
		This leaves us with the integral:
		\[
			\iiint_{V} 3r^{2} \, dV = 3 \int_{0}^{2}\int_{0}^{\pi}\int_{0}^{2\pi}r^{2} (r^{2} \sin(\phi)) \, d\theta \, d\phi \, dr = 3(2\pi)(2) \int_{0}^{2}r^{4} \, dr = 12\pi\left[ \frac{1}{5}r^{5} \right]_{0}^{2} = \boxed{\frac{384\pi}{5}}.
		\]

    }
\end{enumerate}

\end{document}