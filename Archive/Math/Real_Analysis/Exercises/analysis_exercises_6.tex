\vspace*{-0.5cm}

\begin{exercise}
    {4.2.10 (Right and Left Limits)} Introductory calculus courses typically refer to the right-hand limit of a function as the limit obtained by “letting \(x\) approach \(a\) from the right-hand side.”
    \begin{enumerate}
        \item[(a)] Give a proper definition in the style of \defref{4.2.1} for the right-hand and left-hand limit statements:
              \[
                  \lim_{x \to a^+} f(x) = L \quad \text{and} \quad \lim_{x \to a^-} f(x) = M.
              \]

        \item[(b)] Prove that \(\lim_{x \to a} f(x) = L\) if and only if both the right and left-hand limits equal \(L\).
    \end{enumerate}
\end{exercise}

\sol{
    \begin{enumerate}
        \item[(a)] \textbf{Definition of Right-Hand Limit:} \\
        Let \(f: A \to \mathbb{R}\) and let \(c\) be a limit point of \(A\). We say that \(\lim_{x \to c^+} f(x) = L\) if for every \(\epsilon > 0\), there exists \(\delta > 0\) such that whenever \(0 < x - c < \delta\) (and \(x \in A\)), it follows that \(|f(x) - L| < \epsilon\).

        \textbf{Definition of Left-Hand Limit:} \\
        Let \(f: A \to \mathbb{R}\) and let \(c\) be a limit point of \(A\). We say that \(\lim_{x \to c^-} f(x) = M\) if for every \(\epsilon > 0\), there exists \(\delta > 0\) such that whenever \(-\delta < x - c < 0\) (and \(x \in A\)), it follows that \(|f(x) - M| < \epsilon\).

        \item[(b)] 
            \iffpf{
                Suppose that \(\lim_{x \to c} f(x) = L\). By definition, for every \(\epsilon > 0\), there exists \(\delta > 0\) such that whenever \(0 < |x - c| < \delta\), we have \(|f(x) - L| < \epsilon\). This inequality holds for all \(x\) approaching \(c\) from both sides, so both the right-hand and left-hand limits equal \(L\).
                }{
                Conversely, suppose that \(\lim_{x \to c^+} f(x) = L\) and \(\lim_{x \to c^-} f(x) = L\). Then, for every \(\epsilon > 0\), there exist \(\delta_1, \delta_2 > 0\) such that:
            \begin{itemize}
                \item If \(0 < x - c < \delta_1\), then \(|f(x) - L| < \epsilon\).
                \item If \(-\delta_2 < x - c < 0\), then \(|f(x) - L| < \epsilon\).
            \end{itemize}
            Let \(\delta = \min\{\delta_1, \delta_2\}\). Then, for all \(x\) with \(0 < |x - c| < \delta\), we have \(|f(x) - L| < \epsilon\).
            }{
                Therefore, \(\lim_{x \to c} f(x) = L\).
                }
            {xgray}
    \end{enumerate}
}
\begin{exercise}
    {4.2.11 (Squeeze Theorem)}Let \(f\), \(g\), and \(h\) satisfy \(f(x) \leq g(x) \leq h(x)\) for all \(x\) in some common domain \(A\). If \(\lim_{x \to c} f(x) = L\) and \(\lim_{x \to c} h(x) = L\) at some limit point \(c\) of \(A\), show \(\lim_{x \to c} g(x) = L\) as well.
\end{exercise}

\expf{
    Let \(\epsilon > 0\). Since \(\lim_{x \to c} f(x) = L\), there exists \(\delta_1 > 0\) such that whenever \(0 < |x - c| < \delta_1\), we have \(|f(x) - L| < \epsilon\). Similarly, there exists \(\delta_2 > 0\) such that whenever \(0 < |x - c| < \delta_2\), we have \(|h(x) - L| < \epsilon\). Let \(\delta = \min\{\delta_1, \delta_2\}\). Then, for all \(x\) with \(0 < |x - c| < \delta\), we have \(f(x) \leq g(x) \leq h(x)\). Thus, 
    \[
    L - \epsilon < f(x) \leq g(x) \leq h(x) < L + \epsilon.
    \]
    Therefore, \(|g(x) - L| < \epsilon\), and \(\lim_{x \to c} g(x) = L\).
}


\begin{exercise}
    {4.3.1} Let \( g(x) = \sqrt[3]{x} \).
    \begin{enumerate}
        \item Prove that \( g \) is continuous at \( c = 0 \).
        \item Prove that \( g \) is continuous at a point \( c \neq 0 \). (The identity \( a^3 - b^3 = (a - b)(a^2 + ab + b^2) \) will be helpful.)
    \end{enumerate}
\end{exercise}

Possibily use Example 4.3.8 in the book for (b).

\sol{
    \begin{enumerate}
        \item \innerpf{
            To prove that \( g \) is continuous at \( c = 0 \), we need to show that
            \[
                \lim_{x \to 0} g(x) = g(0) = 0.
            \]
            Let \( \epsilon > 0 \). Choose \( \delta = \epsilon^3 \). Then, if \( |x - 0| = |x| < \delta \), we have
            \[
                |g(x) - g(0)| = |\sqrt[3]{x} - 0| = |\sqrt[3]{x}| = |x|^{1/3} < \delta^{1/3} = (\epsilon^3)^{1/3} = \epsilon.
            \]
            Therefore, \( g \) is continuous at \( c = 0 \).
        }
        \item \innerpf{
            Let \( c \neq 0 \) and \( \epsilon > 0 \). We need to find \( \delta > 0 \) such that whenever \( |x - c| < \delta \), it follows that \( |g(x) - g(c)| < \epsilon \).

            Using the identity \( a^3 - b^3 = (a - b)(a^2 + ab + b^2) \), set \( a = \sqrt[3]{x} \) and \( b = \sqrt[3]{c} \). Then,
            \[
                x - c = (\sqrt[3]{x} - \sqrt[3]{c})\left( (\sqrt[3]{x})^2 + \sqrt[3]{x}\sqrt[3]{c} + (\sqrt[3]{c})^2 \right).
            \]
            Therefore,
            \[
                \sqrt[3]{x} - \sqrt[3]{c} = \frac{x - c}{ (\sqrt[3]{x})^2 + \sqrt[3]{x}\sqrt[3]{c} + (\sqrt[3]{c})^2 }.
            \]
            Since \( c \neq 0 \), we can assume x is sufficiently close to \(c\) such that \(\sqrt[3]{x}\) is bounded away from 0. Let \(M = \max(|\sqrt[3]{c}|,1)\). If \(|x - c| < 1\), then \(|\sqrt[3]{x}| \leq M + 1\).

            Thus, the denominator can be bounded as:
            \[
                (\sqrt[3]{x})^2 + \sqrt[3]{x}\sqrt[3]{c} + (\sqrt[3]{c})^2 \geq (\sqrt[3]{c})^{2} > 0, 
            \]
            and for \(x\) close to \(c\), we have:
            \[
                (\sqrt[3]{x})^2 + \sqrt[3]{x}\sqrt[3]{c} + (\sqrt[3]{c})^2 \leq 3(M + 1)^{2}.
            \]
            Now, using the rewritten expression for \(|g(x) - g(c)|\):
            \[
                |g(x) - g(c)| = \left| \frac{x - c}{ (\sqrt[3]{x})^2 + \sqrt[3]{x}\sqrt[3]{c} + (\sqrt[3]{c})^2 } \right|.
            \]
            For \(|x - c| < \delta\), we have:
            \[
                |g(x) - g(c)| \leq \frac{|x - c|}{(\sqrt[3]{c})^{2}}.
            \]
            To ensure \(|g(x) - g(c)| < \epsilon\), choose:
            \[
                \delta = \epsilon(\sqrt[3]{c})^{2}. 
            \]
            If \(|x - c| < \delta\), then 
            \[
                |g(x) - g(c)| \leq \frac{|x - c|}{(\sqrt[3]{c})^{2}} < \frac{\epsilon(\sqrt[3]{c})^{2}}{(\sqrt[3]{c})^{2}} = \epsilon.
            \]
            By choosing \(\delta = \epsilon(\sqrt[3]{c})^{2}\), we ensure that \(|x - c| < \delta\) implies \(|g(x) - g(c)| < \epsilon\). Therefore, \(g(x) = \sqrt[3]{x}\) is continuous at \(c \ne 0\).
        }
    \end{enumerate}
}

\begin{exercise}
    {4.3.8} Decide if the following claims are true or false, providing either a short proof or counterexample to justify each conclusion. Assume throughout that \( g \) is defined and continuous on all of \( \mathbb{R} \).
    \begin{enumerate}
        \item If \( g(x) \geq 0 \) for all \( x < 1 \), then \( g(1) \geq 0 \) as well.
        \item If \( g(r) = 0 \) for all \( r \in \mathbb{Q} \), then \( g(x) = 0 \) for all \( x \in \mathbb{R} \).
    \end{enumerate}
\end{exercise}

\sol{
    \begin{enumerate}
        \item \textbf{True.}

        Since \( g \) is continuous at \( x = 1 \), then by \defref{4.3.1}, we have:
        \[
            g(1) = \lim_{x \to 1} g(x).
        \]
        Since \(g(x) \geq 0\) for all \(x < 1\), any sequence \((x_{n})\) converging to 1 with \(x_{n} < 1\) will have \(g(x_{n}) \geq 0\). Then, by the \namrefthm{Sequential Criterion for Functional Limits} (Theorem 4.2.2), we have:
        \[
            \lim_{n \to \infty} g(x_{n}) \geq 0.
        \]
        By using the definition of continuity again, we have \( g(1) = \lim_{x \to 1}g(x)\) so \(g(1) \geq 0 \).

        \item \textbf{True.}

        Since \( g \) is continuous on \( \mathbb{R} \) and \( g(r) = 0 \) for all rational numbers \( r \), it follows that for any \( x \in \mathbb{R} \), we can find a sequence of rational numbers \( (r_n) \) converging to \( x \). By definition of continuity,
        \[
            g(x) = \lim_{n \to \infty} g(r_n) = \lim_{n \to \infty} 0 = 0.
        \]
        Therefore, \( g(x) = 0 \) for all \( x \in \mathbb{R} \).
    \end{enumerate}
}

\begin{exercise}
    {4.3.9} Assume \( h \colon \mathbb{R} \to \mathbb{R} \) is continuous on \( \mathbb{R} \) and let \( K = \{x \colon h(x) = 0\} \). Show that \( K \) is a closed set.
\end{exercise}

\expf{
    To show that \( K \) is closed, we will show that \( K \) contains all its limit points.

    Let \( (x_n) \) be a sequence in \( K \) that converges to some limit \( x \in \mathbb{R} \). Since \( h \) is continuous on \( \mathbb{R} \) and \( h(x_n) = 0 \) for all \( n \), it follows that:
    \[
        \lim_{n \to \infty} h(x_n) = h\left( \lim_{n \to \infty} x_n \right) = h(x).
    \]
    Therefore, \( h(x) = 0 \), which means \( x \in K \).

    Since every limit point of \( K \) is contained in \( K \), the set \( K \) is closed.
}