\begin{exercise}
    {2.4.7 (Limit Superior)}Let \((a_{n})\) be a bounded sequence.
    \begin{enumerate}
        \item Prove that the sequence defined by \(y_{n} = \sup\{a_{k} \colon k \geq n\}\) converges.
        \item The \textit{limit superior} of \((a_{n})\) or \(\limsupn{a_{n}}\), is defined by \[
                  \limsupn{a_{n}} = \lim_{n \rightarrow \infty} y_{n},
              \]
              where \(y_{n}\) is the sequence from part (a) of this exercise. Provide a reasonable definition for \(\liminfn{a_{n}}\) and briefly explain why it always exists for any bounded sequence.
        \item Prove that \(\liminfn{a_{n}} \leq \limsupn{a_{n}}\) for every bounded sequence, and give an example of a sequence for which the inequality is strict.
        \item Show that \(\liminfn{a_{n}} = \limsupn{a_{n}}\) if and only if \(\lim a_{n}\) exists. In this case, all three share the same value.
    \end{enumerate}
\end{exercise}

\begin{align*}
    \left|\frac{3n - 1}{2n + 1} - \frac{3}{2}\right| & = \left|\frac{2(3n - 1) - 3(2n + 1)}{2(2n + 1)}\right|   \\
                                                     & = \left|\frac{6n - 2 - 6n - 3}{4n + 2}\right|            \\
                                                     & = \left|\frac{-5}{4n + 2}\right|                         \\
                                                     & = \frac{5}{4n + 2}                                       \\
                                                     & = \frac{5}{4N} \quad \text{Let } N = \frac{5}{4\epsilon} \\
                                                     & = \epsilon
\end{align*}

The sequence \((x_n)\) is given by \(x_n = \frac{3n - 1}{2n + 1}\). Determine its limit and give a formal proof.

I have:
Let \(\epsilon > 0\). We want
\[
    \left|\frac{3n - 1}{2n + 1} - \frac{3}{2}\right| = \frac{5}{4n + 2} < \epsilon.
\]
For large \(n\): \(\frac{1}{4n + 2} < \frac{1}{4n}\) so
\[
    \frac{5}{4n + 2} < \frac{5}{4n}.
\]
To ensure \(\frac{5}{4n} < \epsilon\), choose \(N\) such that \(N > \frac{5}{4\epsilon}\). The Archimedean Principle ensures this number.

For \(n \geq N\), we have \(n > \frac{5}{4\epsilon}\), so \(\frac{5}{4n} \leq \frac{5}{4N}\). Since \(N > \frac{5}{4\epsilon}\), we know \(\frac{5}{4N} < \epsilon\). Therefore,
\[
    \left|\frac{3n - 1}{2n + 1} - \frac{3}{2}\right| = \frac{5}{4n + 2} < \epsilon.
\]

\sol{
    \begin{enumerate}
        \item We will show that \((y_{n})\) converges.

              \innerpf{
                  Since \((a_n)\) is bounded, there exists \(M > 0\) such that \(|a_n| \leq M\) for all \(n\).

                  For each \(n\), define \(y_n = \sup\{a_k : k \geq n\}\). As \(n\) increases, the set \(\{a_k : k \geq n\}\) becomes smaller, so the supremum cannot increase. Therefore, the sequence \((y_n)\) is non-increasing:
                  \[
                      y_{n+1} \leq y_n \quad \text{for all } n.
                  \]
                  Additionally, since \((a_n)\) is bounded below, so is \((y_n)\). Therefore, \((y_n)\) is a bounded, non-increasing sequence.

                  By the Monotone Convergence Theorem, every bounded, monotonic sequence converges. Thus, \((y_n)\) converges.
              }

        \item A reasonable definition for \(\liminfn{a_{n}}\) is to define \(z_n = \inf\{a_k \colon k \geq n\}\) for each \(n\). Then, the \textit{limit inferior} of \((a_n)\) is defined by:
              \[
                  \liminfn{a_{n}} = \lim_{n\to\infty} z_n.
              \]
              Since \((a_n)\) is bounded, each \(z_n\) exists and the sequence \((z_n)\) is non-decreasing. As \(n\) increases, the set \(\{a_k : k \geq n\}\) becomes smaller, so the infimum cannot decrease. Therefore, \((z_n)\) is a bounded, non-decreasing sequence, which converges by the \namrefthm{Monotone Convergence Theorem}. Hence, \(\liminfn{a_{n}}\) always exists for any bounded sequence.

        \item We will show that \(\liminfn{a_{n}} \leq \limsupn{a_{n}}\) for every bounded sequence.
              \innerpf{
                  For each \(n\), we have \(z_n = \inf\{a_k : k \geq n\} \leq a_n \leq \sup\{a_k : k \geq n\} = y_n\). This implies:
                  \[
                      z_n \leq y_n \quad \text{for all } n.
                  \]
                  Taking limits as \(n \to \infty\), we get:
                  \[
                      \lim_{n\to\infty} z_n \leq \lim_{n\to\infty} y_n,
                  \]
                  which means:
                  \[
                      \liminfn{a_{n}} \leq \limsupn{a_{n}}.
                  \]
                  For an example where the inequality is strict, consider the sequence \(a_n = (-1)^n\). Then:
                  \[
                      y_n = \sup\{(-1)^k : k \geq n\} = 1, \quad z_n = \inf\{(-1)^k : k \geq n\} = -1.
                  \]
                  Therefore:
                  \[
                      \limsupn{a_{n}} = 1, \quad \liminfn{a_{n}} = -1, \quad \liminfn{a_{n}} < \limsupn{a_{n}}.
                  \]
              }

        \item We will show that \(\liminfn{a_{n}} = \limsupn{a_{n}}\) if and only if \(\lim a_{n}\) exists. In this case, all three share the same value.

              \iffpf{
                  Suppose \(\liminfn{a_{n}} = \limsupn{a_{n}} = L\). We will show that \(\lim a_n\) exists and equals \(L\).

                  Let \(\epsilon > 0\). Since \(\limsupn{a_{n}} = L\), there exists \(N_1\) such that for all \(n \geq N_1\):
                  \[
                      y_n = \sup\{a_k : k \geq n\} < L + \epsilon.
                  \]
                  Similarly, since \(\liminfn{a_{n}} = L\), there exists \(N_2\) such that for all \(n \geq N_2\):
                  \[
                      z_n = \inf\{a_k : k \geq n\} > L - \epsilon.
                  \]
                  Let \(N = \max\{N_1, N_2\}\). Then, for all \(n \geq N\):
                  \[
                      L - \epsilon < z_n \leq a_n \leq y_n < L + \epsilon,
                  \]
                  which implies:
                  \[
                      |a_n - L| < \epsilon.
                  \]
                  Therefore, \(\lim a_n = L\).
              }{Conversely, suppose \(\lim a_n = L\). Then, for every \(\epsilon > 0\), there exists \(N\) such that for all \(n \geq N\):
                  \[
                      |a_n - L| < \epsilon.
                  \]
                  This implies that for all \(n \geq N\), the set \(\{a_k : k \geq n\}\) is contained in \((L - \epsilon, L + \epsilon)\). Therefore:
                  \[
                      y_n = \sup\{a_k : k \geq n\} \leq L + \epsilon, \quad z_n = \inf\{a_k : k \geq n\} \geq L - \epsilon.
                  \]
                  Taking limits, we get:
                  \[
                      \limsupn{a_{n}} \leq L + \epsilon, \quad \liminfn{a_{n}} \geq L - \epsilon.
                  \]
                  Since \(\epsilon > 0\) is arbitrary, it follows that \(\limsupn{a_{n}} = \liminfn{a_{n}} = L\). \qedhere}{}{xgray}
    \end{enumerate}
}

\begin{exercise}
    {2.4.10 (Infinite Products)}

    A close relative of infinite series is the infinite product
    \[
        \prod_{n=1}^{\infty} b_n = b_1 b_2 b_3 \cdots,
    \]
    which is understood in terms of its sequence of partial products
    \[
        p_m = \prod_{n=1}^{m} b_n = b_1 b_2 b_3 \cdots b_m.
    \]

    Consider the special class of infinite products of the form
    \[
        \prod_{n=1}^{\infty} (1 + a_n) = (1 + a_1)(1 + a_2)(1 + a_3) \cdots, \quad \text{where} \quad a_n \geq 0.
    \]

    \begin{enumerate}
        \item Find an explicit formula for the sequence of partial products in the case where \( a_n = \frac{1}{n} \) and decide whether the sequence converges. Write out the first few terms in the sequence of partial products in the case where \( a_n = \frac{1}{n^2} \) and make a conjecture about the convergence of this sequence.
        \item Show, in general, the sequence of partial products converges if and only if \(\sum_{n=1}^{\infty} a_n \text{ converges.}\)
              (The inequality \(1 + x \leq 3^{x} \text{ for positive } x\) will be useful in one direction.)
    \end{enumerate}
\end{exercise}

\sol{
    \begin{enumerate}
        \item For \( a_n = \dfrac{1}{n} \):

              The sequence of partial products is:
              \[
                  p_m = \prod_{n=1}^m \left(1 + \dfrac{1}{n}\right) = \prod_{n=1}^m \dfrac{n + 1}{n}
              \]
              This telescopes:
              \[
                  p_m = \dfrac{2}{1} \times \dfrac{3}{2} \times \dfrac{4}{3} \times \cdots \times \dfrac{m + 1}{m} = \dfrac{m + 1}{1} = m + 1
              \]
              Therefore, the sequence diverges as \( m \to \infty \).

              For \( a_n = \dfrac{1}{n^2} \):

              Compute the first few terms:
              \begin{align*}
                  p_1 & = 1 + \dfrac{1}{1^2} = 2                                                                                \\
                  p_2 & = \left(1 + \dfrac{1}{1^2}\right)\left(1 + \dfrac{1}{2^2}\right) = 2 \times \dfrac{5}{4} = \dfrac{5}{2} \\
                  p_3 & = p_2 \times \left(1 + \dfrac{1}{3^2}\right) = \dfrac{5}{2} \times \dfrac{10}{9} = \dfrac{25}{9}        \\
                  p_4 & = p_3 \times \left(1 + \dfrac{1}{4^2}\right) = \dfrac{25}{9} \times \dfrac{17}{16} = \dfrac{425}{144}
              \end{align*}
              The sequence increases slowly, suggesting that the infinite product is monotone increasing, and thus it converges.

        \item We will provide an if and only if proof below.

              \iffpf[We will show that the infinite product \(\prod_{n=1}^\infty (1 + a_n)\) converges if and only if the series \(\sum_{n=1}^\infty a_n\) converges.]{If \(\sum_{n=1}^\infty a_n\) converges, then \(a_n \to 0\). For \(a_n \geq 0\), we have \(\ln(1 + a_n) \leq a_n\). Thus,
                  \[
                      \sum_{n=1}^\infty \ln(1 + a_n) \leq \sum_{n=1}^\infty a_n < \infty
                  \]
                  So the series \(\sum_{n=1}^\infty \ln(1 + a_n)\) converges, which implies that the product \(\prod_{n=1}^\infty (1 + a_n)\) converges.}{Conversely, if \(\prod_{n=1}^\infty (1 + a_n)\) converges, then the partial products are bounded. For \(a_n \geq 0\) and \(1 + x \geq e^{x/2}\) for small \(x\), we have
                  \[
                      \ln(1 + a_n) \geq \dfrac{a_n}{2}
                  \]
                  For sufficiently large \(n\), this gives
                  \[
                      \sum_{n=1}^\infty a_n \leq 2 \sum_{n=1}^\infty \ln(1 + a_n)
                  \]
                  Since \(\sum_{n=1}^\infty \ln(1 + a_n)\) converges, so does \(\sum_{n=1}^\infty a_n\).\qedhere}{}{xgray}
    \end{enumerate}
}


\begin{exercise}
    {2.5.1}Give an example of each of the following, or argue that such a request is impossible.
    \begin{enumerate}
        \item A sequence that has a subsequence that is bounded but contains no subsequence that converges.
        \item A sequence that does not contain 0 or 1 as a term but contains subsequences converging to each of these values.
    \end{enumerate}
\end{exercise}

\sol{
    \begin{enumerate}
        \item \textbf{Impossible.} This violates the \namrefthm{Bolzano-Weierstrass Theorem}. It assures us that every bounded sequence has a convergent subsequence. If a subsequence is bounded, then it must have a convergent subsequence.
        \item Consider the sequence \((\frac{1}{2},\frac{1}{2},\frac{1}{3},\frac{2}{3},\frac{1}{4},\frac{3}{4}, \cdots \frac{1}{n}, \frac{(n-1)}{n})\). From this, you can have a subsequence \((\frac{1}{2}, \frac{1}{3}, \dots, \frac{1}{n})\) which converges to 0, and also a subsequence \((\frac{1}{2}, \frac{2}{3}, \dots \frac{n - 1}{n})\), which converges to 1.
    \end{enumerate}
}

\begin{exercise}
    {2.5.2} Decide whether the following propositions are true or false, providing a short justification for each conclusion.
    \begin{enumerate}
        \item If every proper subsequence of \((x_n)\) converges, then \((x_n)\) converges as well.
        \item If \((x_{n})\) contains a divergent subsequence, then \((x_n)\) diverges.
        \item If \((x_n)\) is bounded and diverges, then there exist two subsequences of \((x_n)\) that converge to different limits.
    \end{enumerate}
\end{exercise}

\sol{
    \begin{enumerate}
        \item \textbf{True.} If every proper subsequence of \((x_n)\) converges, then \((x_n)\) must converge to the same limit. If \((x_n)\) did not converge, there would exist at least one divergent subsequence or two subsequences converging to different limits, contradicting the assumption.

        \item \textbf{True.} If \((x_n)\) contained a divergent subsequence, then \((x_n)\) cannot converge. A convergent sequence has all its subsequences converging to the same limit, so the existence of a divergent subsequence implies that \((x_n)\) diverges (contrapositive).

        \item \textbf{True.} Since \((x_n)\) is bounded and diverges, the \namrefthm{Bolzano-Weierstrass Theorem} guarantees the existence of at least one convergent subsequence. Let this subsequence converge to \(L_1\). Because \((x_n)\) does not converge to \(L_1\), there is an \(\epsilon > 0\) and infinitely many terms of \((x_n)\) such that \(|x_n - L_1| \geq \epsilon\). Extracting a subsequence from these terms, the Bolzano-Weierstrass Theorem ensures a further subsequence converging to a limit \(L_2 \neq L_1\). Thus, \((x_n)\) has two subsequences converging to different limits.
    \end{enumerate}
}


\begin{exercise}
    {2.5.5} Assume \((a_n)\) is a bounded sequence with the property that every convergent subsequence of \((a_n)\) converges to the same limit \(a \in \mathbb{R}\). Show that \((a_n)\) must converge to \(a\).
\end{exercise}

\expf{
Suppose that \((a_n)\) does not converge to \(a \in \R\). By the definition of convergence, this means there is a positive real number \(\epsilon_0\) such that no matter how large we choose \(N \in \N\), there will always exist some \(n > N\) where \(|a_n - a| \geq \epsilon_0\). In a formal way, this shows that \((a_n)\) does not converge to \(a\) within the \(\epsilon_0\)-neighborhood.

We aim to demonstrate that this leads to a contradiction by constructing a subsequence of \((a_n)\) that stays outside this neighborhood. Begin by selecting \(n_1\) such that \(|a_{n_1} - a| \geq \epsilon_0\). Next, since the condition holds for all \(N \in \N\), we can find another index \(n_2 > n_1\) such that \(|a_{n_2} - a| \geq \epsilon_0\). Continuing this process, we generate an increasing sequence of indices \(n_1 < n_2 < n_3 < \dots\) such that for each \(i \in \N\), \(|a_{n_i} - a| \geq \epsilon_0\).

Now consider the subsequence \((a_{n_i})\) we have built. Since \((a_n)\) is bounded by assumption, its subsequence \((a_{n_i})\) is also bounded. By the \namrefthm{Bolzano-Weierstrass Theorem}, every bounded sequence has a convergent subsequence. Let \((a_{n_{i_k}})\) denote a convergent subsequence of \((a_{n_i})\). According to our assumption, any convergent subsequence of \((a_n)\) must converge to \(a\).

However, each term of \((a_{n_{i_k}})\) remains outside the \(\epsilon_0\)-neighborhood of \(a\). Thus, it is impossible for \((a_{n_{i_k}})\) to converge to \(a\). This contradiction implies that our initial assumption—that \((a_n)\) does not converge to \(a\)—is false. Therefore, the sequence \((a_n)\) must converge to \(a\).
}



\begin{exercise}
    {2.5.6} Use a similar strategy to the one in \numrefthm{2.5.5} to show
    \[\lim b^{1/n} \text{ exists for all } b \geq 0\] and find the value of the limit. (The results in \hyperref[exerc:2.3.1]{Exercise 2.3.1} may be assumed.)
\end{exercise}

\expf{
    We will show that \(\lim_{n \to \infty} b^{1/n}\) exists for all \(b \geq 0\) and find its value.
    \begin{itemize}
        \item \textbf{Case 1:} \(b = 0\).

              When \(b = 0\), the sequence becomes \(a_n = 0^{1/n} = 0\) for all \(n\). Thus, \(\lim_{n \to \infty} b^{1/n} = 0\).

        \item \textbf{Case 2:} \(b > 0\).

              Suppose, for contradiction, that \(\lim_{n \to \infty} b^{1/n} \ne 1\). Then there exists \(\epsilon > 0\) and infinitely many \(n\) such that \(|b^{1/n} - 1| \geq \epsilon\). Extract a subsequence \((b^{1/n_k})\) where this inequality holds for all \(k\).

              Since \(b^{1/n} > 0\) and bounded, by the \namrefthm{Bolzano-Weierstrass Theorem}, the subsequence \((b^{1/n_k})\) has a further subsequence that converges to a limit \(L\). According to \hyperref[ex:2.3.1]{Exercise 2.3.1}, any convergent subsequence of \((b^{1/n})\) must have its limit equal to \(\lim_{n \to \infty} b^{1/n}\).

              Consider \(\ln b^{1/n} = \frac{\ln b}{n}\). As \(n \to \infty\), \(\frac{\ln b}{n} \to 0\), so \(\ln b^{1/n} \to 0\), which implies \(b^{1/n} \to e^{0} = 1\).

              This contradicts the assumption that \(|b^{1/n_k} - 1| \geq \epsilon\), so \(\lim_{n \to \infty} b^{1/n} = 1\).

    \end{itemize}

    \textbf{Conclusion:}

    \[
        \lim_{n \to \infty} b^{1/n} =
        \begin{cases}
            0, & \text{if } b = 0,                \\
            1, & \text{if } b > 0. \hfill\qedhere
        \end{cases}
    \]
}