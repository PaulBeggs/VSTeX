\documentclass[12pt]{article}

\usepackage{fix-cm}
\usepackage{tikz}
\usetikzlibrary{shapes,backgrounds,calc,patterns}
\usepackage{amsmath,amsthm} 
\usepackage{mathtools}
\usepackage{amssymb}
\usepackage[framemethod=tikz]{mdframed}
\usepackage{draculatheme}

\usepackage{mathrsfs}
\usepackage{changepage}
\usepackage{multicol}
\usepackage{hyperref}
\usepackage{slashed}
\usepackage{enumerate}
\usepackage{booktabs}
\usepackage{enumitem}
\setlist[enumerate]{leftmargin=1.3cm}
\usepackage{kantlipsum} 
\usepackage{pgfplots}



%%% AESTHETICS %%%
%-%-%-%-%-%-%-%-%-%-%-%-%-%-%-%-%-%-%-%-%-%-%-%-%-%-%-%-%-%-%-%-%-%-%-%-%-%-%


%%% Dimensions and Spacing %%%
\usepackage[margin=1in]{geometry}
% \usepackage{setspace}
% \linespread{1}
\usepackage{listings}
\usepackage{minted}

%%% Define new colors %%%
\usepackage{xcolor}
\definecolor{orangehdx}{rgb}{0.96, 0.51, 0.16}

% Normal colors
\definecolor{xred}{HTML}{BD4242}
\definecolor{xblue}{HTML}{4268BD}
\definecolor{xgreen}{HTML}{52B256}
\definecolor{xpurple}{HTML}{7F52B2}
\definecolor{xorange}{HTML}{FD9337}
\definecolor{xdotted}{HTML}{999999}
\definecolor{xgray}{HTML}{777777}
\definecolor{xcyan}{HTML}{80F5DC}
\definecolor{xpink}{HTML}{F690EA}
\definecolor{xgrayblue}{HTML}{49B095}
\definecolor{xgraycyan}{HTML}{5AA1B9}

% Dark colors
\colorlet{xdarkred}{red!85!black}
\colorlet{xdarkblue}{xblue!85!black}
\colorlet{xdarkgreen}{xgreen!85!black}
\colorlet{xdarkpurple}{xpurple!85!black}
\colorlet{xdarkorange}{xorange!85!black}
\definecolor{xdarkcyan}{HTML}{008B8B}
\colorlet{xdarkgray}{xgray!85!black}

% Very dark colors
\colorlet{xverydarkblue}{xblue!50!black}

% Document-specific colors
\colorlet{normaltextcolor}{black}
\colorlet{figtextcolor}{xblue}

% Enumerated colors
\colorlet{xcol0}{black}
\colorlet{xcol1}{xred}
\colorlet{xcol2}{xblue}
\colorlet{xcol3}{xgreen}
\colorlet{xcol4}{xpurple}
\colorlet{xcol5}{xorange}
\colorlet{xcol6}{xcyan}
\colorlet{xcol7}{xpink!75!black}

% Blue-Purple (should just used colorbrewer...)
\definecolor{xrainbow0}{HTML}{e41a1c}
\definecolor{xrainbow1}{HTML}{a24057}
\definecolor{xrainbow2}{HTML}{606692}
\definecolor{xrainbow3}{HTML}{3a85a8}
\definecolor{xrainbow4}{HTML}{42977e}
\definecolor{xrainbow5}{HTML}{4aaa54}
\definecolor{xrainbow6}{HTML}{629363}
\definecolor{xrainbow7}{HTML}{7e6e85}
\definecolor{xrainbow8}{HTML}{9c509b}
\definecolor{xrainbow9}{HTML}{c4625d}
\definecolor{xrainbow10}{HTML}{eb751f}
\definecolor{xrainbow11}{HTML}{ff9709}

%%% FIGURES %%%
\usepackage{graphicx}  
\graphicspath{ {images/} }  
% \numberwithin{figure}{section}
\usepackage{float}
\usepackage{caption}

%%% Hyperlinks %%%
\usepackage{hyperref}
\definecolor{horange}{HTML}{f58026}
\hypersetup{
	colorlinks=true,
	linkcolor=horange,
	filecolor=horange,      
	urlcolor=horange,
}

\newcommand{\disabs}[1]{\ensuremath{\left|#1\right|}}
\newcommand{\limn}[1]{\ensuremath{\lim_{n \rightarrow \infty}}#1}
\newcommand{\limx}[2]{\lim_{x \rightarrow #1}#2}

\newcommand{\limsupn}[1]{\displaystyle\limsup_{n\rightarrow \infty}#1}
\newcommand{\liminfn}[1]{\displaystyle\liminf_{n\rightarrow \infty}#1}


\renewcommand{\theenumi}{\arabic{enumi}} 
\renewcommand{\labelenumi}{(\theenumi)}

\title{Real Analysis: Take-Home Final Exam}
\author{Paul Beggs}
\date{\today}

%%% Custom Comands %%%
% Natural Numbers 
\newcommand{\N}{\ensuremath{\mathbb{N}}}

% Whole Numbers
\newcommand{\W}{\ensuremath{\mathbb{W}}}

% Integers
\newcommand{\Z}{\ensuremath{\mathbb{Z}}}

% Rational Numbers
\newcommand{\Q}{\ensuremath{\mathbb{Q}}}

% Real Numbers
\newcommand{\R}{\ensuremath{\mathbb{R}}}

% Complex Numbers
\newcommand{\C}{\ensuremath{\mathbb{C}}}

\newcommand{\I}{\ensuremath{\mathbb{I}}}


\begin{document}

\maketitle


``All work on this take-home exam is my own.''\footnote{Except for the calculator section where I needed to use Wolfram Alpha.} \hfill \textbf{Signature:} Paul Beggs

\vspace{0.5cm}

For this exam, you are going to use results from this semester to set up a big idea for next semester. \\

\noindent\textbf{Part 1.} These two problems will give results that are useful in the next part.

\noindent Throughout this test, \(f^{(j)}(x)\) denotes the \(j^{\text{th}}\) derivative of \(f\) at \(x\).

\begin{enumerate}
    \item
          Let \(c_{0}, c_{1}, c_{2},\dots,c_{k}\) be real numbers. Prove there exists a unique polynomial \(p(x)\) of order at most \(k\) such that for each integer \(j\) between 0 and \(k\), \(p^{(j)}(0) = c_{j}\). In other words,
          \[
              p(0) = c_{0}, \qquad p'(0) = c_{1}, \qquad p''(0) = c_{2}, \qquad \ldots, \qquad  p^{(k)}(0) = c_{k}.
          \]

          If \(p(x) = a_{0} + a_{1}x + a_{2}x^{2} + \cdots + a_{k}x^{k}\), give formulas for \(a_{0},\dots,a_{k}\) in terms of \(c_{0},\dots,c_{k}\).
          \vfill
    \item
          Let \(\varphi\) be a function that is differentiable \(k + 1\) times on an interval \([a,b]\). This means \(\varphi',\varphi'',\dots,\varphi^{k + 1}\) all exist on \([a,b]\). Assume that

          \begin{table}[htbp]
              \centering
              \begin{tabular}{@{}c@{\qquad\quad~}c@{\qquad\quad~~}c@{}}
                  \(\begin{aligned}[t]
                        \varphi(a)       & = 0    \\
                        \varphi'(a)      & = 0    \\
                                         & \vdots \\
                        \varphi^{(k)}(a) & = 0
                    \end{aligned}\)
                   & \text{and} &
                  \(\begin{aligned}[t]
                        \varphi(b) = 0.
                    \end{aligned}\)
              \end{tabular}
          \end{table}


          Prove there exists a point \(c \in (a,b)\) such that \(\varphi^{{k + 1}}(c) = 0\).

\end{enumerate}
\newpage
\noindent\textbf{Part 2.} These problems will walk you through an important concept and result in Calculus.\\

Let \(I\) be an interval with zero in its interior and \(f(x)\) be a function that is \(k + 1\) times differentiable on \(I\).

\begin{enumerate}
    \setcounter{enumi}{2}
    \item
          Construct the unique polynomial \(P_{k}(x)\) of order at most \(k\) which satisfies that for all integers \(j\) between \(0 \text{ and } k\), \(P_{k}^{(j)}(0) = f^{(j)}(0)\). \textit{This should be a direct application of Problem (1).}
    \item
          Let \(x\) be a fixed nonzero point in \(I\). Define a new function \(g\) on \(I\) as follows:
          \[
              g(t) = f(t) - P_{k}(t) - \left(\frac{f(x) - P_{k}(x)}{x^{k + 1}}\right)t^{k + 1}.
          \]
          Show that

          \begin{table}[htbp]
              \centering
              \begin{tabular}{@{}c@{\qquad\quad~}c@{\qquad\quad~~}c@{}}
                  \(\begin{aligned}[t]
                        g(0)       & = 0    \\
                        g'(0)      & = 0    \\
                                   & \vdots \\
                        g^{(k)}(0) & = 0
                    \end{aligned}\)
                   & \text{and} &
                  \(\begin{aligned}[t]
                        g(x) = 0.
                    \end{aligned}\)
              \end{tabular}
          \end{table}

          Conclude there exists a point \(c\) between 0 and \(x\) such that \(g^{(k + 1)}(c) = 0\).
    \item
          Use the above problem to prove the existence of a point \(c\) between 0 and \(x\) for which
          \[
              f(x) = P_{k}(x) + \frac{f^{(k + 1)}(c)}{(k + 1)!}x^{k + 1}.
          \]
    \item
          This polynomial \(P_{k}\) is used as an approximation of \(f\). If it is known that \(|f^{(k + 1)}|\) is bounded by some number \(M\) on the interval \(I\), prove the error bound formula
          \[
              |f(x) - P_{k}(x)| \leq \frac{M |x|^{k + 1}}{(k + 1)!}.
          \]
\end{enumerate}

\noindent\textbf{Part 3.} Now you get to enjoy using your result!

\begin{enumerate}
    \setcounter{enumi}{6}
    \item Consider the function \(f(x) = e^{x}\). Give the expression of the polynomial approximation \(P_{k}\) for an arbitrary \(k \in \N\). Use what you know about \(f\) and its derivatives on the interval \([0,1]\) to determine an integer \(k\) for which you can guarantee that \(|f(1) - P_{k}(1)| < 10^{-12}\). Use this (and a calculator) to generate an approximation of \(e\) to 12 decimal places.
\end{enumerate}
\newpage

\begin{center}
    \textbf{Solutions}
\end{center}

\begin{enumerate}
    \item
          To ensure that we have a polynomial with at order of at most \(k\), we first need to observe some behaviors of derivatives. For example, for the polynomial \(x^{j}\):
          \begin{align*}
              (x^{j})'      & = jx^{j - 1}         \\
              (x^{j})''     & = j(j - 1)x^{j - 2}  \\
                            & ~\vdots              \\
              (x^{j})^{(j)} & = j!\cdot x^{0} = j!
          \end{align*}
          Notice the factorial arises from the recursive application of the power rule. Thus, find the \(j^{\text{th}}\) derivative, we combine this with the coefficients \(a_{0},a_{1},\ldots,a_{k}\) to get:
          \begin{align*}
              p(x)       & = a_0 + a_1 x + a_2 x^2 + \cdots + a_k x^k                                               \\
              p'(x)      & = a_{1} + (a_{2} \cdot 2) x + (a_{3} \cdot 3) x^{2} + \cdots + (a_{k} \cdot k) x^{k - 1} \\
                         & ~\vdots                                                                                  \\
              p^{(j)}(x) & = a_{j} \cdot j! + (\text{terms involving higher values of } x)
          \end{align*}
          To derive formulas for \(a_{1}, a_{2}, \dots, a_{k}\), we need to set \(x = 0\). Thus,
          \[
              p^{(j)}(0) = a_{j} \cdot j!
          \]
          Substitute \(c_{j}\) in for \(p^{(j)}\) and solve for \(a_{j}\):
          \[
              c_j = a_j \cdot j!
          \]

          Therefore:
          \[
              a_j = \frac{c_j}{j!}
          \]
          for each \(j = 0,1,2,\dots,k\). This gives the coefficients of
          \begin{align*}
              a_{0} & = c_{0}             \\
              a_{1} & = \frac{c_{1}}{1!}  \\
              a_{2} & = \frac{c_{2}}{2!}  \\
                    & \vdots              \\
              a_{k} & = \frac{c_{k}}{k!}.
          \end{align*}
          Thus, the polynomial \(p(x)\) exists with unique coefficients defined by \(a_{j} = \dfrac{c_{j}}{j!}\) because each \(a_{j}\) is uniquely determined by \(c_{j}\).
    \item
          Since \(\varphi(a) = 0\) and \(\varphi(b) = 0\), then
          \[
              \varphi(a) = \varphi(b).
          \]

          Thus, by Rolle's Theorem, there exists point \(c_{1} \in (a,b)\) such that
          \[
              \varphi'(c_{1}) = 0.
          \]

          Using this \(c_{1}\) as a point in a new closed interval, we have on \([a,c_{1}]\):
          \[
              \varphi'(a) = \varphi'(c) = 0.
          \]

          Again, by Rolle's Theorem, we have a point \(c_{2} \in (a,c_{1})\) such that
          \[
              \varphi''(c_{2}) = 0.
          \]

          Repeat this up to \(\varphi^{(k)}\) times to create a sequence of points \(c_{1},c_{2},\dots,c_{k}\) such that
          \[
              \varphi^{(j)}(c_{j}) = 0 \text{ for } j = 1,2,\dots,k.
          \]

          After \(k\) applications of Rolle's Theorem, we have
          \[
              \varphi^{(k)}(c_{k}) = 0.
          \]

          Now, consider \(\varphi^{(k)}(x)\) on \((a,b)\) such that
          \[
              \varphi^{(k)}(a) = \varphi^{(k)}(c_{k}) = 0.
          \]

          By Rolle's Theorem, there exists a point \(c \in (c_{k},b)\) such that
          \[
              \varphi^{(k + 1)}(c) = 0.
          \]
    \item
          From Problem (1), we know that the polynomial \(P_{k}(x)\) can be written as:
          \[
              P_{k}(x) = \sum_{j=0}^{k} \frac{f^{(j)}(0)}{j!} x^j
          \]
          This polynomial satisfies \(P_{k}^{(j)}(0) = f^{(j)}(0)\) for all integers \(j\) between 0 and \(k\).
    \item
          First, we show that \(g(0) = 0\):
          \[
              g(0) = f(0) - P_{k}(0) - \left(\frac{f(x) - P_{k}(x)}{x^{k + 1}}\right)0^{k + 1} = f(0) - P_{k}(0) = 0
          \]
          since \(P_{k}(0) = f(0)\).

          Next, we show that \(g'(0) = 0\):
          \[
              g'(t) = f'(t) - P_{k}'(t) - \left(\frac{f(x) - P_{k}(x)}{x^{k + 1}}\right)(k + 1)t^k
          \]
          \[
              g'(0) = f'(0) - P_{k}'(0) = 0
          \]
          since \(P_{k}'(0) = f'(0)\).

          Similarly, we can show that \(g^{(j)}(0) = 0\) for \(j = 0, 1, \ldots, k\).

          Finally, we show that \(g(x) = 0\):
          \[
              g(x) = f(x) - P_{k}(x) - \left(\frac{f(x) - P_{k}(x)}{x^{k + 1}}\right)x^{k + 1} = f(x) - P_{k}(x) - (f(x) - P_{k}(x)) = 0
          \]

          Then, from the hard work we did in (2), we conclude from Rolle's Theorem that there exists a point \(c \in (0, x)\) such that \(g^{(k + 1)}(c) = 0\).
    \item
          From the previous problem, we have \(g^{(k + 1)}(c) = 0\) for some \(c \in (0, x)\). Therefore,
          \[
              g^{(k + 1)}(t) = f^{(k + 1)}(t) - \left(\frac{f(x) - P_{k}(x)}{x^{k + 1}}\right)(k + 1)!
          \]
          Evaluating at \(t = c\), we get:
          \[
              f^{(k + 1)}(c) - \left(\frac{f(x) - P_{k}(x)}{x^{k + 1}}\right)(k + 1)! = 0
          \]
          Solving for \(f(x)\), we obtain:
          \[
              f(x) = P_{k}(x) + \frac{f^{(k + 1)}(c)}{(k + 1)!}x^{k + 1}
          \]
    \item
          From the previous problem, we have:
          \[
              f(x) = P_{k}(x) + \frac{f^{(k + 1)}(c)}{(k + 1)!}x^{k + 1}
          \]
          Therefore, the error term is:
          \[
              f(x) - P_{k}(x) = \frac{f^{(k + 1)}(c)}{(k + 1)!}x^{k + 1}
          \]
          Taking the absolute value and using the bound \(|f^{(k + 1)}(c)| \leq M\), we get:
          \[
              |f(x) - P_{k}(x)| \leq \frac{M |x|^{k + 1}}{(k + 1)!}
          \]
    \item
          For the function \(f(x) = e^x\), all derivatives are \(f^{(j)}(x) = e^x\). At \(x = 0\), we have \(f^{(j)}(0) = 1\) for all \(j\). Therefore, the polynomial approximation \(P_{k}(x)\) is:
          \[
              P_{k}(x) = \sum_{j=0}^{k} \frac{x^j}{j!}
          \]

          To find \(k\) such that \(|f(1) - P_{k}(1)| < 10^{-12}\), we use the error bound formula:
          \[
              |e - P_{k}(1)| \leq \frac{e}{(k + 1)!}
          \]
          We need to find the smallest \(k\) such that:
          \[
              \frac{e}{(k + 1)!} < 10^{-12}
          \]

          Using a calculator (\href{https://www.wolframalpha.com/input?i=solve+for+k%3A+e%2F%28k+%2B+1%29%21+%3C+10%5E-12}{Wolfram Alpha}), we find that \(k = 20\) satisfies this inequality. Therefore, the polynomial approximation \(P_{20}(x)\) is:
          \[
              P_{20}(x) = \sum_{j=0}^{20} \frac{x^j}{j!}
          \]

          Evaluating at \(x = 1\), we get:
          \[
              P_{20}(1) = \sum_{j=0}^{20} \frac{1}{j!} \approx 2.718281828459
          \]

          Thus, an approximation of \(e\) to 12 decimal places is \(2.718281828459\).
\end{enumerate}


\end{document}