\begin{exercise}
    {1.2.13}For this exercise, assume \hyperref[exerc:1.2.5 (De Morgan's Laws)]{Exercise 1.2.5} has been successfully completed.
    \begin{enumerate}[label=(\alph*)]
        \item Show how induction can be used to conclude that
              \[
                  (A_1 \cup A_2 \cup \cdots \cup A_n)^c = A_1^c \cap A_2^c \cap \cdots \cap A_n^c
              \]
              for any finite \(n \in \mathbb{N}\).

        \item It is tempting to appeal to induction to conclude
              \[
                  \left( \bigcup_{i=1}^{\infty} A_i \right)^c = \ \bigcap_{i=1}^{\infty} A_i^c,
              \]
              but induction does not apply here. Induction is used to prove that a particular statement holds for every value of \(n \in \mathbb{N}\), but this does not imply the validity of the infinite case. To illustrate this point, find an example of a collection of sets \(B_1, B_2, B_3, \dots\) where
              \[
                  \bigcap_{i=1}^{n} B_i \neq \emptyset \quad \text{is true for every } n \in \mathbb{N},
              \]
              but
              \[
                  \bigcap_{i=1}^{\infty} B_i = \emptyset
              \]
              fails.

        \item Nevertheless, the infinite version of De Morgan’s Law stated in (b) is a valid statement. Provide a proof that does not use induction.
    \end{enumerate}
\end{exercise}
% \newpage
\begin{customframedproof}[linecolor=horange!75]

    \begin{proof}
        In this proof, we plan to prove (c). Thus, we need to show that:
        \[
            \left( \bigcup_{i=1}^{\infty} A_i \right)^c \subseteq \ \bigcap_{i=1}^{\infty} A_i^c
        \]
        and
        \[
            \left( \bigcup_{i=1}^{\infty} A_i \right)^c \supseteq \ \bigcap_{i=1}^{\infty} A_i^c.
        \]
        \begin{proofpart}[horange]{\((\subseteq)\)}
            Let \(x \in \left( \bigcup_{i=1}^{\infty} A_i \right)^c\). This means \(x\) is in the union set of \(A_i\) for all \(i \in \N\). Then, because we are taking the complement of \(\left( \bigcup_{i=1}^{\infty} A_i \right)\), that means \(x \notin A_i\) for all \(i \in \mathbb{N}\). Hence, \(x\) is in the complement of each \(A_i\). Thus, we can use the definition of intersection to assert \(x \in \bigcap_{i=1}^{\infty} A_i^c\).

            Therefore, we have shown:
            \[
                \left( \bigcup_{i=1}^{\infty} A_i \right)^c \subseteq \ \bigcap_{i=1}^{\infty} A_i^c.
            \]
        \end{proofpart}
        \vspace{2mm}
        \begin{proofpart}[horange]{\((\supseteq)\)}
            Similar to before, let \(x \in \bigcap_{i=1}^{\infty} A_i^c\). Because \(x \in A^c_i\) for all \(i \in \N\) we know \(x \notin A_i\). Hence, \(x \notin \left(\bigcup_{i=1}^{\infty} A_i\right)\), which means \(x \in \left( \bigcup_{i=1}^{\infty} A_i \right)^c\).

            Therefore, we have shown:
            \[
                \left( \bigcup_{i=1}^{\infty} A_i \right)^c \supseteq \ \bigcap_{i=1}^{\infty} A_i^c.
            \]
        \end{proofpart}

        By showing both inclusions, we see that:
        \[
            \left( \bigcup_{i=1}^{\infty} A_i \right)^c = \ \bigcap_{i=1}^{\infty} A_i^c. \qedhere
        \]
    \end{proof}
\end{customframedproof}