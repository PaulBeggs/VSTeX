\setcounter{section}{1}

\newcommand{\xtoc}{\lim_{x \to c}}

\section{Derivates and the Intermediate Value Property}

\begin{definition}
    Let \(g \colon A \to \R\) be a function defined on an interval \(A\). Given \(c \in A\), we define the \textit{derivative} of \(g\) at \(c\) to be
    \[
        g'(c) = \lim_{x \to c} \frac{g(x) - g(c)}{x - c}
    \]
    provided this limit exists. In this case, we say \(g\) is \textit{differentiable} at \(c\). If \(g'\) exists for all points \(c \in A\), then we say \(g\) is \textit{differentiable on} \(A\).
\end{definition}

\begin{example}
    {Differentiation 1}Let \(g(x) = x^{2}\). Use \defref{5.2.1} to find \(g'(c)\).
\end{example}

\lesol{
    \begin{align*}
        g'(c) & = \lim_{x \to c} \frac{g(x) - g(c)}{x - c}    \\
              & = \lim_{x \to c} \frac{x^{2} - c^{2}}{x - c}  \\
              & = \lim_{x \to c} \frac{(x - c)(x + c)}{x - c} \\
              & = \lim_{x \to c} x + c                        \\
              & = 2c
    \end{align*}
}

\begin{example}
    {Differentiation 2}Let \(g(x) = x^{3}\). Use \defref{5.2.1} to find \(g'(c)\).
\end{example}

\lesol{
    \begin{align*}
        g'(c) & = \lim_{x \to c} \frac{g(x) - g(c)}{x - c}                 \\
              & = \lim_{x \to c} \frac{x^{3} - c^{3}}{x - c}               \\
              & = \lim_{x \to c} \frac{(x - c)(x^{2} + xc + c^{2})}{x - c} \\
              & = \lim_{x \to c} x^{2} + xc + c^{2}                        \\
              & = 3c^{2}
    \end{align*}
}

\begin{example}
    {Differentiation 3}Let \(g(x) = x^{4}\). Use \defref{5.2.1} to find \(g'(c)\).
\end{example}

\lesol{
    \begin{align*}
        g'(c) & = \lim_{x \to c} \frac{g(x) - g(c)}{x - c}                              \\
              & = \lim_{x \to c} \frac{x^{4} - c^{4}}{x - c}                            \\
              & = \lim_{x \to c} \frac{(x - c)(x^{3} + x^{2}c + xc^{2} + c^{3})}{x - c} \\
              & = \lim_{x \to c} x^{3} + x^{2}c + xc^{2} + c^{3}                        \\
              & = 4c^{3}
    \end{align*}
}

\begin{ntheorem}
    {Power Rule}For any \(n \in \N\), if \(f(x) = x^{n}\), then \(f'(c) = nc^{n - 1}\).
\end{ntheorem}

\begin{example}
    {Differentiation 4}Let \(f(x) = |x|\). Use \defref{5.2.1} to find \(f'(0)\).
\end{example}

\lesol{
    

    \begin{align*}
        \lim_{x \to 0} \frac{|x| - |0|}{x - 0} & = \lim_{x \to 0} \frac{|x|}{x}
    \end{align*}
    When we view this from the left and right definitions of the limit, we see that the limit does not exist:
    \[
        \lim_{x \to 0^{+}} \frac{|x|}{x} = 1 \quad \text{and} \quad \lim_{x \to 0^{-}} \frac{|x|}{x} = -1.
    \]
    Hence, because these limits are not equal, the limit does not exist, and \(f(x) = |x|\) is not differentiable at 0.
}

\begin{example}
    {Differentiation 5}Let \(f(x) = \begin{cases}
        1 & \text{if } x > 0    \\
        0 & \text{if } x \leq 0 \\
    \end{cases}\)
\end{example}

\lesol{
    Is this differentiable at 0?
    \[
        \lim_{x \to 0^{+}} \frac{f(x)}{x} = \infty \quad \text{and} \quad \lim_{x \to 0^{-}} \frac{f(x)}{x} = 0.
    \]
    Hence, because these limits are not equal, the limit does not exist, and \(f(x)\) is not differentiable at 0.
}

\begin{ntheorem}
    {Algebraic Differentiation Rules} Let \(f\) and \(g\) be differentiable at \(c\) with \(k \in \R\). Then the following functions are differentiable at \(c\):
    \begin{enumerate}
        \item \((f + g)'(c) = f'(c) + g'(c)\)

              \(\Rightarrow \lim_{x \to c} \frac{f(x) + g(x) - f(c) - g(c)}{x - c}\)
        \item \((kf)'(c) = kf'(c)\)
        \item \((fg)'(c) = f'(c)g(c) + f(c)g'(c)\)
        \item \((f/g)'(c) = \frac{f'(c)g(c) - f(c)g'(c)}{g^{2}(c)}\)
    \end{enumerate}
\end{ntheorem}

\tpf{
    For (3), we have
    \begin{align*}
        \lim_{x \to c} \frac{f(x)g(x) - f(c)g(c)}{x - c} & = \lim_{x \to c} \frac{f(x)g(x) - f(x)g(c) + f(x)g(c) - f(c)g(c)}{x - c}                                       \\
                                                         & = \lim_{x \to c} \frac{f(x)(g(x) - g(c)) + g(c)(f(x) - f(c))}{x - c}                                           \\
                                                         & = \lim_{x \to c} f(x) \lim_{x \to c} \frac{g(x) - g(c)}{x - c} + g(c) \lim_{x \to c} \frac{f(x) - f(c)}{x - c} \\
                                                         & = f(c)g'(c) + g(c)f'(c).
    \end{align*}
    For (4), we have
    \begin{align*}
        \lim_{x \to c} \frac{\frac{f(x)}{g(x)} - \frac{f(c)}{g(c)}}{x - c} & = \lim_{x \to c} \frac{f(x)g(c) - f(c)g(x)}{g(x)g(c)(x - c)}                                                      \\
                                                                           & = \lim_{x \to c} \frac{f(x)g(c) - f(c)g(c) + f(c)g(c) - f(c)g(x)}{g(x)g(c)(x - c)}                                \\
                                                                           & = \lim_{x \to c} \frac{f(x) - f(c)}{g(x)g(c)(x - c)} + \lim_{x \to c} \frac{f(c)g(c) - f(c)g(x)}{g(x)g(c)(x - c)} \\
                                                                           & = \frac{f'(c)g(c) - f(c)g'(c)}{g^{2}(c)}. \qedhere
    \end{align*}
}

\begin{ntheorem}
    {Chain Rule} Let \(f\) and \(g\) be differentiable at \(c\). Then the composition \(g \circ f\) is differentiable at \(c\) and \((g \circ f)'(c) = g'(f(c))f'(c)\).
\end{ntheorem}

\tpf{
    \begin{align*}
        \lim_{x \to c} \frac{g(f(x)) - g(f(c))}{x - c} & = \lim_{x \to c} \frac{g(f(x)) - g(f(c))}{f(x) - f(c)} \cdot \frac{f(x) - f(c)}{x - c}          \\
                                                       & = \lim_{x \to c} \frac{g(f(x)) - g(f(c))}{f(x) - f(c)} \lim_{x \to c} \frac{f(x) - f(c)}{x - c} \\
                                                       & = g'(f(c))f'(c). \qedhere
    \end{align*}
}

\begin{ntheorem}
    {Interior Extremum Theorem} If \(f\) is differentiable at \(c\), and \(c\) is a local maximum of \(f\), then \(f'(c) = 0\).
\end{ntheorem}

\tpf{
    We know that \(f'(c) = \lim_{x \to c} \frac{f(x) - f(c)}{x - c}\). Then,
    \[
        f'(x) = \lim_{x \to c^{-}} \frac{f(x) - f(c)}{x - c} \geq 0,
    \]
    and
    \[
        f'(x) = \lim_{x \to c^{+}} \frac{f(x) - f(c)}{x - c} \leq 0,
    \]
    so, \(f'(c) = 0\).
}

\section{The Mean Value Theorems}

\begin{ntheorem}
    {Rolle's Theorem} Let \(f\) be continuous on \([a, b]\) and differentiable on \((a, b)\). If \(f(a) = f(b)\), then there exists a \(c \in (a, b)\) such that \(f'(c) = 0\).
\end{ntheorem}

\tpf{
    Since \(f\) is continuous on \([a,b]\), the \namrefthm{Extreme Value Theorem} says \(f\) achieves a maximum and minimum in \([a,b]\).
    \begin{itemize}
        \item If either maximum or minimum is in the interior, then the \namrefthm{Interior Extremum Theorem} says that \(f'(c) = 0\).
        \item If the maximum or minimum is at the endpoints, then \(f\) is constant and \(f'(c) = 0\) for all \([a,b]\). \qedhere
    \end{itemize}
}

\begin{ntheorem}
    {Mean Value Theorem} Let \(f\) be continuous on \([a, b]\) and differentiable on \((a, b)\). Then there exists a \(c \in (a, b)\) such that
    \[
        f'(c) = \frac{f(b) - f(a)}{b - a}.
    \]
    This is saying that there exists a point where the slope of the tangent line is equal to the slope of the secant line. In other words, there is a point where \(f'(c)\) happens.
\end{ntheorem}

\tpf{
Let \(d(x) = f(x) - [f(a) + \frac{f(b)-f(a)}{b-a} (x - a)]\). Notice \(d(x)\) is continuous on \([a,b]\) and differentiable on \((a,b)\). Thus,
\[
    d(a) = f(a) - (f(a) + 0) = 0 \qquad \text{and} \qquad d(b) = f(b) - (f(a) + f(b) - f(a)) = 0.
\]
So, \namrefthm{Rolle's Theorem} applies to \(d(x)\) and there exists a \(c \in (a,b)\) where \(d'(c) = 0\). Notice
\[
    d'(x) = f'(x) - \frac{f(b) - f(a)}{b - a},
\]
so
\begin{align*}
    0     & = f'(c) - \frac{f(b) - f(a)}{b - a}   \\
    f'(c) & = \frac{f(b) - f(a)}{b - a}. \qedhere
\end{align*}
}

\begin{ntheorem}
    {Generalized Mean Value} Let \(f\) and \(g\) be continuous on \([a,b]\) and differentiable on \((a,b)\). There exists \(c \in (a,b)\) such that
    \[
        (f(b)-f(a))g(x) - (g(b) - g(a))f(x).
    \]
\end{ntheorem}

\tpf{
    Let \(d(x) = (f(b)-f(a))g(x) - (g(b) - g(a))f(x)\). Notice \(d(x)\) is continuous on \([a,b]\) and differentiable on \((a,b)\). Thus,
    \begin{align*}
        d(a) & = (f(b)-f(a))g(a) - (g(b) - g(a))f(a) \\
        d(b) & = (f(b)-f(a))g(b) - (g(b) - g(a))f(b)
    \end{align*}
    We can show \(d(a) = d(b)\), so \namrefthm{Rolle's Theorem} applies to \(d(x)\) and there exists a \(c \in (a,b)\) where \(d'(c) = 0\). Therefore,
    \[
        0 = d'(c) = (f(b)-f(a))g'(c) - (g(b) - g(a))f'(c) \qedhere
    \]
}

\begin{ntheorem}
    {L'Hôspital's Rule} (\(\frac{0}{0}\) case) Let \(f\) and \(g\) be continuous on an interval around \(a \in \R\). Assume \(f\) and \(g\) are differentiable on this interval (except possibly at \(a\)). Assume \(f(a) = g(a) = 0\). Also, \(g'(x) \ne 0\). If all conditions are met, then the following conditional is valid:
    \[
        \text{If }\quad \lim_{x\to a}\frac{f'(x)}{g'(x)} \quad \text{ exists, then } \quad \lim_{x \to a} \frac{f(x)}{g(x)} = \lim_{x \to a} \frac{f'(x)}{g'(x)}.
    \]
\end{ntheorem}

\tpf{
    Choose a number \(b\) that is close to \(a\). The \namrefthm{Generalized Mean Value} Theorem says there exists \(c\) between \(a\) and \(b\) where:
    \begin{align*}
        (f(b) - f(a))g'(c) & = (g(b) - g(a))f'(c)   \\
        f(b)g'(c)          & = g(b)f'(c)            \\
        \frac{f(b)}{g(b)}  & = \frac{f'(c)}{g'(c)},
    \end{align*}
    because \(f(a) = g(a) = 0\).

    Now, suppose \(\lim_{x \to a}\frac{f'(x)}{g'(x)} = L\) exists. Consider any sequence \((b_n) \to a\).
    By the Generalized Mean Value Theorem, for each \(b_n\) there exists \(c_n\) between \(a\) and \(b_n\) such that:
    \[\frac{f(b_n)}{g(b_n)} = \frac{f'(c_n)}{g'(c_n)}\]

    Since \(a < c_n < b_n\) and \(b_n \to a\), we have \(c_n \to a\) as well.
    Therefore:
    \[\lim_{n \to \infty} \frac{f(b_n)}{g(b_n)} = \lim_{n \to \infty} \frac{f'(c_n)}{g'(c_n)} = \lim_{x \to a} \frac{f'(x)}{g'(x)} = L.\]

    Since this holds for any sequence \((b_n) \to a\), we conclude:
    \[\lim_{x \to a} \frac{f(x)}{g(x)} = L. \qedhere\]
}

\begin{ntheorem}
    {Darbaux's Theorem} If \(f\) is differentiable on \([a,b]\) and \(f'(a) < \alpha < f'(b)\), then there exists \(c \in (a,b)\) such that \(f'(c) = \alpha\).
\end{ntheorem}