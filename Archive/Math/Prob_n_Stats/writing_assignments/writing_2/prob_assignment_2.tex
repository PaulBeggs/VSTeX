%-%-%-%-%-%-%-%-%-%-%-%-%-%-%-%-%-%-%-%-%-%-%-%-%-%-%-%-%-%-%-%-%-%-%-%-%-%-%
%%% MAIN DOCUMENT %%%

%-%-%-%-%-%-%-%-%-%-%-%-%-%-%-%-%-%-%-%-%-%-%-%-%-%-%-%-%-%-%-%-%-%-%-%-%-%-%

\documentclass[12pt]{article}
\usepackage{xcolor}
\usepackage{minted}

% \usepackage[T1]{fontenc}

% \usepackage{draculatheme}
% \newcommand{\documentTheme}{draculafg}
% \newcommand{\documentLogo}{../../images/small logo_white.png}
% \newcommand{\documentBigLogo}{../../images/logo_white_text.png}
% \usemintedstyle{dracula}

\newcommand{\documentBigLogo}{../../images/Hendrix Logo.png}
\newcommand{\documentLogo}{../../images/small logo.png}
\newcommand{\documentTheme}{black}

%%% AESTHETICS %%%
%-%-%-%-%-%-%-%-%-%-%-%-%-%-%-%-%-%-%-%-%-%-%-%-%-%-%-%-%-%-%-%-%-%-%-%-%-%-%


%%% Dimensions and Spacing %%%
\usepackage[margin=1in]{geometry}
\setlength{\parindent}{0pt}
\usepackage{setspace}
\usepackage{mathtools}
\usepackage{esint}
\usepackage{adjustbox}
\linespread{1}
\usepackage{listings}
\usepackage{tikz}
\usetikzlibrary{shapes,backgrounds,calc,patterns,positioning}
\usepgflibrary{shadings}
\usepackage{pgfkeys}
\usepackage{pdftexcmds}
\usepackage{shellesc}
\usepackage{algorithm, algorithmic, xspace}
\usepackage{caption}
\usepackage{subcaption}
%%% Define new colors %%%
\definecolor{orangehdx}{rgb}{0.96, 0.51, 0.16}

% Normal colors
\definecolor{xred}{HTML}{BD4242}
\definecolor{xblue}{HTML}{4268BD}
\definecolor{xgreen}{HTML}{52B256}
\definecolor{xpurple}{HTML}{7F52B2}
\definecolor{xorange}{HTML}{FD9337}
\definecolor{xdotted}{HTML}{999999}
\definecolor{xgray}{HTML}{777777}
\definecolor{xcyan}{HTML}{80F5DC}
\definecolor{xpink}{HTML}{F690EA}
\definecolor{xgrayblue}{HTML}{49B095}
\definecolor{xgraycyan}{HTML}{5AA1B9}

% Dark colors
\colorlet{xdarkred}{red!85!black}
\colorlet{xdarkblue}{xblue!85!black}
\colorlet{xdarkgreen}{xgreen!85!black}
\colorlet{xdarkpurple}{xpurple!85!black}
\colorlet{xdarkorange}{xorange!85!black}
\definecolor{xdarkcyan}{HTML}{008B8B}
\colorlet{xdarkgray}{xgray!85!black}

% Very dark colors
\colorlet{xverydarkblue}{xblue!50!black}

% Document-specific colors
\colorlet{normaltextcolor}{black}
\colorlet{figtextcolor}{xblue}

% Enumerated colors
\colorlet{xcol0}{black}
\colorlet{xcol1}{xred}
\colorlet{xcol2}{xblue}
\colorlet{xcol3}{xgreen}
\colorlet{xcol4}{xpurple}
\colorlet{xcol5}{xorange}
\colorlet{xcol6}{xcyan}
\colorlet{xcol7}{xpink!75!black}

% Blue-Purple (should just used colorbrewer...)
\definecolor{xrainbow0}{HTML}{e41a1c}
\definecolor{xrainbow1}{HTML}{a24057}
\definecolor{xrainbow2}{HTML}{606692}
\definecolor{xrainbow3}{HTML}{3a85a8}
\definecolor{xrainbow4}{HTML}{42977e}
\definecolor{xrainbow5}{HTML}{4aaa54}
\definecolor{xrainbow6}{HTML}{629363}
\definecolor{xrainbow7}{HTML}{7e6e85}
\definecolor{xrainbow8}{HTML}{9c509b}
\definecolor{xrainbow9}{HTML}{c4625d}
\definecolor{xrainbow10}{HTML}{eb751f}
\definecolor{xrainbow11}{HTML}{ff9709}

%------- %
% XHFILL %
%------- %



%%% Chapter Headings %%%

\newcommand{\gradientrule}{
    \begin{tikzpicture}
        \shade[left color=orangehdx, right color=black, middle color=gray] (0,0) rectangle (\linewidth,0.4pt);
    \end{tikzpicture}
}

\usepackage[Glenn]{fncychap}
% \ChTitleVar{\bfseries\scshape\color{\documentTheme}} % Needed for Dracula theme
% \ChNumVar{\large\selectfont\color{\documentTheme}} % Needed for Dracula theme
% \ChNameVar{\large\color{\documentTheme}} % Needed for Dracula theme
\usepackage{xpatch}

% \xpatchcmd{\DOCH}
%   {\mghrulefill}{\gradientrule\mghrulefill}
%   {}{\PatchFailed}
% \xpatchcmd{\DOTI}
%   {\mghrulefill}{\gradientrule\mghrulefill}
%   {}{\PatchFailed}
% \xpatchcmd{\DOTIS}
%   {\mghrulefill}{\gradientrule\mghrulefill}
%   {}{\PatchFailed}

\xpatchcmd\DOCH
{\mghrulefill}{\color{orangehdx}\mghrulefill}
{}{\PatchFailed}
\xpatchcmd\DOTI
{\mghrulefill}{\color{orangehdx}\mghrulefill}
{}{\PatchFailed}
\xpatchcmd\DOTIS
{\mghrulefill}{\color{orangehdx}\mghrulefill}
{}{\PatchFailed}


% \usepackage[Bjornstrup]{fncychap}

% \newcommand{\gradient}[1]{
% \begin{tikzpicture}
%     \node (rect) at (0,0) [fill=blue,,path fading=East,minimum width=\linewidth,minimum height=2.5cm] {};
%     \node(title)[above left = 10pt and 10pt of rect.south east, anchor=south east, font=\CTV] {\textcolor{horange}{#1}};
%     \ifnum \thechapter>0\node[left = 10pt of rect.north east,  anchor=center, font=\CNoV] {\textcolor{horange}{\thechapter}};\fi%
% \end{tikzpicture}%
% \vskip 40pt
% }


% \renewcommand{\DOCH}{}
% \renewcommand{\DOTI}[1]{\gradient{#1}}
% \renewcommand{\DOTIS}[1]{\gradient{#1}}

%% Change Chapter Heading Placement %%
\usepackage{etoolbox}
% \makeatletter
% \patchcmd{\@makechapterhead}{\vspace*{50\p@}}{\vspace*{-20\p@}}{}{}
% \patchcmd{\@makeschapterhead}{\vspace*{50\p@}}{\vspace*{-20\p@}}{}{}
% \patchcmd{\DOTI}{\vskip 80\p@}{\vskip 40\p@}{}{}
% \patchcmd{\DOTIS}{\vskip 40\p@}{\vskip 0\p@}{}{}
% \makeatother



% \newcommand*\chapterlabel{}\pmod
% \titleformat{\chapter}
%   {\gdef\chapterlabel{}
%    \normalfont\sffamily\Huge\bfseries\scshape}
%   {\gdef\chapterlabel{\thechapter\ }}{0pt}
%   {\begin{tikzpicture}[remember picture,overlay]
%     \node[yshift=-3cm] at (current page.north west)
%       {\begin{tikzpicture}[remember picture, overlay]
%         \draw[fill=LightSkyBlue] (0,0) rectangle
%           (\paperwidth,3cm);
%         \node[anchor=east,xshift=.9\paperwidth,rectangle,
%               sharp corners=downhill=20pt,inner sep=11pt,
%               fill=MidnightBlue]
%               {\color{white}\chapterlabel#1};
%        \end{tikzpicture}
%       };
%    \end{tikzpicture}
%   }
% \titlespacing*{\chapter}{0pt}{50pt}{-60pt}

% \usepackage[Conny]{fncychap}
% \usepackage[Rejne]{fncychap}
% \ChNameVar{\bfseries}  % Makes the chapter "Chapter #" bold
% \ChNumVar{\bfseries}   % Makes the chapter number bold
% \ChTitleVar{\bfseries} % Makes the chapter title bold
% % Define a custom color, for example:
% \definecolor{mycolor}{RGB}{0,128,255}

% % Redefine the chapter style in Rejne to change the line colors
% \makeatletter
% \ChRuleWidth{2pt}   % Change the thickness of the lines
% \renewcommand{\DOCH}{%
%   \vspace*{-50\p@}% Moves the chapter title up/down if needed
%   {\color{mycolor} \hrule \@chapapp{} \space \thechapter \hrule}% Customizes the chapter header with color
% }
% \renewcommand{\DOTI}[1]{%
%   \vskip 20\p@ % Adjusts the space above the title
%   \bfseries #1\par % Embolden the title
%   \vskip 20\p@ % Adjusts the space below the title
% }
% \makeatother

% %% Change Chapter Heading Placement %%
% \usepackage{etoolbox}
% \makeatletter
% \patchcmd{\@makechapterhead}{\vspace*{50\p@}}{\vspace*{-20\p@}}{}{}
% \patchcmd{\@makeschapterhead}{\vspace*{50\p@}}{\vspace*{-20\p@}}{}{}
% \patchcmd{\DOTI}{\vskip 80\p@}{\vskip 40\p@}{}{}
% \patchcmd{\DOTIS}{\vskip 40\p@}{\vskip 0\p@}{}{}
% \makeatother

% \renewcommand{\thesection}{\thechapter.\arabic{section}} %% Chapter.Section Numbering


%%% FIGURES %%%
\usepackage{graphicx}  
% \numberwithin{figure}{section}
\usepackage{float}
\usepackage{caption}

%%% Hyperlinks %%%
\usepackage{hyperref}
\definecolor{horange}{HTML}{f58026}
\hypersetup{
	colorlinks=true,
	linkcolor=horange,
	filecolor=horange,      
	urlcolor=horange,
}


%% Headers and Footers %%
\usepackage{fancyhdr} % This should be set AFTER setting up the page geometry
\pagestyle{fancy} % options: empty , plain , fancy
\fancyhead[R]{\assignmentname}
\fancyhead[L]{Hendrix College}
\fancyhead[C]{\includegraphics[height=.50cm]{\documentLogo}} % Use for Dracula
\usepackage{xpatch}
\xpretocmd\headrule{\color{orangehdx}}{}{\PatchFailed}
\setlength{\footskip}{0.5in}
\setlength{\headheight}{18.5764pt}


%%%%% Colored Boxes %%%%%
\usepackage{tcolorbox}
\tcbuselibrary{skins}
\tcbuselibrary{theorems}
% \tcbuselibrary{minted}
\newcounter{BoxCounter}
\usepackage{dingbat}

%-%-%-%-%-%-%-%-%-%-%-%-%-%-%-%-%-%-%-%-%-%-%-%-%-%-%-%-%-%-%-%-%-%-%-%-%-%-%

%% MATH PACKAGES, ENVIRONMENTS, COMMANDS %%
%-%-%-%-%-%-%-%-%-%-%-%-%-%-%-%-%-%-%-%-%-%-%-%-%-%-%-%-%-%-%-%-%-%-%-%-%-%-%
%You'll need your own packages, theorem types, and commands.

\usepackage{fix-cm}
\usepackage{amsmath,amsthm} 
\usepackage{array, makecell}
\usepackage{colortbl}
\newcommand{\thickvrule}{\vrule width 1pt}

\newcommand{\thickhrule}{\Xhline{1pt}}

\usepackage{mathtools}
\usepackage{amssymb}
\usepackage[framemethod=tikz]{mdframed}
\usepackage{mathrsfs}
\usepackage{changepage}
\usepackage{multicol}
\usepackage{slashed}
\usepackage{enumerate}
\usepackage{braket}
\usepackage{booktabs}
\usepackage{enumitem}
\usepackage{kantlipsum}  %This package lets us generate random text for example purposes.
\usepackage{pgfplots}


%%% Custom Commands %%%
% Natural Numbers 
\newcommand{\N}{\mathbb{N}}

% Whole Numbers
\newcommand{\W}{\mathbb{W}}

% Integers
\newcommand{\Z}{\mathbb{Z}}

% Rational Numbers
\newcommand{\Q}{\mathbb{Q}}

% Real Numbers
\newcommand{\R}{\mathbb{R}}

% Complex Numbers
\newcommand{\C}{\mathbb{C}}

\newcommand{\I}{\mathbb{I}}

\newcommand{\pfs}{\noindent\makebox[\linewidth]{\rule{\textwidth}{0.4pt}}\vspace{0.5cm}}

\newcommand{\mysqrt}[1]{%
	\mathpalette\foo{#1}%
}
\newcommand{\dmysqrt}[1]{%
	\mathpalette\foodisplay{#1}%
}

% !TeX spellcheck = off
\newcommand{\foo}[2]{%
	% #1: math style, #2: content
	\sbox0{$#1\sqrt{#2}$}% Measure the size of the standard sqrt in the current style
	\begin{tikzpicture}[baseline=(sqrt.base)]
		\node[inner sep=0, outer sep=0] (sqrt) {$#1\sqrt{#2}$}; % Use the current math style
		\draw([yshift=-0.045em]sqrt.north east) -- ++(0,-0.5ex); % Draw the tick
	\end{tikzpicture}%
}
% !TeX spellcheck = off
\newcommand{\foodisplay}[2]{%
	% #1: math style, #2: content
	\sbox0{$#1\sqrt{#2}$}% Measure the size of the standard sqrt in the current style
	\begin{tikzpicture}[baseline=(sqrt.base)]
		\node[inner sep=0, outer sep=0] (sqrt) {$\displaystyle\sqrt{#2}$}; % Force displaystyle
		\draw[line width=0.4pt] ([yshift=-0.044em]sqrt.north east) -- ++(0,-0.5ex); % Draw the tick
	\end{tikzpicture}%
}

% \renewcommand{\theenumi}{\arabic{enumi}}
% \renewcommand{\labelenumi}{\textbf{\theenumi}.}

% \newcommand{}{\marginnote{\includegraphics[width=2em]{caution.png}}}

\newmdenv[
	topline=false,
	bottomline=true,
	rightline=false,
	leftline=true,
	linewidth=1.5pt,
	linecolor=black, % default color, will be overridden in custom commands
	% backgroundcolor=draculafg, % Needed for Dracula theme
	fontcolor=\documentTheme, % Needed for Dracula theme
	innertopmargin=0pt,
	innerbottommargin=5pt,
	innerrightmargin=10pt,
	innerleftmargin=10pt,
	leftmargin=0pt,
	rightmargin=0pt,
	skipabove=\topsep,
	skipbelow=\topsep,
]{customframedproof}

\newenvironment{proofpart}[2][black]{
    \begin{mdframed}[
        topline=false,
        bottomline=false,
        rightline=false,
        leftline=true,
        linewidth=1pt,
        linecolor=#1!40, % Custom color
        % innertopmargin=10pt,
        % innerbottommargin=10pt,
        innerleftmargin=10pt,
        innerrightmargin=10pt,
        leftmargin=0pt,
        rightmargin=0pt,
        % skipabove=\topsep,
        % skipbelow=\topsep%
    ]
    \noindent
    \begin{minipage}[t]{0.08\textwidth}%
        \textbf{#2}%
    \end{minipage}%
    \begin{minipage}[t]{0.90\textwidth}%
        \begin{adjustwidth}{0pt}{0pt}%
}{
    \end{adjustwidth}
    \end{minipage}
    \end{mdframed}
}

% Define a new environment 'proofscratch' for Proof and Scratch Paper
\newenvironment{proofscratch}[2]{%
    \begin{mdframed}[
        topline=false,
        bottomline=false,
        rightline=false,
        leftline=false,
        backgroundcolor=draculabg, % Background color for Dracula theme
        fontcolor=\documentTheme,       % Font color for Dracula theme
        innerleftmargin=-6pt,
        innertopmargin=0pt,
        leftmargin=0pt,
        rightmargin=0pt,
        skipabove=\topsep,
        skipbelow=\topsep,
    ]
    % % Begin TikZ picture for the vertical dotted line
    % \begin{tikzpicture}[overlay, remember picture]
    %     % Draw a vertical dotted line at approximately the center of the mdframed width
    %     \draw[dotted, thick] ([xshift=0.005\linewidth]current bounding box.north west) -- ([xshift=0.005\linewidth]current bounding box.south west);
    % \end{tikzpicture}%
    % % Create the left minipage for Proof
    \begin{minipage}[t]{0.47\textwidth}%
        \noindent\textit{Proof.} #1 \hfill \(\qed\)
    \end{minipage}%
    \hfill
    % Create the right minipage for Scratch Paper
    \begin{minipage}[t]{0.47\textwidth}%
        \textit{Scratch Paper.} #2
    \end{minipage}%
}{
    \end{mdframed}
}

\newenvironment{tipbox}
	{%
		\par\noindent%
		% Top dashed line (using \textwidth)
		\tikz[baseline]{\draw[dash pattern=on 3pt off 2pt, line width=0.5pt, color=draculapurple] (0,0) -- (\textwidth,0);}%
		\par\vspace{0.65em}%
		\noindent\textbf{\textcolor{draculapurple}{TIP:}}\quad%
	}
	{%
		\par%
		% Bottom dashed line
		\noindent%
		\tikz[baseline]{\draw[dash pattern=on 3pt off 2pt, line width=0.5pt, color=draculapurple] (0,0) -- (\textwidth,0);}%
		\par\vspace{0.35em}
	}

\newenvironment{notebox}
	{%
		\par\noindent%
		% Top dashed line (using \textwidth)
		\tikz[baseline]{\draw[dash pattern=on 3pt off 2pt, line width=0.5pt, color=draculagreen] (0,0) -- (\textwidth,0);}%
		\par\vspace{0.65em}%
		\noindent\textbf{\textcolor{draculagreen}{NOTE:}}\quad%
	}
	{%
		\par%
		% Bottom dashed line
		\noindent%
		\tikz[baseline]{\draw[dash pattern=on 3pt off 2pt, line width=0.5pt, color=draculagreen] (0,0) -- (\textwidth,0);}%
		\par\vspace{0.35em}
	}

\newenvironment{warningbox}
	{%
		\par\noindent%
		% Top dashed line (using \textwidth)
		\tikz[baseline]{\draw[dash pattern=on 3pt off 2pt, line width=0.5pt, color=draculared] (0,0) -- (\textwidth,0);}%
		\par\vspace{0.65em}%
		\noindent\textbf{\textcolor{draculared}{WARNING:}}\quad%
	}
	{%
		\par%
		% Bottom dashed line
		\noindent%
		\tikz[baseline]{\draw[dash pattern=on 3pt off 2pt, line width=0.5pt, color=draculared] (0,0) -- (\textwidth,0);}%
		\par\vspace{0.35em}
	}

\newenvironment{solution}{\textit{Solution.}}

\newcommand{\sol}[1]{
    \begin{customframedproof}[linecolor=orangehdx!75,]
        \begin{solution}
        #1
        \end{solution}
    \end{customframedproof}
}

\newcommand{\pf}[1]{
    \begin{customframedproof}[linecolor=orangehdx!75]
        \begin{proof}
        #1
        \end{proof}
    \end{customframedproof}
}

\def \proofDistance {10pt}

\newcommand{\disabs}[1]{\ensuremath{\left|#1\right|}}
\newcommand{\limn}{\lim_{n \rightarrow \infty}}
\newcommand{\limx}[2]{\lim_{x \rightarrow #1}#2}

\newcommand{\limc}[1]{\lim_{x \rightarrow c}#1}

\newcommand{\ninn}{n \in \N}

\newcommand{\mb}[1]{\mathbf{#1}}

\newcommand{\limsupn}[1]{\limsup_{n\rightarrow \infty}#1}
\newcommand{\liminfn}[1]{\liminf_{n\rightarrow \infty}#1}


\newcommand{\proj}{\text{proj}}

\newcommand{\p}{\partial}

\newcommand{\dydx}{\frac{dy}{dx}}
\newcommand{\dxdy}{\frac{dx}{dy}}
\newcommand{\dydt}{\frac{dy}{dt}}
\newcommand{\dxdt}{\frac{dx}{dt}}
\newcommand{\dzdt}{\frac{dz}{dt}}

\newcommand{\barNotationT}[1]{\bigg|_{t = #1}}

\newcommand{\cyanit}[1]{\textit{\textcolor{cyan}{#1}}}

\newcommand{\brackett}[1]{\left\langle #1 \right\rangle}

\newcommand{\norm}[1]{\left\lVert \mathbf{#1}\right\rVert}

\newcommand{\imb}{\mb{i}}
\newcommand{\jmb}{\mb{j}}
\newcommand{\kmb}{\mb{k}}
\newcommand{\rmb}{\mb{r}}
\newcommand{\umb}{\mb{u}}

\newcommand{\vecfuc}[2]{\mb{#1}(#2)}
\newcommand{\dvecfuc}[2]{\mb{#1}'(#2)}
\newcommand{\normdvecfuc}[2]{\|\mb{#1}'(#2)\|}

\newcommand{\squigglyline}{%
    \noindent
    \tikz[baseline=-0.5ex]{
        \draw[decorate, decoration={snake, amplitude=0.5mm, segment length=3mm}] 
        (0,0) -- (\dimexpr\linewidth\relax,0);
    }%
}


% end of preamble
%-%-%-%-%-%-%-%-%-%-%-%-%-%-%-%-%-%-%-%-%-%-%-%-%-%-%-%-%-%-%-%-%-%-%-%-%-%-%

\begin{document}

\numberwithin{BoxCounter}{section}


%-%-%-%-%-%-%-%-%-%-%-%-%-%-%-%-%-%-%-%-%-%-%-%-%-%-%-%-%-%-%-%-%-%-%-%-%-%-%
%%% COVER PAGE %%%
%-%-%-%-%-%-%-%-%-%-%-%-%-%-%-%-%-%-%-%-%-%-%-%-%-%-%-%-%-%-%-%-%-%-%-%-%-%-%

\newcommand{\assignmentname}{Bayesian Inference}

% Do not use all caps.
\newcommand{\cussubtitle}{Writing Assignment 2}
% Date format should be like: "January 1, 2001"
\newcommand{\finaldate}{October 10, 2025}
% Be sure to include degree recognition (e.g., B.S., M.S., Ph.D)
\newcommand{\professor}{Prof.\ Lars Seme, M.S.}

%-%-%-%-%-%-%-%-%-%-%-%-%-%-%-%-%-%-%-%-%-%-%-%-%-%-%-%-%-%-%-%-%-%-%-%-%-%-%

\begin{titlepage}
    \begin{center}

        \vspace*{-2cm}
        \includegraphics[width=0.8\textwidth]{\documentBigLogo}\\
        \vfill

        % Horizontal line above the title in 'horange' color
        \textcolor{horange}{\rule{\textwidth}{1.0pt}}

        \vspace{2em}

        {\huge \textbf{\assignmentname}}

        \vspace{1em} % Space between the title and the bottom line

        \textcolor{horange}{\rule{\textwidth}{1.0pt}}

        \vspace*{1\baselineskip}

        {\LARGE \textbf{\cussubtitle}}

        \begin{large}
            \vspace*{5\baselineskip}

            \vspace*{1\baselineskip}

            \emph{Author} \\[1ex]
            %Submitted by \\[\baselineskip]
            {\Large Paul Beggs \\ \par} % Editor list
            {\href{mailto:BeggsPA@Hendrix.edu}{{BeggsPA@Hendrix.edu}}}\\ % Editor affiliation

            \vspace*{1\baselineskip}

            \textit{Instructor} \\[1ex] % Tagline(s) or further description
            \professor

            \vspace*{1\baselineskip}

            \textit{Due}\\[1ex]
            {\scshape  \finaldate} \\[0.3\baselineskip] % Year published

            \thispagestyle{empty}

        \end{large}
    \end{center}
\end{titlepage}

\section{Introduction}

In probability and statistics, an eponymous theorem developed by Thomas Bayes states
\[
    P(A \mid B) = \frac{P(B \mid A) \cdot P(A)}{P(B)},
\]
read as ``the probability of event $A$ happening given $B$ is equal to the probability of $B$ given $A$, multiplied by the probability of $A$, and divided by the probability of $B$.'' This theorem is the foundation of Bayesian inference, as it allows us to update our beliefs when presented with new evidence. \\

Thus, the goal of this paper is to explore exactly that. Given an initial set of beliefs (referred to as a ``prior''), how does new information, specifically in the form of successful trials, change those beliefs? This updated belief (called a ``posterior'') is important because it provides a nuanced method for learning from new evidence, weighing prior assumptions, and updating those beliefs to be as accurate as possible.

\section{Probabilistic Distributions}

In the following sections, we will examine three different prior distributions and analyze the effect of new information, specifically binomial trials (success or failure), on the resulting posterior distribution. We will use probabilistic distributions that demonstrate the probability density over a given range of values, indicating which value is most probable. We will denote our parameter of interest as $p$, which represents the probability of success. The ``probability density'' is the relative likelihood of $p$ occurring.\\ 

The three prior distributions to be explored are:
\begin{itemize}
    \item \textbf{Uniform:} Each value of $p$ has an equal probability of occurring.
    \item \textbf{Exponential:} Lower $p$ values have a higher probability of occurring.
    \item \textbf{Bell-curve (Normal or Gaussian):} The probability of $p$ occurring is low at the lower and higher ends of the distribution, but high toward the center.
\end{itemize}
For each distribution, we will first detail the initial statistics of the prior, such as the mean, median, mode, and variance, along with the middle 50\% and 90\% credible intervals. Subsequently, we will apply varying amounts of new information from binomial data to the prior to observe how the distribution evolves. We will demonstrate that if we continuously add more trials while maintaining a consistent success rate (specifically 80\%), the value $p = 0.80$ will become increasingly likely as the number of trials grows. Furthermore, we will show that the initial state of the prior is critical to the speed at which the posterior changes. \\

In each of the graphs presented, $p$ ranges from $\{0.01, 0.02,\ldots, 0.99\}$, and the probability density is scaled from $0$ to $0.25$ to ensure a uniform comparison. This normalization is applied so that all graphs can be viewed with identical axes.

\section{Uniform Prior}

Initially, we start off with statistics that are self-evident from \hyperref[fig:uni-prior]{Figure 1}. Our mean and median are both 0.5 because the mean value is just the sum of the lowest \(p\) (0.01), and the highest \(p\) (0.99) divided by 2, and since each value of \(p\) has an equal probability, the halfway point is 0.5. The mode is value that appears the most, but since every \(p\) is of equal likelihood, it can be any \(p\). The variance is 0.08. Finally, we also see that the middle \(50\%\) and \(90\%\) intervals are \([0.25, 0.75]\) and \([0.05, 0.95]\), respectively.

\begin{figure}[h!]
    \centering
    \label{fig:uni-prior}
    \includegraphics[width=0.65\linewidth]{graphs/Uniform/prior.png}
    \caption{Prior Uniform Distribution}
\end{figure}

As we start to add more data to the distribution, our statistics start to quickly shift toward 0.80, as is evident in \hyperref[fig:uni-5]{Figure 2}. The new mean, median, mode, and variance for 5-trials is 0.71, 0.73, 0.8, and 0.03 and for 50-trials is 0.79, 0.79, 0.8, and 0 respectively. The middle 50\% and 90\% intervals for 5-trials is \([0.61, 0.83]\) and \([0.41,0.93]\), and 50-trials is \([0.41, 0.93]\). \\


\begin{figure}[h!]
    \centering
    \label{fig:uni-5}
    \begin{subfigure}{0.475\linewidth}
        \includegraphics[width=\linewidth]{graphs/Uniform/5-trials.png}
        \caption{5 Trials}
    \end{subfigure}
    \begin{subfigure}{0.475\linewidth}
        \includegraphics[width=\linewidth]{graphs/Uniform/50-trials.png}
        \caption{50 Trials}
    \end{subfigure}
    \caption{Posterior Distributions After 5 and 50 Binomial Trials With Uniform Prior}
\end{figure}

\newpage

As we get to 500 trials, all statistics (except the variance) are practically equal to 0.80, with the mean, median, mode, and variance being 0.80, 0.79, 0.80, and 0, respectively. In addition, the middle 50\% and 90\% are also almost entirely centered around .8 with intervals \([0.78, 0.81]\) and \([0.76, 0.82]\). Thus, when we look at \hyperref[fig:uni-500]{Figure 3}, it's clear that the new data has a huge impact upon the posterior, and the prior distribution has almost none. 

\begin{figure}[h!]
    \centering
    \label{fig:uni-500}
    \includegraphics[width=0.65\linewidth]{graphs/Uniform/500-trials.png}
    \caption{Posterior Distribution After 500 Binomial Trials With Uniform Prior}
\end{figure}

\section{Exponential Prior}

The exponential prior distribution looks similar to the uniform distribution, except it has a tail that starts high for low \(p\)'s and drops toward 0, as can be seen in \hyperref[fig:exp-prior]{Figure 4}. The initial mean, median, mode, and variance values for the prior distribution are 0.37, 0.32, 0.01, and 0.07. The intervals are \([0.14, 0.57]\) for 50\%, and \([0.03, 0.88]\) for 90\%. These values make sense as our distribution is skewed toward lower \(p\) values. \\

As we can see in \hyperref[fig:exp-5]{Figure 5}, when we add data from the binomial experiments, our posterior will gravitate toward 0.8, but will move slower than the uniform distribution. Thus, for 5-trials, our statistics for the mean, median, mode, and variance are 0.67, 0.68, 0.74, and 0.03, with 50\% and 90\% intervals \([0.55, 0.80]\) and \([0.36, 0.92]\). We are slowly shifting toward the success rate of 0.8. This becomes more dramatic as we add more trials, such as the 50-trial and 500-trial distributions in \hyperref[fig:exp-50]{Figure 6}. The mean, median, mode, and variance for each of these posteriors are 0.78, 0.78, 0.79, and 0, and 0.8, 0.79, 0.8, and 0, respectively. Their 50\% and 90\% intervals are \([0.74, 0.82]\) and \([0.68, 0.87]\), and \(0.78, 0.81\) and \([0.76, 0.82]\), respectively.


\begin{figure}[h!]
    \centering
    \label{fig:exp-prior}
    \includegraphics[width=0.65\linewidth]{graphs/Exponential/prior.png}
    \caption{Exponential Prior Distribution}
\end{figure}

\begin{figure}[h!]
    \centering
    \label{fig:exp-5}
    \includegraphics[width=0.65\linewidth]{graphs/Exponential/5-trials.png}
    \caption{Posterior Distribution After 5 Binomial Trials With Exponential Prior}
\end{figure}

\begin{figure}[h!]
    \centering
    \label{fig:exp-50}
    \begin{subfigure}{.475\linewidth}
        \includegraphics[width=\linewidth]{graphs/Exponential/50-trials.png}
        \caption{50 Trials}
    \end{subfigure}
    \begin{subfigure}{0.475\linewidth}
        \includegraphics[width=\linewidth]{graphs/Exponential/500-trials.png}
        \caption{500 Trials}
    \end{subfigure}
    \caption{Posterior Distributions After 50 and 500 Binomial Trials With Exponential Prior}
\end{figure}

\newpage

\section{Bell-Shaped Prior}

Unlike the previous two distributions, we start off with a moderately high probability of \(p = 0.5\). This initial spike has a strong effect upon new posteriors, especially for low trials, as can be seen in \hyperref[fig:bell-prior]{Figure 7}. For instance, compare the initial statistics of this prior distribution with the 5-trial posterior distribution:
\begin{itemize}
    \item \textbf{Prior:} The mean, median, and mode are all equal to \(0.5\) with low variance 0.01. The 50\% and 90\% intervals are \([0.43, 0.56]\) and \([0.33, 0.66]\); and
    \item \textbf{5-Trial Posterior:} The mean, median, and mode all slightly shift away from \(0.5\) toward 0.55, with the variance equal to 0. The 50\% and 90\% intervals barely shift too: \([0.48, 0.61]\) and \([0.40, 0.70]\). 
\end{itemize}
This slight change is very important to note. It shows that when we have a strong prior belief, when we add new evidence, we change slightly modify our beliefs, but not by much. 

\begin{figure}[h!]
    \centering
    \label{fig:bell-prior}
    \begin{subfigure}{.475\linewidth}
        \includegraphics[width=\linewidth]{graphs/Bell-shaped/prior.png}
        \caption{Prior Distribution}
    \end{subfigure}
    \begin{subfigure}{0.475\linewidth}
        \includegraphics[width=\linewidth]{graphs/Bell-shaped/5-trials.png}
        \caption{5 Trials}
    \end{subfigure}
    \caption{Bell-Shaped Prior and 5-Trial Posterior Distributions}
\end{figure}

However, when we increase the amount of trials, even though we have a strong prior, the posterior drifts toward 0.8 because of the magnitude of new information, as can be seen in \hyperref[fig:bell-5]{Figure 8}. The statistics of these posterior distributions reflect (albeit, to a lesser degree) what we have seen in the previous ones. For the 50-trial distribution, the mean, median, mode, and variance are 0.71, 0.71, 0.71, and 0, with 50\% and 90\% intervals \([0.67, 0.74]\) and \([0.61, 0.80]\). The 500-trial statistics are even closer to 0.8 with the mean and mode being equal to 0.79, and the median being 0.78 with a variance of 0. The 50\% and 90\% intervals are \([0.77, 0.80]\) and \([0.75, 0.81]\). \\

Thus, we can see that with a bell-shaped, high strong initial prior, when we add more information to the distribution, our posterior slowly shifts toward the stronger, new probability. This is in stark contrast to the previous distributions that almost immediately migrated to \(p = 0.8\). 

\begin{figure}[h!]
    \centering
    \label{fig:bell-5}
    \begin{subfigure}{.475\linewidth}
        \includegraphics[width=\linewidth]{graphs/Bell-shaped/50-trials.png}
        \caption{50 Trials}
    \end{subfigure}
    \begin{subfigure}{0.475\linewidth}
        \includegraphics[width=\linewidth]{graphs/Bell-shaped/500-trials.png}
        \caption{500 Trials}
    \end{subfigure}
    \caption{Posterior Distributions After 50 and 500 Binomial Trials With Bell-Shaped Prior}
\end{figure}

\newpage

\section{Conclusion}

We have reviewed three different distributions, all with different results. We saw that when we start with a uniform prior distribution and add new information, our posterior strongly drifts toward new information with almost no attention being paid to the prior. With the exponential prior, we saw a similar result to the uniform distribution, but with a smaller influence. Finally, we saw that the bell-shaped prior had the strongest opposition to new information because it had an initial high belief centered at \(p= 0.5\). Only when we added 500 trials did we truly see a strong shift toward \(p = 0.8\). \\

In our statistics, we saw common trends. For each distribution, the mean, median, and mode were all close, but when we added new information, they all became almost equal to each other. This resulted in an ever shrinking variance, at some points even being approximately equal to 0. Thus, through this exploration, we have found that when the expected outcome for \(p\) is approximately equal, the prior has little effect on the posterior; conversely, when there is an initial high value for some \(p\), then the prior has a strong effect on the posterior.  

\end{document}