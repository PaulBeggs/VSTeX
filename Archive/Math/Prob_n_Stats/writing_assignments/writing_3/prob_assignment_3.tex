%-%-%-%-%-%-%-%-%-%-%-%-%-%-%-%-%-%-%-%-%-%-%-%-%-%-%-%-%-%-%-%-%-%-%-%-%-%-%
%%% MAIN DOCUMENT %%%

%-%-%-%-%-%-%-%-%-%-%-%-%-%-%-%-%-%-%-%-%-%-%-%-%-%-%-%-%-%-%-%-%-%-%-%-%-%-%

\documentclass[12pt]{article}
\usepackage{xcolor}
\usepackage{minted}

% \usepackage[T1]{fontenc}

% \usepackage{draculatheme}
% \newcommand{\documentTheme}{draculafg}
% \newcommand{\documentLogo}{../../images/small logo_white.png}
% \newcommand{\documentBigLogo}{../../images/logo_white_text.png}
% \usemintedstyle{dracula}

\newcommand{\documentBigLogo}{../../images/Hendrix Logo.png}
\newcommand{\documentLogo}{../../images/small logo.png}
\newcommand{\documentTheme}{black}

%%% AESTHETICS %%%
%-%-%-%-%-%-%-%-%-%-%-%-%-%-%-%-%-%-%-%-%-%-%-%-%-%-%-%-%-%-%-%-%-%-%-%-%-%-%


%%% Dimensions and Spacing %%%
\usepackage[margin=1in]{geometry}
\setlength{\parindent}{0pt}
\usepackage{setspace}
\usepackage{mathtools}
\usepackage{esint}
\usepackage{adjustbox}
\linespread{1}
\usepackage{listings}
\usepackage{tikz}
\usetikzlibrary{shapes,backgrounds,calc,patterns,positioning}
\usepgflibrary{shadings}
\usepackage{lastpage}
\usepackage{pgfkeys}
\usepackage{pdftexcmds}
\usepackage{shellesc}
\usepackage{algorithm, algorithmic, xspace}
\usepackage{caption}
\usepackage{subcaption}

%%% Define new colors %%%
\definecolor{orangehdx}{rgb}{0.96, 0.51, 0.16}

% Normal colors
\definecolor{xred}{HTML}{BD4242}
\definecolor{xblue}{HTML}{4268BD}
\definecolor{xgreen}{HTML}{52B256}
\definecolor{xpurple}{HTML}{7F52B2}
\definecolor{xorange}{HTML}{FD9337}
\definecolor{xdotted}{HTML}{999999}
\definecolor{xgray}{HTML}{777777}
\definecolor{xcyan}{HTML}{80F5DC}
\definecolor{xpink}{HTML}{F690EA}
\definecolor{xgrayblue}{HTML}{49B095}
\definecolor{xgraycyan}{HTML}{5AA1B9}

% Dark colors
\colorlet{xdarkred}{red!85!black}
\colorlet{xdarkblue}{xblue!85!black}
\colorlet{xdarkgreen}{xgreen!85!black}
\colorlet{xdarkpurple}{xpurple!85!black}
\colorlet{xdarkorange}{xorange!85!black}
\definecolor{xdarkcyan}{HTML}{008B8B}
\colorlet{xdarkgray}{xgray!85!black}

% Very dark colors
\colorlet{xverydarkblue}{xblue!50!black}

% Document-specific colors
\colorlet{normaltextcolor}{black}
\colorlet{figtextcolor}{xblue}

% Enumerated colors
\colorlet{xcol0}{black}
\colorlet{xcol1}{xred}
\colorlet{xcol2}{xblue}
\colorlet{xcol3}{xgreen}
\colorlet{xcol4}{xpurple}
\colorlet{xcol5}{xorange}
\colorlet{xcol6}{xcyan}
\colorlet{xcol7}{xpink!75!black}

% Blue-Purple (should just used colorbrewer...)
\definecolor{xrainbow0}{HTML}{e41a1c}
\definecolor{xrainbow1}{HTML}{a24057}
\definecolor{xrainbow2}{HTML}{606692}
\definecolor{xrainbow3}{HTML}{3a85a8}
\definecolor{xrainbow4}{HTML}{42977e}
\definecolor{xrainbow5}{HTML}{4aaa54}
\definecolor{xrainbow6}{HTML}{629363}
\definecolor{xrainbow7}{HTML}{7e6e85}
\definecolor{xrainbow8}{HTML}{9c509b}
\definecolor{xrainbow9}{HTML}{c4625d}
\definecolor{xrainbow10}{HTML}{eb751f}
\definecolor{xrainbow11}{HTML}{ff9709}

%------- %
% XHFILL %
%------- %



%%% Chapter Headings %%%

\newcommand{\gradientrule}{
    \begin{tikzpicture}
        \shade[left color=orangehdx, right color=black, middle color=gray] (0,0) rectangle (\linewidth,0.4pt);
    \end{tikzpicture}
}

\usepackage[Glenn]{fncychap}
% \ChTitleVar{\bfseries\scshape\color{\documentTheme}} % Needed for Dracula theme
% \ChNumVar{\large\selectfont\color{\documentTheme}} % Needed for Dracula theme
% \ChNameVar{\large\color{\documentTheme}} % Needed for Dracula theme
\usepackage{xpatch}

% \xpatchcmd{\DOCH}
%   {\mghrulefill}{\gradientrule\mghrulefill}
%   {}{\PatchFailed}
% \xpatchcmd{\DOTI}
%   {\mghrulefill}{\gradientrule\mghrulefill}
%   {}{\PatchFailed}
% \xpatchcmd{\DOTIS}
%   {\mghrulefill}{\gradientrule\mghrulefill}
%   {}{\PatchFailed}

\xpatchcmd\DOCH
{\mghrulefill}{\color{orangehdx}\mghrulefill}
{}{\PatchFailed}
\xpatchcmd\DOTI
{\mghrulefill}{\color{orangehdx}\mghrulefill}
{}{\PatchFailed}
\xpatchcmd\DOTIS
{\mghrulefill}{\color{orangehdx}\mghrulefill}
{}{\PatchFailed}


% \usepackage[Bjornstrup]{fncychap}

% \newcommand{\gradient}[1]{
% \begin{tikzpicture}
%     \node (rect) at (0,0) [fill=blue,,path fading=East,minimum width=\linewidth,minimum height=2.5cm] {};
%     \node(title)[above left = 10pt and 10pt of rect.south east, anchor=south east, font=\CTV] {\textcolor{horange}{#1}};
%     \ifnum \thechapter>0\node[left = 10pt of rect.north east,  anchor=center, font=\CNoV] {\textcolor{horange}{\thechapter}};\fi%
% \end{tikzpicture}%
% \vskip 40pt
% }


% \renewcommand{\DOCH}{}
% \renewcommand{\DOTI}[1]{\gradient{#1}}
% \renewcommand{\DOTIS}[1]{\gradient{#1}}

%% Change Chapter Heading Placement %%
\usepackage{etoolbox}
% \makeatletter
% \patchcmd{\@makechapterhead}{\vspace*{50\p@}}{\vspace*{-20\p@}}{}{}
% \patchcmd{\@makeschapterhead}{\vspace*{50\p@}}{\vspace*{-20\p@}}{}{}
% \patchcmd{\DOTI}{\vskip 80\p@}{\vskip 40\p@}{}{}
% \patchcmd{\DOTIS}{\vskip 40\p@}{\vskip 0\p@}{}{}
% \makeatother



% \newcommand*\chapterlabel{}\pmod
% \titleformat{\chapter}
%   {\gdef\chapterlabel{}
%    \normalfont\sffamily\Huge\bfseries\scshape}
%   {\gdef\chapterlabel{\thechapter\ }}{0pt}
%   {\begin{tikzpicture}[remember picture,overlay]
%     \node[yshift=-3cm] at (current page.north west)
%       {\begin{tikzpicture}[remember picture, overlay]
%         \draw[fill=LightSkyBlue] (0,0) rectangle
%           (\paperwidth,3cm);
%         \node[anchor=east,xshift=.9\paperwidth,rectangle,
%               sharp corners=downhill=20pt,inner sep=11pt,
%               fill=MidnightBlue]
%               {\color{white}\chapterlabel#1};
%        \end{tikzpicture}
%       };
%    \end{tikzpicture}
%   }
% \titlespacing*{\chapter}{0pt}{50pt}{-60pt}

% \usepackage[Conny]{fncychap}
% \usepackage[Rejne]{fncychap}
% \ChNameVar{\bfseries}  % Makes the chapter "Chapter #" bold
% \ChNumVar{\bfseries}   % Makes the chapter number bold
% \ChTitleVar{\bfseries} % Makes the chapter title bold
% % Define a custom color, for example:
% \definecolor{mycolor}{RGB}{0,128,255}

% % Redefine the chapter style in Rejne to change the line colors
% \makeatletter
% \ChRuleWidth{2pt}   % Change the thickness of the lines
% \renewcommand{\DOCH}{%
%   \vspace*{-50\p@}% Moves the chapter title up/down if needed
%   {\color{mycolor} \hrule \@chapapp{} \space \thechapter \hrule}% Customizes the chapter header with color
% }
% \renewcommand{\DOTI}[1]{%
%   \vskip 20\p@ % Adjusts the space above the title
%   \bfseries #1\par % Embolden the title
%   \vskip 20\p@ % Adjusts the space below the title
% }
% \makeatother

% %% Change Chapter Heading Placement %%
% \usepackage{etoolbox}
% \makeatletter
% \patchcmd{\@makechapterhead}{\vspace*{50\p@}}{\vspace*{-20\p@}}{}{}
% \patchcmd{\@makeschapterhead}{\vspace*{50\p@}}{\vspace*{-20\p@}}{}{}
% \patchcmd{\DOTI}{\vskip 80\p@}{\vskip 40\p@}{}{}
% \patchcmd{\DOTIS}{\vskip 40\p@}{\vskip 0\p@}{}{}
% \makeatother

% \renewcommand{\thesection}{\thechapter.\arabic{section}} %% Chapter.Section Numbering


%%% FIGURES %%%
\usepackage{graphicx}  
% \numberwithin{figure}{section}
\usepackage{float}
\usepackage{caption}

%%% Hyperlinks %%%
\usepackage{hyperref}
\definecolor{horange}{HTML}{f58026}
\hypersetup{
	colorlinks=true,
	linkcolor=horange,
	filecolor=horange,
	urlcolor=horange,
}


%% Headers and Footers %%
\usepackage{fancyhdr} % This should be set AFTER setting up the page geometry
% \pagestyle{standard} % options: empty , plain , fancy
% \fancypagestyle{standard}{
%     \fancyhead[R]{\assignmentname}
%     \fancyhead[L]{Hendrix College}
%     \fancyhead[C]{\includegraphics[height=.50cm]{\documentLogo}} % Use for Dracula
%     \fancyfoot[R]{\rule{\textwidth}{0.5pt} \\ Page \thepage\ of~\pageref*{LastPage}} % Right-aligned 
% }
\usepackage{fancyhdr} % This should be set AFTER setting up the page geometry
\pagestyle{fancy} % options: empty , plain , fancy
\fancyhead[R]{\assignmentname}
\fancyhead[L]{Hendrix College}
\fancyhead[C]{\includegraphics[height=.50cm]{\documentLogo}} % Use for Dracula
\usepackage{xpatch}
\xpretocmd\headrule{\color{orangehdx}}{}{\PatchFailed}
\setlength{\footskip}{0.5in}
\setlength{\headheight}{18.5764pt}


%%%%% Colored Boxes %%%%%
\usepackage{tcolorbox}
\tcbuselibrary{skins}
\tcbuselibrary{theorems}
% \tcbuselibrary{minted}
\newcounter{BoxCounter}
\usepackage{dingbat}

%-%-%-%-%-%-%-%-%-%-%-%-%-%-%-%-%-%-%-%-%-%-%-%-%-%-%-%-%-%-%-%-%-%-%-%-%-%-%

%% MATH PACKAGES, ENVIRONMENTS, COMMANDS %%
%-%-%-%-%-%-%-%-%-%-%-%-%-%-%-%-%-%-%-%-%-%-%-%-%-%-%-%-%-%-%-%-%-%-%-%-%-%-%
%You'll need your own packages, theorem types, and commands.

\usepackage{fix-cm}
\usepackage{amsmath,amsthm} 
\usepackage{array, makecell}
\usepackage{colortbl}
\newcommand{\thickvrule}{\vrule width 1pt}

\newcommand{\thickhrule}{\Xhline{1pt}}

\usepackage{mathtools}
\usepackage{tabularx}
\usepackage{amssymb}
\usepackage[framemethod=tikz]{mdframed}
\usepackage{mathrsfs}
\usepackage{changepage}
\usepackage{multicol}
\usepackage{slashed}
\usepackage{enumerate}
\usepackage{braket}
\usepackage{booktabs}
\usepackage{enumitem}
\usepackage{kantlipsum}  %This package lets us generate random text for example purposes.
\usepackage{pgfplots}


%%% Custom Commands %%%
% Natural Numbers 
\newcommand{\N}{\mathbb{N}}

% Whole Numbers
\newcommand{\W}{\mathbb{W}}

% Integers
\newcommand{\Z}{\mathbb{Z}}

% Rational Numbers
\newcommand{\Q}{\mathbb{Q}}

% Real Numbers
\newcommand{\R}{\mathbb{R}}

% Complex Numbers
\newcommand{\C}{\mathbb{C}}

\newcommand{\I}{\mathbb{I}}

\newcommand{\pfs}{\noindent\makebox[\linewidth]{\rule{\textwidth}{0.4pt}}\vspace{0.5cm}}

\newcommand{\mysqrt}[1]{%
	\mathpalette\foo{#1}%
}
\newcommand{\dmysqrt}[1]{%
	\mathpalette\foodisplay{#1}%
}

% !TeX spellcheck = off
\newcommand{\foo}[2]{%
	% #1: math style, #2: content
	\sbox0{$#1\sqrt{#2}$}% Measure the size of the standard sqrt in the current style
	\begin{tikzpicture}[baseline=(sqrt.base)]
		\node[inner sep=0, outer sep=0] (sqrt) {$#1\sqrt{#2}$}; % Use the current math style
		\draw([yshift=-0.045em]sqrt.north east) -- ++(0,-0.5ex); % Draw the tick
	\end{tikzpicture}%
}
% !TeX spellcheck = off
\newcommand{\foodisplay}[2]{%
	% #1: math style, #2: content
	\sbox0{$#1\sqrt{#2}$}% Measure the size of the standard sqrt in the current style
	\begin{tikzpicture}[baseline=(sqrt.base)]
		\node[inner sep=0, outer sep=0] (sqrt) {$\displaystyle\sqrt{#2}$}; % Force displaystyle
		\draw[line width=0.4pt] ([yshift=-0.044em]sqrt.north east) -- ++(0,-0.5ex); % Draw the tick
	\end{tikzpicture}%
}

% \renewcommand{\theenumi}{\arabic{enumi}}
% \renewcommand{\labelenumi}{\textbf{\theenumi}.}

% \newcommand{}{\marginnote{\includegraphics[width=2em]{caution.png}}}

\newmdenv[
	topline=false,
	bottomline=true,
	rightline=false,
	leftline=true,
	linewidth=1.5pt,
	linecolor=black, % default color, will be overridden in custom commands
	% backgroundcolor=draculafg, % Needed for Dracula theme
	fontcolor=\documentTheme, % Needed for Dracula theme
	innertopmargin=0pt,
	innerbottommargin=5pt,
	innerrightmargin=10pt,
	innerleftmargin=10pt,
	leftmargin=0pt,
	rightmargin=0pt,
	skipabove=\topsep,
	skipbelow=\topsep,
]{customframedproof}

\newenvironment{proofpart}[2][black]{
    \begin{mdframed}[
        topline=false,
        bottomline=false,
        rightline=false,
        leftline=true,
        linewidth=1pt,
        linecolor=#1!40, % Custom color
        % innertopmargin=10pt,
        % innerbottommargin=10pt,
        innerleftmargin=10pt,
        innerrightmargin=10pt,
        leftmargin=0pt,
        rightmargin=0pt,
        % skipabove=\topsep,
        % skipbelow=\topsep%
    ]
    \noindent
    \begin{minipage}[t]{0.08\textwidth}%
        \textbf{#2}%
    \end{minipage}%
    \begin{minipage}[t]{0.90\textwidth}%
        \begin{adjustwidth}{0pt}{0pt}%
}{
    \end{adjustwidth}
    \end{minipage}
    \end{mdframed}
}

% Define a new environment 'proofscratch' for Proof and Scratch Paper
\newenvironment{proofscratch}[2]{%
    \begin{mdframed}[
        topline=false,
        bottomline=false,
        rightline=false,
        leftline=false,
        backgroundcolor=draculabg, % Background color for Dracula theme
        fontcolor=\documentTheme,       % Font color for Dracula theme
        innerleftmargin=-6pt,
        innertopmargin=0pt,
        leftmargin=0pt,
        rightmargin=0pt,
        skipabove=\topsep,
        skipbelow=\topsep,
    ]
    % % Begin TikZ picture for the vertical dotted line
    % \begin{tikzpicture}[overlay, remember picture]
    %     % Draw a vertical dotted line at approximately the center of the mdframed width
    %     \draw[dotted, thick] ([xshift=0.005\linewidth]current bounding box.north west) -- ([xshift=0.005\linewidth]current bounding box.south west);
    % \end{tikzpicture}%
    % % Create the left minipage for Proof
    \begin{minipage}[t]{0.47\textwidth}%
        \noindent\textit{Proof.} #1 \hfill \(\qed\)
    \end{minipage}%
    \hfill
    % Create the right minipage for Scratch Paper
    \begin{minipage}[t]{0.47\textwidth}%
        \textit{Scratch Paper.} #2
    \end{minipage}%
}{
    \end{mdframed}
}

\newenvironment{tipbox}
	{%
		\par\noindent%
		% Top dashed line (using \textwidth)
		\tikz[baseline]{\draw[dash pattern=on 3pt off 2pt, line width=0.5pt, color=draculapurple] (0,0) -- (\textwidth,0);}%
		\par\vspace{0.65em}%
		\noindent\textbf{\textcolor{draculapurple}{TIP:}}\quad%
	}
	{%
		\par%
		% Bottom dashed line
		\noindent%
		\tikz[baseline]{\draw[dash pattern=on 3pt off 2pt, line width=0.5pt, color=draculapurple] (0,0) -- (\textwidth,0);}%
		\par\vspace{0.35em}
	}

\newenvironment{notebox}
	{%
		\par\noindent%
		% Top dashed line (using \textwidth)
		\tikz[baseline]{\draw[dash pattern=on 3pt off 2pt, line width=0.5pt, color=draculagreen] (0,0) -- (\textwidth,0);}%
		\par\vspace{0.65em}%
		\noindent\textbf{\textcolor{draculagreen}{NOTE:}}\quad%
	}
	{%
		\par%
		% Bottom dashed line
		\noindent%
		\tikz[baseline]{\draw[dash pattern=on 3pt off 2pt, line width=0.5pt, color=draculagreen] (0,0) -- (\textwidth,0);}%
		\par\vspace{0.35em}
	}

\newenvironment{warningbox}
	{%
		\par\noindent%
		% Top dashed line (using \textwidth)
		\tikz[baseline]{\draw[dash pattern=on 3pt off 2pt, line width=0.5pt, color=draculared] (0,0) -- (\textwidth,0);}%
		\par\vspace{0.65em}%
		\noindent\textbf{\textcolor{draculared}{WARNING:}}\quad%
	}
	{%
		\par%
		% Bottom dashed line
		\noindent%
		\tikz[baseline]{\draw[dash pattern=on 3pt off 2pt, line width=0.5pt, color=draculared] (0,0) -- (\textwidth,0);}%
		\par\vspace{0.35em}
	}

\newenvironment{solution}{\textit{Solution.}}

\newcommand{\sol}[1]{
    \begin{customframedproof}[linecolor=orangehdx!75,]
        \begin{solution}
        #1
        \end{solution}
    \end{customframedproof}
}

\newcommand{\pf}[1]{
    \begin{customframedproof}[linecolor=orangehdx!75]
        \begin{proof}
        #1
        \end{proof}
    \end{customframedproof}
}

\def \proofDistance {10pt}

\newcommand{\disabs}[1]{\ensuremath{\left|#1\right|}}
\newcommand{\limn}{\lim_{n \rightarrow \infty}}
\newcommand{\limx}[2]{\lim_{x \rightarrow #1}#2}

\newcommand{\limc}[1]{\lim_{x \rightarrow c}#1}

\newcommand{\ninn}{n \in \N}

\newcommand{\mb}[1]{\mathbf{#1}}

\newcommand{\limsupn}[1]{\limsup_{n\rightarrow \infty}#1}
\newcommand{\liminfn}[1]{\liminf_{n\rightarrow \infty}#1}


\newcommand{\proj}{\text{proj}}

\newcommand{\p}{\partial}

\newcommand{\dydx}{\frac{dy}{dx}}
\newcommand{\dxdy}{\frac{dx}{dy}}
\newcommand{\dydt}{\frac{dy}{dt}}
\newcommand{\dxdt}{\frac{dx}{dt}}
\newcommand{\dzdt}{\frac{dz}{dt}}

\newcommand{\barNotationT}[1]{\bigg|_{t = #1}}

\newcommand{\cyanit}[1]{\textit{\textcolor{cyan}{#1}}}

\newcommand{\brackett}[1]{\left\langle #1 \right\rangle}

\newcommand{\norm}[1]{\left\lVert \mathbf{#1}\right\rVert}

\newcommand{\imb}{\mb{i}}
\newcommand{\jmb}{\mb{j}}
\newcommand{\kmb}{\mb{k}}
\newcommand{\rmb}{\mb{r}}
\newcommand{\umb}{\mb{u}}

\newcommand{\vecfuc}[2]{\mb{#1}(#2)}
\newcommand{\dvecfuc}[2]{\mb{#1}'(#2)}
\newcommand{\normdvecfuc}[2]{\|\mb{#1}'(#2)\|}

\newcommand{\squigglyline}{%
    \noindent
    \tikz[baseline=-0.5ex]{
        \draw[decorate, decoration={snake, amplitude=0.5mm, segment length=3mm}] 
        (0,0) -- (\dimexpr\linewidth\relax,0);
    }%
}
\newcolumntype{Y}{>{\centering\arraybackslash}X}

% end of preamble
%-%-%-%-%-%-%-%-%-%-%-%-%-%-%-%-%-%-%-%-%-%-%-%-%-%-%-%-%-%-%-%-%-%-%-%-%-%-%

\begin{document}

\numberwithin{BoxCounter}{section}


%-%-%-%-%-%-%-%-%-%-%-%-%-%-%-%-%-%-%-%-%-%-%-%-%-%-%-%-%-%-%-%-%-%-%-%-%-%-%
%%% COVER PAGE %%%
%-%-%-%-%-%-%-%-%-%-%-%-%-%-%-%-%-%-%-%-%-%-%-%-%-%-%-%-%-%-%-%-%-%-%-%-%-%-%

\newcommand{\assignmentname}{Writing Assignment 3}

% Do not use all caps.
\newcommand{\cussubtitle}{Probability \& Statistics}
% Date format should be like: "January 1, 2001"
\newcommand{\finaldate}{November 21, 2025}
% Be sure to include degree recognition (e.g., B.S., M.S., Ph.D)
\newcommand{\professor}{Prof.\ Lars Seme, M.S.}

%-%-%-%-%-%-%-%-%-%-%-%-%-%-%-%-%-%-%-%-%-%-%-%-%-%-%-%-%-%-%-%-%-%-%-%-%-%-%

\begin{titlepage}
    \begin{center}

        \vspace*{-2cm}
        \includegraphics[width=0.8\textwidth]{\documentBigLogo}\\
        \vfill

        % Horizontal line above the title in 'horange' color
        \textcolor{horange}{\rule{\textwidth}{1.0pt}}

        \vspace{2em}

        {\huge \textbf{\assignmentname}}

        \vspace{1em} % Space between the title and the bottom line

        \textcolor{horange}{\rule{\textwidth}{1.0pt}}

        \vspace*{1\baselineskip}

        {\LARGE \textbf{\cussubtitle}}

        \begin{large}
            \vspace*{5\baselineskip}

            \vspace*{1\baselineskip}

            \emph{Author} \\[1ex]
            %Submitted by \\[\baselineskip]
            {\Large Paul Beggs \\ \par} % Editor list
            {\href{mailto:BeggsPA@Hendrix.edu}{{BeggsPA@Hendrix.edu}}}\\ % Editor affiliation

            \vspace*{1\baselineskip}

            \textit{Instructor} \\[1ex] % Tagline(s) or further description
            \professor

            \vspace*{1\baselineskip}

            \textit{Due}\\[1ex]
            {\scshape  \finaldate} \\[0.3\baselineskip] % Year published

            \thispagestyle{empty}

        \end{large}
    \end{center}
\end{titlepage}
% \pagestyle{standard}

\section{Introduction}

It can often be the case that when we collect empirical data, it can be challenging to distinguish between known theoretical distributions based on summary statistics alone. Thus, the focus of this paper is to investigate three distinct datasets\textemdash\textbf{Data1}, \textbf{Data2}, \textbf{Data3}\textemdash where each has \(n = 200\) samples, and each sample is generated by one of these methods:
\begin{enumerate}
    \item \(X \sim 15 + Gamma(6.25, 0.8)\)
    \item \(Y \sim 20 + \left(\sqrt{4/3}\right)t(3)\)
    \item \(Z \sim 20 + N(0,4)\).
\end{enumerate} 

\section{Data Analysis}

A quick check to determine where the data came from is to find the mean and variance of each theoretical distribution, and compare it with the sample mean and sample variance from the empirical ones. First, let us investigate the theoretical distributions:
\begin{enumerate}
    \item We know that \(X\) is gamma-distributed and shifted by 15, so its mean and variance formulas are \(\mu = 15 + \alpha\theta\) and \(\sigma^{2} = \alpha\theta^{2}\). Substituting \(\alpha\) for 6.25, and \(\theta\) for 0.8, we see that 
    \[
        \mu = 15 + (6.25 \cdot 0.8) = \boxed{20} \quad \text{ and } \quad \sigma^{2} = (0.64)(6.25) = \boxed{4}.
    \]
    \item For \(Y\), since it is \(t\)-distributed, its mean is usually 0, and its variance is found by solving \(\frac{r}{r - 2}\) for \(r > 2\). However, it is shifted by 20 and scaled by a factor of \(\sqrt{4/3}\), so we find its mean and variance to be:
    \[
        \mu = 0 + 20 = \boxed{20} \quad \text{ and } \quad \sigma^{2} = \frac{3}{3 - 2} \cdot \left(\sqrt{4/3}\right)^{2} = \boxed{4}.
    \]
    \item Then for \(Z\), we find the mean and variance from the distribution parameters, but we must also shift the mean by 20:
    \[
        \mu = 0 + 20 = \boxed{20} \quad \text{ and } \quad \sigma^{2} = \boxed{4}.
    \]
\end{enumerate}

The empirical means and sample variances can be seen in \hyperref[tab:dataStats]{Table 1}. These results leave us in a rather awkward situation: all of our theoretical distributions have the same mean and variance, and (perhaps unsurprisingly) the empirical distributions also have similar means with each other, and with the theoretical distributions. Thus, we must tackle this investigation from another angle.
\begin{table}[htbp]
    \centering
    \begin{tabularx}{\textwidth}{lYYYYYYY}
        \toprule
         & \textbf{Min.} & \(Q_{1}\) & \textbf{Median} & \(Q_{3}\) & \textbf{Max.} & \textbf{Mean} & \textbf{S. Var.} \\  \midrule
        \textbf{Data1} & \(9.325\) & \(19.101\) & \(20.095\) & \(20.979\) & \(26.490\) & 19.947 & 3.908 \\ \midrule
        \textbf{Data2} & \(16.179\) & \(18.486\) & \(19.777\) & \(21.163\) & \(27.306\) & 19.978 & 3.962 \\ \midrule
        \textbf{Data3} & \(14.057\) & \(18.665\) & \(20.093\) & \(21.462\) & \(25.354\) & 20.125 & 4.086 \\
        \bottomrule
    \end{tabularx}
    \caption{Dataset Statistics}\label{tab:dataStats}
\end{table} 
\newpage
\section{Box Plots \& Histograms}

For \textbf{Data1}, we see that its  histogram follows a bell-shaped curve (\hyperref[fig:data1histbox]{Figure 1 (a)}), and can see numerous outliers in its box plot (\hyperref[fig:data1histbox]{Figure 1 (b)}). Then, for \textbf{Data2}'s histogram (\hyperref[fig:data2histbox]{Figure 2 (a)}), we see that is also roughly bell shaped, but is right skewed. \textbf{Data3}'s histogram is also bell-shaped (\hyperref[fig:data3histbox]{Figure 3 (a)}), with most of its data being falling within the interquartile range with only 1 outlier, as seen in its box plot (\hyperref[fig:data3histbox]{Figure 3 (b)}). From these observations, we can make the draw the following connections:
\begin{itemize}
    \item \textbf{Gamma Distribution:} We know that the gamma distribution follows a shape similar to that of \textbf{Data2}'s, where there is a right skew with a bell-shape. We also know that gamma's sample space is non-negative, so we know that it must start at 15 (due to \(Y\)'s shifting). This leaves us with the clear choice that \textbf{Data2} must be gamma-distributed, as it is the only distribution that starts above 15 (as per \hyperref[tab:dataStats]{Table 1}).
    \item \textbf{Normal \& T Distributions:} Like the gamma distribution, these both are bell-shaped, but the normal curve has the property where 95\% of the data falls within two standard deviations of the mean. Compare that to \textbf{Data1}'s distribution that has two outliers that are over five standard deviations from the mean. This tells us that the \textbf{Data1} cannot be normally distributed.
\end{itemize}
Thus, through a process of elimination, we have strong evidence that \textbf{Data1} is \(t\)-distributed, \textbf{Data2} is gamma-distributed, and \textbf{Data3} is normally distributed. Now, to ensure our choices are valid, we will now investigate q-q plots.

\section{Q-Q Plots}

Q-q plots are measurements that take a data values at specific percentiles, and compares them to theoretical counterparts. For example, the smallest value in \textbf{Data1}'s set is \(\approx 9.326\) with percentile \(\approx 0.005\). So, we can take that percentile and calculate the corresponding value in one of our distributions and compare values. It is important to note that if the empirical distribution's value and the theoretical distribution's value are equal for every percentile, then that shows that both distributions are equivalent. In q-q plots, that relationship is modeled as \(y = x\), and can be seen as a straight line. Now, in our q-q plots (\hyperref[fig:qqplots]{Figure 4}), we see that \textbf{Data1} and the \(t\) distribution plot has a straight line (albeit with a few outliers); \textbf{Data2} and the gamma distribution plot is almost a near perfect straight line; and \textbf{Data3} and the normal curve plot show the straightest line for all datasets vs.\@ the normal distribution. 


\section{Conclusion}

Through our analysis, we have shown that when you cannot rely on comparing means and variances of empirical distributions to theoretical distributions, you still have other tools that you can rely on. Specifically, we examined histograms and box plots and gained strong evidence that \textbf{Data1} was \(t\)-distributed because it contained outliers that made it so that it could not be gamma-distributed, nor could it be normally distributed because it had outliers that were more than 5 standard deviations from the mean. Then, we also saw that the only theoretical distribution that could be mapped to \textbf{Data2} was gamma because the other two empirical distributions started below 15, which gamma cannot because its support is non-negative. Finally, that left \textbf{Data3} to be mapped to the normal distribution. Finally, we looked at q-q plots that directly supported our conclusions from the histograms and box plots. Thus, through this investigation, we have definitively shown that \textbf{Data1} is \(t\)-distributed, \textbf{Data2} is gamma-distributed, and \textbf{Data3} is normally distributed. 



\newpage

\section*{Appendix A:\@ Histograms \& Box Plots}

\begin{figure}[h!]
    \centering\label{fig:data1histbox}
    \begin{subfigure}[c]{0.475\linewidth}
        \includegraphics[width=\linewidth]{graphs/Histograms/Data1Hist.png}
        \caption{Histogram}
    \end{subfigure}
    \begin{subfigure}[c]{0.475\linewidth}
        \includegraphics[width=\linewidth]{graphs/BoxPlots/Data1Box.png}
        \caption{Box Plot}
    \end{subfigure}
\caption{\textbf{Data1}'s Histogram \& Box Plot}    
\end{figure}

\begin{figure}[h!]
    \centering\label{fig:data2histbox}
    \begin{subfigure}[c]{0.475\linewidth}
        \includegraphics[width=\linewidth]{graphs/Histograms/Data2Hist.png}
        \caption{Histogram}
    \end{subfigure}
    \begin{subfigure}[c]{0.475\linewidth}
        \includegraphics[width=\linewidth]{graphs/BoxPlots/Data2Box.png}
        \caption{Box Plot}
    \end{subfigure}
\caption{\textbf{Data2}'s Histogram \& Box Plot}    
\end{figure}

\newpage

\begin{figure}[ht]
    \centering\label{fig:data3histbox}
    \begin{subfigure}[c]{0.475\linewidth}
        \includegraphics[width=\linewidth]{graphs/Histograms/Data3Hist.png}
        \caption{Histogram}
    \end{subfigure}
    \begin{subfigure}[c]{0.475\linewidth}
        \includegraphics[width=\linewidth]{graphs/BoxPlots/Data3Box.png}
        \caption{Box Plot}
    \end{subfigure}
\caption{\textbf{Data3}'s Histogram \& Box Plot}    
\end{figure}

\newpage
\section*{Appendix B:\@ Q-Q Plots}

\begin{figure}[h]
    \centering
    \label{fig:qqplots}
    \includegraphics[width=\linewidth]{graphs/allqqs.png}
    \caption{All Q-Q Plots; All plots are shown on the same data range for consistency.}
\end{figure}

\end{document}