% This is the "preamble" of the document. This is where the format options get set.
% Pro-tip: things following the % mark will not be compiled by LaTeX. I'll be using them extensively to explain things as we go.
% Note: not to scare you off of LaTeX, but it's normal to have problems. And ya girl has been having some. I've included the copyright info at the bottom of the document from the guy who wrote this package, because his documentation doesn't entirely match how it's actually used. So this is a combination of his working preamble along with my added commentary or explanation. 
%% 


\documentclass[stu,12pt,floatsintext]{apa7}
\usepackage{geometry}
\linespread{2}
% Document class input explanation ________________
% LaTeX files need to start with the document class, so it knows what it's using
% - This file is using the apa7 document class, as it has a lot of the formatting built in
% There are two sets of brackets in LaTeX, for each command (the things that start with the slash \ )
% - The squiggle brackets {} are mandatory for executing the command
% - The square brackets [] are options for that command. There can be more than one set of square brackets for some commands
% Options used in this document (general note - for each of these, if you want to use the other options, swap it out in that spot in the square brackets):
% - stu: this sets the `document mode' as the "student paper" version. Other options are jou (journal), man (manuscript, for journal submission), and doc (a plain document)
% --- The student setting includes things like 'duedate', 'course', and 'professor' on the title page. If these aren't wanted/needed, use the 'man' setting. It also defaults to including the tables and figures at the end of the document. This can be changed by including the 'floatsintext' option, as I have for you. If the instructor wants those at the end, remove that from the square brackets.
% --- The manuscript setting is roughly what you would use to submit to a journal, so uses 'date' instead of 'duedate', and doesn't include the 'course' or 'professor' info. As with 'stu', it defaults to putting the tables and figures at the end rather than in text. The same option will bump those images in text.
% --- Journal ('jou') outputs something similar to a common journal format - double columned text and figurs in place. This can be fun, especially if you are sumbitting this as a writing sample in applications.
% --- Document ('doc') outputs single columned, single spaced text with figures in place. Another option for producing a more polished looking document as a writing sample.
% - 12pt: sets the font size to 12pt. Other options are 10pt or 11pt
% - floatsintext: makes it so tables and figures will appear in text rather than at the end. Unforunately, not having this option set breaks the whole document, and I haven't been able to figure out why. IT's GREAT WHEN THINGS WORK LIKE THEY'RE SUPPOSED TO.

\usepackage[american]{babel}

\usepackage{csquotes} % One of the things you learn about LaTeX is at some level, it's like magic. The references weren't printing as they should without this line, and the guy who wrote the package included it, so here it is. Because LaTeX reasons.
\usepackage[style=apa,sortcites=true,sorting=nyt,backend=biber]{biblatex}
% biblatex: loads the package that will handle the bibliographic info. Other option is natbib, which allows for more customization
% - style=apa: sets the reference format to use apa (albeit the 6th edition)
\DeclareLanguageMapping{american}{american-apa} % Gotta make sure we're patriotic up in here. Seriously, though, there can be local variants to how citations are handled, this sets it to the American idiosyncrasies 
\addbibresource{bibliography.bib} % This is the companion file to the main one you're writing. It contains all of the bibliographic info for your references. It's a little bit of a pain to get used to, but once you do, it's the best. Especially if you recycle references between papers. You only have to get the pieces in the holes once.`

\usepackage{hyperref}

\definecolor{horange}{HTML}{f58026}
\hypersetup{
	colorlinks=true,
	linkcolor=horange,
	filecolor=horange,      
	urlcolor=horange,
}

\usepackage[T1]{fontenc} 
\usepackage{mathptmx} % This is the Times New Roman font, which was the norm back in my day. If you'd like to use a different font, the options are laid out here: https://www.overleaf.com/learn/latex/Font_typefaces
% Alternately, you can comment out or delete these two commands and just use the Overleaf default font. So many choices!


% Title page stuff _____________________
\title{Applying Cognitive Psychology Methods Activity 1: Attention} % The big, long version of the title for the title page
\author{Paul Beggs}
\duedate{\today}
% \date{January 17, 2024} The student version doesn't use the \date command, for whatever reason
\affiliation{Department of Psychology, Hendrix College}
\course{PSYC 319: Cognitive Psychology} % LaTeX gets annoyed (i.e., throws a grumble-error) if this is blank, so I put something here. However, if your instructor will mark you off for this being on the title page, you can leave this entry blank (delete the PSY 4321, but leave the command), and just make peace with the error that will happen. It won't break the document.
\professor{Dr. Carmen Merrick}  % Same situation as for the course info. Some instructors want this, some absolutely don't and will take off points. So do what you gotta.

% \abstract{In this paper, I plan to investigate two functions of attention, those being \href{https://www.psytoolkit.org/lessons/experiment_visualsearch.html}{\textit{Visual Search Task}} and \href{https://www.psytoolkit.org/experiment-library/multitasking.html}{\textit{Divided Attention}}. In doing so, I will report my observations by answering a sequential list of questions that will are outlined on the Evaluation Rubric. In addition, I will include the \hyperlink{results}{results} of both tests.}


%\characterwords{APA style, demonstration} % If you need to have characterwords for your paper, delete the % at the start of this line

\begin{document}

\maketitle % This tells LaTeX to make the title page

\subsection{Visual Search Task}

This study was designed to study how \textit{search}---scanning of the environment for particular features---functions. To be more specific, the researchers utilized \hypertarget{Conjunction}{\textit{conjunctive search}}---sifting through a combination (or conjunction) of two or more features in the same stimulus---in their experimentation to highlight a specific subset of search.

During the experimentation my goal was to identify an upright red `T'. In the procedure, I was to focus on a white cross that would briefly flash. After the flash, a varying number of upside down red `T's that were presented for me to search through. Additionally, there would be blue upright `T's that were designed to mimic the orientation of the target---the upright red `T'. In some cases, the upright `T' would not be present among the noise, but when it was, I would be signaled to press the space bar. In doing this experiment, I aligned the most with the \textit{Guided Search Theory}---an observer looks for a target among several distracting items. In that, I knew exactly what I was looking for, so when I across an upright, blue `T', I was able to tune it out because I knew any object of that color would be a red herring. This allowed me to focus specifically on the red `T's, but this still did not allow me to instantly identify the upright `T's among the others.

Let's examine what would happen if we looked at this theory from the lens of \textit{Signal Similarity Theory}---the task's difficulty is dependent on the similarity of stimulus to distractor. If this theory was utilized during the task of search, then I would not be able to tune out the blue upright `T's because they would be the most familiar to me. Thus, I would take more time, and be more inaccurate.

I think that this experiment correctly isolated the focus of attention that it was after. That being, I was tasked to \textit{search} through a variety of objects that had combined traits in order to find the target. I believe this is the most pertinent type of experiment for this task.

\subsection{Divided Attention}

This study was designed to test \textit{divided attention}---the distribution of attention among multiple tasks (see the \hyperlink{last}{last paragraph} for discussion). Researchers had a rectangle diagram that was halved along the equator, exposing two sections that were labeled either shape or filling. In these sections, there are different rules, as shown below:
\begin{enumerate}
    \item \textbf{Shape:} A shape would flash on the screen, and I was to press either \texttt{b} for a diamond shape, or \texttt{n} for a square.
    \item \textbf{Filling:} In this section, two sets of dots flashed on the screen. I was to use the same characters again, but this time, the letters \texttt{b} corresponded to 2 dots inside of a shape, and the letter \texttt{n} corresponded to 3 dots.
\end{enumerate}

This is a divided attention study because the dots were inside the shape on the top section. Similarly, the shapes containing the dots on the bottom section were the same diamond or square that I had to identify on the top screen. This mix-matching of signals made it difficult to differentiate what character to press. For instance, because the same characters were utilized for both the top and bottom sections, the assignment of the rules for each section `bled' into the other section.

This `bleeding' of rules can be attributed to Treisman's \textit{Feature Integration Theory}---objects are analyzed into their features in the preattentive stage, and the features are later combined with the aid of attention. That is, when I first saw the object that flashed on my screen (preattentive), I mentally noted how many dots were in the shape, and what the shape was. Depending on where the shape spawned in the diagram, I was primed to favor the numerical value or the shape's orientation. In a way, I was preemptively dialing down one aspect of the figure that was dependent on its placement in either half. Interestingly, after a few milliseconds, I would combine the two features into one, and that would complicate things; I learned the longer that I analyze the shape, the slower (and more inaccurate) I became.

This is in stark contrast to a theory posed by Broadbent's \textit{Filter Model of Attention}---selective attention is the ability to focus on one stimuli and ignore other stimuli simultaneously. According to his early selection model, if I were to focus on either the top or bottom of the rectangle and associate the corresponding rule with either section, I would get the correct answer every time by completely tuning out the dots or the shape. This was not the case.

\hypertarget{last}{}The researchers wanted to specifically study divided attention in this study, and I do not think this is the best method to use for this topic. In fact, I think this study is more akin to the \hyperlink{Conjunction}{\textit{Conjunction Search}} that was defined above. I make this claim because conjunction searches are designed to study \textit{binding}---the process by which features such as color, form, motion, and location are combined to create our perception of a coherent object---and that was the primary distractor of the study. One could argue that juggling the rules in your head is divided attention, but I would disagree: the `juggling' of rules is more appropriately attributed to binding. Instead of `juggling the rules' a more fitting term would be `entanglement of rules' (specifically, in dealing with the de-entanglement of the rules to find the correct one). Perhaps a study more akin to David Strayer and William Johnston (2001) would be a more viable research direction for distracted attention.

\newpage


\end{document}

%% 
%% Copyright (C) 2019 by Daniel A. Weiss <daniel.weiss.led at gmail.com>
%% 
%% This work may be distributed and/or modified under the
%% conditions of the LaTeX Project Public License (LPPL), either
%% version 1.3c of this license or (at your option) any later
%% version.  The latest version of this license is in the file:
%% 
%% http://www.latex-project.org/lppl.txt
%% 
%% Users may freely modify these files without permission, as long as the
%% copyright line and this statement are maintained intact.
%% 
%% This work is not endorsed by, affiliated with, or probably even known
%% by, the American Psychological Association.
%% 
%% This work is "maintained" (as per LPPL maintenance status) by
%% Daniel A. Weiss.
%% 
%% This work consists of the file  apa7.dtx
%% and the derived files           apa7.ins,
%%                                 apa7.cls,
%%                                 apa7.pdf,
%%                                 README,
%%                                 APA7american.txt,
%%                                 APA7british.txt,
%%                                 APA7dutch.txt,
%%                                 APA7english.txt,
%%                                 APA7german.txt,
%%                                 APA7ngerman.txt,
%%                                 APA7greek.txt,
%%                                 APA7czech.txt,
%%                                 APA7turkish.txt,
%%                                 APA7endfloat.cfg,
%%                                 Figure1.pdf,
%%                                 shortsample.tex,
%%                                 longsample.tex, and
%%                                 bibliography.bib.
%% 
%%
%%
%% This is file `./samples/shortsample.tex',
%% generated with the docstrip utility.
%%
%% The original source files were:
%%
%% apa7.dtx  (with options: `shortsample')
%% ----------------------------------------------------------------------
%% 
%% apa7 - A LaTeX class for formatting documents in compliance with the
%% American Psychological Association's Publication Manual, 7th edition
%% 
%% Copyright (C) 2019 by Daniel A. Weiss <daniel.weiss.led at gmail.com>
%% 
%% This work may be distributed and/or modified under the
%% conditions of the LaTeX Project Public License (LPPL), either
%% version 1.3c of this license or (at your option) any later
%% version.  The latest version of this license is in the file:
%% 
%% http://www.latex-project.org/lppl.txt
%% 
%% Users may freely modify these files without permission, as long as the
%% copyright line and this statement are maintained intact.
%% 
%% This work is not endorsed by, affiliated with, or probably even known
%% by, the American Psychological Association.
%% 
%% ----------------------------------------------------------------------