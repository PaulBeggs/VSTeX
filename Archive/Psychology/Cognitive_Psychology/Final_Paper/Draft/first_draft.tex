\documentclass[stu]{apa7}
\usepackage{amsmath}
\usepackage{amssymb}
\usepackage{enumitem}
\usepackage{geometry}
\usepackage[style=apa,uniquename=false,backend=biber]{biblatex}
\usepackage{hyperref}
\usepackage{xcolor}
\definecolor{horange}{HTML}{f58026}
\hypersetup{
    colorlinks=true,
    linkcolor=horange,
    filecolor=horange,      
    urlcolor=horange,
    citecolor=horange
}
\usepackage{C:/Users/paulb/VSTeX/local/draculatheme}
\addbibresource{references.bib}

\title{The Effect of ADHD on Working Memory Components: A Literature Review}
\shorttitle{The Effect of ADHD on Working Memory Components}
\author{Paul Beggs}
\authorsaffiliations{Hendrix College}
\course{PSYC 319: Cognitive Psychology}
\professor{Dr.\ Carmen Merrick}
\duedate{\today}


% \abstract{
%     This literature review explores the hypothesis that individuals with ADHD experience significant impairments in working memory, particularly in the central executive, phonological loop, and visuospatial sketchpad. By examining recent research, this paper seeks to identify specific deficits and their implications for cognitive functioning in individuals with ADHD. % Expand Abstract
% }

\begin{document}

\maketitle

Attention-Deficit/Hyperactivity Disorder (ADHD) is a prevalent neurodevelopmental disorder characterized by symptoms of inattention, hyperactivity, and impulsivity \parencite{faraone_attention-deficithyperactivity_2015}. Among the various cognitive functions affected by ADHD, working memory (WM) deficits have garnered significant attention due to their pervasive impact on academic achievement, decision-making, and daily functioning. WM refers to a cognitive system responsible for temporarily storing and manipulating information essential for tasks such as problem-solving, learning, and reasoning \parencite{baddeley_working_2020}.

This literature review explores the hypothesis that ADHD is associated with impairments across all major components of WM, specifically the episodic buffer (EB), phonological loop (PH), visuospatial sketchpad (VS), and central executive (CE) \parencite{baddeley_developments_1994}. By examining these components, the review aims to delineate specific patterns of deficit and their implications for individuals' cognitive functioning. The structure of the paper begins with an overview of WM and its components, followed by a detailed discussion of ADHD-related impairments in each domain, and concludes with how WM affects other cognitive domains such as problem-solving.

\section{Working Memory}

WM comprises multiple interacting components that work in concert to process and maintain information \parencite{baddeley_working_2020}. The CE acts as a supervisory system, coordinating information flow between the phonological loop PH and visuospatial sketchpad VS, while the episodic buffer EB integrates information from these subsystems with long-term memory \parencite{baddeley_developments_1994}. This orchestration allows for complex cognitive tasks such as mental arithmetic, where the PH maintains verbal number representations while the VS supports spatial manipulation \parencite{kofler_working_2020}.

The interaction between these components is particularly evident in tasks requiring cross-modal integration. For example, when reading, the PH processes phonological information while the VS handles visual word forms, with the CE managing attention allocation between these processes \parencite{friedman_reading_2017}. The EB then binds this information with semantic knowledge from long-term memory, creating coherent mental representations \parencite{baddeley_working_2012}. Understanding these interactions is crucial for examining how ADHD affects WM function, as deficits in any component can cascade through the entire system.

\subsection{Episodic Buffer}

The episodic buffer (EB), responsible for integrating information across different modalities and temporal contexts \parencite{baddeley_developments_1994}, provides an intriguing but less consistent picture in ADHD research. Some studies suggest potential deficits in the EB among individuals with ADHD. For instance, \textcite{fabio_working_2020} found impairments in delayed discounting tasks, interpreted as difficulties in integrating temporal information, which is a key function of the EB. Similarly, \textcite{nyhout_episodic_2023} highlights challenges in cross-modal integration tasks among children with ADHD.

However, these findings are not universally observed. Larger-scale studies, such as \textcite{kofler_are_2018}, have not found disproportionate impairments in the EB relative to other working memory components. While methodological differences—such as sample size, assessment tools, and statistical approaches—may account for some of these discrepancies, they collectively point to the EB as an area where deficits are proportional to general working memory impairments rather than uniquely pronounced.

Despite these mixed findings, the EB remains a relevant component of working memory in ADHD. Its proportional deficits align with broader working memory impairments, underscoring the pervasive nature of ADHD's impact on cognitive systems. While the primary focus of this review remains on the PH, VS, and CE, the EB warrants continued investigation as part of a holistic understanding of ADHD-related cognitive challenges.

\subsection{Phonological Loop}

The phonological loop, responsible for verbal and auditory information processing \parencite{baddeley_developments_1994}, shows consistent impairments in individuals with ADHD. Research by \textcite{friedman_reading_2017} demonstrates a medium effect size for phonological short-term memory deficits \((d = -0.70)\), with more pronounced impairments in tasks requiring conversion from visual stimuli to phonological encoding \((d = -0.93)\). These findings suggest that individuals with ADHD struggle not only with maintaining verbal information but also with cross-modal processing involving phonological components.

Further research found similar patterns of impairment in phonological processing and orthographic encoding tasks \parencite{raiker_phonological_2019}. The consistency across these studies points to the phonological loop as a significant area of dysfunction in ADHD, particularly when tasks require both storage and manipulation of verbal information.

\subsection{Visuospatial Sketchpad}


The visuospatial sketchpad manages the temporary storage and manipulation of visual and spatial information \parencite{baddeley_developments_1994}. \textcite{butzbach_basic_2019} examines visuospatial working memory in adults with ADHD and finds that individuals with ADHD exhibit significant deficits in tasks requiring spatial reasoning and visual tracking. These impairments are evident in cross-modality tasks that necessitate the integration of phonological and visuospatial information, suggesting that the visuospatial sketchpad is disproportionately affected in ADHD beyond general working memory impairments.

Interestingly, \textcite{engelmann_motivation_2007} found that elevated motivation leads to improved efficiency in orienting and reorienting of spatial attention, suggesting that motivational factors may modulate visuospatial processing. This finding has important implications for understanding ADHD-related deficits, as it indicates potential compensatory mechanisms through motivational enhancement.

Furthermore, \textcite{elosua_differences_2017} corroborates these findings by demonstrating that children with ADHD perform poorly on visuospatial tasks compared to their non-ADHD peers. The consistent evidence across studies reinforces the conclusion that the visuospatial sketchpad is a critical area of impairment in individuals with ADHD, contributing to the broader cognitive challenges associated with the disorder.

\subsection{Central Executive}

The central executive serves as the supervisory component of working memory, performing three critical functions \parencite{baddeley_working_2012}: focusing attention, managing concurrent tasks, and switching between different cognitive activities. This component is particularly relevant to ADHD, as executive control deficits are central to the disorder's cognitive profile.

Research by \textcite{mccabe_relationship_2010} demonstrates a remarkably strong correlation between working memory capacity and executive function \((r = .97)\), suggesting these constructs may be largely synonymous. Their findings introduce the concept of ``executive attention,'' highlighting the intricate relationship between attentional control and working memory performance.

\subsubsection{CE and motivation}

The relationship between executive control and motivation presents an intriguing dynamic in ADHD. \textcite{pessoa_how_2009} proposes, based on neuroimaging evidence, that motivation can both enhance and potentially impair central executive functioning, depending on attentional focus. This dual effect suggests a complex interaction between motivational states and executive control.

Supporting this perspective, \textcite{elosua_differences_2017} employed multiple executive function measures, including the trail making task, N-back task, stroop task, and dual-task paradigm. Their findings indicate that "intrinsic interest in the task can improve performance," highlighting motivation's crucial role in executive functioning for individuals with ADHD.

\section{Discussion}

The convergence of evidence across multiple studies strengthens our understanding of working memory deficits in ADHD. \textcite{elosua_differences_2017} and \textcite{fried_clinical_2016} identified significant working memory impairments through distinct components—executive function and phonological processing, respectively—suggesting that ADHD affects multiple aspects of working memory processing.

Methodologically robust studies using bifactor models have provided particularly compelling evidence. \textcite{friedman_reading_2017} and \textcite{raiker_phonological_2019} demonstrated consistent findings regarding phonological processing deficits, while \textcite{kofler_working_2020} extended these observations to include both phonological and visuospatial components. The consistency of findings across these studies, which found high rates of cognitive impairment \((75\%\text{--}81\%)\), suggests that working memory deficits represent a core feature of ADHD rather than a secondary characteristic.



\subsection{Problem-Solving}

% In a large study \((N = 517; N_{ADHD} = 276)\) conducted by \textcite{fried_clinical_2016}, they found that deficits in working memory contribute to lower scores for domains such as math and reading; thus contributing to lower grade retention. These results corroborate with \textcite{friedman_reading_2017}.

In a large study \((N = 517; N_{ADHD} = 276)\) conducted by \textcite{fried_clinical_2016}, working memory deficits were found to significantly impact academic performance, particularly in mathematics and reading comprehension, contributing to higher rates of grade retention. These findings align with \textcite{friedman_reading_2017}, who showed a large discrepancy in reading fluency between kids with ADHD and those without \(d = -1.00\).

Further evidence from \textcite{capri_attention_2019} reinforces these conclusions through a focused examination of young adults with ADHD \((N = 24)\). Using computerized assessments including the Stroop Test and Tower of Hanoi, they found significant deficits in attention, problem-solving, and decision-making capabilities. While the sample size was modest, the findings contribute to our understanding of how working memory deficits manifest in real-world cognitive tasks.

\textcite{drigas_executive_2019} suggests that cognitive flexibility and emotional regulation significantly influence decision-making processes, particularly in information processing and goal prioritization. Their research indicates that individuals with ADHD may compensate for cognitive control deficits by developing enhanced problem-solving strategies, offering a potential avenue for intervention and support.

% \textcite{capri_attention_2019} = Objectives: Although attention-deficit/hyperactivity disorder (ADHD) is clearly associated with executive dysfunctions, the neuropsychological profile of adults with ADHD is unclear. The present study aimed at examining neuropsychological performance on tasks measuring attention, problem solving and decision making in young adults with ADHD. Methods: 12 young adults with ADHD (Mean age 18.33; SD= 11.48) and 12 healthy young adults (Mean age 18.41; SD= 18.70) matched for age and gender, performed the following neuropsychological test battery: Stroop Test, Tower of Hanoi and Gambling Task. All the tests were administered via computer using the software Presentations. Results: Results showed that adults with ADHD exhibit deficits in attention, problem solving and decision making. These findings warrant further examination of neuropsychological profile in adults with ADHD to improve the understanding of underlying neurocognitive mechanisms. Conclusions: This study suggests that young adults with ADHD show a considerable impairment in attention, problem solving and decision making. The present study contributes to understanding the neuropsychological picture of young adults with ADHD.

% \textcite{drigas_executive_2019} = Therefore, individuals’ overall cognitive flexibility and emotional regulation can promote the quality, quantity and speed of decision-making processes, such as adaptable and creative information processing as well as efficiency in setting and prioritizing goals. Moreover, individuals with ADHD, Autism Spectrum Disorder, Oppositional Defiant Disorder and individuals with other comorbid states, such as older adults, individuals with Traumatic Brain Injury (TBI) can counterbalance their cognitive control deficits through enhancing their problem solving skills.


\section{Conclusion}

This literature review has examined the relationship between Attention-Deficit/Hyperactivity Disorder (ADHD) and working memory (WM) components. The evidence consistently demonstrates significant impairments across multiple WM domains, with particularly robust findings in the central executive \parencite{mccabe_relationship_2010} and visuospatial sketchpad \parencite{butzbach_basic_2019}. The phonological loop also shows consistent deficits \parencite{friedman_reading_2017}, though the magnitude of impairment varies across tasks and populations.

Methodologically robust studies using diverse assessment tools and larger sample sizes \parencite{fried_clinical_2016} have strengthened our understanding of these deficits, revealing that working memory impairments affect 75-81\% of individuals with ADHD. These findings suggest that working memory dysfunction represents a core feature of ADHD rather than a secondary characteristic. Moreover, the impact extends beyond cognitive measures to practical outcomes, affecting academic achievement and daily functioning \parencite{kofler_working_2020}.

The interaction between motivation and cognitive performance \parencite{engelmann_motivation_2007} offers promising directions for intervention. Understanding how motivational enhancement can modulate working memory performance may lead to more effective therapeutic approaches. Future research should focus on developing standardized assessment protocols and investigating how working memory training might be optimized through motivational strategies to improve functional outcomes for individuals with ADHD.

% \section{Unused sources}


% \textcite{mohamed_basic_2021} = Foundational Processes; Cognitive Processes; Motivation; Methodological Critique

% \textcite{skalski_impact_2021} = Motivation; Cognitive Capacity; ADHD; Attention

\printbibliography

\end{document}