\documentclass[stu]{apa7}
\usepackage{amsmath}
\usepackage{amssymb}
\usepackage{enumitem}
\usepackage{geometry}
\usepackage[style=apa,uniquename=false,backend=biber]{biblatex}
\usepackage{hyperref}
\usepackage{xcolor}
\definecolor{horange}{HTML}{f58026}
\hypersetup{
    colorlinks=true,
    linkcolor=horange,
    filecolor=horange,      
    urlcolor=horange,
    citecolor=horange
}
\usepackage{C:/Users/paulb/VSTeX/local/draculatheme}
\addbibresource{references.bib}

\title{The Effect of ADHD on Working Memory Components: A Literature Review}
\shorttitle{The Effect of ADHD on Working Memory Components}
\author{Paul Beggs}
\authorsaffiliations{Hendrix College}
\course{PSYC 319: Cognitive Psychology}
\professor{Dr.\ Carmen Merrick}
\duedate{\today}


\begin{document}

\maketitle

Attention-Deficit/Hyperactivity Disorder (ADHD) is a prevalent neurodevelopmental disorder characterized by symptoms of inattention, hyperactivity, and impulsivity \parencite{faraone_attention-deficithyperactivity_2015}. Among the various cognitive functions affected by ADHD, working memory (WM) deficits have garnered significant attention due to their pervasive impact on academic achievement, decision-making, and daily functioning. WM refers to a cognitive system responsible for temporarily storing and manipulating information essential for tasks such as problem-solving, learning, and reasoning \parencite{baddeley_working_2020}.

This literature review explores the hypothesis that ADHD is associated with impairments across all major components of WM, specifically the episodic buffer, phonological loop, visuospatial sketchpad, and central executive \parencite{baddeley_developments_1994}. By examining these components, the review aims to delineate specific patterns of deficit and their implications for individuals' cognitive functioning. The structure of the paper begins with an overview of WM and its components, followed by a detailed discussion of ADHD-related impairments in each domain, and concludes with how WM affects other cognitive domains such as problem-solving.

\section{Working Memory}

WM comprises multiple interacting components that work in concert to process and maintain information \parencite{baddeley_working_2020}. The central executive acts as a supervisory system, coordinating information flow between the phonological loop and visuospatial sketchpad, while the episodic buffer integrates information from these subsystems with long-term memory \parencite{baddeley_developments_1994}. This orchestration allows for complex cognitive tasks such as mental arithmetic, where the phonological loop maintains verbal number representations while the visuospatial sketchpad supports spatial manipulation \parencite{kofler_working_2020}.

The interaction between these components is particularly evident in tasks requiring cross-modal integration. For example, when reading, the phonological loop processes verbal information while the visuospatial sketchpad handles visual word forms, with the central executive managing attention allocation between these processes \parencite{friedman_reading_2017}. Then, the episodic buffer binds this information with semantic knowledge from long-term memory, creating coherent mental representations \parencite{baddeley_working_2012}. Understanding these interactions is crucial for examining how ADHD affects WM function, as deficits in any component can cascade through the entire system.

\subsection{Episodic Buffer}

The episodic buffer is responsible for integrating information across different modalities and temporal contexts \parencite{baddeley_developments_1994}. Its presence provides an intriguing but less consistent picture in ADHD research. While some studies  suggest potential deficits in the episodic buffer among individuals with ADHD, others do not. For instance, one article found impairments in delayed discounting tasks, interpreted as difficulties in integrating temporal information, which is a key function of the episodic buffer \parencite{fabio_working_2020}. Similar research highlights challenges in cross-modal integration tasks among children with ADHD \parencite{nyhout_episodic_2023}.

However, these findings are not universally observed. Larger-scale studies, have not found disproportionate impairments in the episodic buffer relative to other working memory components \parencite{kofler_are_2018}. While methodological differences—such as sample size, assessment tools, and statistical approaches—may account for some of these discrepancies, they collectively point to the episodic buffer as an area where deficits are proportional to general working memory impairments rather than uniquely pronounced.

Despite these mixed findings, the episodic buffer remains a relevant component of working memory in ADHD. Its proportional deficits align with broader working memory impairments, underscoring the pervasive nature of ADHD's impact on cognitive systems. While the primary focus of this review remains on the phonological loop, visuospatial sketchpad, and central executive, the episodic buffer warrants continued investigation as part of a holistic understanding of ADHD-related cognitive challenges.

\subsection{Phonological Loop}

The phonological loop is responsible for verbal and auditory information processing \parencite{baddeley_developments_1994}. For people with ADHD, there are consistent impairments in the phonological loop. Research demonstrates a medium effect size for phonological short-term memory deficits, with more pronounced impairments in tasks requiring conversion from visual stimuli to phonological encoding \parencite{friedman_reading_2017}. These findings suggest that individuals with ADHD struggle not only with maintaining verbal information but also with cross-modal processing involving phonological components.

Further research found similar patterns of impairment in phonological processing and orthographic encoding tasks \parencite{raiker_phonological_2019}. The consistency across these studies points to the phonological loop as a significant area of dysfunction in ADHD, particularly when tasks require both storage and manipulation of verbal information.

\subsection{Visuospatial Sketchpad}

The visuospatial sketchpad, responsible for the temporary storage and manipulation of visual and spatial information \parencite{baddeley_developments_1994}, plays a crucial role in cognitive functioning. Research indicates that individuals with ADHD exhibit significant deficits in tasks requiring spatial reasoning and visual tracking, highlighting impairments in the visuospatial sketchpad beyond general working memory challenges \parencite{butzbach_basic_2019,elosua_differences_2017}. These deficits are particularly evident in cross-modality tasks that necessitate the integration of phonological and visuospatial information, suggesting a disproportionate impact on this component of working memory in ADHD.

Moreover, motivational factors may modulate visuospatial processing. Elevated motivation has been found to improve efficiency in orienting and reorienting spatial attention \parencite{engelmann_motivation_2007}, implying potential compensatory mechanisms for the visuospatial deficits observed in ADHD. This has important implications for understanding ADHD-related challenges, as enhancing motivation could mitigate some cognitive impairments associated with the disorder.

Consistent evidence across studies reinforces the conclusion that the visuospatial sketchpad is a critical area of impairment in individuals with ADHD, contributing to broader cognitive difficulties \parencite{butzbach_basic_2019,elosua_differences_2017,engelmann_motivation_2007}. Addressing these impairments may be essential for improving cognitive functioning and overall outcomes for those affected by ADHD.

\subsection{Central Executive}

The central executive serves as the supervisory component of working memory, performing three critical functions \parencite{baddeley_working_2012}: focusing attention, managing concurrent tasks, and switching between different cognitive activities. This component is particularly relevant to ADHD, as executive control deficits are central to the disorder's cognitive profile.

Researchers have demonstrated a remarkably strong correlation between working memory capacity and executive function, suggesting these constructs may be largely synonymous \parencite{mccabe_relationship_2010}. Their findings introduce the concept of ``executive attention,'' highlighting the intricate relationship between attentional control and working memory performance.
% Need to add more about the actual effect of the central executive and not whatever this crap is. Use this source:
% Fosco, W. D., Kofler, M. J., Groves, N. B., Chan, E. S. M., & Raiker, J. S., Jr (2020). Which 'Working' Components of Working Memory aren't Working in Youth with ADHD?. Journal of abnormal child psychology, 48(5), 647–660. https://doi.org/10.1007/s10802-020-00621-y

The relationship between executive control and motivation presents an intriguing dynamic in ADHD. Further research suggests that, based on neuroimaging evidence, motivation can both enhance and potentially impair central executive functioning, depending on attentional focus \parencite{pessoa_how_2009}. This dual effect suggests a complex interaction between motivational states and executive control. Supporting this perspective, additional research has examined multiple executive function measures, including the Trail Making Task, N-Back Task, Stroop Task, and dual-task paradigm. Findings from these studies suggest that an 'intrinsic interest in the task can improve performance,' emphasizing the crucial role of motivation in executive functioning for individuals with ADHD \parencite{elosua_differences_2017}.

\section{Bifactor Models and Problem-Solving}

The convergence of evidence across multiple studies strengthens our understanding of working memory deficits in ADHD. Significant impairments have been identified in executive function and phonological processing, further supporting the notion that ADHD affects multiple aspects of working memory processing \parencite{elosua_differences_2017, fried_clinical_2016}. Methodologically robust studies using bifactor models have consistently demonstrated these findings, with research reporting phonological deficits \parencite{friedman_reading_2017, raiker_phonological_2019} and extending these observations to include visuospatial impairments \parencite{kofler_working_2020}. High rates of cognitive impairment (75\%–81\%) have been reported, suggesting that working memory deficits are a core feature of ADHD.

These deficits have significant implications for problem-solving and academic performance. Impairments in attention, a key function of the central executive, exacerbate challenges in areas such as mathematics and reading comprehension, leading to higher rates of grade retention \parencite{fried_clinical_2016, friedman_reading_2017}. Attention plays a critical role in enabling the central executive to manage and manipulate information. Young adults with ADHD exhibit significant deficits in problem-solving and decision-making, as evidenced by tasks like the Stroop Test and Tower of Hanoi \parencite{capri_attention_2019}.

Cognitive flexibility and emotional regulation, which are closely linked to the central executive, further complicate decision-making processes. \textcite{drigas_executive_2019} suggest that individuals with ADHD often develop compensatory strategies to offset deficits in cognitive control. These strategies highlight the potential for interventions aimed at enhancing central executive functions, offering avenues to mitigate working memory challenges and improve cognitive and academic outcomes.

Collectively, these studies emphasize the interconnectedness of working memory, attention, and problem-solving in ADHD. Addressing central executive deficits is essential for improving both cognitive functioning and real-world outcomes for individuals with ADHD.

\section{Conclusion}

This literature review has examined the relationship between Attention-Deficit/Hyperactivity Disorder (ADHD) and working memory (WM) components. The evidence consistently demonstrates significant impairments across multiple WM domains, with particularly robust findings in the central executive \parencite{mccabe_relationship_2010} and visuospatial sketchpad \parencite{butzbach_basic_2019}. The phonological loop also shows consistent deficits \parencite{friedman_reading_2017}, though the magnitude of impairment varies across tasks and populations.

Methodologically robust studies using diverse assessment tools and larger sample sizes \parencite{fried_clinical_2016} have strengthened our understanding of these deficits, revealing that working memory impairments affect 75-81\% of individuals with ADHD. These findings suggest that working memory dysfunction represents a core feature of ADHD rather than a secondary characteristic. Moreover, the impact extends beyond cognitive measures to practical outcomes, affecting academic achievement and daily functioning \parencite{kofler_working_2020}.

The interaction between motivation and cognitive performance \parencite{engelmann_motivation_2007} offers promising directions for intervention. Understanding how motivational enhancement can modulate working memory performance may lead to more effective therapeutic approaches. Future research should focus on developing standardized assessment protocols and investigating how working memory training might be optimized through motivational strategies to improve functional outcomes for individuals with ADHD.

\printbibliography

\end{document}