\documentclass{article}

\begin{document}

1. For many years, psychologists debated the question of "nature versus nurture" in an "either/or" fashion. In what way has that question changed for today's personality psychologists?

a. Which determines personality-genetics or the environment?

b. Which is more important in determining personality-genetics or the environment?

c. To what extent are our personalities influenced by genetics?

d. To what extent and how do both genetics and the environment influence personality?

Answer: d

2. A woman likes to spend her time with others, enjoys loud music, often acts impulsively, and does not like to be alone. According to Eysenck's model of personality, this womanis a.

sociable.

b. highly extraverted.

c. low in neuroticism.

d. impulsive.

Answer: b

3. Ray and Tray were born at almost the same moment, share 100\% of their genes, and look and act very much alike. Based on this information, you can be sure that they are twins:

a. monozygotic

b. dizygotic

c. difertilized eggs

d. fraternal twins

Answer: a

4. A five-year-old boy is controlled and gentle. He is reluctant to play with new children and hesitates when entering new situations. Temperament researchers would probably identify this child as

a. shy.

b. inhibited.

c. introverted.

d. low in sociability.

Answer: b

5. According to evolutionary personality theory, the existence of anxiety today is

a. the result of natural selection.

b. due to a positive emotion.

c. no longer useful to human beings.

d. learned.

Answer: a

6. A researcher using the twin-study method finds an average correlation of 50 between monozygotic twins' scores on a measure of self-esteem. The researcher finds an average correlation of 30 between dizygotic twins' scores on this measure. The difference between the size of the correlations is statistically significant. What conclusion might the researcher draw from these results?

a. Genetics has more of an influence on self-esteem than environment has.

b. There is evidence for a genetic influence on self-esteem.

c. There is no evidence for a genetic influence on self-esteem.

d. The results are opposite of what would be expected if there were a genetic influence on self-esteem,

Answer: b

7. A researcher reports that men in this country tend to marry younger women and women tend to marry older men. This finding might be used to support which theory?

a. Eysenck's theory of personality

b. The five-factor model

c. The goodness of fit model

d. Evolutionary personality theory

Answer: d

8. Shayla is a very introverted girl. She prefers lower arousal circumstances and has a small group of friends. She has been in a bad mood lately, and her best friend wants to cheer her up while they are at a coffee house with a group of friends. Based on the research of Zelenski and colleagues (2012), what should Shayla's friend tell her to do?

a. Just be yourself, so that you feel authentic.

b. Try acting more extraverted when we're with other people.

c. You should probably go home, so that we aren't disturbing you.

d. Have some more coffee so that the caffeine wakes you up.

Answer: b

9. According to the "goodness of fit" model, educators should ask which question?

a. What temperament characteristics contribute to better school performance?

b. How can we change temperament in problem children?

c. How can we control problem behaviors in children with certain temperaments?

d. What kind of environment and procedures are most conducive to learning for this student?

Answer: d

10. A psychologist shows participants films designed to make them either happy or sad. She measures the participants' brain activity levels with an electroencephalograph (EEG) during the films. This psychologist is probably conducting research on

a. cerebral asymmetry.

b. brain damage.

c. temperament.

d. classical conditioning.

11. The biological approach to personality has each of the following strengths except one. Which one?

a. It has identified some realistic parameters for psychologists interested in behavior change

b. It has provided a bridge between the study of psychology and the discipline of biology.

c. Nearly all of the hypotheses generated from the biological approach can be tested through direct manipulation of the variables of interest.

d. Most of the advocates of the biological approach are academic psychologists interested in testing their ideas through research.

12. Research examining introversion and extraversion in monozygotic and dizygotic twins showed that:

a. Dizygotic twins were more likely than monozygotic twins to have the same trait

b. Monozygotic twins were more likely that dizygotic twins to have the same trait

c. Both mono and dizygotic twins were equally likely to have the same trait

d. Neither mono or dizygotic twins were likely to have the same trait

13. Why is it theorized that extraverts generally experience more happiness than introverts?

a. Introverts tend to have more social contact with friends, which serves as a buffer for stress

b. Research does not support the theory that extraverts experience more happiness than introverts

c. Extraverts typically have higher intelligence levels and are therefore better able to solve personal problems that impede feelings of happiness

d. All of the above

14. Which of the following statements is true regarding mate selection for cisgender women?

a. They are less picky than men when selecting mates due to the need to find a mate and begin having offspring within the limited window of childbearing years

b. They are pickier than men due to needs related to the process of bearing and raising a child

c. Research has not shown a difference in levels of pickiness in mate selection between men and women

d. Research indicates that mate selection is too complex to draw conclusions about gender effects on pickiness

15. Compared to cultures with poor gender equality, research shows that cultures that promote gender equality display:

a. greater differences in mate selection preferences across gender

b. fewer differences in mate selection preferences across gender

c. the same pattern of mate selection preference as cultures with less gender equality

d. a need to improve online dating

16. The "third force" (humanism) in psychology was developed in part to promote a view of humankind different from the ones presented by the

a. Freudian and trait approaches.

b. Freudian and behaviorism approaches.

c. trait and behaviorism approaches.

d. Freudian and social learning approaches.

17. Each of the following is an element of humanistic psychology except one. Which one?

a. Personal responsibility

b. Fully functioning person

c. Personal growth

d. Identity crisis

18. Which term did Abraham Maslow use that was roughly equivalent to Carl Rogers's concept of the fully functioning individual?

a. Ego integrity

b. Optimal experience

c. Flow

d. Self-actualization

19. Carl Rogers proposed a defense mechanism not originally identified by Sigmund Freud.

This process, called subception, allows people to

a. avoid awareness of threatening information.

b. move toward self-actualization.

c. bring their real and ideal selves closer together.

d. come to see themselves as they really are.

20. A father tells his son that he loves him when the child is good, but says "I do not like you right now" when the child misbehaves. The father neglects to remind the child that he is still loved. The father is engaging in

a. denial

b. conditional positive regard.

c. classical conditioning.

d. an anxiety-reduction technique.

21. According to Maslow, a man who has satisfied his physiological and safety needs will probably become highly aware of his need for

a. friends and a romantic relationship.

b. self-respect.

c. respect from his peers.

d. self-actualization.

22. Each of the following has been identified as a component of optimal experience except

one. Which one?

a. Our attention is completely absorbed in the activity.

b. We experience a loss of self-consciousness.

c. People around us recognize and respect our accomplishments.

d. We lose a sense of time.

23. Rogerian therapists measure each of the following with the Q Sort technique except one. Which one?

a. The client's self-concept

b. The client's ideal self

c. The client's progress during the course of psychotherapy

d. The client's ability to see things in a realistic manner

24. You ask a person to provide a real-self profile and an ideal-self profile with a Q-Sort. Which correlation would you expect to find between the two profiles if the person is approaching fully functioning?

a.-.52

b.04

c. .38

d. .75

Answer: d

Please Answer true or false for each of the statements below:

25. Carl Roger's believed people aspire to become self-actualized, while Maslow believed they strive to become fully functioning.

Answer: False

26. The need to perceive oneself as competent is an example of an esteem need.

Answer: True


27. Research indicates genetics can independently explain the development of some personality traits.

Answer: False

28. Between cisgender men and women, women are more likely to consider physical attractiveness when picking a mate than men.

29. Research has found that temperament is not related to intelligence.

30. One example of biological testing of personality is DNA testing.

31. Please fill in each of Maslow's Hierarchy of Needs Below (5pts):

Short Answer Questions:

32. Why is it challenging to test the role that genetics play in personality development, and what can researchers currently do to examine the role our genes play in the development of our personality (10pts)?

33. What is conditional positive regard, and how does it negatively affect us (5pts)?


\end{document}