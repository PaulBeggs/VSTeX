\documentclass[10pt]{book} 
\usepackage[width=5in, height=3in, centering]{geometry} 
\usepackage{geometry}
\usepackage{amsmath, amsfonts, amssymb}

\usepackage{lipsum}
\usepackage{multicol}
\pagestyle{empty}
\usepackage{graphicx}

\usepackage{tcolorbox}
\tcbuselibrary{poster}
\usepackage{xcolor}
\colorlet{darkgreen}{green!85!black}
\newcommand{\topic}[1]{\textcolor{cyan}{\textbf{#1}}}
\newcommand{\subtopic}[1]{\textcolor{red}{#1}}
\newcommand{\shortanswer}[1]{\textcolor{darkgreen}{\textbf{#1}}}

\begin{document}

\begin{center}
\begin{tcbposter}[
  poster = {
    % showframe,
    columns=1,
    rows=1,
    spacing=0mm,
  },
]

\posterbox[
top=1pt,
bottom=1pt,
left=1pt,
right=1pt,
tile,
colback=white,
]{
  column=1, row=1,
}{
  % \tiny
  \scriptsize
  % Custom Commands: \topic{Topic}, \subtopic{Subtopic}, \shortanswer{Short Answer}
 
  \shortanswer{Classical Conditioning Causes Fear:} Classical conditioning can lead to fears and emotional responses by pairing a neutral stimulus with a frightening or unpleasant event, causing the neutral stimulus alone to trigger fear or emotion over time. This was demonstrated in the study of Little Albert, where a young child was conditioned to fear a white rat by pairing it with a loud, frightening noise. \quad \shortanswer{Operant Conditioning:} Operant conditioning shapes behavior through consequences: behaviors followed by rewards are more likely to recur, while those followed by punishments are less likely. Additionally, complex actions can be learned by gradually reinforcing successive approximations (shaping). \quad \topic{Positive Reinforcement:} Increase behavior; give reward following behavior. \quad \topic{Negative Reinforcement:} Increase behavior; remove aversive stimulus following behavior. \quad \topic{Other Terms:} \subtopic{Extinction:} Decrease behavior; do not reward behavior; \subtopic{Punishment:} Decrease behavior; do give averse stimulus following behavior or take away positive stimulus. \quad \topic{Behavior-environment-behavior:} The way people treat you (environment) is partly the result of how you act (behavior). \quad \topic{Reciprocal Determinism:} \(\text{Behavior} \iff \text{External Factors (Rewards, punishment)} \iff \text{Internal Factors (Beliefs, Thoughts, Expectations)}\). \quad \topic{Observational Learning:} Learn by observing, reading, or hearing about other's actions. \quad \topic{Bobo Doll:} Children were able to observe behavior and enact it when asked. When confederate was punished children were less likely to mimic, if otherwise, they mimicked.\quad \topic{Self-Efficacy:} The notion that one is able to, say, quit smoking if they convince themselves to do so. Can be used to overcome childhood depression and other ailments.\quad \topic{Strengths / Weaknesses:} \subtopic{Strengths:} Strong empirical evidence, develops useful therapeutic procedures, can be easily taught and doesn't rely heavily on a 3rd party once teaching is over. \subtopic{Weaknesses:} Too narrow in description of personality, human beings are more complex than lab animals used in behavioral research. \quad \topic{Gender Roles:} \subtopic{Androgynous:} High in aggressiveness, emotionality, independence, and dependence; tend to flip between easily. \subtopic{Undifferentiated:} Low in all the above. \subtopic{Social Pressure:} Men must display aggressive, risk-taking and other behaviors; women must be traditional mothers. To reduce pressure, be aware of subtle reward and punishments for these behaviors. \subtopic{Agency:} (Masculinity) Insensitive to the needs of others. \subtopic{Communion:} (Femininity) So concerned with taking care of others, they sacrifice their own needs and interests.
}
\end{tcbposter}
\end{center}

\newpage

\begin{center}
\begin{tcbposter}[
  poster = {
    % showframe,
    columns=1,
    rows=1,
    spacing=0mm,
  },
]

\posterbox[
top=1pt,
bottom=1pt,
left=1pt,
right=1pt,
tile,
colback=white,
]{
  column=1, row=1,
}{
  \scriptsize
  
   \topic{Locus of Control:} \subtopic{Internal Locus:} General belief that people can affect events around them. \subtopic{External Locus:} Cannot affect events. \subtopic{Psychological Disorders:} Tends to have more external loci. Correlational data. \subtopic{Achievement:} Internal locus folks view themselves as responsible, so they are happier and more ambitious.\quad \topic{Learned Helplessness:} When one has learned there is nothing to do about a aversive event, they will not change. \subtopic{Elderly:} Residential communities cooking and cleaning for elderly is bad because it takes away control.\quad \topic{Personal Construct Theory:} Constantly generate hypotheses about the world. \subtopic{Personal Constructs:} dichotomies like friendly-unfriendly, tall-short, intelligent-unintelligent, masculine-feminine, etc. used to construct impressions. \quad \topic{Self-Schemas:} Beliefs about ourselves---what defines us.\quad \topic{Possible Selves:} Cognitive representations of the kind of person we might become someday. They provide incentives for us to become that representation in the future. Delinquent children add `criminal' to their schema \(\Rightarrow\) adult criminal. \quad \topic{CBT:} Reframing failures to be circumstantial and not due to them can be relieving. \quad \topic{Rational Emotive Therapy:} Dependence upon irrational beliefs. \subtopic{A-B-C Process:} \textbf{A}ctivating experience: The catalyst; emotional \textbf{C}onsequence: depression, happiness, etc.; irrational \textbf{B}elief: e.g., ``I can never be happy without this person.'' \quad\topic{Strengths:} Empirical findings; fits well with current mood of psychology. \quad\topic{Criticisms:} Too abstract; no single, correct model. \quad \shortanswer{Cognitive Triangle:} Thoughts \(\iff\) Feelings \(\iff\) Behaviors.  \quad \topic{Before/During/After Polarizing Conversations:} \subtopic{Before:} Check in with your emotions, How does this person typically treat you, Do you have the bandwidth for this, it's okay to decline. \subtopic{During:} Stay in touch with your emotions, Do you need a break?, How is this person treating you?, It's okay to express your emotions. \subtopic{After:} Recovery/Relaxation time, Processing the conversation, Accepting emotions vs. denying/pushing them away, Self-care. \quad\shortanswer{Application for Hard Conversations:} Keep a calm tone of voice, consider people's background to better understand their perceptual filters, do research and provide facts (if they are open to it), and avoid the use of absolutes ``Always'' and ``Never.''
   
}
\end{tcbposter}
\end{center}

\end{document}