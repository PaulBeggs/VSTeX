\documentclass[stu]{apa7}
\usepackage{amsmath}
\usepackage{amssymb}
\usepackage{enumitem}
\usepackage{geometry}
\usepackage[style=apa,uniquename=false,backend=biber]{biblatex}
\usepackage{hyperref}
\usepackage{xcolor}
\definecolor{horange}{HTML}{f58026}
\hypersetup{
    colorlinks=true,
    linkcolor=horange,
    filecolor=horange,      
    urlcolor=horange,
    citecolor=horange
}
% \usepackage{draculatheme}
\addbibresource{references.bib}

\title{Personality Development: A Dynamic Framework}
% \shorttitle{Personality Development}
\author{Paul Beggs}
\authorsaffiliations{Hendrix College}
\course{PSYC 370: Personality}
\professor{Dr.\ Sarah Root}
\duedate{\today}

\begin{document}
\maketitle

At the start of the semester, I defined personality as a unique grouping of characteristics that shape an individual's thinking, feeling, and behavior. I believed that personality arises from a combination of genetic inheritance, environmental context, life experiences, and cultural influences. Specifically, I posited that (a) genetic predispositions contribute to trait-like consistencies (e.g., shyness passed from parent to child), (b) one's childhood environment (supportive vs. neglectful) molds affective style, (c) significant life events (e.g., trauma) imprint patterns of reactivity, and (d) cultural norms (collectivist vs. individualist) steer one's values and interpersonal tendencies. Thus, I viewed personality as a relatively stable, multifactorial product of nature and nurture. While this perspective has merit, I now see that these views can be expanded upon and refined.

\section{Evolved Perspective on Personality Development}

Over the course of the class, my understanding of personality deepened from a static ``trait \& context'' model to a dynamic, lifespan-oriented, interactionist framework \parencite{burger2018personality}. I now conceptualize personality as a complex system comprising multiple interrelated components.

\subsection{Trait Systems Embedded in Biology}

Originally, I viewed personality largely as a direct inheritance of traits from parents. This perspective focused on the idea that our behavioral predispositions were simply passed down in a relatively unaltered form. However, deeper engagement with the Five-Factor Model (FFM) has revealed a more intricate picture. The FFM, which posits that personality can be distilled into five core dimensions—neuroticism, extraversion, openness, agreeableness, and conscientiousness \parencite{goldberg1990alternative}—initially seemed to limit personality to mere descriptive labels.

Further research, particularly twin studies \parencite{jang1996heritability}, shifted my perspective. These studies indicate that while genetic factors significantly influence personality (with heritability estimates of 41\% for neuroticism, 53\% for extraversion, 61\% for openness, 41\% for agreeableness, and 43\% for conscientiousness), they do so by establishing baseline propensities that interact with environmental influences. This deeper understanding has led me from a simplistic, static view towards appreciating the complex biological underpinnings that provide a foundation for how personality develops over time.

\subsection{Integration of Biological and Environmental Influences}

This evolution in my thinking reflects a significant shift: from seeing personality as a fixed set of inherited traits to recognizing it as a dynamic system where genetic predispositions and environmental experiences continuously interact. My early view of personality as merely inherited has grown into an understanding that environmental contexts, life experiences, and even cultural factors further shape and modify these inherent tendencies. This new approach not only aligns with research but also better explains the variability and development observed throughout the lifespan.

\subsection{Lifespan Plasticity and Narrative Identity}

Building on the integration of biological and environmental influences, research shows that personality is not solidified in early adulthood. Instead, individuals exhibit modest yet meaningful trait evolution throughout life, driven by significant role transitions—such as marriage or career changes—intentional self-regulation, and prevailing cultural scripts \parencite{roberts2006patterns}. Moreover, \textcite{mcadams2001bad} approach suggests that people actively construct and refine their life stories, interweaving past experiences, current contexts, and future goals. This ongoing narrative construction enriches our understanding of personality development by illustrating how the self remains flexible, continuously adapting to both internal drives and external pressures.

\subsection{Cultural and Contextual Embeddedness}

My earlier view of personality underestimated the profound influence of culture—not only in shaping how traits are expressed, but in defining what traits even matter. Trait models like the FFM originated in Western contexts and emphasize characteristics valued in individualistic societies, such as assertiveness and independence. However, cross-cultural research reveals important deviations in how people conceptualize the self and what they consider ideal personality traits.

For example, Americans commonly exhibit self-enhancement bias, rating themselves above average on skills and traits \parencite{taylor1989positive}. In contrast, individuals from collectivist cultures tend to avoid such self-inflation and see humility as a marker of psychological health \parencite{heine2007self}. What might be considered “low self-esteem” in Western frameworks is often a culturally appropriate form of modesty in collectivist societies. This divergence is so pronounced that American athletes, who are encouraged to express confidence and individuality, often struggle in team-centered environments like Japanese baseball clubs, where overt self-assertion is frowned upon \parencite{whiting1989you}.

Furthermore, self-esteem itself appears to be culturally malleable. \textcite{heine1999is} found that Asian individuals who had greater exposure to North American culture gradually developed higher self-esteem scores that aligned with Western norms of personal achievement. In essence, self-concept is not just individually defined—it’s socially and culturally constructed.

Cultural differences also shape what brings people satisfaction in life. In individualistic societies like the United States, happiness and emotional well-being are closely tied to life satisfaction \parencite{diener1995cross}. In contrast, individuals in collectivist cultures derive satisfaction from fulfilling social roles and adhering to group norms
\parencite{oishi2007dynamics,steger2008meaningful}. With such settings, a person’s sense of purpose and belonging—not necessarily their emotional highs—predicts how content they feel with life. These insights challenged my initial assumption that personality—and its outcomes—are universally defined. They are not; they are intricately shaped by the social script of the culture in which a person is embedded.

\section{Conclusion}

My current view positions personality as a dynamic system shaped by biological predispositions that are influenced by environmental influences, narrative identity, and cultural embedding. While genetic factors provide baseline propensities, individuals continuously interpret, enact, and re-narrate their traits through life transitions and cultural contexts. This refined framework moves beyond a static trait model to embrace developmental, situational, and cultural complexities. 


\printbibliography
\end{document}