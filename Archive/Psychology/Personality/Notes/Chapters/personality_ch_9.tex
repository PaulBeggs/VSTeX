\section{Hysenck}

\begin{coloredlist}
    \item \cyanit{Introversion-extraversion}
    \item \cyanit{Stability-neuroticism}
    \item Later added a third: \cyanit{psychoticism} - People on the high end of psychoticism are egocentric, aggressive, and cold.
\end{coloredlist}

\section{Nature Vs. Nurture}

\begin{coloredlist}
    \item If we can't realistically (or ethically) assign people to specific genes or environments, how can we test whether traits are caused by genetics or environment?
    \item Twin studies: Compare identical twins to fraternal twins. If identical twins are more similar, then the trait is genetic.
\end{coloredlist}

\subsection{Temperament}

\begin{coloredlist}
    \item \cyanit{Temperament} -- Evidence of differing personalities in infancy.
    \begin{coloredlist}
        \item Multiple models for temperaments.
        \item Temperament and environmental influences.
    \end{coloredlist}
\end{coloredlist}

\subsection{Evolutionary Psych and Personality}

\begin{coloredlist}
    \item Traits that have survived up to this point are theorized to be adaptive. That is, aids survival and passing on of genes.
    \item Criticisms of this theory say that it is too easy to come up with an explanation for any trait.
\end{coloredlist}

\section{Assessing Biological Contributors to Personality}

\begin{coloredlist}
    \item How can we assess someone's personality biologically?
    \begin{coloredlist}
        \item DNA sequencing
        \item Neuroimaging (fMRI and PET scans)
        \item Heart rate, respiration, galvanic skin response
        \item Hormone levels
        \item Neurotransmitters
        \item Brain lesions
        \item EEG
        \item Twin studies
        \item And many more\ldots
    \end{coloredlist}
\end{coloredlist}