
\begin{table}[htbp]
    \centering
    \begin{tabular}{p{3cm}p{6cm}p{6cm}}
        \toprule
        \textbf{Somatotype} & \textbf{Character} & \textbf{Shape} \\ \midrule
        Endomorph \newline [viscerotonic] & Relaxed, sociable, tolerant,\newline comfort-loving, peaceful & Plump, buxom, developed\newline visceral structure \\ \midrule
        Mesomorph \newline [somatotonic] & Adventurous, assertive, \newline aggressive, vigorous, dominant & Muscular, large-boned, hard,\newline rectangular, strong \\ \midrule
        Ectomorph \newline [cerebrotonic] & Quiet, sensitive, restrained, \newline introverted, fragile & Thin, fragile, delicate, fine-boned \\
        \bottomrule
    \end{tabular}
    \caption{Sheldon's Somatotypes}\label{tab:}
\end{table}

\textbf{Trait}: Categorizes people according to degree to which they manifest a particular characteristic. \\

\textbf{Trait theory assumptions}:
\begin{enumerate}
    \item Characteristics are relatively stable over time.
    \item Characteristics are relatively stable across situations.
\end{enumerate}

\section{Gordon Allport}

\subsection{Nomothetic vs. Idiographic Approaches}

\begin{coloredlist}
    \item \cyanit{Nomothetic}:
    \begin{coloredlist}
        \item All people can be described along a single dimension according to their level of, for example, assertiveness or anxiety.
        \item Each person in a study using the nomothetic approach is tested to see how his or her score for the given trait compares with the scores of other participants.
        \item These traits are called \cyanit{common traits} because they are shared by everyone.
    \end{coloredlist}
    \item \cyanit{Idiographic}:
    \begin{coloredlist}
        \item Focuses on the unique characteristics of an individual.
        \item If you ask most people to list the traits that describe them, they will list a number of traits that are unique to them. These traits are called \cyanit{central traits}.
    \end{coloredlist}
    \item Allport says the best way to know someone is by identifying their central traits.
    \item He proposed that occasionally, a single trait will dominate a person's life and shape almost everything they do. This trait is called a \cyanit{cardinal trait}.
    \begin{coloredlist}
        \item For example, consider someone who is Machiavellian.
    \end{coloredlist}
\end{coloredlist}

\section{Henry Murray}

\begin{coloredlist}
    \item Greatly influenced by Carl Jung.
    \item Theory includes a blend of psycholoanalytic and trait approaches.
    \item \cyanit{Personology}: The study of the whole person.
    \begin{coloredlist}
        \item Basic elements of personality are driven by psychogenic needs.
        \item Identified 27 needs.
        \begin{coloredlist}
            \item Behavior is influenced by the level of importance of each need.
            \item Example: Achievement and affiliation needs.
        \end{coloredlist}
    \end{coloredlist}
\end{coloredlist}

\section{Raymond Cattell}

\begin{coloredlist}
    \item Devoted time to identifying personality traits through statistical analysis. That is, factor analysis.
    \item \cyanit{Factor analysis}: A statistical technique that identifies clusters of related items on a test.
    \item Consists of 16 personality traits:
    \begin{enumerate}
        \begin{multicols}{2}
            \item Warm
            \item Abstract thinker
            \item Emotionally stable
            \item Dominant
            \item Enthusiastic
            \item Conscientious
            \item Bold
            \item Tender-minded
            \item Suspicious
            \item Imaginative
            \item Shrewd
            \item Apprehensive
            \item Experimenting
            \item Self-sufficient
            \item Controlled
            \item Tense
        \end{multicols}
    \end{enumerate}
\end{coloredlist}

\section{The Big Five}

\begin{coloredlist}
    \item \cyanit{Neuroticism}: Degree of emotional instability or stability.
    \item \cyanit{Extraversion}: Degree of sociability or social introversion.
    \item \cyanit{Openness}: Degree of intellectual curiosity or creativity.
    \item \cyanit{Agreeableness}: Degree of friendliness or hostility.
    \item \cyanit{Conscientiousness}: Degree of organization or impulsiveness.
\end{coloredlist}

\begin{table}[htbp]
    \centering
    \begin{tabular}{p{3cm}p{10cm}}
        \toprule
        \textbf{Factor} & \textbf{Characteristics} \\ \midrule
        Openness & Imaginative vs. practical; preference for variety vs. routine; independent vs. conforming \\ \midrule
        Conscientiousness & Organized vs. disorganized; careful vs. careless; disciplined vs. impulsive \\ \midrule
        Extraversion & Sociable vs. retiring; fun-loving vs. sober; affectionate vs. reserved \\ \midrule
        Agreeableness & Softhearted vs. ruthless; trusting vs. suspicious; helpful vs. uncooperative \\ \midrule
        Neuroticism & Worried vs. calm; insecure vs. secure; self-pitying vs. self-satisfied \\
        \bottomrule
    \end{tabular}
    \caption{The Big Five Personality Factors}\label{tab:bigfive}
\end{table}


\section{Strengths and Weaknesses}

\begin{coloredlist}
    \item \cyanit{Strengths}:
    \begin{coloredlist}
        \item Provides a comprehensive framework for understanding personality.
        \item Has been supported by a large body of research.
        \item Has been used to predict a wide range of behaviors.
    \end{coloredlist}
    \item \cyanit{Weaknesses}:
    \begin{coloredlist}
        \item Does not explain:
        \begin{coloredlist}
            \item \textbf{How} or \textbf{why} traits develop.
            \item \textbf{How} traits interact with each other.
            \item \textbf{How} traits are influenced by the environment.
        \end{coloredlist}
        \item Has the chance for people to fake being good or bad. This is called \cyanit{social desirability}.
    \end{coloredlist}
\end{coloredlist}

\section{Summary}

\begin{enumerate}
    \item The trait approach assumes we can identify individual differences in behaviors that are relatively stable across situations and over time. Trait theorists are usually not concerned with any one person’s behavior but rather with describing behavior typical of people at certain points along a trait continuum.
    \item Gordon Allport was the first acknowledged trait theorist. Among his contributions were the notions of central and secondary traits, nomothetic versus idiographic research, and descriptions of the self. Henry Murray identified psychogenic needs as the basic elements of personality. According to Murray, a need will affect behavior depending on where it lies on a person’s need hierarchy and the kind of situation the person is in.
    \item Raymond Cattell was interested in identifying the basic structure of personality. He used a statistical procedure called factor analysis to determine how many basic traits make up human personality. More recent research provides consistent evidence that personality is structured along five basic dimensions. Although questions remain, the evidence to date tends to support the five-factor model.
    \item An enduring controversy in personality concerns the relative importance of traits compared to situational determinants of behavior. Critics have charged that traits do not predict behavior well and that there is little evidence for cross-situational consistency. Trait advocates have answered that if traits and behaviors are measured correctly, a significant relationship can be found. In addition, they maintain that the amount of behavior variance explained by traits is considerable and important.
    \item The development of the five-factor model renewed interest in the relationship between personality and job performance. Although several of the Big Five dimensions are related to performance in the business world, many studies indicate that Conscientiousness may be the best predictor of performance.
    \item Trait researchers typically rely on self-report assessment procedures in their work. One of the most commonly used self-report inventories is the Minnesota Multiphasic Personality Inventory. Test users need to be aware of problems inherent in self-report inventories. These include faking, carelessness and sabotage, and response tendencies.
\end{enumerate}
