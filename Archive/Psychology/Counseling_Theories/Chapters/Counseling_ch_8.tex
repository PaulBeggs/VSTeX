\section{Fritz Perls}

\begin{coloredlist}
    \item Born in 1893 in Berlin, Germany
    \item After serving in WWI, Fritz studied medicine and provided treatment to soldiers with brain injuries.
    \item He fled Germany in 1933 and moved to South Africa.
    \item In 1946, he moved to New York City and began working with Karen Horney.
    \item Coauthored a book on Gestalt Therapy and founded the New York Institute for Gestalt Therapy.
\end{coloredlist}

\section{Nature of Humanity}

\begin{coloredlist}
    \item Basic assumption about people is that they have the capacity to self-regulate when they are aware of what is happening around them.
    \item The view is rooted in existentialism and phenomenological theories.
    \item People need to ``re-own'' and unify all parts of themselves, including the parts they have disowned or denied.
\end{coloredlist}

\section{Goals for Gestalt Therapy}

\begin{coloredlist}
    \item Focuses on the here and now.
    \item Clients need to gain awareness.
    \begin{coloredlist}
        \item Defined as knowing the environment, themselves, accepting themselves, and being able to make contact (interact with others and nature without losing themselves).
        \item Increased awareness itself is curative.
    \end{coloredlist}
\end{coloredlist}

\section{Role of the Therapist}

\begin{coloredlist}
    \item The work of therapy is done by the client.
    \item The therapists' job is to create a climate where clients can try new ways of being and behaving.
    \item Working in the here and now with an ``I/Thou'' relationship.
    \item Fritz Perls methodology: The therapist is directive and confrontational.
\end{coloredlist}

\section{Gestalt Therapy Overview}
\subsection{Evolution of the Approach}
\begin{coloredlist}
    \item From Fritz Perls’ pioneering work emphasizing present-centered awareness and direct confrontation,
    \item to contemporary relational approaches that stress authentic dialogue and mutual influence.
\end{coloredlist}

\subsection{Philosophy and Basic Assumptions}
\begin{coloredlist}
    \item Emphasizes wholeness, personal responsibility, and the integration of thoughts, feelings, and behaviors.
\end{coloredlist}

\subsection{Key Concepts}
\begin{coloredlist}
    \item \cyanit{Holism} – Viewing the person as an integrated whole.
    \item \cyanit{Field Theory} – Understanding behavior in the context of the whole environment.
    \item \cyanit{Figure-Formation Process} – How individuals perceive and organize experience.
    \item \cyanit{Organismic Self-Regulation} – The innate drive toward growth and balance.
\end{coloredlist}

\subsection{Trusting Relationship in Experiments}
\begin{coloredlist}
    \item A strong, trusting client–therapist relationship is central to engaging clients in experiential experiments that promote change.
\end{coloredlist}

\subsection{Role of Confrontation}
\begin{coloredlist}
    \item Contemporary relational Gestalt therapy uses measured confrontation to challenge defensive patterns while maintaining support.
\end{coloredlist}

\subsection{Standard Gestalt Therapy Interventions}
\begin{coloredlist}
    \item Internal Dialogue Exercise – Enhances self-awareness.
    \item Empty-Chair Technique – Facilitates expression of unresolved issues.
    \item Future Projection – Aids clients in envisioning change.
    \item Making the Rounds – Explores multiple perspectives.
    \item Reversal Exercise – Encourages viewing situations from an opposite angle.
    \item Rehearsal Exercise – Allows practice of new behaviors.
    \item Exaggeration Exercise – Amplifies actions for clarity.
    \item Staying with the Feeling – Supports acceptance of emotions.
    \item Dream Work – Interprets dreams as meaningful signals.
\end{coloredlist}

\subsection{Application to Group Counseling}
\begin{coloredlist}
    \item Gestalt techniques foster group cohesion and authentic interaction, enabling members to experiment with new relational patterns.
\end{coloredlist}

\subsection{Application in School Counseling}
\begin{coloredlist}
    \item Emphasizes experiential learning and self-awareness to improve student self-regulation and interpersonal skills.
\end{coloredlist}

\subsection{Multicultural Perspective}
\begin{coloredlist}
    \item Stresses adaptation of interventions to be sensitive to diverse cultural values and worldviews.
\end{coloredlist}

\subsection{Contributions, Strengths, and Limitations}
\begin{coloredlist}
    \item \cyanit{Contributions}: Focus on present experience and relational authenticity.
    \item \cyanit{Strengths}: Flexible, experiential methods that empower client insight.
    \item \cyanit{Limitations}: Potential challenges in structure and cultural adaptability.
\end{coloredlist}