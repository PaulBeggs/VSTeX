\subsection{Managing Your Personal Values}

Personal therapy contributes to the therapist's professional work in 3 ways:

\begin{enumerate}
    \item As part of the therapist's training, personal therapy offers a model of therapeutic practice in which the trainee observes a more experienced therapist.
    \item A beneficial experience in personal therapy can further enhance a therapist's interpersonal skills, which are essential to skillful practicing therapy.
    \item Successful personal therapy can contribute to a therapist's ability to deal with the ongoing stresses associated with clinical work.
\end{enumerate}

\begin{coloredlist}
    \item Therapists have a responsibility to be aware of their own values and set aside personal beliefs, so they do not ``contaminate the counseling process'' (aka, Bracketing). 
\end{coloredlist}

\subsection{Becoming an Effective Multicultural Counselor} 

\begin{coloredlist}
    \item Acquiring Competencies in Multicultural Counseling
    \begin{coloredlist}
        \item Diversity-competent practitioners
        \begin{coloredlist}
            \item Beliefs and attitudes
            \item Knowledge
            \item Flexibility with Intervention Strategies
        \end{coloredlist}
    \end{coloredlist}
\end{coloredlist}

\section{Maintaining Your Vitality as a Person and as a Professional or ``Burnout Prevention''}

\begin{coloredlist}
    \item Therapeutic lifestyle changes (TLCs)
    \begin{coloredlist}
        \item Physical activity
        \item Diet and nutrition
        \item Time in nature
        \item Relationships
        \item Recreation
        \item Religious/spiritual involvement
        \item Providing service to others
    \end{coloredlist}
\end{coloredlist}