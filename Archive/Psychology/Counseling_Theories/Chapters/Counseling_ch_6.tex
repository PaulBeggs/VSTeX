\section{Existentialism: The Origins}
\begin{coloredlist}
    \item Philosophical foundations in 19th/20th-century thinkers:
    \begin{coloredlist}
        \item Søren Kierkegaard (subjective truth, anxiety, freedom)
        \item Friedrich Nietzsche (will to power, ``God is dead,'' self-creation)
        \item Martin Heidegger (being-toward-death, \cyanit{Dasein} [being-in-the-world])
        \item Jean-Paul Sartre (``existence precedes essence,'' radical freedom)
    \end{coloredlist}
    \item Later integrated into psychology by:
    \begin{coloredlist}
        \item Rollo May (pioneer of existential psychology)
        \item Irvin Yalom (four existential ``ultimate concerns'')
    \end{coloredlist}
\end{coloredlist}

\subsection{Key Existential Themes}
\begin{coloredlist}
    \item Confrontation with the ``ultimate concerns'' (Yalom):
    \begin{coloredlist}
        \item Death: Awareness of mortality as a catalyst for authenticity.
        \item Freedom: Responsibility for self-creation in a world without inherent meaning.
        \item Isolation: Existential aloneness despite relationships.
        \item Meaninglessness: The challenge to create meaning without guarantees.
    \end{coloredlist}
    \item Existential anxiety vs. pathological anxiety:
    \begin{coloredlist}
        \item Healthy anxiety: Motivates growth and authenticity.
        \item Neurotic anxiety: Avoidance of freedom/responsibility.
    \end{coloredlist}
\end{coloredlist}

\subsection{Mental Health Effects of The Holocaust}
\begin{coloredlist}
    \item The horrors of the holocaust led to an increase of people struggling with existential crises.
    \item Frankl, a survivor of Auschwitz, developed a form of therapy called \cyanit{logotherapy}. This therapy focuses on the search for meaning in life.
\end{coloredlist}

\section{Goals of Existential Therapy}
\begin{coloredlist}
    \item Assisting clients in moving toward authenticity and learning to recognize when they are deceiving themselves.
    \item Helping clients face anxiety and engage in action that is based on creating a worthy existence.
    \item Helping clients to reclaim and re-own their lives; teaching them to listen to what they already know about themselves.
    \item Encouraging clients to confront existential "givens" (e.g., death, freedom) to live more fully.
    \item Facilitating acceptance of uncertainty and the courage to act despite it.
\end{coloredlist}

\subsection{The Therapeutic Relationship}
\begin{coloredlist}
    \item Therapy is a journey taken by therapist and client
    \begin{coloredlist}
        \item The person-to-person relationship is key.
        \item The relationship demands that therapists be in contact with their own experience in the world.
    \end{coloredlist}
    \item The core of the therapeutic relationship.
    \item Therapist as a \cyanit{authentic companion}, not an authority:
    \begin{coloredlist}
        \item Focus on the "here-and-now" interaction.
        \item Use of self-disclosure to model authenticity.
    \end{coloredlist}
    \item Emphasis on presence, deep listening, and phenomenological exploration (understanding the client’s subjective world).
\end{coloredlist}

\subsection{Techniques and Practices}
\begin{coloredlist}
    \item Less technique-driven; focuses on dialogue and process:
    \begin{coloredlist}
        \item Phenomenological method: Unpacking the client’s lived experience without judgment.
        \item Existential reflection: Exploring choices, values, and responsibility.
        \item Paradoxical intention (Frankl): Facing fears by exaggerating or embracing them.
    \end{coloredlist}
    \item Applications: Effective for grief, identity crises, life transitions, and terminal illness.
\end{coloredlist}

\subsection{Critiques and Limitations}
\begin{coloredlist}
    \item Challenges: Lack of structured techniques, abstract concepts.
    \item Criticisms: Overemphasis on individualism; less focus on systemic/social factors.
\end{coloredlist}