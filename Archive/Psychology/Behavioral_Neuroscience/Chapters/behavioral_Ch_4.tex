\begin{coloredlist}
    \item \cyanit{Corpus callosum}
    \begin{coloredlist}
        \item Connects the two hemispheres.
        \item Remember that the neurons in this structure go from lateral to lateral, and not from dorsal to ventral.
        \item Creates the roof of the lateral ventricles.
        \item Agenesis of the cc
        \begin{coloredlist}
            \item AKA: Callosal Agenesis
            \item Vision impairments,
            \item hypotonia,
            \item poor motor coordination,
            \item delays in motor milestones,
            \begin{coloredlist}
                \item (Such as sitting and walking.)
            \end{coloredlist}
            \item cognitive disability,
            \begin{coloredlist}
                \item (Disability in complex problem solving.)
            \end{coloredlist}
            \item and social difficulties.
            \begin{coloredlist}
                \item (Missing subtle social cues maybe cause of impaired fair processing.)
            \end{coloredlist}
        \end{coloredlist}
        \item \cyanit{Corpus Callosotomy} -- Split brain surgeries.
        \begin{coloredlist}
            \item Used to treat epilepsy.
            \item Gives information about lateralization of hemispheres.
            \item \textbf{Left Hemisphere}
            \begin{coloredlist}
                \item language
                \item serial events
            \end{coloredlist}
            \item \textbf{Right Hemisphere}
            \begin{coloredlist}
                \item creativity
                \item synthesis
            \end{coloredlist}
        \end{coloredlist}
    \end{coloredlist}
    \begin{center}
        \hrule
        \textbf{NEW NOTES FOR 02/12/2025}
    \end{center}
    \item \cyanit{Basal Ganglia}
    \begin{coloredlist}
    \item \textbf{Function:}
    \begin{coloredlist}
        \item Initiation of Voluntary Movements.
        \item \hyperref[parkinson]{CLick here for Parkinson's continuation.}
    \end{coloredlist}
    \item Curls laterally around the thalamus.
    \item \cyanit{Striatum}
    \begin{coloredlist}
        \item \cyanit{Caudate Nucleus} = ``Nucleus with a Tail''
        \begin{coloredlist}
            \item Obsessive Compulsive Disorder (OCD)
            \begin{coloredlist}
                \item MIXED RESULTS
                \begin{coloredlist}
                    \item Too much activity, too large.
                \end{coloredlist}
            \end{coloredlist}
            \item Romantic Love
            \begin{coloredlist}
                \item Fisher, Aron, and Brown
                \begin{coloredlist}
                    \item Anthropologist used fMRI with a picture of neutral and romantic partners.
                    \item The CN activity was increased for loved one.
                \end{coloredlist}
            \end{coloredlist}
            \item Larger in folks with incredible episodic memories (superior autobiographical memory).
            \begin{coloredlist}
                \item How large? 7-8 SDs larger.
            \end{coloredlist}
        \end{coloredlist}
        \item \cyanit{Putamen} = ``Shell''
        \item \cyanit{Nucleus Accumbens}
        \begin{coloredlist}
            \item Nucleus Accumbens Septi = ``Nucleus leaning against the septum.''
            \begin{coloredlist}
                \item Where the head of the caudate and the most anterior portion of the putamen come together.
                \item Plays an important role in reinforcement, pleasure, and addiction.
            \end{coloredlist}
        \end{coloredlist}
    \end{coloredlist}
    \item \cyanit{Globus Pallidus} = ``Pale Globe''
    \item \textit{Note:} When people mention the putamen and the globus pallidus, they call it the lentiform nucleus.
    \begin{coloredlist}
        \item \cyanit{Limbic System} = ``Border''
        \item First thing that comes to your mind should be that the limbic system is in charge of emotions.
        \begin{coloredlist}
            \item \cyanit{Hippocampus} = ``Seahorse''
            \begin{coloredlist}
                \item In charge of moving memories from short-term to long-term storage.
                \item Emotion, selective attention, learning, and memory.
                \item Buried in the temporal lobe.
                \item On the anterior tip of the hippocampus is the amygdala.
                \item Study on H.M. (Henry Molaison)
                \item Does not store declarative memories, but stories procedural memories.
            \end{coloredlist}
            \item \cyanit{Amygdala}
            \begin{coloredlist}
                \item In charge of emotions.
                \item Fear and aggression.
            \end{coloredlist}
            \item \cyanit{Cingulate Gyrus}
            \begin{coloredlist}
                \item Above the corpus callosum.
            \end{coloredlist}
            \item \cyanit{Fornix}
            \begin{coloredlist}
                \item Its tail huge the inside of the hippocampus.
            \end{coloredlist}
            \item \cyanit{Mammillary Bodies}
            \item \cyanit{Septal Nucleus}
            \begin{coloredlist}
                \item Where the fornix and the corpus callosum meet.
            \end{coloredlist}
        \end{coloredlist}
        \item \cyanit{Cerebral Cortex}
        \item \cyanit{Nucleus Accumbens}
    \end{coloredlist}

\subsection{H.M.}

\begin{coloredlist}
    \item Knocked unconscious by a bike accident at age 7.
    \item Epilepsy
    \begin{coloredlist}
        \item \cyanit{Absent seizures} at age 10.
        \begin{coloredlist}
            \item These seizures are hard to notice. They are like staring off into space.
        \end{coloredlist}
        \item \cyanit{Grand mal seizures} at age 16.
        \begin{coloredlist}
            \item ``Great bad'' seizures.
            \item These are the ones where you fall to the ground and shake.
            \item Can cause death.
            \item Has them once a week.
        \end{coloredlist}
    \end{coloredlist}
    \item 1953 at age 27, he meets Dr. Scoville.
    \begin{coloredlist}
        \item Bilateral medial temporal lobectomy.
        \item Scoville removes the hippocampus on both sides.
        \item This surgery had been done on ``psychotic'' patients in the past.
        \item Expected some disorientation and memory loss.
    \end{coloredlist}
    \item \textbf{Results:}
    \begin{coloredlist}
        \item No more grand mal seizures.
        \item IQ remained the same.
        \item Anterograde Amnesia
        \begin{coloredlist}
            \item Inability to form new memories.
        \end{coloredlist}
        \item Procedural memory intact.
        \begin{coloredlist}
            \item Declarative memories were not stored, but procedural memories were.
            \item Mirror drawing task
            \begin{coloredlist}
                \item Drawing a star while looking at the reflection of the star.
                \item Would be taught the rules, then he would forget. This happened repeatedly.
                \item Even though he didn't know the rules (or his practice), he would get better.
            \end{coloredlist}
            \item Tower of Hanoi
            \begin{coloredlist}
                \item He would learn the trick to solving the puzzle (declarative), but he would forget the trick.
                \item  
            \end{coloredlist}
        \end{coloredlist}
    \end{coloredlist}

\end{coloredlist}